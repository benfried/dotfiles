% Fix ``.'' spacing in lists
\documentstyle[12pt,aimemo]{article}

\setlength{\textwidth}{6.0in}
\setlength{\oddsidemargin}{0.25in}
\setlength{\evensidemargin}{0.25in}
\setlength{\topmargin}{0pt}
\setlength{\headsep}{24pt}
\setlength{\textheight}{8.5in}

\let\origpar=\par

% somehow, LaTex fucks with things so that the plain Tex \obeyspaces
% doesn't work.  This is the same, except we use an mbox containing
% a space instead of just a space.

\newcommand{\spc}{\mbox{ }}
\newcommand{\tAb}{\mbox{        }}
\def\obspaces{\catcode`\ =\active\catcode`\^^I=\active}
{\obspaces \global\let =\spc\global\let^^I=\tAb}

\newenvironment{program}{\medskip\samepage\tt\obeylines\obspaces}{\medskip}

% use | for \tt
\catcode`\|=13
\newif\ifvbar
\def|{\ifvbar
         \endgroup\vbarfalse
      \else 
         \begingroup\tt\vbartrue 
      \fi}



\newcommand{\lisp}{\tt}
\newcommand{\iter}{|iterate|}
\newcommand{\iterpg}{|iterate-pg|}
\newcommand{\setf}{|setf|}
\newcommand{\nil}{|nil|}
\newcommand{\nonnil}{non-\nil}
\newcommand{\opt}{{\lisp \&optional}}
\newcommand{\yields}{$\Rightarrow$}



\newbox\Dots
\setbox\Dots=\hbox{\small ...}
\renewcommand{\dots}{\copy\Dots}

% Here's a way to define an environment which will treat all
% paragraphs inside it in the same manner.  It's, as near as I can
% figure, the way that Latex does things.  Using \everypar won't work;
% apparently Latex resets that a lot.  What you have to do is: at the
% top of the file do \let\origpar=\par.  In the
% begin part of the environment, do the parshape (or whatever), then say
% \def\par{{\origpar}}.  The extra braces are necessary.   That's all
% you have to do; at the end of the environment things will return to
% normal automatically.  Why does this all work?  I have no idea.
% Another important thing: there should be a blank line (i.e. a \par)
% between the end of the last paragraph and the \end{environment}.

\newlength{\clindent}
\setlength{\clindent}{0.5in}
\newlength{\clparindent}
\setlength{\clparindent}{\parindent}

\newenvironment{clauses}{\advance\linewidth -\parindent
                        \hangindent\clindent\relax
                         \hangafter=0
                          \parindent=0pt
                          \def\par{{\origpar}}}{}

\newcommand{\cpar}{\hspace \clparindent}

\newcommand{\startitem}{\bigskip\pagebreak[3]}
\newcommand{\finishitem}{\smallskip}

\newcommand{\clause}[1]{\startitem\Clause{#1}\finishitem}
\newcommand{\Clause}[1]{\hbox{#1}}
\newcommand{\clindex}[1]{\index{{\lisp #1}}}
\newcommand{\clindexx}[2]{\index{{\lisp #1\dots #2}}}
\newcommand{\clindexxx}[3]{\index{{\lisp #1\dots #2\dots #3}}}
\newcommand{\clausex}[2]{\startitem\Clausex{#1}{#2}\finishitem}
\newcommand{\Clausex}[2]{\Clause{|#1| #2}\clindex{#1}}

\newcommand{\defvar}[1]{\startitem\deff{#1}{}{Variable}\finishitem}
\newcommand{\defunexpvar}[1]{\startitem\deffnoindex{iterate::#1}{}{Variable}
                                        \lispindex{#1}\finishitem}
\newcommand{\defun}[2]{\startitem\deff{#1}{#2}{Function}\finishitem}
\newcommand{\defmacro}[2]{\startitem\deff{#1}{#2}{Macro}\finishitem}
\newcommand{\defunx}[3]{\startitem\deffx{#1}{#2}{#3}{Function}\finishitem}
\newcommand{\defmacrox}[3]{\startitem\deffx{#1}{#2}{#3}{Macro}\finishitem}
\newcommand{\defunxx}[4]{\startitem\deffxx{#1}{#2}{#3}{#4}{Function}\finishitem}
\newcommand{\defmacroxx}[4]{\startitem\deffxx{#1}{#2}{#3}{#4}{Macro}\finishitem}

\newcommand{\deff}[3]{\deffnoindex{#1}{#2}{#3}\lispindex{#1}}
\newcommand{\deffnoindex}[3]{\hbox to \hsize{{\tt #1} {\it #2}\hfill [{\it #3}]}}
\newcommand{\deffx}[4]{\deff{#1}{#2}{#4}\moreargs{#1}{#3}}
\newcommand{\deffxx}[5]{\deffx{#1}{#2}{#3}{#5}\moreargs{#1}{#4}}

\newcommand{\moreargs}[2]{\hbox to \hsize{\phantom{\lisp #1} {\it #2}\hfill}}

\newcommand{\lispindex}[1]{\index{{\lisp #1}}}

 \newenvironment{note}[1]{\pagebreak[2]\bigskip
                          \hrule\smallskip\small
                         {\setlength{\parindent}{0pt}
                           \par{\bf #1:}}}{\smallskip\hrule\bigskip}



\makeindex

%%%%%%%%%%%%%%%%%%%%%%%%%%%%%%%%%%%%%%%%%%%%%%%%%%%%%%%%%%%%%%%%

%\thispagestyle{empty}

%\vspace*{2in}

%\title{The |Iterate| Manual}
%\author{Jonathan Amsterdam}

%\maketitle

%\thispagestyle{empty}

%\begin{center}
%\Large *** DRAFT ***
%\end{center}
%This manual is in beta-test. It will become an AI Memo in a couple of
%months.  I would appreciate any comments or
%corrections.  You can reach me at email address |jba@ai.ai.mit.edu|.

\begin{document}

\def\writtenby{Jonathan Amsterdam}
\def\contractno{This report describes research done at the Artificial
Intelligence Laboratory of the Massachusetts Institute of Technology.
Support for the laboratory's artificial intelligence research is
provided in part by the Advanced Research Projects Agency of the
Department of Defense under Office
of Naval Research contract N00014-85-K-0124.}

\AIMEMO{1236}{\Large The Iterate Manual \\[1in]}
{This is the manual for version 1.4 of \iter, a powerful iteration
macro for Common  
Lisp. \iter\ is similar to {\tt loop} but provides numerous additional
features, is well integrated with Lisp, and is extensible. 
\vfill }


% use ~ for \em
\catcode`\~=13
\newif\ifem
\def~{\ifem
         \/\endgroup\emfalse
      \else 
         \begingroup\em\emtrue 
      \fi}



%\pagebreak
%\setcounter{page}{1}

%\thispagestyle{empty}


\tableofcontents

\pagebreak
\pagestyle{headings}

\section{Introduction}

\begin{sloppypar}
This manual describes
\iter, a powerful iteration facility for Common Lisp.
\iter\ provides abstractions for many common iteration
patterns and allows for the definition of additional patterns.  \iter\ is
a macro that expands into ordinary Lisp at
compile-time, so it is more efficient than higher-order
functions like |map| and |reduce.|  
While it is similar to |loop|, \iter\ offers a more Lisp-like syntax
and enhanced extensibility.
(For a more complete
comparison of \iter\ with other iteration constructs, see MIT AI Lab Working
Paper No. 324, ~Don't Loop, Iterate.~)
\end{sloppypar}

An \iter\ form consists of the symbol |iter|\footnote{You can also use
|iterate|, but |iter| is preferred because it avoids potential conflicts with
possible future additions to Common Lisp, and because it saves
horizontal space when writing code.}
followed by one or more
forms, some of which may be \iter\ 
~clauses.~  Here is a simple example of \iter\ which collects the
numbers from 1 to 10 into a list, and returns the list.  The return
value is shown following the arrow.
\begin{program}
(iter (for i from 1 to 10)
      (collect i))          \yields (1 2 3 4 5 6 7 8 9 10)
\end{program}
This form contains two clauses: a |for| clause that steps the
variable |i| over the integers from 1 to 10, and a |collect|
clause that accumulates its argument into a list.  With a few
exceptions, all \iter\ clauses have the same format: alternating
symbols (called ~keywords~) and expressions (called
~arguments~).  The syntax and terminology are those of Common
Lisp's keyword lambda lists.  One difference is that \iter's keywords
do not have to begin with a colon---though they may, except for the
first symbol of a clause.  So you can also write  |(for i :from
1 :to 10)| if you prefer.

Any Lisp form can appear in the body of an \iter, where it will have
its usual meaning.  \iter\ walks the entire body, expanding macros,
and recognizing clauses at any level.  This example collects all the
odd numbers in a list:
\begin{program}
(iter (for el in list)
      (if (and (numberp el) (oddp el))
          (collect el)))
\end{program}

There are clauses for iterating over numbers, lists, arrays and other
objects, and for
collecting, summing, counting, maximizing and other useful operations.
\iter\ also supports the creation of new variable bindings, stepping
over multiple sequences at once, destructuring, and compiler
declarations of variable types.  The following example illustrates
some of these features: 
\begin{program}
(iter (for (key . item) in alist)
         (for i from 0)
         (declare (fixnum i))
         (collect (cons i key)))
\end{program}

This loop takes the keys of an alist and returns a new alist
associating the keys with their positions in the original list. The
compiler declaration for |i| will appear in the generated code
in the appropriate place.

%%%%%%%%%%%%%%%%%%%%%%%%%%%%%%%%%%%%%%%%%%%%%%%%%%%%%%%%%%%%%%%%
\section{Clauses}

Most of \iter's clauses will be familiar to |loop| programmers.  
(|loop| is an iteration macro that has been incorporated into Common
Lisp.  Seen Guy Steele's {\em Common Lisp, 2nd Edition}.)
In
nearly all cases they behave the same as their |loop| counterparts, so
a |loop| user can switch to \iter\ with little pain (and much gain).

All clauses with the standard keyword-argument syntax consist of two
parts: a ~required~ part, 
containing keywords that must be present and in the right order; and
an ~optional~ part, containing keywords that may be omitted and,
if present, may occur in any order.  In the descriptions below, the
parts are separated by the Lisp lambda-list keyword \opt.


\subsection{Drivers}

An iteration-driving clause
conceptually causes the iteration to go forward.  Driver clauses in
\iter\ allow iteration over numbers, lists, vectors, hashtables, packages,
files and streams.  Iteration-driving clauses must
appear at the top level of an \iter\ form; they cannot be nested
inside another clause.  The driver variable is updated at the point
where the driver clause occurs.  Before the clause is executed for the
first time, the value of the variable is undefined.

%Also, regardless of where the driver clause
%appears in the body, the driver variable is stepped at the top of the
%loop; hence it is stylistically preferable, though
%not required, to place driver clauses at the beginning of the \iter.

Multiple drivers may appear in a single \iter\ form, in which case all
of the driver variables are updated each time through the loop, in the
order in which the clauses appear.  The first driver to terminate will
terminate the entire loop.

In all cases, the value of the driver variable on exit from the loop,
including within the epilogue code (see the |finally| clause), is
undefined.

All the parameters of a driver clause are evaluated once, before the
loop begins.  Hence it is not possible to change the bounds or other
properties of an iteration by side-effect from within the loop.

With one exception, driver clauses begin with the word |for| (or
the synonym |as|) and mention an iteration variable, which is
given a binding within the \iter\ form.
The exception is |repeat|, which just executes a loop a
specified number of times:

\begin{clauses}

\clausex{repeat}{~n~}
Repeats the loop ~n~ times.  For example:
\begin{program}
(iter (repeat 100)
      (print "I will not talk in class."))
\end{program}
\cpar If $n \leq 0$, the
loop will never be executed.  If ~n~ is not an integer, the
actual number of executions will be $\lceil n \rceil$.

\end{clauses}

\pagebreak[3]

\subsubsection{Numerical Iteration}

\begin{clauses}
\clausex{for}{~var~ |\&sequence|} 
The general form for iterating over a sequence of numbers requires a
variable and, optionally, one or more keywords that provide the bounds
and step size of the iteration. The |\&sequence|
lambda-list keyword is a shorthand for these sequence keywords.  They are:
|from|, |upfrom|, |downfrom|, |to|, |downto|, |above|, |below| and |by|.
|from| provides the starting value for ~var~
and defaults to zero. |to| provides a final value and implies that the
successive values of ~var~ will be increasing; |downto| implies that
they will be decreasing.  The loop terminates when ~var~ passes
the final value (i.e. becomes smaller or larger than it, depending on
the direction of iteration); in other  words, the loop body
will never be 
executed for values of ~var~ past the final value.
|below| and |above| are similar to |to| and
|downto|, except that the loop terminates when ~var~ equals or passes
the final value.

\cpar If no final value is specified, the variable will be stepped
forever.  Using |from| or |upfrom| will result in increasing
values, while |downfrom| will give decreasing values.

\cpar On each iteration,
~var~ is incremented or decremented by the value of the sequence
keyword |by|, which defaults to 1.  It should always be a positive
number, even for downward iterations.

\cpar In the following examples, the sequence of numbers generated is shown
next to the clause.

\begin{program}
(for i upfrom 0) \yields\ 0 1 2 \ldots
(for i from 5)   \yields\ 5 6 7 \ldots    ; either from or upfrom is okay
(for i downfrom 0) \yields\ 0 -1 -2 \ldots
(for i from 1 to 3) \yields\ 1 2 3
(for i from 1 below 3) \yields\ 1 2
(for i from 1 to 3 by 2) \yields\ 1 3
(for i from 1 below 3 by 2) \yields\ 1
(for i from 5 downto 3) \yields\ 5 4 3
\end{program}

\end{clauses}

\subsubsection{Sequence Iteration}

There are a number of clauses for iterating over sequences.  In all of
them, the argument following |for| may be a list instead of a
symbol, in which case destructuring is performed.  See section
\ref{destructuring}. 

\begin{clauses}

\clause{|for| ~var~ |in| ~list~ \opt\ |by| ~step-function~}
\clindexx{for}{in}
~var~ is set to successive elements of list.  
~step-function,~ which 
defaults to |cdr|, is used to obtain the next sublist. 

\clause{|for| ~var~ |on| ~list~ \opt\ |by| ~step-function~}
\clindexx{for}{on}
~var~ is set to successive sublists of list. 
~step-function~ (default |cdr|) is used as in |for\dots in|.

\end{clauses}

\medskip

These two clauses use |atom| to test for the end
of a list.  Hence, given a list whose final |cdr| is not \nil,
they will silently ignore the last |cdr|.  Other choices are
|endp|,  which would signal an error, and |null|, which
would probably result in an error somewhere else.  If you wish to use
an end-test other than |atom|, set the variable
|iterate::*list-end-test*|\lispindex{*list-end-test*} to the name of
the desired function. 

\begin{clauses}
\clause{|for| ~var~ |in-vector| ~vector~ |\&sequence|}
\clindexx{for}{in-vector}
~var~ takes on successive elements from ~vector.~   The vector's
fill-pointer is observed.  Here and in subsequent clauses, the |\&sequence|
keywords include |with-index|, which takes a symbol as argument
and uses it for the index variable instead of an internally generated
symbol.  The other |\&sequence| keywords behave as in numerical
iteration, except that the default iteration bounds are the bounds of
the vector.  E.g. in |(for i in-vector v downto 3)|,
|i| will start off being bound to the last element in |v|,
and will  
be set to preceding elements down to and including the element with
index 3.

\clause{|for| ~var~ |in-sequence| ~seq~ |\&sequence|}
\clindexx{for}{in-sequence}
This uses Common Lisp's generalized sequence functions, |elt|
and |length|, to obtain elements and determine the length of
~seq.~   Hence it will work for any sequence, including lists, and will
observe the fill-pointers of vectors.

\clause{|for| ~var~ |in-string| ~string~ |\&sequence|}
\clindexx{for}{in-string}
~var~ is set to successive characters of ~string.~

\startitem
\Clause{|for| ~var~ |index-of-vector| ~vector~ |\&sequence|}
\clindexx{for}{index-of-vector}
\Clause{|for| ~var~ |index-of-sequence| ~sequence~ |\&sequence|}
\clindexx{for}{index-of-sequence}
\Clause{|for| ~var~ |index-of-string| ~string~ |\&sequence|}
\clindexx{for}{index-of-string}
\finishitem
~var~ is set to successive indices of the sequence.
These clauses avoid the overhead of accessing the sequence elements
for those applications where they do not need to be examined, or are
examined rarely.  They admit all the optional keywords of the other
sequence drivers except the (redundant) |with-index| keyword.

\clause{|for| ~(key value)~ |in-hashtable| ~table~}
\clindexx{for}{in-hashtable}

~key~ and ~value,~ which must appear as shown in a list
and may be destructuring templates, are set to the keys and values of
~table~.  If ~key~ is |nil|, then the hashtable's keys will be ignored;
similarly for ~value~.
The order in which 
elements of ~table~ will be retrieved is unpredictable.

\clause{|for| ~var~ |in-package| ~package~ \opt\ |external-only| ~ext~}
\clindexx{for}{in-package}
Iterates over all the symbols in ~package,~ or over only the
external symbols if ~ext~ is specified and non-|nil|.  ~ext~ is not
evaluated.  The same symbol may appear more than once.

\clause{|for| ~var~ |in-packages| ~packages~ \opt\ |having-access| ~symbol-types~}
\clindexx{for}{in-packages}
Iterates over all the symbols from the list of packages denoted by the
descriptor ~packages~ and having accessibility (or visibility) given by
~symbol-types~.  This defaults to the list |(:external :internal :inherited)|
and is not evaluated.  ~var~ must be a list of up to three variables: in each
iteration, these will be set to a symbol, its access-type and package (as per
|with-package-iterator| in ANSI-CL).  The same symbol may appear more than
once.

\end{clauses}

\begin{clauses}

\clause{|for| ~var~ |in-file| ~name~ \opt\ |using| ~reader~}
\clindexx{for}{in-file}
Opens the file ~name~ (which may be a string or pathname) for input,
and iterates 
over its contents. ~reader~ defaults to |read|, so by default ~var~
will be bound to the successive forms in the file.  The \iter\ body is
wrapped in an unwind-protect to ensure that the file is closed no
matter how the \iter\ is exited.

\clause{|for| ~var~ |in-stream| ~stream~ \opt\ |using| ~reader~}
\clindexx{for}{in-stream}
Like |for\dots in-file|, except that ~stream~ should be an existing
stream object that supports input operations.

\end{clauses}

\subsubsection{Generalized Drivers}

These are primarily useful for writing drivers that can also be used
as generators (see section \ref{generators}, below).
 
\begin{clauses}

\clause{|for| ~var~ |next| ~expr~}
\clindexx{for}{next}

~var~ is set to ~expr~ each time through the loop.  Destructuring is
performed.  When the clause is used as a generator, ~expr~ is the code
that is executed when |(next ~var~)| is 
encountered (see section \ref{generators}, below).
~expr~ should compute the first value for ~var~, as well
as all subsequent values, and is responsible for terminating the loop.
For compatibility with future versions of \iter, this termination
should be done with |terminate|\clindex{terminate}, which can be
considered a synonym for |finish| (see section \ref{control-flow}).

\cpar As an example, the following clauses are equivalent to |(for
i from 1 to 10)|:
\begin{program}
(initially (setq i 0))
(for i next (if (> i 10) (terminate) (incf i)))
\end{program}

\clause{|for| ~var~ |do-next| ~form~}
\clindexx{for}{do-next}
~form~ is evaluated each time through the loop.  Its value is ~not~
set to ~var~; that is ~form's~ job. ~var~ is only present so that
\iter\ knows it is a driver variable.  \linebreak |(for ~var~ next
~expr~)| is equivalent to |(for ~var~ do-next (dsetq ~var~ ~expr~))|.
(See section \ref{destructuring} for an explanation of |dsetq|.) 

\end{clauses}

\subsubsection{Generators}
\label{generators}

In all of the above clauses, the driver variable is updated on each
iteration.
Sometimes it is desirable to have greater control over updating.  
For instance, consider the problem of associating numbers, in
increasing order and with no gaps, with the
\nonnil\ elements of a list.  One obvious first pass at writing this is:

\begin{program}
(iter (for el in list)
      (for i upfrom 1)
      (if el (collect (cons el i))))
\end{program}
But on the list |(a b nil c)| this produces |((a . 1) (b . 2) (c .
4))| instead of the desired |((a . 1) (b . 2) (c . 3))|.  The problem
is that |i| is incremented each time through the loop, even when |el|
is |nil|.  

The problem could be solved elegantly if we could step |i| only when
we wished to.  This
can be accomplished for any \iter\ driver by writing |generate| (or
its synonym |generating|)
instead of |for|.  Doing so produces a ~generator~---a driver whose
values are yielded explicitly.  To obtain the next value of a
generator variable ~v~, write \linebreak |(next ~v~)|.  The value of a |next|
form is the next value of ~v~, as determined by its associated driver
clause.  |next| also has the side-effect of updating ~v~ to that
value.  If there is no next value, |next| will terminate the loop,
just as with a normal driver.


Using generators, we can now write our example like this:
\begin{program}
(iter (for el in list)
      (generate i upfrom 1)
      (if el (collect (cons el (next i)))))
\end{program}
Now |i| is updated only when |(next i)| is executed, and this occurs
only when |el| is \nonnil.

To better understand the relationship between ordinary drivers and
generators, observe that we can rewrite an ordinary driver using its
generator form immediately followed by |next|, as this example shows:
\begin{program}
(iter (generating i from 1 to 10)
      (next i)
      ...)
\end{program}
Provided that the loop body contains no |(next i)| forms, this will
behave just as if we had written |(for i from 1 to 10)|.

We can still refer to a driver variable ~v~ without using |next|; in
this case, its value is that given to it by the last evaluation of
|(next ~v~)|.  Before |(next ~v~)| has been called the first time, the
value of ~v~ is undefined.  

This semantics is more flexible than
one in which ~v~ begins the loop bound to its first value and calls
of |next| supply subsequent values, because it means the loop will not
terminate too soon if the generator's sequence is empty.  For
instance, consider the following code, which tags \nonnil\ elements of
a list using a list of tags, and also counts the null elements.
(We assume there are at least as many tags as \nonnil\ elements.)
\begin{program}
(let* ((counter 0)
       (tagged-list (iter (for el in list)
                          (generating tag in tag-list) 
                          (if (null el)
                              (incf counter)
                              (collect (cons el (next tag)))))))
  ...)
\end{program}
It may be that there are just as many tags as non-null elements of
|list|.  If all the elements of |list| are null, we still want the
counting to proceed, even though |tag-list| is \nil.  If |tag| had to be
assigned its first value before the loop begins, we would have had to
terminate the loop before the first iteration, since when |tag-list|
is \nil, |tag| 
has no first value.  With the existing semantics, however, |(next
tag)| will never execute, so the iteration will cover all the elements of
|list|. 
        
When the ``variable'' of a driver clause is actually a destructuring
template containing several variables, all the variables are eligible
for use with |next|.  As before, |(next ~v~)| evaluates to ~v's~ next
value; but the effect is to update all of the template's variables.
For instance, the following code will return the list |(a 2 c)|.
\begin{program}
(iter (generating (key . item) in '((a . 1) (b . 2) (c . 3)))
      (collect (next key))
      (collect (next item)))
\end{program}

Only driver clauses with variables can be made into generators.  This
includes all clauses mentioned so far except for |repeat|.  It does
~not~ include |for\dots previous|, |for\dots =|,
|for\dots initially\dots then| or |for\dots first\dots then| (see
below). 

\subsubsection{Previous Values of Driver Variables}

Often one would like to access the value of a variable on a previous
iteration.  \iter\ provides a special clause for accomplishing this.

\begin{clauses}

\clause{|for| ~pvar~ |previous| ~var~ \opt\ |initially| ~init~
|back| ~n~}
\clindexx{for}{previous}
Sets ~pvar~ to the previous value of ~var,~ which should be a driver
variable, a variable from another |for\dots previous| clause, or a
variable established by a |for\dots =|, 
|for\dots initially\dots then| or |for\dots first\dots then| clause
(see section \ref{setting}).
Initially, ~pvar~ is given the value ~init~ (which defaults to \nil).  
The ~init~ expression will be moved outside the loop body, so it
should not depend on anything computed within the loop.
~pvar~ retains the value of ~init~ until ~var~ is set to its second
value, at which point ~pvar~ is set to ~var's~ first value; and so on.  

\cpar The
argument ~n~ to |back| must be a constant, positive integer, and
defaults to 1.  It determines how many iterations back ~pvar~ should
track ~var~.  For example, when ~n~ is 2, then ~pvar~ will be assigned
~var's~ first value when ~var~ is set to its third value.

\cpar A |for\dots previous| clause may occur after or before its
associated driver clause. |for\dots previous| works with generators as
well as ordinary drivers.

\pagebreak[3]

\cpar Example:
\begin{program}
(iter (for el in '(1 2 3 4))
      (for p-el previous el)
      (for pp-el previous p-el initially 0)
      (collect pp-el))
\end{program}
This evaluates to |(0 0 1 2)|.  It could have been written more
economically as 
\begin{program}
(iter (for el in '(1 2 3 4))
      (for pp-el previous el back 2 initially 0)
      (collect pp-el))
\end{program}


\end{clauses}

\subsection{Variable Binding and Setting}
\label{setting}

Several clauses exist for establishing new variable bindings or for
setting variables in the loop.  They all support destructuring.

\begin{clauses}

\clausex{with}{~var~ \opt\ |=| ~value~}
Causes ~var~ to be bound to value before the loop body is entered.
If ~value~ is not supplied, ~var~ assumes a default
binding, which will be 
\nil\ in the absence of declarations.  Also, if ~value~ is not
supplied, no destructuring is performed; instead, ~var~ may be a list of
symbols, all of which are given default 
bindings.  If ~value~ is supplied, ~var~ is bound to it, with
destructuring.  

\cpar Because |with|
creates bindings whose scope includes the entire \iter\ form, it is
good style to put all |with| clauses at the beginning.

\cpar Successive occurrences of |with| result in sequential
bindings (as with 
|let*|).  There is no way to obtain parallel bindings; see
section \ref{bindings} for a rationale.  


\clause{|for| ~var~ |=| ~expr~}
\clindexx{for}{=}
On each iteration, ~expr~ is evaluated and ~var~ is set
to its value. 

\cpar This clause may appear to do the same thing as |for\dots next|.
In fact, they are quite different.  |for\dots =| provides only three
services: it sets up a binding for ~var~, sets it to ~expr~ on each
iteration, and makes it possible to use |for\dots previous| with
~var~.  |for\dots next| provides these services in addition to the
ability to turn the driver into a generator.
%Also, the code which sets ~var~ appears in the loop body in the same
%place as the |for\dots =| clause; the code for |for\dots next| appears
%at the top of the loop, as with other drivers (except when being used
%as a generator).

\clause{|for| ~var~ |initially| ~init-expr~ |then| ~then-expr~}
\clindexxx{for}{initially}{then}
Before the loop begins, ~var~ is set to ~init-expr;~ on all
iterations after the first it is set to ~then-expr.~  This clause
must occur at top-level.  ~init-expr~ will be moved outside the loop
body and ~then-expr~ will be moved to the end of the loop body, so
they are subject to code motion problems (see section
\ref{code-movement}). 

\cpar This clause may appear to be similar to |for\dots next|, but in
fact they differ significantly.  |for\dots initially\dots then| is
typically used to give ~var~ its first value before the loop begins,
and subsequent values on following iterations.  This is incompatible
with generators, whose first value and subsequent values must all be
computed by |(next ~var~)|.  Also, the update of ~var~ in 
|for\dots initially\dots then| does not occur at the location of the
clause. 
Use |for\dots initially\dots then| for
one-shot computations where its idiom is more convenient, but use
|for\dots next| for extending \iter\ with new drivers (see section
\ref{extend}).

\clause{|for| ~var~ |first| ~first-expr~ |then| ~then-expr~}
\clindexxx{for}{first}{then}
The first time through the loop, ~var~ is set to ~first-expr;~ on
subsequent iterations, it is set to ~then-expr.~  This differs from
|for\dots initially| in that ~var~ is set to ~first-expr~
inside the loop body, 
so ~first-expr~ may depend on the results of other clauses.  For
instance, 
\begin{program}
(iter (for num in list)
      (for i first num then (1+ i))
      ...)
\end{program}
will set |i| to the first element of |list| on the first
iteration, whereas 
\begin{program}
(iter (for num in list)
      (for i initially num then (1+ i))
      ...)
\end{program}
is probably erroneous; |i| will be bound to |num|'s
default binding (usually \nil) for the first iteration.

\end{clauses}

\begin{note}{Compatibility note}
|loop|'s |for\dots =| works like \iter's, but |loop| used the syntax
|for\dots =\dots then| to mean |for\dots initially\dots then|.  It was
felt that these two operations were sufficiently different to warrant
different keywords.

Also, the |for| in the above three clauses is misleading,
since none is true driver (e.g. none has a corresponding |generate|
form).  |setting| would have been a better choice, but |for| was used
to retain some compatibility with |loop|.  
\end{note}


%%%%%%%%%%%%%%%%%%%%%%%%%%%%%%%%
\subsection{Gathering Clauses}
Many of \iter's clauses accumulate values into a variable, or set a
variable under certain conditions.  At the end of the loop, this
variable contains the desired result.  All these clauses have an
optional |into| keyword, whose argument should be a symbol.  If the
|into| keyword is not supplied, the accumulation variable will be
internally generated and its value will be returned at the end of the
loop; if a variable is specified, that variable is used for the
accumulation, and is not returned as a result---it is up to the user
to return it explicitly, in the loop's epilogue code (see |finally|).
It is safe to examine the accumulation variable during the loop, but
it should not be modified.

These clauses all begin with a verb.  When the verb does
not conflict with an existing Common Lisp function, then it may be
used in either its infinitival or present-participle form (e.g. |sum|,
|summing|).  However, when there is a conflict with Common Lisp, only
the present-participle form may be used (e.g. |unioning|).  This is to
prevent \iter\ clauses from clashing with Common Lisp
functions. 

%although these clauses are described as ``producing a value,'' it is a
%mistake to think of the lisp list representing the clause as a
%value-producing form in the usual way.  clauses may legally be written
%where a value is expected, e.g. |(setq x (sum i))|, but the
%lisp value of a clause in such a context is undefined.

\subsubsection{Reductions}

~Reduction~ is an extremely common iteration pattern in which
the results of successive applications of a
binary operation are accumulated.
For example, a loop that computes the sum of the
elements of a list is performing a reduction with the addition
operation.  This could be written in Common Lisp as  |(reduce \#'+
list)| or with \iter\ as
\begin{program}
(iter (for el in list)
      (sum el))
\end{program}

\begin{clauses}

\clausex{sum}{~expr~ \opt\ |into| ~var~}
Each time through the loop, ~expr~ is evaluated and added to a
variable, which is bound initially to zero.  If ~expr~ has a type,
it is ~not~ used as the type of the sum variable, which is always
|number|.  To get the result variable to be of a more specific
type, use an explicit variable, as in
\begin{program}
(iter (for el in number-list)
      (sum el into x)
      (declare (fixnum x))
      (finally (return x)))
\end{program}

\clausex{multiply}{~expr~ \opt\ |into| ~var~}
Like |sum|, but the initial value of the result variable is $1$,
and the variable is updated by multiplying ~expr~ into it.

\clausex{counting}{~expr~ \opt\ |into| ~var~}
~expr~ is evaluated on each iteration.  If it is \nonnil, the
accumulation variable, initially zero, is incremented.

\startitem
\Clausex{maximize}{~expr~ \opt\ |into| ~var~}
\Clausex{minimize}{~expr~ \opt\ |into| ~var~}
\finishitem
~expr~ is evaluated on each iteration and its extremum (maximum or
minimum) is stored in the accumulation variable.  If ~expr~ is never
evaluated, then the result is \nil\ (if the accumulation variable is
untyped) or $0$ (if it has a numeric type).

\clausex{reducing}{~expr~ |by| ~func~ \opt\ |initial-value| ~init-val~ |into| ~var~}
This is a general way to perform reductions. ~func~ should be a
function of two arguments, the first of which will be the value
computed so far and
the second of which will be the value of ~expr~.  It should return the
new value.
|reducing| is roughly equivalent to the
Common Lisp |(reduce ~func~ ~sequence~ :key ~expr-function~)|, where
~expr-function~ is used to derive values from the successive elements
of ~sequence~.

\cpar If the |reducing| clause is never executed, the result is
undefined.

\cpar It is not necessary to provide an
initial value, but better code can be generated if one is supplied.
Regardless of its location in the \iter\ body, ~init-val~ will be
evaluated before the loop is entered, so it should not depend on any
value computed inside the \iter\ form.  

%\cpar if a ~var~ is not specified, you can get \iter\ to declare the
%type of the internal variable by putting a |the| expression around
%~func.~  see section \ref{types}.

\end{clauses}

%\begin{note}{implementation note}
%in principle, |maximize| and |minimize| can be thought of
%as reductions where the initial value is the smallest (or largest)
%value that the accumulation variable can assume.  because lisp's
%bignums can represent arbitrary integers, these clauses cannot be
%implemented as reductions in general.  if, however, the type of 
%~expr~ or ~var~ can be determined to be a fixnum
%or a float, \iter\ will implement the clause as a true reduction,
%using one of the constants |most-negative-fixnum|, |%most-positive-fixnum|, 
%|most-negative-short-float|, etc. as appropriate.
%\end{note}

\subsubsection{Accumulations}

All the predefined accumulation clauses add values to a sequence.  If
the sequence is a list, they all 
have the property that the partial list is kept in the correct order
and available for inspection at any point in the loop.

\begin{clauses}

\clausex{collect}{~expr~ \opt\ |into| ~var~ |at| ~place~ |result-type| ~type~}
Produces a sequence of the values of ~expr~ on each iteration. ~place~
indicates where the next value of ~expr~ is added to the list and may
be one of the symbols |start|, |beginning| (a synonym for |start|) or
|end|.  The symbol may be quoted, but need not be.  The default is
|end|.  For example,
\begin{program}
(iter (for i from 1 to 5)
      (collect i))
\end{program}
produces |(1 2 3 4 5)|, whereas
\begin{program}
(iter (for i from 1 to 5)
      (collect i at beginning))
\end{program}
produces |(5 4 3 2 1)| (and is likely to be faster in most Common Lisp
implementations).

\cpar If ~type~ is provided, it should be a subtype of |sequence|.
The default is |list|.  Specifying a type other than |list| will
result in |collect| returning a sequence of that type.  ~However,~ the
type of the sequence being constructed when inside the loop body is
undefined when a non-|list| type is specified.  (As with ~place~,
quoting ~type~ is optional.)

\clausex{adjoining}{~expr~ \opt\ |into| ~var~ |test| ~test~ |at| ~place~ |result-type| ~type~}
Like |collect|, but only adds the value of ~expr~ if it is not
already present.  ~test,~ which defaults to |\#'eql|, is
the test to be used with |member|.

\startitem
\Clausex{appending}{~expr~ \opt\ |into| ~var~ |at| ~place~}
\Clausex{nconcing}{\ ~expr~ \opt\ |into| ~var~ |at| ~place~}
\Clausex{unioning}{\ ~expr~ \opt\ |into| ~var~ |test| ~test~ |at| ~place~}
\Clausex{nunioning}{~expr~ \opt\ |into| ~var~ |test| ~test~ |at| ~place~}
\finishitem
These are like |collect|, but behave like the Common Lisp functions
|append|, |nconc|, |union| or |nunion|.  
As in Common Lisp, they work only on lists.  Also as in Common Lisp,
|unioning| and |nunioning| assume that the value of ~expr~ contains no
duplicates. 

\clausex{accumulate}{~expr~ |by| ~func~ \opt\ |initial-value| ~init-val~ |into| ~var~}
This is a general-purpose accumulation clause. ~func~ should be a
function of two arguments, the value of ~expr~ and the value
accumulated so far in the iteration, and it should return the updated
value.  If no initial value is supplied, \nil\ is used.  

%\cpar If a ~var~ is not specified, you can get \iter\ to declare the
%type of the internal variable by putting a |the| expression around
%~func.~  see section \ref{types}.

\cpar The differences between |accumulate| and |reducing| are slight.
One difference is that the functions take their arguments in a
different order.  Another is that in the absence of ~init-val~,
|accumulate| will use \nil, whereas |reducing| will generate different
code that avoids any dependence on the initial value.
The reason for having both clauses is that one usually
thinks of reductions (like |sum|) and accumulations (like |collect|)
as different beasts.

\end{clauses}


\subsubsection{Finders}

A ~finder~ is a clause whose value is an expression that meets some
condition.

\begin{clauses}

\clause{|finding| ~expr~ |such-that| ~test~ \opt\ |into| ~var~ |on-failure| ~failure-value~}
\clindexx{finding}{such-that}
If ~test~ (which is an expression) ever evaluates to \nonnil, the loop
is terminated, the 
epilogue code is run and the value of ~expr~ is returned.  Otherwise,
\nil\ (or ~failure-value,~ if provided) is returned.  If ~var~ is
provided, it will have either the \nonnil\ value of ~expr~ or
~failure-value~ when the epilogue code is run.

\cpar As a special case, if the ~test~ expression is a sharp-quoted
function, then it is applied to ~expr~ instead of being simply
evaluated.  E.g. |(finding x such-that \#'evenp)| is equivalent to
|(finding x such-that (evenp x))|.  

%\cpar although ~test~ need have nothing to do with ~%expr~ as in
%|(finding j such-that (> i 3))|, it usually 
%will: |(finding (length el) such-that (oddp (length el)))|.  to
%avoid performing the |length| computation twice, you could write
%|(finding (length el) such-that \#'oddp)| or |(finding (length
%el) such-that 'oddp)|; for these cases, \iter\ generates code that
%executes ~expr~ only once.  the code for |\#'oddp|
%is slightly different from that for {\lisp 'oddp}; see the discussion
%under {\lisp for\dots in} and {\lisp for\dots on}.

\cpar |On-failure| is a misnomer. Because it is always evaluated, it behaves
more like the default third argument to the |gethash| function. As a result,
|on-failure (error "Not found")| makes no sense. Instead, the clauses |leave|
or |thereis| can be used in conjunction with |finally| as follows:
\begin{program}
(iter (for x in '(1 2 3))
      (if (evenp x) (leave x))
      (finally (error "not found")))
\end{program}

\cpar This clause may appear multiple times when all defaults are
identical. It can also be used together with either |always|/|never| or
|thereis| if their defaults match. More specifically, |on-failure nil| is
compatible with |thereis|, while |on-failure t| is compatible with |always|
and |never| clauses.
\begin{program}
(iter (for i in '(7 -4 2 -3))
      (if (plusp i)
	  (finding    i  such-that (evenp i))
          (finding (- i) such-that (oddp i))))
\end{program}

\startitem
\Clause{|finding| ~expr~ |maximizing| ~m-expr~ \opt\ |into| ~var~}
\clindexx{finding}{maximizing}
\Clause{|finding| ~expr~ |minimizing| ~m-expr~ \opt\ |into| ~var~}
\clindexx{finding}{minimizing}
\finishitem
Computes the extremum (maximum or minimum) value of ~m-expr~ over all
iterations, and returns the value of ~expr~ corresponding to the
extremum.  ~expr~ is evaluated inside the loop at the time the new
extremum is established.  If ~m-expr~ is never evaluated (due to, for
example, being embedded in a conditional clause), then the returned
value depends on the type, if any, of ~expr~ (or ~var,~ if
one is supplied).  If there is no type, the returned
value will be \nil; if the type is numeric, the returned value will be
zero.

\cpar For these two clauses, ~var~ may be a list of two
symbols; in that case, the first is used to record ~expr~ and
the second, ~m-expr.~  

\cpar As with |finding\dots such-that|, if ~m-expr~ is a sharp-quoted
function, then it is called on ~expr~ instead of being evaluated.

\end{clauses}

\subsubsection{Boolean Tests}

\begin{clauses}

\clausex{first-iteration-p}{}
Returns |t| in the first cycle of the loop, otherwise \nil.

\clausex{first-time-p}{}
Returns |t| the first time the expression is evaluated, and then \nil\ forever.
This clause comes handy when printing (optional) elements separated
by a comma:

\begin{program}
(iter (for el in '(nil 1 2 nil 3))
      (when el
        (unless (first-time-p)
          (princ ", "))
        (princ el)))
\end{program}
produces |"1, 2, 3"|.

\end{clauses}

\subsubsection{Aggregated Boolean Tests}

\begin{clauses}

\clausex{always}{~expr~}
If ~expr~ ever evaluates to
\nil, then \nil\ is immediately returned; the epilogue code is not
executed.  If ~expr~ never evaluates to \nil, the epilogue code
is executed and the last value of ~expr~ (or |t| if ~expr~ was never
evaluated) is returned (whereas |loop| would constantly return |t|).

% mention last evaluated clause when multiple always clauses?

\clausex{never}{~expr~}
Like |(always (not ~expr~))|, except it does not influence the last
value returned by a possible other |always| clause. That is,
\begin{program}
(iter (repeat 2)
      (always 2)
      (never nil)) \yields 2 ; not t
\end{program}

\clausex{thereis}{~expr~}
If ~expr~ is ever \nonnil,
its value is immediately returned without running epilogue code.
Otherwise, the epilogue code is performed and \nil\ is returned.

This clause cannot be used together with |always| or |never|, because their
defaults are opposed (similarly, |(loop always 3 thereis nil)| refuses to
compile in some implementations of |loop|).

\end{clauses}


\subsection{Control Flow}
\label{control-flow}
Several clauses can be used to alter the usual flow of control in a loop.

Note: the clauses of this and subsequent sections don't adhere to \iter's
usual syntax, but instead use standard Common Lisp syntax.  Hence the
format for describing syntax subsequently is
like the standard format used in the Common Lisp manual, not like the
descriptions of clauses above.

\begin{clauses}

\clausex{finish}{}
Stops the loop and runs the epilogue code.

%for example:
%
%\begin{program}
%(iter (with answer = nil)
%         (initially (make-a-mess))
%         (for i from 1 to 10)
%         (when (correct? i)
%           (setq answer i)
%           (finish))
%         (finally (cleanup)))
%\end{program}
%
%this code will execute |cleanup| whether or not the test |(correct?
%i)| ever succeeds. 
%the (more elegant) formulation,
%\begin{program}
%(iter (initially (make-a-mess))
%         (for i from 1 to 10)
%         (finding i such-that (correct? i))
%         (finally (cleanup)))
%\end{program}
%would not execute |cleanup| if |(correct? i)| succeeded; it
%would do an immediate return.

\clausex{leave}{\opt\ ~value~}
Immediately returns ~value~ (default \nil) from the current \iter\
form, skipping the epilogue code.  Equivalent to using |return-from|.

\clausex{next-iteration}{}
Skips the remainder of the loop body and begins the next iteration of
the loop.

\clausex{while}{~expr~}
If ~expr~ ever evaluates to \nil, the loop is terminated and the
epilogue code executed.  Equivalent to |(if (not ~expr~) (finish))|.

\clausex{until}{~expr~}
Equivalent to |(if ~expr~ (finish))|.

\clausex{if-first-time}{~then~ \opt\ ~else~}
If this clause is being executed for the first time in this invocation
of the \iter\ form, then the ~then~ code is evaluated; otherwise the
~else~ code is evaluated.

\cpar |(for ~var~ first ~expr1~ then ~expr2~)| is almost equivalent to
\begin{program}
(if-first-time (dsetq ~var~ ~expr1~)
               (dsetq ~var~ ~expr2~))
\end{program}
The only difference is that the |for| version makes ~var~ available
for use with |for\dots previous|.

\end{clauses}

\subsection{Code Placement}
When fine control is desired over where code appears in a loop
generated by \iter, the following special clauses may be useful.
They are all subject to code-motion problems (see section
\ref{code-movement}).  

\begin{clauses}

\clausex{initially}{|\&rest| ~forms~}
The lisp ~forms~ are placed in the prologue section of the loop, where
they are executed once, before the loop body is entered.

\clausex{after-each}{|\&rest| ~forms~}
The ~forms~ are placed at the end of the loop body, where they
are executed after each iteration.  Unlike the other clauses in this
section, ~forms~ may contain \iter\ clauses.

\clausex{else}{|\&rest| ~forms~}
The lisp ~forms~ are placed in the epilogue section of the loop, where they
are executed if this |else| clause is never met during execution of the
loop and the loop terminates normally.

\clausex{finally}{|\&rest| ~forms~}
The lisp ~forms~ are placed in the epilogue section of the loop, where
they are executed after the loop has terminated normally.

\clausex{finally-protected}{|\&rest| ~forms~}
The lisp ~forms~ are placed in the second form of an unwind-protect
outside the loop.  They are always executed after the loop has
terminated, regardless of how the termination occurred.

\end{clauses}


%%%%%%%%%%%%%%%%%%%%%%%%%%%%%%%%%%%%%%%%%%%%%%%%%%%%%%%%%%%%%%%%
\section{Other Features}

\subsection{Multiple Accumulations}
\label{multiple}

\begin{sloppypar}
It is permitted to have more than one clause accumulate into the same
variable, as in the following:
\begin{program}
(iter (for i from 1 to 10)
      (collect i into nums)
      (collect (sqrt i) into nums)
      (finally (return nums)))
\end{program}
Clauses can only accumulate into the same variable if they are
compatible.  |collect|, |adjoining|, |appending|, |nconcing|,
|unioning| and |nunioning| are compatible with each other; |sum|,
|multiply| and |counting| are compatible; |always| and |never| are compatible;
|finding| \dots |such-that| is compatible with either |thereis| or |always|
and |never| when their defaults match; and |maximize| and |minimize| clauses
are compatible only with other |maximize| and |minimize| clauses,
respectively.

%note that the same variable ~cannot~ be both an accumulation
%variable and an ordinary variable; there can be only one variable with
%a given name within an \iter\ form. 
\end{sloppypar}

\subsection{Named Blocks}

Like Common Lisp |block|s, \iter\ forms can be given names.  The
name should be a single symbol, and it must be the first form in the
\iter.  The generated code behaves exactly like a named block; in
particular, |(return-from ~name~)| can be used to exit it:
\begin{program}
(iter fred
      (for i from 1 to 10)
      (iter barney
            (for j from i to 10)
            (if (> (* i j) 17)
                (return-from fred j))))
\end{program}
An \iter\ form that is not given a name is implicitly named \nil.

Sometimes one would like to write an expression in an inner \iter\ form,
but have it processed by an outer \iter\ form.  This is possible with
the |in| clause.  

\begin{clauses}

\clausex{in}{~name~ |\&rest| ~forms~}
Evaluates ~forms~ as if they were part of the \iter\ form named
~name~.  In other words, \iter\ clauses are processed by the \iter\
form named ~name,~ and not by any \iter\ forms that occur inside ~name.~

\cpar As an example, consider the problem of collecting a list of the
elements in a two-dimensional array.  The naive solution,
\begin{program}
(iter (for i below (array-dimension ar 0))
      (iter (for j below (array-dimension ar 1))
            (collect (aref ar i j))))
\end{program}
\noindent is wrong because the list created by the inner \iter\ is simply
ignored by the outer one.  But using |in| we can write:
\begin{program}
(iter outer (for i below (array-dimension ar 0))
      (iter (for j below (array-dimension ar 1))
            (in outer (collect (aref ar i j)))))
\end{program}
\noindent which has the desired result.

\end{clauses}

\subsection{Destructuring}
\label{destructuring}

In many places within \iter\ clauses where a variable is expected, a
list can be written instead.  In these cases, the value to be assigned
is ~destructured~ according to the pattern
described by the list.  As a simple example, the clause
\begin{program}
(for (key . item) in alist)
\end{program}
\noindent will result in |key| being set to the |car| of
each element in |alist|, and |item| being set to the |cdr|.  The
pattern list may be nested to arbitrary depth, and (as the example
shows) need not be terminated with \nil; the only requirement is that
each leaf be a bindable symbol (or \nil, in which case no binding is
generated for that piece of the structure).

Sometimes, you might like to do the equivalent of a
|multiple-value-setq| in a clause.  This 
``multiple-value destructuring'' can be expressed by writing 
\linebreak |(values $pat_1$ $pat_2 \ldots$)| for a destructuring
pattern, as in 
\begin{program}
(for (values (a . b) c d) = (three-valued-function ...))
\end{program}
\begin{sloppypar}
Note that the $pat_i$ can themselves be destructuring patterns (though
not multiple-value destructuring patterns).  You can't do multiple-value
destructuring in a |with| clause; instead wrap the whole \iter\
form in a |multiple-value-bind|.
\end{sloppypar}

\begin{note}{Rationale}
There are subtle interactions between variable declarations and
evaluation order that make the correct implementation of
multiple-value destructuring in a |with| somewhat tricky.
\end{note}

The destructuring feature of \iter\ is available as a separate
mechanism, using the |dsetq| macro:

\begin{clauses}
\defmacro{dsetq}{template expr}
Performs destructuring of ~expr~ using ~template~.   May be used
outside of an \iter\ form. Yields the primary value of ~expr~.

\end{clauses}

\subsection{On-line Help}

\begin{sloppypar}
There is a limited facility for on-line help, in the form of the 
|display-iterate-clauses| function.
\end{sloppypar}

\begin{clauses}

\defun{display-iterate-clauses}{\opt\ clause-spec}
Displays a list of \iter\ clauses.  If ~clause-spec~ is not
provided, all clauses are shown; if it is a symbol, all clauses
beginning with that symbol are shown; and if it is a list of symbols,
all clauses for which ~clause-spec~ is a prefix are shown.

\end{clauses}

\subsection{Parallel Binding and Stepping}
\label{bindings}

The parallel binding and stepping of variables is a feature that
\iter\ does ~not~ have.  This section attempts to provide a rationale.

We say that two variables are bound ~in parallel~ if neither
binding shadows the other.  This is the usual semantics of |let|
(as opposed to |let*|).  Similarly, we can say that iteration
variables are stepped in parallel if neither variable is updated
before the other, conceptually speaking; in other words, if the code
to update each variable can reference the old values of both variables.

|loop| allows parallel binding of variables and parallel stepping of
driver variables.  My view is that if you are depending on the 
serial/parallel distinction, you are doing something obscure.  If you
need to bind 
variables in parallel using |with|, then you must be using a
variable name that shadows a name in the existing lexical environment.
Don't do that.  The most common use for parallel stepping is to track
the values of variables on the previous iteration, but in fact this
does not require parallel stepping at all; the following will work:
\begin{program}
(iter (for current in list)
      (for prev previous current)
      ...)
\end{program}


%%%%%%%%%%%%%%%%%%%%%%%%%%%%%%%%%%%%%%%%%%%%%%%%%%%%%%%%%%%%%%%%
\section{Types and Declarations}
\label{types}

\subsection{Discussion}

Sometimes efficiency dictates that the types of variables be declared.
This type information needs to be communicated to \iter\ so it can
bind variables to appropriate values.  Furthermore, \iter\ must often
generate internal variables invisible to the user; there needs to be a
way for these to be declared.

As an example, consider this code, which will return the number of
odd elements in |number-list|:
\begin{program}
(iter (for el in number-list)
      (count (oddp el)))
\end{program}
In processing this form,
\iter\ will create an internal variable, let us call it |list17|, to
hold the successive |cdr|s of |number-list|, and  
will bind the variable to |number-list|.  It will also generate a
default binding for |el|; only inside the body of the loop will |el|
be set to the |car| of |list17|.  Finally, \iter\ will generate a
variable, call it |result|, to hold the result of the count, and will
bind it to zero.

When dealing with type declarations, \iter\ observes one simple rule:
~it will never generate a declaration unless requested to do so.~  The
reason is that such declarations might mask errors in compiled code by
avoiding error-checks; the resulting problems would be doubly hard to
track down because the declarations would be hidden from the
programmer.  Of course, a compiler might omit error-checks even in the
absence of declarations, though this behavior can usually be avoided,
e.g. by saying |(declaim (optimize (safety 3)))|.

So, the above \iter\ form will generate code with no declarations.
But say we wish to declare the types of |el| and the internal
variables |list17| and |result|.  How is this done?

Declaring the type of |el| is easy, since the programmer knows
the variable's name:
\begin{program}
(iter (for el in number-list)
      (declare (fixnum el))
      (counting (oddp el)))
\end{program}
\iter\ can read variable type declarations like this one.  Before
processing any clauses, it scans the entire top-level form for type
declarations 
and records the types, so that variable bindings can be performed
correctly.  In this case, |el| will be bound to zero
instead of \nil.  Also, \iter\ collects all the top-level declarations
and puts them at the begining of the generated code, so it is not
necessary to place all declarations at the beginning of an \iter\
form; instead, they can be written near the variables whose types they
declare.  

Since \iter\ is not part of the compiler, it will not know
about declarations that occur outside an \iter\ form; these
declarations must be repeated inside the form.

Here is another way we could have declared the type of |el|:
\begin{program}
(iter (for (the fixnum el) in number-list)
      (counting (oddp el)))
\end{program}
\iter\ extends the Common Lisp |the|\lispindex{the}
form to apply to variables as well as value-producing forms; anywhere
a variable is allowed---in a |with| clause, as the iteration
variable in a driver clause, as the |into| argument of an
accumulation clause, even inside a destructuring template---you can
write |(the ~type~ ~symbol~)| instead.

There is one crucial difference between using a |the| form and
actually declaring the variable: explicit declarations are always
placed in the generated code, but type information from a |the|
form is not turned into an actual declaration unless you tell \iter\
to do so using |iterate:declare-variables|.  See below.

\begin{sloppypar}
Declaring the types of internal variables is harder than declaring the
types of explicitly mentioned variables, since their names
are unknown.  You do it by declaring |iterate:declare-variables|
somewhere inside the top level of the \iter\ form.  (This will also
generate declarations for variables declared using |the|.)
\iter\ does not provide much selectivity here: it's all or none. 
And unfortunately, since \iter\ is not privy to compiler information
but instead reads declarations itself, it will not hear if you
|(declaim (iterate:declare-variables))|.  Instead, set the variable 
|iterate::*always-declare-variables*| to |t| at
compile-time, using |eval-when|.
\end{sloppypar}

To determine the appropriate types for internal variables, \iter\ uses
three sources of information:

\begin{itemize}

\item Often, the particular clause dictates a certain type for a
variable; \iter\ will use this information when available.  In the
current example, the variable |list17| will be given the type
|list|, since that is the only type that makes sense; and the
variable |result| will be given the type |fixnum|, on the
assumption that you will not be counting high enough to need bignums.
You can override this assumption only by using and explicitly declaring a
variable: 
\begin{verbatim}
(iter (declare (iterate:declare-variables))
      (for el in number-list)
      (count (oddp el) into my-result)
      (declare (integer my-result))
      (finally (return my-result)))
\end{verbatim}

\begin{sloppypar}
Other examples of the type assumptions that \iter\ makes are: type
|list| for |into| variables of collection clauses; type |list| for
expressions that are to be destructured; type |vector| for the
variable holding the vector in a |for\dots in-vector| clause, and
similarly for |string| and the |for\dots in-string| clause;
and the implementation-dependent type
|(type-of array-dimension-limit)| for the index and limit
variables generated by sequence iteration drivers like |for\dots
in-vector| and |for\dots in-string| (but not |for\dots in-sequence|,
because it may be used to iterate over a list). 
\end{sloppypar}

\item Sometimes, \iter\ will examine expressions and try to determine
their types in a simple-minded way.  If the expression is
self-evaluating (like a number, for instance), \iter\ knows that the
expression's type is the same as the type of the value it denotes, so
it can use that type.  If the expression is of the form |(the ~type~
~expr~)|, \iter\ is smart enough to extract ~type~ and use it.
However, the current version of \iter\ does not  
examine declarations of function result types or do any type
inference.  It will not determine, for 
example, that the type of |(+ 3 4)| is |fixnum|, or even
|number|.

\item In some cases, the type of an internal variable should match the
type of some other variable.  For instance, \iter\ generates an
internal variable for |(f x)| in the
clause |(for i from 1 to (f x))|, and in the absence of other
information will give it the same type as |i|.  If, however, the
expression had been written |(the fixnum (f x))|, then \iter\
would have given the internal variable the type |fixnum|
regardless of |i|'s type.  The type incompatibility errors that
could arise in this situation are not checked for.

\end{itemize}

Note that if you do declare |iterate:declare-variables|, then
\iter\ may declare user variables as well as internal ones if they do
not already have declarations, though only for variables that it
binds.  For instance, in this code:

\begin{program}
(iter (declare (iterate:declare-variables))
      (for i from 1 to 10)
      (collect i into var))
\end{program}
the variable |var| will be declared to be of type |list|.


\subsection{Summary}

\iter\ understands standard Common Lisp variable type declarations
that occur within an \iter\ form and
will pass them through to the generated code.  If the declaration
|(iterate:declare-variables)|\lispindex{declare-variables}
appears at the top level of an 
\iter\ form, or if 
|iterate::*always-declare-variables*|\lispindex{*always-declare-variables*} 
is \nonnil, then \iter\ will use the type information gleaned from user
declarations, self-evaluating expressions and |the| expressions,
combined with reasonable assumptions, to determine variable
types and declare them.


%%%%%%%%%%%%%%%%%%%%%%%%%%%%%%%%%%%%%%%%%%%%%%%%%%%%%%%%%%%%%%%%
\section{Problems with Code Movement}
\label{code-movement}

Some \iter\ clauses, or parts of clauses, result in code being
moved from the location of the clause to other parts of the loop.
Drivers behave this way, as do code-placement clauses like |initially|
and |finally|.  When using these clauses, there is a danger of writing
an expression that makes sense in its apparent location but will be
invalid or have a different meaning in another location.  For example:
\begin{program}
(iter (for i from 1 to 10)
      (let ((x 3))
        (initially (setq x 4))))
\end{program}
While it may appear that the |x| of |(initially (setq x 4))| is the
same as the |x| of |(let ((x 3)) \dots|, in fact they are not:
|initially| moves its code outside the loop body, so |x| would refer
to a global variable.  Here is another example of the same problem:
\begin{program}
(iter (for i from 1 to 10)
      (let ((x 3))
        (collect i into x)))
\end{program}
If this code were executed, |collect| would create a binding for its
|x| at the top level of the \iter\ form that the |let| will shadow.

Happily, \iter\ is smart enough to catch these errors; it
walks all problematical code to ensure that free variables are not
bound inside the loop body, and checks all variables it binds for the
same problem.

However, some errors cannot be caught:

\begin{program}
(iter (with x = 3)
      (for el in list)
      (setq x 1)
      (reducing el by \#'+ initial-value x))
\end{program}
|reducing| moves its |initial-value| argument to the initialization
part of the loop in order to produce more efficient code.  Since
\iter\ does not perform data-flow analysis, it cannot determine that
|x| is changed inside the loop; all it can establish is that |x| is
not bound internally.  Hence this code will not signal an
error and will use $3$ as the initial value of the reduction.

The following list summarizes all cases that are subject to these code
motion and variable-shadowing problems.
\begin{itemize}
\item Any variable for which \iter\ creates a binding, including those
used in |with| and the |into| keyword of many clauses.

\begin{sloppypar}
\item The special clauses which place code: |initially|, |after-each|, |else|,
|finally| and |finally-protected|. 
\end{sloppypar}

\item The variables of a |next| or |do-next| form.

\item The |initially| arguments of |for\dots initially\dots then| and
|for\dots previous|. 

\item The |then| argument of |for\dots initially\dots then|.

\item The |initial-value| arguments of |reducing| and |accumulate|.

\item The |on-failure| argument of |finding\dots such-that|.
\end{itemize}

%%%%%%%%%%%%%%%%%%%%%%%%%%%%%%%%%%%%%%%%%%%%%%%%%%%%%%%%%%%%%%%%
\section{Differences Between {\tt Iterate} and {\tt Loop}}
|loop| contains a great deal of complexity which \iter\ tries to
avoid.  Hence many esoteric features of |loop| don't exist in \iter.
Other features have been carried over, but in a cleaned-up form.  
And of course, many new features have been added; they are not
mentioned in this list.

\begin{itemize}

\item \iter's syntax is more Lisp-like than |loop|'s, having a higher
density of parens.

\item The current implementation of \iter, unlike the current version
of  |loop| (as documented in {\em Common Lisp, 2nd Ed.\/}), is
extensible (see section \ref{extend}).  

\item |loop| puts the updates of all driver variables at the top of
the loop; \iter\ leaves them where the driver clauses appear.

\item While for the most part \iter\ clauses that resemble |loop| clauses
behave similarly, there are some differences.  For instance, there is
no |for\dots =\dots then| in \iter; instead use 
|for\dots initially\dots then|.  

\item |loop| binds the variable |it| at certain times
to allow pseudo-English expressions like |when ~expr~ return it|. 
In \iter, you must bind ~expr~ to a variable yourself.  Note that
|when ~expr~ return it| is like |thereis ~expr~| except that the latter is an
accumulation clause and therefore competes with other accumulations
(remember section \ref{multiple} above).

% repeat different behaviour of |always| clause here?

\item |loop| has a special |return| clause, illustrated in the
previous item.  \iter\ doesn't need one,  since an ordinary Lisp
|return| has the same effect.

\item |loop| allows for parallel binding and stepping of iteration
variables.  \iter\ does not.  (See section \ref{bindings}.)

\item |loop| and \iter\ handle variable type declarations very
differently.  |loop| provides a special syntax for declaring variable
types, and does not examine declarations.  Moreover, the standard
implementation of |loop| will
generate declarations when none are requested.
\iter\ parses standard Common Lisp type declarations, and will never
declare a variable itself unless declarations are 
specifically requested.

\end{itemize}

%%%%%%%%%%%%%%%%%%%%%%%%%%%%%%%%%%%%%%%%%%%%%%%%%%%%%%%%%%%%%%%%
\section{Rolling Your Own}
\label{extend}

\subsection{Introduction}

\iter\ is extensible---you can write new clauses that embody new
iteration patterns.  You might want to write a new driver clause for a
data structure of your own, or you might want to write a clause that
collects or manipulates elements in a way not provided by \iter. 

This section describes how to write clauses for \iter.  Writing a
clause is like writing a macro.  In fact, writing a clause ~is~
writing a macro: since \iter\ code-walks its body and macroexpands,
you can add new abstractions to \iter\ with good old |defmacro|.

Actually, there are two extensions you can make to \iter\ that are
even easier than writing a macro.  They are adding a synonym for an
existing clause and defining a driver clause for an indexable
sequence.  These can be done with |defsynonym| and
|defclause-sequence|, respectively.  See section \ref{aids}, below.

The rest of this section explains how to write macros that expand into
\iter\ clauses.
Here's how you could add a simplified version of \iter's |multiply|
clause, if \iter\ didn't already have one:
\begin{program}
(defmacro multiply (expr)
  `(reducing ,expr by \#'* initial-value 1))
\end{program}

If you found yourself summing the square of an expression often, you
might want to write a macro for that.  A first cut might be
\begin{program}
(defmacro sum-of-squares (expr)
  `(sum (* ,expr ,expr)))
\end{program}
but if you are an experienced macro writer, you will realize that this
code will evaluate ~expr~ twice, which is probably a bad idea.  A
better version would use a temporary:
\begin{program}
(defmacro sum-of-squares (expr)
  (let ((temp (gensym)))
    `(let ((,temp ,expr))
       (sum (* ,temp ,temp)))))
\end{program}
Although this may seem complex, it is just the sort of thing you'd
have to go through to write any macro, which illustrates the point of
this section: if you can write macros, you can extend \iter.

Our macros don't use \iter's keyword-argument
syntax.  We could just use keywords with |defmacro|, but we would
still not be using \iter's clause indexing mechanism.  Unlike Lisp,
which uses just the first symbol of a form to determine what function
to call, \iter\ individuates clauses by the list of required keywords.
For instance, |for\dots in| and |for\dots in-vector| are different clauses
implemented by distinct Lisp functions.  

To buy into this indexing scheme, as well as the keyword-argument
syntax, use |defmacro-clause|:

\begin{clauses}

\defmacro{defmacro-clause}{arglist |\&body| body}
Defines a new \iter\ clause.  ~arglist~ is a list of symbols which are
alternating keywords and arguments.  \opt\ may be used, and the list
may be terminated by |\&sequence|.  ~body~ is an ordinary macro body,
as with |defmacro|.  If the first form of ~body~ is a string, it is
considered a documentation string and will be shown by
|display-iterate-clauses|. |defmacro-clause| will signal an error if 
defining the clause would result in an ambiguity.  E.g. you cannot
define the clause |for\dots from| because there would be no way to
distinguish it from a use of the |for| clause with optional keyword |from|.

\end{clauses}

\medskip

Here is |multiply| using |defmacro-clause|.  The keywords are capitalized
for readability.
\begin{program}
(defmacro-clause (MULTIPLY expr \opt\ INTO var)
  `(reducing ,expr by \#'* into ,var initial-value 1))
\end{program}
You don't have to worry about the case when |var| is not supplied; for
any clause with an |into| keyword, saying |into nil| is equivalent to
omitting the |into| entirely.

As another, more extended example, consider the fairly common
iteration pattern that 
involves finding the sequence element that maximizes (or minimizes) some
function.  \iter\ provides this as |finding\dots maximizing|, but it's
instructive to see how to write it.
Here, in pseudocode, is how you might write such a loop for
maximizing a function F:

\begin{quote}
\begin{tabbing}
set variable MAX-VAL to NIL; \\
set variable WINNER to NIL; \\
for\=\ each element EL in the sequence \\
\>  if\=\ MAX-VAL\=\ is NIL or F(EL) $>$ MAX-VAL then \\
\>\>    set MAX-VAL to F(EL); \\
\>\>    set WINNER to EL; \\
\>  end if; \\
end for; \\
return WINNER.
\end{tabbing}
\end{quote}

Here is the macro:
\begin{program}
(defmacro-clause (FINDING expr MAXIMIZING func \opt\ INTO var)
  (let ((max-val (gensym))
        (temp1 (gensym))
        (temp2 (gensym))
        (winner (or var iterate::*result-var*)))
    `(progn 
       (with ,max-val = nil)
       (with ,winner = nil)
       (cond 
        ((null ,max-val)
         (setq ,winner ,expr)
         (setq ,max-val (funcall ,func ,winner))
        (t
         (let* ((,temp1 ,expr)
                (,temp2 (funcall ,func ,temp1)))
           (when (> ,temp2 ,max-val)
             (setq ,max-val ,temp2)
             (setq ,winner ,temp1))))))
       (finally (leave ,winner)))))
\end{program}
Note that if no |into| variable is supplied, we use
|iterate::*result-var*|, which contains the internal variable into
which all clauses place their results.  If this variable is bound by
some clause, then \iter\ will return its value automatically;
otherwise, \nil\ will be returned.  

\subsection{Writing Drivers}

In principle, drivers can be implemented just as easily as other
\iter\ clauses.  In practice, they are a little harder to get right. 
As an example, consider writing a driver that
iterates over all the 
elements of a vector, ignoring its fill-pointer.  |for\dots in-vector|
won't work for this, because it observes the fill-pointer.  It's necessary to
use |array-dimension| instead of |length| to obtain the size of the
vector.  Here is one approach:

\begin{program}
(defmacro-clause (FOR var IN-WHOLE-VECTOR v)
  "All the elements of a vector (disregards fill-pointer)"
  (let ((vect (gensym))
        (index (gensym)))
    `(progn
       (with ,vect = ,v)
       (for ,index from 0 below (array-dimension ,vect 0))
       (for ,var = (aref ,vect ,index)))))
\end{program}
Note that we immediately put |v| in a variable, in case it is an
expression.  Again, this is just good Lisp macrology.  It also has a
subtle effect on the semantics of the driver: |v| is evaluated only
once, at the beginning of the loop, so changes to |v| in the loop have
no effect on the driver.  Similarly, the bounds for numerical iteration
e.g. the above |array-dimension| are also evaluated once only.  This is how
all of \iter's drivers work.


There is an important point concerning the |progn| in this code.  We
need the |progn|, of course, because we are returning several forms,
one of which is a driver.  But \iter\ drivers must occur at top-level.
Is this code in error?  No, because ~top-level~ is defined in \iter\
to include forms inside a |progn|.  This is just the definition of
top-level that Common Lisp uses, and for the same reason: to allow
macros to return multiple forms at top-level.

While our |for\dots in-whole-vector| clause will work, it is not
ideal.  In particular, it does not support generating.  Do do so, we
need to use |for\dots next| or |for\dots do-next|.  The job is
simplified by the |defmacro-driver| macro.

\begin{clauses}

\defmacro{defmacro-driver}{arglist |\&body| body}
Defines a driver clause in
both the |for| and |generate| forms, and provides a parameter
|generate| which ~body~ can examine to determine how it was invoked.
~arglist~ is as in |defmacro-clause|, and should begin with the symbol
|for|. 

\end{clauses}

With |defmacro-driver|, our driver looks like this:
\begin{program}
(defmacro-driver (FOR var IN-WHOLE-VECTOR v)
  "All the elements of a vector (disregards fill-pointer)"
   (let ((vect (gensym))
         (end (gensym))
         (index (gensym))
         (kwd (if generate 'generate 'for)))
     `(progn
        (with ,vect = ,v)
        (with ,end = (array-dimension ,vect 0))
        (with ,index = -1)
        (,kwd ,var next (progn (incf ,index)
                               (if (>= ,index ,end) (terminate))
                               (aref ,vect ,index))))))
\end{program}


We are still missing one thing: the |\&sequence| keywords.  
We can get them easily enough, by writing 
\begin{program}
(defmacro-driver (FOR var IN-WHOLE-VECTOR v \&sequence) 
  ...)
\end{program}
We can now refer to parameters |from|, |to|, |by|, etc. which contain
either the values for the corresponding keyword, or \nil\ if the
keyword was not supplied.  Implementing the right code for these
keywords is cumbersome but not difficult; it is left as an exercise.
But before you begin, see |defclause-sequence| below for an easier way.
       
\subsection{Extensibility Aids}
\label{aids}

This section documents assorted features that may be of use in
extending \iter.

\begin{clauses}

\defunexpvar{*result-var*}
Holds the variable that is used to return a value as a result of the
\iter\ form.  You may examine this and use it in a |with| clause, but
you should not change it.

\defmacro{defsynonym}{syn word}
 Makes ~syn~ a synonym for the existing \iter\ keyword ~word.~  Only
the first word in each clause can have synonyms. 

\defmacroxx{defclause-sequence}{element-name index-name}{|\&key|
access-fn size-fn sequence-type}{element-type
element-doc-string index-doc-string}  
Provides
a simple way to define sequence clauses.  Generates two
clauses, one for iterating over the sequence's elements, the other
for iterating over its indices.  The first symbol of both
clauses will have print-name |for|.
~element-name~  and ~index-name~ should be symbols.
~element-name~ is the second keyword of the element iterator (typically of
the form 
|in-~sequence-type~|), and ~index-name~ is the second keyword
of the index-iterator (typically of the form
|index-of-~sequence-type~|).  Either name may be 
\nil, in which case the corresponding clause is not defined.  If both
symbols are supplied, they should be in the same package.  The |for|
that begins the clauses will be in this package.

\cpar ~access-fn~ is the function to be used to
access elements of the sequence in the element iterator.  The function
should take two 
arguments, a sequence and an index, and return the appropriate element.
~size-fn~ should denote a function of one argument, a sequence, that
returns its size.  Both ~access-fn~ and ~size-fn~ are required for the
element iterator, but only ~size-fn~ is needed for the index iterator.

\cpar The ~sequence-type~ and ~element-type~ keywords are used to
suggest types for the variables 
used to hold the sequence and the
sequence elements, respectively.  The usual rules about \iter's
treatment of variable type declarations apply (see section \ref{types}).

\cpar ~element-doc-string~ and ~index-doc-string~ are
the documentation strings, for use with |display-iterate-clauses|. 

\cpar The generated element-iterator performs destructuring on the
element variable.

\cpar As an example, the above |for\dots in-whole-vector| example
could have been written:
\begin{program}
(defclause-sequence IN-WHOLE-VECTOR INDEX-OF-WHOLE-VECTOR
  :access-fn 'aref
  :size-fn \#'(lambda (v) (array-dimension v 0))
  :sequence-type 'vector
  :element-type t
  :element-doc-string 
     "Elements of a vector, disregarding fill-pointer"
  :index-doc-string 
     "Indices of vector, disregarding fill-pointer")
\end{program}

\end{clauses}

\subsection{Subtleties}

There are some subtleties to be aware of when writing \iter\ clauses.
First, the code returned by your macros may be |nconc|'ed into a list,
so you should always returned freshly consed lists, rather than
constants.  Second, \iter\ matches clauses by using |eq| on the first
symbol and |string=| on the subsequent ones, so the package of the
first symbol of a clause is relevant.  All of the clauses in this manual
have their first word in the \iter\ package.  
You can use the package system in the usual way to shadow
\iter\ clauses without replacing them.

%%% say more here, about the badness that only the first word of a
%%% clause is packagey.

\section{Non-portable Extensions to Iterate (Contribs)}
\label{contribs}

Currently, there is only one non-portable extension to iterate in the
distribution: iterate-pg. If you have made an extension that depends
on non-portable features, feel free to send them to |asf@boinkor.net|
for inclusion in the iterate distribution.

\subsection{An SQL query driver for iterate}

The pg package by Eric Marsden (see |http://cliki.net/pg|) provides an
interface to the PostgreSQL database. Using the \iterpg\ extension, it
is possible to handle the results of SQL queries using \iter.

This usage example should give you an idea of how to use it:

\begin{program}
(pg:with-pg-connection (c "somedb" "someuser")
  (iter (for (impl version date) in-relation "select * from version"
                                 on-connection *dbconn*)
        (collect version)))
\end{program}

To use the extension via |ASDF|, simply make your system depend on the
|iterate-pg| system instead of the |iterate| system. To load it
manually, use:

\begin{program}
  (asdf:oos 'asdf:load-op :iterate-pg)
\end{program}

%%%%%%%%%%%%%%%%%%%%%%%%%%%%%%%%%%%%%%%%%%%%%%%%%%%%%%%%%%%%%%%%
\section{Obtaining {\tt Iterate}}

\begin{sloppypar}
The information in this chapter is obsolete but included for
completeness's sake; Currently, the most up-to-date information on
\iter\ can be found at |http://boinkor.net/iterate.html|.
\end{sloppypar}

\begin{sloppypar}
\iter\ currently runs on Lisp Machines, and on
HP's, Sun3's and Sparcstations under Lucid.  
\iter\ source and binaries are available at the MIT AI Lab in the
subdirectories of |/src/local/lisplib/|.  The source file,
|iterate.lisp|, is also available for anonymous FTP in the directory
|/com/fpt/pub/| on the machine |TRIX.AI.MIT.EDU| (Internet number
128.52.32.6).  If you are unable to obtain |iterate| in one of these
ways, send mail to |jba@ai.mit.edu| and I will send you the source
file. 
\end{sloppypar}

\begin{sloppypar}
\iter\ resides in the |iterate| package (nickname |iter|).  Just say
\linebreak |(use-package :iterate)| to make all the necessary symbols
available. 
If a symbol is not exported, it appears in this manual with an
``|iterate::|'' prefix.
\end{sloppypar}

Send bug reports to |bug-iterate@ai.mit.edu|.  The |info-iterate|
mailing list will have notices of changes and problems; to have
yourself added, send mail to |info-iterate-request@ai.mit.edu|.


\medskip

\begin{flushleft}
 \bf Acknowledgements
\end{flushleft}
\smallskip

Richard Waters provided invaluable criticism which spurred me to improve
\iter\ greatly.  As early users, David Clemens, Oren Etzioni and Jeff
Siskind helped ferret out many bugs.

%\begin{theindex}
% The index files must be generated with the genindex program
% or from makeindex iterate-manual.idx which creates iterate-manual.ind.

%\baselineskip\normalbaselineskip
%\advance\baselineskip by -2pt

% Fix ``.'' spacing in lists
\documentstyle[12pt,aimemo]{article}

\setlength{\textwidth}{6.0in}
\setlength{\oddsidemargin}{0.25in}
\setlength{\evensidemargin}{0.25in}
\setlength{\topmargin}{0pt}
\setlength{\headsep}{24pt}
\setlength{\textheight}{8.5in}

\let\origpar=\par

% somehow, LaTex fucks with things so that the plain Tex \obeyspaces
% doesn't work.  This is the same, except we use an mbox containing
% a space instead of just a space.

\newcommand{\spc}{\mbox{ }}
\newcommand{\tAb}{\mbox{        }}
\def\obspaces{\catcode`\ =\active\catcode`\^^I=\active}
{\obspaces \global\let =\spc\global\let^^I=\tAb}

\newenvironment{program}{\medskip\samepage\tt\obeylines\obspaces}{\medskip}

% use | for \tt
\catcode`\|=13
\newif\ifvbar
\def|{\ifvbar
         \endgroup\vbarfalse
      \else 
         \begingroup\tt\vbartrue 
      \fi}



\newcommand{\lisp}{\tt}
\newcommand{\iter}{|iterate|}
\newcommand{\iterpg}{|iterate-pg|}
\newcommand{\setf}{|setf|}
\newcommand{\nil}{|nil|}
\newcommand{\nonnil}{non-\nil}
\newcommand{\opt}{{\lisp \&optional}}
\newcommand{\yields}{$\Rightarrow$}



\newbox\Dots
\setbox\Dots=\hbox{\small ...}
\renewcommand{\dots}{\copy\Dots}

% Here's a way to define an environment which will treat all
% paragraphs inside it in the same manner.  It's, as near as I can
% figure, the way that Latex does things.  Using \everypar won't work;
% apparently Latex resets that a lot.  What you have to do is: at the
% top of the file do \let\origpar=\par.  In the
% begin part of the environment, do the parshape (or whatever), then say
% \def\par{{\origpar}}.  The extra braces are necessary.   That's all
% you have to do; at the end of the environment things will return to
% normal automatically.  Why does this all work?  I have no idea.
% Another important thing: there should be a blank line (i.e. a \par)
% between the end of the last paragraph and the \end{environment}.

\newlength{\clindent}
\setlength{\clindent}{0.5in}
\newlength{\clparindent}
\setlength{\clparindent}{\parindent}

\newenvironment{clauses}{\advance\linewidth -\parindent
                        \hangindent\clindent\relax
                         \hangafter=0
                          \parindent=0pt
                          \def\par{{\origpar}}}{}

\newcommand{\cpar}{\hspace \clparindent}

\newcommand{\startitem}{\bigskip\pagebreak[3]}
\newcommand{\finishitem}{\smallskip}

\newcommand{\clause}[1]{\startitem\Clause{#1}\finishitem}
\newcommand{\Clause}[1]{\hbox{#1}}
\newcommand{\clindex}[1]{\index{{\lisp #1}}}
\newcommand{\clindexx}[2]{\index{{\lisp #1\dots #2}}}
\newcommand{\clindexxx}[3]{\index{{\lisp #1\dots #2\dots #3}}}
\newcommand{\clausex}[2]{\startitem\Clausex{#1}{#2}\finishitem}
\newcommand{\Clausex}[2]{\Clause{|#1| #2}\clindex{#1}}

\newcommand{\defvar}[1]{\startitem\deff{#1}{}{Variable}\finishitem}
\newcommand{\defunexpvar}[1]{\startitem\deffnoindex{iterate::#1}{}{Variable}
                                        \lispindex{#1}\finishitem}
\newcommand{\defun}[2]{\startitem\deff{#1}{#2}{Function}\finishitem}
\newcommand{\defmacro}[2]{\startitem\deff{#1}{#2}{Macro}\finishitem}
\newcommand{\defunx}[3]{\startitem\deffx{#1}{#2}{#3}{Function}\finishitem}
\newcommand{\defmacrox}[3]{\startitem\deffx{#1}{#2}{#3}{Macro}\finishitem}
\newcommand{\defunxx}[4]{\startitem\deffxx{#1}{#2}{#3}{#4}{Function}\finishitem}
\newcommand{\defmacroxx}[4]{\startitem\deffxx{#1}{#2}{#3}{#4}{Macro}\finishitem}

\newcommand{\deff}[3]{\deffnoindex{#1}{#2}{#3}\lispindex{#1}}
\newcommand{\deffnoindex}[3]{\hbox to \hsize{{\tt #1} {\it #2}\hfill [{\it #3}]}}
\newcommand{\deffx}[4]{\deff{#1}{#2}{#4}\moreargs{#1}{#3}}
\newcommand{\deffxx}[5]{\deffx{#1}{#2}{#3}{#5}\moreargs{#1}{#4}}

\newcommand{\moreargs}[2]{\hbox to \hsize{\phantom{\lisp #1} {\it #2}\hfill}}

\newcommand{\lispindex}[1]{\index{{\lisp #1}}}

 \newenvironment{note}[1]{\pagebreak[2]\bigskip
                          \hrule\smallskip\small
                         {\setlength{\parindent}{0pt}
                           \par{\bf #1:}}}{\smallskip\hrule\bigskip}



\makeindex

%%%%%%%%%%%%%%%%%%%%%%%%%%%%%%%%%%%%%%%%%%%%%%%%%%%%%%%%%%%%%%%%

%\thispagestyle{empty}

%\vspace*{2in}

%\title{The |Iterate| Manual}
%\author{Jonathan Amsterdam}

%\maketitle

%\thispagestyle{empty}

%\begin{center}
%\Large *** DRAFT ***
%\end{center}
%This manual is in beta-test. It will become an AI Memo in a couple of
%months.  I would appreciate any comments or
%corrections.  You can reach me at email address |jba@ai.ai.mit.edu|.

\begin{document}

\def\writtenby{Jonathan Amsterdam}
\def\contractno{This report describes research done at the Artificial
Intelligence Laboratory of the Massachusetts Institute of Technology.
Support for the laboratory's artificial intelligence research is
provided in part by the Advanced Research Projects Agency of the
Department of Defense under Office
of Naval Research contract N00014-85-K-0124.}

\AIMEMO{1236}{\Large The Iterate Manual \\[1in]}
{This is the manual for version 1.4 of \iter, a powerful iteration
macro for Common  
Lisp. \iter\ is similar to {\tt loop} but provides numerous additional
features, is well integrated with Lisp, and is extensible. 
\vfill }


% use ~ for \em
\catcode`\~=13
\newif\ifem
\def~{\ifem
         \/\endgroup\emfalse
      \else 
         \begingroup\em\emtrue 
      \fi}



%\pagebreak
%\setcounter{page}{1}

%\thispagestyle{empty}


\tableofcontents

\pagebreak
\pagestyle{headings}

\section{Introduction}

\begin{sloppypar}
This manual describes
\iter, a powerful iteration facility for Common Lisp.
\iter\ provides abstractions for many common iteration
patterns and allows for the definition of additional patterns.  \iter\ is
a macro that expands into ordinary Lisp at
compile-time, so it is more efficient than higher-order
functions like |map| and |reduce.|  
While it is similar to |loop|, \iter\ offers a more Lisp-like syntax
and enhanced extensibility.
(For a more complete
comparison of \iter\ with other iteration constructs, see MIT AI Lab Working
Paper No. 324, ~Don't Loop, Iterate.~)
\end{sloppypar}

An \iter\ form consists of the symbol |iter|\footnote{You can also use
|iterate|, but |iter| is preferred because it avoids potential conflicts with
possible future additions to Common Lisp, and because it saves
horizontal space when writing code.}
followed by one or more
forms, some of which may be \iter\ 
~clauses.~  Here is a simple example of \iter\ which collects the
numbers from 1 to 10 into a list, and returns the list.  The return
value is shown following the arrow.
\begin{program}
(iter (for i from 1 to 10)
      (collect i))          \yields (1 2 3 4 5 6 7 8 9 10)
\end{program}
This form contains two clauses: a |for| clause that steps the
variable |i| over the integers from 1 to 10, and a |collect|
clause that accumulates its argument into a list.  With a few
exceptions, all \iter\ clauses have the same format: alternating
symbols (called ~keywords~) and expressions (called
~arguments~).  The syntax and terminology are those of Common
Lisp's keyword lambda lists.  One difference is that \iter's keywords
do not have to begin with a colon---though they may, except for the
first symbol of a clause.  So you can also write  |(for i :from
1 :to 10)| if you prefer.

Any Lisp form can appear in the body of an \iter, where it will have
its usual meaning.  \iter\ walks the entire body, expanding macros,
and recognizing clauses at any level.  This example collects all the
odd numbers in a list:
\begin{program}
(iter (for el in list)
      (if (and (numberp el) (oddp el))
          (collect el)))
\end{program}

There are clauses for iterating over numbers, lists, arrays and other
objects, and for
collecting, summing, counting, maximizing and other useful operations.
\iter\ also supports the creation of new variable bindings, stepping
over multiple sequences at once, destructuring, and compiler
declarations of variable types.  The following example illustrates
some of these features: 
\begin{program}
(iter (for (key . item) in alist)
         (for i from 0)
         (declare (fixnum i))
         (collect (cons i key)))
\end{program}

This loop takes the keys of an alist and returns a new alist
associating the keys with their positions in the original list. The
compiler declaration for |i| will appear in the generated code
in the appropriate place.

%%%%%%%%%%%%%%%%%%%%%%%%%%%%%%%%%%%%%%%%%%%%%%%%%%%%%%%%%%%%%%%%
\section{Clauses}

Most of \iter's clauses will be familiar to |loop| programmers.  
(|loop| is an iteration macro that has been incorporated into Common
Lisp.  Seen Guy Steele's {\em Common Lisp, 2nd Edition}.)
In
nearly all cases they behave the same as their |loop| counterparts, so
a |loop| user can switch to \iter\ with little pain (and much gain).

All clauses with the standard keyword-argument syntax consist of two
parts: a ~required~ part, 
containing keywords that must be present and in the right order; and
an ~optional~ part, containing keywords that may be omitted and,
if present, may occur in any order.  In the descriptions below, the
parts are separated by the Lisp lambda-list keyword \opt.


\subsection{Drivers}

An iteration-driving clause
conceptually causes the iteration to go forward.  Driver clauses in
\iter\ allow iteration over numbers, lists, vectors, hashtables, packages,
files and streams.  Iteration-driving clauses must
appear at the top level of an \iter\ form; they cannot be nested
inside another clause.  The driver variable is updated at the point
where the driver clause occurs.  Before the clause is executed for the
first time, the value of the variable is undefined.

%Also, regardless of where the driver clause
%appears in the body, the driver variable is stepped at the top of the
%loop; hence it is stylistically preferable, though
%not required, to place driver clauses at the beginning of the \iter.

Multiple drivers may appear in a single \iter\ form, in which case all
of the driver variables are updated each time through the loop, in the
order in which the clauses appear.  The first driver to terminate will
terminate the entire loop.

In all cases, the value of the driver variable on exit from the loop,
including within the epilogue code (see the |finally| clause), is
undefined.

All the parameters of a driver clause are evaluated once, before the
loop begins.  Hence it is not possible to change the bounds or other
properties of an iteration by side-effect from within the loop.

With one exception, driver clauses begin with the word |for| (or
the synonym |as|) and mention an iteration variable, which is
given a binding within the \iter\ form.
The exception is |repeat|, which just executes a loop a
specified number of times:

\begin{clauses}

\clausex{repeat}{~n~}
Repeats the loop ~n~ times.  For example:
\begin{program}
(iter (repeat 100)
      (print "I will not talk in class."))
\end{program}
\cpar If $n \leq 0$, the
loop will never be executed.  If ~n~ is not an integer, the
actual number of executions will be $\lceil n \rceil$.

\end{clauses}

\pagebreak[3]

\subsubsection{Numerical Iteration}

\begin{clauses}
\clausex{for}{~var~ |\&sequence|} 
The general form for iterating over a sequence of numbers requires a
variable and, optionally, one or more keywords that provide the bounds
and step size of the iteration. The |\&sequence|
lambda-list keyword is a shorthand for these sequence keywords.  They are:
|from|, |upfrom|, |downfrom|, |to|, |downto|, |above|, |below| and |by|.
|from| provides the starting value for ~var~
and defaults to zero. |to| provides a final value and implies that the
successive values of ~var~ will be increasing; |downto| implies that
they will be decreasing.  The loop terminates when ~var~ passes
the final value (i.e. becomes smaller or larger than it, depending on
the direction of iteration); in other  words, the loop body
will never be 
executed for values of ~var~ past the final value.
|below| and |above| are similar to |to| and
|downto|, except that the loop terminates when ~var~ equals or passes
the final value.

\cpar If no final value is specified, the variable will be stepped
forever.  Using |from| or |upfrom| will result in increasing
values, while |downfrom| will give decreasing values.

\cpar On each iteration,
~var~ is incremented or decremented by the value of the sequence
keyword |by|, which defaults to 1.  It should always be a positive
number, even for downward iterations.

\cpar In the following examples, the sequence of numbers generated is shown
next to the clause.

\begin{program}
(for i upfrom 0) \yields\ 0 1 2 \ldots
(for i from 5)   \yields\ 5 6 7 \ldots    ; either from or upfrom is okay
(for i downfrom 0) \yields\ 0 -1 -2 \ldots
(for i from 1 to 3) \yields\ 1 2 3
(for i from 1 below 3) \yields\ 1 2
(for i from 1 to 3 by 2) \yields\ 1 3
(for i from 1 below 3 by 2) \yields\ 1
(for i from 5 downto 3) \yields\ 5 4 3
\end{program}

\end{clauses}

\subsubsection{Sequence Iteration}

There are a number of clauses for iterating over sequences.  In all of
them, the argument following |for| may be a list instead of a
symbol, in which case destructuring is performed.  See section
\ref{destructuring}. 

\begin{clauses}

\clause{|for| ~var~ |in| ~list~ \opt\ |by| ~step-function~}
\clindexx{for}{in}
~var~ is set to successive elements of list.  
~step-function,~ which 
defaults to |cdr|, is used to obtain the next sublist. 

\clause{|for| ~var~ |on| ~list~ \opt\ |by| ~step-function~}
\clindexx{for}{on}
~var~ is set to successive sublists of list. 
~step-function~ (default |cdr|) is used as in |for\dots in|.

\end{clauses}

\medskip

These two clauses use |atom| to test for the end
of a list.  Hence, given a list whose final |cdr| is not \nil,
they will silently ignore the last |cdr|.  Other choices are
|endp|,  which would signal an error, and |null|, which
would probably result in an error somewhere else.  If you wish to use
an end-test other than |atom|, set the variable
|iterate::*list-end-test*|\lispindex{*list-end-test*} to the name of
the desired function. 

\begin{clauses}
\clause{|for| ~var~ |in-vector| ~vector~ |\&sequence|}
\clindexx{for}{in-vector}
~var~ takes on successive elements from ~vector.~   The vector's
fill-pointer is observed.  Here and in subsequent clauses, the |\&sequence|
keywords include |with-index|, which takes a symbol as argument
and uses it for the index variable instead of an internally generated
symbol.  The other |\&sequence| keywords behave as in numerical
iteration, except that the default iteration bounds are the bounds of
the vector.  E.g. in |(for i in-vector v downto 3)|,
|i| will start off being bound to the last element in |v|,
and will  
be set to preceding elements down to and including the element with
index 3.

\clause{|for| ~var~ |in-sequence| ~seq~ |\&sequence|}
\clindexx{for}{in-sequence}
This uses Common Lisp's generalized sequence functions, |elt|
and |length|, to obtain elements and determine the length of
~seq.~   Hence it will work for any sequence, including lists, and will
observe the fill-pointers of vectors.

\clause{|for| ~var~ |in-string| ~string~ |\&sequence|}
\clindexx{for}{in-string}
~var~ is set to successive characters of ~string.~

\startitem
\Clause{|for| ~var~ |index-of-vector| ~vector~ |\&sequence|}
\clindexx{for}{index-of-vector}
\Clause{|for| ~var~ |index-of-sequence| ~sequence~ |\&sequence|}
\clindexx{for}{index-of-sequence}
\Clause{|for| ~var~ |index-of-string| ~string~ |\&sequence|}
\clindexx{for}{index-of-string}
\finishitem
~var~ is set to successive indices of the sequence.
These clauses avoid the overhead of accessing the sequence elements
for those applications where they do not need to be examined, or are
examined rarely.  They admit all the optional keywords of the other
sequence drivers except the (redundant) |with-index| keyword.

\clause{|for| ~(key value)~ |in-hashtable| ~table~}
\clindexx{for}{in-hashtable}

~key~ and ~value,~ which must appear as shown in a list
and may be destructuring templates, are set to the keys and values of
~table~.  If ~key~ is |nil|, then the hashtable's keys will be ignored;
similarly for ~value~.
The order in which 
elements of ~table~ will be retrieved is unpredictable.

\clause{|for| ~var~ |in-package| ~package~ \opt\ |external-only| ~ext~}
\clindexx{for}{in-package}
Iterates over all the symbols in ~package,~ or over only the
external symbols if ~ext~ is specified and non-|nil|.  ~ext~ is not
evaluated.  The same symbol may appear more than once.

\clause{|for| ~var~ |in-packages| ~packages~ \opt\ |having-access| ~symbol-types~}
\clindexx{for}{in-packages}
Iterates over all the symbols from the list of packages denoted by the
descriptor ~packages~ and having accessibility (or visibility) given by
~symbol-types~.  This defaults to the list |(:external :internal :inherited)|
and is not evaluated.  ~var~ must be a list of up to three variables: in each
iteration, these will be set to a symbol, its access-type and package (as per
|with-package-iterator| in ANSI-CL).  The same symbol may appear more than
once.

\end{clauses}

\begin{clauses}

\clause{|for| ~var~ |in-file| ~name~ \opt\ |using| ~reader~}
\clindexx{for}{in-file}
Opens the file ~name~ (which may be a string or pathname) for input,
and iterates 
over its contents. ~reader~ defaults to |read|, so by default ~var~
will be bound to the successive forms in the file.  The \iter\ body is
wrapped in an unwind-protect to ensure that the file is closed no
matter how the \iter\ is exited.

\clause{|for| ~var~ |in-stream| ~stream~ \opt\ |using| ~reader~}
\clindexx{for}{in-stream}
Like |for\dots in-file|, except that ~stream~ should be an existing
stream object that supports input operations.

\end{clauses}

\subsubsection{Generalized Drivers}

These are primarily useful for writing drivers that can also be used
as generators (see section \ref{generators}, below).
 
\begin{clauses}

\clause{|for| ~var~ |next| ~expr~}
\clindexx{for}{next}

~var~ is set to ~expr~ each time through the loop.  Destructuring is
performed.  When the clause is used as a generator, ~expr~ is the code
that is executed when |(next ~var~)| is 
encountered (see section \ref{generators}, below).
~expr~ should compute the first value for ~var~, as well
as all subsequent values, and is responsible for terminating the loop.
For compatibility with future versions of \iter, this termination
should be done with |terminate|\clindex{terminate}, which can be
considered a synonym for |finish| (see section \ref{control-flow}).

\cpar As an example, the following clauses are equivalent to |(for
i from 1 to 10)|:
\begin{program}
(initially (setq i 0))
(for i next (if (> i 10) (terminate) (incf i)))
\end{program}

\clause{|for| ~var~ |do-next| ~form~}
\clindexx{for}{do-next}
~form~ is evaluated each time through the loop.  Its value is ~not~
set to ~var~; that is ~form's~ job. ~var~ is only present so that
\iter\ knows it is a driver variable.  \linebreak |(for ~var~ next
~expr~)| is equivalent to |(for ~var~ do-next (dsetq ~var~ ~expr~))|.
(See section \ref{destructuring} for an explanation of |dsetq|.) 

\end{clauses}

\subsubsection{Generators}
\label{generators}

In all of the above clauses, the driver variable is updated on each
iteration.
Sometimes it is desirable to have greater control over updating.  
For instance, consider the problem of associating numbers, in
increasing order and with no gaps, with the
\nonnil\ elements of a list.  One obvious first pass at writing this is:

\begin{program}
(iter (for el in list)
      (for i upfrom 1)
      (if el (collect (cons el i))))
\end{program}
But on the list |(a b nil c)| this produces |((a . 1) (b . 2) (c .
4))| instead of the desired |((a . 1) (b . 2) (c . 3))|.  The problem
is that |i| is incremented each time through the loop, even when |el|
is |nil|.  

The problem could be solved elegantly if we could step |i| only when
we wished to.  This
can be accomplished for any \iter\ driver by writing |generate| (or
its synonym |generating|)
instead of |for|.  Doing so produces a ~generator~---a driver whose
values are yielded explicitly.  To obtain the next value of a
generator variable ~v~, write \linebreak |(next ~v~)|.  The value of a |next|
form is the next value of ~v~, as determined by its associated driver
clause.  |next| also has the side-effect of updating ~v~ to that
value.  If there is no next value, |next| will terminate the loop,
just as with a normal driver.


Using generators, we can now write our example like this:
\begin{program}
(iter (for el in list)
      (generate i upfrom 1)
      (if el (collect (cons el (next i)))))
\end{program}
Now |i| is updated only when |(next i)| is executed, and this occurs
only when |el| is \nonnil.

To better understand the relationship between ordinary drivers and
generators, observe that we can rewrite an ordinary driver using its
generator form immediately followed by |next|, as this example shows:
\begin{program}
(iter (generating i from 1 to 10)
      (next i)
      ...)
\end{program}
Provided that the loop body contains no |(next i)| forms, this will
behave just as if we had written |(for i from 1 to 10)|.

We can still refer to a driver variable ~v~ without using |next|; in
this case, its value is that given to it by the last evaluation of
|(next ~v~)|.  Before |(next ~v~)| has been called the first time, the
value of ~v~ is undefined.  

This semantics is more flexible than
one in which ~v~ begins the loop bound to its first value and calls
of |next| supply subsequent values, because it means the loop will not
terminate too soon if the generator's sequence is empty.  For
instance, consider the following code, which tags \nonnil\ elements of
a list using a list of tags, and also counts the null elements.
(We assume there are at least as many tags as \nonnil\ elements.)
\begin{program}
(let* ((counter 0)
       (tagged-list (iter (for el in list)
                          (generating tag in tag-list) 
                          (if (null el)
                              (incf counter)
                              (collect (cons el (next tag)))))))
  ...)
\end{program}
It may be that there are just as many tags as non-null elements of
|list|.  If all the elements of |list| are null, we still want the
counting to proceed, even though |tag-list| is \nil.  If |tag| had to be
assigned its first value before the loop begins, we would have had to
terminate the loop before the first iteration, since when |tag-list|
is \nil, |tag| 
has no first value.  With the existing semantics, however, |(next
tag)| will never execute, so the iteration will cover all the elements of
|list|. 
        
When the ``variable'' of a driver clause is actually a destructuring
template containing several variables, all the variables are eligible
for use with |next|.  As before, |(next ~v~)| evaluates to ~v's~ next
value; but the effect is to update all of the template's variables.
For instance, the following code will return the list |(a 2 c)|.
\begin{program}
(iter (generating (key . item) in '((a . 1) (b . 2) (c . 3)))
      (collect (next key))
      (collect (next item)))
\end{program}

Only driver clauses with variables can be made into generators.  This
includes all clauses mentioned so far except for |repeat|.  It does
~not~ include |for\dots previous|, |for\dots =|,
|for\dots initially\dots then| or |for\dots first\dots then| (see
below). 

\subsubsection{Previous Values of Driver Variables}

Often one would like to access the value of a variable on a previous
iteration.  \iter\ provides a special clause for accomplishing this.

\begin{clauses}

\clause{|for| ~pvar~ |previous| ~var~ \opt\ |initially| ~init~
|back| ~n~}
\clindexx{for}{previous}
Sets ~pvar~ to the previous value of ~var,~ which should be a driver
variable, a variable from another |for\dots previous| clause, or a
variable established by a |for\dots =|, 
|for\dots initially\dots then| or |for\dots first\dots then| clause
(see section \ref{setting}).
Initially, ~pvar~ is given the value ~init~ (which defaults to \nil).  
The ~init~ expression will be moved outside the loop body, so it
should not depend on anything computed within the loop.
~pvar~ retains the value of ~init~ until ~var~ is set to its second
value, at which point ~pvar~ is set to ~var's~ first value; and so on.  

\cpar The
argument ~n~ to |back| must be a constant, positive integer, and
defaults to 1.  It determines how many iterations back ~pvar~ should
track ~var~.  For example, when ~n~ is 2, then ~pvar~ will be assigned
~var's~ first value when ~var~ is set to its third value.

\cpar A |for\dots previous| clause may occur after or before its
associated driver clause. |for\dots previous| works with generators as
well as ordinary drivers.

\pagebreak[3]

\cpar Example:
\begin{program}
(iter (for el in '(1 2 3 4))
      (for p-el previous el)
      (for pp-el previous p-el initially 0)
      (collect pp-el))
\end{program}
This evaluates to |(0 0 1 2)|.  It could have been written more
economically as 
\begin{program}
(iter (for el in '(1 2 3 4))
      (for pp-el previous el back 2 initially 0)
      (collect pp-el))
\end{program}


\end{clauses}

\subsection{Variable Binding and Setting}
\label{setting}

Several clauses exist for establishing new variable bindings or for
setting variables in the loop.  They all support destructuring.

\begin{clauses}

\clausex{with}{~var~ \opt\ |=| ~value~}
Causes ~var~ to be bound to value before the loop body is entered.
If ~value~ is not supplied, ~var~ assumes a default
binding, which will be 
\nil\ in the absence of declarations.  Also, if ~value~ is not
supplied, no destructuring is performed; instead, ~var~ may be a list of
symbols, all of which are given default 
bindings.  If ~value~ is supplied, ~var~ is bound to it, with
destructuring.  

\cpar Because |with|
creates bindings whose scope includes the entire \iter\ form, it is
good style to put all |with| clauses at the beginning.

\cpar Successive occurrences of |with| result in sequential
bindings (as with 
|let*|).  There is no way to obtain parallel bindings; see
section \ref{bindings} for a rationale.  


\clause{|for| ~var~ |=| ~expr~}
\clindexx{for}{=}
On each iteration, ~expr~ is evaluated and ~var~ is set
to its value. 

\cpar This clause may appear to do the same thing as |for\dots next|.
In fact, they are quite different.  |for\dots =| provides only three
services: it sets up a binding for ~var~, sets it to ~expr~ on each
iteration, and makes it possible to use |for\dots previous| with
~var~.  |for\dots next| provides these services in addition to the
ability to turn the driver into a generator.
%Also, the code which sets ~var~ appears in the loop body in the same
%place as the |for\dots =| clause; the code for |for\dots next| appears
%at the top of the loop, as with other drivers (except when being used
%as a generator).

\clause{|for| ~var~ |initially| ~init-expr~ |then| ~then-expr~}
\clindexxx{for}{initially}{then}
Before the loop begins, ~var~ is set to ~init-expr;~ on all
iterations after the first it is set to ~then-expr.~  This clause
must occur at top-level.  ~init-expr~ will be moved outside the loop
body and ~then-expr~ will be moved to the end of the loop body, so
they are subject to code motion problems (see section
\ref{code-movement}). 

\cpar This clause may appear to be similar to |for\dots next|, but in
fact they differ significantly.  |for\dots initially\dots then| is
typically used to give ~var~ its first value before the loop begins,
and subsequent values on following iterations.  This is incompatible
with generators, whose first value and subsequent values must all be
computed by |(next ~var~)|.  Also, the update of ~var~ in 
|for\dots initially\dots then| does not occur at the location of the
clause. 
Use |for\dots initially\dots then| for
one-shot computations where its idiom is more convenient, but use
|for\dots next| for extending \iter\ with new drivers (see section
\ref{extend}).

\clause{|for| ~var~ |first| ~first-expr~ |then| ~then-expr~}
\clindexxx{for}{first}{then}
The first time through the loop, ~var~ is set to ~first-expr;~ on
subsequent iterations, it is set to ~then-expr.~  This differs from
|for\dots initially| in that ~var~ is set to ~first-expr~
inside the loop body, 
so ~first-expr~ may depend on the results of other clauses.  For
instance, 
\begin{program}
(iter (for num in list)
      (for i first num then (1+ i))
      ...)
\end{program}
will set |i| to the first element of |list| on the first
iteration, whereas 
\begin{program}
(iter (for num in list)
      (for i initially num then (1+ i))
      ...)
\end{program}
is probably erroneous; |i| will be bound to |num|'s
default binding (usually \nil) for the first iteration.

\end{clauses}

\begin{note}{Compatibility note}
|loop|'s |for\dots =| works like \iter's, but |loop| used the syntax
|for\dots =\dots then| to mean |for\dots initially\dots then|.  It was
felt that these two operations were sufficiently different to warrant
different keywords.

Also, the |for| in the above three clauses is misleading,
since none is true driver (e.g. none has a corresponding |generate|
form).  |setting| would have been a better choice, but |for| was used
to retain some compatibility with |loop|.  
\end{note}


%%%%%%%%%%%%%%%%%%%%%%%%%%%%%%%%
\subsection{Gathering Clauses}
Many of \iter's clauses accumulate values into a variable, or set a
variable under certain conditions.  At the end of the loop, this
variable contains the desired result.  All these clauses have an
optional |into| keyword, whose argument should be a symbol.  If the
|into| keyword is not supplied, the accumulation variable will be
internally generated and its value will be returned at the end of the
loop; if a variable is specified, that variable is used for the
accumulation, and is not returned as a result---it is up to the user
to return it explicitly, in the loop's epilogue code (see |finally|).
It is safe to examine the accumulation variable during the loop, but
it should not be modified.

These clauses all begin with a verb.  When the verb does
not conflict with an existing Common Lisp function, then it may be
used in either its infinitival or present-participle form (e.g. |sum|,
|summing|).  However, when there is a conflict with Common Lisp, only
the present-participle form may be used (e.g. |unioning|).  This is to
prevent \iter\ clauses from clashing with Common Lisp
functions. 

%although these clauses are described as ``producing a value,'' it is a
%mistake to think of the lisp list representing the clause as a
%value-producing form in the usual way.  clauses may legally be written
%where a value is expected, e.g. |(setq x (sum i))|, but the
%lisp value of a clause in such a context is undefined.

\subsubsection{Reductions}

~Reduction~ is an extremely common iteration pattern in which
the results of successive applications of a
binary operation are accumulated.
For example, a loop that computes the sum of the
elements of a list is performing a reduction with the addition
operation.  This could be written in Common Lisp as  |(reduce \#'+
list)| or with \iter\ as
\begin{program}
(iter (for el in list)
      (sum el))
\end{program}

\begin{clauses}

\clausex{sum}{~expr~ \opt\ |into| ~var~}
Each time through the loop, ~expr~ is evaluated and added to a
variable, which is bound initially to zero.  If ~expr~ has a type,
it is ~not~ used as the type of the sum variable, which is always
|number|.  To get the result variable to be of a more specific
type, use an explicit variable, as in
\begin{program}
(iter (for el in number-list)
      (sum el into x)
      (declare (fixnum x))
      (finally (return x)))
\end{program}

\clausex{multiply}{~expr~ \opt\ |into| ~var~}
Like |sum|, but the initial value of the result variable is $1$,
and the variable is updated by multiplying ~expr~ into it.

\clausex{counting}{~expr~ \opt\ |into| ~var~}
~expr~ is evaluated on each iteration.  If it is \nonnil, the
accumulation variable, initially zero, is incremented.

\startitem
\Clausex{maximize}{~expr~ \opt\ |into| ~var~}
\Clausex{minimize}{~expr~ \opt\ |into| ~var~}
\finishitem
~expr~ is evaluated on each iteration and its extremum (maximum or
minimum) is stored in the accumulation variable.  If ~expr~ is never
evaluated, then the result is \nil\ (if the accumulation variable is
untyped) or $0$ (if it has a numeric type).

\clausex{reducing}{~expr~ |by| ~func~ \opt\ |initial-value| ~init-val~ |into| ~var~}
This is a general way to perform reductions. ~func~ should be a
function of two arguments, the first of which will be the value
computed so far and
the second of which will be the value of ~expr~.  It should return the
new value.
|reducing| is roughly equivalent to the
Common Lisp |(reduce ~func~ ~sequence~ :key ~expr-function~)|, where
~expr-function~ is used to derive values from the successive elements
of ~sequence~.

\cpar If the |reducing| clause is never executed, the result is
undefined.

\cpar It is not necessary to provide an
initial value, but better code can be generated if one is supplied.
Regardless of its location in the \iter\ body, ~init-val~ will be
evaluated before the loop is entered, so it should not depend on any
value computed inside the \iter\ form.  

%\cpar if a ~var~ is not specified, you can get \iter\ to declare the
%type of the internal variable by putting a |the| expression around
%~func.~  see section \ref{types}.

\end{clauses}

%\begin{note}{implementation note}
%in principle, |maximize| and |minimize| can be thought of
%as reductions where the initial value is the smallest (or largest)
%value that the accumulation variable can assume.  because lisp's
%bignums can represent arbitrary integers, these clauses cannot be
%implemented as reductions in general.  if, however, the type of 
%~expr~ or ~var~ can be determined to be a fixnum
%or a float, \iter\ will implement the clause as a true reduction,
%using one of the constants |most-negative-fixnum|, |%most-positive-fixnum|, 
%|most-negative-short-float|, etc. as appropriate.
%\end{note}

\subsubsection{Accumulations}

All the predefined accumulation clauses add values to a sequence.  If
the sequence is a list, they all 
have the property that the partial list is kept in the correct order
and available for inspection at any point in the loop.

\begin{clauses}

\clausex{collect}{~expr~ \opt\ |into| ~var~ |at| ~place~ |result-type| ~type~}
Produces a sequence of the values of ~expr~ on each iteration. ~place~
indicates where the next value of ~expr~ is added to the list and may
be one of the symbols |start|, |beginning| (a synonym for |start|) or
|end|.  The symbol may be quoted, but need not be.  The default is
|end|.  For example,
\begin{program}
(iter (for i from 1 to 5)
      (collect i))
\end{program}
produces |(1 2 3 4 5)|, whereas
\begin{program}
(iter (for i from 1 to 5)
      (collect i at beginning))
\end{program}
produces |(5 4 3 2 1)| (and is likely to be faster in most Common Lisp
implementations).

\cpar If ~type~ is provided, it should be a subtype of |sequence|.
The default is |list|.  Specifying a type other than |list| will
result in |collect| returning a sequence of that type.  ~However,~ the
type of the sequence being constructed when inside the loop body is
undefined when a non-|list| type is specified.  (As with ~place~,
quoting ~type~ is optional.)

\clausex{adjoining}{~expr~ \opt\ |into| ~var~ |test| ~test~ |at| ~place~ |result-type| ~type~}
Like |collect|, but only adds the value of ~expr~ if it is not
already present.  ~test,~ which defaults to |\#'eql|, is
the test to be used with |member|.

\startitem
\Clausex{appending}{~expr~ \opt\ |into| ~var~ |at| ~place~}
\Clausex{nconcing}{\ ~expr~ \opt\ |into| ~var~ |at| ~place~}
\Clausex{unioning}{\ ~expr~ \opt\ |into| ~var~ |test| ~test~ |at| ~place~}
\Clausex{nunioning}{~expr~ \opt\ |into| ~var~ |test| ~test~ |at| ~place~}
\finishitem
These are like |collect|, but behave like the Common Lisp functions
|append|, |nconc|, |union| or |nunion|.  
As in Common Lisp, they work only on lists.  Also as in Common Lisp,
|unioning| and |nunioning| assume that the value of ~expr~ contains no
duplicates. 

\clausex{accumulate}{~expr~ |by| ~func~ \opt\ |initial-value| ~init-val~ |into| ~var~}
This is a general-purpose accumulation clause. ~func~ should be a
function of two arguments, the value of ~expr~ and the value
accumulated so far in the iteration, and it should return the updated
value.  If no initial value is supplied, \nil\ is used.  

%\cpar If a ~var~ is not specified, you can get \iter\ to declare the
%type of the internal variable by putting a |the| expression around
%~func.~  see section \ref{types}.

\cpar The differences between |accumulate| and |reducing| are slight.
One difference is that the functions take their arguments in a
different order.  Another is that in the absence of ~init-val~,
|accumulate| will use \nil, whereas |reducing| will generate different
code that avoids any dependence on the initial value.
The reason for having both clauses is that one usually
thinks of reductions (like |sum|) and accumulations (like |collect|)
as different beasts.

\end{clauses}


\subsubsection{Finders}

A ~finder~ is a clause whose value is an expression that meets some
condition.

\begin{clauses}

\clause{|finding| ~expr~ |such-that| ~test~ \opt\ |into| ~var~ |on-failure| ~failure-value~}
\clindexx{finding}{such-that}
If ~test~ (which is an expression) ever evaluates to \nonnil, the loop
is terminated, the 
epilogue code is run and the value of ~expr~ is returned.  Otherwise,
\nil\ (or ~failure-value,~ if provided) is returned.  If ~var~ is
provided, it will have either the \nonnil\ value of ~expr~ or
~failure-value~ when the epilogue code is run.

\cpar As a special case, if the ~test~ expression is a sharp-quoted
function, then it is applied to ~expr~ instead of being simply
evaluated.  E.g. |(finding x such-that \#'evenp)| is equivalent to
|(finding x such-that (evenp x))|.  

%\cpar although ~test~ need have nothing to do with ~%expr~ as in
%|(finding j such-that (> i 3))|, it usually 
%will: |(finding (length el) such-that (oddp (length el)))|.  to
%avoid performing the |length| computation twice, you could write
%|(finding (length el) such-that \#'oddp)| or |(finding (length
%el) such-that 'oddp)|; for these cases, \iter\ generates code that
%executes ~expr~ only once.  the code for |\#'oddp|
%is slightly different from that for {\lisp 'oddp}; see the discussion
%under {\lisp for\dots in} and {\lisp for\dots on}.

\cpar |On-failure| is a misnomer. Because it is always evaluated, it behaves
more like the default third argument to the |gethash| function. As a result,
|on-failure (error "Not found")| makes no sense. Instead, the clauses |leave|
or |thereis| can be used in conjunction with |finally| as follows:
\begin{program}
(iter (for x in '(1 2 3))
      (if (evenp x) (leave x))
      (finally (error "not found")))
\end{program}

\cpar This clause may appear multiple times when all defaults are
identical. It can also be used together with either |always|/|never| or
|thereis| if their defaults match. More specifically, |on-failure nil| is
compatible with |thereis|, while |on-failure t| is compatible with |always|
and |never| clauses.
\begin{program}
(iter (for i in '(7 -4 2 -3))
      (if (plusp i)
	  (finding    i  such-that (evenp i))
          (finding (- i) such-that (oddp i))))
\end{program}

\startitem
\Clause{|finding| ~expr~ |maximizing| ~m-expr~ \opt\ |into| ~var~}
\clindexx{finding}{maximizing}
\Clause{|finding| ~expr~ |minimizing| ~m-expr~ \opt\ |into| ~var~}
\clindexx{finding}{minimizing}
\finishitem
Computes the extremum (maximum or minimum) value of ~m-expr~ over all
iterations, and returns the value of ~expr~ corresponding to the
extremum.  ~expr~ is evaluated inside the loop at the time the new
extremum is established.  If ~m-expr~ is never evaluated (due to, for
example, being embedded in a conditional clause), then the returned
value depends on the type, if any, of ~expr~ (or ~var,~ if
one is supplied).  If there is no type, the returned
value will be \nil; if the type is numeric, the returned value will be
zero.

\cpar For these two clauses, ~var~ may be a list of two
symbols; in that case, the first is used to record ~expr~ and
the second, ~m-expr.~  

\cpar As with |finding\dots such-that|, if ~m-expr~ is a sharp-quoted
function, then it is called on ~expr~ instead of being evaluated.

\end{clauses}

\subsubsection{Boolean Tests}

\begin{clauses}

\clausex{first-iteration-p}{}
Returns |t| in the first cycle of the loop, otherwise \nil.

\clausex{first-time-p}{}
Returns |t| the first time the expression is evaluated, and then \nil\ forever.
This clause comes handy when printing (optional) elements separated
by a comma:

\begin{program}
(iter (for el in '(nil 1 2 nil 3))
      (when el
        (unless (first-time-p)
          (princ ", "))
        (princ el)))
\end{program}
produces |"1, 2, 3"|.

\end{clauses}

\subsubsection{Aggregated Boolean Tests}

\begin{clauses}

\clausex{always}{~expr~}
If ~expr~ ever evaluates to
\nil, then \nil\ is immediately returned; the epilogue code is not
executed.  If ~expr~ never evaluates to \nil, the epilogue code
is executed and the last value of ~expr~ (or |t| if ~expr~ was never
evaluated) is returned (whereas |loop| would constantly return |t|).

% mention last evaluated clause when multiple always clauses?

\clausex{never}{~expr~}
Like |(always (not ~expr~))|, except it does not influence the last
value returned by a possible other |always| clause. That is,
\begin{program}
(iter (repeat 2)
      (always 2)
      (never nil)) \yields 2 ; not t
\end{program}

\clausex{thereis}{~expr~}
If ~expr~ is ever \nonnil,
its value is immediately returned without running epilogue code.
Otherwise, the epilogue code is performed and \nil\ is returned.

This clause cannot be used together with |always| or |never|, because their
defaults are opposed (similarly, |(loop always 3 thereis nil)| refuses to
compile in some implementations of |loop|).

\end{clauses}


\subsection{Control Flow}
\label{control-flow}
Several clauses can be used to alter the usual flow of control in a loop.

Note: the clauses of this and subsequent sections don't adhere to \iter's
usual syntax, but instead use standard Common Lisp syntax.  Hence the
format for describing syntax subsequently is
like the standard format used in the Common Lisp manual, not like the
descriptions of clauses above.

\begin{clauses}

\clausex{finish}{}
Stops the loop and runs the epilogue code.

%for example:
%
%\begin{program}
%(iter (with answer = nil)
%         (initially (make-a-mess))
%         (for i from 1 to 10)
%         (when (correct? i)
%           (setq answer i)
%           (finish))
%         (finally (cleanup)))
%\end{program}
%
%this code will execute |cleanup| whether or not the test |(correct?
%i)| ever succeeds. 
%the (more elegant) formulation,
%\begin{program}
%(iter (initially (make-a-mess))
%         (for i from 1 to 10)
%         (finding i such-that (correct? i))
%         (finally (cleanup)))
%\end{program}
%would not execute |cleanup| if |(correct? i)| succeeded; it
%would do an immediate return.

\clausex{leave}{\opt\ ~value~}
Immediately returns ~value~ (default \nil) from the current \iter\
form, skipping the epilogue code.  Equivalent to using |return-from|.

\clausex{next-iteration}{}
Skips the remainder of the loop body and begins the next iteration of
the loop.

\clausex{while}{~expr~}
If ~expr~ ever evaluates to \nil, the loop is terminated and the
epilogue code executed.  Equivalent to |(if (not ~expr~) (finish))|.

\clausex{until}{~expr~}
Equivalent to |(if ~expr~ (finish))|.

\clausex{if-first-time}{~then~ \opt\ ~else~}
If this clause is being executed for the first time in this invocation
of the \iter\ form, then the ~then~ code is evaluated; otherwise the
~else~ code is evaluated.

\cpar |(for ~var~ first ~expr1~ then ~expr2~)| is almost equivalent to
\begin{program}
(if-first-time (dsetq ~var~ ~expr1~)
               (dsetq ~var~ ~expr2~))
\end{program}
The only difference is that the |for| version makes ~var~ available
for use with |for\dots previous|.

\end{clauses}

\subsection{Code Placement}
When fine control is desired over where code appears in a loop
generated by \iter, the following special clauses may be useful.
They are all subject to code-motion problems (see section
\ref{code-movement}).  

\begin{clauses}

\clausex{initially}{|\&rest| ~forms~}
The lisp ~forms~ are placed in the prologue section of the loop, where
they are executed once, before the loop body is entered.

\clausex{after-each}{|\&rest| ~forms~}
The ~forms~ are placed at the end of the loop body, where they
are executed after each iteration.  Unlike the other clauses in this
section, ~forms~ may contain \iter\ clauses.

\clausex{else}{|\&rest| ~forms~}
The lisp ~forms~ are placed in the epilogue section of the loop, where they
are executed if this |else| clause is never met during execution of the
loop and the loop terminates normally.

\clausex{finally}{|\&rest| ~forms~}
The lisp ~forms~ are placed in the epilogue section of the loop, where
they are executed after the loop has terminated normally.

\clausex{finally-protected}{|\&rest| ~forms~}
The lisp ~forms~ are placed in the second form of an unwind-protect
outside the loop.  They are always executed after the loop has
terminated, regardless of how the termination occurred.

\end{clauses}


%%%%%%%%%%%%%%%%%%%%%%%%%%%%%%%%%%%%%%%%%%%%%%%%%%%%%%%%%%%%%%%%
\section{Other Features}

\subsection{Multiple Accumulations}
\label{multiple}

\begin{sloppypar}
It is permitted to have more than one clause accumulate into the same
variable, as in the following:
\begin{program}
(iter (for i from 1 to 10)
      (collect i into nums)
      (collect (sqrt i) into nums)
      (finally (return nums)))
\end{program}
Clauses can only accumulate into the same variable if they are
compatible.  |collect|, |adjoining|, |appending|, |nconcing|,
|unioning| and |nunioning| are compatible with each other; |sum|,
|multiply| and |counting| are compatible; |always| and |never| are compatible;
|finding| \dots |such-that| is compatible with either |thereis| or |always|
and |never| when their defaults match; and |maximize| and |minimize| clauses
are compatible only with other |maximize| and |minimize| clauses,
respectively.

%note that the same variable ~cannot~ be both an accumulation
%variable and an ordinary variable; there can be only one variable with
%a given name within an \iter\ form. 
\end{sloppypar}

\subsection{Named Blocks}

Like Common Lisp |block|s, \iter\ forms can be given names.  The
name should be a single symbol, and it must be the first form in the
\iter.  The generated code behaves exactly like a named block; in
particular, |(return-from ~name~)| can be used to exit it:
\begin{program}
(iter fred
      (for i from 1 to 10)
      (iter barney
            (for j from i to 10)
            (if (> (* i j) 17)
                (return-from fred j))))
\end{program}
An \iter\ form that is not given a name is implicitly named \nil.

Sometimes one would like to write an expression in an inner \iter\ form,
but have it processed by an outer \iter\ form.  This is possible with
the |in| clause.  

\begin{clauses}

\clausex{in}{~name~ |\&rest| ~forms~}
Evaluates ~forms~ as if they were part of the \iter\ form named
~name~.  In other words, \iter\ clauses are processed by the \iter\
form named ~name,~ and not by any \iter\ forms that occur inside ~name.~

\cpar As an example, consider the problem of collecting a list of the
elements in a two-dimensional array.  The naive solution,
\begin{program}
(iter (for i below (array-dimension ar 0))
      (iter (for j below (array-dimension ar 1))
            (collect (aref ar i j))))
\end{program}
\noindent is wrong because the list created by the inner \iter\ is simply
ignored by the outer one.  But using |in| we can write:
\begin{program}
(iter outer (for i below (array-dimension ar 0))
      (iter (for j below (array-dimension ar 1))
            (in outer (collect (aref ar i j)))))
\end{program}
\noindent which has the desired result.

\end{clauses}

\subsection{Destructuring}
\label{destructuring}

In many places within \iter\ clauses where a variable is expected, a
list can be written instead.  In these cases, the value to be assigned
is ~destructured~ according to the pattern
described by the list.  As a simple example, the clause
\begin{program}
(for (key . item) in alist)
\end{program}
\noindent will result in |key| being set to the |car| of
each element in |alist|, and |item| being set to the |cdr|.  The
pattern list may be nested to arbitrary depth, and (as the example
shows) need not be terminated with \nil; the only requirement is that
each leaf be a bindable symbol (or \nil, in which case no binding is
generated for that piece of the structure).

Sometimes, you might like to do the equivalent of a
|multiple-value-setq| in a clause.  This 
``multiple-value destructuring'' can be expressed by writing 
\linebreak |(values $pat_1$ $pat_2 \ldots$)| for a destructuring
pattern, as in 
\begin{program}
(for (values (a . b) c d) = (three-valued-function ...))
\end{program}
\begin{sloppypar}
Note that the $pat_i$ can themselves be destructuring patterns (though
not multiple-value destructuring patterns).  You can't do multiple-value
destructuring in a |with| clause; instead wrap the whole \iter\
form in a |multiple-value-bind|.
\end{sloppypar}

\begin{note}{Rationale}
There are subtle interactions between variable declarations and
evaluation order that make the correct implementation of
multiple-value destructuring in a |with| somewhat tricky.
\end{note}

The destructuring feature of \iter\ is available as a separate
mechanism, using the |dsetq| macro:

\begin{clauses}
\defmacro{dsetq}{template expr}
Performs destructuring of ~expr~ using ~template~.   May be used
outside of an \iter\ form. Yields the primary value of ~expr~.

\end{clauses}

\subsection{On-line Help}

\begin{sloppypar}
There is a limited facility for on-line help, in the form of the 
|display-iterate-clauses| function.
\end{sloppypar}

\begin{clauses}

\defun{display-iterate-clauses}{\opt\ clause-spec}
Displays a list of \iter\ clauses.  If ~clause-spec~ is not
provided, all clauses are shown; if it is a symbol, all clauses
beginning with that symbol are shown; and if it is a list of symbols,
all clauses for which ~clause-spec~ is a prefix are shown.

\end{clauses}

\subsection{Parallel Binding and Stepping}
\label{bindings}

The parallel binding and stepping of variables is a feature that
\iter\ does ~not~ have.  This section attempts to provide a rationale.

We say that two variables are bound ~in parallel~ if neither
binding shadows the other.  This is the usual semantics of |let|
(as opposed to |let*|).  Similarly, we can say that iteration
variables are stepped in parallel if neither variable is updated
before the other, conceptually speaking; in other words, if the code
to update each variable can reference the old values of both variables.

|loop| allows parallel binding of variables and parallel stepping of
driver variables.  My view is that if you are depending on the 
serial/parallel distinction, you are doing something obscure.  If you
need to bind 
variables in parallel using |with|, then you must be using a
variable name that shadows a name in the existing lexical environment.
Don't do that.  The most common use for parallel stepping is to track
the values of variables on the previous iteration, but in fact this
does not require parallel stepping at all; the following will work:
\begin{program}
(iter (for current in list)
      (for prev previous current)
      ...)
\end{program}


%%%%%%%%%%%%%%%%%%%%%%%%%%%%%%%%%%%%%%%%%%%%%%%%%%%%%%%%%%%%%%%%
\section{Types and Declarations}
\label{types}

\subsection{Discussion}

Sometimes efficiency dictates that the types of variables be declared.
This type information needs to be communicated to \iter\ so it can
bind variables to appropriate values.  Furthermore, \iter\ must often
generate internal variables invisible to the user; there needs to be a
way for these to be declared.

As an example, consider this code, which will return the number of
odd elements in |number-list|:
\begin{program}
(iter (for el in number-list)
      (count (oddp el)))
\end{program}
In processing this form,
\iter\ will create an internal variable, let us call it |list17|, to
hold the successive |cdr|s of |number-list|, and  
will bind the variable to |number-list|.  It will also generate a
default binding for |el|; only inside the body of the loop will |el|
be set to the |car| of |list17|.  Finally, \iter\ will generate a
variable, call it |result|, to hold the result of the count, and will
bind it to zero.

When dealing with type declarations, \iter\ observes one simple rule:
~it will never generate a declaration unless requested to do so.~  The
reason is that such declarations might mask errors in compiled code by
avoiding error-checks; the resulting problems would be doubly hard to
track down because the declarations would be hidden from the
programmer.  Of course, a compiler might omit error-checks even in the
absence of declarations, though this behavior can usually be avoided,
e.g. by saying |(declaim (optimize (safety 3)))|.

So, the above \iter\ form will generate code with no declarations.
But say we wish to declare the types of |el| and the internal
variables |list17| and |result|.  How is this done?

Declaring the type of |el| is easy, since the programmer knows
the variable's name:
\begin{program}
(iter (for el in number-list)
      (declare (fixnum el))
      (counting (oddp el)))
\end{program}
\iter\ can read variable type declarations like this one.  Before
processing any clauses, it scans the entire top-level form for type
declarations 
and records the types, so that variable bindings can be performed
correctly.  In this case, |el| will be bound to zero
instead of \nil.  Also, \iter\ collects all the top-level declarations
and puts them at the begining of the generated code, so it is not
necessary to place all declarations at the beginning of an \iter\
form; instead, they can be written near the variables whose types they
declare.  

Since \iter\ is not part of the compiler, it will not know
about declarations that occur outside an \iter\ form; these
declarations must be repeated inside the form.

Here is another way we could have declared the type of |el|:
\begin{program}
(iter (for (the fixnum el) in number-list)
      (counting (oddp el)))
\end{program}
\iter\ extends the Common Lisp |the|\lispindex{the}
form to apply to variables as well as value-producing forms; anywhere
a variable is allowed---in a |with| clause, as the iteration
variable in a driver clause, as the |into| argument of an
accumulation clause, even inside a destructuring template---you can
write |(the ~type~ ~symbol~)| instead.

There is one crucial difference between using a |the| form and
actually declaring the variable: explicit declarations are always
placed in the generated code, but type information from a |the|
form is not turned into an actual declaration unless you tell \iter\
to do so using |iterate:declare-variables|.  See below.

\begin{sloppypar}
Declaring the types of internal variables is harder than declaring the
types of explicitly mentioned variables, since their names
are unknown.  You do it by declaring |iterate:declare-variables|
somewhere inside the top level of the \iter\ form.  (This will also
generate declarations for variables declared using |the|.)
\iter\ does not provide much selectivity here: it's all or none. 
And unfortunately, since \iter\ is not privy to compiler information
but instead reads declarations itself, it will not hear if you
|(declaim (iterate:declare-variables))|.  Instead, set the variable 
|iterate::*always-declare-variables*| to |t| at
compile-time, using |eval-when|.
\end{sloppypar}

To determine the appropriate types for internal variables, \iter\ uses
three sources of information:

\begin{itemize}

\item Often, the particular clause dictates a certain type for a
variable; \iter\ will use this information when available.  In the
current example, the variable |list17| will be given the type
|list|, since that is the only type that makes sense; and the
variable |result| will be given the type |fixnum|, on the
assumption that you will not be counting high enough to need bignums.
You can override this assumption only by using and explicitly declaring a
variable: 
\begin{verbatim}
(iter (declare (iterate:declare-variables))
      (for el in number-list)
      (count (oddp el) into my-result)
      (declare (integer my-result))
      (finally (return my-result)))
\end{verbatim}

\begin{sloppypar}
Other examples of the type assumptions that \iter\ makes are: type
|list| for |into| variables of collection clauses; type |list| for
expressions that are to be destructured; type |vector| for the
variable holding the vector in a |for\dots in-vector| clause, and
similarly for |string| and the |for\dots in-string| clause;
and the implementation-dependent type
|(type-of array-dimension-limit)| for the index and limit
variables generated by sequence iteration drivers like |for\dots
in-vector| and |for\dots in-string| (but not |for\dots in-sequence|,
because it may be used to iterate over a list). 
\end{sloppypar}

\item Sometimes, \iter\ will examine expressions and try to determine
their types in a simple-minded way.  If the expression is
self-evaluating (like a number, for instance), \iter\ knows that the
expression's type is the same as the type of the value it denotes, so
it can use that type.  If the expression is of the form |(the ~type~
~expr~)|, \iter\ is smart enough to extract ~type~ and use it.
However, the current version of \iter\ does not  
examine declarations of function result types or do any type
inference.  It will not determine, for 
example, that the type of |(+ 3 4)| is |fixnum|, or even
|number|.

\item In some cases, the type of an internal variable should match the
type of some other variable.  For instance, \iter\ generates an
internal variable for |(f x)| in the
clause |(for i from 1 to (f x))|, and in the absence of other
information will give it the same type as |i|.  If, however, the
expression had been written |(the fixnum (f x))|, then \iter\
would have given the internal variable the type |fixnum|
regardless of |i|'s type.  The type incompatibility errors that
could arise in this situation are not checked for.

\end{itemize}

Note that if you do declare |iterate:declare-variables|, then
\iter\ may declare user variables as well as internal ones if they do
not already have declarations, though only for variables that it
binds.  For instance, in this code:

\begin{program}
(iter (declare (iterate:declare-variables))
      (for i from 1 to 10)
      (collect i into var))
\end{program}
the variable |var| will be declared to be of type |list|.


\subsection{Summary}

\iter\ understands standard Common Lisp variable type declarations
that occur within an \iter\ form and
will pass them through to the generated code.  If the declaration
|(iterate:declare-variables)|\lispindex{declare-variables}
appears at the top level of an 
\iter\ form, or if 
|iterate::*always-declare-variables*|\lispindex{*always-declare-variables*} 
is \nonnil, then \iter\ will use the type information gleaned from user
declarations, self-evaluating expressions and |the| expressions,
combined with reasonable assumptions, to determine variable
types and declare them.


%%%%%%%%%%%%%%%%%%%%%%%%%%%%%%%%%%%%%%%%%%%%%%%%%%%%%%%%%%%%%%%%
\section{Problems with Code Movement}
\label{code-movement}

Some \iter\ clauses, or parts of clauses, result in code being
moved from the location of the clause to other parts of the loop.
Drivers behave this way, as do code-placement clauses like |initially|
and |finally|.  When using these clauses, there is a danger of writing
an expression that makes sense in its apparent location but will be
invalid or have a different meaning in another location.  For example:
\begin{program}
(iter (for i from 1 to 10)
      (let ((x 3))
        (initially (setq x 4))))
\end{program}
While it may appear that the |x| of |(initially (setq x 4))| is the
same as the |x| of |(let ((x 3)) \dots|, in fact they are not:
|initially| moves its code outside the loop body, so |x| would refer
to a global variable.  Here is another example of the same problem:
\begin{program}
(iter (for i from 1 to 10)
      (let ((x 3))
        (collect i into x)))
\end{program}
If this code were executed, |collect| would create a binding for its
|x| at the top level of the \iter\ form that the |let| will shadow.

Happily, \iter\ is smart enough to catch these errors; it
walks all problematical code to ensure that free variables are not
bound inside the loop body, and checks all variables it binds for the
same problem.

However, some errors cannot be caught:

\begin{program}
(iter (with x = 3)
      (for el in list)
      (setq x 1)
      (reducing el by \#'+ initial-value x))
\end{program}
|reducing| moves its |initial-value| argument to the initialization
part of the loop in order to produce more efficient code.  Since
\iter\ does not perform data-flow analysis, it cannot determine that
|x| is changed inside the loop; all it can establish is that |x| is
not bound internally.  Hence this code will not signal an
error and will use $3$ as the initial value of the reduction.

The following list summarizes all cases that are subject to these code
motion and variable-shadowing problems.
\begin{itemize}
\item Any variable for which \iter\ creates a binding, including those
used in |with| and the |into| keyword of many clauses.

\begin{sloppypar}
\item The special clauses which place code: |initially|, |after-each|, |else|,
|finally| and |finally-protected|. 
\end{sloppypar}

\item The variables of a |next| or |do-next| form.

\item The |initially| arguments of |for\dots initially\dots then| and
|for\dots previous|. 

\item The |then| argument of |for\dots initially\dots then|.

\item The |initial-value| arguments of |reducing| and |accumulate|.

\item The |on-failure| argument of |finding\dots such-that|.
\end{itemize}

%%%%%%%%%%%%%%%%%%%%%%%%%%%%%%%%%%%%%%%%%%%%%%%%%%%%%%%%%%%%%%%%
\section{Differences Between {\tt Iterate} and {\tt Loop}}
|loop| contains a great deal of complexity which \iter\ tries to
avoid.  Hence many esoteric features of |loop| don't exist in \iter.
Other features have been carried over, but in a cleaned-up form.  
And of course, many new features have been added; they are not
mentioned in this list.

\begin{itemize}

\item \iter's syntax is more Lisp-like than |loop|'s, having a higher
density of parens.

\item The current implementation of \iter, unlike the current version
of  |loop| (as documented in {\em Common Lisp, 2nd Ed.\/}), is
extensible (see section \ref{extend}).  

\item |loop| puts the updates of all driver variables at the top of
the loop; \iter\ leaves them where the driver clauses appear.

\item While for the most part \iter\ clauses that resemble |loop| clauses
behave similarly, there are some differences.  For instance, there is
no |for\dots =\dots then| in \iter; instead use 
|for\dots initially\dots then|.  

\item |loop| binds the variable |it| at certain times
to allow pseudo-English expressions like |when ~expr~ return it|. 
In \iter, you must bind ~expr~ to a variable yourself.  Note that
|when ~expr~ return it| is like |thereis ~expr~| except that the latter is an
accumulation clause and therefore competes with other accumulations
(remember section \ref{multiple} above).

% repeat different behaviour of |always| clause here?

\item |loop| has a special |return| clause, illustrated in the
previous item.  \iter\ doesn't need one,  since an ordinary Lisp
|return| has the same effect.

\item |loop| allows for parallel binding and stepping of iteration
variables.  \iter\ does not.  (See section \ref{bindings}.)

\item |loop| and \iter\ handle variable type declarations very
differently.  |loop| provides a special syntax for declaring variable
types, and does not examine declarations.  Moreover, the standard
implementation of |loop| will
generate declarations when none are requested.
\iter\ parses standard Common Lisp type declarations, and will never
declare a variable itself unless declarations are 
specifically requested.

\end{itemize}

%%%%%%%%%%%%%%%%%%%%%%%%%%%%%%%%%%%%%%%%%%%%%%%%%%%%%%%%%%%%%%%%
\section{Rolling Your Own}
\label{extend}

\subsection{Introduction}

\iter\ is extensible---you can write new clauses that embody new
iteration patterns.  You might want to write a new driver clause for a
data structure of your own, or you might want to write a clause that
collects or manipulates elements in a way not provided by \iter. 

This section describes how to write clauses for \iter.  Writing a
clause is like writing a macro.  In fact, writing a clause ~is~
writing a macro: since \iter\ code-walks its body and macroexpands,
you can add new abstractions to \iter\ with good old |defmacro|.

Actually, there are two extensions you can make to \iter\ that are
even easier than writing a macro.  They are adding a synonym for an
existing clause and defining a driver clause for an indexable
sequence.  These can be done with |defsynonym| and
|defclause-sequence|, respectively.  See section \ref{aids}, below.

The rest of this section explains how to write macros that expand into
\iter\ clauses.
Here's how you could add a simplified version of \iter's |multiply|
clause, if \iter\ didn't already have one:
\begin{program}
(defmacro multiply (expr)
  `(reducing ,expr by \#'* initial-value 1))
\end{program}

If you found yourself summing the square of an expression often, you
might want to write a macro for that.  A first cut might be
\begin{program}
(defmacro sum-of-squares (expr)
  `(sum (* ,expr ,expr)))
\end{program}
but if you are an experienced macro writer, you will realize that this
code will evaluate ~expr~ twice, which is probably a bad idea.  A
better version would use a temporary:
\begin{program}
(defmacro sum-of-squares (expr)
  (let ((temp (gensym)))
    `(let ((,temp ,expr))
       (sum (* ,temp ,temp)))))
\end{program}
Although this may seem complex, it is just the sort of thing you'd
have to go through to write any macro, which illustrates the point of
this section: if you can write macros, you can extend \iter.

Our macros don't use \iter's keyword-argument
syntax.  We could just use keywords with |defmacro|, but we would
still not be using \iter's clause indexing mechanism.  Unlike Lisp,
which uses just the first symbol of a form to determine what function
to call, \iter\ individuates clauses by the list of required keywords.
For instance, |for\dots in| and |for\dots in-vector| are different clauses
implemented by distinct Lisp functions.  

To buy into this indexing scheme, as well as the keyword-argument
syntax, use |defmacro-clause|:

\begin{clauses}

\defmacro{defmacro-clause}{arglist |\&body| body}
Defines a new \iter\ clause.  ~arglist~ is a list of symbols which are
alternating keywords and arguments.  \opt\ may be used, and the list
may be terminated by |\&sequence|.  ~body~ is an ordinary macro body,
as with |defmacro|.  If the first form of ~body~ is a string, it is
considered a documentation string and will be shown by
|display-iterate-clauses|. |defmacro-clause| will signal an error if 
defining the clause would result in an ambiguity.  E.g. you cannot
define the clause |for\dots from| because there would be no way to
distinguish it from a use of the |for| clause with optional keyword |from|.

\end{clauses}

\medskip

Here is |multiply| using |defmacro-clause|.  The keywords are capitalized
for readability.
\begin{program}
(defmacro-clause (MULTIPLY expr \opt\ INTO var)
  `(reducing ,expr by \#'* into ,var initial-value 1))
\end{program}
You don't have to worry about the case when |var| is not supplied; for
any clause with an |into| keyword, saying |into nil| is equivalent to
omitting the |into| entirely.

As another, more extended example, consider the fairly common
iteration pattern that 
involves finding the sequence element that maximizes (or minimizes) some
function.  \iter\ provides this as |finding\dots maximizing|, but it's
instructive to see how to write it.
Here, in pseudocode, is how you might write such a loop for
maximizing a function F:

\begin{quote}
\begin{tabbing}
set variable MAX-VAL to NIL; \\
set variable WINNER to NIL; \\
for\=\ each element EL in the sequence \\
\>  if\=\ MAX-VAL\=\ is NIL or F(EL) $>$ MAX-VAL then \\
\>\>    set MAX-VAL to F(EL); \\
\>\>    set WINNER to EL; \\
\>  end if; \\
end for; \\
return WINNER.
\end{tabbing}
\end{quote}

Here is the macro:
\begin{program}
(defmacro-clause (FINDING expr MAXIMIZING func \opt\ INTO var)
  (let ((max-val (gensym))
        (temp1 (gensym))
        (temp2 (gensym))
        (winner (or var iterate::*result-var*)))
    `(progn 
       (with ,max-val = nil)
       (with ,winner = nil)
       (cond 
        ((null ,max-val)
         (setq ,winner ,expr)
         (setq ,max-val (funcall ,func ,winner))
        (t
         (let* ((,temp1 ,expr)
                (,temp2 (funcall ,func ,temp1)))
           (when (> ,temp2 ,max-val)
             (setq ,max-val ,temp2)
             (setq ,winner ,temp1))))))
       (finally (leave ,winner)))))
\end{program}
Note that if no |into| variable is supplied, we use
|iterate::*result-var*|, which contains the internal variable into
which all clauses place their results.  If this variable is bound by
some clause, then \iter\ will return its value automatically;
otherwise, \nil\ will be returned.  

\subsection{Writing Drivers}

In principle, drivers can be implemented just as easily as other
\iter\ clauses.  In practice, they are a little harder to get right. 
As an example, consider writing a driver that
iterates over all the 
elements of a vector, ignoring its fill-pointer.  |for\dots in-vector|
won't work for this, because it observes the fill-pointer.  It's necessary to
use |array-dimension| instead of |length| to obtain the size of the
vector.  Here is one approach:

\begin{program}
(defmacro-clause (FOR var IN-WHOLE-VECTOR v)
  "All the elements of a vector (disregards fill-pointer)"
  (let ((vect (gensym))
        (index (gensym)))
    `(progn
       (with ,vect = ,v)
       (for ,index from 0 below (array-dimension ,vect 0))
       (for ,var = (aref ,vect ,index)))))
\end{program}
Note that we immediately put |v| in a variable, in case it is an
expression.  Again, this is just good Lisp macrology.  It also has a
subtle effect on the semantics of the driver: |v| is evaluated only
once, at the beginning of the loop, so changes to |v| in the loop have
no effect on the driver.  Similarly, the bounds for numerical iteration
e.g. the above |array-dimension| are also evaluated once only.  This is how
all of \iter's drivers work.


There is an important point concerning the |progn| in this code.  We
need the |progn|, of course, because we are returning several forms,
one of which is a driver.  But \iter\ drivers must occur at top-level.
Is this code in error?  No, because ~top-level~ is defined in \iter\
to include forms inside a |progn|.  This is just the definition of
top-level that Common Lisp uses, and for the same reason: to allow
macros to return multiple forms at top-level.

While our |for\dots in-whole-vector| clause will work, it is not
ideal.  In particular, it does not support generating.  Do do so, we
need to use |for\dots next| or |for\dots do-next|.  The job is
simplified by the |defmacro-driver| macro.

\begin{clauses}

\defmacro{defmacro-driver}{arglist |\&body| body}
Defines a driver clause in
both the |for| and |generate| forms, and provides a parameter
|generate| which ~body~ can examine to determine how it was invoked.
~arglist~ is as in |defmacro-clause|, and should begin with the symbol
|for|. 

\end{clauses}

With |defmacro-driver|, our driver looks like this:
\begin{program}
(defmacro-driver (FOR var IN-WHOLE-VECTOR v)
  "All the elements of a vector (disregards fill-pointer)"
   (let ((vect (gensym))
         (end (gensym))
         (index (gensym))
         (kwd (if generate 'generate 'for)))
     `(progn
        (with ,vect = ,v)
        (with ,end = (array-dimension ,vect 0))
        (with ,index = -1)
        (,kwd ,var next (progn (incf ,index)
                               (if (>= ,index ,end) (terminate))
                               (aref ,vect ,index))))))
\end{program}


We are still missing one thing: the |\&sequence| keywords.  
We can get them easily enough, by writing 
\begin{program}
(defmacro-driver (FOR var IN-WHOLE-VECTOR v \&sequence) 
  ...)
\end{program}
We can now refer to parameters |from|, |to|, |by|, etc. which contain
either the values for the corresponding keyword, or \nil\ if the
keyword was not supplied.  Implementing the right code for these
keywords is cumbersome but not difficult; it is left as an exercise.
But before you begin, see |defclause-sequence| below for an easier way.
       
\subsection{Extensibility Aids}
\label{aids}

This section documents assorted features that may be of use in
extending \iter.

\begin{clauses}

\defunexpvar{*result-var*}
Holds the variable that is used to return a value as a result of the
\iter\ form.  You may examine this and use it in a |with| clause, but
you should not change it.

\defmacro{defsynonym}{syn word}
 Makes ~syn~ a synonym for the existing \iter\ keyword ~word.~  Only
the first word in each clause can have synonyms. 

\defmacroxx{defclause-sequence}{element-name index-name}{|\&key|
access-fn size-fn sequence-type}{element-type
element-doc-string index-doc-string}  
Provides
a simple way to define sequence clauses.  Generates two
clauses, one for iterating over the sequence's elements, the other
for iterating over its indices.  The first symbol of both
clauses will have print-name |for|.
~element-name~  and ~index-name~ should be symbols.
~element-name~ is the second keyword of the element iterator (typically of
the form 
|in-~sequence-type~|), and ~index-name~ is the second keyword
of the index-iterator (typically of the form
|index-of-~sequence-type~|).  Either name may be 
\nil, in which case the corresponding clause is not defined.  If both
symbols are supplied, they should be in the same package.  The |for|
that begins the clauses will be in this package.

\cpar ~access-fn~ is the function to be used to
access elements of the sequence in the element iterator.  The function
should take two 
arguments, a sequence and an index, and return the appropriate element.
~size-fn~ should denote a function of one argument, a sequence, that
returns its size.  Both ~access-fn~ and ~size-fn~ are required for the
element iterator, but only ~size-fn~ is needed for the index iterator.

\cpar The ~sequence-type~ and ~element-type~ keywords are used to
suggest types for the variables 
used to hold the sequence and the
sequence elements, respectively.  The usual rules about \iter's
treatment of variable type declarations apply (see section \ref{types}).

\cpar ~element-doc-string~ and ~index-doc-string~ are
the documentation strings, for use with |display-iterate-clauses|. 

\cpar The generated element-iterator performs destructuring on the
element variable.

\cpar As an example, the above |for\dots in-whole-vector| example
could have been written:
\begin{program}
(defclause-sequence IN-WHOLE-VECTOR INDEX-OF-WHOLE-VECTOR
  :access-fn 'aref
  :size-fn \#'(lambda (v) (array-dimension v 0))
  :sequence-type 'vector
  :element-type t
  :element-doc-string 
     "Elements of a vector, disregarding fill-pointer"
  :index-doc-string 
     "Indices of vector, disregarding fill-pointer")
\end{program}

\end{clauses}

\subsection{Subtleties}

There are some subtleties to be aware of when writing \iter\ clauses.
First, the code returned by your macros may be |nconc|'ed into a list,
so you should always returned freshly consed lists, rather than
constants.  Second, \iter\ matches clauses by using |eq| on the first
symbol and |string=| on the subsequent ones, so the package of the
first symbol of a clause is relevant.  All of the clauses in this manual
have their first word in the \iter\ package.  
You can use the package system in the usual way to shadow
\iter\ clauses without replacing them.

%%% say more here, about the badness that only the first word of a
%%% clause is packagey.

\section{Non-portable Extensions to Iterate (Contribs)}
\label{contribs}

Currently, there is only one non-portable extension to iterate in the
distribution: iterate-pg. If you have made an extension that depends
on non-portable features, feel free to send them to |asf@boinkor.net|
for inclusion in the iterate distribution.

\subsection{An SQL query driver for iterate}

The pg package by Eric Marsden (see |http://cliki.net/pg|) provides an
interface to the PostgreSQL database. Using the \iterpg\ extension, it
is possible to handle the results of SQL queries using \iter.

This usage example should give you an idea of how to use it:

\begin{program}
(pg:with-pg-connection (c "somedb" "someuser")
  (iter (for (impl version date) in-relation "select * from version"
                                 on-connection *dbconn*)
        (collect version)))
\end{program}

To use the extension via |ASDF|, simply make your system depend on the
|iterate-pg| system instead of the |iterate| system. To load it
manually, use:

\begin{program}
  (asdf:oos 'asdf:load-op :iterate-pg)
\end{program}

%%%%%%%%%%%%%%%%%%%%%%%%%%%%%%%%%%%%%%%%%%%%%%%%%%%%%%%%%%%%%%%%
\section{Obtaining {\tt Iterate}}

\begin{sloppypar}
The information in this chapter is obsolete but included for
completeness's sake; Currently, the most up-to-date information on
\iter\ can be found at |http://boinkor.net/iterate.html|.
\end{sloppypar}

\begin{sloppypar}
\iter\ currently runs on Lisp Machines, and on
HP's, Sun3's and Sparcstations under Lucid.  
\iter\ source and binaries are available at the MIT AI Lab in the
subdirectories of |/src/local/lisplib/|.  The source file,
|iterate.lisp|, is also available for anonymous FTP in the directory
|/com/fpt/pub/| on the machine |TRIX.AI.MIT.EDU| (Internet number
128.52.32.6).  If you are unable to obtain |iterate| in one of these
ways, send mail to |jba@ai.mit.edu| and I will send you the source
file. 
\end{sloppypar}

\begin{sloppypar}
\iter\ resides in the |iterate| package (nickname |iter|).  Just say
\linebreak |(use-package :iterate)| to make all the necessary symbols
available. 
If a symbol is not exported, it appears in this manual with an
``|iterate::|'' prefix.
\end{sloppypar}

Send bug reports to |bug-iterate@ai.mit.edu|.  The |info-iterate|
mailing list will have notices of changes and problems; to have
yourself added, send mail to |info-iterate-request@ai.mit.edu|.


\medskip

\begin{flushleft}
 \bf Acknowledgements
\end{flushleft}
\smallskip

Richard Waters provided invaluable criticism which spurred me to improve
\iter\ greatly.  As early users, David Clemens, Oren Etzioni and Jeff
Siskind helped ferret out many bugs.

%\begin{theindex}
% The index files must be generated with the genindex program
% or from makeindex iterate-manual.idx which creates iterate-manual.ind.

%\baselineskip\normalbaselineskip
%\advance\baselineskip by -2pt

% Fix ``.'' spacing in lists
\documentstyle[12pt,aimemo]{article}

\setlength{\textwidth}{6.0in}
\setlength{\oddsidemargin}{0.25in}
\setlength{\evensidemargin}{0.25in}
\setlength{\topmargin}{0pt}
\setlength{\headsep}{24pt}
\setlength{\textheight}{8.5in}

\let\origpar=\par

% somehow, LaTex fucks with things so that the plain Tex \obeyspaces
% doesn't work.  This is the same, except we use an mbox containing
% a space instead of just a space.

\newcommand{\spc}{\mbox{ }}
\newcommand{\tAb}{\mbox{        }}
\def\obspaces{\catcode`\ =\active\catcode`\^^I=\active}
{\obspaces \global\let =\spc\global\let^^I=\tAb}

\newenvironment{program}{\medskip\samepage\tt\obeylines\obspaces}{\medskip}

% use | for \tt
\catcode`\|=13
\newif\ifvbar
\def|{\ifvbar
         \endgroup\vbarfalse
      \else 
         \begingroup\tt\vbartrue 
      \fi}



\newcommand{\lisp}{\tt}
\newcommand{\iter}{|iterate|}
\newcommand{\iterpg}{|iterate-pg|}
\newcommand{\setf}{|setf|}
\newcommand{\nil}{|nil|}
\newcommand{\nonnil}{non-\nil}
\newcommand{\opt}{{\lisp \&optional}}
\newcommand{\yields}{$\Rightarrow$}



\newbox\Dots
\setbox\Dots=\hbox{\small ...}
\renewcommand{\dots}{\copy\Dots}

% Here's a way to define an environment which will treat all
% paragraphs inside it in the same manner.  It's, as near as I can
% figure, the way that Latex does things.  Using \everypar won't work;
% apparently Latex resets that a lot.  What you have to do is: at the
% top of the file do \let\origpar=\par.  In the
% begin part of the environment, do the parshape (or whatever), then say
% \def\par{{\origpar}}.  The extra braces are necessary.   That's all
% you have to do; at the end of the environment things will return to
% normal automatically.  Why does this all work?  I have no idea.
% Another important thing: there should be a blank line (i.e. a \par)
% between the end of the last paragraph and the \end{environment}.

\newlength{\clindent}
\setlength{\clindent}{0.5in}
\newlength{\clparindent}
\setlength{\clparindent}{\parindent}

\newenvironment{clauses}{\advance\linewidth -\parindent
                        \hangindent\clindent\relax
                         \hangafter=0
                          \parindent=0pt
                          \def\par{{\origpar}}}{}

\newcommand{\cpar}{\hspace \clparindent}

\newcommand{\startitem}{\bigskip\pagebreak[3]}
\newcommand{\finishitem}{\smallskip}

\newcommand{\clause}[1]{\startitem\Clause{#1}\finishitem}
\newcommand{\Clause}[1]{\hbox{#1}}
\newcommand{\clindex}[1]{\index{{\lisp #1}}}
\newcommand{\clindexx}[2]{\index{{\lisp #1\dots #2}}}
\newcommand{\clindexxx}[3]{\index{{\lisp #1\dots #2\dots #3}}}
\newcommand{\clausex}[2]{\startitem\Clausex{#1}{#2}\finishitem}
\newcommand{\Clausex}[2]{\Clause{|#1| #2}\clindex{#1}}

\newcommand{\defvar}[1]{\startitem\deff{#1}{}{Variable}\finishitem}
\newcommand{\defunexpvar}[1]{\startitem\deffnoindex{iterate::#1}{}{Variable}
                                        \lispindex{#1}\finishitem}
\newcommand{\defun}[2]{\startitem\deff{#1}{#2}{Function}\finishitem}
\newcommand{\defmacro}[2]{\startitem\deff{#1}{#2}{Macro}\finishitem}
\newcommand{\defunx}[3]{\startitem\deffx{#1}{#2}{#3}{Function}\finishitem}
\newcommand{\defmacrox}[3]{\startitem\deffx{#1}{#2}{#3}{Macro}\finishitem}
\newcommand{\defunxx}[4]{\startitem\deffxx{#1}{#2}{#3}{#4}{Function}\finishitem}
\newcommand{\defmacroxx}[4]{\startitem\deffxx{#1}{#2}{#3}{#4}{Macro}\finishitem}

\newcommand{\deff}[3]{\deffnoindex{#1}{#2}{#3}\lispindex{#1}}
\newcommand{\deffnoindex}[3]{\hbox to \hsize{{\tt #1} {\it #2}\hfill [{\it #3}]}}
\newcommand{\deffx}[4]{\deff{#1}{#2}{#4}\moreargs{#1}{#3}}
\newcommand{\deffxx}[5]{\deffx{#1}{#2}{#3}{#5}\moreargs{#1}{#4}}

\newcommand{\moreargs}[2]{\hbox to \hsize{\phantom{\lisp #1} {\it #2}\hfill}}

\newcommand{\lispindex}[1]{\index{{\lisp #1}}}

 \newenvironment{note}[1]{\pagebreak[2]\bigskip
                          \hrule\smallskip\small
                         {\setlength{\parindent}{0pt}
                           \par{\bf #1:}}}{\smallskip\hrule\bigskip}



\makeindex

%%%%%%%%%%%%%%%%%%%%%%%%%%%%%%%%%%%%%%%%%%%%%%%%%%%%%%%%%%%%%%%%

%\thispagestyle{empty}

%\vspace*{2in}

%\title{The |Iterate| Manual}
%\author{Jonathan Amsterdam}

%\maketitle

%\thispagestyle{empty}

%\begin{center}
%\Large *** DRAFT ***
%\end{center}
%This manual is in beta-test. It will become an AI Memo in a couple of
%months.  I would appreciate any comments or
%corrections.  You can reach me at email address |jba@ai.ai.mit.edu|.

\begin{document}

\def\writtenby{Jonathan Amsterdam}
\def\contractno{This report describes research done at the Artificial
Intelligence Laboratory of the Massachusetts Institute of Technology.
Support for the laboratory's artificial intelligence research is
provided in part by the Advanced Research Projects Agency of the
Department of Defense under Office
of Naval Research contract N00014-85-K-0124.}

\AIMEMO{1236}{\Large The Iterate Manual \\[1in]}
{This is the manual for version 1.4 of \iter, a powerful iteration
macro for Common  
Lisp. \iter\ is similar to {\tt loop} but provides numerous additional
features, is well integrated with Lisp, and is extensible. 
\vfill }


% use ~ for \em
\catcode`\~=13
\newif\ifem
\def~{\ifem
         \/\endgroup\emfalse
      \else 
         \begingroup\em\emtrue 
      \fi}



%\pagebreak
%\setcounter{page}{1}

%\thispagestyle{empty}


\tableofcontents

\pagebreak
\pagestyle{headings}

\section{Introduction}

\begin{sloppypar}
This manual describes
\iter, a powerful iteration facility for Common Lisp.
\iter\ provides abstractions for many common iteration
patterns and allows for the definition of additional patterns.  \iter\ is
a macro that expands into ordinary Lisp at
compile-time, so it is more efficient than higher-order
functions like |map| and |reduce.|  
While it is similar to |loop|, \iter\ offers a more Lisp-like syntax
and enhanced extensibility.
(For a more complete
comparison of \iter\ with other iteration constructs, see MIT AI Lab Working
Paper No. 324, ~Don't Loop, Iterate.~)
\end{sloppypar}

An \iter\ form consists of the symbol |iter|\footnote{You can also use
|iterate|, but |iter| is preferred because it avoids potential conflicts with
possible future additions to Common Lisp, and because it saves
horizontal space when writing code.}
followed by one or more
forms, some of which may be \iter\ 
~clauses.~  Here is a simple example of \iter\ which collects the
numbers from 1 to 10 into a list, and returns the list.  The return
value is shown following the arrow.
\begin{program}
(iter (for i from 1 to 10)
      (collect i))          \yields (1 2 3 4 5 6 7 8 9 10)
\end{program}
This form contains two clauses: a |for| clause that steps the
variable |i| over the integers from 1 to 10, and a |collect|
clause that accumulates its argument into a list.  With a few
exceptions, all \iter\ clauses have the same format: alternating
symbols (called ~keywords~) and expressions (called
~arguments~).  The syntax and terminology are those of Common
Lisp's keyword lambda lists.  One difference is that \iter's keywords
do not have to begin with a colon---though they may, except for the
first symbol of a clause.  So you can also write  |(for i :from
1 :to 10)| if you prefer.

Any Lisp form can appear in the body of an \iter, where it will have
its usual meaning.  \iter\ walks the entire body, expanding macros,
and recognizing clauses at any level.  This example collects all the
odd numbers in a list:
\begin{program}
(iter (for el in list)
      (if (and (numberp el) (oddp el))
          (collect el)))
\end{program}

There are clauses for iterating over numbers, lists, arrays and other
objects, and for
collecting, summing, counting, maximizing and other useful operations.
\iter\ also supports the creation of new variable bindings, stepping
over multiple sequences at once, destructuring, and compiler
declarations of variable types.  The following example illustrates
some of these features: 
\begin{program}
(iter (for (key . item) in alist)
         (for i from 0)
         (declare (fixnum i))
         (collect (cons i key)))
\end{program}

This loop takes the keys of an alist and returns a new alist
associating the keys with their positions in the original list. The
compiler declaration for |i| will appear in the generated code
in the appropriate place.

%%%%%%%%%%%%%%%%%%%%%%%%%%%%%%%%%%%%%%%%%%%%%%%%%%%%%%%%%%%%%%%%
\section{Clauses}

Most of \iter's clauses will be familiar to |loop| programmers.  
(|loop| is an iteration macro that has been incorporated into Common
Lisp.  Seen Guy Steele's {\em Common Lisp, 2nd Edition}.)
In
nearly all cases they behave the same as their |loop| counterparts, so
a |loop| user can switch to \iter\ with little pain (and much gain).

All clauses with the standard keyword-argument syntax consist of two
parts: a ~required~ part, 
containing keywords that must be present and in the right order; and
an ~optional~ part, containing keywords that may be omitted and,
if present, may occur in any order.  In the descriptions below, the
parts are separated by the Lisp lambda-list keyword \opt.


\subsection{Drivers}

An iteration-driving clause
conceptually causes the iteration to go forward.  Driver clauses in
\iter\ allow iteration over numbers, lists, vectors, hashtables, packages,
files and streams.  Iteration-driving clauses must
appear at the top level of an \iter\ form; they cannot be nested
inside another clause.  The driver variable is updated at the point
where the driver clause occurs.  Before the clause is executed for the
first time, the value of the variable is undefined.

%Also, regardless of where the driver clause
%appears in the body, the driver variable is stepped at the top of the
%loop; hence it is stylistically preferable, though
%not required, to place driver clauses at the beginning of the \iter.

Multiple drivers may appear in a single \iter\ form, in which case all
of the driver variables are updated each time through the loop, in the
order in which the clauses appear.  The first driver to terminate will
terminate the entire loop.

In all cases, the value of the driver variable on exit from the loop,
including within the epilogue code (see the |finally| clause), is
undefined.

All the parameters of a driver clause are evaluated once, before the
loop begins.  Hence it is not possible to change the bounds or other
properties of an iteration by side-effect from within the loop.

With one exception, driver clauses begin with the word |for| (or
the synonym |as|) and mention an iteration variable, which is
given a binding within the \iter\ form.
The exception is |repeat|, which just executes a loop a
specified number of times:

\begin{clauses}

\clausex{repeat}{~n~}
Repeats the loop ~n~ times.  For example:
\begin{program}
(iter (repeat 100)
      (print "I will not talk in class."))
\end{program}
\cpar If $n \leq 0$, the
loop will never be executed.  If ~n~ is not an integer, the
actual number of executions will be $\lceil n \rceil$.

\end{clauses}

\pagebreak[3]

\subsubsection{Numerical Iteration}

\begin{clauses}
\clausex{for}{~var~ |\&sequence|} 
The general form for iterating over a sequence of numbers requires a
variable and, optionally, one or more keywords that provide the bounds
and step size of the iteration. The |\&sequence|
lambda-list keyword is a shorthand for these sequence keywords.  They are:
|from|, |upfrom|, |downfrom|, |to|, |downto|, |above|, |below| and |by|.
|from| provides the starting value for ~var~
and defaults to zero. |to| provides a final value and implies that the
successive values of ~var~ will be increasing; |downto| implies that
they will be decreasing.  The loop terminates when ~var~ passes
the final value (i.e. becomes smaller or larger than it, depending on
the direction of iteration); in other  words, the loop body
will never be 
executed for values of ~var~ past the final value.
|below| and |above| are similar to |to| and
|downto|, except that the loop terminates when ~var~ equals or passes
the final value.

\cpar If no final value is specified, the variable will be stepped
forever.  Using |from| or |upfrom| will result in increasing
values, while |downfrom| will give decreasing values.

\cpar On each iteration,
~var~ is incremented or decremented by the value of the sequence
keyword |by|, which defaults to 1.  It should always be a positive
number, even for downward iterations.

\cpar In the following examples, the sequence of numbers generated is shown
next to the clause.

\begin{program}
(for i upfrom 0) \yields\ 0 1 2 \ldots
(for i from 5)   \yields\ 5 6 7 \ldots    ; either from or upfrom is okay
(for i downfrom 0) \yields\ 0 -1 -2 \ldots
(for i from 1 to 3) \yields\ 1 2 3
(for i from 1 below 3) \yields\ 1 2
(for i from 1 to 3 by 2) \yields\ 1 3
(for i from 1 below 3 by 2) \yields\ 1
(for i from 5 downto 3) \yields\ 5 4 3
\end{program}

\end{clauses}

\subsubsection{Sequence Iteration}

There are a number of clauses for iterating over sequences.  In all of
them, the argument following |for| may be a list instead of a
symbol, in which case destructuring is performed.  See section
\ref{destructuring}. 

\begin{clauses}

\clause{|for| ~var~ |in| ~list~ \opt\ |by| ~step-function~}
\clindexx{for}{in}
~var~ is set to successive elements of list.  
~step-function,~ which 
defaults to |cdr|, is used to obtain the next sublist. 

\clause{|for| ~var~ |on| ~list~ \opt\ |by| ~step-function~}
\clindexx{for}{on}
~var~ is set to successive sublists of list. 
~step-function~ (default |cdr|) is used as in |for\dots in|.

\end{clauses}

\medskip

These two clauses use |atom| to test for the end
of a list.  Hence, given a list whose final |cdr| is not \nil,
they will silently ignore the last |cdr|.  Other choices are
|endp|,  which would signal an error, and |null|, which
would probably result in an error somewhere else.  If you wish to use
an end-test other than |atom|, set the variable
|iterate::*list-end-test*|\lispindex{*list-end-test*} to the name of
the desired function. 

\begin{clauses}
\clause{|for| ~var~ |in-vector| ~vector~ |\&sequence|}
\clindexx{for}{in-vector}
~var~ takes on successive elements from ~vector.~   The vector's
fill-pointer is observed.  Here and in subsequent clauses, the |\&sequence|
keywords include |with-index|, which takes a symbol as argument
and uses it for the index variable instead of an internally generated
symbol.  The other |\&sequence| keywords behave as in numerical
iteration, except that the default iteration bounds are the bounds of
the vector.  E.g. in |(for i in-vector v downto 3)|,
|i| will start off being bound to the last element in |v|,
and will  
be set to preceding elements down to and including the element with
index 3.

\clause{|for| ~var~ |in-sequence| ~seq~ |\&sequence|}
\clindexx{for}{in-sequence}
This uses Common Lisp's generalized sequence functions, |elt|
and |length|, to obtain elements and determine the length of
~seq.~   Hence it will work for any sequence, including lists, and will
observe the fill-pointers of vectors.

\clause{|for| ~var~ |in-string| ~string~ |\&sequence|}
\clindexx{for}{in-string}
~var~ is set to successive characters of ~string.~

\startitem
\Clause{|for| ~var~ |index-of-vector| ~vector~ |\&sequence|}
\clindexx{for}{index-of-vector}
\Clause{|for| ~var~ |index-of-sequence| ~sequence~ |\&sequence|}
\clindexx{for}{index-of-sequence}
\Clause{|for| ~var~ |index-of-string| ~string~ |\&sequence|}
\clindexx{for}{index-of-string}
\finishitem
~var~ is set to successive indices of the sequence.
These clauses avoid the overhead of accessing the sequence elements
for those applications where they do not need to be examined, or are
examined rarely.  They admit all the optional keywords of the other
sequence drivers except the (redundant) |with-index| keyword.

\clause{|for| ~(key value)~ |in-hashtable| ~table~}
\clindexx{for}{in-hashtable}

~key~ and ~value,~ which must appear as shown in a list
and may be destructuring templates, are set to the keys and values of
~table~.  If ~key~ is |nil|, then the hashtable's keys will be ignored;
similarly for ~value~.
The order in which 
elements of ~table~ will be retrieved is unpredictable.

\clause{|for| ~var~ |in-package| ~package~ \opt\ |external-only| ~ext~}
\clindexx{for}{in-package}
Iterates over all the symbols in ~package,~ or over only the
external symbols if ~ext~ is specified and non-|nil|.  ~ext~ is not
evaluated.  The same symbol may appear more than once.

\clause{|for| ~var~ |in-packages| ~packages~ \opt\ |having-access| ~symbol-types~}
\clindexx{for}{in-packages}
Iterates over all the symbols from the list of packages denoted by the
descriptor ~packages~ and having accessibility (or visibility) given by
~symbol-types~.  This defaults to the list |(:external :internal :inherited)|
and is not evaluated.  ~var~ must be a list of up to three variables: in each
iteration, these will be set to a symbol, its access-type and package (as per
|with-package-iterator| in ANSI-CL).  The same symbol may appear more than
once.

\end{clauses}

\begin{clauses}

\clause{|for| ~var~ |in-file| ~name~ \opt\ |using| ~reader~}
\clindexx{for}{in-file}
Opens the file ~name~ (which may be a string or pathname) for input,
and iterates 
over its contents. ~reader~ defaults to |read|, so by default ~var~
will be bound to the successive forms in the file.  The \iter\ body is
wrapped in an unwind-protect to ensure that the file is closed no
matter how the \iter\ is exited.

\clause{|for| ~var~ |in-stream| ~stream~ \opt\ |using| ~reader~}
\clindexx{for}{in-stream}
Like |for\dots in-file|, except that ~stream~ should be an existing
stream object that supports input operations.

\end{clauses}

\subsubsection{Generalized Drivers}

These are primarily useful for writing drivers that can also be used
as generators (see section \ref{generators}, below).
 
\begin{clauses}

\clause{|for| ~var~ |next| ~expr~}
\clindexx{for}{next}

~var~ is set to ~expr~ each time through the loop.  Destructuring is
performed.  When the clause is used as a generator, ~expr~ is the code
that is executed when |(next ~var~)| is 
encountered (see section \ref{generators}, below).
~expr~ should compute the first value for ~var~, as well
as all subsequent values, and is responsible for terminating the loop.
For compatibility with future versions of \iter, this termination
should be done with |terminate|\clindex{terminate}, which can be
considered a synonym for |finish| (see section \ref{control-flow}).

\cpar As an example, the following clauses are equivalent to |(for
i from 1 to 10)|:
\begin{program}
(initially (setq i 0))
(for i next (if (> i 10) (terminate) (incf i)))
\end{program}

\clause{|for| ~var~ |do-next| ~form~}
\clindexx{for}{do-next}
~form~ is evaluated each time through the loop.  Its value is ~not~
set to ~var~; that is ~form's~ job. ~var~ is only present so that
\iter\ knows it is a driver variable.  \linebreak |(for ~var~ next
~expr~)| is equivalent to |(for ~var~ do-next (dsetq ~var~ ~expr~))|.
(See section \ref{destructuring} for an explanation of |dsetq|.) 

\end{clauses}

\subsubsection{Generators}
\label{generators}

In all of the above clauses, the driver variable is updated on each
iteration.
Sometimes it is desirable to have greater control over updating.  
For instance, consider the problem of associating numbers, in
increasing order and with no gaps, with the
\nonnil\ elements of a list.  One obvious first pass at writing this is:

\begin{program}
(iter (for el in list)
      (for i upfrom 1)
      (if el (collect (cons el i))))
\end{program}
But on the list |(a b nil c)| this produces |((a . 1) (b . 2) (c .
4))| instead of the desired |((a . 1) (b . 2) (c . 3))|.  The problem
is that |i| is incremented each time through the loop, even when |el|
is |nil|.  

The problem could be solved elegantly if we could step |i| only when
we wished to.  This
can be accomplished for any \iter\ driver by writing |generate| (or
its synonym |generating|)
instead of |for|.  Doing so produces a ~generator~---a driver whose
values are yielded explicitly.  To obtain the next value of a
generator variable ~v~, write \linebreak |(next ~v~)|.  The value of a |next|
form is the next value of ~v~, as determined by its associated driver
clause.  |next| also has the side-effect of updating ~v~ to that
value.  If there is no next value, |next| will terminate the loop,
just as with a normal driver.


Using generators, we can now write our example like this:
\begin{program}
(iter (for el in list)
      (generate i upfrom 1)
      (if el (collect (cons el (next i)))))
\end{program}
Now |i| is updated only when |(next i)| is executed, and this occurs
only when |el| is \nonnil.

To better understand the relationship between ordinary drivers and
generators, observe that we can rewrite an ordinary driver using its
generator form immediately followed by |next|, as this example shows:
\begin{program}
(iter (generating i from 1 to 10)
      (next i)
      ...)
\end{program}
Provided that the loop body contains no |(next i)| forms, this will
behave just as if we had written |(for i from 1 to 10)|.

We can still refer to a driver variable ~v~ without using |next|; in
this case, its value is that given to it by the last evaluation of
|(next ~v~)|.  Before |(next ~v~)| has been called the first time, the
value of ~v~ is undefined.  

This semantics is more flexible than
one in which ~v~ begins the loop bound to its first value and calls
of |next| supply subsequent values, because it means the loop will not
terminate too soon if the generator's sequence is empty.  For
instance, consider the following code, which tags \nonnil\ elements of
a list using a list of tags, and also counts the null elements.
(We assume there are at least as many tags as \nonnil\ elements.)
\begin{program}
(let* ((counter 0)
       (tagged-list (iter (for el in list)
                          (generating tag in tag-list) 
                          (if (null el)
                              (incf counter)
                              (collect (cons el (next tag)))))))
  ...)
\end{program}
It may be that there are just as many tags as non-null elements of
|list|.  If all the elements of |list| are null, we still want the
counting to proceed, even though |tag-list| is \nil.  If |tag| had to be
assigned its first value before the loop begins, we would have had to
terminate the loop before the first iteration, since when |tag-list|
is \nil, |tag| 
has no first value.  With the existing semantics, however, |(next
tag)| will never execute, so the iteration will cover all the elements of
|list|. 
        
When the ``variable'' of a driver clause is actually a destructuring
template containing several variables, all the variables are eligible
for use with |next|.  As before, |(next ~v~)| evaluates to ~v's~ next
value; but the effect is to update all of the template's variables.
For instance, the following code will return the list |(a 2 c)|.
\begin{program}
(iter (generating (key . item) in '((a . 1) (b . 2) (c . 3)))
      (collect (next key))
      (collect (next item)))
\end{program}

Only driver clauses with variables can be made into generators.  This
includes all clauses mentioned so far except for |repeat|.  It does
~not~ include |for\dots previous|, |for\dots =|,
|for\dots initially\dots then| or |for\dots first\dots then| (see
below). 

\subsubsection{Previous Values of Driver Variables}

Often one would like to access the value of a variable on a previous
iteration.  \iter\ provides a special clause for accomplishing this.

\begin{clauses}

\clause{|for| ~pvar~ |previous| ~var~ \opt\ |initially| ~init~
|back| ~n~}
\clindexx{for}{previous}
Sets ~pvar~ to the previous value of ~var,~ which should be a driver
variable, a variable from another |for\dots previous| clause, or a
variable established by a |for\dots =|, 
|for\dots initially\dots then| or |for\dots first\dots then| clause
(see section \ref{setting}).
Initially, ~pvar~ is given the value ~init~ (which defaults to \nil).  
The ~init~ expression will be moved outside the loop body, so it
should not depend on anything computed within the loop.
~pvar~ retains the value of ~init~ until ~var~ is set to its second
value, at which point ~pvar~ is set to ~var's~ first value; and so on.  

\cpar The
argument ~n~ to |back| must be a constant, positive integer, and
defaults to 1.  It determines how many iterations back ~pvar~ should
track ~var~.  For example, when ~n~ is 2, then ~pvar~ will be assigned
~var's~ first value when ~var~ is set to its third value.

\cpar A |for\dots previous| clause may occur after or before its
associated driver clause. |for\dots previous| works with generators as
well as ordinary drivers.

\pagebreak[3]

\cpar Example:
\begin{program}
(iter (for el in '(1 2 3 4))
      (for p-el previous el)
      (for pp-el previous p-el initially 0)
      (collect pp-el))
\end{program}
This evaluates to |(0 0 1 2)|.  It could have been written more
economically as 
\begin{program}
(iter (for el in '(1 2 3 4))
      (for pp-el previous el back 2 initially 0)
      (collect pp-el))
\end{program}


\end{clauses}

\subsection{Variable Binding and Setting}
\label{setting}

Several clauses exist for establishing new variable bindings or for
setting variables in the loop.  They all support destructuring.

\begin{clauses}

\clausex{with}{~var~ \opt\ |=| ~value~}
Causes ~var~ to be bound to value before the loop body is entered.
If ~value~ is not supplied, ~var~ assumes a default
binding, which will be 
\nil\ in the absence of declarations.  Also, if ~value~ is not
supplied, no destructuring is performed; instead, ~var~ may be a list of
symbols, all of which are given default 
bindings.  If ~value~ is supplied, ~var~ is bound to it, with
destructuring.  

\cpar Because |with|
creates bindings whose scope includes the entire \iter\ form, it is
good style to put all |with| clauses at the beginning.

\cpar Successive occurrences of |with| result in sequential
bindings (as with 
|let*|).  There is no way to obtain parallel bindings; see
section \ref{bindings} for a rationale.  


\clause{|for| ~var~ |=| ~expr~}
\clindexx{for}{=}
On each iteration, ~expr~ is evaluated and ~var~ is set
to its value. 

\cpar This clause may appear to do the same thing as |for\dots next|.
In fact, they are quite different.  |for\dots =| provides only three
services: it sets up a binding for ~var~, sets it to ~expr~ on each
iteration, and makes it possible to use |for\dots previous| with
~var~.  |for\dots next| provides these services in addition to the
ability to turn the driver into a generator.
%Also, the code which sets ~var~ appears in the loop body in the same
%place as the |for\dots =| clause; the code for |for\dots next| appears
%at the top of the loop, as with other drivers (except when being used
%as a generator).

\clause{|for| ~var~ |initially| ~init-expr~ |then| ~then-expr~}
\clindexxx{for}{initially}{then}
Before the loop begins, ~var~ is set to ~init-expr;~ on all
iterations after the first it is set to ~then-expr.~  This clause
must occur at top-level.  ~init-expr~ will be moved outside the loop
body and ~then-expr~ will be moved to the end of the loop body, so
they are subject to code motion problems (see section
\ref{code-movement}). 

\cpar This clause may appear to be similar to |for\dots next|, but in
fact they differ significantly.  |for\dots initially\dots then| is
typically used to give ~var~ its first value before the loop begins,
and subsequent values on following iterations.  This is incompatible
with generators, whose first value and subsequent values must all be
computed by |(next ~var~)|.  Also, the update of ~var~ in 
|for\dots initially\dots then| does not occur at the location of the
clause. 
Use |for\dots initially\dots then| for
one-shot computations where its idiom is more convenient, but use
|for\dots next| for extending \iter\ with new drivers (see section
\ref{extend}).

\clause{|for| ~var~ |first| ~first-expr~ |then| ~then-expr~}
\clindexxx{for}{first}{then}
The first time through the loop, ~var~ is set to ~first-expr;~ on
subsequent iterations, it is set to ~then-expr.~  This differs from
|for\dots initially| in that ~var~ is set to ~first-expr~
inside the loop body, 
so ~first-expr~ may depend on the results of other clauses.  For
instance, 
\begin{program}
(iter (for num in list)
      (for i first num then (1+ i))
      ...)
\end{program}
will set |i| to the first element of |list| on the first
iteration, whereas 
\begin{program}
(iter (for num in list)
      (for i initially num then (1+ i))
      ...)
\end{program}
is probably erroneous; |i| will be bound to |num|'s
default binding (usually \nil) for the first iteration.

\end{clauses}

\begin{note}{Compatibility note}
|loop|'s |for\dots =| works like \iter's, but |loop| used the syntax
|for\dots =\dots then| to mean |for\dots initially\dots then|.  It was
felt that these two operations were sufficiently different to warrant
different keywords.

Also, the |for| in the above three clauses is misleading,
since none is true driver (e.g. none has a corresponding |generate|
form).  |setting| would have been a better choice, but |for| was used
to retain some compatibility with |loop|.  
\end{note}


%%%%%%%%%%%%%%%%%%%%%%%%%%%%%%%%
\subsection{Gathering Clauses}
Many of \iter's clauses accumulate values into a variable, or set a
variable under certain conditions.  At the end of the loop, this
variable contains the desired result.  All these clauses have an
optional |into| keyword, whose argument should be a symbol.  If the
|into| keyword is not supplied, the accumulation variable will be
internally generated and its value will be returned at the end of the
loop; if a variable is specified, that variable is used for the
accumulation, and is not returned as a result---it is up to the user
to return it explicitly, in the loop's epilogue code (see |finally|).
It is safe to examine the accumulation variable during the loop, but
it should not be modified.

These clauses all begin with a verb.  When the verb does
not conflict with an existing Common Lisp function, then it may be
used in either its infinitival or present-participle form (e.g. |sum|,
|summing|).  However, when there is a conflict with Common Lisp, only
the present-participle form may be used (e.g. |unioning|).  This is to
prevent \iter\ clauses from clashing with Common Lisp
functions. 

%although these clauses are described as ``producing a value,'' it is a
%mistake to think of the lisp list representing the clause as a
%value-producing form in the usual way.  clauses may legally be written
%where a value is expected, e.g. |(setq x (sum i))|, but the
%lisp value of a clause in such a context is undefined.

\subsubsection{Reductions}

~Reduction~ is an extremely common iteration pattern in which
the results of successive applications of a
binary operation are accumulated.
For example, a loop that computes the sum of the
elements of a list is performing a reduction with the addition
operation.  This could be written in Common Lisp as  |(reduce \#'+
list)| or with \iter\ as
\begin{program}
(iter (for el in list)
      (sum el))
\end{program}

\begin{clauses}

\clausex{sum}{~expr~ \opt\ |into| ~var~}
Each time through the loop, ~expr~ is evaluated and added to a
variable, which is bound initially to zero.  If ~expr~ has a type,
it is ~not~ used as the type of the sum variable, which is always
|number|.  To get the result variable to be of a more specific
type, use an explicit variable, as in
\begin{program}
(iter (for el in number-list)
      (sum el into x)
      (declare (fixnum x))
      (finally (return x)))
\end{program}

\clausex{multiply}{~expr~ \opt\ |into| ~var~}
Like |sum|, but the initial value of the result variable is $1$,
and the variable is updated by multiplying ~expr~ into it.

\clausex{counting}{~expr~ \opt\ |into| ~var~}
~expr~ is evaluated on each iteration.  If it is \nonnil, the
accumulation variable, initially zero, is incremented.

\startitem
\Clausex{maximize}{~expr~ \opt\ |into| ~var~}
\Clausex{minimize}{~expr~ \opt\ |into| ~var~}
\finishitem
~expr~ is evaluated on each iteration and its extremum (maximum or
minimum) is stored in the accumulation variable.  If ~expr~ is never
evaluated, then the result is \nil\ (if the accumulation variable is
untyped) or $0$ (if it has a numeric type).

\clausex{reducing}{~expr~ |by| ~func~ \opt\ |initial-value| ~init-val~ |into| ~var~}
This is a general way to perform reductions. ~func~ should be a
function of two arguments, the first of which will be the value
computed so far and
the second of which will be the value of ~expr~.  It should return the
new value.
|reducing| is roughly equivalent to the
Common Lisp |(reduce ~func~ ~sequence~ :key ~expr-function~)|, where
~expr-function~ is used to derive values from the successive elements
of ~sequence~.

\cpar If the |reducing| clause is never executed, the result is
undefined.

\cpar It is not necessary to provide an
initial value, but better code can be generated if one is supplied.
Regardless of its location in the \iter\ body, ~init-val~ will be
evaluated before the loop is entered, so it should not depend on any
value computed inside the \iter\ form.  

%\cpar if a ~var~ is not specified, you can get \iter\ to declare the
%type of the internal variable by putting a |the| expression around
%~func.~  see section \ref{types}.

\end{clauses}

%\begin{note}{implementation note}
%in principle, |maximize| and |minimize| can be thought of
%as reductions where the initial value is the smallest (or largest)
%value that the accumulation variable can assume.  because lisp's
%bignums can represent arbitrary integers, these clauses cannot be
%implemented as reductions in general.  if, however, the type of 
%~expr~ or ~var~ can be determined to be a fixnum
%or a float, \iter\ will implement the clause as a true reduction,
%using one of the constants |most-negative-fixnum|, |%most-positive-fixnum|, 
%|most-negative-short-float|, etc. as appropriate.
%\end{note}

\subsubsection{Accumulations}

All the predefined accumulation clauses add values to a sequence.  If
the sequence is a list, they all 
have the property that the partial list is kept in the correct order
and available for inspection at any point in the loop.

\begin{clauses}

\clausex{collect}{~expr~ \opt\ |into| ~var~ |at| ~place~ |result-type| ~type~}
Produces a sequence of the values of ~expr~ on each iteration. ~place~
indicates where the next value of ~expr~ is added to the list and may
be one of the symbols |start|, |beginning| (a synonym for |start|) or
|end|.  The symbol may be quoted, but need not be.  The default is
|end|.  For example,
\begin{program}
(iter (for i from 1 to 5)
      (collect i))
\end{program}
produces |(1 2 3 4 5)|, whereas
\begin{program}
(iter (for i from 1 to 5)
      (collect i at beginning))
\end{program}
produces |(5 4 3 2 1)| (and is likely to be faster in most Common Lisp
implementations).

\cpar If ~type~ is provided, it should be a subtype of |sequence|.
The default is |list|.  Specifying a type other than |list| will
result in |collect| returning a sequence of that type.  ~However,~ the
type of the sequence being constructed when inside the loop body is
undefined when a non-|list| type is specified.  (As with ~place~,
quoting ~type~ is optional.)

\clausex{adjoining}{~expr~ \opt\ |into| ~var~ |test| ~test~ |at| ~place~ |result-type| ~type~}
Like |collect|, but only adds the value of ~expr~ if it is not
already present.  ~test,~ which defaults to |\#'eql|, is
the test to be used with |member|.

\startitem
\Clausex{appending}{~expr~ \opt\ |into| ~var~ |at| ~place~}
\Clausex{nconcing}{\ ~expr~ \opt\ |into| ~var~ |at| ~place~}
\Clausex{unioning}{\ ~expr~ \opt\ |into| ~var~ |test| ~test~ |at| ~place~}
\Clausex{nunioning}{~expr~ \opt\ |into| ~var~ |test| ~test~ |at| ~place~}
\finishitem
These are like |collect|, but behave like the Common Lisp functions
|append|, |nconc|, |union| or |nunion|.  
As in Common Lisp, they work only on lists.  Also as in Common Lisp,
|unioning| and |nunioning| assume that the value of ~expr~ contains no
duplicates. 

\clausex{accumulate}{~expr~ |by| ~func~ \opt\ |initial-value| ~init-val~ |into| ~var~}
This is a general-purpose accumulation clause. ~func~ should be a
function of two arguments, the value of ~expr~ and the value
accumulated so far in the iteration, and it should return the updated
value.  If no initial value is supplied, \nil\ is used.  

%\cpar If a ~var~ is not specified, you can get \iter\ to declare the
%type of the internal variable by putting a |the| expression around
%~func.~  see section \ref{types}.

\cpar The differences between |accumulate| and |reducing| are slight.
One difference is that the functions take their arguments in a
different order.  Another is that in the absence of ~init-val~,
|accumulate| will use \nil, whereas |reducing| will generate different
code that avoids any dependence on the initial value.
The reason for having both clauses is that one usually
thinks of reductions (like |sum|) and accumulations (like |collect|)
as different beasts.

\end{clauses}


\subsubsection{Finders}

A ~finder~ is a clause whose value is an expression that meets some
condition.

\begin{clauses}

\clause{|finding| ~expr~ |such-that| ~test~ \opt\ |into| ~var~ |on-failure| ~failure-value~}
\clindexx{finding}{such-that}
If ~test~ (which is an expression) ever evaluates to \nonnil, the loop
is terminated, the 
epilogue code is run and the value of ~expr~ is returned.  Otherwise,
\nil\ (or ~failure-value,~ if provided) is returned.  If ~var~ is
provided, it will have either the \nonnil\ value of ~expr~ or
~failure-value~ when the epilogue code is run.

\cpar As a special case, if the ~test~ expression is a sharp-quoted
function, then it is applied to ~expr~ instead of being simply
evaluated.  E.g. |(finding x such-that \#'evenp)| is equivalent to
|(finding x such-that (evenp x))|.  

%\cpar although ~test~ need have nothing to do with ~%expr~ as in
%|(finding j such-that (> i 3))|, it usually 
%will: |(finding (length el) such-that (oddp (length el)))|.  to
%avoid performing the |length| computation twice, you could write
%|(finding (length el) such-that \#'oddp)| or |(finding (length
%el) such-that 'oddp)|; for these cases, \iter\ generates code that
%executes ~expr~ only once.  the code for |\#'oddp|
%is slightly different from that for {\lisp 'oddp}; see the discussion
%under {\lisp for\dots in} and {\lisp for\dots on}.

\cpar |On-failure| is a misnomer. Because it is always evaluated, it behaves
more like the default third argument to the |gethash| function. As a result,
|on-failure (error "Not found")| makes no sense. Instead, the clauses |leave|
or |thereis| can be used in conjunction with |finally| as follows:
\begin{program}
(iter (for x in '(1 2 3))
      (if (evenp x) (leave x))
      (finally (error "not found")))
\end{program}

\cpar This clause may appear multiple times when all defaults are
identical. It can also be used together with either |always|/|never| or
|thereis| if their defaults match. More specifically, |on-failure nil| is
compatible with |thereis|, while |on-failure t| is compatible with |always|
and |never| clauses.
\begin{program}
(iter (for i in '(7 -4 2 -3))
      (if (plusp i)
	  (finding    i  such-that (evenp i))
          (finding (- i) such-that (oddp i))))
\end{program}

\startitem
\Clause{|finding| ~expr~ |maximizing| ~m-expr~ \opt\ |into| ~var~}
\clindexx{finding}{maximizing}
\Clause{|finding| ~expr~ |minimizing| ~m-expr~ \opt\ |into| ~var~}
\clindexx{finding}{minimizing}
\finishitem
Computes the extremum (maximum or minimum) value of ~m-expr~ over all
iterations, and returns the value of ~expr~ corresponding to the
extremum.  ~expr~ is evaluated inside the loop at the time the new
extremum is established.  If ~m-expr~ is never evaluated (due to, for
example, being embedded in a conditional clause), then the returned
value depends on the type, if any, of ~expr~ (or ~var,~ if
one is supplied).  If there is no type, the returned
value will be \nil; if the type is numeric, the returned value will be
zero.

\cpar For these two clauses, ~var~ may be a list of two
symbols; in that case, the first is used to record ~expr~ and
the second, ~m-expr.~  

\cpar As with |finding\dots such-that|, if ~m-expr~ is a sharp-quoted
function, then it is called on ~expr~ instead of being evaluated.

\end{clauses}

\subsubsection{Boolean Tests}

\begin{clauses}

\clausex{first-iteration-p}{}
Returns |t| in the first cycle of the loop, otherwise \nil.

\clausex{first-time-p}{}
Returns |t| the first time the expression is evaluated, and then \nil\ forever.
This clause comes handy when printing (optional) elements separated
by a comma:

\begin{program}
(iter (for el in '(nil 1 2 nil 3))
      (when el
        (unless (first-time-p)
          (princ ", "))
        (princ el)))
\end{program}
produces |"1, 2, 3"|.

\end{clauses}

\subsubsection{Aggregated Boolean Tests}

\begin{clauses}

\clausex{always}{~expr~}
If ~expr~ ever evaluates to
\nil, then \nil\ is immediately returned; the epilogue code is not
executed.  If ~expr~ never evaluates to \nil, the epilogue code
is executed and the last value of ~expr~ (or |t| if ~expr~ was never
evaluated) is returned (whereas |loop| would constantly return |t|).

% mention last evaluated clause when multiple always clauses?

\clausex{never}{~expr~}
Like |(always (not ~expr~))|, except it does not influence the last
value returned by a possible other |always| clause. That is,
\begin{program}
(iter (repeat 2)
      (always 2)
      (never nil)) \yields 2 ; not t
\end{program}

\clausex{thereis}{~expr~}
If ~expr~ is ever \nonnil,
its value is immediately returned without running epilogue code.
Otherwise, the epilogue code is performed and \nil\ is returned.

This clause cannot be used together with |always| or |never|, because their
defaults are opposed (similarly, |(loop always 3 thereis nil)| refuses to
compile in some implementations of |loop|).

\end{clauses}


\subsection{Control Flow}
\label{control-flow}
Several clauses can be used to alter the usual flow of control in a loop.

Note: the clauses of this and subsequent sections don't adhere to \iter's
usual syntax, but instead use standard Common Lisp syntax.  Hence the
format for describing syntax subsequently is
like the standard format used in the Common Lisp manual, not like the
descriptions of clauses above.

\begin{clauses}

\clausex{finish}{}
Stops the loop and runs the epilogue code.

%for example:
%
%\begin{program}
%(iter (with answer = nil)
%         (initially (make-a-mess))
%         (for i from 1 to 10)
%         (when (correct? i)
%           (setq answer i)
%           (finish))
%         (finally (cleanup)))
%\end{program}
%
%this code will execute |cleanup| whether or not the test |(correct?
%i)| ever succeeds. 
%the (more elegant) formulation,
%\begin{program}
%(iter (initially (make-a-mess))
%         (for i from 1 to 10)
%         (finding i such-that (correct? i))
%         (finally (cleanup)))
%\end{program}
%would not execute |cleanup| if |(correct? i)| succeeded; it
%would do an immediate return.

\clausex{leave}{\opt\ ~value~}
Immediately returns ~value~ (default \nil) from the current \iter\
form, skipping the epilogue code.  Equivalent to using |return-from|.

\clausex{next-iteration}{}
Skips the remainder of the loop body and begins the next iteration of
the loop.

\clausex{while}{~expr~}
If ~expr~ ever evaluates to \nil, the loop is terminated and the
epilogue code executed.  Equivalent to |(if (not ~expr~) (finish))|.

\clausex{until}{~expr~}
Equivalent to |(if ~expr~ (finish))|.

\clausex{if-first-time}{~then~ \opt\ ~else~}
If this clause is being executed for the first time in this invocation
of the \iter\ form, then the ~then~ code is evaluated; otherwise the
~else~ code is evaluated.

\cpar |(for ~var~ first ~expr1~ then ~expr2~)| is almost equivalent to
\begin{program}
(if-first-time (dsetq ~var~ ~expr1~)
               (dsetq ~var~ ~expr2~))
\end{program}
The only difference is that the |for| version makes ~var~ available
for use with |for\dots previous|.

\end{clauses}

\subsection{Code Placement}
When fine control is desired over where code appears in a loop
generated by \iter, the following special clauses may be useful.
They are all subject to code-motion problems (see section
\ref{code-movement}).  

\begin{clauses}

\clausex{initially}{|\&rest| ~forms~}
The lisp ~forms~ are placed in the prologue section of the loop, where
they are executed once, before the loop body is entered.

\clausex{after-each}{|\&rest| ~forms~}
The ~forms~ are placed at the end of the loop body, where they
are executed after each iteration.  Unlike the other clauses in this
section, ~forms~ may contain \iter\ clauses.

\clausex{else}{|\&rest| ~forms~}
The lisp ~forms~ are placed in the epilogue section of the loop, where they
are executed if this |else| clause is never met during execution of the
loop and the loop terminates normally.

\clausex{finally}{|\&rest| ~forms~}
The lisp ~forms~ are placed in the epilogue section of the loop, where
they are executed after the loop has terminated normally.

\clausex{finally-protected}{|\&rest| ~forms~}
The lisp ~forms~ are placed in the second form of an unwind-protect
outside the loop.  They are always executed after the loop has
terminated, regardless of how the termination occurred.

\end{clauses}


%%%%%%%%%%%%%%%%%%%%%%%%%%%%%%%%%%%%%%%%%%%%%%%%%%%%%%%%%%%%%%%%
\section{Other Features}

\subsection{Multiple Accumulations}
\label{multiple}

\begin{sloppypar}
It is permitted to have more than one clause accumulate into the same
variable, as in the following:
\begin{program}
(iter (for i from 1 to 10)
      (collect i into nums)
      (collect (sqrt i) into nums)
      (finally (return nums)))
\end{program}
Clauses can only accumulate into the same variable if they are
compatible.  |collect|, |adjoining|, |appending|, |nconcing|,
|unioning| and |nunioning| are compatible with each other; |sum|,
|multiply| and |counting| are compatible; |always| and |never| are compatible;
|finding| \dots |such-that| is compatible with either |thereis| or |always|
and |never| when their defaults match; and |maximize| and |minimize| clauses
are compatible only with other |maximize| and |minimize| clauses,
respectively.

%note that the same variable ~cannot~ be both an accumulation
%variable and an ordinary variable; there can be only one variable with
%a given name within an \iter\ form. 
\end{sloppypar}

\subsection{Named Blocks}

Like Common Lisp |block|s, \iter\ forms can be given names.  The
name should be a single symbol, and it must be the first form in the
\iter.  The generated code behaves exactly like a named block; in
particular, |(return-from ~name~)| can be used to exit it:
\begin{program}
(iter fred
      (for i from 1 to 10)
      (iter barney
            (for j from i to 10)
            (if (> (* i j) 17)
                (return-from fred j))))
\end{program}
An \iter\ form that is not given a name is implicitly named \nil.

Sometimes one would like to write an expression in an inner \iter\ form,
but have it processed by an outer \iter\ form.  This is possible with
the |in| clause.  

\begin{clauses}

\clausex{in}{~name~ |\&rest| ~forms~}
Evaluates ~forms~ as if they were part of the \iter\ form named
~name~.  In other words, \iter\ clauses are processed by the \iter\
form named ~name,~ and not by any \iter\ forms that occur inside ~name.~

\cpar As an example, consider the problem of collecting a list of the
elements in a two-dimensional array.  The naive solution,
\begin{program}
(iter (for i below (array-dimension ar 0))
      (iter (for j below (array-dimension ar 1))
            (collect (aref ar i j))))
\end{program}
\noindent is wrong because the list created by the inner \iter\ is simply
ignored by the outer one.  But using |in| we can write:
\begin{program}
(iter outer (for i below (array-dimension ar 0))
      (iter (for j below (array-dimension ar 1))
            (in outer (collect (aref ar i j)))))
\end{program}
\noindent which has the desired result.

\end{clauses}

\subsection{Destructuring}
\label{destructuring}

In many places within \iter\ clauses where a variable is expected, a
list can be written instead.  In these cases, the value to be assigned
is ~destructured~ according to the pattern
described by the list.  As a simple example, the clause
\begin{program}
(for (key . item) in alist)
\end{program}
\noindent will result in |key| being set to the |car| of
each element in |alist|, and |item| being set to the |cdr|.  The
pattern list may be nested to arbitrary depth, and (as the example
shows) need not be terminated with \nil; the only requirement is that
each leaf be a bindable symbol (or \nil, in which case no binding is
generated for that piece of the structure).

Sometimes, you might like to do the equivalent of a
|multiple-value-setq| in a clause.  This 
``multiple-value destructuring'' can be expressed by writing 
\linebreak |(values $pat_1$ $pat_2 \ldots$)| for a destructuring
pattern, as in 
\begin{program}
(for (values (a . b) c d) = (three-valued-function ...))
\end{program}
\begin{sloppypar}
Note that the $pat_i$ can themselves be destructuring patterns (though
not multiple-value destructuring patterns).  You can't do multiple-value
destructuring in a |with| clause; instead wrap the whole \iter\
form in a |multiple-value-bind|.
\end{sloppypar}

\begin{note}{Rationale}
There are subtle interactions between variable declarations and
evaluation order that make the correct implementation of
multiple-value destructuring in a |with| somewhat tricky.
\end{note}

The destructuring feature of \iter\ is available as a separate
mechanism, using the |dsetq| macro:

\begin{clauses}
\defmacro{dsetq}{template expr}
Performs destructuring of ~expr~ using ~template~.   May be used
outside of an \iter\ form. Yields the primary value of ~expr~.

\end{clauses}

\subsection{On-line Help}

\begin{sloppypar}
There is a limited facility for on-line help, in the form of the 
|display-iterate-clauses| function.
\end{sloppypar}

\begin{clauses}

\defun{display-iterate-clauses}{\opt\ clause-spec}
Displays a list of \iter\ clauses.  If ~clause-spec~ is not
provided, all clauses are shown; if it is a symbol, all clauses
beginning with that symbol are shown; and if it is a list of symbols,
all clauses for which ~clause-spec~ is a prefix are shown.

\end{clauses}

\subsection{Parallel Binding and Stepping}
\label{bindings}

The parallel binding and stepping of variables is a feature that
\iter\ does ~not~ have.  This section attempts to provide a rationale.

We say that two variables are bound ~in parallel~ if neither
binding shadows the other.  This is the usual semantics of |let|
(as opposed to |let*|).  Similarly, we can say that iteration
variables are stepped in parallel if neither variable is updated
before the other, conceptually speaking; in other words, if the code
to update each variable can reference the old values of both variables.

|loop| allows parallel binding of variables and parallel stepping of
driver variables.  My view is that if you are depending on the 
serial/parallel distinction, you are doing something obscure.  If you
need to bind 
variables in parallel using |with|, then you must be using a
variable name that shadows a name in the existing lexical environment.
Don't do that.  The most common use for parallel stepping is to track
the values of variables on the previous iteration, but in fact this
does not require parallel stepping at all; the following will work:
\begin{program}
(iter (for current in list)
      (for prev previous current)
      ...)
\end{program}


%%%%%%%%%%%%%%%%%%%%%%%%%%%%%%%%%%%%%%%%%%%%%%%%%%%%%%%%%%%%%%%%
\section{Types and Declarations}
\label{types}

\subsection{Discussion}

Sometimes efficiency dictates that the types of variables be declared.
This type information needs to be communicated to \iter\ so it can
bind variables to appropriate values.  Furthermore, \iter\ must often
generate internal variables invisible to the user; there needs to be a
way for these to be declared.

As an example, consider this code, which will return the number of
odd elements in |number-list|:
\begin{program}
(iter (for el in number-list)
      (count (oddp el)))
\end{program}
In processing this form,
\iter\ will create an internal variable, let us call it |list17|, to
hold the successive |cdr|s of |number-list|, and  
will bind the variable to |number-list|.  It will also generate a
default binding for |el|; only inside the body of the loop will |el|
be set to the |car| of |list17|.  Finally, \iter\ will generate a
variable, call it |result|, to hold the result of the count, and will
bind it to zero.

When dealing with type declarations, \iter\ observes one simple rule:
~it will never generate a declaration unless requested to do so.~  The
reason is that such declarations might mask errors in compiled code by
avoiding error-checks; the resulting problems would be doubly hard to
track down because the declarations would be hidden from the
programmer.  Of course, a compiler might omit error-checks even in the
absence of declarations, though this behavior can usually be avoided,
e.g. by saying |(declaim (optimize (safety 3)))|.

So, the above \iter\ form will generate code with no declarations.
But say we wish to declare the types of |el| and the internal
variables |list17| and |result|.  How is this done?

Declaring the type of |el| is easy, since the programmer knows
the variable's name:
\begin{program}
(iter (for el in number-list)
      (declare (fixnum el))
      (counting (oddp el)))
\end{program}
\iter\ can read variable type declarations like this one.  Before
processing any clauses, it scans the entire top-level form for type
declarations 
and records the types, so that variable bindings can be performed
correctly.  In this case, |el| will be bound to zero
instead of \nil.  Also, \iter\ collects all the top-level declarations
and puts them at the begining of the generated code, so it is not
necessary to place all declarations at the beginning of an \iter\
form; instead, they can be written near the variables whose types they
declare.  

Since \iter\ is not part of the compiler, it will not know
about declarations that occur outside an \iter\ form; these
declarations must be repeated inside the form.

Here is another way we could have declared the type of |el|:
\begin{program}
(iter (for (the fixnum el) in number-list)
      (counting (oddp el)))
\end{program}
\iter\ extends the Common Lisp |the|\lispindex{the}
form to apply to variables as well as value-producing forms; anywhere
a variable is allowed---in a |with| clause, as the iteration
variable in a driver clause, as the |into| argument of an
accumulation clause, even inside a destructuring template---you can
write |(the ~type~ ~symbol~)| instead.

There is one crucial difference between using a |the| form and
actually declaring the variable: explicit declarations are always
placed in the generated code, but type information from a |the|
form is not turned into an actual declaration unless you tell \iter\
to do so using |iterate:declare-variables|.  See below.

\begin{sloppypar}
Declaring the types of internal variables is harder than declaring the
types of explicitly mentioned variables, since their names
are unknown.  You do it by declaring |iterate:declare-variables|
somewhere inside the top level of the \iter\ form.  (This will also
generate declarations for variables declared using |the|.)
\iter\ does not provide much selectivity here: it's all or none. 
And unfortunately, since \iter\ is not privy to compiler information
but instead reads declarations itself, it will not hear if you
|(declaim (iterate:declare-variables))|.  Instead, set the variable 
|iterate::*always-declare-variables*| to |t| at
compile-time, using |eval-when|.
\end{sloppypar}

To determine the appropriate types for internal variables, \iter\ uses
three sources of information:

\begin{itemize}

\item Often, the particular clause dictates a certain type for a
variable; \iter\ will use this information when available.  In the
current example, the variable |list17| will be given the type
|list|, since that is the only type that makes sense; and the
variable |result| will be given the type |fixnum|, on the
assumption that you will not be counting high enough to need bignums.
You can override this assumption only by using and explicitly declaring a
variable: 
\begin{verbatim}
(iter (declare (iterate:declare-variables))
      (for el in number-list)
      (count (oddp el) into my-result)
      (declare (integer my-result))
      (finally (return my-result)))
\end{verbatim}

\begin{sloppypar}
Other examples of the type assumptions that \iter\ makes are: type
|list| for |into| variables of collection clauses; type |list| for
expressions that are to be destructured; type |vector| for the
variable holding the vector in a |for\dots in-vector| clause, and
similarly for |string| and the |for\dots in-string| clause;
and the implementation-dependent type
|(type-of array-dimension-limit)| for the index and limit
variables generated by sequence iteration drivers like |for\dots
in-vector| and |for\dots in-string| (but not |for\dots in-sequence|,
because it may be used to iterate over a list). 
\end{sloppypar}

\item Sometimes, \iter\ will examine expressions and try to determine
their types in a simple-minded way.  If the expression is
self-evaluating (like a number, for instance), \iter\ knows that the
expression's type is the same as the type of the value it denotes, so
it can use that type.  If the expression is of the form |(the ~type~
~expr~)|, \iter\ is smart enough to extract ~type~ and use it.
However, the current version of \iter\ does not  
examine declarations of function result types or do any type
inference.  It will not determine, for 
example, that the type of |(+ 3 4)| is |fixnum|, or even
|number|.

\item In some cases, the type of an internal variable should match the
type of some other variable.  For instance, \iter\ generates an
internal variable for |(f x)| in the
clause |(for i from 1 to (f x))|, and in the absence of other
information will give it the same type as |i|.  If, however, the
expression had been written |(the fixnum (f x))|, then \iter\
would have given the internal variable the type |fixnum|
regardless of |i|'s type.  The type incompatibility errors that
could arise in this situation are not checked for.

\end{itemize}

Note that if you do declare |iterate:declare-variables|, then
\iter\ may declare user variables as well as internal ones if they do
not already have declarations, though only for variables that it
binds.  For instance, in this code:

\begin{program}
(iter (declare (iterate:declare-variables))
      (for i from 1 to 10)
      (collect i into var))
\end{program}
the variable |var| will be declared to be of type |list|.


\subsection{Summary}

\iter\ understands standard Common Lisp variable type declarations
that occur within an \iter\ form and
will pass them through to the generated code.  If the declaration
|(iterate:declare-variables)|\lispindex{declare-variables}
appears at the top level of an 
\iter\ form, or if 
|iterate::*always-declare-variables*|\lispindex{*always-declare-variables*} 
is \nonnil, then \iter\ will use the type information gleaned from user
declarations, self-evaluating expressions and |the| expressions,
combined with reasonable assumptions, to determine variable
types and declare them.


%%%%%%%%%%%%%%%%%%%%%%%%%%%%%%%%%%%%%%%%%%%%%%%%%%%%%%%%%%%%%%%%
\section{Problems with Code Movement}
\label{code-movement}

Some \iter\ clauses, or parts of clauses, result in code being
moved from the location of the clause to other parts of the loop.
Drivers behave this way, as do code-placement clauses like |initially|
and |finally|.  When using these clauses, there is a danger of writing
an expression that makes sense in its apparent location but will be
invalid or have a different meaning in another location.  For example:
\begin{program}
(iter (for i from 1 to 10)
      (let ((x 3))
        (initially (setq x 4))))
\end{program}
While it may appear that the |x| of |(initially (setq x 4))| is the
same as the |x| of |(let ((x 3)) \dots|, in fact they are not:
|initially| moves its code outside the loop body, so |x| would refer
to a global variable.  Here is another example of the same problem:
\begin{program}
(iter (for i from 1 to 10)
      (let ((x 3))
        (collect i into x)))
\end{program}
If this code were executed, |collect| would create a binding for its
|x| at the top level of the \iter\ form that the |let| will shadow.

Happily, \iter\ is smart enough to catch these errors; it
walks all problematical code to ensure that free variables are not
bound inside the loop body, and checks all variables it binds for the
same problem.

However, some errors cannot be caught:

\begin{program}
(iter (with x = 3)
      (for el in list)
      (setq x 1)
      (reducing el by \#'+ initial-value x))
\end{program}
|reducing| moves its |initial-value| argument to the initialization
part of the loop in order to produce more efficient code.  Since
\iter\ does not perform data-flow analysis, it cannot determine that
|x| is changed inside the loop; all it can establish is that |x| is
not bound internally.  Hence this code will not signal an
error and will use $3$ as the initial value of the reduction.

The following list summarizes all cases that are subject to these code
motion and variable-shadowing problems.
\begin{itemize}
\item Any variable for which \iter\ creates a binding, including those
used in |with| and the |into| keyword of many clauses.

\begin{sloppypar}
\item The special clauses which place code: |initially|, |after-each|, |else|,
|finally| and |finally-protected|. 
\end{sloppypar}

\item The variables of a |next| or |do-next| form.

\item The |initially| arguments of |for\dots initially\dots then| and
|for\dots previous|. 

\item The |then| argument of |for\dots initially\dots then|.

\item The |initial-value| arguments of |reducing| and |accumulate|.

\item The |on-failure| argument of |finding\dots such-that|.
\end{itemize}

%%%%%%%%%%%%%%%%%%%%%%%%%%%%%%%%%%%%%%%%%%%%%%%%%%%%%%%%%%%%%%%%
\section{Differences Between {\tt Iterate} and {\tt Loop}}
|loop| contains a great deal of complexity which \iter\ tries to
avoid.  Hence many esoteric features of |loop| don't exist in \iter.
Other features have been carried over, but in a cleaned-up form.  
And of course, many new features have been added; they are not
mentioned in this list.

\begin{itemize}

\item \iter's syntax is more Lisp-like than |loop|'s, having a higher
density of parens.

\item The current implementation of \iter, unlike the current version
of  |loop| (as documented in {\em Common Lisp, 2nd Ed.\/}), is
extensible (see section \ref{extend}).  

\item |loop| puts the updates of all driver variables at the top of
the loop; \iter\ leaves them where the driver clauses appear.

\item While for the most part \iter\ clauses that resemble |loop| clauses
behave similarly, there are some differences.  For instance, there is
no |for\dots =\dots then| in \iter; instead use 
|for\dots initially\dots then|.  

\item |loop| binds the variable |it| at certain times
to allow pseudo-English expressions like |when ~expr~ return it|. 
In \iter, you must bind ~expr~ to a variable yourself.  Note that
|when ~expr~ return it| is like |thereis ~expr~| except that the latter is an
accumulation clause and therefore competes with other accumulations
(remember section \ref{multiple} above).

% repeat different behaviour of |always| clause here?

\item |loop| has a special |return| clause, illustrated in the
previous item.  \iter\ doesn't need one,  since an ordinary Lisp
|return| has the same effect.

\item |loop| allows for parallel binding and stepping of iteration
variables.  \iter\ does not.  (See section \ref{bindings}.)

\item |loop| and \iter\ handle variable type declarations very
differently.  |loop| provides a special syntax for declaring variable
types, and does not examine declarations.  Moreover, the standard
implementation of |loop| will
generate declarations when none are requested.
\iter\ parses standard Common Lisp type declarations, and will never
declare a variable itself unless declarations are 
specifically requested.

\end{itemize}

%%%%%%%%%%%%%%%%%%%%%%%%%%%%%%%%%%%%%%%%%%%%%%%%%%%%%%%%%%%%%%%%
\section{Rolling Your Own}
\label{extend}

\subsection{Introduction}

\iter\ is extensible---you can write new clauses that embody new
iteration patterns.  You might want to write a new driver clause for a
data structure of your own, or you might want to write a clause that
collects or manipulates elements in a way not provided by \iter. 

This section describes how to write clauses for \iter.  Writing a
clause is like writing a macro.  In fact, writing a clause ~is~
writing a macro: since \iter\ code-walks its body and macroexpands,
you can add new abstractions to \iter\ with good old |defmacro|.

Actually, there are two extensions you can make to \iter\ that are
even easier than writing a macro.  They are adding a synonym for an
existing clause and defining a driver clause for an indexable
sequence.  These can be done with |defsynonym| and
|defclause-sequence|, respectively.  See section \ref{aids}, below.

The rest of this section explains how to write macros that expand into
\iter\ clauses.
Here's how you could add a simplified version of \iter's |multiply|
clause, if \iter\ didn't already have one:
\begin{program}
(defmacro multiply (expr)
  `(reducing ,expr by \#'* initial-value 1))
\end{program}

If you found yourself summing the square of an expression often, you
might want to write a macro for that.  A first cut might be
\begin{program}
(defmacro sum-of-squares (expr)
  `(sum (* ,expr ,expr)))
\end{program}
but if you are an experienced macro writer, you will realize that this
code will evaluate ~expr~ twice, which is probably a bad idea.  A
better version would use a temporary:
\begin{program}
(defmacro sum-of-squares (expr)
  (let ((temp (gensym)))
    `(let ((,temp ,expr))
       (sum (* ,temp ,temp)))))
\end{program}
Although this may seem complex, it is just the sort of thing you'd
have to go through to write any macro, which illustrates the point of
this section: if you can write macros, you can extend \iter.

Our macros don't use \iter's keyword-argument
syntax.  We could just use keywords with |defmacro|, but we would
still not be using \iter's clause indexing mechanism.  Unlike Lisp,
which uses just the first symbol of a form to determine what function
to call, \iter\ individuates clauses by the list of required keywords.
For instance, |for\dots in| and |for\dots in-vector| are different clauses
implemented by distinct Lisp functions.  

To buy into this indexing scheme, as well as the keyword-argument
syntax, use |defmacro-clause|:

\begin{clauses}

\defmacro{defmacro-clause}{arglist |\&body| body}
Defines a new \iter\ clause.  ~arglist~ is a list of symbols which are
alternating keywords and arguments.  \opt\ may be used, and the list
may be terminated by |\&sequence|.  ~body~ is an ordinary macro body,
as with |defmacro|.  If the first form of ~body~ is a string, it is
considered a documentation string and will be shown by
|display-iterate-clauses|. |defmacro-clause| will signal an error if 
defining the clause would result in an ambiguity.  E.g. you cannot
define the clause |for\dots from| because there would be no way to
distinguish it from a use of the |for| clause with optional keyword |from|.

\end{clauses}

\medskip

Here is |multiply| using |defmacro-clause|.  The keywords are capitalized
for readability.
\begin{program}
(defmacro-clause (MULTIPLY expr \opt\ INTO var)
  `(reducing ,expr by \#'* into ,var initial-value 1))
\end{program}
You don't have to worry about the case when |var| is not supplied; for
any clause with an |into| keyword, saying |into nil| is equivalent to
omitting the |into| entirely.

As another, more extended example, consider the fairly common
iteration pattern that 
involves finding the sequence element that maximizes (or minimizes) some
function.  \iter\ provides this as |finding\dots maximizing|, but it's
instructive to see how to write it.
Here, in pseudocode, is how you might write such a loop for
maximizing a function F:

\begin{quote}
\begin{tabbing}
set variable MAX-VAL to NIL; \\
set variable WINNER to NIL; \\
for\=\ each element EL in the sequence \\
\>  if\=\ MAX-VAL\=\ is NIL or F(EL) $>$ MAX-VAL then \\
\>\>    set MAX-VAL to F(EL); \\
\>\>    set WINNER to EL; \\
\>  end if; \\
end for; \\
return WINNER.
\end{tabbing}
\end{quote}

Here is the macro:
\begin{program}
(defmacro-clause (FINDING expr MAXIMIZING func \opt\ INTO var)
  (let ((max-val (gensym))
        (temp1 (gensym))
        (temp2 (gensym))
        (winner (or var iterate::*result-var*)))
    `(progn 
       (with ,max-val = nil)
       (with ,winner = nil)
       (cond 
        ((null ,max-val)
         (setq ,winner ,expr)
         (setq ,max-val (funcall ,func ,winner))
        (t
         (let* ((,temp1 ,expr)
                (,temp2 (funcall ,func ,temp1)))
           (when (> ,temp2 ,max-val)
             (setq ,max-val ,temp2)
             (setq ,winner ,temp1))))))
       (finally (leave ,winner)))))
\end{program}
Note that if no |into| variable is supplied, we use
|iterate::*result-var*|, which contains the internal variable into
which all clauses place their results.  If this variable is bound by
some clause, then \iter\ will return its value automatically;
otherwise, \nil\ will be returned.  

\subsection{Writing Drivers}

In principle, drivers can be implemented just as easily as other
\iter\ clauses.  In practice, they are a little harder to get right. 
As an example, consider writing a driver that
iterates over all the 
elements of a vector, ignoring its fill-pointer.  |for\dots in-vector|
won't work for this, because it observes the fill-pointer.  It's necessary to
use |array-dimension| instead of |length| to obtain the size of the
vector.  Here is one approach:

\begin{program}
(defmacro-clause (FOR var IN-WHOLE-VECTOR v)
  "All the elements of a vector (disregards fill-pointer)"
  (let ((vect (gensym))
        (index (gensym)))
    `(progn
       (with ,vect = ,v)
       (for ,index from 0 below (array-dimension ,vect 0))
       (for ,var = (aref ,vect ,index)))))
\end{program}
Note that we immediately put |v| in a variable, in case it is an
expression.  Again, this is just good Lisp macrology.  It also has a
subtle effect on the semantics of the driver: |v| is evaluated only
once, at the beginning of the loop, so changes to |v| in the loop have
no effect on the driver.  Similarly, the bounds for numerical iteration
e.g. the above |array-dimension| are also evaluated once only.  This is how
all of \iter's drivers work.


There is an important point concerning the |progn| in this code.  We
need the |progn|, of course, because we are returning several forms,
one of which is a driver.  But \iter\ drivers must occur at top-level.
Is this code in error?  No, because ~top-level~ is defined in \iter\
to include forms inside a |progn|.  This is just the definition of
top-level that Common Lisp uses, and for the same reason: to allow
macros to return multiple forms at top-level.

While our |for\dots in-whole-vector| clause will work, it is not
ideal.  In particular, it does not support generating.  Do do so, we
need to use |for\dots next| or |for\dots do-next|.  The job is
simplified by the |defmacro-driver| macro.

\begin{clauses}

\defmacro{defmacro-driver}{arglist |\&body| body}
Defines a driver clause in
both the |for| and |generate| forms, and provides a parameter
|generate| which ~body~ can examine to determine how it was invoked.
~arglist~ is as in |defmacro-clause|, and should begin with the symbol
|for|. 

\end{clauses}

With |defmacro-driver|, our driver looks like this:
\begin{program}
(defmacro-driver (FOR var IN-WHOLE-VECTOR v)
  "All the elements of a vector (disregards fill-pointer)"
   (let ((vect (gensym))
         (end (gensym))
         (index (gensym))
         (kwd (if generate 'generate 'for)))
     `(progn
        (with ,vect = ,v)
        (with ,end = (array-dimension ,vect 0))
        (with ,index = -1)
        (,kwd ,var next (progn (incf ,index)
                               (if (>= ,index ,end) (terminate))
                               (aref ,vect ,index))))))
\end{program}


We are still missing one thing: the |\&sequence| keywords.  
We can get them easily enough, by writing 
\begin{program}
(defmacro-driver (FOR var IN-WHOLE-VECTOR v \&sequence) 
  ...)
\end{program}
We can now refer to parameters |from|, |to|, |by|, etc. which contain
either the values for the corresponding keyword, or \nil\ if the
keyword was not supplied.  Implementing the right code for these
keywords is cumbersome but not difficult; it is left as an exercise.
But before you begin, see |defclause-sequence| below for an easier way.
       
\subsection{Extensibility Aids}
\label{aids}

This section documents assorted features that may be of use in
extending \iter.

\begin{clauses}

\defunexpvar{*result-var*}
Holds the variable that is used to return a value as a result of the
\iter\ form.  You may examine this and use it in a |with| clause, but
you should not change it.

\defmacro{defsynonym}{syn word}
 Makes ~syn~ a synonym for the existing \iter\ keyword ~word.~  Only
the first word in each clause can have synonyms. 

\defmacroxx{defclause-sequence}{element-name index-name}{|\&key|
access-fn size-fn sequence-type}{element-type
element-doc-string index-doc-string}  
Provides
a simple way to define sequence clauses.  Generates two
clauses, one for iterating over the sequence's elements, the other
for iterating over its indices.  The first symbol of both
clauses will have print-name |for|.
~element-name~  and ~index-name~ should be symbols.
~element-name~ is the second keyword of the element iterator (typically of
the form 
|in-~sequence-type~|), and ~index-name~ is the second keyword
of the index-iterator (typically of the form
|index-of-~sequence-type~|).  Either name may be 
\nil, in which case the corresponding clause is not defined.  If both
symbols are supplied, they should be in the same package.  The |for|
that begins the clauses will be in this package.

\cpar ~access-fn~ is the function to be used to
access elements of the sequence in the element iterator.  The function
should take two 
arguments, a sequence and an index, and return the appropriate element.
~size-fn~ should denote a function of one argument, a sequence, that
returns its size.  Both ~access-fn~ and ~size-fn~ are required for the
element iterator, but only ~size-fn~ is needed for the index iterator.

\cpar The ~sequence-type~ and ~element-type~ keywords are used to
suggest types for the variables 
used to hold the sequence and the
sequence elements, respectively.  The usual rules about \iter's
treatment of variable type declarations apply (see section \ref{types}).

\cpar ~element-doc-string~ and ~index-doc-string~ are
the documentation strings, for use with |display-iterate-clauses|. 

\cpar The generated element-iterator performs destructuring on the
element variable.

\cpar As an example, the above |for\dots in-whole-vector| example
could have been written:
\begin{program}
(defclause-sequence IN-WHOLE-VECTOR INDEX-OF-WHOLE-VECTOR
  :access-fn 'aref
  :size-fn \#'(lambda (v) (array-dimension v 0))
  :sequence-type 'vector
  :element-type t
  :element-doc-string 
     "Elements of a vector, disregarding fill-pointer"
  :index-doc-string 
     "Indices of vector, disregarding fill-pointer")
\end{program}

\end{clauses}

\subsection{Subtleties}

There are some subtleties to be aware of when writing \iter\ clauses.
First, the code returned by your macros may be |nconc|'ed into a list,
so you should always returned freshly consed lists, rather than
constants.  Second, \iter\ matches clauses by using |eq| on the first
symbol and |string=| on the subsequent ones, so the package of the
first symbol of a clause is relevant.  All of the clauses in this manual
have their first word in the \iter\ package.  
You can use the package system in the usual way to shadow
\iter\ clauses without replacing them.

%%% say more here, about the badness that only the first word of a
%%% clause is packagey.

\section{Non-portable Extensions to Iterate (Contribs)}
\label{contribs}

Currently, there is only one non-portable extension to iterate in the
distribution: iterate-pg. If you have made an extension that depends
on non-portable features, feel free to send them to |asf@boinkor.net|
for inclusion in the iterate distribution.

\subsection{An SQL query driver for iterate}

The pg package by Eric Marsden (see |http://cliki.net/pg|) provides an
interface to the PostgreSQL database. Using the \iterpg\ extension, it
is possible to handle the results of SQL queries using \iter.

This usage example should give you an idea of how to use it:

\begin{program}
(pg:with-pg-connection (c "somedb" "someuser")
  (iter (for (impl version date) in-relation "select * from version"
                                 on-connection *dbconn*)
        (collect version)))
\end{program}

To use the extension via |ASDF|, simply make your system depend on the
|iterate-pg| system instead of the |iterate| system. To load it
manually, use:

\begin{program}
  (asdf:oos 'asdf:load-op :iterate-pg)
\end{program}

%%%%%%%%%%%%%%%%%%%%%%%%%%%%%%%%%%%%%%%%%%%%%%%%%%%%%%%%%%%%%%%%
\section{Obtaining {\tt Iterate}}

\begin{sloppypar}
The information in this chapter is obsolete but included for
completeness's sake; Currently, the most up-to-date information on
\iter\ can be found at |http://boinkor.net/iterate.html|.
\end{sloppypar}

\begin{sloppypar}
\iter\ currently runs on Lisp Machines, and on
HP's, Sun3's and Sparcstations under Lucid.  
\iter\ source and binaries are available at the MIT AI Lab in the
subdirectories of |/src/local/lisplib/|.  The source file,
|iterate.lisp|, is also available for anonymous FTP in the directory
|/com/fpt/pub/| on the machine |TRIX.AI.MIT.EDU| (Internet number
128.52.32.6).  If you are unable to obtain |iterate| in one of these
ways, send mail to |jba@ai.mit.edu| and I will send you the source
file. 
\end{sloppypar}

\begin{sloppypar}
\iter\ resides in the |iterate| package (nickname |iter|).  Just say
\linebreak |(use-package :iterate)| to make all the necessary symbols
available. 
If a symbol is not exported, it appears in this manual with an
``|iterate::|'' prefix.
\end{sloppypar}

Send bug reports to |bug-iterate@ai.mit.edu|.  The |info-iterate|
mailing list will have notices of changes and problems; to have
yourself added, send mail to |info-iterate-request@ai.mit.edu|.


\medskip

\begin{flushleft}
 \bf Acknowledgements
\end{flushleft}
\smallskip

Richard Waters provided invaluable criticism which spurred me to improve
\iter\ greatly.  As early users, David Clemens, Oren Etzioni and Jeff
Siskind helped ferret out many bugs.

%\begin{theindex}
% The index files must be generated with the genindex program
% or from makeindex iterate-manual.idx which creates iterate-manual.ind.

%\baselineskip\normalbaselineskip
%\advance\baselineskip by -2pt

% Fix ``.'' spacing in lists
\documentstyle[12pt,aimemo]{article}

\setlength{\textwidth}{6.0in}
\setlength{\oddsidemargin}{0.25in}
\setlength{\evensidemargin}{0.25in}
\setlength{\topmargin}{0pt}
\setlength{\headsep}{24pt}
\setlength{\textheight}{8.5in}

\let\origpar=\par

% somehow, LaTex fucks with things so that the plain Tex \obeyspaces
% doesn't work.  This is the same, except we use an mbox containing
% a space instead of just a space.

\newcommand{\spc}{\mbox{ }}
\newcommand{\tAb}{\mbox{        }}
\def\obspaces{\catcode`\ =\active\catcode`\^^I=\active}
{\obspaces \global\let =\spc\global\let^^I=\tAb}

\newenvironment{program}{\medskip\samepage\tt\obeylines\obspaces}{\medskip}

% use | for \tt
\catcode`\|=13
\newif\ifvbar
\def|{\ifvbar
         \endgroup\vbarfalse
      \else 
         \begingroup\tt\vbartrue 
      \fi}



\newcommand{\lisp}{\tt}
\newcommand{\iter}{|iterate|}
\newcommand{\iterpg}{|iterate-pg|}
\newcommand{\setf}{|setf|}
\newcommand{\nil}{|nil|}
\newcommand{\nonnil}{non-\nil}
\newcommand{\opt}{{\lisp \&optional}}
\newcommand{\yields}{$\Rightarrow$}



\newbox\Dots
\setbox\Dots=\hbox{\small ...}
\renewcommand{\dots}{\copy\Dots}

% Here's a way to define an environment which will treat all
% paragraphs inside it in the same manner.  It's, as near as I can
% figure, the way that Latex does things.  Using \everypar won't work;
% apparently Latex resets that a lot.  What you have to do is: at the
% top of the file do \let\origpar=\par.  In the
% begin part of the environment, do the parshape (or whatever), then say
% \def\par{{\origpar}}.  The extra braces are necessary.   That's all
% you have to do; at the end of the environment things will return to
% normal automatically.  Why does this all work?  I have no idea.
% Another important thing: there should be a blank line (i.e. a \par)
% between the end of the last paragraph and the \end{environment}.

\newlength{\clindent}
\setlength{\clindent}{0.5in}
\newlength{\clparindent}
\setlength{\clparindent}{\parindent}

\newenvironment{clauses}{\advance\linewidth -\parindent
                        \hangindent\clindent\relax
                         \hangafter=0
                          \parindent=0pt
                          \def\par{{\origpar}}}{}

\newcommand{\cpar}{\hspace \clparindent}

\newcommand{\startitem}{\bigskip\pagebreak[3]}
\newcommand{\finishitem}{\smallskip}

\newcommand{\clause}[1]{\startitem\Clause{#1}\finishitem}
\newcommand{\Clause}[1]{\hbox{#1}}
\newcommand{\clindex}[1]{\index{{\lisp #1}}}
\newcommand{\clindexx}[2]{\index{{\lisp #1\dots #2}}}
\newcommand{\clindexxx}[3]{\index{{\lisp #1\dots #2\dots #3}}}
\newcommand{\clausex}[2]{\startitem\Clausex{#1}{#2}\finishitem}
\newcommand{\Clausex}[2]{\Clause{|#1| #2}\clindex{#1}}

\newcommand{\defvar}[1]{\startitem\deff{#1}{}{Variable}\finishitem}
\newcommand{\defunexpvar}[1]{\startitem\deffnoindex{iterate::#1}{}{Variable}
                                        \lispindex{#1}\finishitem}
\newcommand{\defun}[2]{\startitem\deff{#1}{#2}{Function}\finishitem}
\newcommand{\defmacro}[2]{\startitem\deff{#1}{#2}{Macro}\finishitem}
\newcommand{\defunx}[3]{\startitem\deffx{#1}{#2}{#3}{Function}\finishitem}
\newcommand{\defmacrox}[3]{\startitem\deffx{#1}{#2}{#3}{Macro}\finishitem}
\newcommand{\defunxx}[4]{\startitem\deffxx{#1}{#2}{#3}{#4}{Function}\finishitem}
\newcommand{\defmacroxx}[4]{\startitem\deffxx{#1}{#2}{#3}{#4}{Macro}\finishitem}

\newcommand{\deff}[3]{\deffnoindex{#1}{#2}{#3}\lispindex{#1}}
\newcommand{\deffnoindex}[3]{\hbox to \hsize{{\tt #1} {\it #2}\hfill [{\it #3}]}}
\newcommand{\deffx}[4]{\deff{#1}{#2}{#4}\moreargs{#1}{#3}}
\newcommand{\deffxx}[5]{\deffx{#1}{#2}{#3}{#5}\moreargs{#1}{#4}}

\newcommand{\moreargs}[2]{\hbox to \hsize{\phantom{\lisp #1} {\it #2}\hfill}}

\newcommand{\lispindex}[1]{\index{{\lisp #1}}}

 \newenvironment{note}[1]{\pagebreak[2]\bigskip
                          \hrule\smallskip\small
                         {\setlength{\parindent}{0pt}
                           \par{\bf #1:}}}{\smallskip\hrule\bigskip}



\makeindex

%%%%%%%%%%%%%%%%%%%%%%%%%%%%%%%%%%%%%%%%%%%%%%%%%%%%%%%%%%%%%%%%

%\thispagestyle{empty}

%\vspace*{2in}

%\title{The |Iterate| Manual}
%\author{Jonathan Amsterdam}

%\maketitle

%\thispagestyle{empty}

%\begin{center}
%\Large *** DRAFT ***
%\end{center}
%This manual is in beta-test. It will become an AI Memo in a couple of
%months.  I would appreciate any comments or
%corrections.  You can reach me at email address |jba@ai.ai.mit.edu|.

\begin{document}

\def\writtenby{Jonathan Amsterdam}
\def\contractno{This report describes research done at the Artificial
Intelligence Laboratory of the Massachusetts Institute of Technology.
Support for the laboratory's artificial intelligence research is
provided in part by the Advanced Research Projects Agency of the
Department of Defense under Office
of Naval Research contract N00014-85-K-0124.}

\AIMEMO{1236}{\Large The Iterate Manual \\[1in]}
{This is the manual for version 1.4 of \iter, a powerful iteration
macro for Common  
Lisp. \iter\ is similar to {\tt loop} but provides numerous additional
features, is well integrated with Lisp, and is extensible. 
\vfill }


% use ~ for \em
\catcode`\~=13
\newif\ifem
\def~{\ifem
         \/\endgroup\emfalse
      \else 
         \begingroup\em\emtrue 
      \fi}



%\pagebreak
%\setcounter{page}{1}

%\thispagestyle{empty}


\tableofcontents

\pagebreak
\pagestyle{headings}

\section{Introduction}

\begin{sloppypar}
This manual describes
\iter, a powerful iteration facility for Common Lisp.
\iter\ provides abstractions for many common iteration
patterns and allows for the definition of additional patterns.  \iter\ is
a macro that expands into ordinary Lisp at
compile-time, so it is more efficient than higher-order
functions like |map| and |reduce.|  
While it is similar to |loop|, \iter\ offers a more Lisp-like syntax
and enhanced extensibility.
(For a more complete
comparison of \iter\ with other iteration constructs, see MIT AI Lab Working
Paper No. 324, ~Don't Loop, Iterate.~)
\end{sloppypar}

An \iter\ form consists of the symbol |iter|\footnote{You can also use
|iterate|, but |iter| is preferred because it avoids potential conflicts with
possible future additions to Common Lisp, and because it saves
horizontal space when writing code.}
followed by one or more
forms, some of which may be \iter\ 
~clauses.~  Here is a simple example of \iter\ which collects the
numbers from 1 to 10 into a list, and returns the list.  The return
value is shown following the arrow.
\begin{program}
(iter (for i from 1 to 10)
      (collect i))          \yields (1 2 3 4 5 6 7 8 9 10)
\end{program}
This form contains two clauses: a |for| clause that steps the
variable |i| over the integers from 1 to 10, and a |collect|
clause that accumulates its argument into a list.  With a few
exceptions, all \iter\ clauses have the same format: alternating
symbols (called ~keywords~) and expressions (called
~arguments~).  The syntax and terminology are those of Common
Lisp's keyword lambda lists.  One difference is that \iter's keywords
do not have to begin with a colon---though they may, except for the
first symbol of a clause.  So you can also write  |(for i :from
1 :to 10)| if you prefer.

Any Lisp form can appear in the body of an \iter, where it will have
its usual meaning.  \iter\ walks the entire body, expanding macros,
and recognizing clauses at any level.  This example collects all the
odd numbers in a list:
\begin{program}
(iter (for el in list)
      (if (and (numberp el) (oddp el))
          (collect el)))
\end{program}

There are clauses for iterating over numbers, lists, arrays and other
objects, and for
collecting, summing, counting, maximizing and other useful operations.
\iter\ also supports the creation of new variable bindings, stepping
over multiple sequences at once, destructuring, and compiler
declarations of variable types.  The following example illustrates
some of these features: 
\begin{program}
(iter (for (key . item) in alist)
         (for i from 0)
         (declare (fixnum i))
         (collect (cons i key)))
\end{program}

This loop takes the keys of an alist and returns a new alist
associating the keys with their positions in the original list. The
compiler declaration for |i| will appear in the generated code
in the appropriate place.

%%%%%%%%%%%%%%%%%%%%%%%%%%%%%%%%%%%%%%%%%%%%%%%%%%%%%%%%%%%%%%%%
\section{Clauses}

Most of \iter's clauses will be familiar to |loop| programmers.  
(|loop| is an iteration macro that has been incorporated into Common
Lisp.  Seen Guy Steele's {\em Common Lisp, 2nd Edition}.)
In
nearly all cases they behave the same as their |loop| counterparts, so
a |loop| user can switch to \iter\ with little pain (and much gain).

All clauses with the standard keyword-argument syntax consist of two
parts: a ~required~ part, 
containing keywords that must be present and in the right order; and
an ~optional~ part, containing keywords that may be omitted and,
if present, may occur in any order.  In the descriptions below, the
parts are separated by the Lisp lambda-list keyword \opt.


\subsection{Drivers}

An iteration-driving clause
conceptually causes the iteration to go forward.  Driver clauses in
\iter\ allow iteration over numbers, lists, vectors, hashtables, packages,
files and streams.  Iteration-driving clauses must
appear at the top level of an \iter\ form; they cannot be nested
inside another clause.  The driver variable is updated at the point
where the driver clause occurs.  Before the clause is executed for the
first time, the value of the variable is undefined.

%Also, regardless of where the driver clause
%appears in the body, the driver variable is stepped at the top of the
%loop; hence it is stylistically preferable, though
%not required, to place driver clauses at the beginning of the \iter.

Multiple drivers may appear in a single \iter\ form, in which case all
of the driver variables are updated each time through the loop, in the
order in which the clauses appear.  The first driver to terminate will
terminate the entire loop.

In all cases, the value of the driver variable on exit from the loop,
including within the epilogue code (see the |finally| clause), is
undefined.

All the parameters of a driver clause are evaluated once, before the
loop begins.  Hence it is not possible to change the bounds or other
properties of an iteration by side-effect from within the loop.

With one exception, driver clauses begin with the word |for| (or
the synonym |as|) and mention an iteration variable, which is
given a binding within the \iter\ form.
The exception is |repeat|, which just executes a loop a
specified number of times:

\begin{clauses}

\clausex{repeat}{~n~}
Repeats the loop ~n~ times.  For example:
\begin{program}
(iter (repeat 100)
      (print "I will not talk in class."))
\end{program}
\cpar If $n \leq 0$, the
loop will never be executed.  If ~n~ is not an integer, the
actual number of executions will be $\lceil n \rceil$.

\end{clauses}

\pagebreak[3]

\subsubsection{Numerical Iteration}

\begin{clauses}
\clausex{for}{~var~ |\&sequence|} 
The general form for iterating over a sequence of numbers requires a
variable and, optionally, one or more keywords that provide the bounds
and step size of the iteration. The |\&sequence|
lambda-list keyword is a shorthand for these sequence keywords.  They are:
|from|, |upfrom|, |downfrom|, |to|, |downto|, |above|, |below| and |by|.
|from| provides the starting value for ~var~
and defaults to zero. |to| provides a final value and implies that the
successive values of ~var~ will be increasing; |downto| implies that
they will be decreasing.  The loop terminates when ~var~ passes
the final value (i.e. becomes smaller or larger than it, depending on
the direction of iteration); in other  words, the loop body
will never be 
executed for values of ~var~ past the final value.
|below| and |above| are similar to |to| and
|downto|, except that the loop terminates when ~var~ equals or passes
the final value.

\cpar If no final value is specified, the variable will be stepped
forever.  Using |from| or |upfrom| will result in increasing
values, while |downfrom| will give decreasing values.

\cpar On each iteration,
~var~ is incremented or decremented by the value of the sequence
keyword |by|, which defaults to 1.  It should always be a positive
number, even for downward iterations.

\cpar In the following examples, the sequence of numbers generated is shown
next to the clause.

\begin{program}
(for i upfrom 0) \yields\ 0 1 2 \ldots
(for i from 5)   \yields\ 5 6 7 \ldots    ; either from or upfrom is okay
(for i downfrom 0) \yields\ 0 -1 -2 \ldots
(for i from 1 to 3) \yields\ 1 2 3
(for i from 1 below 3) \yields\ 1 2
(for i from 1 to 3 by 2) \yields\ 1 3
(for i from 1 below 3 by 2) \yields\ 1
(for i from 5 downto 3) \yields\ 5 4 3
\end{program}

\end{clauses}

\subsubsection{Sequence Iteration}

There are a number of clauses for iterating over sequences.  In all of
them, the argument following |for| may be a list instead of a
symbol, in which case destructuring is performed.  See section
\ref{destructuring}. 

\begin{clauses}

\clause{|for| ~var~ |in| ~list~ \opt\ |by| ~step-function~}
\clindexx{for}{in}
~var~ is set to successive elements of list.  
~step-function,~ which 
defaults to |cdr|, is used to obtain the next sublist. 

\clause{|for| ~var~ |on| ~list~ \opt\ |by| ~step-function~}
\clindexx{for}{on}
~var~ is set to successive sublists of list. 
~step-function~ (default |cdr|) is used as in |for\dots in|.

\end{clauses}

\medskip

These two clauses use |atom| to test for the end
of a list.  Hence, given a list whose final |cdr| is not \nil,
they will silently ignore the last |cdr|.  Other choices are
|endp|,  which would signal an error, and |null|, which
would probably result in an error somewhere else.  If you wish to use
an end-test other than |atom|, set the variable
|iterate::*list-end-test*|\lispindex{*list-end-test*} to the name of
the desired function. 

\begin{clauses}
\clause{|for| ~var~ |in-vector| ~vector~ |\&sequence|}
\clindexx{for}{in-vector}
~var~ takes on successive elements from ~vector.~   The vector's
fill-pointer is observed.  Here and in subsequent clauses, the |\&sequence|
keywords include |with-index|, which takes a symbol as argument
and uses it for the index variable instead of an internally generated
symbol.  The other |\&sequence| keywords behave as in numerical
iteration, except that the default iteration bounds are the bounds of
the vector.  E.g. in |(for i in-vector v downto 3)|,
|i| will start off being bound to the last element in |v|,
and will  
be set to preceding elements down to and including the element with
index 3.

\clause{|for| ~var~ |in-sequence| ~seq~ |\&sequence|}
\clindexx{for}{in-sequence}
This uses Common Lisp's generalized sequence functions, |elt|
and |length|, to obtain elements and determine the length of
~seq.~   Hence it will work for any sequence, including lists, and will
observe the fill-pointers of vectors.

\clause{|for| ~var~ |in-string| ~string~ |\&sequence|}
\clindexx{for}{in-string}
~var~ is set to successive characters of ~string.~

\startitem
\Clause{|for| ~var~ |index-of-vector| ~vector~ |\&sequence|}
\clindexx{for}{index-of-vector}
\Clause{|for| ~var~ |index-of-sequence| ~sequence~ |\&sequence|}
\clindexx{for}{index-of-sequence}
\Clause{|for| ~var~ |index-of-string| ~string~ |\&sequence|}
\clindexx{for}{index-of-string}
\finishitem
~var~ is set to successive indices of the sequence.
These clauses avoid the overhead of accessing the sequence elements
for those applications where they do not need to be examined, or are
examined rarely.  They admit all the optional keywords of the other
sequence drivers except the (redundant) |with-index| keyword.

\clause{|for| ~(key value)~ |in-hashtable| ~table~}
\clindexx{for}{in-hashtable}

~key~ and ~value,~ which must appear as shown in a list
and may be destructuring templates, are set to the keys and values of
~table~.  If ~key~ is |nil|, then the hashtable's keys will be ignored;
similarly for ~value~.
The order in which 
elements of ~table~ will be retrieved is unpredictable.

\clause{|for| ~var~ |in-package| ~package~ \opt\ |external-only| ~ext~}
\clindexx{for}{in-package}
Iterates over all the symbols in ~package,~ or over only the
external symbols if ~ext~ is specified and non-|nil|.  ~ext~ is not
evaluated.  The same symbol may appear more than once.

\clause{|for| ~var~ |in-packages| ~packages~ \opt\ |having-access| ~symbol-types~}
\clindexx{for}{in-packages}
Iterates over all the symbols from the list of packages denoted by the
descriptor ~packages~ and having accessibility (or visibility) given by
~symbol-types~.  This defaults to the list |(:external :internal :inherited)|
and is not evaluated.  ~var~ must be a list of up to three variables: in each
iteration, these will be set to a symbol, its access-type and package (as per
|with-package-iterator| in ANSI-CL).  The same symbol may appear more than
once.

\end{clauses}

\begin{clauses}

\clause{|for| ~var~ |in-file| ~name~ \opt\ |using| ~reader~}
\clindexx{for}{in-file}
Opens the file ~name~ (which may be a string or pathname) for input,
and iterates 
over its contents. ~reader~ defaults to |read|, so by default ~var~
will be bound to the successive forms in the file.  The \iter\ body is
wrapped in an unwind-protect to ensure that the file is closed no
matter how the \iter\ is exited.

\clause{|for| ~var~ |in-stream| ~stream~ \opt\ |using| ~reader~}
\clindexx{for}{in-stream}
Like |for\dots in-file|, except that ~stream~ should be an existing
stream object that supports input operations.

\end{clauses}

\subsubsection{Generalized Drivers}

These are primarily useful for writing drivers that can also be used
as generators (see section \ref{generators}, below).
 
\begin{clauses}

\clause{|for| ~var~ |next| ~expr~}
\clindexx{for}{next}

~var~ is set to ~expr~ each time through the loop.  Destructuring is
performed.  When the clause is used as a generator, ~expr~ is the code
that is executed when |(next ~var~)| is 
encountered (see section \ref{generators}, below).
~expr~ should compute the first value for ~var~, as well
as all subsequent values, and is responsible for terminating the loop.
For compatibility with future versions of \iter, this termination
should be done with |terminate|\clindex{terminate}, which can be
considered a synonym for |finish| (see section \ref{control-flow}).

\cpar As an example, the following clauses are equivalent to |(for
i from 1 to 10)|:
\begin{program}
(initially (setq i 0))
(for i next (if (> i 10) (terminate) (incf i)))
\end{program}

\clause{|for| ~var~ |do-next| ~form~}
\clindexx{for}{do-next}
~form~ is evaluated each time through the loop.  Its value is ~not~
set to ~var~; that is ~form's~ job. ~var~ is only present so that
\iter\ knows it is a driver variable.  \linebreak |(for ~var~ next
~expr~)| is equivalent to |(for ~var~ do-next (dsetq ~var~ ~expr~))|.
(See section \ref{destructuring} for an explanation of |dsetq|.) 

\end{clauses}

\subsubsection{Generators}
\label{generators}

In all of the above clauses, the driver variable is updated on each
iteration.
Sometimes it is desirable to have greater control over updating.  
For instance, consider the problem of associating numbers, in
increasing order and with no gaps, with the
\nonnil\ elements of a list.  One obvious first pass at writing this is:

\begin{program}
(iter (for el in list)
      (for i upfrom 1)
      (if el (collect (cons el i))))
\end{program}
But on the list |(a b nil c)| this produces |((a . 1) (b . 2) (c .
4))| instead of the desired |((a . 1) (b . 2) (c . 3))|.  The problem
is that |i| is incremented each time through the loop, even when |el|
is |nil|.  

The problem could be solved elegantly if we could step |i| only when
we wished to.  This
can be accomplished for any \iter\ driver by writing |generate| (or
its synonym |generating|)
instead of |for|.  Doing so produces a ~generator~---a driver whose
values are yielded explicitly.  To obtain the next value of a
generator variable ~v~, write \linebreak |(next ~v~)|.  The value of a |next|
form is the next value of ~v~, as determined by its associated driver
clause.  |next| also has the side-effect of updating ~v~ to that
value.  If there is no next value, |next| will terminate the loop,
just as with a normal driver.


Using generators, we can now write our example like this:
\begin{program}
(iter (for el in list)
      (generate i upfrom 1)
      (if el (collect (cons el (next i)))))
\end{program}
Now |i| is updated only when |(next i)| is executed, and this occurs
only when |el| is \nonnil.

To better understand the relationship between ordinary drivers and
generators, observe that we can rewrite an ordinary driver using its
generator form immediately followed by |next|, as this example shows:
\begin{program}
(iter (generating i from 1 to 10)
      (next i)
      ...)
\end{program}
Provided that the loop body contains no |(next i)| forms, this will
behave just as if we had written |(for i from 1 to 10)|.

We can still refer to a driver variable ~v~ without using |next|; in
this case, its value is that given to it by the last evaluation of
|(next ~v~)|.  Before |(next ~v~)| has been called the first time, the
value of ~v~ is undefined.  

This semantics is more flexible than
one in which ~v~ begins the loop bound to its first value and calls
of |next| supply subsequent values, because it means the loop will not
terminate too soon if the generator's sequence is empty.  For
instance, consider the following code, which tags \nonnil\ elements of
a list using a list of tags, and also counts the null elements.
(We assume there are at least as many tags as \nonnil\ elements.)
\begin{program}
(let* ((counter 0)
       (tagged-list (iter (for el in list)
                          (generating tag in tag-list) 
                          (if (null el)
                              (incf counter)
                              (collect (cons el (next tag)))))))
  ...)
\end{program}
It may be that there are just as many tags as non-null elements of
|list|.  If all the elements of |list| are null, we still want the
counting to proceed, even though |tag-list| is \nil.  If |tag| had to be
assigned its first value before the loop begins, we would have had to
terminate the loop before the first iteration, since when |tag-list|
is \nil, |tag| 
has no first value.  With the existing semantics, however, |(next
tag)| will never execute, so the iteration will cover all the elements of
|list|. 
        
When the ``variable'' of a driver clause is actually a destructuring
template containing several variables, all the variables are eligible
for use with |next|.  As before, |(next ~v~)| evaluates to ~v's~ next
value; but the effect is to update all of the template's variables.
For instance, the following code will return the list |(a 2 c)|.
\begin{program}
(iter (generating (key . item) in '((a . 1) (b . 2) (c . 3)))
      (collect (next key))
      (collect (next item)))
\end{program}

Only driver clauses with variables can be made into generators.  This
includes all clauses mentioned so far except for |repeat|.  It does
~not~ include |for\dots previous|, |for\dots =|,
|for\dots initially\dots then| or |for\dots first\dots then| (see
below). 

\subsubsection{Previous Values of Driver Variables}

Often one would like to access the value of a variable on a previous
iteration.  \iter\ provides a special clause for accomplishing this.

\begin{clauses}

\clause{|for| ~pvar~ |previous| ~var~ \opt\ |initially| ~init~
|back| ~n~}
\clindexx{for}{previous}
Sets ~pvar~ to the previous value of ~var,~ which should be a driver
variable, a variable from another |for\dots previous| clause, or a
variable established by a |for\dots =|, 
|for\dots initially\dots then| or |for\dots first\dots then| clause
(see section \ref{setting}).
Initially, ~pvar~ is given the value ~init~ (which defaults to \nil).  
The ~init~ expression will be moved outside the loop body, so it
should not depend on anything computed within the loop.
~pvar~ retains the value of ~init~ until ~var~ is set to its second
value, at which point ~pvar~ is set to ~var's~ first value; and so on.  

\cpar The
argument ~n~ to |back| must be a constant, positive integer, and
defaults to 1.  It determines how many iterations back ~pvar~ should
track ~var~.  For example, when ~n~ is 2, then ~pvar~ will be assigned
~var's~ first value when ~var~ is set to its third value.

\cpar A |for\dots previous| clause may occur after or before its
associated driver clause. |for\dots previous| works with generators as
well as ordinary drivers.

\pagebreak[3]

\cpar Example:
\begin{program}
(iter (for el in '(1 2 3 4))
      (for p-el previous el)
      (for pp-el previous p-el initially 0)
      (collect pp-el))
\end{program}
This evaluates to |(0 0 1 2)|.  It could have been written more
economically as 
\begin{program}
(iter (for el in '(1 2 3 4))
      (for pp-el previous el back 2 initially 0)
      (collect pp-el))
\end{program}


\end{clauses}

\subsection{Variable Binding and Setting}
\label{setting}

Several clauses exist for establishing new variable bindings or for
setting variables in the loop.  They all support destructuring.

\begin{clauses}

\clausex{with}{~var~ \opt\ |=| ~value~}
Causes ~var~ to be bound to value before the loop body is entered.
If ~value~ is not supplied, ~var~ assumes a default
binding, which will be 
\nil\ in the absence of declarations.  Also, if ~value~ is not
supplied, no destructuring is performed; instead, ~var~ may be a list of
symbols, all of which are given default 
bindings.  If ~value~ is supplied, ~var~ is bound to it, with
destructuring.  

\cpar Because |with|
creates bindings whose scope includes the entire \iter\ form, it is
good style to put all |with| clauses at the beginning.

\cpar Successive occurrences of |with| result in sequential
bindings (as with 
|let*|).  There is no way to obtain parallel bindings; see
section \ref{bindings} for a rationale.  


\clause{|for| ~var~ |=| ~expr~}
\clindexx{for}{=}
On each iteration, ~expr~ is evaluated and ~var~ is set
to its value. 

\cpar This clause may appear to do the same thing as |for\dots next|.
In fact, they are quite different.  |for\dots =| provides only three
services: it sets up a binding for ~var~, sets it to ~expr~ on each
iteration, and makes it possible to use |for\dots previous| with
~var~.  |for\dots next| provides these services in addition to the
ability to turn the driver into a generator.
%Also, the code which sets ~var~ appears in the loop body in the same
%place as the |for\dots =| clause; the code for |for\dots next| appears
%at the top of the loop, as with other drivers (except when being used
%as a generator).

\clause{|for| ~var~ |initially| ~init-expr~ |then| ~then-expr~}
\clindexxx{for}{initially}{then}
Before the loop begins, ~var~ is set to ~init-expr;~ on all
iterations after the first it is set to ~then-expr.~  This clause
must occur at top-level.  ~init-expr~ will be moved outside the loop
body and ~then-expr~ will be moved to the end of the loop body, so
they are subject to code motion problems (see section
\ref{code-movement}). 

\cpar This clause may appear to be similar to |for\dots next|, but in
fact they differ significantly.  |for\dots initially\dots then| is
typically used to give ~var~ its first value before the loop begins,
and subsequent values on following iterations.  This is incompatible
with generators, whose first value and subsequent values must all be
computed by |(next ~var~)|.  Also, the update of ~var~ in 
|for\dots initially\dots then| does not occur at the location of the
clause. 
Use |for\dots initially\dots then| for
one-shot computations where its idiom is more convenient, but use
|for\dots next| for extending \iter\ with new drivers (see section
\ref{extend}).

\clause{|for| ~var~ |first| ~first-expr~ |then| ~then-expr~}
\clindexxx{for}{first}{then}
The first time through the loop, ~var~ is set to ~first-expr;~ on
subsequent iterations, it is set to ~then-expr.~  This differs from
|for\dots initially| in that ~var~ is set to ~first-expr~
inside the loop body, 
so ~first-expr~ may depend on the results of other clauses.  For
instance, 
\begin{program}
(iter (for num in list)
      (for i first num then (1+ i))
      ...)
\end{program}
will set |i| to the first element of |list| on the first
iteration, whereas 
\begin{program}
(iter (for num in list)
      (for i initially num then (1+ i))
      ...)
\end{program}
is probably erroneous; |i| will be bound to |num|'s
default binding (usually \nil) for the first iteration.

\end{clauses}

\begin{note}{Compatibility note}
|loop|'s |for\dots =| works like \iter's, but |loop| used the syntax
|for\dots =\dots then| to mean |for\dots initially\dots then|.  It was
felt that these two operations were sufficiently different to warrant
different keywords.

Also, the |for| in the above three clauses is misleading,
since none is true driver (e.g. none has a corresponding |generate|
form).  |setting| would have been a better choice, but |for| was used
to retain some compatibility with |loop|.  
\end{note}


%%%%%%%%%%%%%%%%%%%%%%%%%%%%%%%%
\subsection{Gathering Clauses}
Many of \iter's clauses accumulate values into a variable, or set a
variable under certain conditions.  At the end of the loop, this
variable contains the desired result.  All these clauses have an
optional |into| keyword, whose argument should be a symbol.  If the
|into| keyword is not supplied, the accumulation variable will be
internally generated and its value will be returned at the end of the
loop; if a variable is specified, that variable is used for the
accumulation, and is not returned as a result---it is up to the user
to return it explicitly, in the loop's epilogue code (see |finally|).
It is safe to examine the accumulation variable during the loop, but
it should not be modified.

These clauses all begin with a verb.  When the verb does
not conflict with an existing Common Lisp function, then it may be
used in either its infinitival or present-participle form (e.g. |sum|,
|summing|).  However, when there is a conflict with Common Lisp, only
the present-participle form may be used (e.g. |unioning|).  This is to
prevent \iter\ clauses from clashing with Common Lisp
functions. 

%although these clauses are described as ``producing a value,'' it is a
%mistake to think of the lisp list representing the clause as a
%value-producing form in the usual way.  clauses may legally be written
%where a value is expected, e.g. |(setq x (sum i))|, but the
%lisp value of a clause in such a context is undefined.

\subsubsection{Reductions}

~Reduction~ is an extremely common iteration pattern in which
the results of successive applications of a
binary operation are accumulated.
For example, a loop that computes the sum of the
elements of a list is performing a reduction with the addition
operation.  This could be written in Common Lisp as  |(reduce \#'+
list)| or with \iter\ as
\begin{program}
(iter (for el in list)
      (sum el))
\end{program}

\begin{clauses}

\clausex{sum}{~expr~ \opt\ |into| ~var~}
Each time through the loop, ~expr~ is evaluated and added to a
variable, which is bound initially to zero.  If ~expr~ has a type,
it is ~not~ used as the type of the sum variable, which is always
|number|.  To get the result variable to be of a more specific
type, use an explicit variable, as in
\begin{program}
(iter (for el in number-list)
      (sum el into x)
      (declare (fixnum x))
      (finally (return x)))
\end{program}

\clausex{multiply}{~expr~ \opt\ |into| ~var~}
Like |sum|, but the initial value of the result variable is $1$,
and the variable is updated by multiplying ~expr~ into it.

\clausex{counting}{~expr~ \opt\ |into| ~var~}
~expr~ is evaluated on each iteration.  If it is \nonnil, the
accumulation variable, initially zero, is incremented.

\startitem
\Clausex{maximize}{~expr~ \opt\ |into| ~var~}
\Clausex{minimize}{~expr~ \opt\ |into| ~var~}
\finishitem
~expr~ is evaluated on each iteration and its extremum (maximum or
minimum) is stored in the accumulation variable.  If ~expr~ is never
evaluated, then the result is \nil\ (if the accumulation variable is
untyped) or $0$ (if it has a numeric type).

\clausex{reducing}{~expr~ |by| ~func~ \opt\ |initial-value| ~init-val~ |into| ~var~}
This is a general way to perform reductions. ~func~ should be a
function of two arguments, the first of which will be the value
computed so far and
the second of which will be the value of ~expr~.  It should return the
new value.
|reducing| is roughly equivalent to the
Common Lisp |(reduce ~func~ ~sequence~ :key ~expr-function~)|, where
~expr-function~ is used to derive values from the successive elements
of ~sequence~.

\cpar If the |reducing| clause is never executed, the result is
undefined.

\cpar It is not necessary to provide an
initial value, but better code can be generated if one is supplied.
Regardless of its location in the \iter\ body, ~init-val~ will be
evaluated before the loop is entered, so it should not depend on any
value computed inside the \iter\ form.  

%\cpar if a ~var~ is not specified, you can get \iter\ to declare the
%type of the internal variable by putting a |the| expression around
%~func.~  see section \ref{types}.

\end{clauses}

%\begin{note}{implementation note}
%in principle, |maximize| and |minimize| can be thought of
%as reductions where the initial value is the smallest (or largest)
%value that the accumulation variable can assume.  because lisp's
%bignums can represent arbitrary integers, these clauses cannot be
%implemented as reductions in general.  if, however, the type of 
%~expr~ or ~var~ can be determined to be a fixnum
%or a float, \iter\ will implement the clause as a true reduction,
%using one of the constants |most-negative-fixnum|, |%most-positive-fixnum|, 
%|most-negative-short-float|, etc. as appropriate.
%\end{note}

\subsubsection{Accumulations}

All the predefined accumulation clauses add values to a sequence.  If
the sequence is a list, they all 
have the property that the partial list is kept in the correct order
and available for inspection at any point in the loop.

\begin{clauses}

\clausex{collect}{~expr~ \opt\ |into| ~var~ |at| ~place~ |result-type| ~type~}
Produces a sequence of the values of ~expr~ on each iteration. ~place~
indicates where the next value of ~expr~ is added to the list and may
be one of the symbols |start|, |beginning| (a synonym for |start|) or
|end|.  The symbol may be quoted, but need not be.  The default is
|end|.  For example,
\begin{program}
(iter (for i from 1 to 5)
      (collect i))
\end{program}
produces |(1 2 3 4 5)|, whereas
\begin{program}
(iter (for i from 1 to 5)
      (collect i at beginning))
\end{program}
produces |(5 4 3 2 1)| (and is likely to be faster in most Common Lisp
implementations).

\cpar If ~type~ is provided, it should be a subtype of |sequence|.
The default is |list|.  Specifying a type other than |list| will
result in |collect| returning a sequence of that type.  ~However,~ the
type of the sequence being constructed when inside the loop body is
undefined when a non-|list| type is specified.  (As with ~place~,
quoting ~type~ is optional.)

\clausex{adjoining}{~expr~ \opt\ |into| ~var~ |test| ~test~ |at| ~place~ |result-type| ~type~}
Like |collect|, but only adds the value of ~expr~ if it is not
already present.  ~test,~ which defaults to |\#'eql|, is
the test to be used with |member|.

\startitem
\Clausex{appending}{~expr~ \opt\ |into| ~var~ |at| ~place~}
\Clausex{nconcing}{\ ~expr~ \opt\ |into| ~var~ |at| ~place~}
\Clausex{unioning}{\ ~expr~ \opt\ |into| ~var~ |test| ~test~ |at| ~place~}
\Clausex{nunioning}{~expr~ \opt\ |into| ~var~ |test| ~test~ |at| ~place~}
\finishitem
These are like |collect|, but behave like the Common Lisp functions
|append|, |nconc|, |union| or |nunion|.  
As in Common Lisp, they work only on lists.  Also as in Common Lisp,
|unioning| and |nunioning| assume that the value of ~expr~ contains no
duplicates. 

\clausex{accumulate}{~expr~ |by| ~func~ \opt\ |initial-value| ~init-val~ |into| ~var~}
This is a general-purpose accumulation clause. ~func~ should be a
function of two arguments, the value of ~expr~ and the value
accumulated so far in the iteration, and it should return the updated
value.  If no initial value is supplied, \nil\ is used.  

%\cpar If a ~var~ is not specified, you can get \iter\ to declare the
%type of the internal variable by putting a |the| expression around
%~func.~  see section \ref{types}.

\cpar The differences between |accumulate| and |reducing| are slight.
One difference is that the functions take their arguments in a
different order.  Another is that in the absence of ~init-val~,
|accumulate| will use \nil, whereas |reducing| will generate different
code that avoids any dependence on the initial value.
The reason for having both clauses is that one usually
thinks of reductions (like |sum|) and accumulations (like |collect|)
as different beasts.

\end{clauses}


\subsubsection{Finders}

A ~finder~ is a clause whose value is an expression that meets some
condition.

\begin{clauses}

\clause{|finding| ~expr~ |such-that| ~test~ \opt\ |into| ~var~ |on-failure| ~failure-value~}
\clindexx{finding}{such-that}
If ~test~ (which is an expression) ever evaluates to \nonnil, the loop
is terminated, the 
epilogue code is run and the value of ~expr~ is returned.  Otherwise,
\nil\ (or ~failure-value,~ if provided) is returned.  If ~var~ is
provided, it will have either the \nonnil\ value of ~expr~ or
~failure-value~ when the epilogue code is run.

\cpar As a special case, if the ~test~ expression is a sharp-quoted
function, then it is applied to ~expr~ instead of being simply
evaluated.  E.g. |(finding x such-that \#'evenp)| is equivalent to
|(finding x such-that (evenp x))|.  

%\cpar although ~test~ need have nothing to do with ~%expr~ as in
%|(finding j such-that (> i 3))|, it usually 
%will: |(finding (length el) such-that (oddp (length el)))|.  to
%avoid performing the |length| computation twice, you could write
%|(finding (length el) such-that \#'oddp)| or |(finding (length
%el) such-that 'oddp)|; for these cases, \iter\ generates code that
%executes ~expr~ only once.  the code for |\#'oddp|
%is slightly different from that for {\lisp 'oddp}; see the discussion
%under {\lisp for\dots in} and {\lisp for\dots on}.

\cpar |On-failure| is a misnomer. Because it is always evaluated, it behaves
more like the default third argument to the |gethash| function. As a result,
|on-failure (error "Not found")| makes no sense. Instead, the clauses |leave|
or |thereis| can be used in conjunction with |finally| as follows:
\begin{program}
(iter (for x in '(1 2 3))
      (if (evenp x) (leave x))
      (finally (error "not found")))
\end{program}

\cpar This clause may appear multiple times when all defaults are
identical. It can also be used together with either |always|/|never| or
|thereis| if their defaults match. More specifically, |on-failure nil| is
compatible with |thereis|, while |on-failure t| is compatible with |always|
and |never| clauses.
\begin{program}
(iter (for i in '(7 -4 2 -3))
      (if (plusp i)
	  (finding    i  such-that (evenp i))
          (finding (- i) such-that (oddp i))))
\end{program}

\startitem
\Clause{|finding| ~expr~ |maximizing| ~m-expr~ \opt\ |into| ~var~}
\clindexx{finding}{maximizing}
\Clause{|finding| ~expr~ |minimizing| ~m-expr~ \opt\ |into| ~var~}
\clindexx{finding}{minimizing}
\finishitem
Computes the extremum (maximum or minimum) value of ~m-expr~ over all
iterations, and returns the value of ~expr~ corresponding to the
extremum.  ~expr~ is evaluated inside the loop at the time the new
extremum is established.  If ~m-expr~ is never evaluated (due to, for
example, being embedded in a conditional clause), then the returned
value depends on the type, if any, of ~expr~ (or ~var,~ if
one is supplied).  If there is no type, the returned
value will be \nil; if the type is numeric, the returned value will be
zero.

\cpar For these two clauses, ~var~ may be a list of two
symbols; in that case, the first is used to record ~expr~ and
the second, ~m-expr.~  

\cpar As with |finding\dots such-that|, if ~m-expr~ is a sharp-quoted
function, then it is called on ~expr~ instead of being evaluated.

\end{clauses}

\subsubsection{Boolean Tests}

\begin{clauses}

\clausex{first-iteration-p}{}
Returns |t| in the first cycle of the loop, otherwise \nil.

\clausex{first-time-p}{}
Returns |t| the first time the expression is evaluated, and then \nil\ forever.
This clause comes handy when printing (optional) elements separated
by a comma:

\begin{program}
(iter (for el in '(nil 1 2 nil 3))
      (when el
        (unless (first-time-p)
          (princ ", "))
        (princ el)))
\end{program}
produces |"1, 2, 3"|.

\end{clauses}

\subsubsection{Aggregated Boolean Tests}

\begin{clauses}

\clausex{always}{~expr~}
If ~expr~ ever evaluates to
\nil, then \nil\ is immediately returned; the epilogue code is not
executed.  If ~expr~ never evaluates to \nil, the epilogue code
is executed and the last value of ~expr~ (or |t| if ~expr~ was never
evaluated) is returned (whereas |loop| would constantly return |t|).

% mention last evaluated clause when multiple always clauses?

\clausex{never}{~expr~}
Like |(always (not ~expr~))|, except it does not influence the last
value returned by a possible other |always| clause. That is,
\begin{program}
(iter (repeat 2)
      (always 2)
      (never nil)) \yields 2 ; not t
\end{program}

\clausex{thereis}{~expr~}
If ~expr~ is ever \nonnil,
its value is immediately returned without running epilogue code.
Otherwise, the epilogue code is performed and \nil\ is returned.

This clause cannot be used together with |always| or |never|, because their
defaults are opposed (similarly, |(loop always 3 thereis nil)| refuses to
compile in some implementations of |loop|).

\end{clauses}


\subsection{Control Flow}
\label{control-flow}
Several clauses can be used to alter the usual flow of control in a loop.

Note: the clauses of this and subsequent sections don't adhere to \iter's
usual syntax, but instead use standard Common Lisp syntax.  Hence the
format for describing syntax subsequently is
like the standard format used in the Common Lisp manual, not like the
descriptions of clauses above.

\begin{clauses}

\clausex{finish}{}
Stops the loop and runs the epilogue code.

%for example:
%
%\begin{program}
%(iter (with answer = nil)
%         (initially (make-a-mess))
%         (for i from 1 to 10)
%         (when (correct? i)
%           (setq answer i)
%           (finish))
%         (finally (cleanup)))
%\end{program}
%
%this code will execute |cleanup| whether or not the test |(correct?
%i)| ever succeeds. 
%the (more elegant) formulation,
%\begin{program}
%(iter (initially (make-a-mess))
%         (for i from 1 to 10)
%         (finding i such-that (correct? i))
%         (finally (cleanup)))
%\end{program}
%would not execute |cleanup| if |(correct? i)| succeeded; it
%would do an immediate return.

\clausex{leave}{\opt\ ~value~}
Immediately returns ~value~ (default \nil) from the current \iter\
form, skipping the epilogue code.  Equivalent to using |return-from|.

\clausex{next-iteration}{}
Skips the remainder of the loop body and begins the next iteration of
the loop.

\clausex{while}{~expr~}
If ~expr~ ever evaluates to \nil, the loop is terminated and the
epilogue code executed.  Equivalent to |(if (not ~expr~) (finish))|.

\clausex{until}{~expr~}
Equivalent to |(if ~expr~ (finish))|.

\clausex{if-first-time}{~then~ \opt\ ~else~}
If this clause is being executed for the first time in this invocation
of the \iter\ form, then the ~then~ code is evaluated; otherwise the
~else~ code is evaluated.

\cpar |(for ~var~ first ~expr1~ then ~expr2~)| is almost equivalent to
\begin{program}
(if-first-time (dsetq ~var~ ~expr1~)
               (dsetq ~var~ ~expr2~))
\end{program}
The only difference is that the |for| version makes ~var~ available
for use with |for\dots previous|.

\end{clauses}

\subsection{Code Placement}
When fine control is desired over where code appears in a loop
generated by \iter, the following special clauses may be useful.
They are all subject to code-motion problems (see section
\ref{code-movement}).  

\begin{clauses}

\clausex{initially}{|\&rest| ~forms~}
The lisp ~forms~ are placed in the prologue section of the loop, where
they are executed once, before the loop body is entered.

\clausex{after-each}{|\&rest| ~forms~}
The ~forms~ are placed at the end of the loop body, where they
are executed after each iteration.  Unlike the other clauses in this
section, ~forms~ may contain \iter\ clauses.

\clausex{else}{|\&rest| ~forms~}
The lisp ~forms~ are placed in the epilogue section of the loop, where they
are executed if this |else| clause is never met during execution of the
loop and the loop terminates normally.

\clausex{finally}{|\&rest| ~forms~}
The lisp ~forms~ are placed in the epilogue section of the loop, where
they are executed after the loop has terminated normally.

\clausex{finally-protected}{|\&rest| ~forms~}
The lisp ~forms~ are placed in the second form of an unwind-protect
outside the loop.  They are always executed after the loop has
terminated, regardless of how the termination occurred.

\end{clauses}


%%%%%%%%%%%%%%%%%%%%%%%%%%%%%%%%%%%%%%%%%%%%%%%%%%%%%%%%%%%%%%%%
\section{Other Features}

\subsection{Multiple Accumulations}
\label{multiple}

\begin{sloppypar}
It is permitted to have more than one clause accumulate into the same
variable, as in the following:
\begin{program}
(iter (for i from 1 to 10)
      (collect i into nums)
      (collect (sqrt i) into nums)
      (finally (return nums)))
\end{program}
Clauses can only accumulate into the same variable if they are
compatible.  |collect|, |adjoining|, |appending|, |nconcing|,
|unioning| and |nunioning| are compatible with each other; |sum|,
|multiply| and |counting| are compatible; |always| and |never| are compatible;
|finding| \dots |such-that| is compatible with either |thereis| or |always|
and |never| when their defaults match; and |maximize| and |minimize| clauses
are compatible only with other |maximize| and |minimize| clauses,
respectively.

%note that the same variable ~cannot~ be both an accumulation
%variable and an ordinary variable; there can be only one variable with
%a given name within an \iter\ form. 
\end{sloppypar}

\subsection{Named Blocks}

Like Common Lisp |block|s, \iter\ forms can be given names.  The
name should be a single symbol, and it must be the first form in the
\iter.  The generated code behaves exactly like a named block; in
particular, |(return-from ~name~)| can be used to exit it:
\begin{program}
(iter fred
      (for i from 1 to 10)
      (iter barney
            (for j from i to 10)
            (if (> (* i j) 17)
                (return-from fred j))))
\end{program}
An \iter\ form that is not given a name is implicitly named \nil.

Sometimes one would like to write an expression in an inner \iter\ form,
but have it processed by an outer \iter\ form.  This is possible with
the |in| clause.  

\begin{clauses}

\clausex{in}{~name~ |\&rest| ~forms~}
Evaluates ~forms~ as if they were part of the \iter\ form named
~name~.  In other words, \iter\ clauses are processed by the \iter\
form named ~name,~ and not by any \iter\ forms that occur inside ~name.~

\cpar As an example, consider the problem of collecting a list of the
elements in a two-dimensional array.  The naive solution,
\begin{program}
(iter (for i below (array-dimension ar 0))
      (iter (for j below (array-dimension ar 1))
            (collect (aref ar i j))))
\end{program}
\noindent is wrong because the list created by the inner \iter\ is simply
ignored by the outer one.  But using |in| we can write:
\begin{program}
(iter outer (for i below (array-dimension ar 0))
      (iter (for j below (array-dimension ar 1))
            (in outer (collect (aref ar i j)))))
\end{program}
\noindent which has the desired result.

\end{clauses}

\subsection{Destructuring}
\label{destructuring}

In many places within \iter\ clauses where a variable is expected, a
list can be written instead.  In these cases, the value to be assigned
is ~destructured~ according to the pattern
described by the list.  As a simple example, the clause
\begin{program}
(for (key . item) in alist)
\end{program}
\noindent will result in |key| being set to the |car| of
each element in |alist|, and |item| being set to the |cdr|.  The
pattern list may be nested to arbitrary depth, and (as the example
shows) need not be terminated with \nil; the only requirement is that
each leaf be a bindable symbol (or \nil, in which case no binding is
generated for that piece of the structure).

Sometimes, you might like to do the equivalent of a
|multiple-value-setq| in a clause.  This 
``multiple-value destructuring'' can be expressed by writing 
\linebreak |(values $pat_1$ $pat_2 \ldots$)| for a destructuring
pattern, as in 
\begin{program}
(for (values (a . b) c d) = (three-valued-function ...))
\end{program}
\begin{sloppypar}
Note that the $pat_i$ can themselves be destructuring patterns (though
not multiple-value destructuring patterns).  You can't do multiple-value
destructuring in a |with| clause; instead wrap the whole \iter\
form in a |multiple-value-bind|.
\end{sloppypar}

\begin{note}{Rationale}
There are subtle interactions between variable declarations and
evaluation order that make the correct implementation of
multiple-value destructuring in a |with| somewhat tricky.
\end{note}

The destructuring feature of \iter\ is available as a separate
mechanism, using the |dsetq| macro:

\begin{clauses}
\defmacro{dsetq}{template expr}
Performs destructuring of ~expr~ using ~template~.   May be used
outside of an \iter\ form. Yields the primary value of ~expr~.

\end{clauses}

\subsection{On-line Help}

\begin{sloppypar}
There is a limited facility for on-line help, in the form of the 
|display-iterate-clauses| function.
\end{sloppypar}

\begin{clauses}

\defun{display-iterate-clauses}{\opt\ clause-spec}
Displays a list of \iter\ clauses.  If ~clause-spec~ is not
provided, all clauses are shown; if it is a symbol, all clauses
beginning with that symbol are shown; and if it is a list of symbols,
all clauses for which ~clause-spec~ is a prefix are shown.

\end{clauses}

\subsection{Parallel Binding and Stepping}
\label{bindings}

The parallel binding and stepping of variables is a feature that
\iter\ does ~not~ have.  This section attempts to provide a rationale.

We say that two variables are bound ~in parallel~ if neither
binding shadows the other.  This is the usual semantics of |let|
(as opposed to |let*|).  Similarly, we can say that iteration
variables are stepped in parallel if neither variable is updated
before the other, conceptually speaking; in other words, if the code
to update each variable can reference the old values of both variables.

|loop| allows parallel binding of variables and parallel stepping of
driver variables.  My view is that if you are depending on the 
serial/parallel distinction, you are doing something obscure.  If you
need to bind 
variables in parallel using |with|, then you must be using a
variable name that shadows a name in the existing lexical environment.
Don't do that.  The most common use for parallel stepping is to track
the values of variables on the previous iteration, but in fact this
does not require parallel stepping at all; the following will work:
\begin{program}
(iter (for current in list)
      (for prev previous current)
      ...)
\end{program}


%%%%%%%%%%%%%%%%%%%%%%%%%%%%%%%%%%%%%%%%%%%%%%%%%%%%%%%%%%%%%%%%
\section{Types and Declarations}
\label{types}

\subsection{Discussion}

Sometimes efficiency dictates that the types of variables be declared.
This type information needs to be communicated to \iter\ so it can
bind variables to appropriate values.  Furthermore, \iter\ must often
generate internal variables invisible to the user; there needs to be a
way for these to be declared.

As an example, consider this code, which will return the number of
odd elements in |number-list|:
\begin{program}
(iter (for el in number-list)
      (count (oddp el)))
\end{program}
In processing this form,
\iter\ will create an internal variable, let us call it |list17|, to
hold the successive |cdr|s of |number-list|, and  
will bind the variable to |number-list|.  It will also generate a
default binding for |el|; only inside the body of the loop will |el|
be set to the |car| of |list17|.  Finally, \iter\ will generate a
variable, call it |result|, to hold the result of the count, and will
bind it to zero.

When dealing with type declarations, \iter\ observes one simple rule:
~it will never generate a declaration unless requested to do so.~  The
reason is that such declarations might mask errors in compiled code by
avoiding error-checks; the resulting problems would be doubly hard to
track down because the declarations would be hidden from the
programmer.  Of course, a compiler might omit error-checks even in the
absence of declarations, though this behavior can usually be avoided,
e.g. by saying |(declaim (optimize (safety 3)))|.

So, the above \iter\ form will generate code with no declarations.
But say we wish to declare the types of |el| and the internal
variables |list17| and |result|.  How is this done?

Declaring the type of |el| is easy, since the programmer knows
the variable's name:
\begin{program}
(iter (for el in number-list)
      (declare (fixnum el))
      (counting (oddp el)))
\end{program}
\iter\ can read variable type declarations like this one.  Before
processing any clauses, it scans the entire top-level form for type
declarations 
and records the types, so that variable bindings can be performed
correctly.  In this case, |el| will be bound to zero
instead of \nil.  Also, \iter\ collects all the top-level declarations
and puts them at the begining of the generated code, so it is not
necessary to place all declarations at the beginning of an \iter\
form; instead, they can be written near the variables whose types they
declare.  

Since \iter\ is not part of the compiler, it will not know
about declarations that occur outside an \iter\ form; these
declarations must be repeated inside the form.

Here is another way we could have declared the type of |el|:
\begin{program}
(iter (for (the fixnum el) in number-list)
      (counting (oddp el)))
\end{program}
\iter\ extends the Common Lisp |the|\lispindex{the}
form to apply to variables as well as value-producing forms; anywhere
a variable is allowed---in a |with| clause, as the iteration
variable in a driver clause, as the |into| argument of an
accumulation clause, even inside a destructuring template---you can
write |(the ~type~ ~symbol~)| instead.

There is one crucial difference between using a |the| form and
actually declaring the variable: explicit declarations are always
placed in the generated code, but type information from a |the|
form is not turned into an actual declaration unless you tell \iter\
to do so using |iterate:declare-variables|.  See below.

\begin{sloppypar}
Declaring the types of internal variables is harder than declaring the
types of explicitly mentioned variables, since their names
are unknown.  You do it by declaring |iterate:declare-variables|
somewhere inside the top level of the \iter\ form.  (This will also
generate declarations for variables declared using |the|.)
\iter\ does not provide much selectivity here: it's all or none. 
And unfortunately, since \iter\ is not privy to compiler information
but instead reads declarations itself, it will not hear if you
|(declaim (iterate:declare-variables))|.  Instead, set the variable 
|iterate::*always-declare-variables*| to |t| at
compile-time, using |eval-when|.
\end{sloppypar}

To determine the appropriate types for internal variables, \iter\ uses
three sources of information:

\begin{itemize}

\item Often, the particular clause dictates a certain type for a
variable; \iter\ will use this information when available.  In the
current example, the variable |list17| will be given the type
|list|, since that is the only type that makes sense; and the
variable |result| will be given the type |fixnum|, on the
assumption that you will not be counting high enough to need bignums.
You can override this assumption only by using and explicitly declaring a
variable: 
\begin{verbatim}
(iter (declare (iterate:declare-variables))
      (for el in number-list)
      (count (oddp el) into my-result)
      (declare (integer my-result))
      (finally (return my-result)))
\end{verbatim}

\begin{sloppypar}
Other examples of the type assumptions that \iter\ makes are: type
|list| for |into| variables of collection clauses; type |list| for
expressions that are to be destructured; type |vector| for the
variable holding the vector in a |for\dots in-vector| clause, and
similarly for |string| and the |for\dots in-string| clause;
and the implementation-dependent type
|(type-of array-dimension-limit)| for the index and limit
variables generated by sequence iteration drivers like |for\dots
in-vector| and |for\dots in-string| (but not |for\dots in-sequence|,
because it may be used to iterate over a list). 
\end{sloppypar}

\item Sometimes, \iter\ will examine expressions and try to determine
their types in a simple-minded way.  If the expression is
self-evaluating (like a number, for instance), \iter\ knows that the
expression's type is the same as the type of the value it denotes, so
it can use that type.  If the expression is of the form |(the ~type~
~expr~)|, \iter\ is smart enough to extract ~type~ and use it.
However, the current version of \iter\ does not  
examine declarations of function result types or do any type
inference.  It will not determine, for 
example, that the type of |(+ 3 4)| is |fixnum|, or even
|number|.

\item In some cases, the type of an internal variable should match the
type of some other variable.  For instance, \iter\ generates an
internal variable for |(f x)| in the
clause |(for i from 1 to (f x))|, and in the absence of other
information will give it the same type as |i|.  If, however, the
expression had been written |(the fixnum (f x))|, then \iter\
would have given the internal variable the type |fixnum|
regardless of |i|'s type.  The type incompatibility errors that
could arise in this situation are not checked for.

\end{itemize}

Note that if you do declare |iterate:declare-variables|, then
\iter\ may declare user variables as well as internal ones if they do
not already have declarations, though only for variables that it
binds.  For instance, in this code:

\begin{program}
(iter (declare (iterate:declare-variables))
      (for i from 1 to 10)
      (collect i into var))
\end{program}
the variable |var| will be declared to be of type |list|.


\subsection{Summary}

\iter\ understands standard Common Lisp variable type declarations
that occur within an \iter\ form and
will pass them through to the generated code.  If the declaration
|(iterate:declare-variables)|\lispindex{declare-variables}
appears at the top level of an 
\iter\ form, or if 
|iterate::*always-declare-variables*|\lispindex{*always-declare-variables*} 
is \nonnil, then \iter\ will use the type information gleaned from user
declarations, self-evaluating expressions and |the| expressions,
combined with reasonable assumptions, to determine variable
types and declare them.


%%%%%%%%%%%%%%%%%%%%%%%%%%%%%%%%%%%%%%%%%%%%%%%%%%%%%%%%%%%%%%%%
\section{Problems with Code Movement}
\label{code-movement}

Some \iter\ clauses, or parts of clauses, result in code being
moved from the location of the clause to other parts of the loop.
Drivers behave this way, as do code-placement clauses like |initially|
and |finally|.  When using these clauses, there is a danger of writing
an expression that makes sense in its apparent location but will be
invalid or have a different meaning in another location.  For example:
\begin{program}
(iter (for i from 1 to 10)
      (let ((x 3))
        (initially (setq x 4))))
\end{program}
While it may appear that the |x| of |(initially (setq x 4))| is the
same as the |x| of |(let ((x 3)) \dots|, in fact they are not:
|initially| moves its code outside the loop body, so |x| would refer
to a global variable.  Here is another example of the same problem:
\begin{program}
(iter (for i from 1 to 10)
      (let ((x 3))
        (collect i into x)))
\end{program}
If this code were executed, |collect| would create a binding for its
|x| at the top level of the \iter\ form that the |let| will shadow.

Happily, \iter\ is smart enough to catch these errors; it
walks all problematical code to ensure that free variables are not
bound inside the loop body, and checks all variables it binds for the
same problem.

However, some errors cannot be caught:

\begin{program}
(iter (with x = 3)
      (for el in list)
      (setq x 1)
      (reducing el by \#'+ initial-value x))
\end{program}
|reducing| moves its |initial-value| argument to the initialization
part of the loop in order to produce more efficient code.  Since
\iter\ does not perform data-flow analysis, it cannot determine that
|x| is changed inside the loop; all it can establish is that |x| is
not bound internally.  Hence this code will not signal an
error and will use $3$ as the initial value of the reduction.

The following list summarizes all cases that are subject to these code
motion and variable-shadowing problems.
\begin{itemize}
\item Any variable for which \iter\ creates a binding, including those
used in |with| and the |into| keyword of many clauses.

\begin{sloppypar}
\item The special clauses which place code: |initially|, |after-each|, |else|,
|finally| and |finally-protected|. 
\end{sloppypar}

\item The variables of a |next| or |do-next| form.

\item The |initially| arguments of |for\dots initially\dots then| and
|for\dots previous|. 

\item The |then| argument of |for\dots initially\dots then|.

\item The |initial-value| arguments of |reducing| and |accumulate|.

\item The |on-failure| argument of |finding\dots such-that|.
\end{itemize}

%%%%%%%%%%%%%%%%%%%%%%%%%%%%%%%%%%%%%%%%%%%%%%%%%%%%%%%%%%%%%%%%
\section{Differences Between {\tt Iterate} and {\tt Loop}}
|loop| contains a great deal of complexity which \iter\ tries to
avoid.  Hence many esoteric features of |loop| don't exist in \iter.
Other features have been carried over, but in a cleaned-up form.  
And of course, many new features have been added; they are not
mentioned in this list.

\begin{itemize}

\item \iter's syntax is more Lisp-like than |loop|'s, having a higher
density of parens.

\item The current implementation of \iter, unlike the current version
of  |loop| (as documented in {\em Common Lisp, 2nd Ed.\/}), is
extensible (see section \ref{extend}).  

\item |loop| puts the updates of all driver variables at the top of
the loop; \iter\ leaves them where the driver clauses appear.

\item While for the most part \iter\ clauses that resemble |loop| clauses
behave similarly, there are some differences.  For instance, there is
no |for\dots =\dots then| in \iter; instead use 
|for\dots initially\dots then|.  

\item |loop| binds the variable |it| at certain times
to allow pseudo-English expressions like |when ~expr~ return it|. 
In \iter, you must bind ~expr~ to a variable yourself.  Note that
|when ~expr~ return it| is like |thereis ~expr~| except that the latter is an
accumulation clause and therefore competes with other accumulations
(remember section \ref{multiple} above).

% repeat different behaviour of |always| clause here?

\item |loop| has a special |return| clause, illustrated in the
previous item.  \iter\ doesn't need one,  since an ordinary Lisp
|return| has the same effect.

\item |loop| allows for parallel binding and stepping of iteration
variables.  \iter\ does not.  (See section \ref{bindings}.)

\item |loop| and \iter\ handle variable type declarations very
differently.  |loop| provides a special syntax for declaring variable
types, and does not examine declarations.  Moreover, the standard
implementation of |loop| will
generate declarations when none are requested.
\iter\ parses standard Common Lisp type declarations, and will never
declare a variable itself unless declarations are 
specifically requested.

\end{itemize}

%%%%%%%%%%%%%%%%%%%%%%%%%%%%%%%%%%%%%%%%%%%%%%%%%%%%%%%%%%%%%%%%
\section{Rolling Your Own}
\label{extend}

\subsection{Introduction}

\iter\ is extensible---you can write new clauses that embody new
iteration patterns.  You might want to write a new driver clause for a
data structure of your own, or you might want to write a clause that
collects or manipulates elements in a way not provided by \iter. 

This section describes how to write clauses for \iter.  Writing a
clause is like writing a macro.  In fact, writing a clause ~is~
writing a macro: since \iter\ code-walks its body and macroexpands,
you can add new abstractions to \iter\ with good old |defmacro|.

Actually, there are two extensions you can make to \iter\ that are
even easier than writing a macro.  They are adding a synonym for an
existing clause and defining a driver clause for an indexable
sequence.  These can be done with |defsynonym| and
|defclause-sequence|, respectively.  See section \ref{aids}, below.

The rest of this section explains how to write macros that expand into
\iter\ clauses.
Here's how you could add a simplified version of \iter's |multiply|
clause, if \iter\ didn't already have one:
\begin{program}
(defmacro multiply (expr)
  `(reducing ,expr by \#'* initial-value 1))
\end{program}

If you found yourself summing the square of an expression often, you
might want to write a macro for that.  A first cut might be
\begin{program}
(defmacro sum-of-squares (expr)
  `(sum (* ,expr ,expr)))
\end{program}
but if you are an experienced macro writer, you will realize that this
code will evaluate ~expr~ twice, which is probably a bad idea.  A
better version would use a temporary:
\begin{program}
(defmacro sum-of-squares (expr)
  (let ((temp (gensym)))
    `(let ((,temp ,expr))
       (sum (* ,temp ,temp)))))
\end{program}
Although this may seem complex, it is just the sort of thing you'd
have to go through to write any macro, which illustrates the point of
this section: if you can write macros, you can extend \iter.

Our macros don't use \iter's keyword-argument
syntax.  We could just use keywords with |defmacro|, but we would
still not be using \iter's clause indexing mechanism.  Unlike Lisp,
which uses just the first symbol of a form to determine what function
to call, \iter\ individuates clauses by the list of required keywords.
For instance, |for\dots in| and |for\dots in-vector| are different clauses
implemented by distinct Lisp functions.  

To buy into this indexing scheme, as well as the keyword-argument
syntax, use |defmacro-clause|:

\begin{clauses}

\defmacro{defmacro-clause}{arglist |\&body| body}
Defines a new \iter\ clause.  ~arglist~ is a list of symbols which are
alternating keywords and arguments.  \opt\ may be used, and the list
may be terminated by |\&sequence|.  ~body~ is an ordinary macro body,
as with |defmacro|.  If the first form of ~body~ is a string, it is
considered a documentation string and will be shown by
|display-iterate-clauses|. |defmacro-clause| will signal an error if 
defining the clause would result in an ambiguity.  E.g. you cannot
define the clause |for\dots from| because there would be no way to
distinguish it from a use of the |for| clause with optional keyword |from|.

\end{clauses}

\medskip

Here is |multiply| using |defmacro-clause|.  The keywords are capitalized
for readability.
\begin{program}
(defmacro-clause (MULTIPLY expr \opt\ INTO var)
  `(reducing ,expr by \#'* into ,var initial-value 1))
\end{program}
You don't have to worry about the case when |var| is not supplied; for
any clause with an |into| keyword, saying |into nil| is equivalent to
omitting the |into| entirely.

As another, more extended example, consider the fairly common
iteration pattern that 
involves finding the sequence element that maximizes (or minimizes) some
function.  \iter\ provides this as |finding\dots maximizing|, but it's
instructive to see how to write it.
Here, in pseudocode, is how you might write such a loop for
maximizing a function F:

\begin{quote}
\begin{tabbing}
set variable MAX-VAL to NIL; \\
set variable WINNER to NIL; \\
for\=\ each element EL in the sequence \\
\>  if\=\ MAX-VAL\=\ is NIL or F(EL) $>$ MAX-VAL then \\
\>\>    set MAX-VAL to F(EL); \\
\>\>    set WINNER to EL; \\
\>  end if; \\
end for; \\
return WINNER.
\end{tabbing}
\end{quote}

Here is the macro:
\begin{program}
(defmacro-clause (FINDING expr MAXIMIZING func \opt\ INTO var)
  (let ((max-val (gensym))
        (temp1 (gensym))
        (temp2 (gensym))
        (winner (or var iterate::*result-var*)))
    `(progn 
       (with ,max-val = nil)
       (with ,winner = nil)
       (cond 
        ((null ,max-val)
         (setq ,winner ,expr)
         (setq ,max-val (funcall ,func ,winner))
        (t
         (let* ((,temp1 ,expr)
                (,temp2 (funcall ,func ,temp1)))
           (when (> ,temp2 ,max-val)
             (setq ,max-val ,temp2)
             (setq ,winner ,temp1))))))
       (finally (leave ,winner)))))
\end{program}
Note that if no |into| variable is supplied, we use
|iterate::*result-var*|, which contains the internal variable into
which all clauses place their results.  If this variable is bound by
some clause, then \iter\ will return its value automatically;
otherwise, \nil\ will be returned.  

\subsection{Writing Drivers}

In principle, drivers can be implemented just as easily as other
\iter\ clauses.  In practice, they are a little harder to get right. 
As an example, consider writing a driver that
iterates over all the 
elements of a vector, ignoring its fill-pointer.  |for\dots in-vector|
won't work for this, because it observes the fill-pointer.  It's necessary to
use |array-dimension| instead of |length| to obtain the size of the
vector.  Here is one approach:

\begin{program}
(defmacro-clause (FOR var IN-WHOLE-VECTOR v)
  "All the elements of a vector (disregards fill-pointer)"
  (let ((vect (gensym))
        (index (gensym)))
    `(progn
       (with ,vect = ,v)
       (for ,index from 0 below (array-dimension ,vect 0))
       (for ,var = (aref ,vect ,index)))))
\end{program}
Note that we immediately put |v| in a variable, in case it is an
expression.  Again, this is just good Lisp macrology.  It also has a
subtle effect on the semantics of the driver: |v| is evaluated only
once, at the beginning of the loop, so changes to |v| in the loop have
no effect on the driver.  Similarly, the bounds for numerical iteration
e.g. the above |array-dimension| are also evaluated once only.  This is how
all of \iter's drivers work.


There is an important point concerning the |progn| in this code.  We
need the |progn|, of course, because we are returning several forms,
one of which is a driver.  But \iter\ drivers must occur at top-level.
Is this code in error?  No, because ~top-level~ is defined in \iter\
to include forms inside a |progn|.  This is just the definition of
top-level that Common Lisp uses, and for the same reason: to allow
macros to return multiple forms at top-level.

While our |for\dots in-whole-vector| clause will work, it is not
ideal.  In particular, it does not support generating.  Do do so, we
need to use |for\dots next| or |for\dots do-next|.  The job is
simplified by the |defmacro-driver| macro.

\begin{clauses}

\defmacro{defmacro-driver}{arglist |\&body| body}
Defines a driver clause in
both the |for| and |generate| forms, and provides a parameter
|generate| which ~body~ can examine to determine how it was invoked.
~arglist~ is as in |defmacro-clause|, and should begin with the symbol
|for|. 

\end{clauses}

With |defmacro-driver|, our driver looks like this:
\begin{program}
(defmacro-driver (FOR var IN-WHOLE-VECTOR v)
  "All the elements of a vector (disregards fill-pointer)"
   (let ((vect (gensym))
         (end (gensym))
         (index (gensym))
         (kwd (if generate 'generate 'for)))
     `(progn
        (with ,vect = ,v)
        (with ,end = (array-dimension ,vect 0))
        (with ,index = -1)
        (,kwd ,var next (progn (incf ,index)
                               (if (>= ,index ,end) (terminate))
                               (aref ,vect ,index))))))
\end{program}


We are still missing one thing: the |\&sequence| keywords.  
We can get them easily enough, by writing 
\begin{program}
(defmacro-driver (FOR var IN-WHOLE-VECTOR v \&sequence) 
  ...)
\end{program}
We can now refer to parameters |from|, |to|, |by|, etc. which contain
either the values for the corresponding keyword, or \nil\ if the
keyword was not supplied.  Implementing the right code for these
keywords is cumbersome but not difficult; it is left as an exercise.
But before you begin, see |defclause-sequence| below for an easier way.
       
\subsection{Extensibility Aids}
\label{aids}

This section documents assorted features that may be of use in
extending \iter.

\begin{clauses}

\defunexpvar{*result-var*}
Holds the variable that is used to return a value as a result of the
\iter\ form.  You may examine this and use it in a |with| clause, but
you should not change it.

\defmacro{defsynonym}{syn word}
 Makes ~syn~ a synonym for the existing \iter\ keyword ~word.~  Only
the first word in each clause can have synonyms. 

\defmacroxx{defclause-sequence}{element-name index-name}{|\&key|
access-fn size-fn sequence-type}{element-type
element-doc-string index-doc-string}  
Provides
a simple way to define sequence clauses.  Generates two
clauses, one for iterating over the sequence's elements, the other
for iterating over its indices.  The first symbol of both
clauses will have print-name |for|.
~element-name~  and ~index-name~ should be symbols.
~element-name~ is the second keyword of the element iterator (typically of
the form 
|in-~sequence-type~|), and ~index-name~ is the second keyword
of the index-iterator (typically of the form
|index-of-~sequence-type~|).  Either name may be 
\nil, in which case the corresponding clause is not defined.  If both
symbols are supplied, they should be in the same package.  The |for|
that begins the clauses will be in this package.

\cpar ~access-fn~ is the function to be used to
access elements of the sequence in the element iterator.  The function
should take two 
arguments, a sequence and an index, and return the appropriate element.
~size-fn~ should denote a function of one argument, a sequence, that
returns its size.  Both ~access-fn~ and ~size-fn~ are required for the
element iterator, but only ~size-fn~ is needed for the index iterator.

\cpar The ~sequence-type~ and ~element-type~ keywords are used to
suggest types for the variables 
used to hold the sequence and the
sequence elements, respectively.  The usual rules about \iter's
treatment of variable type declarations apply (see section \ref{types}).

\cpar ~element-doc-string~ and ~index-doc-string~ are
the documentation strings, for use with |display-iterate-clauses|. 

\cpar The generated element-iterator performs destructuring on the
element variable.

\cpar As an example, the above |for\dots in-whole-vector| example
could have been written:
\begin{program}
(defclause-sequence IN-WHOLE-VECTOR INDEX-OF-WHOLE-VECTOR
  :access-fn 'aref
  :size-fn \#'(lambda (v) (array-dimension v 0))
  :sequence-type 'vector
  :element-type t
  :element-doc-string 
     "Elements of a vector, disregarding fill-pointer"
  :index-doc-string 
     "Indices of vector, disregarding fill-pointer")
\end{program}

\end{clauses}

\subsection{Subtleties}

There are some subtleties to be aware of when writing \iter\ clauses.
First, the code returned by your macros may be |nconc|'ed into a list,
so you should always returned freshly consed lists, rather than
constants.  Second, \iter\ matches clauses by using |eq| on the first
symbol and |string=| on the subsequent ones, so the package of the
first symbol of a clause is relevant.  All of the clauses in this manual
have their first word in the \iter\ package.  
You can use the package system in the usual way to shadow
\iter\ clauses without replacing them.

%%% say more here, about the badness that only the first word of a
%%% clause is packagey.

\section{Non-portable Extensions to Iterate (Contribs)}
\label{contribs}

Currently, there is only one non-portable extension to iterate in the
distribution: iterate-pg. If you have made an extension that depends
on non-portable features, feel free to send them to |asf@boinkor.net|
for inclusion in the iterate distribution.

\subsection{An SQL query driver for iterate}

The pg package by Eric Marsden (see |http://cliki.net/pg|) provides an
interface to the PostgreSQL database. Using the \iterpg\ extension, it
is possible to handle the results of SQL queries using \iter.

This usage example should give you an idea of how to use it:

\begin{program}
(pg:with-pg-connection (c "somedb" "someuser")
  (iter (for (impl version date) in-relation "select * from version"
                                 on-connection *dbconn*)
        (collect version)))
\end{program}

To use the extension via |ASDF|, simply make your system depend on the
|iterate-pg| system instead of the |iterate| system. To load it
manually, use:

\begin{program}
  (asdf:oos 'asdf:load-op :iterate-pg)
\end{program}

%%%%%%%%%%%%%%%%%%%%%%%%%%%%%%%%%%%%%%%%%%%%%%%%%%%%%%%%%%%%%%%%
\section{Obtaining {\tt Iterate}}

\begin{sloppypar}
The information in this chapter is obsolete but included for
completeness's sake; Currently, the most up-to-date information on
\iter\ can be found at |http://boinkor.net/iterate.html|.
\end{sloppypar}

\begin{sloppypar}
\iter\ currently runs on Lisp Machines, and on
HP's, Sun3's and Sparcstations under Lucid.  
\iter\ source and binaries are available at the MIT AI Lab in the
subdirectories of |/src/local/lisplib/|.  The source file,
|iterate.lisp|, is also available for anonymous FTP in the directory
|/com/fpt/pub/| on the machine |TRIX.AI.MIT.EDU| (Internet number
128.52.32.6).  If you are unable to obtain |iterate| in one of these
ways, send mail to |jba@ai.mit.edu| and I will send you the source
file. 
\end{sloppypar}

\begin{sloppypar}
\iter\ resides in the |iterate| package (nickname |iter|).  Just say
\linebreak |(use-package :iterate)| to make all the necessary symbols
available. 
If a symbol is not exported, it appears in this manual with an
``|iterate::|'' prefix.
\end{sloppypar}

Send bug reports to |bug-iterate@ai.mit.edu|.  The |info-iterate|
mailing list will have notices of changes and problems; to have
yourself added, send mail to |info-iterate-request@ai.mit.edu|.


\medskip

\begin{flushleft}
 \bf Acknowledgements
\end{flushleft}
\smallskip

Richard Waters provided invaluable criticism which spurred me to improve
\iter\ greatly.  As early users, David Clemens, Oren Etzioni and Jeff
Siskind helped ferret out many bugs.

%\begin{theindex}
% The index files must be generated with the genindex program
% or from makeindex iterate-manual.idx which creates iterate-manual.ind.

%\baselineskip\normalbaselineskip
%\advance\baselineskip by -2pt

\input{iterate-manual.ind}

%\underline{Clauses}
%\input{iterate-manual.clindex}
%
%\indexspace
%\underline{Lisp}
%\input{iterate-manual.lispindex}

%\end{theindex}


\end{document}


% arch-tag: "c0f871a6-313c-11d8-abb9-000c76244c24"


%\underline{Clauses}
%% Fix ``.'' spacing in lists
\documentstyle[12pt,aimemo]{article}

\setlength{\textwidth}{6.0in}
\setlength{\oddsidemargin}{0.25in}
\setlength{\evensidemargin}{0.25in}
\setlength{\topmargin}{0pt}
\setlength{\headsep}{24pt}
\setlength{\textheight}{8.5in}

\let\origpar=\par

% somehow, LaTex fucks with things so that the plain Tex \obeyspaces
% doesn't work.  This is the same, except we use an mbox containing
% a space instead of just a space.

\newcommand{\spc}{\mbox{ }}
\newcommand{\tAb}{\mbox{        }}
\def\obspaces{\catcode`\ =\active\catcode`\^^I=\active}
{\obspaces \global\let =\spc\global\let^^I=\tAb}

\newenvironment{program}{\medskip\samepage\tt\obeylines\obspaces}{\medskip}

% use | for \tt
\catcode`\|=13
\newif\ifvbar
\def|{\ifvbar
         \endgroup\vbarfalse
      \else 
         \begingroup\tt\vbartrue 
      \fi}



\newcommand{\lisp}{\tt}
\newcommand{\iter}{|iterate|}
\newcommand{\iterpg}{|iterate-pg|}
\newcommand{\setf}{|setf|}
\newcommand{\nil}{|nil|}
\newcommand{\nonnil}{non-\nil}
\newcommand{\opt}{{\lisp \&optional}}
\newcommand{\yields}{$\Rightarrow$}



\newbox\Dots
\setbox\Dots=\hbox{\small ...}
\renewcommand{\dots}{\copy\Dots}

% Here's a way to define an environment which will treat all
% paragraphs inside it in the same manner.  It's, as near as I can
% figure, the way that Latex does things.  Using \everypar won't work;
% apparently Latex resets that a lot.  What you have to do is: at the
% top of the file do \let\origpar=\par.  In the
% begin part of the environment, do the parshape (or whatever), then say
% \def\par{{\origpar}}.  The extra braces are necessary.   That's all
% you have to do; at the end of the environment things will return to
% normal automatically.  Why does this all work?  I have no idea.
% Another important thing: there should be a blank line (i.e. a \par)
% between the end of the last paragraph and the \end{environment}.

\newlength{\clindent}
\setlength{\clindent}{0.5in}
\newlength{\clparindent}
\setlength{\clparindent}{\parindent}

\newenvironment{clauses}{\advance\linewidth -\parindent
                        \hangindent\clindent\relax
                         \hangafter=0
                          \parindent=0pt
                          \def\par{{\origpar}}}{}

\newcommand{\cpar}{\hspace \clparindent}

\newcommand{\startitem}{\bigskip\pagebreak[3]}
\newcommand{\finishitem}{\smallskip}

\newcommand{\clause}[1]{\startitem\Clause{#1}\finishitem}
\newcommand{\Clause}[1]{\hbox{#1}}
\newcommand{\clindex}[1]{\index{{\lisp #1}}}
\newcommand{\clindexx}[2]{\index{{\lisp #1\dots #2}}}
\newcommand{\clindexxx}[3]{\index{{\lisp #1\dots #2\dots #3}}}
\newcommand{\clausex}[2]{\startitem\Clausex{#1}{#2}\finishitem}
\newcommand{\Clausex}[2]{\Clause{|#1| #2}\clindex{#1}}

\newcommand{\defvar}[1]{\startitem\deff{#1}{}{Variable}\finishitem}
\newcommand{\defunexpvar}[1]{\startitem\deffnoindex{iterate::#1}{}{Variable}
                                        \lispindex{#1}\finishitem}
\newcommand{\defun}[2]{\startitem\deff{#1}{#2}{Function}\finishitem}
\newcommand{\defmacro}[2]{\startitem\deff{#1}{#2}{Macro}\finishitem}
\newcommand{\defunx}[3]{\startitem\deffx{#1}{#2}{#3}{Function}\finishitem}
\newcommand{\defmacrox}[3]{\startitem\deffx{#1}{#2}{#3}{Macro}\finishitem}
\newcommand{\defunxx}[4]{\startitem\deffxx{#1}{#2}{#3}{#4}{Function}\finishitem}
\newcommand{\defmacroxx}[4]{\startitem\deffxx{#1}{#2}{#3}{#4}{Macro}\finishitem}

\newcommand{\deff}[3]{\deffnoindex{#1}{#2}{#3}\lispindex{#1}}
\newcommand{\deffnoindex}[3]{\hbox to \hsize{{\tt #1} {\it #2}\hfill [{\it #3}]}}
\newcommand{\deffx}[4]{\deff{#1}{#2}{#4}\moreargs{#1}{#3}}
\newcommand{\deffxx}[5]{\deffx{#1}{#2}{#3}{#5}\moreargs{#1}{#4}}

\newcommand{\moreargs}[2]{\hbox to \hsize{\phantom{\lisp #1} {\it #2}\hfill}}

\newcommand{\lispindex}[1]{\index{{\lisp #1}}}

 \newenvironment{note}[1]{\pagebreak[2]\bigskip
                          \hrule\smallskip\small
                         {\setlength{\parindent}{0pt}
                           \par{\bf #1:}}}{\smallskip\hrule\bigskip}



\makeindex

%%%%%%%%%%%%%%%%%%%%%%%%%%%%%%%%%%%%%%%%%%%%%%%%%%%%%%%%%%%%%%%%

%\thispagestyle{empty}

%\vspace*{2in}

%\title{The |Iterate| Manual}
%\author{Jonathan Amsterdam}

%\maketitle

%\thispagestyle{empty}

%\begin{center}
%\Large *** DRAFT ***
%\end{center}
%This manual is in beta-test. It will become an AI Memo in a couple of
%months.  I would appreciate any comments or
%corrections.  You can reach me at email address |jba@ai.ai.mit.edu|.

\begin{document}

\def\writtenby{Jonathan Amsterdam}
\def\contractno{This report describes research done at the Artificial
Intelligence Laboratory of the Massachusetts Institute of Technology.
Support for the laboratory's artificial intelligence research is
provided in part by the Advanced Research Projects Agency of the
Department of Defense under Office
of Naval Research contract N00014-85-K-0124.}

\AIMEMO{1236}{\Large The Iterate Manual \\[1in]}
{This is the manual for version 1.4 of \iter, a powerful iteration
macro for Common  
Lisp. \iter\ is similar to {\tt loop} but provides numerous additional
features, is well integrated with Lisp, and is extensible. 
\vfill }


% use ~ for \em
\catcode`\~=13
\newif\ifem
\def~{\ifem
         \/\endgroup\emfalse
      \else 
         \begingroup\em\emtrue 
      \fi}



%\pagebreak
%\setcounter{page}{1}

%\thispagestyle{empty}


\tableofcontents

\pagebreak
\pagestyle{headings}

\section{Introduction}

\begin{sloppypar}
This manual describes
\iter, a powerful iteration facility for Common Lisp.
\iter\ provides abstractions for many common iteration
patterns and allows for the definition of additional patterns.  \iter\ is
a macro that expands into ordinary Lisp at
compile-time, so it is more efficient than higher-order
functions like |map| and |reduce.|  
While it is similar to |loop|, \iter\ offers a more Lisp-like syntax
and enhanced extensibility.
(For a more complete
comparison of \iter\ with other iteration constructs, see MIT AI Lab Working
Paper No. 324, ~Don't Loop, Iterate.~)
\end{sloppypar}

An \iter\ form consists of the symbol |iter|\footnote{You can also use
|iterate|, but |iter| is preferred because it avoids potential conflicts with
possible future additions to Common Lisp, and because it saves
horizontal space when writing code.}
followed by one or more
forms, some of which may be \iter\ 
~clauses.~  Here is a simple example of \iter\ which collects the
numbers from 1 to 10 into a list, and returns the list.  The return
value is shown following the arrow.
\begin{program}
(iter (for i from 1 to 10)
      (collect i))          \yields (1 2 3 4 5 6 7 8 9 10)
\end{program}
This form contains two clauses: a |for| clause that steps the
variable |i| over the integers from 1 to 10, and a |collect|
clause that accumulates its argument into a list.  With a few
exceptions, all \iter\ clauses have the same format: alternating
symbols (called ~keywords~) and expressions (called
~arguments~).  The syntax and terminology are those of Common
Lisp's keyword lambda lists.  One difference is that \iter's keywords
do not have to begin with a colon---though they may, except for the
first symbol of a clause.  So you can also write  |(for i :from
1 :to 10)| if you prefer.

Any Lisp form can appear in the body of an \iter, where it will have
its usual meaning.  \iter\ walks the entire body, expanding macros,
and recognizing clauses at any level.  This example collects all the
odd numbers in a list:
\begin{program}
(iter (for el in list)
      (if (and (numberp el) (oddp el))
          (collect el)))
\end{program}

There are clauses for iterating over numbers, lists, arrays and other
objects, and for
collecting, summing, counting, maximizing and other useful operations.
\iter\ also supports the creation of new variable bindings, stepping
over multiple sequences at once, destructuring, and compiler
declarations of variable types.  The following example illustrates
some of these features: 
\begin{program}
(iter (for (key . item) in alist)
         (for i from 0)
         (declare (fixnum i))
         (collect (cons i key)))
\end{program}

This loop takes the keys of an alist and returns a new alist
associating the keys with their positions in the original list. The
compiler declaration for |i| will appear in the generated code
in the appropriate place.

%%%%%%%%%%%%%%%%%%%%%%%%%%%%%%%%%%%%%%%%%%%%%%%%%%%%%%%%%%%%%%%%
\section{Clauses}

Most of \iter's clauses will be familiar to |loop| programmers.  
(|loop| is an iteration macro that has been incorporated into Common
Lisp.  Seen Guy Steele's {\em Common Lisp, 2nd Edition}.)
In
nearly all cases they behave the same as their |loop| counterparts, so
a |loop| user can switch to \iter\ with little pain (and much gain).

All clauses with the standard keyword-argument syntax consist of two
parts: a ~required~ part, 
containing keywords that must be present and in the right order; and
an ~optional~ part, containing keywords that may be omitted and,
if present, may occur in any order.  In the descriptions below, the
parts are separated by the Lisp lambda-list keyword \opt.


\subsection{Drivers}

An iteration-driving clause
conceptually causes the iteration to go forward.  Driver clauses in
\iter\ allow iteration over numbers, lists, vectors, hashtables, packages,
files and streams.  Iteration-driving clauses must
appear at the top level of an \iter\ form; they cannot be nested
inside another clause.  The driver variable is updated at the point
where the driver clause occurs.  Before the clause is executed for the
first time, the value of the variable is undefined.

%Also, regardless of where the driver clause
%appears in the body, the driver variable is stepped at the top of the
%loop; hence it is stylistically preferable, though
%not required, to place driver clauses at the beginning of the \iter.

Multiple drivers may appear in a single \iter\ form, in which case all
of the driver variables are updated each time through the loop, in the
order in which the clauses appear.  The first driver to terminate will
terminate the entire loop.

In all cases, the value of the driver variable on exit from the loop,
including within the epilogue code (see the |finally| clause), is
undefined.

All the parameters of a driver clause are evaluated once, before the
loop begins.  Hence it is not possible to change the bounds or other
properties of an iteration by side-effect from within the loop.

With one exception, driver clauses begin with the word |for| (or
the synonym |as|) and mention an iteration variable, which is
given a binding within the \iter\ form.
The exception is |repeat|, which just executes a loop a
specified number of times:

\begin{clauses}

\clausex{repeat}{~n~}
Repeats the loop ~n~ times.  For example:
\begin{program}
(iter (repeat 100)
      (print "I will not talk in class."))
\end{program}
\cpar If $n \leq 0$, the
loop will never be executed.  If ~n~ is not an integer, the
actual number of executions will be $\lceil n \rceil$.

\end{clauses}

\pagebreak[3]

\subsubsection{Numerical Iteration}

\begin{clauses}
\clausex{for}{~var~ |\&sequence|} 
The general form for iterating over a sequence of numbers requires a
variable and, optionally, one or more keywords that provide the bounds
and step size of the iteration. The |\&sequence|
lambda-list keyword is a shorthand for these sequence keywords.  They are:
|from|, |upfrom|, |downfrom|, |to|, |downto|, |above|, |below| and |by|.
|from| provides the starting value for ~var~
and defaults to zero. |to| provides a final value and implies that the
successive values of ~var~ will be increasing; |downto| implies that
they will be decreasing.  The loop terminates when ~var~ passes
the final value (i.e. becomes smaller or larger than it, depending on
the direction of iteration); in other  words, the loop body
will never be 
executed for values of ~var~ past the final value.
|below| and |above| are similar to |to| and
|downto|, except that the loop terminates when ~var~ equals or passes
the final value.

\cpar If no final value is specified, the variable will be stepped
forever.  Using |from| or |upfrom| will result in increasing
values, while |downfrom| will give decreasing values.

\cpar On each iteration,
~var~ is incremented or decremented by the value of the sequence
keyword |by|, which defaults to 1.  It should always be a positive
number, even for downward iterations.

\cpar In the following examples, the sequence of numbers generated is shown
next to the clause.

\begin{program}
(for i upfrom 0) \yields\ 0 1 2 \ldots
(for i from 5)   \yields\ 5 6 7 \ldots    ; either from or upfrom is okay
(for i downfrom 0) \yields\ 0 -1 -2 \ldots
(for i from 1 to 3) \yields\ 1 2 3
(for i from 1 below 3) \yields\ 1 2
(for i from 1 to 3 by 2) \yields\ 1 3
(for i from 1 below 3 by 2) \yields\ 1
(for i from 5 downto 3) \yields\ 5 4 3
\end{program}

\end{clauses}

\subsubsection{Sequence Iteration}

There are a number of clauses for iterating over sequences.  In all of
them, the argument following |for| may be a list instead of a
symbol, in which case destructuring is performed.  See section
\ref{destructuring}. 

\begin{clauses}

\clause{|for| ~var~ |in| ~list~ \opt\ |by| ~step-function~}
\clindexx{for}{in}
~var~ is set to successive elements of list.  
~step-function,~ which 
defaults to |cdr|, is used to obtain the next sublist. 

\clause{|for| ~var~ |on| ~list~ \opt\ |by| ~step-function~}
\clindexx{for}{on}
~var~ is set to successive sublists of list. 
~step-function~ (default |cdr|) is used as in |for\dots in|.

\end{clauses}

\medskip

These two clauses use |atom| to test for the end
of a list.  Hence, given a list whose final |cdr| is not \nil,
they will silently ignore the last |cdr|.  Other choices are
|endp|,  which would signal an error, and |null|, which
would probably result in an error somewhere else.  If you wish to use
an end-test other than |atom|, set the variable
|iterate::*list-end-test*|\lispindex{*list-end-test*} to the name of
the desired function. 

\begin{clauses}
\clause{|for| ~var~ |in-vector| ~vector~ |\&sequence|}
\clindexx{for}{in-vector}
~var~ takes on successive elements from ~vector.~   The vector's
fill-pointer is observed.  Here and in subsequent clauses, the |\&sequence|
keywords include |with-index|, which takes a symbol as argument
and uses it for the index variable instead of an internally generated
symbol.  The other |\&sequence| keywords behave as in numerical
iteration, except that the default iteration bounds are the bounds of
the vector.  E.g. in |(for i in-vector v downto 3)|,
|i| will start off being bound to the last element in |v|,
and will  
be set to preceding elements down to and including the element with
index 3.

\clause{|for| ~var~ |in-sequence| ~seq~ |\&sequence|}
\clindexx{for}{in-sequence}
This uses Common Lisp's generalized sequence functions, |elt|
and |length|, to obtain elements and determine the length of
~seq.~   Hence it will work for any sequence, including lists, and will
observe the fill-pointers of vectors.

\clause{|for| ~var~ |in-string| ~string~ |\&sequence|}
\clindexx{for}{in-string}
~var~ is set to successive characters of ~string.~

\startitem
\Clause{|for| ~var~ |index-of-vector| ~vector~ |\&sequence|}
\clindexx{for}{index-of-vector}
\Clause{|for| ~var~ |index-of-sequence| ~sequence~ |\&sequence|}
\clindexx{for}{index-of-sequence}
\Clause{|for| ~var~ |index-of-string| ~string~ |\&sequence|}
\clindexx{for}{index-of-string}
\finishitem
~var~ is set to successive indices of the sequence.
These clauses avoid the overhead of accessing the sequence elements
for those applications where they do not need to be examined, or are
examined rarely.  They admit all the optional keywords of the other
sequence drivers except the (redundant) |with-index| keyword.

\clause{|for| ~(key value)~ |in-hashtable| ~table~}
\clindexx{for}{in-hashtable}

~key~ and ~value,~ which must appear as shown in a list
and may be destructuring templates, are set to the keys and values of
~table~.  If ~key~ is |nil|, then the hashtable's keys will be ignored;
similarly for ~value~.
The order in which 
elements of ~table~ will be retrieved is unpredictable.

\clause{|for| ~var~ |in-package| ~package~ \opt\ |external-only| ~ext~}
\clindexx{for}{in-package}
Iterates over all the symbols in ~package,~ or over only the
external symbols if ~ext~ is specified and non-|nil|.  ~ext~ is not
evaluated.  The same symbol may appear more than once.

\clause{|for| ~var~ |in-packages| ~packages~ \opt\ |having-access| ~symbol-types~}
\clindexx{for}{in-packages}
Iterates over all the symbols from the list of packages denoted by the
descriptor ~packages~ and having accessibility (or visibility) given by
~symbol-types~.  This defaults to the list |(:external :internal :inherited)|
and is not evaluated.  ~var~ must be a list of up to three variables: in each
iteration, these will be set to a symbol, its access-type and package (as per
|with-package-iterator| in ANSI-CL).  The same symbol may appear more than
once.

\end{clauses}

\begin{clauses}

\clause{|for| ~var~ |in-file| ~name~ \opt\ |using| ~reader~}
\clindexx{for}{in-file}
Opens the file ~name~ (which may be a string or pathname) for input,
and iterates 
over its contents. ~reader~ defaults to |read|, so by default ~var~
will be bound to the successive forms in the file.  The \iter\ body is
wrapped in an unwind-protect to ensure that the file is closed no
matter how the \iter\ is exited.

\clause{|for| ~var~ |in-stream| ~stream~ \opt\ |using| ~reader~}
\clindexx{for}{in-stream}
Like |for\dots in-file|, except that ~stream~ should be an existing
stream object that supports input operations.

\end{clauses}

\subsubsection{Generalized Drivers}

These are primarily useful for writing drivers that can also be used
as generators (see section \ref{generators}, below).
 
\begin{clauses}

\clause{|for| ~var~ |next| ~expr~}
\clindexx{for}{next}

~var~ is set to ~expr~ each time through the loop.  Destructuring is
performed.  When the clause is used as a generator, ~expr~ is the code
that is executed when |(next ~var~)| is 
encountered (see section \ref{generators}, below).
~expr~ should compute the first value for ~var~, as well
as all subsequent values, and is responsible for terminating the loop.
For compatibility with future versions of \iter, this termination
should be done with |terminate|\clindex{terminate}, which can be
considered a synonym for |finish| (see section \ref{control-flow}).

\cpar As an example, the following clauses are equivalent to |(for
i from 1 to 10)|:
\begin{program}
(initially (setq i 0))
(for i next (if (> i 10) (terminate) (incf i)))
\end{program}

\clause{|for| ~var~ |do-next| ~form~}
\clindexx{for}{do-next}
~form~ is evaluated each time through the loop.  Its value is ~not~
set to ~var~; that is ~form's~ job. ~var~ is only present so that
\iter\ knows it is a driver variable.  \linebreak |(for ~var~ next
~expr~)| is equivalent to |(for ~var~ do-next (dsetq ~var~ ~expr~))|.
(See section \ref{destructuring} for an explanation of |dsetq|.) 

\end{clauses}

\subsubsection{Generators}
\label{generators}

In all of the above clauses, the driver variable is updated on each
iteration.
Sometimes it is desirable to have greater control over updating.  
For instance, consider the problem of associating numbers, in
increasing order and with no gaps, with the
\nonnil\ elements of a list.  One obvious first pass at writing this is:

\begin{program}
(iter (for el in list)
      (for i upfrom 1)
      (if el (collect (cons el i))))
\end{program}
But on the list |(a b nil c)| this produces |((a . 1) (b . 2) (c .
4))| instead of the desired |((a . 1) (b . 2) (c . 3))|.  The problem
is that |i| is incremented each time through the loop, even when |el|
is |nil|.  

The problem could be solved elegantly if we could step |i| only when
we wished to.  This
can be accomplished for any \iter\ driver by writing |generate| (or
its synonym |generating|)
instead of |for|.  Doing so produces a ~generator~---a driver whose
values are yielded explicitly.  To obtain the next value of a
generator variable ~v~, write \linebreak |(next ~v~)|.  The value of a |next|
form is the next value of ~v~, as determined by its associated driver
clause.  |next| also has the side-effect of updating ~v~ to that
value.  If there is no next value, |next| will terminate the loop,
just as with a normal driver.


Using generators, we can now write our example like this:
\begin{program}
(iter (for el in list)
      (generate i upfrom 1)
      (if el (collect (cons el (next i)))))
\end{program}
Now |i| is updated only when |(next i)| is executed, and this occurs
only when |el| is \nonnil.

To better understand the relationship between ordinary drivers and
generators, observe that we can rewrite an ordinary driver using its
generator form immediately followed by |next|, as this example shows:
\begin{program}
(iter (generating i from 1 to 10)
      (next i)
      ...)
\end{program}
Provided that the loop body contains no |(next i)| forms, this will
behave just as if we had written |(for i from 1 to 10)|.

We can still refer to a driver variable ~v~ without using |next|; in
this case, its value is that given to it by the last evaluation of
|(next ~v~)|.  Before |(next ~v~)| has been called the first time, the
value of ~v~ is undefined.  

This semantics is more flexible than
one in which ~v~ begins the loop bound to its first value and calls
of |next| supply subsequent values, because it means the loop will not
terminate too soon if the generator's sequence is empty.  For
instance, consider the following code, which tags \nonnil\ elements of
a list using a list of tags, and also counts the null elements.
(We assume there are at least as many tags as \nonnil\ elements.)
\begin{program}
(let* ((counter 0)
       (tagged-list (iter (for el in list)
                          (generating tag in tag-list) 
                          (if (null el)
                              (incf counter)
                              (collect (cons el (next tag)))))))
  ...)
\end{program}
It may be that there are just as many tags as non-null elements of
|list|.  If all the elements of |list| are null, we still want the
counting to proceed, even though |tag-list| is \nil.  If |tag| had to be
assigned its first value before the loop begins, we would have had to
terminate the loop before the first iteration, since when |tag-list|
is \nil, |tag| 
has no first value.  With the existing semantics, however, |(next
tag)| will never execute, so the iteration will cover all the elements of
|list|. 
        
When the ``variable'' of a driver clause is actually a destructuring
template containing several variables, all the variables are eligible
for use with |next|.  As before, |(next ~v~)| evaluates to ~v's~ next
value; but the effect is to update all of the template's variables.
For instance, the following code will return the list |(a 2 c)|.
\begin{program}
(iter (generating (key . item) in '((a . 1) (b . 2) (c . 3)))
      (collect (next key))
      (collect (next item)))
\end{program}

Only driver clauses with variables can be made into generators.  This
includes all clauses mentioned so far except for |repeat|.  It does
~not~ include |for\dots previous|, |for\dots =|,
|for\dots initially\dots then| or |for\dots first\dots then| (see
below). 

\subsubsection{Previous Values of Driver Variables}

Often one would like to access the value of a variable on a previous
iteration.  \iter\ provides a special clause for accomplishing this.

\begin{clauses}

\clause{|for| ~pvar~ |previous| ~var~ \opt\ |initially| ~init~
|back| ~n~}
\clindexx{for}{previous}
Sets ~pvar~ to the previous value of ~var,~ which should be a driver
variable, a variable from another |for\dots previous| clause, or a
variable established by a |for\dots =|, 
|for\dots initially\dots then| or |for\dots first\dots then| clause
(see section \ref{setting}).
Initially, ~pvar~ is given the value ~init~ (which defaults to \nil).  
The ~init~ expression will be moved outside the loop body, so it
should not depend on anything computed within the loop.
~pvar~ retains the value of ~init~ until ~var~ is set to its second
value, at which point ~pvar~ is set to ~var's~ first value; and so on.  

\cpar The
argument ~n~ to |back| must be a constant, positive integer, and
defaults to 1.  It determines how many iterations back ~pvar~ should
track ~var~.  For example, when ~n~ is 2, then ~pvar~ will be assigned
~var's~ first value when ~var~ is set to its third value.

\cpar A |for\dots previous| clause may occur after or before its
associated driver clause. |for\dots previous| works with generators as
well as ordinary drivers.

\pagebreak[3]

\cpar Example:
\begin{program}
(iter (for el in '(1 2 3 4))
      (for p-el previous el)
      (for pp-el previous p-el initially 0)
      (collect pp-el))
\end{program}
This evaluates to |(0 0 1 2)|.  It could have been written more
economically as 
\begin{program}
(iter (for el in '(1 2 3 4))
      (for pp-el previous el back 2 initially 0)
      (collect pp-el))
\end{program}


\end{clauses}

\subsection{Variable Binding and Setting}
\label{setting}

Several clauses exist for establishing new variable bindings or for
setting variables in the loop.  They all support destructuring.

\begin{clauses}

\clausex{with}{~var~ \opt\ |=| ~value~}
Causes ~var~ to be bound to value before the loop body is entered.
If ~value~ is not supplied, ~var~ assumes a default
binding, which will be 
\nil\ in the absence of declarations.  Also, if ~value~ is not
supplied, no destructuring is performed; instead, ~var~ may be a list of
symbols, all of which are given default 
bindings.  If ~value~ is supplied, ~var~ is bound to it, with
destructuring.  

\cpar Because |with|
creates bindings whose scope includes the entire \iter\ form, it is
good style to put all |with| clauses at the beginning.

\cpar Successive occurrences of |with| result in sequential
bindings (as with 
|let*|).  There is no way to obtain parallel bindings; see
section \ref{bindings} for a rationale.  


\clause{|for| ~var~ |=| ~expr~}
\clindexx{for}{=}
On each iteration, ~expr~ is evaluated and ~var~ is set
to its value. 

\cpar This clause may appear to do the same thing as |for\dots next|.
In fact, they are quite different.  |for\dots =| provides only three
services: it sets up a binding for ~var~, sets it to ~expr~ on each
iteration, and makes it possible to use |for\dots previous| with
~var~.  |for\dots next| provides these services in addition to the
ability to turn the driver into a generator.
%Also, the code which sets ~var~ appears in the loop body in the same
%place as the |for\dots =| clause; the code for |for\dots next| appears
%at the top of the loop, as with other drivers (except when being used
%as a generator).

\clause{|for| ~var~ |initially| ~init-expr~ |then| ~then-expr~}
\clindexxx{for}{initially}{then}
Before the loop begins, ~var~ is set to ~init-expr;~ on all
iterations after the first it is set to ~then-expr.~  This clause
must occur at top-level.  ~init-expr~ will be moved outside the loop
body and ~then-expr~ will be moved to the end of the loop body, so
they are subject to code motion problems (see section
\ref{code-movement}). 

\cpar This clause may appear to be similar to |for\dots next|, but in
fact they differ significantly.  |for\dots initially\dots then| is
typically used to give ~var~ its first value before the loop begins,
and subsequent values on following iterations.  This is incompatible
with generators, whose first value and subsequent values must all be
computed by |(next ~var~)|.  Also, the update of ~var~ in 
|for\dots initially\dots then| does not occur at the location of the
clause. 
Use |for\dots initially\dots then| for
one-shot computations where its idiom is more convenient, but use
|for\dots next| for extending \iter\ with new drivers (see section
\ref{extend}).

\clause{|for| ~var~ |first| ~first-expr~ |then| ~then-expr~}
\clindexxx{for}{first}{then}
The first time through the loop, ~var~ is set to ~first-expr;~ on
subsequent iterations, it is set to ~then-expr.~  This differs from
|for\dots initially| in that ~var~ is set to ~first-expr~
inside the loop body, 
so ~first-expr~ may depend on the results of other clauses.  For
instance, 
\begin{program}
(iter (for num in list)
      (for i first num then (1+ i))
      ...)
\end{program}
will set |i| to the first element of |list| on the first
iteration, whereas 
\begin{program}
(iter (for num in list)
      (for i initially num then (1+ i))
      ...)
\end{program}
is probably erroneous; |i| will be bound to |num|'s
default binding (usually \nil) for the first iteration.

\end{clauses}

\begin{note}{Compatibility note}
|loop|'s |for\dots =| works like \iter's, but |loop| used the syntax
|for\dots =\dots then| to mean |for\dots initially\dots then|.  It was
felt that these two operations were sufficiently different to warrant
different keywords.

Also, the |for| in the above three clauses is misleading,
since none is true driver (e.g. none has a corresponding |generate|
form).  |setting| would have been a better choice, but |for| was used
to retain some compatibility with |loop|.  
\end{note}


%%%%%%%%%%%%%%%%%%%%%%%%%%%%%%%%
\subsection{Gathering Clauses}
Many of \iter's clauses accumulate values into a variable, or set a
variable under certain conditions.  At the end of the loop, this
variable contains the desired result.  All these clauses have an
optional |into| keyword, whose argument should be a symbol.  If the
|into| keyword is not supplied, the accumulation variable will be
internally generated and its value will be returned at the end of the
loop; if a variable is specified, that variable is used for the
accumulation, and is not returned as a result---it is up to the user
to return it explicitly, in the loop's epilogue code (see |finally|).
It is safe to examine the accumulation variable during the loop, but
it should not be modified.

These clauses all begin with a verb.  When the verb does
not conflict with an existing Common Lisp function, then it may be
used in either its infinitival or present-participle form (e.g. |sum|,
|summing|).  However, when there is a conflict with Common Lisp, only
the present-participle form may be used (e.g. |unioning|).  This is to
prevent \iter\ clauses from clashing with Common Lisp
functions. 

%although these clauses are described as ``producing a value,'' it is a
%mistake to think of the lisp list representing the clause as a
%value-producing form in the usual way.  clauses may legally be written
%where a value is expected, e.g. |(setq x (sum i))|, but the
%lisp value of a clause in such a context is undefined.

\subsubsection{Reductions}

~Reduction~ is an extremely common iteration pattern in which
the results of successive applications of a
binary operation are accumulated.
For example, a loop that computes the sum of the
elements of a list is performing a reduction with the addition
operation.  This could be written in Common Lisp as  |(reduce \#'+
list)| or with \iter\ as
\begin{program}
(iter (for el in list)
      (sum el))
\end{program}

\begin{clauses}

\clausex{sum}{~expr~ \opt\ |into| ~var~}
Each time through the loop, ~expr~ is evaluated and added to a
variable, which is bound initially to zero.  If ~expr~ has a type,
it is ~not~ used as the type of the sum variable, which is always
|number|.  To get the result variable to be of a more specific
type, use an explicit variable, as in
\begin{program}
(iter (for el in number-list)
      (sum el into x)
      (declare (fixnum x))
      (finally (return x)))
\end{program}

\clausex{multiply}{~expr~ \opt\ |into| ~var~}
Like |sum|, but the initial value of the result variable is $1$,
and the variable is updated by multiplying ~expr~ into it.

\clausex{counting}{~expr~ \opt\ |into| ~var~}
~expr~ is evaluated on each iteration.  If it is \nonnil, the
accumulation variable, initially zero, is incremented.

\startitem
\Clausex{maximize}{~expr~ \opt\ |into| ~var~}
\Clausex{minimize}{~expr~ \opt\ |into| ~var~}
\finishitem
~expr~ is evaluated on each iteration and its extremum (maximum or
minimum) is stored in the accumulation variable.  If ~expr~ is never
evaluated, then the result is \nil\ (if the accumulation variable is
untyped) or $0$ (if it has a numeric type).

\clausex{reducing}{~expr~ |by| ~func~ \opt\ |initial-value| ~init-val~ |into| ~var~}
This is a general way to perform reductions. ~func~ should be a
function of two arguments, the first of which will be the value
computed so far and
the second of which will be the value of ~expr~.  It should return the
new value.
|reducing| is roughly equivalent to the
Common Lisp |(reduce ~func~ ~sequence~ :key ~expr-function~)|, where
~expr-function~ is used to derive values from the successive elements
of ~sequence~.

\cpar If the |reducing| clause is never executed, the result is
undefined.

\cpar It is not necessary to provide an
initial value, but better code can be generated if one is supplied.
Regardless of its location in the \iter\ body, ~init-val~ will be
evaluated before the loop is entered, so it should not depend on any
value computed inside the \iter\ form.  

%\cpar if a ~var~ is not specified, you can get \iter\ to declare the
%type of the internal variable by putting a |the| expression around
%~func.~  see section \ref{types}.

\end{clauses}

%\begin{note}{implementation note}
%in principle, |maximize| and |minimize| can be thought of
%as reductions where the initial value is the smallest (or largest)
%value that the accumulation variable can assume.  because lisp's
%bignums can represent arbitrary integers, these clauses cannot be
%implemented as reductions in general.  if, however, the type of 
%~expr~ or ~var~ can be determined to be a fixnum
%or a float, \iter\ will implement the clause as a true reduction,
%using one of the constants |most-negative-fixnum|, |%most-positive-fixnum|, 
%|most-negative-short-float|, etc. as appropriate.
%\end{note}

\subsubsection{Accumulations}

All the predefined accumulation clauses add values to a sequence.  If
the sequence is a list, they all 
have the property that the partial list is kept in the correct order
and available for inspection at any point in the loop.

\begin{clauses}

\clausex{collect}{~expr~ \opt\ |into| ~var~ |at| ~place~ |result-type| ~type~}
Produces a sequence of the values of ~expr~ on each iteration. ~place~
indicates where the next value of ~expr~ is added to the list and may
be one of the symbols |start|, |beginning| (a synonym for |start|) or
|end|.  The symbol may be quoted, but need not be.  The default is
|end|.  For example,
\begin{program}
(iter (for i from 1 to 5)
      (collect i))
\end{program}
produces |(1 2 3 4 5)|, whereas
\begin{program}
(iter (for i from 1 to 5)
      (collect i at beginning))
\end{program}
produces |(5 4 3 2 1)| (and is likely to be faster in most Common Lisp
implementations).

\cpar If ~type~ is provided, it should be a subtype of |sequence|.
The default is |list|.  Specifying a type other than |list| will
result in |collect| returning a sequence of that type.  ~However,~ the
type of the sequence being constructed when inside the loop body is
undefined when a non-|list| type is specified.  (As with ~place~,
quoting ~type~ is optional.)

\clausex{adjoining}{~expr~ \opt\ |into| ~var~ |test| ~test~ |at| ~place~ |result-type| ~type~}
Like |collect|, but only adds the value of ~expr~ if it is not
already present.  ~test,~ which defaults to |\#'eql|, is
the test to be used with |member|.

\startitem
\Clausex{appending}{~expr~ \opt\ |into| ~var~ |at| ~place~}
\Clausex{nconcing}{\ ~expr~ \opt\ |into| ~var~ |at| ~place~}
\Clausex{unioning}{\ ~expr~ \opt\ |into| ~var~ |test| ~test~ |at| ~place~}
\Clausex{nunioning}{~expr~ \opt\ |into| ~var~ |test| ~test~ |at| ~place~}
\finishitem
These are like |collect|, but behave like the Common Lisp functions
|append|, |nconc|, |union| or |nunion|.  
As in Common Lisp, they work only on lists.  Also as in Common Lisp,
|unioning| and |nunioning| assume that the value of ~expr~ contains no
duplicates. 

\clausex{accumulate}{~expr~ |by| ~func~ \opt\ |initial-value| ~init-val~ |into| ~var~}
This is a general-purpose accumulation clause. ~func~ should be a
function of two arguments, the value of ~expr~ and the value
accumulated so far in the iteration, and it should return the updated
value.  If no initial value is supplied, \nil\ is used.  

%\cpar If a ~var~ is not specified, you can get \iter\ to declare the
%type of the internal variable by putting a |the| expression around
%~func.~  see section \ref{types}.

\cpar The differences between |accumulate| and |reducing| are slight.
One difference is that the functions take their arguments in a
different order.  Another is that in the absence of ~init-val~,
|accumulate| will use \nil, whereas |reducing| will generate different
code that avoids any dependence on the initial value.
The reason for having both clauses is that one usually
thinks of reductions (like |sum|) and accumulations (like |collect|)
as different beasts.

\end{clauses}


\subsubsection{Finders}

A ~finder~ is a clause whose value is an expression that meets some
condition.

\begin{clauses}

\clause{|finding| ~expr~ |such-that| ~test~ \opt\ |into| ~var~ |on-failure| ~failure-value~}
\clindexx{finding}{such-that}
If ~test~ (which is an expression) ever evaluates to \nonnil, the loop
is terminated, the 
epilogue code is run and the value of ~expr~ is returned.  Otherwise,
\nil\ (or ~failure-value,~ if provided) is returned.  If ~var~ is
provided, it will have either the \nonnil\ value of ~expr~ or
~failure-value~ when the epilogue code is run.

\cpar As a special case, if the ~test~ expression is a sharp-quoted
function, then it is applied to ~expr~ instead of being simply
evaluated.  E.g. |(finding x such-that \#'evenp)| is equivalent to
|(finding x such-that (evenp x))|.  

%\cpar although ~test~ need have nothing to do with ~%expr~ as in
%|(finding j such-that (> i 3))|, it usually 
%will: |(finding (length el) such-that (oddp (length el)))|.  to
%avoid performing the |length| computation twice, you could write
%|(finding (length el) such-that \#'oddp)| or |(finding (length
%el) such-that 'oddp)|; for these cases, \iter\ generates code that
%executes ~expr~ only once.  the code for |\#'oddp|
%is slightly different from that for {\lisp 'oddp}; see the discussion
%under {\lisp for\dots in} and {\lisp for\dots on}.

\cpar |On-failure| is a misnomer. Because it is always evaluated, it behaves
more like the default third argument to the |gethash| function. As a result,
|on-failure (error "Not found")| makes no sense. Instead, the clauses |leave|
or |thereis| can be used in conjunction with |finally| as follows:
\begin{program}
(iter (for x in '(1 2 3))
      (if (evenp x) (leave x))
      (finally (error "not found")))
\end{program}

\cpar This clause may appear multiple times when all defaults are
identical. It can also be used together with either |always|/|never| or
|thereis| if their defaults match. More specifically, |on-failure nil| is
compatible with |thereis|, while |on-failure t| is compatible with |always|
and |never| clauses.
\begin{program}
(iter (for i in '(7 -4 2 -3))
      (if (plusp i)
	  (finding    i  such-that (evenp i))
          (finding (- i) such-that (oddp i))))
\end{program}

\startitem
\Clause{|finding| ~expr~ |maximizing| ~m-expr~ \opt\ |into| ~var~}
\clindexx{finding}{maximizing}
\Clause{|finding| ~expr~ |minimizing| ~m-expr~ \opt\ |into| ~var~}
\clindexx{finding}{minimizing}
\finishitem
Computes the extremum (maximum or minimum) value of ~m-expr~ over all
iterations, and returns the value of ~expr~ corresponding to the
extremum.  ~expr~ is evaluated inside the loop at the time the new
extremum is established.  If ~m-expr~ is never evaluated (due to, for
example, being embedded in a conditional clause), then the returned
value depends on the type, if any, of ~expr~ (or ~var,~ if
one is supplied).  If there is no type, the returned
value will be \nil; if the type is numeric, the returned value will be
zero.

\cpar For these two clauses, ~var~ may be a list of two
symbols; in that case, the first is used to record ~expr~ and
the second, ~m-expr.~  

\cpar As with |finding\dots such-that|, if ~m-expr~ is a sharp-quoted
function, then it is called on ~expr~ instead of being evaluated.

\end{clauses}

\subsubsection{Boolean Tests}

\begin{clauses}

\clausex{first-iteration-p}{}
Returns |t| in the first cycle of the loop, otherwise \nil.

\clausex{first-time-p}{}
Returns |t| the first time the expression is evaluated, and then \nil\ forever.
This clause comes handy when printing (optional) elements separated
by a comma:

\begin{program}
(iter (for el in '(nil 1 2 nil 3))
      (when el
        (unless (first-time-p)
          (princ ", "))
        (princ el)))
\end{program}
produces |"1, 2, 3"|.

\end{clauses}

\subsubsection{Aggregated Boolean Tests}

\begin{clauses}

\clausex{always}{~expr~}
If ~expr~ ever evaluates to
\nil, then \nil\ is immediately returned; the epilogue code is not
executed.  If ~expr~ never evaluates to \nil, the epilogue code
is executed and the last value of ~expr~ (or |t| if ~expr~ was never
evaluated) is returned (whereas |loop| would constantly return |t|).

% mention last evaluated clause when multiple always clauses?

\clausex{never}{~expr~}
Like |(always (not ~expr~))|, except it does not influence the last
value returned by a possible other |always| clause. That is,
\begin{program}
(iter (repeat 2)
      (always 2)
      (never nil)) \yields 2 ; not t
\end{program}

\clausex{thereis}{~expr~}
If ~expr~ is ever \nonnil,
its value is immediately returned without running epilogue code.
Otherwise, the epilogue code is performed and \nil\ is returned.

This clause cannot be used together with |always| or |never|, because their
defaults are opposed (similarly, |(loop always 3 thereis nil)| refuses to
compile in some implementations of |loop|).

\end{clauses}


\subsection{Control Flow}
\label{control-flow}
Several clauses can be used to alter the usual flow of control in a loop.

Note: the clauses of this and subsequent sections don't adhere to \iter's
usual syntax, but instead use standard Common Lisp syntax.  Hence the
format for describing syntax subsequently is
like the standard format used in the Common Lisp manual, not like the
descriptions of clauses above.

\begin{clauses}

\clausex{finish}{}
Stops the loop and runs the epilogue code.

%for example:
%
%\begin{program}
%(iter (with answer = nil)
%         (initially (make-a-mess))
%         (for i from 1 to 10)
%         (when (correct? i)
%           (setq answer i)
%           (finish))
%         (finally (cleanup)))
%\end{program}
%
%this code will execute |cleanup| whether or not the test |(correct?
%i)| ever succeeds. 
%the (more elegant) formulation,
%\begin{program}
%(iter (initially (make-a-mess))
%         (for i from 1 to 10)
%         (finding i such-that (correct? i))
%         (finally (cleanup)))
%\end{program}
%would not execute |cleanup| if |(correct? i)| succeeded; it
%would do an immediate return.

\clausex{leave}{\opt\ ~value~}
Immediately returns ~value~ (default \nil) from the current \iter\
form, skipping the epilogue code.  Equivalent to using |return-from|.

\clausex{next-iteration}{}
Skips the remainder of the loop body and begins the next iteration of
the loop.

\clausex{while}{~expr~}
If ~expr~ ever evaluates to \nil, the loop is terminated and the
epilogue code executed.  Equivalent to |(if (not ~expr~) (finish))|.

\clausex{until}{~expr~}
Equivalent to |(if ~expr~ (finish))|.

\clausex{if-first-time}{~then~ \opt\ ~else~}
If this clause is being executed for the first time in this invocation
of the \iter\ form, then the ~then~ code is evaluated; otherwise the
~else~ code is evaluated.

\cpar |(for ~var~ first ~expr1~ then ~expr2~)| is almost equivalent to
\begin{program}
(if-first-time (dsetq ~var~ ~expr1~)
               (dsetq ~var~ ~expr2~))
\end{program}
The only difference is that the |for| version makes ~var~ available
for use with |for\dots previous|.

\end{clauses}

\subsection{Code Placement}
When fine control is desired over where code appears in a loop
generated by \iter, the following special clauses may be useful.
They are all subject to code-motion problems (see section
\ref{code-movement}).  

\begin{clauses}

\clausex{initially}{|\&rest| ~forms~}
The lisp ~forms~ are placed in the prologue section of the loop, where
they are executed once, before the loop body is entered.

\clausex{after-each}{|\&rest| ~forms~}
The ~forms~ are placed at the end of the loop body, where they
are executed after each iteration.  Unlike the other clauses in this
section, ~forms~ may contain \iter\ clauses.

\clausex{else}{|\&rest| ~forms~}
The lisp ~forms~ are placed in the epilogue section of the loop, where they
are executed if this |else| clause is never met during execution of the
loop and the loop terminates normally.

\clausex{finally}{|\&rest| ~forms~}
The lisp ~forms~ are placed in the epilogue section of the loop, where
they are executed after the loop has terminated normally.

\clausex{finally-protected}{|\&rest| ~forms~}
The lisp ~forms~ are placed in the second form of an unwind-protect
outside the loop.  They are always executed after the loop has
terminated, regardless of how the termination occurred.

\end{clauses}


%%%%%%%%%%%%%%%%%%%%%%%%%%%%%%%%%%%%%%%%%%%%%%%%%%%%%%%%%%%%%%%%
\section{Other Features}

\subsection{Multiple Accumulations}
\label{multiple}

\begin{sloppypar}
It is permitted to have more than one clause accumulate into the same
variable, as in the following:
\begin{program}
(iter (for i from 1 to 10)
      (collect i into nums)
      (collect (sqrt i) into nums)
      (finally (return nums)))
\end{program}
Clauses can only accumulate into the same variable if they are
compatible.  |collect|, |adjoining|, |appending|, |nconcing|,
|unioning| and |nunioning| are compatible with each other; |sum|,
|multiply| and |counting| are compatible; |always| and |never| are compatible;
|finding| \dots |such-that| is compatible with either |thereis| or |always|
and |never| when their defaults match; and |maximize| and |minimize| clauses
are compatible only with other |maximize| and |minimize| clauses,
respectively.

%note that the same variable ~cannot~ be both an accumulation
%variable and an ordinary variable; there can be only one variable with
%a given name within an \iter\ form. 
\end{sloppypar}

\subsection{Named Blocks}

Like Common Lisp |block|s, \iter\ forms can be given names.  The
name should be a single symbol, and it must be the first form in the
\iter.  The generated code behaves exactly like a named block; in
particular, |(return-from ~name~)| can be used to exit it:
\begin{program}
(iter fred
      (for i from 1 to 10)
      (iter barney
            (for j from i to 10)
            (if (> (* i j) 17)
                (return-from fred j))))
\end{program}
An \iter\ form that is not given a name is implicitly named \nil.

Sometimes one would like to write an expression in an inner \iter\ form,
but have it processed by an outer \iter\ form.  This is possible with
the |in| clause.  

\begin{clauses}

\clausex{in}{~name~ |\&rest| ~forms~}
Evaluates ~forms~ as if they were part of the \iter\ form named
~name~.  In other words, \iter\ clauses are processed by the \iter\
form named ~name,~ and not by any \iter\ forms that occur inside ~name.~

\cpar As an example, consider the problem of collecting a list of the
elements in a two-dimensional array.  The naive solution,
\begin{program}
(iter (for i below (array-dimension ar 0))
      (iter (for j below (array-dimension ar 1))
            (collect (aref ar i j))))
\end{program}
\noindent is wrong because the list created by the inner \iter\ is simply
ignored by the outer one.  But using |in| we can write:
\begin{program}
(iter outer (for i below (array-dimension ar 0))
      (iter (for j below (array-dimension ar 1))
            (in outer (collect (aref ar i j)))))
\end{program}
\noindent which has the desired result.

\end{clauses}

\subsection{Destructuring}
\label{destructuring}

In many places within \iter\ clauses where a variable is expected, a
list can be written instead.  In these cases, the value to be assigned
is ~destructured~ according to the pattern
described by the list.  As a simple example, the clause
\begin{program}
(for (key . item) in alist)
\end{program}
\noindent will result in |key| being set to the |car| of
each element in |alist|, and |item| being set to the |cdr|.  The
pattern list may be nested to arbitrary depth, and (as the example
shows) need not be terminated with \nil; the only requirement is that
each leaf be a bindable symbol (or \nil, in which case no binding is
generated for that piece of the structure).

Sometimes, you might like to do the equivalent of a
|multiple-value-setq| in a clause.  This 
``multiple-value destructuring'' can be expressed by writing 
\linebreak |(values $pat_1$ $pat_2 \ldots$)| for a destructuring
pattern, as in 
\begin{program}
(for (values (a . b) c d) = (three-valued-function ...))
\end{program}
\begin{sloppypar}
Note that the $pat_i$ can themselves be destructuring patterns (though
not multiple-value destructuring patterns).  You can't do multiple-value
destructuring in a |with| clause; instead wrap the whole \iter\
form in a |multiple-value-bind|.
\end{sloppypar}

\begin{note}{Rationale}
There are subtle interactions between variable declarations and
evaluation order that make the correct implementation of
multiple-value destructuring in a |with| somewhat tricky.
\end{note}

The destructuring feature of \iter\ is available as a separate
mechanism, using the |dsetq| macro:

\begin{clauses}
\defmacro{dsetq}{template expr}
Performs destructuring of ~expr~ using ~template~.   May be used
outside of an \iter\ form. Yields the primary value of ~expr~.

\end{clauses}

\subsection{On-line Help}

\begin{sloppypar}
There is a limited facility for on-line help, in the form of the 
|display-iterate-clauses| function.
\end{sloppypar}

\begin{clauses}

\defun{display-iterate-clauses}{\opt\ clause-spec}
Displays a list of \iter\ clauses.  If ~clause-spec~ is not
provided, all clauses are shown; if it is a symbol, all clauses
beginning with that symbol are shown; and if it is a list of symbols,
all clauses for which ~clause-spec~ is a prefix are shown.

\end{clauses}

\subsection{Parallel Binding and Stepping}
\label{bindings}

The parallel binding and stepping of variables is a feature that
\iter\ does ~not~ have.  This section attempts to provide a rationale.

We say that two variables are bound ~in parallel~ if neither
binding shadows the other.  This is the usual semantics of |let|
(as opposed to |let*|).  Similarly, we can say that iteration
variables are stepped in parallel if neither variable is updated
before the other, conceptually speaking; in other words, if the code
to update each variable can reference the old values of both variables.

|loop| allows parallel binding of variables and parallel stepping of
driver variables.  My view is that if you are depending on the 
serial/parallel distinction, you are doing something obscure.  If you
need to bind 
variables in parallel using |with|, then you must be using a
variable name that shadows a name in the existing lexical environment.
Don't do that.  The most common use for parallel stepping is to track
the values of variables on the previous iteration, but in fact this
does not require parallel stepping at all; the following will work:
\begin{program}
(iter (for current in list)
      (for prev previous current)
      ...)
\end{program}


%%%%%%%%%%%%%%%%%%%%%%%%%%%%%%%%%%%%%%%%%%%%%%%%%%%%%%%%%%%%%%%%
\section{Types and Declarations}
\label{types}

\subsection{Discussion}

Sometimes efficiency dictates that the types of variables be declared.
This type information needs to be communicated to \iter\ so it can
bind variables to appropriate values.  Furthermore, \iter\ must often
generate internal variables invisible to the user; there needs to be a
way for these to be declared.

As an example, consider this code, which will return the number of
odd elements in |number-list|:
\begin{program}
(iter (for el in number-list)
      (count (oddp el)))
\end{program}
In processing this form,
\iter\ will create an internal variable, let us call it |list17|, to
hold the successive |cdr|s of |number-list|, and  
will bind the variable to |number-list|.  It will also generate a
default binding for |el|; only inside the body of the loop will |el|
be set to the |car| of |list17|.  Finally, \iter\ will generate a
variable, call it |result|, to hold the result of the count, and will
bind it to zero.

When dealing with type declarations, \iter\ observes one simple rule:
~it will never generate a declaration unless requested to do so.~  The
reason is that such declarations might mask errors in compiled code by
avoiding error-checks; the resulting problems would be doubly hard to
track down because the declarations would be hidden from the
programmer.  Of course, a compiler might omit error-checks even in the
absence of declarations, though this behavior can usually be avoided,
e.g. by saying |(declaim (optimize (safety 3)))|.

So, the above \iter\ form will generate code with no declarations.
But say we wish to declare the types of |el| and the internal
variables |list17| and |result|.  How is this done?

Declaring the type of |el| is easy, since the programmer knows
the variable's name:
\begin{program}
(iter (for el in number-list)
      (declare (fixnum el))
      (counting (oddp el)))
\end{program}
\iter\ can read variable type declarations like this one.  Before
processing any clauses, it scans the entire top-level form for type
declarations 
and records the types, so that variable bindings can be performed
correctly.  In this case, |el| will be bound to zero
instead of \nil.  Also, \iter\ collects all the top-level declarations
and puts them at the begining of the generated code, so it is not
necessary to place all declarations at the beginning of an \iter\
form; instead, they can be written near the variables whose types they
declare.  

Since \iter\ is not part of the compiler, it will not know
about declarations that occur outside an \iter\ form; these
declarations must be repeated inside the form.

Here is another way we could have declared the type of |el|:
\begin{program}
(iter (for (the fixnum el) in number-list)
      (counting (oddp el)))
\end{program}
\iter\ extends the Common Lisp |the|\lispindex{the}
form to apply to variables as well as value-producing forms; anywhere
a variable is allowed---in a |with| clause, as the iteration
variable in a driver clause, as the |into| argument of an
accumulation clause, even inside a destructuring template---you can
write |(the ~type~ ~symbol~)| instead.

There is one crucial difference between using a |the| form and
actually declaring the variable: explicit declarations are always
placed in the generated code, but type information from a |the|
form is not turned into an actual declaration unless you tell \iter\
to do so using |iterate:declare-variables|.  See below.

\begin{sloppypar}
Declaring the types of internal variables is harder than declaring the
types of explicitly mentioned variables, since their names
are unknown.  You do it by declaring |iterate:declare-variables|
somewhere inside the top level of the \iter\ form.  (This will also
generate declarations for variables declared using |the|.)
\iter\ does not provide much selectivity here: it's all or none. 
And unfortunately, since \iter\ is not privy to compiler information
but instead reads declarations itself, it will not hear if you
|(declaim (iterate:declare-variables))|.  Instead, set the variable 
|iterate::*always-declare-variables*| to |t| at
compile-time, using |eval-when|.
\end{sloppypar}

To determine the appropriate types for internal variables, \iter\ uses
three sources of information:

\begin{itemize}

\item Often, the particular clause dictates a certain type for a
variable; \iter\ will use this information when available.  In the
current example, the variable |list17| will be given the type
|list|, since that is the only type that makes sense; and the
variable |result| will be given the type |fixnum|, on the
assumption that you will not be counting high enough to need bignums.
You can override this assumption only by using and explicitly declaring a
variable: 
\begin{verbatim}
(iter (declare (iterate:declare-variables))
      (for el in number-list)
      (count (oddp el) into my-result)
      (declare (integer my-result))
      (finally (return my-result)))
\end{verbatim}

\begin{sloppypar}
Other examples of the type assumptions that \iter\ makes are: type
|list| for |into| variables of collection clauses; type |list| for
expressions that are to be destructured; type |vector| for the
variable holding the vector in a |for\dots in-vector| clause, and
similarly for |string| and the |for\dots in-string| clause;
and the implementation-dependent type
|(type-of array-dimension-limit)| for the index and limit
variables generated by sequence iteration drivers like |for\dots
in-vector| and |for\dots in-string| (but not |for\dots in-sequence|,
because it may be used to iterate over a list). 
\end{sloppypar}

\item Sometimes, \iter\ will examine expressions and try to determine
their types in a simple-minded way.  If the expression is
self-evaluating (like a number, for instance), \iter\ knows that the
expression's type is the same as the type of the value it denotes, so
it can use that type.  If the expression is of the form |(the ~type~
~expr~)|, \iter\ is smart enough to extract ~type~ and use it.
However, the current version of \iter\ does not  
examine declarations of function result types or do any type
inference.  It will not determine, for 
example, that the type of |(+ 3 4)| is |fixnum|, or even
|number|.

\item In some cases, the type of an internal variable should match the
type of some other variable.  For instance, \iter\ generates an
internal variable for |(f x)| in the
clause |(for i from 1 to (f x))|, and in the absence of other
information will give it the same type as |i|.  If, however, the
expression had been written |(the fixnum (f x))|, then \iter\
would have given the internal variable the type |fixnum|
regardless of |i|'s type.  The type incompatibility errors that
could arise in this situation are not checked for.

\end{itemize}

Note that if you do declare |iterate:declare-variables|, then
\iter\ may declare user variables as well as internal ones if they do
not already have declarations, though only for variables that it
binds.  For instance, in this code:

\begin{program}
(iter (declare (iterate:declare-variables))
      (for i from 1 to 10)
      (collect i into var))
\end{program}
the variable |var| will be declared to be of type |list|.


\subsection{Summary}

\iter\ understands standard Common Lisp variable type declarations
that occur within an \iter\ form and
will pass them through to the generated code.  If the declaration
|(iterate:declare-variables)|\lispindex{declare-variables}
appears at the top level of an 
\iter\ form, or if 
|iterate::*always-declare-variables*|\lispindex{*always-declare-variables*} 
is \nonnil, then \iter\ will use the type information gleaned from user
declarations, self-evaluating expressions and |the| expressions,
combined with reasonable assumptions, to determine variable
types and declare them.


%%%%%%%%%%%%%%%%%%%%%%%%%%%%%%%%%%%%%%%%%%%%%%%%%%%%%%%%%%%%%%%%
\section{Problems with Code Movement}
\label{code-movement}

Some \iter\ clauses, or parts of clauses, result in code being
moved from the location of the clause to other parts of the loop.
Drivers behave this way, as do code-placement clauses like |initially|
and |finally|.  When using these clauses, there is a danger of writing
an expression that makes sense in its apparent location but will be
invalid or have a different meaning in another location.  For example:
\begin{program}
(iter (for i from 1 to 10)
      (let ((x 3))
        (initially (setq x 4))))
\end{program}
While it may appear that the |x| of |(initially (setq x 4))| is the
same as the |x| of |(let ((x 3)) \dots|, in fact they are not:
|initially| moves its code outside the loop body, so |x| would refer
to a global variable.  Here is another example of the same problem:
\begin{program}
(iter (for i from 1 to 10)
      (let ((x 3))
        (collect i into x)))
\end{program}
If this code were executed, |collect| would create a binding for its
|x| at the top level of the \iter\ form that the |let| will shadow.

Happily, \iter\ is smart enough to catch these errors; it
walks all problematical code to ensure that free variables are not
bound inside the loop body, and checks all variables it binds for the
same problem.

However, some errors cannot be caught:

\begin{program}
(iter (with x = 3)
      (for el in list)
      (setq x 1)
      (reducing el by \#'+ initial-value x))
\end{program}
|reducing| moves its |initial-value| argument to the initialization
part of the loop in order to produce more efficient code.  Since
\iter\ does not perform data-flow analysis, it cannot determine that
|x| is changed inside the loop; all it can establish is that |x| is
not bound internally.  Hence this code will not signal an
error and will use $3$ as the initial value of the reduction.

The following list summarizes all cases that are subject to these code
motion and variable-shadowing problems.
\begin{itemize}
\item Any variable for which \iter\ creates a binding, including those
used in |with| and the |into| keyword of many clauses.

\begin{sloppypar}
\item The special clauses which place code: |initially|, |after-each|, |else|,
|finally| and |finally-protected|. 
\end{sloppypar}

\item The variables of a |next| or |do-next| form.

\item The |initially| arguments of |for\dots initially\dots then| and
|for\dots previous|. 

\item The |then| argument of |for\dots initially\dots then|.

\item The |initial-value| arguments of |reducing| and |accumulate|.

\item The |on-failure| argument of |finding\dots such-that|.
\end{itemize}

%%%%%%%%%%%%%%%%%%%%%%%%%%%%%%%%%%%%%%%%%%%%%%%%%%%%%%%%%%%%%%%%
\section{Differences Between {\tt Iterate} and {\tt Loop}}
|loop| contains a great deal of complexity which \iter\ tries to
avoid.  Hence many esoteric features of |loop| don't exist in \iter.
Other features have been carried over, but in a cleaned-up form.  
And of course, many new features have been added; they are not
mentioned in this list.

\begin{itemize}

\item \iter's syntax is more Lisp-like than |loop|'s, having a higher
density of parens.

\item The current implementation of \iter, unlike the current version
of  |loop| (as documented in {\em Common Lisp, 2nd Ed.\/}), is
extensible (see section \ref{extend}).  

\item |loop| puts the updates of all driver variables at the top of
the loop; \iter\ leaves them where the driver clauses appear.

\item While for the most part \iter\ clauses that resemble |loop| clauses
behave similarly, there are some differences.  For instance, there is
no |for\dots =\dots then| in \iter; instead use 
|for\dots initially\dots then|.  

\item |loop| binds the variable |it| at certain times
to allow pseudo-English expressions like |when ~expr~ return it|. 
In \iter, you must bind ~expr~ to a variable yourself.  Note that
|when ~expr~ return it| is like |thereis ~expr~| except that the latter is an
accumulation clause and therefore competes with other accumulations
(remember section \ref{multiple} above).

% repeat different behaviour of |always| clause here?

\item |loop| has a special |return| clause, illustrated in the
previous item.  \iter\ doesn't need one,  since an ordinary Lisp
|return| has the same effect.

\item |loop| allows for parallel binding and stepping of iteration
variables.  \iter\ does not.  (See section \ref{bindings}.)

\item |loop| and \iter\ handle variable type declarations very
differently.  |loop| provides a special syntax for declaring variable
types, and does not examine declarations.  Moreover, the standard
implementation of |loop| will
generate declarations when none are requested.
\iter\ parses standard Common Lisp type declarations, and will never
declare a variable itself unless declarations are 
specifically requested.

\end{itemize}

%%%%%%%%%%%%%%%%%%%%%%%%%%%%%%%%%%%%%%%%%%%%%%%%%%%%%%%%%%%%%%%%
\section{Rolling Your Own}
\label{extend}

\subsection{Introduction}

\iter\ is extensible---you can write new clauses that embody new
iteration patterns.  You might want to write a new driver clause for a
data structure of your own, or you might want to write a clause that
collects or manipulates elements in a way not provided by \iter. 

This section describes how to write clauses for \iter.  Writing a
clause is like writing a macro.  In fact, writing a clause ~is~
writing a macro: since \iter\ code-walks its body and macroexpands,
you can add new abstractions to \iter\ with good old |defmacro|.

Actually, there are two extensions you can make to \iter\ that are
even easier than writing a macro.  They are adding a synonym for an
existing clause and defining a driver clause for an indexable
sequence.  These can be done with |defsynonym| and
|defclause-sequence|, respectively.  See section \ref{aids}, below.

The rest of this section explains how to write macros that expand into
\iter\ clauses.
Here's how you could add a simplified version of \iter's |multiply|
clause, if \iter\ didn't already have one:
\begin{program}
(defmacro multiply (expr)
  `(reducing ,expr by \#'* initial-value 1))
\end{program}

If you found yourself summing the square of an expression often, you
might want to write a macro for that.  A first cut might be
\begin{program}
(defmacro sum-of-squares (expr)
  `(sum (* ,expr ,expr)))
\end{program}
but if you are an experienced macro writer, you will realize that this
code will evaluate ~expr~ twice, which is probably a bad idea.  A
better version would use a temporary:
\begin{program}
(defmacro sum-of-squares (expr)
  (let ((temp (gensym)))
    `(let ((,temp ,expr))
       (sum (* ,temp ,temp)))))
\end{program}
Although this may seem complex, it is just the sort of thing you'd
have to go through to write any macro, which illustrates the point of
this section: if you can write macros, you can extend \iter.

Our macros don't use \iter's keyword-argument
syntax.  We could just use keywords with |defmacro|, but we would
still not be using \iter's clause indexing mechanism.  Unlike Lisp,
which uses just the first symbol of a form to determine what function
to call, \iter\ individuates clauses by the list of required keywords.
For instance, |for\dots in| and |for\dots in-vector| are different clauses
implemented by distinct Lisp functions.  

To buy into this indexing scheme, as well as the keyword-argument
syntax, use |defmacro-clause|:

\begin{clauses}

\defmacro{defmacro-clause}{arglist |\&body| body}
Defines a new \iter\ clause.  ~arglist~ is a list of symbols which are
alternating keywords and arguments.  \opt\ may be used, and the list
may be terminated by |\&sequence|.  ~body~ is an ordinary macro body,
as with |defmacro|.  If the first form of ~body~ is a string, it is
considered a documentation string and will be shown by
|display-iterate-clauses|. |defmacro-clause| will signal an error if 
defining the clause would result in an ambiguity.  E.g. you cannot
define the clause |for\dots from| because there would be no way to
distinguish it from a use of the |for| clause with optional keyword |from|.

\end{clauses}

\medskip

Here is |multiply| using |defmacro-clause|.  The keywords are capitalized
for readability.
\begin{program}
(defmacro-clause (MULTIPLY expr \opt\ INTO var)
  `(reducing ,expr by \#'* into ,var initial-value 1))
\end{program}
You don't have to worry about the case when |var| is not supplied; for
any clause with an |into| keyword, saying |into nil| is equivalent to
omitting the |into| entirely.

As another, more extended example, consider the fairly common
iteration pattern that 
involves finding the sequence element that maximizes (or minimizes) some
function.  \iter\ provides this as |finding\dots maximizing|, but it's
instructive to see how to write it.
Here, in pseudocode, is how you might write such a loop for
maximizing a function F:

\begin{quote}
\begin{tabbing}
set variable MAX-VAL to NIL; \\
set variable WINNER to NIL; \\
for\=\ each element EL in the sequence \\
\>  if\=\ MAX-VAL\=\ is NIL or F(EL) $>$ MAX-VAL then \\
\>\>    set MAX-VAL to F(EL); \\
\>\>    set WINNER to EL; \\
\>  end if; \\
end for; \\
return WINNER.
\end{tabbing}
\end{quote}

Here is the macro:
\begin{program}
(defmacro-clause (FINDING expr MAXIMIZING func \opt\ INTO var)
  (let ((max-val (gensym))
        (temp1 (gensym))
        (temp2 (gensym))
        (winner (or var iterate::*result-var*)))
    `(progn 
       (with ,max-val = nil)
       (with ,winner = nil)
       (cond 
        ((null ,max-val)
         (setq ,winner ,expr)
         (setq ,max-val (funcall ,func ,winner))
        (t
         (let* ((,temp1 ,expr)
                (,temp2 (funcall ,func ,temp1)))
           (when (> ,temp2 ,max-val)
             (setq ,max-val ,temp2)
             (setq ,winner ,temp1))))))
       (finally (leave ,winner)))))
\end{program}
Note that if no |into| variable is supplied, we use
|iterate::*result-var*|, which contains the internal variable into
which all clauses place their results.  If this variable is bound by
some clause, then \iter\ will return its value automatically;
otherwise, \nil\ will be returned.  

\subsection{Writing Drivers}

In principle, drivers can be implemented just as easily as other
\iter\ clauses.  In practice, they are a little harder to get right. 
As an example, consider writing a driver that
iterates over all the 
elements of a vector, ignoring its fill-pointer.  |for\dots in-vector|
won't work for this, because it observes the fill-pointer.  It's necessary to
use |array-dimension| instead of |length| to obtain the size of the
vector.  Here is one approach:

\begin{program}
(defmacro-clause (FOR var IN-WHOLE-VECTOR v)
  "All the elements of a vector (disregards fill-pointer)"
  (let ((vect (gensym))
        (index (gensym)))
    `(progn
       (with ,vect = ,v)
       (for ,index from 0 below (array-dimension ,vect 0))
       (for ,var = (aref ,vect ,index)))))
\end{program}
Note that we immediately put |v| in a variable, in case it is an
expression.  Again, this is just good Lisp macrology.  It also has a
subtle effect on the semantics of the driver: |v| is evaluated only
once, at the beginning of the loop, so changes to |v| in the loop have
no effect on the driver.  Similarly, the bounds for numerical iteration
e.g. the above |array-dimension| are also evaluated once only.  This is how
all of \iter's drivers work.


There is an important point concerning the |progn| in this code.  We
need the |progn|, of course, because we are returning several forms,
one of which is a driver.  But \iter\ drivers must occur at top-level.
Is this code in error?  No, because ~top-level~ is defined in \iter\
to include forms inside a |progn|.  This is just the definition of
top-level that Common Lisp uses, and for the same reason: to allow
macros to return multiple forms at top-level.

While our |for\dots in-whole-vector| clause will work, it is not
ideal.  In particular, it does not support generating.  Do do so, we
need to use |for\dots next| or |for\dots do-next|.  The job is
simplified by the |defmacro-driver| macro.

\begin{clauses}

\defmacro{defmacro-driver}{arglist |\&body| body}
Defines a driver clause in
both the |for| and |generate| forms, and provides a parameter
|generate| which ~body~ can examine to determine how it was invoked.
~arglist~ is as in |defmacro-clause|, and should begin with the symbol
|for|. 

\end{clauses}

With |defmacro-driver|, our driver looks like this:
\begin{program}
(defmacro-driver (FOR var IN-WHOLE-VECTOR v)
  "All the elements of a vector (disregards fill-pointer)"
   (let ((vect (gensym))
         (end (gensym))
         (index (gensym))
         (kwd (if generate 'generate 'for)))
     `(progn
        (with ,vect = ,v)
        (with ,end = (array-dimension ,vect 0))
        (with ,index = -1)
        (,kwd ,var next (progn (incf ,index)
                               (if (>= ,index ,end) (terminate))
                               (aref ,vect ,index))))))
\end{program}


We are still missing one thing: the |\&sequence| keywords.  
We can get them easily enough, by writing 
\begin{program}
(defmacro-driver (FOR var IN-WHOLE-VECTOR v \&sequence) 
  ...)
\end{program}
We can now refer to parameters |from|, |to|, |by|, etc. which contain
either the values for the corresponding keyword, or \nil\ if the
keyword was not supplied.  Implementing the right code for these
keywords is cumbersome but not difficult; it is left as an exercise.
But before you begin, see |defclause-sequence| below for an easier way.
       
\subsection{Extensibility Aids}
\label{aids}

This section documents assorted features that may be of use in
extending \iter.

\begin{clauses}

\defunexpvar{*result-var*}
Holds the variable that is used to return a value as a result of the
\iter\ form.  You may examine this and use it in a |with| clause, but
you should not change it.

\defmacro{defsynonym}{syn word}
 Makes ~syn~ a synonym for the existing \iter\ keyword ~word.~  Only
the first word in each clause can have synonyms. 

\defmacroxx{defclause-sequence}{element-name index-name}{|\&key|
access-fn size-fn sequence-type}{element-type
element-doc-string index-doc-string}  
Provides
a simple way to define sequence clauses.  Generates two
clauses, one for iterating over the sequence's elements, the other
for iterating over its indices.  The first symbol of both
clauses will have print-name |for|.
~element-name~  and ~index-name~ should be symbols.
~element-name~ is the second keyword of the element iterator (typically of
the form 
|in-~sequence-type~|), and ~index-name~ is the second keyword
of the index-iterator (typically of the form
|index-of-~sequence-type~|).  Either name may be 
\nil, in which case the corresponding clause is not defined.  If both
symbols are supplied, they should be in the same package.  The |for|
that begins the clauses will be in this package.

\cpar ~access-fn~ is the function to be used to
access elements of the sequence in the element iterator.  The function
should take two 
arguments, a sequence and an index, and return the appropriate element.
~size-fn~ should denote a function of one argument, a sequence, that
returns its size.  Both ~access-fn~ and ~size-fn~ are required for the
element iterator, but only ~size-fn~ is needed for the index iterator.

\cpar The ~sequence-type~ and ~element-type~ keywords are used to
suggest types for the variables 
used to hold the sequence and the
sequence elements, respectively.  The usual rules about \iter's
treatment of variable type declarations apply (see section \ref{types}).

\cpar ~element-doc-string~ and ~index-doc-string~ are
the documentation strings, for use with |display-iterate-clauses|. 

\cpar The generated element-iterator performs destructuring on the
element variable.

\cpar As an example, the above |for\dots in-whole-vector| example
could have been written:
\begin{program}
(defclause-sequence IN-WHOLE-VECTOR INDEX-OF-WHOLE-VECTOR
  :access-fn 'aref
  :size-fn \#'(lambda (v) (array-dimension v 0))
  :sequence-type 'vector
  :element-type t
  :element-doc-string 
     "Elements of a vector, disregarding fill-pointer"
  :index-doc-string 
     "Indices of vector, disregarding fill-pointer")
\end{program}

\end{clauses}

\subsection{Subtleties}

There are some subtleties to be aware of when writing \iter\ clauses.
First, the code returned by your macros may be |nconc|'ed into a list,
so you should always returned freshly consed lists, rather than
constants.  Second, \iter\ matches clauses by using |eq| on the first
symbol and |string=| on the subsequent ones, so the package of the
first symbol of a clause is relevant.  All of the clauses in this manual
have their first word in the \iter\ package.  
You can use the package system in the usual way to shadow
\iter\ clauses without replacing them.

%%% say more here, about the badness that only the first word of a
%%% clause is packagey.

\section{Non-portable Extensions to Iterate (Contribs)}
\label{contribs}

Currently, there is only one non-portable extension to iterate in the
distribution: iterate-pg. If you have made an extension that depends
on non-portable features, feel free to send them to |asf@boinkor.net|
for inclusion in the iterate distribution.

\subsection{An SQL query driver for iterate}

The pg package by Eric Marsden (see |http://cliki.net/pg|) provides an
interface to the PostgreSQL database. Using the \iterpg\ extension, it
is possible to handle the results of SQL queries using \iter.

This usage example should give you an idea of how to use it:

\begin{program}
(pg:with-pg-connection (c "somedb" "someuser")
  (iter (for (impl version date) in-relation "select * from version"
                                 on-connection *dbconn*)
        (collect version)))
\end{program}

To use the extension via |ASDF|, simply make your system depend on the
|iterate-pg| system instead of the |iterate| system. To load it
manually, use:

\begin{program}
  (asdf:oos 'asdf:load-op :iterate-pg)
\end{program}

%%%%%%%%%%%%%%%%%%%%%%%%%%%%%%%%%%%%%%%%%%%%%%%%%%%%%%%%%%%%%%%%
\section{Obtaining {\tt Iterate}}

\begin{sloppypar}
The information in this chapter is obsolete but included for
completeness's sake; Currently, the most up-to-date information on
\iter\ can be found at |http://boinkor.net/iterate.html|.
\end{sloppypar}

\begin{sloppypar}
\iter\ currently runs on Lisp Machines, and on
HP's, Sun3's and Sparcstations under Lucid.  
\iter\ source and binaries are available at the MIT AI Lab in the
subdirectories of |/src/local/lisplib/|.  The source file,
|iterate.lisp|, is also available for anonymous FTP in the directory
|/com/fpt/pub/| on the machine |TRIX.AI.MIT.EDU| (Internet number
128.52.32.6).  If you are unable to obtain |iterate| in one of these
ways, send mail to |jba@ai.mit.edu| and I will send you the source
file. 
\end{sloppypar}

\begin{sloppypar}
\iter\ resides in the |iterate| package (nickname |iter|).  Just say
\linebreak |(use-package :iterate)| to make all the necessary symbols
available. 
If a symbol is not exported, it appears in this manual with an
``|iterate::|'' prefix.
\end{sloppypar}

Send bug reports to |bug-iterate@ai.mit.edu|.  The |info-iterate|
mailing list will have notices of changes and problems; to have
yourself added, send mail to |info-iterate-request@ai.mit.edu|.


\medskip

\begin{flushleft}
 \bf Acknowledgements
\end{flushleft}
\smallskip

Richard Waters provided invaluable criticism which spurred me to improve
\iter\ greatly.  As early users, David Clemens, Oren Etzioni and Jeff
Siskind helped ferret out many bugs.

%\begin{theindex}
% The index files must be generated with the genindex program
% or from makeindex iterate-manual.idx which creates iterate-manual.ind.

%\baselineskip\normalbaselineskip
%\advance\baselineskip by -2pt

\input{iterate-manual.ind}

%\underline{Clauses}
%\input{iterate-manual.clindex}
%
%\indexspace
%\underline{Lisp}
%\input{iterate-manual.lispindex}

%\end{theindex}


\end{document}


% arch-tag: "c0f871a6-313c-11d8-abb9-000c76244c24"

%
%\indexspace
%\underline{Lisp}
%% Fix ``.'' spacing in lists
\documentstyle[12pt,aimemo]{article}

\setlength{\textwidth}{6.0in}
\setlength{\oddsidemargin}{0.25in}
\setlength{\evensidemargin}{0.25in}
\setlength{\topmargin}{0pt}
\setlength{\headsep}{24pt}
\setlength{\textheight}{8.5in}

\let\origpar=\par

% somehow, LaTex fucks with things so that the plain Tex \obeyspaces
% doesn't work.  This is the same, except we use an mbox containing
% a space instead of just a space.

\newcommand{\spc}{\mbox{ }}
\newcommand{\tAb}{\mbox{        }}
\def\obspaces{\catcode`\ =\active\catcode`\^^I=\active}
{\obspaces \global\let =\spc\global\let^^I=\tAb}

\newenvironment{program}{\medskip\samepage\tt\obeylines\obspaces}{\medskip}

% use | for \tt
\catcode`\|=13
\newif\ifvbar
\def|{\ifvbar
         \endgroup\vbarfalse
      \else 
         \begingroup\tt\vbartrue 
      \fi}



\newcommand{\lisp}{\tt}
\newcommand{\iter}{|iterate|}
\newcommand{\iterpg}{|iterate-pg|}
\newcommand{\setf}{|setf|}
\newcommand{\nil}{|nil|}
\newcommand{\nonnil}{non-\nil}
\newcommand{\opt}{{\lisp \&optional}}
\newcommand{\yields}{$\Rightarrow$}



\newbox\Dots
\setbox\Dots=\hbox{\small ...}
\renewcommand{\dots}{\copy\Dots}

% Here's a way to define an environment which will treat all
% paragraphs inside it in the same manner.  It's, as near as I can
% figure, the way that Latex does things.  Using \everypar won't work;
% apparently Latex resets that a lot.  What you have to do is: at the
% top of the file do \let\origpar=\par.  In the
% begin part of the environment, do the parshape (or whatever), then say
% \def\par{{\origpar}}.  The extra braces are necessary.   That's all
% you have to do; at the end of the environment things will return to
% normal automatically.  Why does this all work?  I have no idea.
% Another important thing: there should be a blank line (i.e. a \par)
% between the end of the last paragraph and the \end{environment}.

\newlength{\clindent}
\setlength{\clindent}{0.5in}
\newlength{\clparindent}
\setlength{\clparindent}{\parindent}

\newenvironment{clauses}{\advance\linewidth -\parindent
                        \hangindent\clindent\relax
                         \hangafter=0
                          \parindent=0pt
                          \def\par{{\origpar}}}{}

\newcommand{\cpar}{\hspace \clparindent}

\newcommand{\startitem}{\bigskip\pagebreak[3]}
\newcommand{\finishitem}{\smallskip}

\newcommand{\clause}[1]{\startitem\Clause{#1}\finishitem}
\newcommand{\Clause}[1]{\hbox{#1}}
\newcommand{\clindex}[1]{\index{{\lisp #1}}}
\newcommand{\clindexx}[2]{\index{{\lisp #1\dots #2}}}
\newcommand{\clindexxx}[3]{\index{{\lisp #1\dots #2\dots #3}}}
\newcommand{\clausex}[2]{\startitem\Clausex{#1}{#2}\finishitem}
\newcommand{\Clausex}[2]{\Clause{|#1| #2}\clindex{#1}}

\newcommand{\defvar}[1]{\startitem\deff{#1}{}{Variable}\finishitem}
\newcommand{\defunexpvar}[1]{\startitem\deffnoindex{iterate::#1}{}{Variable}
                                        \lispindex{#1}\finishitem}
\newcommand{\defun}[2]{\startitem\deff{#1}{#2}{Function}\finishitem}
\newcommand{\defmacro}[2]{\startitem\deff{#1}{#2}{Macro}\finishitem}
\newcommand{\defunx}[3]{\startitem\deffx{#1}{#2}{#3}{Function}\finishitem}
\newcommand{\defmacrox}[3]{\startitem\deffx{#1}{#2}{#3}{Macro}\finishitem}
\newcommand{\defunxx}[4]{\startitem\deffxx{#1}{#2}{#3}{#4}{Function}\finishitem}
\newcommand{\defmacroxx}[4]{\startitem\deffxx{#1}{#2}{#3}{#4}{Macro}\finishitem}

\newcommand{\deff}[3]{\deffnoindex{#1}{#2}{#3}\lispindex{#1}}
\newcommand{\deffnoindex}[3]{\hbox to \hsize{{\tt #1} {\it #2}\hfill [{\it #3}]}}
\newcommand{\deffx}[4]{\deff{#1}{#2}{#4}\moreargs{#1}{#3}}
\newcommand{\deffxx}[5]{\deffx{#1}{#2}{#3}{#5}\moreargs{#1}{#4}}

\newcommand{\moreargs}[2]{\hbox to \hsize{\phantom{\lisp #1} {\it #2}\hfill}}

\newcommand{\lispindex}[1]{\index{{\lisp #1}}}

 \newenvironment{note}[1]{\pagebreak[2]\bigskip
                          \hrule\smallskip\small
                         {\setlength{\parindent}{0pt}
                           \par{\bf #1:}}}{\smallskip\hrule\bigskip}



\makeindex

%%%%%%%%%%%%%%%%%%%%%%%%%%%%%%%%%%%%%%%%%%%%%%%%%%%%%%%%%%%%%%%%

%\thispagestyle{empty}

%\vspace*{2in}

%\title{The |Iterate| Manual}
%\author{Jonathan Amsterdam}

%\maketitle

%\thispagestyle{empty}

%\begin{center}
%\Large *** DRAFT ***
%\end{center}
%This manual is in beta-test. It will become an AI Memo in a couple of
%months.  I would appreciate any comments or
%corrections.  You can reach me at email address |jba@ai.ai.mit.edu|.

\begin{document}

\def\writtenby{Jonathan Amsterdam}
\def\contractno{This report describes research done at the Artificial
Intelligence Laboratory of the Massachusetts Institute of Technology.
Support for the laboratory's artificial intelligence research is
provided in part by the Advanced Research Projects Agency of the
Department of Defense under Office
of Naval Research contract N00014-85-K-0124.}

\AIMEMO{1236}{\Large The Iterate Manual \\[1in]}
{This is the manual for version 1.4 of \iter, a powerful iteration
macro for Common  
Lisp. \iter\ is similar to {\tt loop} but provides numerous additional
features, is well integrated with Lisp, and is extensible. 
\vfill }


% use ~ for \em
\catcode`\~=13
\newif\ifem
\def~{\ifem
         \/\endgroup\emfalse
      \else 
         \begingroup\em\emtrue 
      \fi}



%\pagebreak
%\setcounter{page}{1}

%\thispagestyle{empty}


\tableofcontents

\pagebreak
\pagestyle{headings}

\section{Introduction}

\begin{sloppypar}
This manual describes
\iter, a powerful iteration facility for Common Lisp.
\iter\ provides abstractions for many common iteration
patterns and allows for the definition of additional patterns.  \iter\ is
a macro that expands into ordinary Lisp at
compile-time, so it is more efficient than higher-order
functions like |map| and |reduce.|  
While it is similar to |loop|, \iter\ offers a more Lisp-like syntax
and enhanced extensibility.
(For a more complete
comparison of \iter\ with other iteration constructs, see MIT AI Lab Working
Paper No. 324, ~Don't Loop, Iterate.~)
\end{sloppypar}

An \iter\ form consists of the symbol |iter|\footnote{You can also use
|iterate|, but |iter| is preferred because it avoids potential conflicts with
possible future additions to Common Lisp, and because it saves
horizontal space when writing code.}
followed by one or more
forms, some of which may be \iter\ 
~clauses.~  Here is a simple example of \iter\ which collects the
numbers from 1 to 10 into a list, and returns the list.  The return
value is shown following the arrow.
\begin{program}
(iter (for i from 1 to 10)
      (collect i))          \yields (1 2 3 4 5 6 7 8 9 10)
\end{program}
This form contains two clauses: a |for| clause that steps the
variable |i| over the integers from 1 to 10, and a |collect|
clause that accumulates its argument into a list.  With a few
exceptions, all \iter\ clauses have the same format: alternating
symbols (called ~keywords~) and expressions (called
~arguments~).  The syntax and terminology are those of Common
Lisp's keyword lambda lists.  One difference is that \iter's keywords
do not have to begin with a colon---though they may, except for the
first symbol of a clause.  So you can also write  |(for i :from
1 :to 10)| if you prefer.

Any Lisp form can appear in the body of an \iter, where it will have
its usual meaning.  \iter\ walks the entire body, expanding macros,
and recognizing clauses at any level.  This example collects all the
odd numbers in a list:
\begin{program}
(iter (for el in list)
      (if (and (numberp el) (oddp el))
          (collect el)))
\end{program}

There are clauses for iterating over numbers, lists, arrays and other
objects, and for
collecting, summing, counting, maximizing and other useful operations.
\iter\ also supports the creation of new variable bindings, stepping
over multiple sequences at once, destructuring, and compiler
declarations of variable types.  The following example illustrates
some of these features: 
\begin{program}
(iter (for (key . item) in alist)
         (for i from 0)
         (declare (fixnum i))
         (collect (cons i key)))
\end{program}

This loop takes the keys of an alist and returns a new alist
associating the keys with their positions in the original list. The
compiler declaration for |i| will appear in the generated code
in the appropriate place.

%%%%%%%%%%%%%%%%%%%%%%%%%%%%%%%%%%%%%%%%%%%%%%%%%%%%%%%%%%%%%%%%
\section{Clauses}

Most of \iter's clauses will be familiar to |loop| programmers.  
(|loop| is an iteration macro that has been incorporated into Common
Lisp.  Seen Guy Steele's {\em Common Lisp, 2nd Edition}.)
In
nearly all cases they behave the same as their |loop| counterparts, so
a |loop| user can switch to \iter\ with little pain (and much gain).

All clauses with the standard keyword-argument syntax consist of two
parts: a ~required~ part, 
containing keywords that must be present and in the right order; and
an ~optional~ part, containing keywords that may be omitted and,
if present, may occur in any order.  In the descriptions below, the
parts are separated by the Lisp lambda-list keyword \opt.


\subsection{Drivers}

An iteration-driving clause
conceptually causes the iteration to go forward.  Driver clauses in
\iter\ allow iteration over numbers, lists, vectors, hashtables, packages,
files and streams.  Iteration-driving clauses must
appear at the top level of an \iter\ form; they cannot be nested
inside another clause.  The driver variable is updated at the point
where the driver clause occurs.  Before the clause is executed for the
first time, the value of the variable is undefined.

%Also, regardless of where the driver clause
%appears in the body, the driver variable is stepped at the top of the
%loop; hence it is stylistically preferable, though
%not required, to place driver clauses at the beginning of the \iter.

Multiple drivers may appear in a single \iter\ form, in which case all
of the driver variables are updated each time through the loop, in the
order in which the clauses appear.  The first driver to terminate will
terminate the entire loop.

In all cases, the value of the driver variable on exit from the loop,
including within the epilogue code (see the |finally| clause), is
undefined.

All the parameters of a driver clause are evaluated once, before the
loop begins.  Hence it is not possible to change the bounds or other
properties of an iteration by side-effect from within the loop.

With one exception, driver clauses begin with the word |for| (or
the synonym |as|) and mention an iteration variable, which is
given a binding within the \iter\ form.
The exception is |repeat|, which just executes a loop a
specified number of times:

\begin{clauses}

\clausex{repeat}{~n~}
Repeats the loop ~n~ times.  For example:
\begin{program}
(iter (repeat 100)
      (print "I will not talk in class."))
\end{program}
\cpar If $n \leq 0$, the
loop will never be executed.  If ~n~ is not an integer, the
actual number of executions will be $\lceil n \rceil$.

\end{clauses}

\pagebreak[3]

\subsubsection{Numerical Iteration}

\begin{clauses}
\clausex{for}{~var~ |\&sequence|} 
The general form for iterating over a sequence of numbers requires a
variable and, optionally, one or more keywords that provide the bounds
and step size of the iteration. The |\&sequence|
lambda-list keyword is a shorthand for these sequence keywords.  They are:
|from|, |upfrom|, |downfrom|, |to|, |downto|, |above|, |below| and |by|.
|from| provides the starting value for ~var~
and defaults to zero. |to| provides a final value and implies that the
successive values of ~var~ will be increasing; |downto| implies that
they will be decreasing.  The loop terminates when ~var~ passes
the final value (i.e. becomes smaller or larger than it, depending on
the direction of iteration); in other  words, the loop body
will never be 
executed for values of ~var~ past the final value.
|below| and |above| are similar to |to| and
|downto|, except that the loop terminates when ~var~ equals or passes
the final value.

\cpar If no final value is specified, the variable will be stepped
forever.  Using |from| or |upfrom| will result in increasing
values, while |downfrom| will give decreasing values.

\cpar On each iteration,
~var~ is incremented or decremented by the value of the sequence
keyword |by|, which defaults to 1.  It should always be a positive
number, even for downward iterations.

\cpar In the following examples, the sequence of numbers generated is shown
next to the clause.

\begin{program}
(for i upfrom 0) \yields\ 0 1 2 \ldots
(for i from 5)   \yields\ 5 6 7 \ldots    ; either from or upfrom is okay
(for i downfrom 0) \yields\ 0 -1 -2 \ldots
(for i from 1 to 3) \yields\ 1 2 3
(for i from 1 below 3) \yields\ 1 2
(for i from 1 to 3 by 2) \yields\ 1 3
(for i from 1 below 3 by 2) \yields\ 1
(for i from 5 downto 3) \yields\ 5 4 3
\end{program}

\end{clauses}

\subsubsection{Sequence Iteration}

There are a number of clauses for iterating over sequences.  In all of
them, the argument following |for| may be a list instead of a
symbol, in which case destructuring is performed.  See section
\ref{destructuring}. 

\begin{clauses}

\clause{|for| ~var~ |in| ~list~ \opt\ |by| ~step-function~}
\clindexx{for}{in}
~var~ is set to successive elements of list.  
~step-function,~ which 
defaults to |cdr|, is used to obtain the next sublist. 

\clause{|for| ~var~ |on| ~list~ \opt\ |by| ~step-function~}
\clindexx{for}{on}
~var~ is set to successive sublists of list. 
~step-function~ (default |cdr|) is used as in |for\dots in|.

\end{clauses}

\medskip

These two clauses use |atom| to test for the end
of a list.  Hence, given a list whose final |cdr| is not \nil,
they will silently ignore the last |cdr|.  Other choices are
|endp|,  which would signal an error, and |null|, which
would probably result in an error somewhere else.  If you wish to use
an end-test other than |atom|, set the variable
|iterate::*list-end-test*|\lispindex{*list-end-test*} to the name of
the desired function. 

\begin{clauses}
\clause{|for| ~var~ |in-vector| ~vector~ |\&sequence|}
\clindexx{for}{in-vector}
~var~ takes on successive elements from ~vector.~   The vector's
fill-pointer is observed.  Here and in subsequent clauses, the |\&sequence|
keywords include |with-index|, which takes a symbol as argument
and uses it for the index variable instead of an internally generated
symbol.  The other |\&sequence| keywords behave as in numerical
iteration, except that the default iteration bounds are the bounds of
the vector.  E.g. in |(for i in-vector v downto 3)|,
|i| will start off being bound to the last element in |v|,
and will  
be set to preceding elements down to and including the element with
index 3.

\clause{|for| ~var~ |in-sequence| ~seq~ |\&sequence|}
\clindexx{for}{in-sequence}
This uses Common Lisp's generalized sequence functions, |elt|
and |length|, to obtain elements and determine the length of
~seq.~   Hence it will work for any sequence, including lists, and will
observe the fill-pointers of vectors.

\clause{|for| ~var~ |in-string| ~string~ |\&sequence|}
\clindexx{for}{in-string}
~var~ is set to successive characters of ~string.~

\startitem
\Clause{|for| ~var~ |index-of-vector| ~vector~ |\&sequence|}
\clindexx{for}{index-of-vector}
\Clause{|for| ~var~ |index-of-sequence| ~sequence~ |\&sequence|}
\clindexx{for}{index-of-sequence}
\Clause{|for| ~var~ |index-of-string| ~string~ |\&sequence|}
\clindexx{for}{index-of-string}
\finishitem
~var~ is set to successive indices of the sequence.
These clauses avoid the overhead of accessing the sequence elements
for those applications where they do not need to be examined, or are
examined rarely.  They admit all the optional keywords of the other
sequence drivers except the (redundant) |with-index| keyword.

\clause{|for| ~(key value)~ |in-hashtable| ~table~}
\clindexx{for}{in-hashtable}

~key~ and ~value,~ which must appear as shown in a list
and may be destructuring templates, are set to the keys and values of
~table~.  If ~key~ is |nil|, then the hashtable's keys will be ignored;
similarly for ~value~.
The order in which 
elements of ~table~ will be retrieved is unpredictable.

\clause{|for| ~var~ |in-package| ~package~ \opt\ |external-only| ~ext~}
\clindexx{for}{in-package}
Iterates over all the symbols in ~package,~ or over only the
external symbols if ~ext~ is specified and non-|nil|.  ~ext~ is not
evaluated.  The same symbol may appear more than once.

\clause{|for| ~var~ |in-packages| ~packages~ \opt\ |having-access| ~symbol-types~}
\clindexx{for}{in-packages}
Iterates over all the symbols from the list of packages denoted by the
descriptor ~packages~ and having accessibility (or visibility) given by
~symbol-types~.  This defaults to the list |(:external :internal :inherited)|
and is not evaluated.  ~var~ must be a list of up to three variables: in each
iteration, these will be set to a symbol, its access-type and package (as per
|with-package-iterator| in ANSI-CL).  The same symbol may appear more than
once.

\end{clauses}

\begin{clauses}

\clause{|for| ~var~ |in-file| ~name~ \opt\ |using| ~reader~}
\clindexx{for}{in-file}
Opens the file ~name~ (which may be a string or pathname) for input,
and iterates 
over its contents. ~reader~ defaults to |read|, so by default ~var~
will be bound to the successive forms in the file.  The \iter\ body is
wrapped in an unwind-protect to ensure that the file is closed no
matter how the \iter\ is exited.

\clause{|for| ~var~ |in-stream| ~stream~ \opt\ |using| ~reader~}
\clindexx{for}{in-stream}
Like |for\dots in-file|, except that ~stream~ should be an existing
stream object that supports input operations.

\end{clauses}

\subsubsection{Generalized Drivers}

These are primarily useful for writing drivers that can also be used
as generators (see section \ref{generators}, below).
 
\begin{clauses}

\clause{|for| ~var~ |next| ~expr~}
\clindexx{for}{next}

~var~ is set to ~expr~ each time through the loop.  Destructuring is
performed.  When the clause is used as a generator, ~expr~ is the code
that is executed when |(next ~var~)| is 
encountered (see section \ref{generators}, below).
~expr~ should compute the first value for ~var~, as well
as all subsequent values, and is responsible for terminating the loop.
For compatibility with future versions of \iter, this termination
should be done with |terminate|\clindex{terminate}, which can be
considered a synonym for |finish| (see section \ref{control-flow}).

\cpar As an example, the following clauses are equivalent to |(for
i from 1 to 10)|:
\begin{program}
(initially (setq i 0))
(for i next (if (> i 10) (terminate) (incf i)))
\end{program}

\clause{|for| ~var~ |do-next| ~form~}
\clindexx{for}{do-next}
~form~ is evaluated each time through the loop.  Its value is ~not~
set to ~var~; that is ~form's~ job. ~var~ is only present so that
\iter\ knows it is a driver variable.  \linebreak |(for ~var~ next
~expr~)| is equivalent to |(for ~var~ do-next (dsetq ~var~ ~expr~))|.
(See section \ref{destructuring} for an explanation of |dsetq|.) 

\end{clauses}

\subsubsection{Generators}
\label{generators}

In all of the above clauses, the driver variable is updated on each
iteration.
Sometimes it is desirable to have greater control over updating.  
For instance, consider the problem of associating numbers, in
increasing order and with no gaps, with the
\nonnil\ elements of a list.  One obvious first pass at writing this is:

\begin{program}
(iter (for el in list)
      (for i upfrom 1)
      (if el (collect (cons el i))))
\end{program}
But on the list |(a b nil c)| this produces |((a . 1) (b . 2) (c .
4))| instead of the desired |((a . 1) (b . 2) (c . 3))|.  The problem
is that |i| is incremented each time through the loop, even when |el|
is |nil|.  

The problem could be solved elegantly if we could step |i| only when
we wished to.  This
can be accomplished for any \iter\ driver by writing |generate| (or
its synonym |generating|)
instead of |for|.  Doing so produces a ~generator~---a driver whose
values are yielded explicitly.  To obtain the next value of a
generator variable ~v~, write \linebreak |(next ~v~)|.  The value of a |next|
form is the next value of ~v~, as determined by its associated driver
clause.  |next| also has the side-effect of updating ~v~ to that
value.  If there is no next value, |next| will terminate the loop,
just as with a normal driver.


Using generators, we can now write our example like this:
\begin{program}
(iter (for el in list)
      (generate i upfrom 1)
      (if el (collect (cons el (next i)))))
\end{program}
Now |i| is updated only when |(next i)| is executed, and this occurs
only when |el| is \nonnil.

To better understand the relationship between ordinary drivers and
generators, observe that we can rewrite an ordinary driver using its
generator form immediately followed by |next|, as this example shows:
\begin{program}
(iter (generating i from 1 to 10)
      (next i)
      ...)
\end{program}
Provided that the loop body contains no |(next i)| forms, this will
behave just as if we had written |(for i from 1 to 10)|.

We can still refer to a driver variable ~v~ without using |next|; in
this case, its value is that given to it by the last evaluation of
|(next ~v~)|.  Before |(next ~v~)| has been called the first time, the
value of ~v~ is undefined.  

This semantics is more flexible than
one in which ~v~ begins the loop bound to its first value and calls
of |next| supply subsequent values, because it means the loop will not
terminate too soon if the generator's sequence is empty.  For
instance, consider the following code, which tags \nonnil\ elements of
a list using a list of tags, and also counts the null elements.
(We assume there are at least as many tags as \nonnil\ elements.)
\begin{program}
(let* ((counter 0)
       (tagged-list (iter (for el in list)
                          (generating tag in tag-list) 
                          (if (null el)
                              (incf counter)
                              (collect (cons el (next tag)))))))
  ...)
\end{program}
It may be that there are just as many tags as non-null elements of
|list|.  If all the elements of |list| are null, we still want the
counting to proceed, even though |tag-list| is \nil.  If |tag| had to be
assigned its first value before the loop begins, we would have had to
terminate the loop before the first iteration, since when |tag-list|
is \nil, |tag| 
has no first value.  With the existing semantics, however, |(next
tag)| will never execute, so the iteration will cover all the elements of
|list|. 
        
When the ``variable'' of a driver clause is actually a destructuring
template containing several variables, all the variables are eligible
for use with |next|.  As before, |(next ~v~)| evaluates to ~v's~ next
value; but the effect is to update all of the template's variables.
For instance, the following code will return the list |(a 2 c)|.
\begin{program}
(iter (generating (key . item) in '((a . 1) (b . 2) (c . 3)))
      (collect (next key))
      (collect (next item)))
\end{program}

Only driver clauses with variables can be made into generators.  This
includes all clauses mentioned so far except for |repeat|.  It does
~not~ include |for\dots previous|, |for\dots =|,
|for\dots initially\dots then| or |for\dots first\dots then| (see
below). 

\subsubsection{Previous Values of Driver Variables}

Often one would like to access the value of a variable on a previous
iteration.  \iter\ provides a special clause for accomplishing this.

\begin{clauses}

\clause{|for| ~pvar~ |previous| ~var~ \opt\ |initially| ~init~
|back| ~n~}
\clindexx{for}{previous}
Sets ~pvar~ to the previous value of ~var,~ which should be a driver
variable, a variable from another |for\dots previous| clause, or a
variable established by a |for\dots =|, 
|for\dots initially\dots then| or |for\dots first\dots then| clause
(see section \ref{setting}).
Initially, ~pvar~ is given the value ~init~ (which defaults to \nil).  
The ~init~ expression will be moved outside the loop body, so it
should not depend on anything computed within the loop.
~pvar~ retains the value of ~init~ until ~var~ is set to its second
value, at which point ~pvar~ is set to ~var's~ first value; and so on.  

\cpar The
argument ~n~ to |back| must be a constant, positive integer, and
defaults to 1.  It determines how many iterations back ~pvar~ should
track ~var~.  For example, when ~n~ is 2, then ~pvar~ will be assigned
~var's~ first value when ~var~ is set to its third value.

\cpar A |for\dots previous| clause may occur after or before its
associated driver clause. |for\dots previous| works with generators as
well as ordinary drivers.

\pagebreak[3]

\cpar Example:
\begin{program}
(iter (for el in '(1 2 3 4))
      (for p-el previous el)
      (for pp-el previous p-el initially 0)
      (collect pp-el))
\end{program}
This evaluates to |(0 0 1 2)|.  It could have been written more
economically as 
\begin{program}
(iter (for el in '(1 2 3 4))
      (for pp-el previous el back 2 initially 0)
      (collect pp-el))
\end{program}


\end{clauses}

\subsection{Variable Binding and Setting}
\label{setting}

Several clauses exist for establishing new variable bindings or for
setting variables in the loop.  They all support destructuring.

\begin{clauses}

\clausex{with}{~var~ \opt\ |=| ~value~}
Causes ~var~ to be bound to value before the loop body is entered.
If ~value~ is not supplied, ~var~ assumes a default
binding, which will be 
\nil\ in the absence of declarations.  Also, if ~value~ is not
supplied, no destructuring is performed; instead, ~var~ may be a list of
symbols, all of which are given default 
bindings.  If ~value~ is supplied, ~var~ is bound to it, with
destructuring.  

\cpar Because |with|
creates bindings whose scope includes the entire \iter\ form, it is
good style to put all |with| clauses at the beginning.

\cpar Successive occurrences of |with| result in sequential
bindings (as with 
|let*|).  There is no way to obtain parallel bindings; see
section \ref{bindings} for a rationale.  


\clause{|for| ~var~ |=| ~expr~}
\clindexx{for}{=}
On each iteration, ~expr~ is evaluated and ~var~ is set
to its value. 

\cpar This clause may appear to do the same thing as |for\dots next|.
In fact, they are quite different.  |for\dots =| provides only three
services: it sets up a binding for ~var~, sets it to ~expr~ on each
iteration, and makes it possible to use |for\dots previous| with
~var~.  |for\dots next| provides these services in addition to the
ability to turn the driver into a generator.
%Also, the code which sets ~var~ appears in the loop body in the same
%place as the |for\dots =| clause; the code for |for\dots next| appears
%at the top of the loop, as with other drivers (except when being used
%as a generator).

\clause{|for| ~var~ |initially| ~init-expr~ |then| ~then-expr~}
\clindexxx{for}{initially}{then}
Before the loop begins, ~var~ is set to ~init-expr;~ on all
iterations after the first it is set to ~then-expr.~  This clause
must occur at top-level.  ~init-expr~ will be moved outside the loop
body and ~then-expr~ will be moved to the end of the loop body, so
they are subject to code motion problems (see section
\ref{code-movement}). 

\cpar This clause may appear to be similar to |for\dots next|, but in
fact they differ significantly.  |for\dots initially\dots then| is
typically used to give ~var~ its first value before the loop begins,
and subsequent values on following iterations.  This is incompatible
with generators, whose first value and subsequent values must all be
computed by |(next ~var~)|.  Also, the update of ~var~ in 
|for\dots initially\dots then| does not occur at the location of the
clause. 
Use |for\dots initially\dots then| for
one-shot computations where its idiom is more convenient, but use
|for\dots next| for extending \iter\ with new drivers (see section
\ref{extend}).

\clause{|for| ~var~ |first| ~first-expr~ |then| ~then-expr~}
\clindexxx{for}{first}{then}
The first time through the loop, ~var~ is set to ~first-expr;~ on
subsequent iterations, it is set to ~then-expr.~  This differs from
|for\dots initially| in that ~var~ is set to ~first-expr~
inside the loop body, 
so ~first-expr~ may depend on the results of other clauses.  For
instance, 
\begin{program}
(iter (for num in list)
      (for i first num then (1+ i))
      ...)
\end{program}
will set |i| to the first element of |list| on the first
iteration, whereas 
\begin{program}
(iter (for num in list)
      (for i initially num then (1+ i))
      ...)
\end{program}
is probably erroneous; |i| will be bound to |num|'s
default binding (usually \nil) for the first iteration.

\end{clauses}

\begin{note}{Compatibility note}
|loop|'s |for\dots =| works like \iter's, but |loop| used the syntax
|for\dots =\dots then| to mean |for\dots initially\dots then|.  It was
felt that these two operations were sufficiently different to warrant
different keywords.

Also, the |for| in the above three clauses is misleading,
since none is true driver (e.g. none has a corresponding |generate|
form).  |setting| would have been a better choice, but |for| was used
to retain some compatibility with |loop|.  
\end{note}


%%%%%%%%%%%%%%%%%%%%%%%%%%%%%%%%
\subsection{Gathering Clauses}
Many of \iter's clauses accumulate values into a variable, or set a
variable under certain conditions.  At the end of the loop, this
variable contains the desired result.  All these clauses have an
optional |into| keyword, whose argument should be a symbol.  If the
|into| keyword is not supplied, the accumulation variable will be
internally generated and its value will be returned at the end of the
loop; if a variable is specified, that variable is used for the
accumulation, and is not returned as a result---it is up to the user
to return it explicitly, in the loop's epilogue code (see |finally|).
It is safe to examine the accumulation variable during the loop, but
it should not be modified.

These clauses all begin with a verb.  When the verb does
not conflict with an existing Common Lisp function, then it may be
used in either its infinitival or present-participle form (e.g. |sum|,
|summing|).  However, when there is a conflict with Common Lisp, only
the present-participle form may be used (e.g. |unioning|).  This is to
prevent \iter\ clauses from clashing with Common Lisp
functions. 

%although these clauses are described as ``producing a value,'' it is a
%mistake to think of the lisp list representing the clause as a
%value-producing form in the usual way.  clauses may legally be written
%where a value is expected, e.g. |(setq x (sum i))|, but the
%lisp value of a clause in such a context is undefined.

\subsubsection{Reductions}

~Reduction~ is an extremely common iteration pattern in which
the results of successive applications of a
binary operation are accumulated.
For example, a loop that computes the sum of the
elements of a list is performing a reduction with the addition
operation.  This could be written in Common Lisp as  |(reduce \#'+
list)| or with \iter\ as
\begin{program}
(iter (for el in list)
      (sum el))
\end{program}

\begin{clauses}

\clausex{sum}{~expr~ \opt\ |into| ~var~}
Each time through the loop, ~expr~ is evaluated and added to a
variable, which is bound initially to zero.  If ~expr~ has a type,
it is ~not~ used as the type of the sum variable, which is always
|number|.  To get the result variable to be of a more specific
type, use an explicit variable, as in
\begin{program}
(iter (for el in number-list)
      (sum el into x)
      (declare (fixnum x))
      (finally (return x)))
\end{program}

\clausex{multiply}{~expr~ \opt\ |into| ~var~}
Like |sum|, but the initial value of the result variable is $1$,
and the variable is updated by multiplying ~expr~ into it.

\clausex{counting}{~expr~ \opt\ |into| ~var~}
~expr~ is evaluated on each iteration.  If it is \nonnil, the
accumulation variable, initially zero, is incremented.

\startitem
\Clausex{maximize}{~expr~ \opt\ |into| ~var~}
\Clausex{minimize}{~expr~ \opt\ |into| ~var~}
\finishitem
~expr~ is evaluated on each iteration and its extremum (maximum or
minimum) is stored in the accumulation variable.  If ~expr~ is never
evaluated, then the result is \nil\ (if the accumulation variable is
untyped) or $0$ (if it has a numeric type).

\clausex{reducing}{~expr~ |by| ~func~ \opt\ |initial-value| ~init-val~ |into| ~var~}
This is a general way to perform reductions. ~func~ should be a
function of two arguments, the first of which will be the value
computed so far and
the second of which will be the value of ~expr~.  It should return the
new value.
|reducing| is roughly equivalent to the
Common Lisp |(reduce ~func~ ~sequence~ :key ~expr-function~)|, where
~expr-function~ is used to derive values from the successive elements
of ~sequence~.

\cpar If the |reducing| clause is never executed, the result is
undefined.

\cpar It is not necessary to provide an
initial value, but better code can be generated if one is supplied.
Regardless of its location in the \iter\ body, ~init-val~ will be
evaluated before the loop is entered, so it should not depend on any
value computed inside the \iter\ form.  

%\cpar if a ~var~ is not specified, you can get \iter\ to declare the
%type of the internal variable by putting a |the| expression around
%~func.~  see section \ref{types}.

\end{clauses}

%\begin{note}{implementation note}
%in principle, |maximize| and |minimize| can be thought of
%as reductions where the initial value is the smallest (or largest)
%value that the accumulation variable can assume.  because lisp's
%bignums can represent arbitrary integers, these clauses cannot be
%implemented as reductions in general.  if, however, the type of 
%~expr~ or ~var~ can be determined to be a fixnum
%or a float, \iter\ will implement the clause as a true reduction,
%using one of the constants |most-negative-fixnum|, |%most-positive-fixnum|, 
%|most-negative-short-float|, etc. as appropriate.
%\end{note}

\subsubsection{Accumulations}

All the predefined accumulation clauses add values to a sequence.  If
the sequence is a list, they all 
have the property that the partial list is kept in the correct order
and available for inspection at any point in the loop.

\begin{clauses}

\clausex{collect}{~expr~ \opt\ |into| ~var~ |at| ~place~ |result-type| ~type~}
Produces a sequence of the values of ~expr~ on each iteration. ~place~
indicates where the next value of ~expr~ is added to the list and may
be one of the symbols |start|, |beginning| (a synonym for |start|) or
|end|.  The symbol may be quoted, but need not be.  The default is
|end|.  For example,
\begin{program}
(iter (for i from 1 to 5)
      (collect i))
\end{program}
produces |(1 2 3 4 5)|, whereas
\begin{program}
(iter (for i from 1 to 5)
      (collect i at beginning))
\end{program}
produces |(5 4 3 2 1)| (and is likely to be faster in most Common Lisp
implementations).

\cpar If ~type~ is provided, it should be a subtype of |sequence|.
The default is |list|.  Specifying a type other than |list| will
result in |collect| returning a sequence of that type.  ~However,~ the
type of the sequence being constructed when inside the loop body is
undefined when a non-|list| type is specified.  (As with ~place~,
quoting ~type~ is optional.)

\clausex{adjoining}{~expr~ \opt\ |into| ~var~ |test| ~test~ |at| ~place~ |result-type| ~type~}
Like |collect|, but only adds the value of ~expr~ if it is not
already present.  ~test,~ which defaults to |\#'eql|, is
the test to be used with |member|.

\startitem
\Clausex{appending}{~expr~ \opt\ |into| ~var~ |at| ~place~}
\Clausex{nconcing}{\ ~expr~ \opt\ |into| ~var~ |at| ~place~}
\Clausex{unioning}{\ ~expr~ \opt\ |into| ~var~ |test| ~test~ |at| ~place~}
\Clausex{nunioning}{~expr~ \opt\ |into| ~var~ |test| ~test~ |at| ~place~}
\finishitem
These are like |collect|, but behave like the Common Lisp functions
|append|, |nconc|, |union| or |nunion|.  
As in Common Lisp, they work only on lists.  Also as in Common Lisp,
|unioning| and |nunioning| assume that the value of ~expr~ contains no
duplicates. 

\clausex{accumulate}{~expr~ |by| ~func~ \opt\ |initial-value| ~init-val~ |into| ~var~}
This is a general-purpose accumulation clause. ~func~ should be a
function of two arguments, the value of ~expr~ and the value
accumulated so far in the iteration, and it should return the updated
value.  If no initial value is supplied, \nil\ is used.  

%\cpar If a ~var~ is not specified, you can get \iter\ to declare the
%type of the internal variable by putting a |the| expression around
%~func.~  see section \ref{types}.

\cpar The differences between |accumulate| and |reducing| are slight.
One difference is that the functions take their arguments in a
different order.  Another is that in the absence of ~init-val~,
|accumulate| will use \nil, whereas |reducing| will generate different
code that avoids any dependence on the initial value.
The reason for having both clauses is that one usually
thinks of reductions (like |sum|) and accumulations (like |collect|)
as different beasts.

\end{clauses}


\subsubsection{Finders}

A ~finder~ is a clause whose value is an expression that meets some
condition.

\begin{clauses}

\clause{|finding| ~expr~ |such-that| ~test~ \opt\ |into| ~var~ |on-failure| ~failure-value~}
\clindexx{finding}{such-that}
If ~test~ (which is an expression) ever evaluates to \nonnil, the loop
is terminated, the 
epilogue code is run and the value of ~expr~ is returned.  Otherwise,
\nil\ (or ~failure-value,~ if provided) is returned.  If ~var~ is
provided, it will have either the \nonnil\ value of ~expr~ or
~failure-value~ when the epilogue code is run.

\cpar As a special case, if the ~test~ expression is a sharp-quoted
function, then it is applied to ~expr~ instead of being simply
evaluated.  E.g. |(finding x such-that \#'evenp)| is equivalent to
|(finding x such-that (evenp x))|.  

%\cpar although ~test~ need have nothing to do with ~%expr~ as in
%|(finding j such-that (> i 3))|, it usually 
%will: |(finding (length el) such-that (oddp (length el)))|.  to
%avoid performing the |length| computation twice, you could write
%|(finding (length el) such-that \#'oddp)| or |(finding (length
%el) such-that 'oddp)|; for these cases, \iter\ generates code that
%executes ~expr~ only once.  the code for |\#'oddp|
%is slightly different from that for {\lisp 'oddp}; see the discussion
%under {\lisp for\dots in} and {\lisp for\dots on}.

\cpar |On-failure| is a misnomer. Because it is always evaluated, it behaves
more like the default third argument to the |gethash| function. As a result,
|on-failure (error "Not found")| makes no sense. Instead, the clauses |leave|
or |thereis| can be used in conjunction with |finally| as follows:
\begin{program}
(iter (for x in '(1 2 3))
      (if (evenp x) (leave x))
      (finally (error "not found")))
\end{program}

\cpar This clause may appear multiple times when all defaults are
identical. It can also be used together with either |always|/|never| or
|thereis| if their defaults match. More specifically, |on-failure nil| is
compatible with |thereis|, while |on-failure t| is compatible with |always|
and |never| clauses.
\begin{program}
(iter (for i in '(7 -4 2 -3))
      (if (plusp i)
	  (finding    i  such-that (evenp i))
          (finding (- i) such-that (oddp i))))
\end{program}

\startitem
\Clause{|finding| ~expr~ |maximizing| ~m-expr~ \opt\ |into| ~var~}
\clindexx{finding}{maximizing}
\Clause{|finding| ~expr~ |minimizing| ~m-expr~ \opt\ |into| ~var~}
\clindexx{finding}{minimizing}
\finishitem
Computes the extremum (maximum or minimum) value of ~m-expr~ over all
iterations, and returns the value of ~expr~ corresponding to the
extremum.  ~expr~ is evaluated inside the loop at the time the new
extremum is established.  If ~m-expr~ is never evaluated (due to, for
example, being embedded in a conditional clause), then the returned
value depends on the type, if any, of ~expr~ (or ~var,~ if
one is supplied).  If there is no type, the returned
value will be \nil; if the type is numeric, the returned value will be
zero.

\cpar For these two clauses, ~var~ may be a list of two
symbols; in that case, the first is used to record ~expr~ and
the second, ~m-expr.~  

\cpar As with |finding\dots such-that|, if ~m-expr~ is a sharp-quoted
function, then it is called on ~expr~ instead of being evaluated.

\end{clauses}

\subsubsection{Boolean Tests}

\begin{clauses}

\clausex{first-iteration-p}{}
Returns |t| in the first cycle of the loop, otherwise \nil.

\clausex{first-time-p}{}
Returns |t| the first time the expression is evaluated, and then \nil\ forever.
This clause comes handy when printing (optional) elements separated
by a comma:

\begin{program}
(iter (for el in '(nil 1 2 nil 3))
      (when el
        (unless (first-time-p)
          (princ ", "))
        (princ el)))
\end{program}
produces |"1, 2, 3"|.

\end{clauses}

\subsubsection{Aggregated Boolean Tests}

\begin{clauses}

\clausex{always}{~expr~}
If ~expr~ ever evaluates to
\nil, then \nil\ is immediately returned; the epilogue code is not
executed.  If ~expr~ never evaluates to \nil, the epilogue code
is executed and the last value of ~expr~ (or |t| if ~expr~ was never
evaluated) is returned (whereas |loop| would constantly return |t|).

% mention last evaluated clause when multiple always clauses?

\clausex{never}{~expr~}
Like |(always (not ~expr~))|, except it does not influence the last
value returned by a possible other |always| clause. That is,
\begin{program}
(iter (repeat 2)
      (always 2)
      (never nil)) \yields 2 ; not t
\end{program}

\clausex{thereis}{~expr~}
If ~expr~ is ever \nonnil,
its value is immediately returned without running epilogue code.
Otherwise, the epilogue code is performed and \nil\ is returned.

This clause cannot be used together with |always| or |never|, because their
defaults are opposed (similarly, |(loop always 3 thereis nil)| refuses to
compile in some implementations of |loop|).

\end{clauses}


\subsection{Control Flow}
\label{control-flow}
Several clauses can be used to alter the usual flow of control in a loop.

Note: the clauses of this and subsequent sections don't adhere to \iter's
usual syntax, but instead use standard Common Lisp syntax.  Hence the
format for describing syntax subsequently is
like the standard format used in the Common Lisp manual, not like the
descriptions of clauses above.

\begin{clauses}

\clausex{finish}{}
Stops the loop and runs the epilogue code.

%for example:
%
%\begin{program}
%(iter (with answer = nil)
%         (initially (make-a-mess))
%         (for i from 1 to 10)
%         (when (correct? i)
%           (setq answer i)
%           (finish))
%         (finally (cleanup)))
%\end{program}
%
%this code will execute |cleanup| whether or not the test |(correct?
%i)| ever succeeds. 
%the (more elegant) formulation,
%\begin{program}
%(iter (initially (make-a-mess))
%         (for i from 1 to 10)
%         (finding i such-that (correct? i))
%         (finally (cleanup)))
%\end{program}
%would not execute |cleanup| if |(correct? i)| succeeded; it
%would do an immediate return.

\clausex{leave}{\opt\ ~value~}
Immediately returns ~value~ (default \nil) from the current \iter\
form, skipping the epilogue code.  Equivalent to using |return-from|.

\clausex{next-iteration}{}
Skips the remainder of the loop body and begins the next iteration of
the loop.

\clausex{while}{~expr~}
If ~expr~ ever evaluates to \nil, the loop is terminated and the
epilogue code executed.  Equivalent to |(if (not ~expr~) (finish))|.

\clausex{until}{~expr~}
Equivalent to |(if ~expr~ (finish))|.

\clausex{if-first-time}{~then~ \opt\ ~else~}
If this clause is being executed for the first time in this invocation
of the \iter\ form, then the ~then~ code is evaluated; otherwise the
~else~ code is evaluated.

\cpar |(for ~var~ first ~expr1~ then ~expr2~)| is almost equivalent to
\begin{program}
(if-first-time (dsetq ~var~ ~expr1~)
               (dsetq ~var~ ~expr2~))
\end{program}
The only difference is that the |for| version makes ~var~ available
for use with |for\dots previous|.

\end{clauses}

\subsection{Code Placement}
When fine control is desired over where code appears in a loop
generated by \iter, the following special clauses may be useful.
They are all subject to code-motion problems (see section
\ref{code-movement}).  

\begin{clauses}

\clausex{initially}{|\&rest| ~forms~}
The lisp ~forms~ are placed in the prologue section of the loop, where
they are executed once, before the loop body is entered.

\clausex{after-each}{|\&rest| ~forms~}
The ~forms~ are placed at the end of the loop body, where they
are executed after each iteration.  Unlike the other clauses in this
section, ~forms~ may contain \iter\ clauses.

\clausex{else}{|\&rest| ~forms~}
The lisp ~forms~ are placed in the epilogue section of the loop, where they
are executed if this |else| clause is never met during execution of the
loop and the loop terminates normally.

\clausex{finally}{|\&rest| ~forms~}
The lisp ~forms~ are placed in the epilogue section of the loop, where
they are executed after the loop has terminated normally.

\clausex{finally-protected}{|\&rest| ~forms~}
The lisp ~forms~ are placed in the second form of an unwind-protect
outside the loop.  They are always executed after the loop has
terminated, regardless of how the termination occurred.

\end{clauses}


%%%%%%%%%%%%%%%%%%%%%%%%%%%%%%%%%%%%%%%%%%%%%%%%%%%%%%%%%%%%%%%%
\section{Other Features}

\subsection{Multiple Accumulations}
\label{multiple}

\begin{sloppypar}
It is permitted to have more than one clause accumulate into the same
variable, as in the following:
\begin{program}
(iter (for i from 1 to 10)
      (collect i into nums)
      (collect (sqrt i) into nums)
      (finally (return nums)))
\end{program}
Clauses can only accumulate into the same variable if they are
compatible.  |collect|, |adjoining|, |appending|, |nconcing|,
|unioning| and |nunioning| are compatible with each other; |sum|,
|multiply| and |counting| are compatible; |always| and |never| are compatible;
|finding| \dots |such-that| is compatible with either |thereis| or |always|
and |never| when their defaults match; and |maximize| and |minimize| clauses
are compatible only with other |maximize| and |minimize| clauses,
respectively.

%note that the same variable ~cannot~ be both an accumulation
%variable and an ordinary variable; there can be only one variable with
%a given name within an \iter\ form. 
\end{sloppypar}

\subsection{Named Blocks}

Like Common Lisp |block|s, \iter\ forms can be given names.  The
name should be a single symbol, and it must be the first form in the
\iter.  The generated code behaves exactly like a named block; in
particular, |(return-from ~name~)| can be used to exit it:
\begin{program}
(iter fred
      (for i from 1 to 10)
      (iter barney
            (for j from i to 10)
            (if (> (* i j) 17)
                (return-from fred j))))
\end{program}
An \iter\ form that is not given a name is implicitly named \nil.

Sometimes one would like to write an expression in an inner \iter\ form,
but have it processed by an outer \iter\ form.  This is possible with
the |in| clause.  

\begin{clauses}

\clausex{in}{~name~ |\&rest| ~forms~}
Evaluates ~forms~ as if they were part of the \iter\ form named
~name~.  In other words, \iter\ clauses are processed by the \iter\
form named ~name,~ and not by any \iter\ forms that occur inside ~name.~

\cpar As an example, consider the problem of collecting a list of the
elements in a two-dimensional array.  The naive solution,
\begin{program}
(iter (for i below (array-dimension ar 0))
      (iter (for j below (array-dimension ar 1))
            (collect (aref ar i j))))
\end{program}
\noindent is wrong because the list created by the inner \iter\ is simply
ignored by the outer one.  But using |in| we can write:
\begin{program}
(iter outer (for i below (array-dimension ar 0))
      (iter (for j below (array-dimension ar 1))
            (in outer (collect (aref ar i j)))))
\end{program}
\noindent which has the desired result.

\end{clauses}

\subsection{Destructuring}
\label{destructuring}

In many places within \iter\ clauses where a variable is expected, a
list can be written instead.  In these cases, the value to be assigned
is ~destructured~ according to the pattern
described by the list.  As a simple example, the clause
\begin{program}
(for (key . item) in alist)
\end{program}
\noindent will result in |key| being set to the |car| of
each element in |alist|, and |item| being set to the |cdr|.  The
pattern list may be nested to arbitrary depth, and (as the example
shows) need not be terminated with \nil; the only requirement is that
each leaf be a bindable symbol (or \nil, in which case no binding is
generated for that piece of the structure).

Sometimes, you might like to do the equivalent of a
|multiple-value-setq| in a clause.  This 
``multiple-value destructuring'' can be expressed by writing 
\linebreak |(values $pat_1$ $pat_2 \ldots$)| for a destructuring
pattern, as in 
\begin{program}
(for (values (a . b) c d) = (three-valued-function ...))
\end{program}
\begin{sloppypar}
Note that the $pat_i$ can themselves be destructuring patterns (though
not multiple-value destructuring patterns).  You can't do multiple-value
destructuring in a |with| clause; instead wrap the whole \iter\
form in a |multiple-value-bind|.
\end{sloppypar}

\begin{note}{Rationale}
There are subtle interactions between variable declarations and
evaluation order that make the correct implementation of
multiple-value destructuring in a |with| somewhat tricky.
\end{note}

The destructuring feature of \iter\ is available as a separate
mechanism, using the |dsetq| macro:

\begin{clauses}
\defmacro{dsetq}{template expr}
Performs destructuring of ~expr~ using ~template~.   May be used
outside of an \iter\ form. Yields the primary value of ~expr~.

\end{clauses}

\subsection{On-line Help}

\begin{sloppypar}
There is a limited facility for on-line help, in the form of the 
|display-iterate-clauses| function.
\end{sloppypar}

\begin{clauses}

\defun{display-iterate-clauses}{\opt\ clause-spec}
Displays a list of \iter\ clauses.  If ~clause-spec~ is not
provided, all clauses are shown; if it is a symbol, all clauses
beginning with that symbol are shown; and if it is a list of symbols,
all clauses for which ~clause-spec~ is a prefix are shown.

\end{clauses}

\subsection{Parallel Binding and Stepping}
\label{bindings}

The parallel binding and stepping of variables is a feature that
\iter\ does ~not~ have.  This section attempts to provide a rationale.

We say that two variables are bound ~in parallel~ if neither
binding shadows the other.  This is the usual semantics of |let|
(as opposed to |let*|).  Similarly, we can say that iteration
variables are stepped in parallel if neither variable is updated
before the other, conceptually speaking; in other words, if the code
to update each variable can reference the old values of both variables.

|loop| allows parallel binding of variables and parallel stepping of
driver variables.  My view is that if you are depending on the 
serial/parallel distinction, you are doing something obscure.  If you
need to bind 
variables in parallel using |with|, then you must be using a
variable name that shadows a name in the existing lexical environment.
Don't do that.  The most common use for parallel stepping is to track
the values of variables on the previous iteration, but in fact this
does not require parallel stepping at all; the following will work:
\begin{program}
(iter (for current in list)
      (for prev previous current)
      ...)
\end{program}


%%%%%%%%%%%%%%%%%%%%%%%%%%%%%%%%%%%%%%%%%%%%%%%%%%%%%%%%%%%%%%%%
\section{Types and Declarations}
\label{types}

\subsection{Discussion}

Sometimes efficiency dictates that the types of variables be declared.
This type information needs to be communicated to \iter\ so it can
bind variables to appropriate values.  Furthermore, \iter\ must often
generate internal variables invisible to the user; there needs to be a
way for these to be declared.

As an example, consider this code, which will return the number of
odd elements in |number-list|:
\begin{program}
(iter (for el in number-list)
      (count (oddp el)))
\end{program}
In processing this form,
\iter\ will create an internal variable, let us call it |list17|, to
hold the successive |cdr|s of |number-list|, and  
will bind the variable to |number-list|.  It will also generate a
default binding for |el|; only inside the body of the loop will |el|
be set to the |car| of |list17|.  Finally, \iter\ will generate a
variable, call it |result|, to hold the result of the count, and will
bind it to zero.

When dealing with type declarations, \iter\ observes one simple rule:
~it will never generate a declaration unless requested to do so.~  The
reason is that such declarations might mask errors in compiled code by
avoiding error-checks; the resulting problems would be doubly hard to
track down because the declarations would be hidden from the
programmer.  Of course, a compiler might omit error-checks even in the
absence of declarations, though this behavior can usually be avoided,
e.g. by saying |(declaim (optimize (safety 3)))|.

So, the above \iter\ form will generate code with no declarations.
But say we wish to declare the types of |el| and the internal
variables |list17| and |result|.  How is this done?

Declaring the type of |el| is easy, since the programmer knows
the variable's name:
\begin{program}
(iter (for el in number-list)
      (declare (fixnum el))
      (counting (oddp el)))
\end{program}
\iter\ can read variable type declarations like this one.  Before
processing any clauses, it scans the entire top-level form for type
declarations 
and records the types, so that variable bindings can be performed
correctly.  In this case, |el| will be bound to zero
instead of \nil.  Also, \iter\ collects all the top-level declarations
and puts them at the begining of the generated code, so it is not
necessary to place all declarations at the beginning of an \iter\
form; instead, they can be written near the variables whose types they
declare.  

Since \iter\ is not part of the compiler, it will not know
about declarations that occur outside an \iter\ form; these
declarations must be repeated inside the form.

Here is another way we could have declared the type of |el|:
\begin{program}
(iter (for (the fixnum el) in number-list)
      (counting (oddp el)))
\end{program}
\iter\ extends the Common Lisp |the|\lispindex{the}
form to apply to variables as well as value-producing forms; anywhere
a variable is allowed---in a |with| clause, as the iteration
variable in a driver clause, as the |into| argument of an
accumulation clause, even inside a destructuring template---you can
write |(the ~type~ ~symbol~)| instead.

There is one crucial difference between using a |the| form and
actually declaring the variable: explicit declarations are always
placed in the generated code, but type information from a |the|
form is not turned into an actual declaration unless you tell \iter\
to do so using |iterate:declare-variables|.  See below.

\begin{sloppypar}
Declaring the types of internal variables is harder than declaring the
types of explicitly mentioned variables, since their names
are unknown.  You do it by declaring |iterate:declare-variables|
somewhere inside the top level of the \iter\ form.  (This will also
generate declarations for variables declared using |the|.)
\iter\ does not provide much selectivity here: it's all or none. 
And unfortunately, since \iter\ is not privy to compiler information
but instead reads declarations itself, it will not hear if you
|(declaim (iterate:declare-variables))|.  Instead, set the variable 
|iterate::*always-declare-variables*| to |t| at
compile-time, using |eval-when|.
\end{sloppypar}

To determine the appropriate types for internal variables, \iter\ uses
three sources of information:

\begin{itemize}

\item Often, the particular clause dictates a certain type for a
variable; \iter\ will use this information when available.  In the
current example, the variable |list17| will be given the type
|list|, since that is the only type that makes sense; and the
variable |result| will be given the type |fixnum|, on the
assumption that you will not be counting high enough to need bignums.
You can override this assumption only by using and explicitly declaring a
variable: 
\begin{verbatim}
(iter (declare (iterate:declare-variables))
      (for el in number-list)
      (count (oddp el) into my-result)
      (declare (integer my-result))
      (finally (return my-result)))
\end{verbatim}

\begin{sloppypar}
Other examples of the type assumptions that \iter\ makes are: type
|list| for |into| variables of collection clauses; type |list| for
expressions that are to be destructured; type |vector| for the
variable holding the vector in a |for\dots in-vector| clause, and
similarly for |string| and the |for\dots in-string| clause;
and the implementation-dependent type
|(type-of array-dimension-limit)| for the index and limit
variables generated by sequence iteration drivers like |for\dots
in-vector| and |for\dots in-string| (but not |for\dots in-sequence|,
because it may be used to iterate over a list). 
\end{sloppypar}

\item Sometimes, \iter\ will examine expressions and try to determine
their types in a simple-minded way.  If the expression is
self-evaluating (like a number, for instance), \iter\ knows that the
expression's type is the same as the type of the value it denotes, so
it can use that type.  If the expression is of the form |(the ~type~
~expr~)|, \iter\ is smart enough to extract ~type~ and use it.
However, the current version of \iter\ does not  
examine declarations of function result types or do any type
inference.  It will not determine, for 
example, that the type of |(+ 3 4)| is |fixnum|, or even
|number|.

\item In some cases, the type of an internal variable should match the
type of some other variable.  For instance, \iter\ generates an
internal variable for |(f x)| in the
clause |(for i from 1 to (f x))|, and in the absence of other
information will give it the same type as |i|.  If, however, the
expression had been written |(the fixnum (f x))|, then \iter\
would have given the internal variable the type |fixnum|
regardless of |i|'s type.  The type incompatibility errors that
could arise in this situation are not checked for.

\end{itemize}

Note that if you do declare |iterate:declare-variables|, then
\iter\ may declare user variables as well as internal ones if they do
not already have declarations, though only for variables that it
binds.  For instance, in this code:

\begin{program}
(iter (declare (iterate:declare-variables))
      (for i from 1 to 10)
      (collect i into var))
\end{program}
the variable |var| will be declared to be of type |list|.


\subsection{Summary}

\iter\ understands standard Common Lisp variable type declarations
that occur within an \iter\ form and
will pass them through to the generated code.  If the declaration
|(iterate:declare-variables)|\lispindex{declare-variables}
appears at the top level of an 
\iter\ form, or if 
|iterate::*always-declare-variables*|\lispindex{*always-declare-variables*} 
is \nonnil, then \iter\ will use the type information gleaned from user
declarations, self-evaluating expressions and |the| expressions,
combined with reasonable assumptions, to determine variable
types and declare them.


%%%%%%%%%%%%%%%%%%%%%%%%%%%%%%%%%%%%%%%%%%%%%%%%%%%%%%%%%%%%%%%%
\section{Problems with Code Movement}
\label{code-movement}

Some \iter\ clauses, or parts of clauses, result in code being
moved from the location of the clause to other parts of the loop.
Drivers behave this way, as do code-placement clauses like |initially|
and |finally|.  When using these clauses, there is a danger of writing
an expression that makes sense in its apparent location but will be
invalid or have a different meaning in another location.  For example:
\begin{program}
(iter (for i from 1 to 10)
      (let ((x 3))
        (initially (setq x 4))))
\end{program}
While it may appear that the |x| of |(initially (setq x 4))| is the
same as the |x| of |(let ((x 3)) \dots|, in fact they are not:
|initially| moves its code outside the loop body, so |x| would refer
to a global variable.  Here is another example of the same problem:
\begin{program}
(iter (for i from 1 to 10)
      (let ((x 3))
        (collect i into x)))
\end{program}
If this code were executed, |collect| would create a binding for its
|x| at the top level of the \iter\ form that the |let| will shadow.

Happily, \iter\ is smart enough to catch these errors; it
walks all problematical code to ensure that free variables are not
bound inside the loop body, and checks all variables it binds for the
same problem.

However, some errors cannot be caught:

\begin{program}
(iter (with x = 3)
      (for el in list)
      (setq x 1)
      (reducing el by \#'+ initial-value x))
\end{program}
|reducing| moves its |initial-value| argument to the initialization
part of the loop in order to produce more efficient code.  Since
\iter\ does not perform data-flow analysis, it cannot determine that
|x| is changed inside the loop; all it can establish is that |x| is
not bound internally.  Hence this code will not signal an
error and will use $3$ as the initial value of the reduction.

The following list summarizes all cases that are subject to these code
motion and variable-shadowing problems.
\begin{itemize}
\item Any variable for which \iter\ creates a binding, including those
used in |with| and the |into| keyword of many clauses.

\begin{sloppypar}
\item The special clauses which place code: |initially|, |after-each|, |else|,
|finally| and |finally-protected|. 
\end{sloppypar}

\item The variables of a |next| or |do-next| form.

\item The |initially| arguments of |for\dots initially\dots then| and
|for\dots previous|. 

\item The |then| argument of |for\dots initially\dots then|.

\item The |initial-value| arguments of |reducing| and |accumulate|.

\item The |on-failure| argument of |finding\dots such-that|.
\end{itemize}

%%%%%%%%%%%%%%%%%%%%%%%%%%%%%%%%%%%%%%%%%%%%%%%%%%%%%%%%%%%%%%%%
\section{Differences Between {\tt Iterate} and {\tt Loop}}
|loop| contains a great deal of complexity which \iter\ tries to
avoid.  Hence many esoteric features of |loop| don't exist in \iter.
Other features have been carried over, but in a cleaned-up form.  
And of course, many new features have been added; they are not
mentioned in this list.

\begin{itemize}

\item \iter's syntax is more Lisp-like than |loop|'s, having a higher
density of parens.

\item The current implementation of \iter, unlike the current version
of  |loop| (as documented in {\em Common Lisp, 2nd Ed.\/}), is
extensible (see section \ref{extend}).  

\item |loop| puts the updates of all driver variables at the top of
the loop; \iter\ leaves them where the driver clauses appear.

\item While for the most part \iter\ clauses that resemble |loop| clauses
behave similarly, there are some differences.  For instance, there is
no |for\dots =\dots then| in \iter; instead use 
|for\dots initially\dots then|.  

\item |loop| binds the variable |it| at certain times
to allow pseudo-English expressions like |when ~expr~ return it|. 
In \iter, you must bind ~expr~ to a variable yourself.  Note that
|when ~expr~ return it| is like |thereis ~expr~| except that the latter is an
accumulation clause and therefore competes with other accumulations
(remember section \ref{multiple} above).

% repeat different behaviour of |always| clause here?

\item |loop| has a special |return| clause, illustrated in the
previous item.  \iter\ doesn't need one,  since an ordinary Lisp
|return| has the same effect.

\item |loop| allows for parallel binding and stepping of iteration
variables.  \iter\ does not.  (See section \ref{bindings}.)

\item |loop| and \iter\ handle variable type declarations very
differently.  |loop| provides a special syntax for declaring variable
types, and does not examine declarations.  Moreover, the standard
implementation of |loop| will
generate declarations when none are requested.
\iter\ parses standard Common Lisp type declarations, and will never
declare a variable itself unless declarations are 
specifically requested.

\end{itemize}

%%%%%%%%%%%%%%%%%%%%%%%%%%%%%%%%%%%%%%%%%%%%%%%%%%%%%%%%%%%%%%%%
\section{Rolling Your Own}
\label{extend}

\subsection{Introduction}

\iter\ is extensible---you can write new clauses that embody new
iteration patterns.  You might want to write a new driver clause for a
data structure of your own, or you might want to write a clause that
collects or manipulates elements in a way not provided by \iter. 

This section describes how to write clauses for \iter.  Writing a
clause is like writing a macro.  In fact, writing a clause ~is~
writing a macro: since \iter\ code-walks its body and macroexpands,
you can add new abstractions to \iter\ with good old |defmacro|.

Actually, there are two extensions you can make to \iter\ that are
even easier than writing a macro.  They are adding a synonym for an
existing clause and defining a driver clause for an indexable
sequence.  These can be done with |defsynonym| and
|defclause-sequence|, respectively.  See section \ref{aids}, below.

The rest of this section explains how to write macros that expand into
\iter\ clauses.
Here's how you could add a simplified version of \iter's |multiply|
clause, if \iter\ didn't already have one:
\begin{program}
(defmacro multiply (expr)
  `(reducing ,expr by \#'* initial-value 1))
\end{program}

If you found yourself summing the square of an expression often, you
might want to write a macro for that.  A first cut might be
\begin{program}
(defmacro sum-of-squares (expr)
  `(sum (* ,expr ,expr)))
\end{program}
but if you are an experienced macro writer, you will realize that this
code will evaluate ~expr~ twice, which is probably a bad idea.  A
better version would use a temporary:
\begin{program}
(defmacro sum-of-squares (expr)
  (let ((temp (gensym)))
    `(let ((,temp ,expr))
       (sum (* ,temp ,temp)))))
\end{program}
Although this may seem complex, it is just the sort of thing you'd
have to go through to write any macro, which illustrates the point of
this section: if you can write macros, you can extend \iter.

Our macros don't use \iter's keyword-argument
syntax.  We could just use keywords with |defmacro|, but we would
still not be using \iter's clause indexing mechanism.  Unlike Lisp,
which uses just the first symbol of a form to determine what function
to call, \iter\ individuates clauses by the list of required keywords.
For instance, |for\dots in| and |for\dots in-vector| are different clauses
implemented by distinct Lisp functions.  

To buy into this indexing scheme, as well as the keyword-argument
syntax, use |defmacro-clause|:

\begin{clauses}

\defmacro{defmacro-clause}{arglist |\&body| body}
Defines a new \iter\ clause.  ~arglist~ is a list of symbols which are
alternating keywords and arguments.  \opt\ may be used, and the list
may be terminated by |\&sequence|.  ~body~ is an ordinary macro body,
as with |defmacro|.  If the first form of ~body~ is a string, it is
considered a documentation string and will be shown by
|display-iterate-clauses|. |defmacro-clause| will signal an error if 
defining the clause would result in an ambiguity.  E.g. you cannot
define the clause |for\dots from| because there would be no way to
distinguish it from a use of the |for| clause with optional keyword |from|.

\end{clauses}

\medskip

Here is |multiply| using |defmacro-clause|.  The keywords are capitalized
for readability.
\begin{program}
(defmacro-clause (MULTIPLY expr \opt\ INTO var)
  `(reducing ,expr by \#'* into ,var initial-value 1))
\end{program}
You don't have to worry about the case when |var| is not supplied; for
any clause with an |into| keyword, saying |into nil| is equivalent to
omitting the |into| entirely.

As another, more extended example, consider the fairly common
iteration pattern that 
involves finding the sequence element that maximizes (or minimizes) some
function.  \iter\ provides this as |finding\dots maximizing|, but it's
instructive to see how to write it.
Here, in pseudocode, is how you might write such a loop for
maximizing a function F:

\begin{quote}
\begin{tabbing}
set variable MAX-VAL to NIL; \\
set variable WINNER to NIL; \\
for\=\ each element EL in the sequence \\
\>  if\=\ MAX-VAL\=\ is NIL or F(EL) $>$ MAX-VAL then \\
\>\>    set MAX-VAL to F(EL); \\
\>\>    set WINNER to EL; \\
\>  end if; \\
end for; \\
return WINNER.
\end{tabbing}
\end{quote}

Here is the macro:
\begin{program}
(defmacro-clause (FINDING expr MAXIMIZING func \opt\ INTO var)
  (let ((max-val (gensym))
        (temp1 (gensym))
        (temp2 (gensym))
        (winner (or var iterate::*result-var*)))
    `(progn 
       (with ,max-val = nil)
       (with ,winner = nil)
       (cond 
        ((null ,max-val)
         (setq ,winner ,expr)
         (setq ,max-val (funcall ,func ,winner))
        (t
         (let* ((,temp1 ,expr)
                (,temp2 (funcall ,func ,temp1)))
           (when (> ,temp2 ,max-val)
             (setq ,max-val ,temp2)
             (setq ,winner ,temp1))))))
       (finally (leave ,winner)))))
\end{program}
Note that if no |into| variable is supplied, we use
|iterate::*result-var*|, which contains the internal variable into
which all clauses place their results.  If this variable is bound by
some clause, then \iter\ will return its value automatically;
otherwise, \nil\ will be returned.  

\subsection{Writing Drivers}

In principle, drivers can be implemented just as easily as other
\iter\ clauses.  In practice, they are a little harder to get right. 
As an example, consider writing a driver that
iterates over all the 
elements of a vector, ignoring its fill-pointer.  |for\dots in-vector|
won't work for this, because it observes the fill-pointer.  It's necessary to
use |array-dimension| instead of |length| to obtain the size of the
vector.  Here is one approach:

\begin{program}
(defmacro-clause (FOR var IN-WHOLE-VECTOR v)
  "All the elements of a vector (disregards fill-pointer)"
  (let ((vect (gensym))
        (index (gensym)))
    `(progn
       (with ,vect = ,v)
       (for ,index from 0 below (array-dimension ,vect 0))
       (for ,var = (aref ,vect ,index)))))
\end{program}
Note that we immediately put |v| in a variable, in case it is an
expression.  Again, this is just good Lisp macrology.  It also has a
subtle effect on the semantics of the driver: |v| is evaluated only
once, at the beginning of the loop, so changes to |v| in the loop have
no effect on the driver.  Similarly, the bounds for numerical iteration
e.g. the above |array-dimension| are also evaluated once only.  This is how
all of \iter's drivers work.


There is an important point concerning the |progn| in this code.  We
need the |progn|, of course, because we are returning several forms,
one of which is a driver.  But \iter\ drivers must occur at top-level.
Is this code in error?  No, because ~top-level~ is defined in \iter\
to include forms inside a |progn|.  This is just the definition of
top-level that Common Lisp uses, and for the same reason: to allow
macros to return multiple forms at top-level.

While our |for\dots in-whole-vector| clause will work, it is not
ideal.  In particular, it does not support generating.  Do do so, we
need to use |for\dots next| or |for\dots do-next|.  The job is
simplified by the |defmacro-driver| macro.

\begin{clauses}

\defmacro{defmacro-driver}{arglist |\&body| body}
Defines a driver clause in
both the |for| and |generate| forms, and provides a parameter
|generate| which ~body~ can examine to determine how it was invoked.
~arglist~ is as in |defmacro-clause|, and should begin with the symbol
|for|. 

\end{clauses}

With |defmacro-driver|, our driver looks like this:
\begin{program}
(defmacro-driver (FOR var IN-WHOLE-VECTOR v)
  "All the elements of a vector (disregards fill-pointer)"
   (let ((vect (gensym))
         (end (gensym))
         (index (gensym))
         (kwd (if generate 'generate 'for)))
     `(progn
        (with ,vect = ,v)
        (with ,end = (array-dimension ,vect 0))
        (with ,index = -1)
        (,kwd ,var next (progn (incf ,index)
                               (if (>= ,index ,end) (terminate))
                               (aref ,vect ,index))))))
\end{program}


We are still missing one thing: the |\&sequence| keywords.  
We can get them easily enough, by writing 
\begin{program}
(defmacro-driver (FOR var IN-WHOLE-VECTOR v \&sequence) 
  ...)
\end{program}
We can now refer to parameters |from|, |to|, |by|, etc. which contain
either the values for the corresponding keyword, or \nil\ if the
keyword was not supplied.  Implementing the right code for these
keywords is cumbersome but not difficult; it is left as an exercise.
But before you begin, see |defclause-sequence| below for an easier way.
       
\subsection{Extensibility Aids}
\label{aids}

This section documents assorted features that may be of use in
extending \iter.

\begin{clauses}

\defunexpvar{*result-var*}
Holds the variable that is used to return a value as a result of the
\iter\ form.  You may examine this and use it in a |with| clause, but
you should not change it.

\defmacro{defsynonym}{syn word}
 Makes ~syn~ a synonym for the existing \iter\ keyword ~word.~  Only
the first word in each clause can have synonyms. 

\defmacroxx{defclause-sequence}{element-name index-name}{|\&key|
access-fn size-fn sequence-type}{element-type
element-doc-string index-doc-string}  
Provides
a simple way to define sequence clauses.  Generates two
clauses, one for iterating over the sequence's elements, the other
for iterating over its indices.  The first symbol of both
clauses will have print-name |for|.
~element-name~  and ~index-name~ should be symbols.
~element-name~ is the second keyword of the element iterator (typically of
the form 
|in-~sequence-type~|), and ~index-name~ is the second keyword
of the index-iterator (typically of the form
|index-of-~sequence-type~|).  Either name may be 
\nil, in which case the corresponding clause is not defined.  If both
symbols are supplied, they should be in the same package.  The |for|
that begins the clauses will be in this package.

\cpar ~access-fn~ is the function to be used to
access elements of the sequence in the element iterator.  The function
should take two 
arguments, a sequence and an index, and return the appropriate element.
~size-fn~ should denote a function of one argument, a sequence, that
returns its size.  Both ~access-fn~ and ~size-fn~ are required for the
element iterator, but only ~size-fn~ is needed for the index iterator.

\cpar The ~sequence-type~ and ~element-type~ keywords are used to
suggest types for the variables 
used to hold the sequence and the
sequence elements, respectively.  The usual rules about \iter's
treatment of variable type declarations apply (see section \ref{types}).

\cpar ~element-doc-string~ and ~index-doc-string~ are
the documentation strings, for use with |display-iterate-clauses|. 

\cpar The generated element-iterator performs destructuring on the
element variable.

\cpar As an example, the above |for\dots in-whole-vector| example
could have been written:
\begin{program}
(defclause-sequence IN-WHOLE-VECTOR INDEX-OF-WHOLE-VECTOR
  :access-fn 'aref
  :size-fn \#'(lambda (v) (array-dimension v 0))
  :sequence-type 'vector
  :element-type t
  :element-doc-string 
     "Elements of a vector, disregarding fill-pointer"
  :index-doc-string 
     "Indices of vector, disregarding fill-pointer")
\end{program}

\end{clauses}

\subsection{Subtleties}

There are some subtleties to be aware of when writing \iter\ clauses.
First, the code returned by your macros may be |nconc|'ed into a list,
so you should always returned freshly consed lists, rather than
constants.  Second, \iter\ matches clauses by using |eq| on the first
symbol and |string=| on the subsequent ones, so the package of the
first symbol of a clause is relevant.  All of the clauses in this manual
have their first word in the \iter\ package.  
You can use the package system in the usual way to shadow
\iter\ clauses without replacing them.

%%% say more here, about the badness that only the first word of a
%%% clause is packagey.

\section{Non-portable Extensions to Iterate (Contribs)}
\label{contribs}

Currently, there is only one non-portable extension to iterate in the
distribution: iterate-pg. If you have made an extension that depends
on non-portable features, feel free to send them to |asf@boinkor.net|
for inclusion in the iterate distribution.

\subsection{An SQL query driver for iterate}

The pg package by Eric Marsden (see |http://cliki.net/pg|) provides an
interface to the PostgreSQL database. Using the \iterpg\ extension, it
is possible to handle the results of SQL queries using \iter.

This usage example should give you an idea of how to use it:

\begin{program}
(pg:with-pg-connection (c "somedb" "someuser")
  (iter (for (impl version date) in-relation "select * from version"
                                 on-connection *dbconn*)
        (collect version)))
\end{program}

To use the extension via |ASDF|, simply make your system depend on the
|iterate-pg| system instead of the |iterate| system. To load it
manually, use:

\begin{program}
  (asdf:oos 'asdf:load-op :iterate-pg)
\end{program}

%%%%%%%%%%%%%%%%%%%%%%%%%%%%%%%%%%%%%%%%%%%%%%%%%%%%%%%%%%%%%%%%
\section{Obtaining {\tt Iterate}}

\begin{sloppypar}
The information in this chapter is obsolete but included for
completeness's sake; Currently, the most up-to-date information on
\iter\ can be found at |http://boinkor.net/iterate.html|.
\end{sloppypar}

\begin{sloppypar}
\iter\ currently runs on Lisp Machines, and on
HP's, Sun3's and Sparcstations under Lucid.  
\iter\ source and binaries are available at the MIT AI Lab in the
subdirectories of |/src/local/lisplib/|.  The source file,
|iterate.lisp|, is also available for anonymous FTP in the directory
|/com/fpt/pub/| on the machine |TRIX.AI.MIT.EDU| (Internet number
128.52.32.6).  If you are unable to obtain |iterate| in one of these
ways, send mail to |jba@ai.mit.edu| and I will send you the source
file. 
\end{sloppypar}

\begin{sloppypar}
\iter\ resides in the |iterate| package (nickname |iter|).  Just say
\linebreak |(use-package :iterate)| to make all the necessary symbols
available. 
If a symbol is not exported, it appears in this manual with an
``|iterate::|'' prefix.
\end{sloppypar}

Send bug reports to |bug-iterate@ai.mit.edu|.  The |info-iterate|
mailing list will have notices of changes and problems; to have
yourself added, send mail to |info-iterate-request@ai.mit.edu|.


\medskip

\begin{flushleft}
 \bf Acknowledgements
\end{flushleft}
\smallskip

Richard Waters provided invaluable criticism which spurred me to improve
\iter\ greatly.  As early users, David Clemens, Oren Etzioni and Jeff
Siskind helped ferret out many bugs.

%\begin{theindex}
% The index files must be generated with the genindex program
% or from makeindex iterate-manual.idx which creates iterate-manual.ind.

%\baselineskip\normalbaselineskip
%\advance\baselineskip by -2pt

\input{iterate-manual.ind}

%\underline{Clauses}
%\input{iterate-manual.clindex}
%
%\indexspace
%\underline{Lisp}
%\input{iterate-manual.lispindex}

%\end{theindex}


\end{document}


% arch-tag: "c0f871a6-313c-11d8-abb9-000c76244c24"


%\end{theindex}


\end{document}


% arch-tag: "c0f871a6-313c-11d8-abb9-000c76244c24"


%\underline{Clauses}
%% Fix ``.'' spacing in lists
\documentstyle[12pt,aimemo]{article}

\setlength{\textwidth}{6.0in}
\setlength{\oddsidemargin}{0.25in}
\setlength{\evensidemargin}{0.25in}
\setlength{\topmargin}{0pt}
\setlength{\headsep}{24pt}
\setlength{\textheight}{8.5in}

\let\origpar=\par

% somehow, LaTex fucks with things so that the plain Tex \obeyspaces
% doesn't work.  This is the same, except we use an mbox containing
% a space instead of just a space.

\newcommand{\spc}{\mbox{ }}
\newcommand{\tAb}{\mbox{        }}
\def\obspaces{\catcode`\ =\active\catcode`\^^I=\active}
{\obspaces \global\let =\spc\global\let^^I=\tAb}

\newenvironment{program}{\medskip\samepage\tt\obeylines\obspaces}{\medskip}

% use | for \tt
\catcode`\|=13
\newif\ifvbar
\def|{\ifvbar
         \endgroup\vbarfalse
      \else 
         \begingroup\tt\vbartrue 
      \fi}



\newcommand{\lisp}{\tt}
\newcommand{\iter}{|iterate|}
\newcommand{\iterpg}{|iterate-pg|}
\newcommand{\setf}{|setf|}
\newcommand{\nil}{|nil|}
\newcommand{\nonnil}{non-\nil}
\newcommand{\opt}{{\lisp \&optional}}
\newcommand{\yields}{$\Rightarrow$}



\newbox\Dots
\setbox\Dots=\hbox{\small ...}
\renewcommand{\dots}{\copy\Dots}

% Here's a way to define an environment which will treat all
% paragraphs inside it in the same manner.  It's, as near as I can
% figure, the way that Latex does things.  Using \everypar won't work;
% apparently Latex resets that a lot.  What you have to do is: at the
% top of the file do \let\origpar=\par.  In the
% begin part of the environment, do the parshape (or whatever), then say
% \def\par{{\origpar}}.  The extra braces are necessary.   That's all
% you have to do; at the end of the environment things will return to
% normal automatically.  Why does this all work?  I have no idea.
% Another important thing: there should be a blank line (i.e. a \par)
% between the end of the last paragraph and the \end{environment}.

\newlength{\clindent}
\setlength{\clindent}{0.5in}
\newlength{\clparindent}
\setlength{\clparindent}{\parindent}

\newenvironment{clauses}{\advance\linewidth -\parindent
                        \hangindent\clindent\relax
                         \hangafter=0
                          \parindent=0pt
                          \def\par{{\origpar}}}{}

\newcommand{\cpar}{\hspace \clparindent}

\newcommand{\startitem}{\bigskip\pagebreak[3]}
\newcommand{\finishitem}{\smallskip}

\newcommand{\clause}[1]{\startitem\Clause{#1}\finishitem}
\newcommand{\Clause}[1]{\hbox{#1}}
\newcommand{\clindex}[1]{\index{{\lisp #1}}}
\newcommand{\clindexx}[2]{\index{{\lisp #1\dots #2}}}
\newcommand{\clindexxx}[3]{\index{{\lisp #1\dots #2\dots #3}}}
\newcommand{\clausex}[2]{\startitem\Clausex{#1}{#2}\finishitem}
\newcommand{\Clausex}[2]{\Clause{|#1| #2}\clindex{#1}}

\newcommand{\defvar}[1]{\startitem\deff{#1}{}{Variable}\finishitem}
\newcommand{\defunexpvar}[1]{\startitem\deffnoindex{iterate::#1}{}{Variable}
                                        \lispindex{#1}\finishitem}
\newcommand{\defun}[2]{\startitem\deff{#1}{#2}{Function}\finishitem}
\newcommand{\defmacro}[2]{\startitem\deff{#1}{#2}{Macro}\finishitem}
\newcommand{\defunx}[3]{\startitem\deffx{#1}{#2}{#3}{Function}\finishitem}
\newcommand{\defmacrox}[3]{\startitem\deffx{#1}{#2}{#3}{Macro}\finishitem}
\newcommand{\defunxx}[4]{\startitem\deffxx{#1}{#2}{#3}{#4}{Function}\finishitem}
\newcommand{\defmacroxx}[4]{\startitem\deffxx{#1}{#2}{#3}{#4}{Macro}\finishitem}

\newcommand{\deff}[3]{\deffnoindex{#1}{#2}{#3}\lispindex{#1}}
\newcommand{\deffnoindex}[3]{\hbox to \hsize{{\tt #1} {\it #2}\hfill [{\it #3}]}}
\newcommand{\deffx}[4]{\deff{#1}{#2}{#4}\moreargs{#1}{#3}}
\newcommand{\deffxx}[5]{\deffx{#1}{#2}{#3}{#5}\moreargs{#1}{#4}}

\newcommand{\moreargs}[2]{\hbox to \hsize{\phantom{\lisp #1} {\it #2}\hfill}}

\newcommand{\lispindex}[1]{\index{{\lisp #1}}}

 \newenvironment{note}[1]{\pagebreak[2]\bigskip
                          \hrule\smallskip\small
                         {\setlength{\parindent}{0pt}
                           \par{\bf #1:}}}{\smallskip\hrule\bigskip}



\makeindex

%%%%%%%%%%%%%%%%%%%%%%%%%%%%%%%%%%%%%%%%%%%%%%%%%%%%%%%%%%%%%%%%

%\thispagestyle{empty}

%\vspace*{2in}

%\title{The |Iterate| Manual}
%\author{Jonathan Amsterdam}

%\maketitle

%\thispagestyle{empty}

%\begin{center}
%\Large *** DRAFT ***
%\end{center}
%This manual is in beta-test. It will become an AI Memo in a couple of
%months.  I would appreciate any comments or
%corrections.  You can reach me at email address |jba@ai.ai.mit.edu|.

\begin{document}

\def\writtenby{Jonathan Amsterdam}
\def\contractno{This report describes research done at the Artificial
Intelligence Laboratory of the Massachusetts Institute of Technology.
Support for the laboratory's artificial intelligence research is
provided in part by the Advanced Research Projects Agency of the
Department of Defense under Office
of Naval Research contract N00014-85-K-0124.}

\AIMEMO{1236}{\Large The Iterate Manual \\[1in]}
{This is the manual for version 1.4 of \iter, a powerful iteration
macro for Common  
Lisp. \iter\ is similar to {\tt loop} but provides numerous additional
features, is well integrated with Lisp, and is extensible. 
\vfill }


% use ~ for \em
\catcode`\~=13
\newif\ifem
\def~{\ifem
         \/\endgroup\emfalse
      \else 
         \begingroup\em\emtrue 
      \fi}



%\pagebreak
%\setcounter{page}{1}

%\thispagestyle{empty}


\tableofcontents

\pagebreak
\pagestyle{headings}

\section{Introduction}

\begin{sloppypar}
This manual describes
\iter, a powerful iteration facility for Common Lisp.
\iter\ provides abstractions for many common iteration
patterns and allows for the definition of additional patterns.  \iter\ is
a macro that expands into ordinary Lisp at
compile-time, so it is more efficient than higher-order
functions like |map| and |reduce.|  
While it is similar to |loop|, \iter\ offers a more Lisp-like syntax
and enhanced extensibility.
(For a more complete
comparison of \iter\ with other iteration constructs, see MIT AI Lab Working
Paper No. 324, ~Don't Loop, Iterate.~)
\end{sloppypar}

An \iter\ form consists of the symbol |iter|\footnote{You can also use
|iterate|, but |iter| is preferred because it avoids potential conflicts with
possible future additions to Common Lisp, and because it saves
horizontal space when writing code.}
followed by one or more
forms, some of which may be \iter\ 
~clauses.~  Here is a simple example of \iter\ which collects the
numbers from 1 to 10 into a list, and returns the list.  The return
value is shown following the arrow.
\begin{program}
(iter (for i from 1 to 10)
      (collect i))          \yields (1 2 3 4 5 6 7 8 9 10)
\end{program}
This form contains two clauses: a |for| clause that steps the
variable |i| over the integers from 1 to 10, and a |collect|
clause that accumulates its argument into a list.  With a few
exceptions, all \iter\ clauses have the same format: alternating
symbols (called ~keywords~) and expressions (called
~arguments~).  The syntax and terminology are those of Common
Lisp's keyword lambda lists.  One difference is that \iter's keywords
do not have to begin with a colon---though they may, except for the
first symbol of a clause.  So you can also write  |(for i :from
1 :to 10)| if you prefer.

Any Lisp form can appear in the body of an \iter, where it will have
its usual meaning.  \iter\ walks the entire body, expanding macros,
and recognizing clauses at any level.  This example collects all the
odd numbers in a list:
\begin{program}
(iter (for el in list)
      (if (and (numberp el) (oddp el))
          (collect el)))
\end{program}

There are clauses for iterating over numbers, lists, arrays and other
objects, and for
collecting, summing, counting, maximizing and other useful operations.
\iter\ also supports the creation of new variable bindings, stepping
over multiple sequences at once, destructuring, and compiler
declarations of variable types.  The following example illustrates
some of these features: 
\begin{program}
(iter (for (key . item) in alist)
         (for i from 0)
         (declare (fixnum i))
         (collect (cons i key)))
\end{program}

This loop takes the keys of an alist and returns a new alist
associating the keys with their positions in the original list. The
compiler declaration for |i| will appear in the generated code
in the appropriate place.

%%%%%%%%%%%%%%%%%%%%%%%%%%%%%%%%%%%%%%%%%%%%%%%%%%%%%%%%%%%%%%%%
\section{Clauses}

Most of \iter's clauses will be familiar to |loop| programmers.  
(|loop| is an iteration macro that has been incorporated into Common
Lisp.  Seen Guy Steele's {\em Common Lisp, 2nd Edition}.)
In
nearly all cases they behave the same as their |loop| counterparts, so
a |loop| user can switch to \iter\ with little pain (and much gain).

All clauses with the standard keyword-argument syntax consist of two
parts: a ~required~ part, 
containing keywords that must be present and in the right order; and
an ~optional~ part, containing keywords that may be omitted and,
if present, may occur in any order.  In the descriptions below, the
parts are separated by the Lisp lambda-list keyword \opt.


\subsection{Drivers}

An iteration-driving clause
conceptually causes the iteration to go forward.  Driver clauses in
\iter\ allow iteration over numbers, lists, vectors, hashtables, packages,
files and streams.  Iteration-driving clauses must
appear at the top level of an \iter\ form; they cannot be nested
inside another clause.  The driver variable is updated at the point
where the driver clause occurs.  Before the clause is executed for the
first time, the value of the variable is undefined.

%Also, regardless of where the driver clause
%appears in the body, the driver variable is stepped at the top of the
%loop; hence it is stylistically preferable, though
%not required, to place driver clauses at the beginning of the \iter.

Multiple drivers may appear in a single \iter\ form, in which case all
of the driver variables are updated each time through the loop, in the
order in which the clauses appear.  The first driver to terminate will
terminate the entire loop.

In all cases, the value of the driver variable on exit from the loop,
including within the epilogue code (see the |finally| clause), is
undefined.

All the parameters of a driver clause are evaluated once, before the
loop begins.  Hence it is not possible to change the bounds or other
properties of an iteration by side-effect from within the loop.

With one exception, driver clauses begin with the word |for| (or
the synonym |as|) and mention an iteration variable, which is
given a binding within the \iter\ form.
The exception is |repeat|, which just executes a loop a
specified number of times:

\begin{clauses}

\clausex{repeat}{~n~}
Repeats the loop ~n~ times.  For example:
\begin{program}
(iter (repeat 100)
      (print "I will not talk in class."))
\end{program}
\cpar If $n \leq 0$, the
loop will never be executed.  If ~n~ is not an integer, the
actual number of executions will be $\lceil n \rceil$.

\end{clauses}

\pagebreak[3]

\subsubsection{Numerical Iteration}

\begin{clauses}
\clausex{for}{~var~ |\&sequence|} 
The general form for iterating over a sequence of numbers requires a
variable and, optionally, one or more keywords that provide the bounds
and step size of the iteration. The |\&sequence|
lambda-list keyword is a shorthand for these sequence keywords.  They are:
|from|, |upfrom|, |downfrom|, |to|, |downto|, |above|, |below| and |by|.
|from| provides the starting value for ~var~
and defaults to zero. |to| provides a final value and implies that the
successive values of ~var~ will be increasing; |downto| implies that
they will be decreasing.  The loop terminates when ~var~ passes
the final value (i.e. becomes smaller or larger than it, depending on
the direction of iteration); in other  words, the loop body
will never be 
executed for values of ~var~ past the final value.
|below| and |above| are similar to |to| and
|downto|, except that the loop terminates when ~var~ equals or passes
the final value.

\cpar If no final value is specified, the variable will be stepped
forever.  Using |from| or |upfrom| will result in increasing
values, while |downfrom| will give decreasing values.

\cpar On each iteration,
~var~ is incremented or decremented by the value of the sequence
keyword |by|, which defaults to 1.  It should always be a positive
number, even for downward iterations.

\cpar In the following examples, the sequence of numbers generated is shown
next to the clause.

\begin{program}
(for i upfrom 0) \yields\ 0 1 2 \ldots
(for i from 5)   \yields\ 5 6 7 \ldots    ; either from or upfrom is okay
(for i downfrom 0) \yields\ 0 -1 -2 \ldots
(for i from 1 to 3) \yields\ 1 2 3
(for i from 1 below 3) \yields\ 1 2
(for i from 1 to 3 by 2) \yields\ 1 3
(for i from 1 below 3 by 2) \yields\ 1
(for i from 5 downto 3) \yields\ 5 4 3
\end{program}

\end{clauses}

\subsubsection{Sequence Iteration}

There are a number of clauses for iterating over sequences.  In all of
them, the argument following |for| may be a list instead of a
symbol, in which case destructuring is performed.  See section
\ref{destructuring}. 

\begin{clauses}

\clause{|for| ~var~ |in| ~list~ \opt\ |by| ~step-function~}
\clindexx{for}{in}
~var~ is set to successive elements of list.  
~step-function,~ which 
defaults to |cdr|, is used to obtain the next sublist. 

\clause{|for| ~var~ |on| ~list~ \opt\ |by| ~step-function~}
\clindexx{for}{on}
~var~ is set to successive sublists of list. 
~step-function~ (default |cdr|) is used as in |for\dots in|.

\end{clauses}

\medskip

These two clauses use |atom| to test for the end
of a list.  Hence, given a list whose final |cdr| is not \nil,
they will silently ignore the last |cdr|.  Other choices are
|endp|,  which would signal an error, and |null|, which
would probably result in an error somewhere else.  If you wish to use
an end-test other than |atom|, set the variable
|iterate::*list-end-test*|\lispindex{*list-end-test*} to the name of
the desired function. 

\begin{clauses}
\clause{|for| ~var~ |in-vector| ~vector~ |\&sequence|}
\clindexx{for}{in-vector}
~var~ takes on successive elements from ~vector.~   The vector's
fill-pointer is observed.  Here and in subsequent clauses, the |\&sequence|
keywords include |with-index|, which takes a symbol as argument
and uses it for the index variable instead of an internally generated
symbol.  The other |\&sequence| keywords behave as in numerical
iteration, except that the default iteration bounds are the bounds of
the vector.  E.g. in |(for i in-vector v downto 3)|,
|i| will start off being bound to the last element in |v|,
and will  
be set to preceding elements down to and including the element with
index 3.

\clause{|for| ~var~ |in-sequence| ~seq~ |\&sequence|}
\clindexx{for}{in-sequence}
This uses Common Lisp's generalized sequence functions, |elt|
and |length|, to obtain elements and determine the length of
~seq.~   Hence it will work for any sequence, including lists, and will
observe the fill-pointers of vectors.

\clause{|for| ~var~ |in-string| ~string~ |\&sequence|}
\clindexx{for}{in-string}
~var~ is set to successive characters of ~string.~

\startitem
\Clause{|for| ~var~ |index-of-vector| ~vector~ |\&sequence|}
\clindexx{for}{index-of-vector}
\Clause{|for| ~var~ |index-of-sequence| ~sequence~ |\&sequence|}
\clindexx{for}{index-of-sequence}
\Clause{|for| ~var~ |index-of-string| ~string~ |\&sequence|}
\clindexx{for}{index-of-string}
\finishitem
~var~ is set to successive indices of the sequence.
These clauses avoid the overhead of accessing the sequence elements
for those applications where they do not need to be examined, or are
examined rarely.  They admit all the optional keywords of the other
sequence drivers except the (redundant) |with-index| keyword.

\clause{|for| ~(key value)~ |in-hashtable| ~table~}
\clindexx{for}{in-hashtable}

~key~ and ~value,~ which must appear as shown in a list
and may be destructuring templates, are set to the keys and values of
~table~.  If ~key~ is |nil|, then the hashtable's keys will be ignored;
similarly for ~value~.
The order in which 
elements of ~table~ will be retrieved is unpredictable.

\clause{|for| ~var~ |in-package| ~package~ \opt\ |external-only| ~ext~}
\clindexx{for}{in-package}
Iterates over all the symbols in ~package,~ or over only the
external symbols if ~ext~ is specified and non-|nil|.  ~ext~ is not
evaluated.  The same symbol may appear more than once.

\clause{|for| ~var~ |in-packages| ~packages~ \opt\ |having-access| ~symbol-types~}
\clindexx{for}{in-packages}
Iterates over all the symbols from the list of packages denoted by the
descriptor ~packages~ and having accessibility (or visibility) given by
~symbol-types~.  This defaults to the list |(:external :internal :inherited)|
and is not evaluated.  ~var~ must be a list of up to three variables: in each
iteration, these will be set to a symbol, its access-type and package (as per
|with-package-iterator| in ANSI-CL).  The same symbol may appear more than
once.

\end{clauses}

\begin{clauses}

\clause{|for| ~var~ |in-file| ~name~ \opt\ |using| ~reader~}
\clindexx{for}{in-file}
Opens the file ~name~ (which may be a string or pathname) for input,
and iterates 
over its contents. ~reader~ defaults to |read|, so by default ~var~
will be bound to the successive forms in the file.  The \iter\ body is
wrapped in an unwind-protect to ensure that the file is closed no
matter how the \iter\ is exited.

\clause{|for| ~var~ |in-stream| ~stream~ \opt\ |using| ~reader~}
\clindexx{for}{in-stream}
Like |for\dots in-file|, except that ~stream~ should be an existing
stream object that supports input operations.

\end{clauses}

\subsubsection{Generalized Drivers}

These are primarily useful for writing drivers that can also be used
as generators (see section \ref{generators}, below).
 
\begin{clauses}

\clause{|for| ~var~ |next| ~expr~}
\clindexx{for}{next}

~var~ is set to ~expr~ each time through the loop.  Destructuring is
performed.  When the clause is used as a generator, ~expr~ is the code
that is executed when |(next ~var~)| is 
encountered (see section \ref{generators}, below).
~expr~ should compute the first value for ~var~, as well
as all subsequent values, and is responsible for terminating the loop.
For compatibility with future versions of \iter, this termination
should be done with |terminate|\clindex{terminate}, which can be
considered a synonym for |finish| (see section \ref{control-flow}).

\cpar As an example, the following clauses are equivalent to |(for
i from 1 to 10)|:
\begin{program}
(initially (setq i 0))
(for i next (if (> i 10) (terminate) (incf i)))
\end{program}

\clause{|for| ~var~ |do-next| ~form~}
\clindexx{for}{do-next}
~form~ is evaluated each time through the loop.  Its value is ~not~
set to ~var~; that is ~form's~ job. ~var~ is only present so that
\iter\ knows it is a driver variable.  \linebreak |(for ~var~ next
~expr~)| is equivalent to |(for ~var~ do-next (dsetq ~var~ ~expr~))|.
(See section \ref{destructuring} for an explanation of |dsetq|.) 

\end{clauses}

\subsubsection{Generators}
\label{generators}

In all of the above clauses, the driver variable is updated on each
iteration.
Sometimes it is desirable to have greater control over updating.  
For instance, consider the problem of associating numbers, in
increasing order and with no gaps, with the
\nonnil\ elements of a list.  One obvious first pass at writing this is:

\begin{program}
(iter (for el in list)
      (for i upfrom 1)
      (if el (collect (cons el i))))
\end{program}
But on the list |(a b nil c)| this produces |((a . 1) (b . 2) (c .
4))| instead of the desired |((a . 1) (b . 2) (c . 3))|.  The problem
is that |i| is incremented each time through the loop, even when |el|
is |nil|.  

The problem could be solved elegantly if we could step |i| only when
we wished to.  This
can be accomplished for any \iter\ driver by writing |generate| (or
its synonym |generating|)
instead of |for|.  Doing so produces a ~generator~---a driver whose
values are yielded explicitly.  To obtain the next value of a
generator variable ~v~, write \linebreak |(next ~v~)|.  The value of a |next|
form is the next value of ~v~, as determined by its associated driver
clause.  |next| also has the side-effect of updating ~v~ to that
value.  If there is no next value, |next| will terminate the loop,
just as with a normal driver.


Using generators, we can now write our example like this:
\begin{program}
(iter (for el in list)
      (generate i upfrom 1)
      (if el (collect (cons el (next i)))))
\end{program}
Now |i| is updated only when |(next i)| is executed, and this occurs
only when |el| is \nonnil.

To better understand the relationship between ordinary drivers and
generators, observe that we can rewrite an ordinary driver using its
generator form immediately followed by |next|, as this example shows:
\begin{program}
(iter (generating i from 1 to 10)
      (next i)
      ...)
\end{program}
Provided that the loop body contains no |(next i)| forms, this will
behave just as if we had written |(for i from 1 to 10)|.

We can still refer to a driver variable ~v~ without using |next|; in
this case, its value is that given to it by the last evaluation of
|(next ~v~)|.  Before |(next ~v~)| has been called the first time, the
value of ~v~ is undefined.  

This semantics is more flexible than
one in which ~v~ begins the loop bound to its first value and calls
of |next| supply subsequent values, because it means the loop will not
terminate too soon if the generator's sequence is empty.  For
instance, consider the following code, which tags \nonnil\ elements of
a list using a list of tags, and also counts the null elements.
(We assume there are at least as many tags as \nonnil\ elements.)
\begin{program}
(let* ((counter 0)
       (tagged-list (iter (for el in list)
                          (generating tag in tag-list) 
                          (if (null el)
                              (incf counter)
                              (collect (cons el (next tag)))))))
  ...)
\end{program}
It may be that there are just as many tags as non-null elements of
|list|.  If all the elements of |list| are null, we still want the
counting to proceed, even though |tag-list| is \nil.  If |tag| had to be
assigned its first value before the loop begins, we would have had to
terminate the loop before the first iteration, since when |tag-list|
is \nil, |tag| 
has no first value.  With the existing semantics, however, |(next
tag)| will never execute, so the iteration will cover all the elements of
|list|. 
        
When the ``variable'' of a driver clause is actually a destructuring
template containing several variables, all the variables are eligible
for use with |next|.  As before, |(next ~v~)| evaluates to ~v's~ next
value; but the effect is to update all of the template's variables.
For instance, the following code will return the list |(a 2 c)|.
\begin{program}
(iter (generating (key . item) in '((a . 1) (b . 2) (c . 3)))
      (collect (next key))
      (collect (next item)))
\end{program}

Only driver clauses with variables can be made into generators.  This
includes all clauses mentioned so far except for |repeat|.  It does
~not~ include |for\dots previous|, |for\dots =|,
|for\dots initially\dots then| or |for\dots first\dots then| (see
below). 

\subsubsection{Previous Values of Driver Variables}

Often one would like to access the value of a variable on a previous
iteration.  \iter\ provides a special clause for accomplishing this.

\begin{clauses}

\clause{|for| ~pvar~ |previous| ~var~ \opt\ |initially| ~init~
|back| ~n~}
\clindexx{for}{previous}
Sets ~pvar~ to the previous value of ~var,~ which should be a driver
variable, a variable from another |for\dots previous| clause, or a
variable established by a |for\dots =|, 
|for\dots initially\dots then| or |for\dots first\dots then| clause
(see section \ref{setting}).
Initially, ~pvar~ is given the value ~init~ (which defaults to \nil).  
The ~init~ expression will be moved outside the loop body, so it
should not depend on anything computed within the loop.
~pvar~ retains the value of ~init~ until ~var~ is set to its second
value, at which point ~pvar~ is set to ~var's~ first value; and so on.  

\cpar The
argument ~n~ to |back| must be a constant, positive integer, and
defaults to 1.  It determines how many iterations back ~pvar~ should
track ~var~.  For example, when ~n~ is 2, then ~pvar~ will be assigned
~var's~ first value when ~var~ is set to its third value.

\cpar A |for\dots previous| clause may occur after or before its
associated driver clause. |for\dots previous| works with generators as
well as ordinary drivers.

\pagebreak[3]

\cpar Example:
\begin{program}
(iter (for el in '(1 2 3 4))
      (for p-el previous el)
      (for pp-el previous p-el initially 0)
      (collect pp-el))
\end{program}
This evaluates to |(0 0 1 2)|.  It could have been written more
economically as 
\begin{program}
(iter (for el in '(1 2 3 4))
      (for pp-el previous el back 2 initially 0)
      (collect pp-el))
\end{program}


\end{clauses}

\subsection{Variable Binding and Setting}
\label{setting}

Several clauses exist for establishing new variable bindings or for
setting variables in the loop.  They all support destructuring.

\begin{clauses}

\clausex{with}{~var~ \opt\ |=| ~value~}
Causes ~var~ to be bound to value before the loop body is entered.
If ~value~ is not supplied, ~var~ assumes a default
binding, which will be 
\nil\ in the absence of declarations.  Also, if ~value~ is not
supplied, no destructuring is performed; instead, ~var~ may be a list of
symbols, all of which are given default 
bindings.  If ~value~ is supplied, ~var~ is bound to it, with
destructuring.  

\cpar Because |with|
creates bindings whose scope includes the entire \iter\ form, it is
good style to put all |with| clauses at the beginning.

\cpar Successive occurrences of |with| result in sequential
bindings (as with 
|let*|).  There is no way to obtain parallel bindings; see
section \ref{bindings} for a rationale.  


\clause{|for| ~var~ |=| ~expr~}
\clindexx{for}{=}
On each iteration, ~expr~ is evaluated and ~var~ is set
to its value. 

\cpar This clause may appear to do the same thing as |for\dots next|.
In fact, they are quite different.  |for\dots =| provides only three
services: it sets up a binding for ~var~, sets it to ~expr~ on each
iteration, and makes it possible to use |for\dots previous| with
~var~.  |for\dots next| provides these services in addition to the
ability to turn the driver into a generator.
%Also, the code which sets ~var~ appears in the loop body in the same
%place as the |for\dots =| clause; the code for |for\dots next| appears
%at the top of the loop, as with other drivers (except when being used
%as a generator).

\clause{|for| ~var~ |initially| ~init-expr~ |then| ~then-expr~}
\clindexxx{for}{initially}{then}
Before the loop begins, ~var~ is set to ~init-expr;~ on all
iterations after the first it is set to ~then-expr.~  This clause
must occur at top-level.  ~init-expr~ will be moved outside the loop
body and ~then-expr~ will be moved to the end of the loop body, so
they are subject to code motion problems (see section
\ref{code-movement}). 

\cpar This clause may appear to be similar to |for\dots next|, but in
fact they differ significantly.  |for\dots initially\dots then| is
typically used to give ~var~ its first value before the loop begins,
and subsequent values on following iterations.  This is incompatible
with generators, whose first value and subsequent values must all be
computed by |(next ~var~)|.  Also, the update of ~var~ in 
|for\dots initially\dots then| does not occur at the location of the
clause. 
Use |for\dots initially\dots then| for
one-shot computations where its idiom is more convenient, but use
|for\dots next| for extending \iter\ with new drivers (see section
\ref{extend}).

\clause{|for| ~var~ |first| ~first-expr~ |then| ~then-expr~}
\clindexxx{for}{first}{then}
The first time through the loop, ~var~ is set to ~first-expr;~ on
subsequent iterations, it is set to ~then-expr.~  This differs from
|for\dots initially| in that ~var~ is set to ~first-expr~
inside the loop body, 
so ~first-expr~ may depend on the results of other clauses.  For
instance, 
\begin{program}
(iter (for num in list)
      (for i first num then (1+ i))
      ...)
\end{program}
will set |i| to the first element of |list| on the first
iteration, whereas 
\begin{program}
(iter (for num in list)
      (for i initially num then (1+ i))
      ...)
\end{program}
is probably erroneous; |i| will be bound to |num|'s
default binding (usually \nil) for the first iteration.

\end{clauses}

\begin{note}{Compatibility note}
|loop|'s |for\dots =| works like \iter's, but |loop| used the syntax
|for\dots =\dots then| to mean |for\dots initially\dots then|.  It was
felt that these two operations were sufficiently different to warrant
different keywords.

Also, the |for| in the above three clauses is misleading,
since none is true driver (e.g. none has a corresponding |generate|
form).  |setting| would have been a better choice, but |for| was used
to retain some compatibility with |loop|.  
\end{note}


%%%%%%%%%%%%%%%%%%%%%%%%%%%%%%%%
\subsection{Gathering Clauses}
Many of \iter's clauses accumulate values into a variable, or set a
variable under certain conditions.  At the end of the loop, this
variable contains the desired result.  All these clauses have an
optional |into| keyword, whose argument should be a symbol.  If the
|into| keyword is not supplied, the accumulation variable will be
internally generated and its value will be returned at the end of the
loop; if a variable is specified, that variable is used for the
accumulation, and is not returned as a result---it is up to the user
to return it explicitly, in the loop's epilogue code (see |finally|).
It is safe to examine the accumulation variable during the loop, but
it should not be modified.

These clauses all begin with a verb.  When the verb does
not conflict with an existing Common Lisp function, then it may be
used in either its infinitival or present-participle form (e.g. |sum|,
|summing|).  However, when there is a conflict with Common Lisp, only
the present-participle form may be used (e.g. |unioning|).  This is to
prevent \iter\ clauses from clashing with Common Lisp
functions. 

%although these clauses are described as ``producing a value,'' it is a
%mistake to think of the lisp list representing the clause as a
%value-producing form in the usual way.  clauses may legally be written
%where a value is expected, e.g. |(setq x (sum i))|, but the
%lisp value of a clause in such a context is undefined.

\subsubsection{Reductions}

~Reduction~ is an extremely common iteration pattern in which
the results of successive applications of a
binary operation are accumulated.
For example, a loop that computes the sum of the
elements of a list is performing a reduction with the addition
operation.  This could be written in Common Lisp as  |(reduce \#'+
list)| or with \iter\ as
\begin{program}
(iter (for el in list)
      (sum el))
\end{program}

\begin{clauses}

\clausex{sum}{~expr~ \opt\ |into| ~var~}
Each time through the loop, ~expr~ is evaluated and added to a
variable, which is bound initially to zero.  If ~expr~ has a type,
it is ~not~ used as the type of the sum variable, which is always
|number|.  To get the result variable to be of a more specific
type, use an explicit variable, as in
\begin{program}
(iter (for el in number-list)
      (sum el into x)
      (declare (fixnum x))
      (finally (return x)))
\end{program}

\clausex{multiply}{~expr~ \opt\ |into| ~var~}
Like |sum|, but the initial value of the result variable is $1$,
and the variable is updated by multiplying ~expr~ into it.

\clausex{counting}{~expr~ \opt\ |into| ~var~}
~expr~ is evaluated on each iteration.  If it is \nonnil, the
accumulation variable, initially zero, is incremented.

\startitem
\Clausex{maximize}{~expr~ \opt\ |into| ~var~}
\Clausex{minimize}{~expr~ \opt\ |into| ~var~}
\finishitem
~expr~ is evaluated on each iteration and its extremum (maximum or
minimum) is stored in the accumulation variable.  If ~expr~ is never
evaluated, then the result is \nil\ (if the accumulation variable is
untyped) or $0$ (if it has a numeric type).

\clausex{reducing}{~expr~ |by| ~func~ \opt\ |initial-value| ~init-val~ |into| ~var~}
This is a general way to perform reductions. ~func~ should be a
function of two arguments, the first of which will be the value
computed so far and
the second of which will be the value of ~expr~.  It should return the
new value.
|reducing| is roughly equivalent to the
Common Lisp |(reduce ~func~ ~sequence~ :key ~expr-function~)|, where
~expr-function~ is used to derive values from the successive elements
of ~sequence~.

\cpar If the |reducing| clause is never executed, the result is
undefined.

\cpar It is not necessary to provide an
initial value, but better code can be generated if one is supplied.
Regardless of its location in the \iter\ body, ~init-val~ will be
evaluated before the loop is entered, so it should not depend on any
value computed inside the \iter\ form.  

%\cpar if a ~var~ is not specified, you can get \iter\ to declare the
%type of the internal variable by putting a |the| expression around
%~func.~  see section \ref{types}.

\end{clauses}

%\begin{note}{implementation note}
%in principle, |maximize| and |minimize| can be thought of
%as reductions where the initial value is the smallest (or largest)
%value that the accumulation variable can assume.  because lisp's
%bignums can represent arbitrary integers, these clauses cannot be
%implemented as reductions in general.  if, however, the type of 
%~expr~ or ~var~ can be determined to be a fixnum
%or a float, \iter\ will implement the clause as a true reduction,
%using one of the constants |most-negative-fixnum|, |%most-positive-fixnum|, 
%|most-negative-short-float|, etc. as appropriate.
%\end{note}

\subsubsection{Accumulations}

All the predefined accumulation clauses add values to a sequence.  If
the sequence is a list, they all 
have the property that the partial list is kept in the correct order
and available for inspection at any point in the loop.

\begin{clauses}

\clausex{collect}{~expr~ \opt\ |into| ~var~ |at| ~place~ |result-type| ~type~}
Produces a sequence of the values of ~expr~ on each iteration. ~place~
indicates where the next value of ~expr~ is added to the list and may
be one of the symbols |start|, |beginning| (a synonym for |start|) or
|end|.  The symbol may be quoted, but need not be.  The default is
|end|.  For example,
\begin{program}
(iter (for i from 1 to 5)
      (collect i))
\end{program}
produces |(1 2 3 4 5)|, whereas
\begin{program}
(iter (for i from 1 to 5)
      (collect i at beginning))
\end{program}
produces |(5 4 3 2 1)| (and is likely to be faster in most Common Lisp
implementations).

\cpar If ~type~ is provided, it should be a subtype of |sequence|.
The default is |list|.  Specifying a type other than |list| will
result in |collect| returning a sequence of that type.  ~However,~ the
type of the sequence being constructed when inside the loop body is
undefined when a non-|list| type is specified.  (As with ~place~,
quoting ~type~ is optional.)

\clausex{adjoining}{~expr~ \opt\ |into| ~var~ |test| ~test~ |at| ~place~ |result-type| ~type~}
Like |collect|, but only adds the value of ~expr~ if it is not
already present.  ~test,~ which defaults to |\#'eql|, is
the test to be used with |member|.

\startitem
\Clausex{appending}{~expr~ \opt\ |into| ~var~ |at| ~place~}
\Clausex{nconcing}{\ ~expr~ \opt\ |into| ~var~ |at| ~place~}
\Clausex{unioning}{\ ~expr~ \opt\ |into| ~var~ |test| ~test~ |at| ~place~}
\Clausex{nunioning}{~expr~ \opt\ |into| ~var~ |test| ~test~ |at| ~place~}
\finishitem
These are like |collect|, but behave like the Common Lisp functions
|append|, |nconc|, |union| or |nunion|.  
As in Common Lisp, they work only on lists.  Also as in Common Lisp,
|unioning| and |nunioning| assume that the value of ~expr~ contains no
duplicates. 

\clausex{accumulate}{~expr~ |by| ~func~ \opt\ |initial-value| ~init-val~ |into| ~var~}
This is a general-purpose accumulation clause. ~func~ should be a
function of two arguments, the value of ~expr~ and the value
accumulated so far in the iteration, and it should return the updated
value.  If no initial value is supplied, \nil\ is used.  

%\cpar If a ~var~ is not specified, you can get \iter\ to declare the
%type of the internal variable by putting a |the| expression around
%~func.~  see section \ref{types}.

\cpar The differences between |accumulate| and |reducing| are slight.
One difference is that the functions take their arguments in a
different order.  Another is that in the absence of ~init-val~,
|accumulate| will use \nil, whereas |reducing| will generate different
code that avoids any dependence on the initial value.
The reason for having both clauses is that one usually
thinks of reductions (like |sum|) and accumulations (like |collect|)
as different beasts.

\end{clauses}


\subsubsection{Finders}

A ~finder~ is a clause whose value is an expression that meets some
condition.

\begin{clauses}

\clause{|finding| ~expr~ |such-that| ~test~ \opt\ |into| ~var~ |on-failure| ~failure-value~}
\clindexx{finding}{such-that}
If ~test~ (which is an expression) ever evaluates to \nonnil, the loop
is terminated, the 
epilogue code is run and the value of ~expr~ is returned.  Otherwise,
\nil\ (or ~failure-value,~ if provided) is returned.  If ~var~ is
provided, it will have either the \nonnil\ value of ~expr~ or
~failure-value~ when the epilogue code is run.

\cpar As a special case, if the ~test~ expression is a sharp-quoted
function, then it is applied to ~expr~ instead of being simply
evaluated.  E.g. |(finding x such-that \#'evenp)| is equivalent to
|(finding x such-that (evenp x))|.  

%\cpar although ~test~ need have nothing to do with ~%expr~ as in
%|(finding j such-that (> i 3))|, it usually 
%will: |(finding (length el) such-that (oddp (length el)))|.  to
%avoid performing the |length| computation twice, you could write
%|(finding (length el) such-that \#'oddp)| or |(finding (length
%el) such-that 'oddp)|; for these cases, \iter\ generates code that
%executes ~expr~ only once.  the code for |\#'oddp|
%is slightly different from that for {\lisp 'oddp}; see the discussion
%under {\lisp for\dots in} and {\lisp for\dots on}.

\cpar |On-failure| is a misnomer. Because it is always evaluated, it behaves
more like the default third argument to the |gethash| function. As a result,
|on-failure (error "Not found")| makes no sense. Instead, the clauses |leave|
or |thereis| can be used in conjunction with |finally| as follows:
\begin{program}
(iter (for x in '(1 2 3))
      (if (evenp x) (leave x))
      (finally (error "not found")))
\end{program}

\cpar This clause may appear multiple times when all defaults are
identical. It can also be used together with either |always|/|never| or
|thereis| if their defaults match. More specifically, |on-failure nil| is
compatible with |thereis|, while |on-failure t| is compatible with |always|
and |never| clauses.
\begin{program}
(iter (for i in '(7 -4 2 -3))
      (if (plusp i)
	  (finding    i  such-that (evenp i))
          (finding (- i) such-that (oddp i))))
\end{program}

\startitem
\Clause{|finding| ~expr~ |maximizing| ~m-expr~ \opt\ |into| ~var~}
\clindexx{finding}{maximizing}
\Clause{|finding| ~expr~ |minimizing| ~m-expr~ \opt\ |into| ~var~}
\clindexx{finding}{minimizing}
\finishitem
Computes the extremum (maximum or minimum) value of ~m-expr~ over all
iterations, and returns the value of ~expr~ corresponding to the
extremum.  ~expr~ is evaluated inside the loop at the time the new
extremum is established.  If ~m-expr~ is never evaluated (due to, for
example, being embedded in a conditional clause), then the returned
value depends on the type, if any, of ~expr~ (or ~var,~ if
one is supplied).  If there is no type, the returned
value will be \nil; if the type is numeric, the returned value will be
zero.

\cpar For these two clauses, ~var~ may be a list of two
symbols; in that case, the first is used to record ~expr~ and
the second, ~m-expr.~  

\cpar As with |finding\dots such-that|, if ~m-expr~ is a sharp-quoted
function, then it is called on ~expr~ instead of being evaluated.

\end{clauses}

\subsubsection{Boolean Tests}

\begin{clauses}

\clausex{first-iteration-p}{}
Returns |t| in the first cycle of the loop, otherwise \nil.

\clausex{first-time-p}{}
Returns |t| the first time the expression is evaluated, and then \nil\ forever.
This clause comes handy when printing (optional) elements separated
by a comma:

\begin{program}
(iter (for el in '(nil 1 2 nil 3))
      (when el
        (unless (first-time-p)
          (princ ", "))
        (princ el)))
\end{program}
produces |"1, 2, 3"|.

\end{clauses}

\subsubsection{Aggregated Boolean Tests}

\begin{clauses}

\clausex{always}{~expr~}
If ~expr~ ever evaluates to
\nil, then \nil\ is immediately returned; the epilogue code is not
executed.  If ~expr~ never evaluates to \nil, the epilogue code
is executed and the last value of ~expr~ (or |t| if ~expr~ was never
evaluated) is returned (whereas |loop| would constantly return |t|).

% mention last evaluated clause when multiple always clauses?

\clausex{never}{~expr~}
Like |(always (not ~expr~))|, except it does not influence the last
value returned by a possible other |always| clause. That is,
\begin{program}
(iter (repeat 2)
      (always 2)
      (never nil)) \yields 2 ; not t
\end{program}

\clausex{thereis}{~expr~}
If ~expr~ is ever \nonnil,
its value is immediately returned without running epilogue code.
Otherwise, the epilogue code is performed and \nil\ is returned.

This clause cannot be used together with |always| or |never|, because their
defaults are opposed (similarly, |(loop always 3 thereis nil)| refuses to
compile in some implementations of |loop|).

\end{clauses}


\subsection{Control Flow}
\label{control-flow}
Several clauses can be used to alter the usual flow of control in a loop.

Note: the clauses of this and subsequent sections don't adhere to \iter's
usual syntax, but instead use standard Common Lisp syntax.  Hence the
format for describing syntax subsequently is
like the standard format used in the Common Lisp manual, not like the
descriptions of clauses above.

\begin{clauses}

\clausex{finish}{}
Stops the loop and runs the epilogue code.

%for example:
%
%\begin{program}
%(iter (with answer = nil)
%         (initially (make-a-mess))
%         (for i from 1 to 10)
%         (when (correct? i)
%           (setq answer i)
%           (finish))
%         (finally (cleanup)))
%\end{program}
%
%this code will execute |cleanup| whether or not the test |(correct?
%i)| ever succeeds. 
%the (more elegant) formulation,
%\begin{program}
%(iter (initially (make-a-mess))
%         (for i from 1 to 10)
%         (finding i such-that (correct? i))
%         (finally (cleanup)))
%\end{program}
%would not execute |cleanup| if |(correct? i)| succeeded; it
%would do an immediate return.

\clausex{leave}{\opt\ ~value~}
Immediately returns ~value~ (default \nil) from the current \iter\
form, skipping the epilogue code.  Equivalent to using |return-from|.

\clausex{next-iteration}{}
Skips the remainder of the loop body and begins the next iteration of
the loop.

\clausex{while}{~expr~}
If ~expr~ ever evaluates to \nil, the loop is terminated and the
epilogue code executed.  Equivalent to |(if (not ~expr~) (finish))|.

\clausex{until}{~expr~}
Equivalent to |(if ~expr~ (finish))|.

\clausex{if-first-time}{~then~ \opt\ ~else~}
If this clause is being executed for the first time in this invocation
of the \iter\ form, then the ~then~ code is evaluated; otherwise the
~else~ code is evaluated.

\cpar |(for ~var~ first ~expr1~ then ~expr2~)| is almost equivalent to
\begin{program}
(if-first-time (dsetq ~var~ ~expr1~)
               (dsetq ~var~ ~expr2~))
\end{program}
The only difference is that the |for| version makes ~var~ available
for use with |for\dots previous|.

\end{clauses}

\subsection{Code Placement}
When fine control is desired over where code appears in a loop
generated by \iter, the following special clauses may be useful.
They are all subject to code-motion problems (see section
\ref{code-movement}).  

\begin{clauses}

\clausex{initially}{|\&rest| ~forms~}
The lisp ~forms~ are placed in the prologue section of the loop, where
they are executed once, before the loop body is entered.

\clausex{after-each}{|\&rest| ~forms~}
The ~forms~ are placed at the end of the loop body, where they
are executed after each iteration.  Unlike the other clauses in this
section, ~forms~ may contain \iter\ clauses.

\clausex{else}{|\&rest| ~forms~}
The lisp ~forms~ are placed in the epilogue section of the loop, where they
are executed if this |else| clause is never met during execution of the
loop and the loop terminates normally.

\clausex{finally}{|\&rest| ~forms~}
The lisp ~forms~ are placed in the epilogue section of the loop, where
they are executed after the loop has terminated normally.

\clausex{finally-protected}{|\&rest| ~forms~}
The lisp ~forms~ are placed in the second form of an unwind-protect
outside the loop.  They are always executed after the loop has
terminated, regardless of how the termination occurred.

\end{clauses}


%%%%%%%%%%%%%%%%%%%%%%%%%%%%%%%%%%%%%%%%%%%%%%%%%%%%%%%%%%%%%%%%
\section{Other Features}

\subsection{Multiple Accumulations}
\label{multiple}

\begin{sloppypar}
It is permitted to have more than one clause accumulate into the same
variable, as in the following:
\begin{program}
(iter (for i from 1 to 10)
      (collect i into nums)
      (collect (sqrt i) into nums)
      (finally (return nums)))
\end{program}
Clauses can only accumulate into the same variable if they are
compatible.  |collect|, |adjoining|, |appending|, |nconcing|,
|unioning| and |nunioning| are compatible with each other; |sum|,
|multiply| and |counting| are compatible; |always| and |never| are compatible;
|finding| \dots |such-that| is compatible with either |thereis| or |always|
and |never| when their defaults match; and |maximize| and |minimize| clauses
are compatible only with other |maximize| and |minimize| clauses,
respectively.

%note that the same variable ~cannot~ be both an accumulation
%variable and an ordinary variable; there can be only one variable with
%a given name within an \iter\ form. 
\end{sloppypar}

\subsection{Named Blocks}

Like Common Lisp |block|s, \iter\ forms can be given names.  The
name should be a single symbol, and it must be the first form in the
\iter.  The generated code behaves exactly like a named block; in
particular, |(return-from ~name~)| can be used to exit it:
\begin{program}
(iter fred
      (for i from 1 to 10)
      (iter barney
            (for j from i to 10)
            (if (> (* i j) 17)
                (return-from fred j))))
\end{program}
An \iter\ form that is not given a name is implicitly named \nil.

Sometimes one would like to write an expression in an inner \iter\ form,
but have it processed by an outer \iter\ form.  This is possible with
the |in| clause.  

\begin{clauses}

\clausex{in}{~name~ |\&rest| ~forms~}
Evaluates ~forms~ as if they were part of the \iter\ form named
~name~.  In other words, \iter\ clauses are processed by the \iter\
form named ~name,~ and not by any \iter\ forms that occur inside ~name.~

\cpar As an example, consider the problem of collecting a list of the
elements in a two-dimensional array.  The naive solution,
\begin{program}
(iter (for i below (array-dimension ar 0))
      (iter (for j below (array-dimension ar 1))
            (collect (aref ar i j))))
\end{program}
\noindent is wrong because the list created by the inner \iter\ is simply
ignored by the outer one.  But using |in| we can write:
\begin{program}
(iter outer (for i below (array-dimension ar 0))
      (iter (for j below (array-dimension ar 1))
            (in outer (collect (aref ar i j)))))
\end{program}
\noindent which has the desired result.

\end{clauses}

\subsection{Destructuring}
\label{destructuring}

In many places within \iter\ clauses where a variable is expected, a
list can be written instead.  In these cases, the value to be assigned
is ~destructured~ according to the pattern
described by the list.  As a simple example, the clause
\begin{program}
(for (key . item) in alist)
\end{program}
\noindent will result in |key| being set to the |car| of
each element in |alist|, and |item| being set to the |cdr|.  The
pattern list may be nested to arbitrary depth, and (as the example
shows) need not be terminated with \nil; the only requirement is that
each leaf be a bindable symbol (or \nil, in which case no binding is
generated for that piece of the structure).

Sometimes, you might like to do the equivalent of a
|multiple-value-setq| in a clause.  This 
``multiple-value destructuring'' can be expressed by writing 
\linebreak |(values $pat_1$ $pat_2 \ldots$)| for a destructuring
pattern, as in 
\begin{program}
(for (values (a . b) c d) = (three-valued-function ...))
\end{program}
\begin{sloppypar}
Note that the $pat_i$ can themselves be destructuring patterns (though
not multiple-value destructuring patterns).  You can't do multiple-value
destructuring in a |with| clause; instead wrap the whole \iter\
form in a |multiple-value-bind|.
\end{sloppypar}

\begin{note}{Rationale}
There are subtle interactions between variable declarations and
evaluation order that make the correct implementation of
multiple-value destructuring in a |with| somewhat tricky.
\end{note}

The destructuring feature of \iter\ is available as a separate
mechanism, using the |dsetq| macro:

\begin{clauses}
\defmacro{dsetq}{template expr}
Performs destructuring of ~expr~ using ~template~.   May be used
outside of an \iter\ form. Yields the primary value of ~expr~.

\end{clauses}

\subsection{On-line Help}

\begin{sloppypar}
There is a limited facility for on-line help, in the form of the 
|display-iterate-clauses| function.
\end{sloppypar}

\begin{clauses}

\defun{display-iterate-clauses}{\opt\ clause-spec}
Displays a list of \iter\ clauses.  If ~clause-spec~ is not
provided, all clauses are shown; if it is a symbol, all clauses
beginning with that symbol are shown; and if it is a list of symbols,
all clauses for which ~clause-spec~ is a prefix are shown.

\end{clauses}

\subsection{Parallel Binding and Stepping}
\label{bindings}

The parallel binding and stepping of variables is a feature that
\iter\ does ~not~ have.  This section attempts to provide a rationale.

We say that two variables are bound ~in parallel~ if neither
binding shadows the other.  This is the usual semantics of |let|
(as opposed to |let*|).  Similarly, we can say that iteration
variables are stepped in parallel if neither variable is updated
before the other, conceptually speaking; in other words, if the code
to update each variable can reference the old values of both variables.

|loop| allows parallel binding of variables and parallel stepping of
driver variables.  My view is that if you are depending on the 
serial/parallel distinction, you are doing something obscure.  If you
need to bind 
variables in parallel using |with|, then you must be using a
variable name that shadows a name in the existing lexical environment.
Don't do that.  The most common use for parallel stepping is to track
the values of variables on the previous iteration, but in fact this
does not require parallel stepping at all; the following will work:
\begin{program}
(iter (for current in list)
      (for prev previous current)
      ...)
\end{program}


%%%%%%%%%%%%%%%%%%%%%%%%%%%%%%%%%%%%%%%%%%%%%%%%%%%%%%%%%%%%%%%%
\section{Types and Declarations}
\label{types}

\subsection{Discussion}

Sometimes efficiency dictates that the types of variables be declared.
This type information needs to be communicated to \iter\ so it can
bind variables to appropriate values.  Furthermore, \iter\ must often
generate internal variables invisible to the user; there needs to be a
way for these to be declared.

As an example, consider this code, which will return the number of
odd elements in |number-list|:
\begin{program}
(iter (for el in number-list)
      (count (oddp el)))
\end{program}
In processing this form,
\iter\ will create an internal variable, let us call it |list17|, to
hold the successive |cdr|s of |number-list|, and  
will bind the variable to |number-list|.  It will also generate a
default binding for |el|; only inside the body of the loop will |el|
be set to the |car| of |list17|.  Finally, \iter\ will generate a
variable, call it |result|, to hold the result of the count, and will
bind it to zero.

When dealing with type declarations, \iter\ observes one simple rule:
~it will never generate a declaration unless requested to do so.~  The
reason is that such declarations might mask errors in compiled code by
avoiding error-checks; the resulting problems would be doubly hard to
track down because the declarations would be hidden from the
programmer.  Of course, a compiler might omit error-checks even in the
absence of declarations, though this behavior can usually be avoided,
e.g. by saying |(declaim (optimize (safety 3)))|.

So, the above \iter\ form will generate code with no declarations.
But say we wish to declare the types of |el| and the internal
variables |list17| and |result|.  How is this done?

Declaring the type of |el| is easy, since the programmer knows
the variable's name:
\begin{program}
(iter (for el in number-list)
      (declare (fixnum el))
      (counting (oddp el)))
\end{program}
\iter\ can read variable type declarations like this one.  Before
processing any clauses, it scans the entire top-level form for type
declarations 
and records the types, so that variable bindings can be performed
correctly.  In this case, |el| will be bound to zero
instead of \nil.  Also, \iter\ collects all the top-level declarations
and puts them at the begining of the generated code, so it is not
necessary to place all declarations at the beginning of an \iter\
form; instead, they can be written near the variables whose types they
declare.  

Since \iter\ is not part of the compiler, it will not know
about declarations that occur outside an \iter\ form; these
declarations must be repeated inside the form.

Here is another way we could have declared the type of |el|:
\begin{program}
(iter (for (the fixnum el) in number-list)
      (counting (oddp el)))
\end{program}
\iter\ extends the Common Lisp |the|\lispindex{the}
form to apply to variables as well as value-producing forms; anywhere
a variable is allowed---in a |with| clause, as the iteration
variable in a driver clause, as the |into| argument of an
accumulation clause, even inside a destructuring template---you can
write |(the ~type~ ~symbol~)| instead.

There is one crucial difference between using a |the| form and
actually declaring the variable: explicit declarations are always
placed in the generated code, but type information from a |the|
form is not turned into an actual declaration unless you tell \iter\
to do so using |iterate:declare-variables|.  See below.

\begin{sloppypar}
Declaring the types of internal variables is harder than declaring the
types of explicitly mentioned variables, since their names
are unknown.  You do it by declaring |iterate:declare-variables|
somewhere inside the top level of the \iter\ form.  (This will also
generate declarations for variables declared using |the|.)
\iter\ does not provide much selectivity here: it's all or none. 
And unfortunately, since \iter\ is not privy to compiler information
but instead reads declarations itself, it will not hear if you
|(declaim (iterate:declare-variables))|.  Instead, set the variable 
|iterate::*always-declare-variables*| to |t| at
compile-time, using |eval-when|.
\end{sloppypar}

To determine the appropriate types for internal variables, \iter\ uses
three sources of information:

\begin{itemize}

\item Often, the particular clause dictates a certain type for a
variable; \iter\ will use this information when available.  In the
current example, the variable |list17| will be given the type
|list|, since that is the only type that makes sense; and the
variable |result| will be given the type |fixnum|, on the
assumption that you will not be counting high enough to need bignums.
You can override this assumption only by using and explicitly declaring a
variable: 
\begin{verbatim}
(iter (declare (iterate:declare-variables))
      (for el in number-list)
      (count (oddp el) into my-result)
      (declare (integer my-result))
      (finally (return my-result)))
\end{verbatim}

\begin{sloppypar}
Other examples of the type assumptions that \iter\ makes are: type
|list| for |into| variables of collection clauses; type |list| for
expressions that are to be destructured; type |vector| for the
variable holding the vector in a |for\dots in-vector| clause, and
similarly for |string| and the |for\dots in-string| clause;
and the implementation-dependent type
|(type-of array-dimension-limit)| for the index and limit
variables generated by sequence iteration drivers like |for\dots
in-vector| and |for\dots in-string| (but not |for\dots in-sequence|,
because it may be used to iterate over a list). 
\end{sloppypar}

\item Sometimes, \iter\ will examine expressions and try to determine
their types in a simple-minded way.  If the expression is
self-evaluating (like a number, for instance), \iter\ knows that the
expression's type is the same as the type of the value it denotes, so
it can use that type.  If the expression is of the form |(the ~type~
~expr~)|, \iter\ is smart enough to extract ~type~ and use it.
However, the current version of \iter\ does not  
examine declarations of function result types or do any type
inference.  It will not determine, for 
example, that the type of |(+ 3 4)| is |fixnum|, or even
|number|.

\item In some cases, the type of an internal variable should match the
type of some other variable.  For instance, \iter\ generates an
internal variable for |(f x)| in the
clause |(for i from 1 to (f x))|, and in the absence of other
information will give it the same type as |i|.  If, however, the
expression had been written |(the fixnum (f x))|, then \iter\
would have given the internal variable the type |fixnum|
regardless of |i|'s type.  The type incompatibility errors that
could arise in this situation are not checked for.

\end{itemize}

Note that if you do declare |iterate:declare-variables|, then
\iter\ may declare user variables as well as internal ones if they do
not already have declarations, though only for variables that it
binds.  For instance, in this code:

\begin{program}
(iter (declare (iterate:declare-variables))
      (for i from 1 to 10)
      (collect i into var))
\end{program}
the variable |var| will be declared to be of type |list|.


\subsection{Summary}

\iter\ understands standard Common Lisp variable type declarations
that occur within an \iter\ form and
will pass them through to the generated code.  If the declaration
|(iterate:declare-variables)|\lispindex{declare-variables}
appears at the top level of an 
\iter\ form, or if 
|iterate::*always-declare-variables*|\lispindex{*always-declare-variables*} 
is \nonnil, then \iter\ will use the type information gleaned from user
declarations, self-evaluating expressions and |the| expressions,
combined with reasonable assumptions, to determine variable
types and declare them.


%%%%%%%%%%%%%%%%%%%%%%%%%%%%%%%%%%%%%%%%%%%%%%%%%%%%%%%%%%%%%%%%
\section{Problems with Code Movement}
\label{code-movement}

Some \iter\ clauses, or parts of clauses, result in code being
moved from the location of the clause to other parts of the loop.
Drivers behave this way, as do code-placement clauses like |initially|
and |finally|.  When using these clauses, there is a danger of writing
an expression that makes sense in its apparent location but will be
invalid or have a different meaning in another location.  For example:
\begin{program}
(iter (for i from 1 to 10)
      (let ((x 3))
        (initially (setq x 4))))
\end{program}
While it may appear that the |x| of |(initially (setq x 4))| is the
same as the |x| of |(let ((x 3)) \dots|, in fact they are not:
|initially| moves its code outside the loop body, so |x| would refer
to a global variable.  Here is another example of the same problem:
\begin{program}
(iter (for i from 1 to 10)
      (let ((x 3))
        (collect i into x)))
\end{program}
If this code were executed, |collect| would create a binding for its
|x| at the top level of the \iter\ form that the |let| will shadow.

Happily, \iter\ is smart enough to catch these errors; it
walks all problematical code to ensure that free variables are not
bound inside the loop body, and checks all variables it binds for the
same problem.

However, some errors cannot be caught:

\begin{program}
(iter (with x = 3)
      (for el in list)
      (setq x 1)
      (reducing el by \#'+ initial-value x))
\end{program}
|reducing| moves its |initial-value| argument to the initialization
part of the loop in order to produce more efficient code.  Since
\iter\ does not perform data-flow analysis, it cannot determine that
|x| is changed inside the loop; all it can establish is that |x| is
not bound internally.  Hence this code will not signal an
error and will use $3$ as the initial value of the reduction.

The following list summarizes all cases that are subject to these code
motion and variable-shadowing problems.
\begin{itemize}
\item Any variable for which \iter\ creates a binding, including those
used in |with| and the |into| keyword of many clauses.

\begin{sloppypar}
\item The special clauses which place code: |initially|, |after-each|, |else|,
|finally| and |finally-protected|. 
\end{sloppypar}

\item The variables of a |next| or |do-next| form.

\item The |initially| arguments of |for\dots initially\dots then| and
|for\dots previous|. 

\item The |then| argument of |for\dots initially\dots then|.

\item The |initial-value| arguments of |reducing| and |accumulate|.

\item The |on-failure| argument of |finding\dots such-that|.
\end{itemize}

%%%%%%%%%%%%%%%%%%%%%%%%%%%%%%%%%%%%%%%%%%%%%%%%%%%%%%%%%%%%%%%%
\section{Differences Between {\tt Iterate} and {\tt Loop}}
|loop| contains a great deal of complexity which \iter\ tries to
avoid.  Hence many esoteric features of |loop| don't exist in \iter.
Other features have been carried over, but in a cleaned-up form.  
And of course, many new features have been added; they are not
mentioned in this list.

\begin{itemize}

\item \iter's syntax is more Lisp-like than |loop|'s, having a higher
density of parens.

\item The current implementation of \iter, unlike the current version
of  |loop| (as documented in {\em Common Lisp, 2nd Ed.\/}), is
extensible (see section \ref{extend}).  

\item |loop| puts the updates of all driver variables at the top of
the loop; \iter\ leaves them where the driver clauses appear.

\item While for the most part \iter\ clauses that resemble |loop| clauses
behave similarly, there are some differences.  For instance, there is
no |for\dots =\dots then| in \iter; instead use 
|for\dots initially\dots then|.  

\item |loop| binds the variable |it| at certain times
to allow pseudo-English expressions like |when ~expr~ return it|. 
In \iter, you must bind ~expr~ to a variable yourself.  Note that
|when ~expr~ return it| is like |thereis ~expr~| except that the latter is an
accumulation clause and therefore competes with other accumulations
(remember section \ref{multiple} above).

% repeat different behaviour of |always| clause here?

\item |loop| has a special |return| clause, illustrated in the
previous item.  \iter\ doesn't need one,  since an ordinary Lisp
|return| has the same effect.

\item |loop| allows for parallel binding and stepping of iteration
variables.  \iter\ does not.  (See section \ref{bindings}.)

\item |loop| and \iter\ handle variable type declarations very
differently.  |loop| provides a special syntax for declaring variable
types, and does not examine declarations.  Moreover, the standard
implementation of |loop| will
generate declarations when none are requested.
\iter\ parses standard Common Lisp type declarations, and will never
declare a variable itself unless declarations are 
specifically requested.

\end{itemize}

%%%%%%%%%%%%%%%%%%%%%%%%%%%%%%%%%%%%%%%%%%%%%%%%%%%%%%%%%%%%%%%%
\section{Rolling Your Own}
\label{extend}

\subsection{Introduction}

\iter\ is extensible---you can write new clauses that embody new
iteration patterns.  You might want to write a new driver clause for a
data structure of your own, or you might want to write a clause that
collects or manipulates elements in a way not provided by \iter. 

This section describes how to write clauses for \iter.  Writing a
clause is like writing a macro.  In fact, writing a clause ~is~
writing a macro: since \iter\ code-walks its body and macroexpands,
you can add new abstractions to \iter\ with good old |defmacro|.

Actually, there are two extensions you can make to \iter\ that are
even easier than writing a macro.  They are adding a synonym for an
existing clause and defining a driver clause for an indexable
sequence.  These can be done with |defsynonym| and
|defclause-sequence|, respectively.  See section \ref{aids}, below.

The rest of this section explains how to write macros that expand into
\iter\ clauses.
Here's how you could add a simplified version of \iter's |multiply|
clause, if \iter\ didn't already have one:
\begin{program}
(defmacro multiply (expr)
  `(reducing ,expr by \#'* initial-value 1))
\end{program}

If you found yourself summing the square of an expression often, you
might want to write a macro for that.  A first cut might be
\begin{program}
(defmacro sum-of-squares (expr)
  `(sum (* ,expr ,expr)))
\end{program}
but if you are an experienced macro writer, you will realize that this
code will evaluate ~expr~ twice, which is probably a bad idea.  A
better version would use a temporary:
\begin{program}
(defmacro sum-of-squares (expr)
  (let ((temp (gensym)))
    `(let ((,temp ,expr))
       (sum (* ,temp ,temp)))))
\end{program}
Although this may seem complex, it is just the sort of thing you'd
have to go through to write any macro, which illustrates the point of
this section: if you can write macros, you can extend \iter.

Our macros don't use \iter's keyword-argument
syntax.  We could just use keywords with |defmacro|, but we would
still not be using \iter's clause indexing mechanism.  Unlike Lisp,
which uses just the first symbol of a form to determine what function
to call, \iter\ individuates clauses by the list of required keywords.
For instance, |for\dots in| and |for\dots in-vector| are different clauses
implemented by distinct Lisp functions.  

To buy into this indexing scheme, as well as the keyword-argument
syntax, use |defmacro-clause|:

\begin{clauses}

\defmacro{defmacro-clause}{arglist |\&body| body}
Defines a new \iter\ clause.  ~arglist~ is a list of symbols which are
alternating keywords and arguments.  \opt\ may be used, and the list
may be terminated by |\&sequence|.  ~body~ is an ordinary macro body,
as with |defmacro|.  If the first form of ~body~ is a string, it is
considered a documentation string and will be shown by
|display-iterate-clauses|. |defmacro-clause| will signal an error if 
defining the clause would result in an ambiguity.  E.g. you cannot
define the clause |for\dots from| because there would be no way to
distinguish it from a use of the |for| clause with optional keyword |from|.

\end{clauses}

\medskip

Here is |multiply| using |defmacro-clause|.  The keywords are capitalized
for readability.
\begin{program}
(defmacro-clause (MULTIPLY expr \opt\ INTO var)
  `(reducing ,expr by \#'* into ,var initial-value 1))
\end{program}
You don't have to worry about the case when |var| is not supplied; for
any clause with an |into| keyword, saying |into nil| is equivalent to
omitting the |into| entirely.

As another, more extended example, consider the fairly common
iteration pattern that 
involves finding the sequence element that maximizes (or minimizes) some
function.  \iter\ provides this as |finding\dots maximizing|, but it's
instructive to see how to write it.
Here, in pseudocode, is how you might write such a loop for
maximizing a function F:

\begin{quote}
\begin{tabbing}
set variable MAX-VAL to NIL; \\
set variable WINNER to NIL; \\
for\=\ each element EL in the sequence \\
\>  if\=\ MAX-VAL\=\ is NIL or F(EL) $>$ MAX-VAL then \\
\>\>    set MAX-VAL to F(EL); \\
\>\>    set WINNER to EL; \\
\>  end if; \\
end for; \\
return WINNER.
\end{tabbing}
\end{quote}

Here is the macro:
\begin{program}
(defmacro-clause (FINDING expr MAXIMIZING func \opt\ INTO var)
  (let ((max-val (gensym))
        (temp1 (gensym))
        (temp2 (gensym))
        (winner (or var iterate::*result-var*)))
    `(progn 
       (with ,max-val = nil)
       (with ,winner = nil)
       (cond 
        ((null ,max-val)
         (setq ,winner ,expr)
         (setq ,max-val (funcall ,func ,winner))
        (t
         (let* ((,temp1 ,expr)
                (,temp2 (funcall ,func ,temp1)))
           (when (> ,temp2 ,max-val)
             (setq ,max-val ,temp2)
             (setq ,winner ,temp1))))))
       (finally (leave ,winner)))))
\end{program}
Note that if no |into| variable is supplied, we use
|iterate::*result-var*|, which contains the internal variable into
which all clauses place their results.  If this variable is bound by
some clause, then \iter\ will return its value automatically;
otherwise, \nil\ will be returned.  

\subsection{Writing Drivers}

In principle, drivers can be implemented just as easily as other
\iter\ clauses.  In practice, they are a little harder to get right. 
As an example, consider writing a driver that
iterates over all the 
elements of a vector, ignoring its fill-pointer.  |for\dots in-vector|
won't work for this, because it observes the fill-pointer.  It's necessary to
use |array-dimension| instead of |length| to obtain the size of the
vector.  Here is one approach:

\begin{program}
(defmacro-clause (FOR var IN-WHOLE-VECTOR v)
  "All the elements of a vector (disregards fill-pointer)"
  (let ((vect (gensym))
        (index (gensym)))
    `(progn
       (with ,vect = ,v)
       (for ,index from 0 below (array-dimension ,vect 0))
       (for ,var = (aref ,vect ,index)))))
\end{program}
Note that we immediately put |v| in a variable, in case it is an
expression.  Again, this is just good Lisp macrology.  It also has a
subtle effect on the semantics of the driver: |v| is evaluated only
once, at the beginning of the loop, so changes to |v| in the loop have
no effect on the driver.  Similarly, the bounds for numerical iteration
e.g. the above |array-dimension| are also evaluated once only.  This is how
all of \iter's drivers work.


There is an important point concerning the |progn| in this code.  We
need the |progn|, of course, because we are returning several forms,
one of which is a driver.  But \iter\ drivers must occur at top-level.
Is this code in error?  No, because ~top-level~ is defined in \iter\
to include forms inside a |progn|.  This is just the definition of
top-level that Common Lisp uses, and for the same reason: to allow
macros to return multiple forms at top-level.

While our |for\dots in-whole-vector| clause will work, it is not
ideal.  In particular, it does not support generating.  Do do so, we
need to use |for\dots next| or |for\dots do-next|.  The job is
simplified by the |defmacro-driver| macro.

\begin{clauses}

\defmacro{defmacro-driver}{arglist |\&body| body}
Defines a driver clause in
both the |for| and |generate| forms, and provides a parameter
|generate| which ~body~ can examine to determine how it was invoked.
~arglist~ is as in |defmacro-clause|, and should begin with the symbol
|for|. 

\end{clauses}

With |defmacro-driver|, our driver looks like this:
\begin{program}
(defmacro-driver (FOR var IN-WHOLE-VECTOR v)
  "All the elements of a vector (disregards fill-pointer)"
   (let ((vect (gensym))
         (end (gensym))
         (index (gensym))
         (kwd (if generate 'generate 'for)))
     `(progn
        (with ,vect = ,v)
        (with ,end = (array-dimension ,vect 0))
        (with ,index = -1)
        (,kwd ,var next (progn (incf ,index)
                               (if (>= ,index ,end) (terminate))
                               (aref ,vect ,index))))))
\end{program}


We are still missing one thing: the |\&sequence| keywords.  
We can get them easily enough, by writing 
\begin{program}
(defmacro-driver (FOR var IN-WHOLE-VECTOR v \&sequence) 
  ...)
\end{program}
We can now refer to parameters |from|, |to|, |by|, etc. which contain
either the values for the corresponding keyword, or \nil\ if the
keyword was not supplied.  Implementing the right code for these
keywords is cumbersome but not difficult; it is left as an exercise.
But before you begin, see |defclause-sequence| below for an easier way.
       
\subsection{Extensibility Aids}
\label{aids}

This section documents assorted features that may be of use in
extending \iter.

\begin{clauses}

\defunexpvar{*result-var*}
Holds the variable that is used to return a value as a result of the
\iter\ form.  You may examine this and use it in a |with| clause, but
you should not change it.

\defmacro{defsynonym}{syn word}
 Makes ~syn~ a synonym for the existing \iter\ keyword ~word.~  Only
the first word in each clause can have synonyms. 

\defmacroxx{defclause-sequence}{element-name index-name}{|\&key|
access-fn size-fn sequence-type}{element-type
element-doc-string index-doc-string}  
Provides
a simple way to define sequence clauses.  Generates two
clauses, one for iterating over the sequence's elements, the other
for iterating over its indices.  The first symbol of both
clauses will have print-name |for|.
~element-name~  and ~index-name~ should be symbols.
~element-name~ is the second keyword of the element iterator (typically of
the form 
|in-~sequence-type~|), and ~index-name~ is the second keyword
of the index-iterator (typically of the form
|index-of-~sequence-type~|).  Either name may be 
\nil, in which case the corresponding clause is not defined.  If both
symbols are supplied, they should be in the same package.  The |for|
that begins the clauses will be in this package.

\cpar ~access-fn~ is the function to be used to
access elements of the sequence in the element iterator.  The function
should take two 
arguments, a sequence and an index, and return the appropriate element.
~size-fn~ should denote a function of one argument, a sequence, that
returns its size.  Both ~access-fn~ and ~size-fn~ are required for the
element iterator, but only ~size-fn~ is needed for the index iterator.

\cpar The ~sequence-type~ and ~element-type~ keywords are used to
suggest types for the variables 
used to hold the sequence and the
sequence elements, respectively.  The usual rules about \iter's
treatment of variable type declarations apply (see section \ref{types}).

\cpar ~element-doc-string~ and ~index-doc-string~ are
the documentation strings, for use with |display-iterate-clauses|. 

\cpar The generated element-iterator performs destructuring on the
element variable.

\cpar As an example, the above |for\dots in-whole-vector| example
could have been written:
\begin{program}
(defclause-sequence IN-WHOLE-VECTOR INDEX-OF-WHOLE-VECTOR
  :access-fn 'aref
  :size-fn \#'(lambda (v) (array-dimension v 0))
  :sequence-type 'vector
  :element-type t
  :element-doc-string 
     "Elements of a vector, disregarding fill-pointer"
  :index-doc-string 
     "Indices of vector, disregarding fill-pointer")
\end{program}

\end{clauses}

\subsection{Subtleties}

There are some subtleties to be aware of when writing \iter\ clauses.
First, the code returned by your macros may be |nconc|'ed into a list,
so you should always returned freshly consed lists, rather than
constants.  Second, \iter\ matches clauses by using |eq| on the first
symbol and |string=| on the subsequent ones, so the package of the
first symbol of a clause is relevant.  All of the clauses in this manual
have their first word in the \iter\ package.  
You can use the package system in the usual way to shadow
\iter\ clauses without replacing them.

%%% say more here, about the badness that only the first word of a
%%% clause is packagey.

\section{Non-portable Extensions to Iterate (Contribs)}
\label{contribs}

Currently, there is only one non-portable extension to iterate in the
distribution: iterate-pg. If you have made an extension that depends
on non-portable features, feel free to send them to |asf@boinkor.net|
for inclusion in the iterate distribution.

\subsection{An SQL query driver for iterate}

The pg package by Eric Marsden (see |http://cliki.net/pg|) provides an
interface to the PostgreSQL database. Using the \iterpg\ extension, it
is possible to handle the results of SQL queries using \iter.

This usage example should give you an idea of how to use it:

\begin{program}
(pg:with-pg-connection (c "somedb" "someuser")
  (iter (for (impl version date) in-relation "select * from version"
                                 on-connection *dbconn*)
        (collect version)))
\end{program}

To use the extension via |ASDF|, simply make your system depend on the
|iterate-pg| system instead of the |iterate| system. To load it
manually, use:

\begin{program}
  (asdf:oos 'asdf:load-op :iterate-pg)
\end{program}

%%%%%%%%%%%%%%%%%%%%%%%%%%%%%%%%%%%%%%%%%%%%%%%%%%%%%%%%%%%%%%%%
\section{Obtaining {\tt Iterate}}

\begin{sloppypar}
The information in this chapter is obsolete but included for
completeness's sake; Currently, the most up-to-date information on
\iter\ can be found at |http://boinkor.net/iterate.html|.
\end{sloppypar}

\begin{sloppypar}
\iter\ currently runs on Lisp Machines, and on
HP's, Sun3's and Sparcstations under Lucid.  
\iter\ source and binaries are available at the MIT AI Lab in the
subdirectories of |/src/local/lisplib/|.  The source file,
|iterate.lisp|, is also available for anonymous FTP in the directory
|/com/fpt/pub/| on the machine |TRIX.AI.MIT.EDU| (Internet number
128.52.32.6).  If you are unable to obtain |iterate| in one of these
ways, send mail to |jba@ai.mit.edu| and I will send you the source
file. 
\end{sloppypar}

\begin{sloppypar}
\iter\ resides in the |iterate| package (nickname |iter|).  Just say
\linebreak |(use-package :iterate)| to make all the necessary symbols
available. 
If a symbol is not exported, it appears in this manual with an
``|iterate::|'' prefix.
\end{sloppypar}

Send bug reports to |bug-iterate@ai.mit.edu|.  The |info-iterate|
mailing list will have notices of changes and problems; to have
yourself added, send mail to |info-iterate-request@ai.mit.edu|.


\medskip

\begin{flushleft}
 \bf Acknowledgements
\end{flushleft}
\smallskip

Richard Waters provided invaluable criticism which spurred me to improve
\iter\ greatly.  As early users, David Clemens, Oren Etzioni and Jeff
Siskind helped ferret out many bugs.

%\begin{theindex}
% The index files must be generated with the genindex program
% or from makeindex iterate-manual.idx which creates iterate-manual.ind.

%\baselineskip\normalbaselineskip
%\advance\baselineskip by -2pt

% Fix ``.'' spacing in lists
\documentstyle[12pt,aimemo]{article}

\setlength{\textwidth}{6.0in}
\setlength{\oddsidemargin}{0.25in}
\setlength{\evensidemargin}{0.25in}
\setlength{\topmargin}{0pt}
\setlength{\headsep}{24pt}
\setlength{\textheight}{8.5in}

\let\origpar=\par

% somehow, LaTex fucks with things so that the plain Tex \obeyspaces
% doesn't work.  This is the same, except we use an mbox containing
% a space instead of just a space.

\newcommand{\spc}{\mbox{ }}
\newcommand{\tAb}{\mbox{        }}
\def\obspaces{\catcode`\ =\active\catcode`\^^I=\active}
{\obspaces \global\let =\spc\global\let^^I=\tAb}

\newenvironment{program}{\medskip\samepage\tt\obeylines\obspaces}{\medskip}

% use | for \tt
\catcode`\|=13
\newif\ifvbar
\def|{\ifvbar
         \endgroup\vbarfalse
      \else 
         \begingroup\tt\vbartrue 
      \fi}



\newcommand{\lisp}{\tt}
\newcommand{\iter}{|iterate|}
\newcommand{\iterpg}{|iterate-pg|}
\newcommand{\setf}{|setf|}
\newcommand{\nil}{|nil|}
\newcommand{\nonnil}{non-\nil}
\newcommand{\opt}{{\lisp \&optional}}
\newcommand{\yields}{$\Rightarrow$}



\newbox\Dots
\setbox\Dots=\hbox{\small ...}
\renewcommand{\dots}{\copy\Dots}

% Here's a way to define an environment which will treat all
% paragraphs inside it in the same manner.  It's, as near as I can
% figure, the way that Latex does things.  Using \everypar won't work;
% apparently Latex resets that a lot.  What you have to do is: at the
% top of the file do \let\origpar=\par.  In the
% begin part of the environment, do the parshape (or whatever), then say
% \def\par{{\origpar}}.  The extra braces are necessary.   That's all
% you have to do; at the end of the environment things will return to
% normal automatically.  Why does this all work?  I have no idea.
% Another important thing: there should be a blank line (i.e. a \par)
% between the end of the last paragraph and the \end{environment}.

\newlength{\clindent}
\setlength{\clindent}{0.5in}
\newlength{\clparindent}
\setlength{\clparindent}{\parindent}

\newenvironment{clauses}{\advance\linewidth -\parindent
                        \hangindent\clindent\relax
                         \hangafter=0
                          \parindent=0pt
                          \def\par{{\origpar}}}{}

\newcommand{\cpar}{\hspace \clparindent}

\newcommand{\startitem}{\bigskip\pagebreak[3]}
\newcommand{\finishitem}{\smallskip}

\newcommand{\clause}[1]{\startitem\Clause{#1}\finishitem}
\newcommand{\Clause}[1]{\hbox{#1}}
\newcommand{\clindex}[1]{\index{{\lisp #1}}}
\newcommand{\clindexx}[2]{\index{{\lisp #1\dots #2}}}
\newcommand{\clindexxx}[3]{\index{{\lisp #1\dots #2\dots #3}}}
\newcommand{\clausex}[2]{\startitem\Clausex{#1}{#2}\finishitem}
\newcommand{\Clausex}[2]{\Clause{|#1| #2}\clindex{#1}}

\newcommand{\defvar}[1]{\startitem\deff{#1}{}{Variable}\finishitem}
\newcommand{\defunexpvar}[1]{\startitem\deffnoindex{iterate::#1}{}{Variable}
                                        \lispindex{#1}\finishitem}
\newcommand{\defun}[2]{\startitem\deff{#1}{#2}{Function}\finishitem}
\newcommand{\defmacro}[2]{\startitem\deff{#1}{#2}{Macro}\finishitem}
\newcommand{\defunx}[3]{\startitem\deffx{#1}{#2}{#3}{Function}\finishitem}
\newcommand{\defmacrox}[3]{\startitem\deffx{#1}{#2}{#3}{Macro}\finishitem}
\newcommand{\defunxx}[4]{\startitem\deffxx{#1}{#2}{#3}{#4}{Function}\finishitem}
\newcommand{\defmacroxx}[4]{\startitem\deffxx{#1}{#2}{#3}{#4}{Macro}\finishitem}

\newcommand{\deff}[3]{\deffnoindex{#1}{#2}{#3}\lispindex{#1}}
\newcommand{\deffnoindex}[3]{\hbox to \hsize{{\tt #1} {\it #2}\hfill [{\it #3}]}}
\newcommand{\deffx}[4]{\deff{#1}{#2}{#4}\moreargs{#1}{#3}}
\newcommand{\deffxx}[5]{\deffx{#1}{#2}{#3}{#5}\moreargs{#1}{#4}}

\newcommand{\moreargs}[2]{\hbox to \hsize{\phantom{\lisp #1} {\it #2}\hfill}}

\newcommand{\lispindex}[1]{\index{{\lisp #1}}}

 \newenvironment{note}[1]{\pagebreak[2]\bigskip
                          \hrule\smallskip\small
                         {\setlength{\parindent}{0pt}
                           \par{\bf #1:}}}{\smallskip\hrule\bigskip}



\makeindex

%%%%%%%%%%%%%%%%%%%%%%%%%%%%%%%%%%%%%%%%%%%%%%%%%%%%%%%%%%%%%%%%

%\thispagestyle{empty}

%\vspace*{2in}

%\title{The |Iterate| Manual}
%\author{Jonathan Amsterdam}

%\maketitle

%\thispagestyle{empty}

%\begin{center}
%\Large *** DRAFT ***
%\end{center}
%This manual is in beta-test. It will become an AI Memo in a couple of
%months.  I would appreciate any comments or
%corrections.  You can reach me at email address |jba@ai.ai.mit.edu|.

\begin{document}

\def\writtenby{Jonathan Amsterdam}
\def\contractno{This report describes research done at the Artificial
Intelligence Laboratory of the Massachusetts Institute of Technology.
Support for the laboratory's artificial intelligence research is
provided in part by the Advanced Research Projects Agency of the
Department of Defense under Office
of Naval Research contract N00014-85-K-0124.}

\AIMEMO{1236}{\Large The Iterate Manual \\[1in]}
{This is the manual for version 1.4 of \iter, a powerful iteration
macro for Common  
Lisp. \iter\ is similar to {\tt loop} but provides numerous additional
features, is well integrated with Lisp, and is extensible. 
\vfill }


% use ~ for \em
\catcode`\~=13
\newif\ifem
\def~{\ifem
         \/\endgroup\emfalse
      \else 
         \begingroup\em\emtrue 
      \fi}



%\pagebreak
%\setcounter{page}{1}

%\thispagestyle{empty}


\tableofcontents

\pagebreak
\pagestyle{headings}

\section{Introduction}

\begin{sloppypar}
This manual describes
\iter, a powerful iteration facility for Common Lisp.
\iter\ provides abstractions for many common iteration
patterns and allows for the definition of additional patterns.  \iter\ is
a macro that expands into ordinary Lisp at
compile-time, so it is more efficient than higher-order
functions like |map| and |reduce.|  
While it is similar to |loop|, \iter\ offers a more Lisp-like syntax
and enhanced extensibility.
(For a more complete
comparison of \iter\ with other iteration constructs, see MIT AI Lab Working
Paper No. 324, ~Don't Loop, Iterate.~)
\end{sloppypar}

An \iter\ form consists of the symbol |iter|\footnote{You can also use
|iterate|, but |iter| is preferred because it avoids potential conflicts with
possible future additions to Common Lisp, and because it saves
horizontal space when writing code.}
followed by one or more
forms, some of which may be \iter\ 
~clauses.~  Here is a simple example of \iter\ which collects the
numbers from 1 to 10 into a list, and returns the list.  The return
value is shown following the arrow.
\begin{program}
(iter (for i from 1 to 10)
      (collect i))          \yields (1 2 3 4 5 6 7 8 9 10)
\end{program}
This form contains two clauses: a |for| clause that steps the
variable |i| over the integers from 1 to 10, and a |collect|
clause that accumulates its argument into a list.  With a few
exceptions, all \iter\ clauses have the same format: alternating
symbols (called ~keywords~) and expressions (called
~arguments~).  The syntax and terminology are those of Common
Lisp's keyword lambda lists.  One difference is that \iter's keywords
do not have to begin with a colon---though they may, except for the
first symbol of a clause.  So you can also write  |(for i :from
1 :to 10)| if you prefer.

Any Lisp form can appear in the body of an \iter, where it will have
its usual meaning.  \iter\ walks the entire body, expanding macros,
and recognizing clauses at any level.  This example collects all the
odd numbers in a list:
\begin{program}
(iter (for el in list)
      (if (and (numberp el) (oddp el))
          (collect el)))
\end{program}

There are clauses for iterating over numbers, lists, arrays and other
objects, and for
collecting, summing, counting, maximizing and other useful operations.
\iter\ also supports the creation of new variable bindings, stepping
over multiple sequences at once, destructuring, and compiler
declarations of variable types.  The following example illustrates
some of these features: 
\begin{program}
(iter (for (key . item) in alist)
         (for i from 0)
         (declare (fixnum i))
         (collect (cons i key)))
\end{program}

This loop takes the keys of an alist and returns a new alist
associating the keys with their positions in the original list. The
compiler declaration for |i| will appear in the generated code
in the appropriate place.

%%%%%%%%%%%%%%%%%%%%%%%%%%%%%%%%%%%%%%%%%%%%%%%%%%%%%%%%%%%%%%%%
\section{Clauses}

Most of \iter's clauses will be familiar to |loop| programmers.  
(|loop| is an iteration macro that has been incorporated into Common
Lisp.  Seen Guy Steele's {\em Common Lisp, 2nd Edition}.)
In
nearly all cases they behave the same as their |loop| counterparts, so
a |loop| user can switch to \iter\ with little pain (and much gain).

All clauses with the standard keyword-argument syntax consist of two
parts: a ~required~ part, 
containing keywords that must be present and in the right order; and
an ~optional~ part, containing keywords that may be omitted and,
if present, may occur in any order.  In the descriptions below, the
parts are separated by the Lisp lambda-list keyword \opt.


\subsection{Drivers}

An iteration-driving clause
conceptually causes the iteration to go forward.  Driver clauses in
\iter\ allow iteration over numbers, lists, vectors, hashtables, packages,
files and streams.  Iteration-driving clauses must
appear at the top level of an \iter\ form; they cannot be nested
inside another clause.  The driver variable is updated at the point
where the driver clause occurs.  Before the clause is executed for the
first time, the value of the variable is undefined.

%Also, regardless of where the driver clause
%appears in the body, the driver variable is stepped at the top of the
%loop; hence it is stylistically preferable, though
%not required, to place driver clauses at the beginning of the \iter.

Multiple drivers may appear in a single \iter\ form, in which case all
of the driver variables are updated each time through the loop, in the
order in which the clauses appear.  The first driver to terminate will
terminate the entire loop.

In all cases, the value of the driver variable on exit from the loop,
including within the epilogue code (see the |finally| clause), is
undefined.

All the parameters of a driver clause are evaluated once, before the
loop begins.  Hence it is not possible to change the bounds or other
properties of an iteration by side-effect from within the loop.

With one exception, driver clauses begin with the word |for| (or
the synonym |as|) and mention an iteration variable, which is
given a binding within the \iter\ form.
The exception is |repeat|, which just executes a loop a
specified number of times:

\begin{clauses}

\clausex{repeat}{~n~}
Repeats the loop ~n~ times.  For example:
\begin{program}
(iter (repeat 100)
      (print "I will not talk in class."))
\end{program}
\cpar If $n \leq 0$, the
loop will never be executed.  If ~n~ is not an integer, the
actual number of executions will be $\lceil n \rceil$.

\end{clauses}

\pagebreak[3]

\subsubsection{Numerical Iteration}

\begin{clauses}
\clausex{for}{~var~ |\&sequence|} 
The general form for iterating over a sequence of numbers requires a
variable and, optionally, one or more keywords that provide the bounds
and step size of the iteration. The |\&sequence|
lambda-list keyword is a shorthand for these sequence keywords.  They are:
|from|, |upfrom|, |downfrom|, |to|, |downto|, |above|, |below| and |by|.
|from| provides the starting value for ~var~
and defaults to zero. |to| provides a final value and implies that the
successive values of ~var~ will be increasing; |downto| implies that
they will be decreasing.  The loop terminates when ~var~ passes
the final value (i.e. becomes smaller or larger than it, depending on
the direction of iteration); in other  words, the loop body
will never be 
executed for values of ~var~ past the final value.
|below| and |above| are similar to |to| and
|downto|, except that the loop terminates when ~var~ equals or passes
the final value.

\cpar If no final value is specified, the variable will be stepped
forever.  Using |from| or |upfrom| will result in increasing
values, while |downfrom| will give decreasing values.

\cpar On each iteration,
~var~ is incremented or decremented by the value of the sequence
keyword |by|, which defaults to 1.  It should always be a positive
number, even for downward iterations.

\cpar In the following examples, the sequence of numbers generated is shown
next to the clause.

\begin{program}
(for i upfrom 0) \yields\ 0 1 2 \ldots
(for i from 5)   \yields\ 5 6 7 \ldots    ; either from or upfrom is okay
(for i downfrom 0) \yields\ 0 -1 -2 \ldots
(for i from 1 to 3) \yields\ 1 2 3
(for i from 1 below 3) \yields\ 1 2
(for i from 1 to 3 by 2) \yields\ 1 3
(for i from 1 below 3 by 2) \yields\ 1
(for i from 5 downto 3) \yields\ 5 4 3
\end{program}

\end{clauses}

\subsubsection{Sequence Iteration}

There are a number of clauses for iterating over sequences.  In all of
them, the argument following |for| may be a list instead of a
symbol, in which case destructuring is performed.  See section
\ref{destructuring}. 

\begin{clauses}

\clause{|for| ~var~ |in| ~list~ \opt\ |by| ~step-function~}
\clindexx{for}{in}
~var~ is set to successive elements of list.  
~step-function,~ which 
defaults to |cdr|, is used to obtain the next sublist. 

\clause{|for| ~var~ |on| ~list~ \opt\ |by| ~step-function~}
\clindexx{for}{on}
~var~ is set to successive sublists of list. 
~step-function~ (default |cdr|) is used as in |for\dots in|.

\end{clauses}

\medskip

These two clauses use |atom| to test for the end
of a list.  Hence, given a list whose final |cdr| is not \nil,
they will silently ignore the last |cdr|.  Other choices are
|endp|,  which would signal an error, and |null|, which
would probably result in an error somewhere else.  If you wish to use
an end-test other than |atom|, set the variable
|iterate::*list-end-test*|\lispindex{*list-end-test*} to the name of
the desired function. 

\begin{clauses}
\clause{|for| ~var~ |in-vector| ~vector~ |\&sequence|}
\clindexx{for}{in-vector}
~var~ takes on successive elements from ~vector.~   The vector's
fill-pointer is observed.  Here and in subsequent clauses, the |\&sequence|
keywords include |with-index|, which takes a symbol as argument
and uses it for the index variable instead of an internally generated
symbol.  The other |\&sequence| keywords behave as in numerical
iteration, except that the default iteration bounds are the bounds of
the vector.  E.g. in |(for i in-vector v downto 3)|,
|i| will start off being bound to the last element in |v|,
and will  
be set to preceding elements down to and including the element with
index 3.

\clause{|for| ~var~ |in-sequence| ~seq~ |\&sequence|}
\clindexx{for}{in-sequence}
This uses Common Lisp's generalized sequence functions, |elt|
and |length|, to obtain elements and determine the length of
~seq.~   Hence it will work for any sequence, including lists, and will
observe the fill-pointers of vectors.

\clause{|for| ~var~ |in-string| ~string~ |\&sequence|}
\clindexx{for}{in-string}
~var~ is set to successive characters of ~string.~

\startitem
\Clause{|for| ~var~ |index-of-vector| ~vector~ |\&sequence|}
\clindexx{for}{index-of-vector}
\Clause{|for| ~var~ |index-of-sequence| ~sequence~ |\&sequence|}
\clindexx{for}{index-of-sequence}
\Clause{|for| ~var~ |index-of-string| ~string~ |\&sequence|}
\clindexx{for}{index-of-string}
\finishitem
~var~ is set to successive indices of the sequence.
These clauses avoid the overhead of accessing the sequence elements
for those applications where they do not need to be examined, or are
examined rarely.  They admit all the optional keywords of the other
sequence drivers except the (redundant) |with-index| keyword.

\clause{|for| ~(key value)~ |in-hashtable| ~table~}
\clindexx{for}{in-hashtable}

~key~ and ~value,~ which must appear as shown in a list
and may be destructuring templates, are set to the keys and values of
~table~.  If ~key~ is |nil|, then the hashtable's keys will be ignored;
similarly for ~value~.
The order in which 
elements of ~table~ will be retrieved is unpredictable.

\clause{|for| ~var~ |in-package| ~package~ \opt\ |external-only| ~ext~}
\clindexx{for}{in-package}
Iterates over all the symbols in ~package,~ or over only the
external symbols if ~ext~ is specified and non-|nil|.  ~ext~ is not
evaluated.  The same symbol may appear more than once.

\clause{|for| ~var~ |in-packages| ~packages~ \opt\ |having-access| ~symbol-types~}
\clindexx{for}{in-packages}
Iterates over all the symbols from the list of packages denoted by the
descriptor ~packages~ and having accessibility (or visibility) given by
~symbol-types~.  This defaults to the list |(:external :internal :inherited)|
and is not evaluated.  ~var~ must be a list of up to three variables: in each
iteration, these will be set to a symbol, its access-type and package (as per
|with-package-iterator| in ANSI-CL).  The same symbol may appear more than
once.

\end{clauses}

\begin{clauses}

\clause{|for| ~var~ |in-file| ~name~ \opt\ |using| ~reader~}
\clindexx{for}{in-file}
Opens the file ~name~ (which may be a string or pathname) for input,
and iterates 
over its contents. ~reader~ defaults to |read|, so by default ~var~
will be bound to the successive forms in the file.  The \iter\ body is
wrapped in an unwind-protect to ensure that the file is closed no
matter how the \iter\ is exited.

\clause{|for| ~var~ |in-stream| ~stream~ \opt\ |using| ~reader~}
\clindexx{for}{in-stream}
Like |for\dots in-file|, except that ~stream~ should be an existing
stream object that supports input operations.

\end{clauses}

\subsubsection{Generalized Drivers}

These are primarily useful for writing drivers that can also be used
as generators (see section \ref{generators}, below).
 
\begin{clauses}

\clause{|for| ~var~ |next| ~expr~}
\clindexx{for}{next}

~var~ is set to ~expr~ each time through the loop.  Destructuring is
performed.  When the clause is used as a generator, ~expr~ is the code
that is executed when |(next ~var~)| is 
encountered (see section \ref{generators}, below).
~expr~ should compute the first value for ~var~, as well
as all subsequent values, and is responsible for terminating the loop.
For compatibility with future versions of \iter, this termination
should be done with |terminate|\clindex{terminate}, which can be
considered a synonym for |finish| (see section \ref{control-flow}).

\cpar As an example, the following clauses are equivalent to |(for
i from 1 to 10)|:
\begin{program}
(initially (setq i 0))
(for i next (if (> i 10) (terminate) (incf i)))
\end{program}

\clause{|for| ~var~ |do-next| ~form~}
\clindexx{for}{do-next}
~form~ is evaluated each time through the loop.  Its value is ~not~
set to ~var~; that is ~form's~ job. ~var~ is only present so that
\iter\ knows it is a driver variable.  \linebreak |(for ~var~ next
~expr~)| is equivalent to |(for ~var~ do-next (dsetq ~var~ ~expr~))|.
(See section \ref{destructuring} for an explanation of |dsetq|.) 

\end{clauses}

\subsubsection{Generators}
\label{generators}

In all of the above clauses, the driver variable is updated on each
iteration.
Sometimes it is desirable to have greater control over updating.  
For instance, consider the problem of associating numbers, in
increasing order and with no gaps, with the
\nonnil\ elements of a list.  One obvious first pass at writing this is:

\begin{program}
(iter (for el in list)
      (for i upfrom 1)
      (if el (collect (cons el i))))
\end{program}
But on the list |(a b nil c)| this produces |((a . 1) (b . 2) (c .
4))| instead of the desired |((a . 1) (b . 2) (c . 3))|.  The problem
is that |i| is incremented each time through the loop, even when |el|
is |nil|.  

The problem could be solved elegantly if we could step |i| only when
we wished to.  This
can be accomplished for any \iter\ driver by writing |generate| (or
its synonym |generating|)
instead of |for|.  Doing so produces a ~generator~---a driver whose
values are yielded explicitly.  To obtain the next value of a
generator variable ~v~, write \linebreak |(next ~v~)|.  The value of a |next|
form is the next value of ~v~, as determined by its associated driver
clause.  |next| also has the side-effect of updating ~v~ to that
value.  If there is no next value, |next| will terminate the loop,
just as with a normal driver.


Using generators, we can now write our example like this:
\begin{program}
(iter (for el in list)
      (generate i upfrom 1)
      (if el (collect (cons el (next i)))))
\end{program}
Now |i| is updated only when |(next i)| is executed, and this occurs
only when |el| is \nonnil.

To better understand the relationship between ordinary drivers and
generators, observe that we can rewrite an ordinary driver using its
generator form immediately followed by |next|, as this example shows:
\begin{program}
(iter (generating i from 1 to 10)
      (next i)
      ...)
\end{program}
Provided that the loop body contains no |(next i)| forms, this will
behave just as if we had written |(for i from 1 to 10)|.

We can still refer to a driver variable ~v~ without using |next|; in
this case, its value is that given to it by the last evaluation of
|(next ~v~)|.  Before |(next ~v~)| has been called the first time, the
value of ~v~ is undefined.  

This semantics is more flexible than
one in which ~v~ begins the loop bound to its first value and calls
of |next| supply subsequent values, because it means the loop will not
terminate too soon if the generator's sequence is empty.  For
instance, consider the following code, which tags \nonnil\ elements of
a list using a list of tags, and also counts the null elements.
(We assume there are at least as many tags as \nonnil\ elements.)
\begin{program}
(let* ((counter 0)
       (tagged-list (iter (for el in list)
                          (generating tag in tag-list) 
                          (if (null el)
                              (incf counter)
                              (collect (cons el (next tag)))))))
  ...)
\end{program}
It may be that there are just as many tags as non-null elements of
|list|.  If all the elements of |list| are null, we still want the
counting to proceed, even though |tag-list| is \nil.  If |tag| had to be
assigned its first value before the loop begins, we would have had to
terminate the loop before the first iteration, since when |tag-list|
is \nil, |tag| 
has no first value.  With the existing semantics, however, |(next
tag)| will never execute, so the iteration will cover all the elements of
|list|. 
        
When the ``variable'' of a driver clause is actually a destructuring
template containing several variables, all the variables are eligible
for use with |next|.  As before, |(next ~v~)| evaluates to ~v's~ next
value; but the effect is to update all of the template's variables.
For instance, the following code will return the list |(a 2 c)|.
\begin{program}
(iter (generating (key . item) in '((a . 1) (b . 2) (c . 3)))
      (collect (next key))
      (collect (next item)))
\end{program}

Only driver clauses with variables can be made into generators.  This
includes all clauses mentioned so far except for |repeat|.  It does
~not~ include |for\dots previous|, |for\dots =|,
|for\dots initially\dots then| or |for\dots first\dots then| (see
below). 

\subsubsection{Previous Values of Driver Variables}

Often one would like to access the value of a variable on a previous
iteration.  \iter\ provides a special clause for accomplishing this.

\begin{clauses}

\clause{|for| ~pvar~ |previous| ~var~ \opt\ |initially| ~init~
|back| ~n~}
\clindexx{for}{previous}
Sets ~pvar~ to the previous value of ~var,~ which should be a driver
variable, a variable from another |for\dots previous| clause, or a
variable established by a |for\dots =|, 
|for\dots initially\dots then| or |for\dots first\dots then| clause
(see section \ref{setting}).
Initially, ~pvar~ is given the value ~init~ (which defaults to \nil).  
The ~init~ expression will be moved outside the loop body, so it
should not depend on anything computed within the loop.
~pvar~ retains the value of ~init~ until ~var~ is set to its second
value, at which point ~pvar~ is set to ~var's~ first value; and so on.  

\cpar The
argument ~n~ to |back| must be a constant, positive integer, and
defaults to 1.  It determines how many iterations back ~pvar~ should
track ~var~.  For example, when ~n~ is 2, then ~pvar~ will be assigned
~var's~ first value when ~var~ is set to its third value.

\cpar A |for\dots previous| clause may occur after or before its
associated driver clause. |for\dots previous| works with generators as
well as ordinary drivers.

\pagebreak[3]

\cpar Example:
\begin{program}
(iter (for el in '(1 2 3 4))
      (for p-el previous el)
      (for pp-el previous p-el initially 0)
      (collect pp-el))
\end{program}
This evaluates to |(0 0 1 2)|.  It could have been written more
economically as 
\begin{program}
(iter (for el in '(1 2 3 4))
      (for pp-el previous el back 2 initially 0)
      (collect pp-el))
\end{program}


\end{clauses}

\subsection{Variable Binding and Setting}
\label{setting}

Several clauses exist for establishing new variable bindings or for
setting variables in the loop.  They all support destructuring.

\begin{clauses}

\clausex{with}{~var~ \opt\ |=| ~value~}
Causes ~var~ to be bound to value before the loop body is entered.
If ~value~ is not supplied, ~var~ assumes a default
binding, which will be 
\nil\ in the absence of declarations.  Also, if ~value~ is not
supplied, no destructuring is performed; instead, ~var~ may be a list of
symbols, all of which are given default 
bindings.  If ~value~ is supplied, ~var~ is bound to it, with
destructuring.  

\cpar Because |with|
creates bindings whose scope includes the entire \iter\ form, it is
good style to put all |with| clauses at the beginning.

\cpar Successive occurrences of |with| result in sequential
bindings (as with 
|let*|).  There is no way to obtain parallel bindings; see
section \ref{bindings} for a rationale.  


\clause{|for| ~var~ |=| ~expr~}
\clindexx{for}{=}
On each iteration, ~expr~ is evaluated and ~var~ is set
to its value. 

\cpar This clause may appear to do the same thing as |for\dots next|.
In fact, they are quite different.  |for\dots =| provides only three
services: it sets up a binding for ~var~, sets it to ~expr~ on each
iteration, and makes it possible to use |for\dots previous| with
~var~.  |for\dots next| provides these services in addition to the
ability to turn the driver into a generator.
%Also, the code which sets ~var~ appears in the loop body in the same
%place as the |for\dots =| clause; the code for |for\dots next| appears
%at the top of the loop, as with other drivers (except when being used
%as a generator).

\clause{|for| ~var~ |initially| ~init-expr~ |then| ~then-expr~}
\clindexxx{for}{initially}{then}
Before the loop begins, ~var~ is set to ~init-expr;~ on all
iterations after the first it is set to ~then-expr.~  This clause
must occur at top-level.  ~init-expr~ will be moved outside the loop
body and ~then-expr~ will be moved to the end of the loop body, so
they are subject to code motion problems (see section
\ref{code-movement}). 

\cpar This clause may appear to be similar to |for\dots next|, but in
fact they differ significantly.  |for\dots initially\dots then| is
typically used to give ~var~ its first value before the loop begins,
and subsequent values on following iterations.  This is incompatible
with generators, whose first value and subsequent values must all be
computed by |(next ~var~)|.  Also, the update of ~var~ in 
|for\dots initially\dots then| does not occur at the location of the
clause. 
Use |for\dots initially\dots then| for
one-shot computations where its idiom is more convenient, but use
|for\dots next| for extending \iter\ with new drivers (see section
\ref{extend}).

\clause{|for| ~var~ |first| ~first-expr~ |then| ~then-expr~}
\clindexxx{for}{first}{then}
The first time through the loop, ~var~ is set to ~first-expr;~ on
subsequent iterations, it is set to ~then-expr.~  This differs from
|for\dots initially| in that ~var~ is set to ~first-expr~
inside the loop body, 
so ~first-expr~ may depend on the results of other clauses.  For
instance, 
\begin{program}
(iter (for num in list)
      (for i first num then (1+ i))
      ...)
\end{program}
will set |i| to the first element of |list| on the first
iteration, whereas 
\begin{program}
(iter (for num in list)
      (for i initially num then (1+ i))
      ...)
\end{program}
is probably erroneous; |i| will be bound to |num|'s
default binding (usually \nil) for the first iteration.

\end{clauses}

\begin{note}{Compatibility note}
|loop|'s |for\dots =| works like \iter's, but |loop| used the syntax
|for\dots =\dots then| to mean |for\dots initially\dots then|.  It was
felt that these two operations were sufficiently different to warrant
different keywords.

Also, the |for| in the above three clauses is misleading,
since none is true driver (e.g. none has a corresponding |generate|
form).  |setting| would have been a better choice, but |for| was used
to retain some compatibility with |loop|.  
\end{note}


%%%%%%%%%%%%%%%%%%%%%%%%%%%%%%%%
\subsection{Gathering Clauses}
Many of \iter's clauses accumulate values into a variable, or set a
variable under certain conditions.  At the end of the loop, this
variable contains the desired result.  All these clauses have an
optional |into| keyword, whose argument should be a symbol.  If the
|into| keyword is not supplied, the accumulation variable will be
internally generated and its value will be returned at the end of the
loop; if a variable is specified, that variable is used for the
accumulation, and is not returned as a result---it is up to the user
to return it explicitly, in the loop's epilogue code (see |finally|).
It is safe to examine the accumulation variable during the loop, but
it should not be modified.

These clauses all begin with a verb.  When the verb does
not conflict with an existing Common Lisp function, then it may be
used in either its infinitival or present-participle form (e.g. |sum|,
|summing|).  However, when there is a conflict with Common Lisp, only
the present-participle form may be used (e.g. |unioning|).  This is to
prevent \iter\ clauses from clashing with Common Lisp
functions. 

%although these clauses are described as ``producing a value,'' it is a
%mistake to think of the lisp list representing the clause as a
%value-producing form in the usual way.  clauses may legally be written
%where a value is expected, e.g. |(setq x (sum i))|, but the
%lisp value of a clause in such a context is undefined.

\subsubsection{Reductions}

~Reduction~ is an extremely common iteration pattern in which
the results of successive applications of a
binary operation are accumulated.
For example, a loop that computes the sum of the
elements of a list is performing a reduction with the addition
operation.  This could be written in Common Lisp as  |(reduce \#'+
list)| or with \iter\ as
\begin{program}
(iter (for el in list)
      (sum el))
\end{program}

\begin{clauses}

\clausex{sum}{~expr~ \opt\ |into| ~var~}
Each time through the loop, ~expr~ is evaluated and added to a
variable, which is bound initially to zero.  If ~expr~ has a type,
it is ~not~ used as the type of the sum variable, which is always
|number|.  To get the result variable to be of a more specific
type, use an explicit variable, as in
\begin{program}
(iter (for el in number-list)
      (sum el into x)
      (declare (fixnum x))
      (finally (return x)))
\end{program}

\clausex{multiply}{~expr~ \opt\ |into| ~var~}
Like |sum|, but the initial value of the result variable is $1$,
and the variable is updated by multiplying ~expr~ into it.

\clausex{counting}{~expr~ \opt\ |into| ~var~}
~expr~ is evaluated on each iteration.  If it is \nonnil, the
accumulation variable, initially zero, is incremented.

\startitem
\Clausex{maximize}{~expr~ \opt\ |into| ~var~}
\Clausex{minimize}{~expr~ \opt\ |into| ~var~}
\finishitem
~expr~ is evaluated on each iteration and its extremum (maximum or
minimum) is stored in the accumulation variable.  If ~expr~ is never
evaluated, then the result is \nil\ (if the accumulation variable is
untyped) or $0$ (if it has a numeric type).

\clausex{reducing}{~expr~ |by| ~func~ \opt\ |initial-value| ~init-val~ |into| ~var~}
This is a general way to perform reductions. ~func~ should be a
function of two arguments, the first of which will be the value
computed so far and
the second of which will be the value of ~expr~.  It should return the
new value.
|reducing| is roughly equivalent to the
Common Lisp |(reduce ~func~ ~sequence~ :key ~expr-function~)|, where
~expr-function~ is used to derive values from the successive elements
of ~sequence~.

\cpar If the |reducing| clause is never executed, the result is
undefined.

\cpar It is not necessary to provide an
initial value, but better code can be generated if one is supplied.
Regardless of its location in the \iter\ body, ~init-val~ will be
evaluated before the loop is entered, so it should not depend on any
value computed inside the \iter\ form.  

%\cpar if a ~var~ is not specified, you can get \iter\ to declare the
%type of the internal variable by putting a |the| expression around
%~func.~  see section \ref{types}.

\end{clauses}

%\begin{note}{implementation note}
%in principle, |maximize| and |minimize| can be thought of
%as reductions where the initial value is the smallest (or largest)
%value that the accumulation variable can assume.  because lisp's
%bignums can represent arbitrary integers, these clauses cannot be
%implemented as reductions in general.  if, however, the type of 
%~expr~ or ~var~ can be determined to be a fixnum
%or a float, \iter\ will implement the clause as a true reduction,
%using one of the constants |most-negative-fixnum|, |%most-positive-fixnum|, 
%|most-negative-short-float|, etc. as appropriate.
%\end{note}

\subsubsection{Accumulations}

All the predefined accumulation clauses add values to a sequence.  If
the sequence is a list, they all 
have the property that the partial list is kept in the correct order
and available for inspection at any point in the loop.

\begin{clauses}

\clausex{collect}{~expr~ \opt\ |into| ~var~ |at| ~place~ |result-type| ~type~}
Produces a sequence of the values of ~expr~ on each iteration. ~place~
indicates where the next value of ~expr~ is added to the list and may
be one of the symbols |start|, |beginning| (a synonym for |start|) or
|end|.  The symbol may be quoted, but need not be.  The default is
|end|.  For example,
\begin{program}
(iter (for i from 1 to 5)
      (collect i))
\end{program}
produces |(1 2 3 4 5)|, whereas
\begin{program}
(iter (for i from 1 to 5)
      (collect i at beginning))
\end{program}
produces |(5 4 3 2 1)| (and is likely to be faster in most Common Lisp
implementations).

\cpar If ~type~ is provided, it should be a subtype of |sequence|.
The default is |list|.  Specifying a type other than |list| will
result in |collect| returning a sequence of that type.  ~However,~ the
type of the sequence being constructed when inside the loop body is
undefined when a non-|list| type is specified.  (As with ~place~,
quoting ~type~ is optional.)

\clausex{adjoining}{~expr~ \opt\ |into| ~var~ |test| ~test~ |at| ~place~ |result-type| ~type~}
Like |collect|, but only adds the value of ~expr~ if it is not
already present.  ~test,~ which defaults to |\#'eql|, is
the test to be used with |member|.

\startitem
\Clausex{appending}{~expr~ \opt\ |into| ~var~ |at| ~place~}
\Clausex{nconcing}{\ ~expr~ \opt\ |into| ~var~ |at| ~place~}
\Clausex{unioning}{\ ~expr~ \opt\ |into| ~var~ |test| ~test~ |at| ~place~}
\Clausex{nunioning}{~expr~ \opt\ |into| ~var~ |test| ~test~ |at| ~place~}
\finishitem
These are like |collect|, but behave like the Common Lisp functions
|append|, |nconc|, |union| or |nunion|.  
As in Common Lisp, they work only on lists.  Also as in Common Lisp,
|unioning| and |nunioning| assume that the value of ~expr~ contains no
duplicates. 

\clausex{accumulate}{~expr~ |by| ~func~ \opt\ |initial-value| ~init-val~ |into| ~var~}
This is a general-purpose accumulation clause. ~func~ should be a
function of two arguments, the value of ~expr~ and the value
accumulated so far in the iteration, and it should return the updated
value.  If no initial value is supplied, \nil\ is used.  

%\cpar If a ~var~ is not specified, you can get \iter\ to declare the
%type of the internal variable by putting a |the| expression around
%~func.~  see section \ref{types}.

\cpar The differences between |accumulate| and |reducing| are slight.
One difference is that the functions take their arguments in a
different order.  Another is that in the absence of ~init-val~,
|accumulate| will use \nil, whereas |reducing| will generate different
code that avoids any dependence on the initial value.
The reason for having both clauses is that one usually
thinks of reductions (like |sum|) and accumulations (like |collect|)
as different beasts.

\end{clauses}


\subsubsection{Finders}

A ~finder~ is a clause whose value is an expression that meets some
condition.

\begin{clauses}

\clause{|finding| ~expr~ |such-that| ~test~ \opt\ |into| ~var~ |on-failure| ~failure-value~}
\clindexx{finding}{such-that}
If ~test~ (which is an expression) ever evaluates to \nonnil, the loop
is terminated, the 
epilogue code is run and the value of ~expr~ is returned.  Otherwise,
\nil\ (or ~failure-value,~ if provided) is returned.  If ~var~ is
provided, it will have either the \nonnil\ value of ~expr~ or
~failure-value~ when the epilogue code is run.

\cpar As a special case, if the ~test~ expression is a sharp-quoted
function, then it is applied to ~expr~ instead of being simply
evaluated.  E.g. |(finding x such-that \#'evenp)| is equivalent to
|(finding x such-that (evenp x))|.  

%\cpar although ~test~ need have nothing to do with ~%expr~ as in
%|(finding j such-that (> i 3))|, it usually 
%will: |(finding (length el) such-that (oddp (length el)))|.  to
%avoid performing the |length| computation twice, you could write
%|(finding (length el) such-that \#'oddp)| or |(finding (length
%el) such-that 'oddp)|; for these cases, \iter\ generates code that
%executes ~expr~ only once.  the code for |\#'oddp|
%is slightly different from that for {\lisp 'oddp}; see the discussion
%under {\lisp for\dots in} and {\lisp for\dots on}.

\cpar |On-failure| is a misnomer. Because it is always evaluated, it behaves
more like the default third argument to the |gethash| function. As a result,
|on-failure (error "Not found")| makes no sense. Instead, the clauses |leave|
or |thereis| can be used in conjunction with |finally| as follows:
\begin{program}
(iter (for x in '(1 2 3))
      (if (evenp x) (leave x))
      (finally (error "not found")))
\end{program}

\cpar This clause may appear multiple times when all defaults are
identical. It can also be used together with either |always|/|never| or
|thereis| if their defaults match. More specifically, |on-failure nil| is
compatible with |thereis|, while |on-failure t| is compatible with |always|
and |never| clauses.
\begin{program}
(iter (for i in '(7 -4 2 -3))
      (if (plusp i)
	  (finding    i  such-that (evenp i))
          (finding (- i) such-that (oddp i))))
\end{program}

\startitem
\Clause{|finding| ~expr~ |maximizing| ~m-expr~ \opt\ |into| ~var~}
\clindexx{finding}{maximizing}
\Clause{|finding| ~expr~ |minimizing| ~m-expr~ \opt\ |into| ~var~}
\clindexx{finding}{minimizing}
\finishitem
Computes the extremum (maximum or minimum) value of ~m-expr~ over all
iterations, and returns the value of ~expr~ corresponding to the
extremum.  ~expr~ is evaluated inside the loop at the time the new
extremum is established.  If ~m-expr~ is never evaluated (due to, for
example, being embedded in a conditional clause), then the returned
value depends on the type, if any, of ~expr~ (or ~var,~ if
one is supplied).  If there is no type, the returned
value will be \nil; if the type is numeric, the returned value will be
zero.

\cpar For these two clauses, ~var~ may be a list of two
symbols; in that case, the first is used to record ~expr~ and
the second, ~m-expr.~  

\cpar As with |finding\dots such-that|, if ~m-expr~ is a sharp-quoted
function, then it is called on ~expr~ instead of being evaluated.

\end{clauses}

\subsubsection{Boolean Tests}

\begin{clauses}

\clausex{first-iteration-p}{}
Returns |t| in the first cycle of the loop, otherwise \nil.

\clausex{first-time-p}{}
Returns |t| the first time the expression is evaluated, and then \nil\ forever.
This clause comes handy when printing (optional) elements separated
by a comma:

\begin{program}
(iter (for el in '(nil 1 2 nil 3))
      (when el
        (unless (first-time-p)
          (princ ", "))
        (princ el)))
\end{program}
produces |"1, 2, 3"|.

\end{clauses}

\subsubsection{Aggregated Boolean Tests}

\begin{clauses}

\clausex{always}{~expr~}
If ~expr~ ever evaluates to
\nil, then \nil\ is immediately returned; the epilogue code is not
executed.  If ~expr~ never evaluates to \nil, the epilogue code
is executed and the last value of ~expr~ (or |t| if ~expr~ was never
evaluated) is returned (whereas |loop| would constantly return |t|).

% mention last evaluated clause when multiple always clauses?

\clausex{never}{~expr~}
Like |(always (not ~expr~))|, except it does not influence the last
value returned by a possible other |always| clause. That is,
\begin{program}
(iter (repeat 2)
      (always 2)
      (never nil)) \yields 2 ; not t
\end{program}

\clausex{thereis}{~expr~}
If ~expr~ is ever \nonnil,
its value is immediately returned without running epilogue code.
Otherwise, the epilogue code is performed and \nil\ is returned.

This clause cannot be used together with |always| or |never|, because their
defaults are opposed (similarly, |(loop always 3 thereis nil)| refuses to
compile in some implementations of |loop|).

\end{clauses}


\subsection{Control Flow}
\label{control-flow}
Several clauses can be used to alter the usual flow of control in a loop.

Note: the clauses of this and subsequent sections don't adhere to \iter's
usual syntax, but instead use standard Common Lisp syntax.  Hence the
format for describing syntax subsequently is
like the standard format used in the Common Lisp manual, not like the
descriptions of clauses above.

\begin{clauses}

\clausex{finish}{}
Stops the loop and runs the epilogue code.

%for example:
%
%\begin{program}
%(iter (with answer = nil)
%         (initially (make-a-mess))
%         (for i from 1 to 10)
%         (when (correct? i)
%           (setq answer i)
%           (finish))
%         (finally (cleanup)))
%\end{program}
%
%this code will execute |cleanup| whether or not the test |(correct?
%i)| ever succeeds. 
%the (more elegant) formulation,
%\begin{program}
%(iter (initially (make-a-mess))
%         (for i from 1 to 10)
%         (finding i such-that (correct? i))
%         (finally (cleanup)))
%\end{program}
%would not execute |cleanup| if |(correct? i)| succeeded; it
%would do an immediate return.

\clausex{leave}{\opt\ ~value~}
Immediately returns ~value~ (default \nil) from the current \iter\
form, skipping the epilogue code.  Equivalent to using |return-from|.

\clausex{next-iteration}{}
Skips the remainder of the loop body and begins the next iteration of
the loop.

\clausex{while}{~expr~}
If ~expr~ ever evaluates to \nil, the loop is terminated and the
epilogue code executed.  Equivalent to |(if (not ~expr~) (finish))|.

\clausex{until}{~expr~}
Equivalent to |(if ~expr~ (finish))|.

\clausex{if-first-time}{~then~ \opt\ ~else~}
If this clause is being executed for the first time in this invocation
of the \iter\ form, then the ~then~ code is evaluated; otherwise the
~else~ code is evaluated.

\cpar |(for ~var~ first ~expr1~ then ~expr2~)| is almost equivalent to
\begin{program}
(if-first-time (dsetq ~var~ ~expr1~)
               (dsetq ~var~ ~expr2~))
\end{program}
The only difference is that the |for| version makes ~var~ available
for use with |for\dots previous|.

\end{clauses}

\subsection{Code Placement}
When fine control is desired over where code appears in a loop
generated by \iter, the following special clauses may be useful.
They are all subject to code-motion problems (see section
\ref{code-movement}).  

\begin{clauses}

\clausex{initially}{|\&rest| ~forms~}
The lisp ~forms~ are placed in the prologue section of the loop, where
they are executed once, before the loop body is entered.

\clausex{after-each}{|\&rest| ~forms~}
The ~forms~ are placed at the end of the loop body, where they
are executed after each iteration.  Unlike the other clauses in this
section, ~forms~ may contain \iter\ clauses.

\clausex{else}{|\&rest| ~forms~}
The lisp ~forms~ are placed in the epilogue section of the loop, where they
are executed if this |else| clause is never met during execution of the
loop and the loop terminates normally.

\clausex{finally}{|\&rest| ~forms~}
The lisp ~forms~ are placed in the epilogue section of the loop, where
they are executed after the loop has terminated normally.

\clausex{finally-protected}{|\&rest| ~forms~}
The lisp ~forms~ are placed in the second form of an unwind-protect
outside the loop.  They are always executed after the loop has
terminated, regardless of how the termination occurred.

\end{clauses}


%%%%%%%%%%%%%%%%%%%%%%%%%%%%%%%%%%%%%%%%%%%%%%%%%%%%%%%%%%%%%%%%
\section{Other Features}

\subsection{Multiple Accumulations}
\label{multiple}

\begin{sloppypar}
It is permitted to have more than one clause accumulate into the same
variable, as in the following:
\begin{program}
(iter (for i from 1 to 10)
      (collect i into nums)
      (collect (sqrt i) into nums)
      (finally (return nums)))
\end{program}
Clauses can only accumulate into the same variable if they are
compatible.  |collect|, |adjoining|, |appending|, |nconcing|,
|unioning| and |nunioning| are compatible with each other; |sum|,
|multiply| and |counting| are compatible; |always| and |never| are compatible;
|finding| \dots |such-that| is compatible with either |thereis| or |always|
and |never| when their defaults match; and |maximize| and |minimize| clauses
are compatible only with other |maximize| and |minimize| clauses,
respectively.

%note that the same variable ~cannot~ be both an accumulation
%variable and an ordinary variable; there can be only one variable with
%a given name within an \iter\ form. 
\end{sloppypar}

\subsection{Named Blocks}

Like Common Lisp |block|s, \iter\ forms can be given names.  The
name should be a single symbol, and it must be the first form in the
\iter.  The generated code behaves exactly like a named block; in
particular, |(return-from ~name~)| can be used to exit it:
\begin{program}
(iter fred
      (for i from 1 to 10)
      (iter barney
            (for j from i to 10)
            (if (> (* i j) 17)
                (return-from fred j))))
\end{program}
An \iter\ form that is not given a name is implicitly named \nil.

Sometimes one would like to write an expression in an inner \iter\ form,
but have it processed by an outer \iter\ form.  This is possible with
the |in| clause.  

\begin{clauses}

\clausex{in}{~name~ |\&rest| ~forms~}
Evaluates ~forms~ as if they were part of the \iter\ form named
~name~.  In other words, \iter\ clauses are processed by the \iter\
form named ~name,~ and not by any \iter\ forms that occur inside ~name.~

\cpar As an example, consider the problem of collecting a list of the
elements in a two-dimensional array.  The naive solution,
\begin{program}
(iter (for i below (array-dimension ar 0))
      (iter (for j below (array-dimension ar 1))
            (collect (aref ar i j))))
\end{program}
\noindent is wrong because the list created by the inner \iter\ is simply
ignored by the outer one.  But using |in| we can write:
\begin{program}
(iter outer (for i below (array-dimension ar 0))
      (iter (for j below (array-dimension ar 1))
            (in outer (collect (aref ar i j)))))
\end{program}
\noindent which has the desired result.

\end{clauses}

\subsection{Destructuring}
\label{destructuring}

In many places within \iter\ clauses where a variable is expected, a
list can be written instead.  In these cases, the value to be assigned
is ~destructured~ according to the pattern
described by the list.  As a simple example, the clause
\begin{program}
(for (key . item) in alist)
\end{program}
\noindent will result in |key| being set to the |car| of
each element in |alist|, and |item| being set to the |cdr|.  The
pattern list may be nested to arbitrary depth, and (as the example
shows) need not be terminated with \nil; the only requirement is that
each leaf be a bindable symbol (or \nil, in which case no binding is
generated for that piece of the structure).

Sometimes, you might like to do the equivalent of a
|multiple-value-setq| in a clause.  This 
``multiple-value destructuring'' can be expressed by writing 
\linebreak |(values $pat_1$ $pat_2 \ldots$)| for a destructuring
pattern, as in 
\begin{program}
(for (values (a . b) c d) = (three-valued-function ...))
\end{program}
\begin{sloppypar}
Note that the $pat_i$ can themselves be destructuring patterns (though
not multiple-value destructuring patterns).  You can't do multiple-value
destructuring in a |with| clause; instead wrap the whole \iter\
form in a |multiple-value-bind|.
\end{sloppypar}

\begin{note}{Rationale}
There are subtle interactions between variable declarations and
evaluation order that make the correct implementation of
multiple-value destructuring in a |with| somewhat tricky.
\end{note}

The destructuring feature of \iter\ is available as a separate
mechanism, using the |dsetq| macro:

\begin{clauses}
\defmacro{dsetq}{template expr}
Performs destructuring of ~expr~ using ~template~.   May be used
outside of an \iter\ form. Yields the primary value of ~expr~.

\end{clauses}

\subsection{On-line Help}

\begin{sloppypar}
There is a limited facility for on-line help, in the form of the 
|display-iterate-clauses| function.
\end{sloppypar}

\begin{clauses}

\defun{display-iterate-clauses}{\opt\ clause-spec}
Displays a list of \iter\ clauses.  If ~clause-spec~ is not
provided, all clauses are shown; if it is a symbol, all clauses
beginning with that symbol are shown; and if it is a list of symbols,
all clauses for which ~clause-spec~ is a prefix are shown.

\end{clauses}

\subsection{Parallel Binding and Stepping}
\label{bindings}

The parallel binding and stepping of variables is a feature that
\iter\ does ~not~ have.  This section attempts to provide a rationale.

We say that two variables are bound ~in parallel~ if neither
binding shadows the other.  This is the usual semantics of |let|
(as opposed to |let*|).  Similarly, we can say that iteration
variables are stepped in parallel if neither variable is updated
before the other, conceptually speaking; in other words, if the code
to update each variable can reference the old values of both variables.

|loop| allows parallel binding of variables and parallel stepping of
driver variables.  My view is that if you are depending on the 
serial/parallel distinction, you are doing something obscure.  If you
need to bind 
variables in parallel using |with|, then you must be using a
variable name that shadows a name in the existing lexical environment.
Don't do that.  The most common use for parallel stepping is to track
the values of variables on the previous iteration, but in fact this
does not require parallel stepping at all; the following will work:
\begin{program}
(iter (for current in list)
      (for prev previous current)
      ...)
\end{program}


%%%%%%%%%%%%%%%%%%%%%%%%%%%%%%%%%%%%%%%%%%%%%%%%%%%%%%%%%%%%%%%%
\section{Types and Declarations}
\label{types}

\subsection{Discussion}

Sometimes efficiency dictates that the types of variables be declared.
This type information needs to be communicated to \iter\ so it can
bind variables to appropriate values.  Furthermore, \iter\ must often
generate internal variables invisible to the user; there needs to be a
way for these to be declared.

As an example, consider this code, which will return the number of
odd elements in |number-list|:
\begin{program}
(iter (for el in number-list)
      (count (oddp el)))
\end{program}
In processing this form,
\iter\ will create an internal variable, let us call it |list17|, to
hold the successive |cdr|s of |number-list|, and  
will bind the variable to |number-list|.  It will also generate a
default binding for |el|; only inside the body of the loop will |el|
be set to the |car| of |list17|.  Finally, \iter\ will generate a
variable, call it |result|, to hold the result of the count, and will
bind it to zero.

When dealing with type declarations, \iter\ observes one simple rule:
~it will never generate a declaration unless requested to do so.~  The
reason is that such declarations might mask errors in compiled code by
avoiding error-checks; the resulting problems would be doubly hard to
track down because the declarations would be hidden from the
programmer.  Of course, a compiler might omit error-checks even in the
absence of declarations, though this behavior can usually be avoided,
e.g. by saying |(declaim (optimize (safety 3)))|.

So, the above \iter\ form will generate code with no declarations.
But say we wish to declare the types of |el| and the internal
variables |list17| and |result|.  How is this done?

Declaring the type of |el| is easy, since the programmer knows
the variable's name:
\begin{program}
(iter (for el in number-list)
      (declare (fixnum el))
      (counting (oddp el)))
\end{program}
\iter\ can read variable type declarations like this one.  Before
processing any clauses, it scans the entire top-level form for type
declarations 
and records the types, so that variable bindings can be performed
correctly.  In this case, |el| will be bound to zero
instead of \nil.  Also, \iter\ collects all the top-level declarations
and puts them at the begining of the generated code, so it is not
necessary to place all declarations at the beginning of an \iter\
form; instead, they can be written near the variables whose types they
declare.  

Since \iter\ is not part of the compiler, it will not know
about declarations that occur outside an \iter\ form; these
declarations must be repeated inside the form.

Here is another way we could have declared the type of |el|:
\begin{program}
(iter (for (the fixnum el) in number-list)
      (counting (oddp el)))
\end{program}
\iter\ extends the Common Lisp |the|\lispindex{the}
form to apply to variables as well as value-producing forms; anywhere
a variable is allowed---in a |with| clause, as the iteration
variable in a driver clause, as the |into| argument of an
accumulation clause, even inside a destructuring template---you can
write |(the ~type~ ~symbol~)| instead.

There is one crucial difference between using a |the| form and
actually declaring the variable: explicit declarations are always
placed in the generated code, but type information from a |the|
form is not turned into an actual declaration unless you tell \iter\
to do so using |iterate:declare-variables|.  See below.

\begin{sloppypar}
Declaring the types of internal variables is harder than declaring the
types of explicitly mentioned variables, since their names
are unknown.  You do it by declaring |iterate:declare-variables|
somewhere inside the top level of the \iter\ form.  (This will also
generate declarations for variables declared using |the|.)
\iter\ does not provide much selectivity here: it's all or none. 
And unfortunately, since \iter\ is not privy to compiler information
but instead reads declarations itself, it will not hear if you
|(declaim (iterate:declare-variables))|.  Instead, set the variable 
|iterate::*always-declare-variables*| to |t| at
compile-time, using |eval-when|.
\end{sloppypar}

To determine the appropriate types for internal variables, \iter\ uses
three sources of information:

\begin{itemize}

\item Often, the particular clause dictates a certain type for a
variable; \iter\ will use this information when available.  In the
current example, the variable |list17| will be given the type
|list|, since that is the only type that makes sense; and the
variable |result| will be given the type |fixnum|, on the
assumption that you will not be counting high enough to need bignums.
You can override this assumption only by using and explicitly declaring a
variable: 
\begin{verbatim}
(iter (declare (iterate:declare-variables))
      (for el in number-list)
      (count (oddp el) into my-result)
      (declare (integer my-result))
      (finally (return my-result)))
\end{verbatim}

\begin{sloppypar}
Other examples of the type assumptions that \iter\ makes are: type
|list| for |into| variables of collection clauses; type |list| for
expressions that are to be destructured; type |vector| for the
variable holding the vector in a |for\dots in-vector| clause, and
similarly for |string| and the |for\dots in-string| clause;
and the implementation-dependent type
|(type-of array-dimension-limit)| for the index and limit
variables generated by sequence iteration drivers like |for\dots
in-vector| and |for\dots in-string| (but not |for\dots in-sequence|,
because it may be used to iterate over a list). 
\end{sloppypar}

\item Sometimes, \iter\ will examine expressions and try to determine
their types in a simple-minded way.  If the expression is
self-evaluating (like a number, for instance), \iter\ knows that the
expression's type is the same as the type of the value it denotes, so
it can use that type.  If the expression is of the form |(the ~type~
~expr~)|, \iter\ is smart enough to extract ~type~ and use it.
However, the current version of \iter\ does not  
examine declarations of function result types or do any type
inference.  It will not determine, for 
example, that the type of |(+ 3 4)| is |fixnum|, or even
|number|.

\item In some cases, the type of an internal variable should match the
type of some other variable.  For instance, \iter\ generates an
internal variable for |(f x)| in the
clause |(for i from 1 to (f x))|, and in the absence of other
information will give it the same type as |i|.  If, however, the
expression had been written |(the fixnum (f x))|, then \iter\
would have given the internal variable the type |fixnum|
regardless of |i|'s type.  The type incompatibility errors that
could arise in this situation are not checked for.

\end{itemize}

Note that if you do declare |iterate:declare-variables|, then
\iter\ may declare user variables as well as internal ones if they do
not already have declarations, though only for variables that it
binds.  For instance, in this code:

\begin{program}
(iter (declare (iterate:declare-variables))
      (for i from 1 to 10)
      (collect i into var))
\end{program}
the variable |var| will be declared to be of type |list|.


\subsection{Summary}

\iter\ understands standard Common Lisp variable type declarations
that occur within an \iter\ form and
will pass them through to the generated code.  If the declaration
|(iterate:declare-variables)|\lispindex{declare-variables}
appears at the top level of an 
\iter\ form, or if 
|iterate::*always-declare-variables*|\lispindex{*always-declare-variables*} 
is \nonnil, then \iter\ will use the type information gleaned from user
declarations, self-evaluating expressions and |the| expressions,
combined with reasonable assumptions, to determine variable
types and declare them.


%%%%%%%%%%%%%%%%%%%%%%%%%%%%%%%%%%%%%%%%%%%%%%%%%%%%%%%%%%%%%%%%
\section{Problems with Code Movement}
\label{code-movement}

Some \iter\ clauses, or parts of clauses, result in code being
moved from the location of the clause to other parts of the loop.
Drivers behave this way, as do code-placement clauses like |initially|
and |finally|.  When using these clauses, there is a danger of writing
an expression that makes sense in its apparent location but will be
invalid or have a different meaning in another location.  For example:
\begin{program}
(iter (for i from 1 to 10)
      (let ((x 3))
        (initially (setq x 4))))
\end{program}
While it may appear that the |x| of |(initially (setq x 4))| is the
same as the |x| of |(let ((x 3)) \dots|, in fact they are not:
|initially| moves its code outside the loop body, so |x| would refer
to a global variable.  Here is another example of the same problem:
\begin{program}
(iter (for i from 1 to 10)
      (let ((x 3))
        (collect i into x)))
\end{program}
If this code were executed, |collect| would create a binding for its
|x| at the top level of the \iter\ form that the |let| will shadow.

Happily, \iter\ is smart enough to catch these errors; it
walks all problematical code to ensure that free variables are not
bound inside the loop body, and checks all variables it binds for the
same problem.

However, some errors cannot be caught:

\begin{program}
(iter (with x = 3)
      (for el in list)
      (setq x 1)
      (reducing el by \#'+ initial-value x))
\end{program}
|reducing| moves its |initial-value| argument to the initialization
part of the loop in order to produce more efficient code.  Since
\iter\ does not perform data-flow analysis, it cannot determine that
|x| is changed inside the loop; all it can establish is that |x| is
not bound internally.  Hence this code will not signal an
error and will use $3$ as the initial value of the reduction.

The following list summarizes all cases that are subject to these code
motion and variable-shadowing problems.
\begin{itemize}
\item Any variable for which \iter\ creates a binding, including those
used in |with| and the |into| keyword of many clauses.

\begin{sloppypar}
\item The special clauses which place code: |initially|, |after-each|, |else|,
|finally| and |finally-protected|. 
\end{sloppypar}

\item The variables of a |next| or |do-next| form.

\item The |initially| arguments of |for\dots initially\dots then| and
|for\dots previous|. 

\item The |then| argument of |for\dots initially\dots then|.

\item The |initial-value| arguments of |reducing| and |accumulate|.

\item The |on-failure| argument of |finding\dots such-that|.
\end{itemize}

%%%%%%%%%%%%%%%%%%%%%%%%%%%%%%%%%%%%%%%%%%%%%%%%%%%%%%%%%%%%%%%%
\section{Differences Between {\tt Iterate} and {\tt Loop}}
|loop| contains a great deal of complexity which \iter\ tries to
avoid.  Hence many esoteric features of |loop| don't exist in \iter.
Other features have been carried over, but in a cleaned-up form.  
And of course, many new features have been added; they are not
mentioned in this list.

\begin{itemize}

\item \iter's syntax is more Lisp-like than |loop|'s, having a higher
density of parens.

\item The current implementation of \iter, unlike the current version
of  |loop| (as documented in {\em Common Lisp, 2nd Ed.\/}), is
extensible (see section \ref{extend}).  

\item |loop| puts the updates of all driver variables at the top of
the loop; \iter\ leaves them where the driver clauses appear.

\item While for the most part \iter\ clauses that resemble |loop| clauses
behave similarly, there are some differences.  For instance, there is
no |for\dots =\dots then| in \iter; instead use 
|for\dots initially\dots then|.  

\item |loop| binds the variable |it| at certain times
to allow pseudo-English expressions like |when ~expr~ return it|. 
In \iter, you must bind ~expr~ to a variable yourself.  Note that
|when ~expr~ return it| is like |thereis ~expr~| except that the latter is an
accumulation clause and therefore competes with other accumulations
(remember section \ref{multiple} above).

% repeat different behaviour of |always| clause here?

\item |loop| has a special |return| clause, illustrated in the
previous item.  \iter\ doesn't need one,  since an ordinary Lisp
|return| has the same effect.

\item |loop| allows for parallel binding and stepping of iteration
variables.  \iter\ does not.  (See section \ref{bindings}.)

\item |loop| and \iter\ handle variable type declarations very
differently.  |loop| provides a special syntax for declaring variable
types, and does not examine declarations.  Moreover, the standard
implementation of |loop| will
generate declarations when none are requested.
\iter\ parses standard Common Lisp type declarations, and will never
declare a variable itself unless declarations are 
specifically requested.

\end{itemize}

%%%%%%%%%%%%%%%%%%%%%%%%%%%%%%%%%%%%%%%%%%%%%%%%%%%%%%%%%%%%%%%%
\section{Rolling Your Own}
\label{extend}

\subsection{Introduction}

\iter\ is extensible---you can write new clauses that embody new
iteration patterns.  You might want to write a new driver clause for a
data structure of your own, or you might want to write a clause that
collects or manipulates elements in a way not provided by \iter. 

This section describes how to write clauses for \iter.  Writing a
clause is like writing a macro.  In fact, writing a clause ~is~
writing a macro: since \iter\ code-walks its body and macroexpands,
you can add new abstractions to \iter\ with good old |defmacro|.

Actually, there are two extensions you can make to \iter\ that are
even easier than writing a macro.  They are adding a synonym for an
existing clause and defining a driver clause for an indexable
sequence.  These can be done with |defsynonym| and
|defclause-sequence|, respectively.  See section \ref{aids}, below.

The rest of this section explains how to write macros that expand into
\iter\ clauses.
Here's how you could add a simplified version of \iter's |multiply|
clause, if \iter\ didn't already have one:
\begin{program}
(defmacro multiply (expr)
  `(reducing ,expr by \#'* initial-value 1))
\end{program}

If you found yourself summing the square of an expression often, you
might want to write a macro for that.  A first cut might be
\begin{program}
(defmacro sum-of-squares (expr)
  `(sum (* ,expr ,expr)))
\end{program}
but if you are an experienced macro writer, you will realize that this
code will evaluate ~expr~ twice, which is probably a bad idea.  A
better version would use a temporary:
\begin{program}
(defmacro sum-of-squares (expr)
  (let ((temp (gensym)))
    `(let ((,temp ,expr))
       (sum (* ,temp ,temp)))))
\end{program}
Although this may seem complex, it is just the sort of thing you'd
have to go through to write any macro, which illustrates the point of
this section: if you can write macros, you can extend \iter.

Our macros don't use \iter's keyword-argument
syntax.  We could just use keywords with |defmacro|, but we would
still not be using \iter's clause indexing mechanism.  Unlike Lisp,
which uses just the first symbol of a form to determine what function
to call, \iter\ individuates clauses by the list of required keywords.
For instance, |for\dots in| and |for\dots in-vector| are different clauses
implemented by distinct Lisp functions.  

To buy into this indexing scheme, as well as the keyword-argument
syntax, use |defmacro-clause|:

\begin{clauses}

\defmacro{defmacro-clause}{arglist |\&body| body}
Defines a new \iter\ clause.  ~arglist~ is a list of symbols which are
alternating keywords and arguments.  \opt\ may be used, and the list
may be terminated by |\&sequence|.  ~body~ is an ordinary macro body,
as with |defmacro|.  If the first form of ~body~ is a string, it is
considered a documentation string and will be shown by
|display-iterate-clauses|. |defmacro-clause| will signal an error if 
defining the clause would result in an ambiguity.  E.g. you cannot
define the clause |for\dots from| because there would be no way to
distinguish it from a use of the |for| clause with optional keyword |from|.

\end{clauses}

\medskip

Here is |multiply| using |defmacro-clause|.  The keywords are capitalized
for readability.
\begin{program}
(defmacro-clause (MULTIPLY expr \opt\ INTO var)
  `(reducing ,expr by \#'* into ,var initial-value 1))
\end{program}
You don't have to worry about the case when |var| is not supplied; for
any clause with an |into| keyword, saying |into nil| is equivalent to
omitting the |into| entirely.

As another, more extended example, consider the fairly common
iteration pattern that 
involves finding the sequence element that maximizes (or minimizes) some
function.  \iter\ provides this as |finding\dots maximizing|, but it's
instructive to see how to write it.
Here, in pseudocode, is how you might write such a loop for
maximizing a function F:

\begin{quote}
\begin{tabbing}
set variable MAX-VAL to NIL; \\
set variable WINNER to NIL; \\
for\=\ each element EL in the sequence \\
\>  if\=\ MAX-VAL\=\ is NIL or F(EL) $>$ MAX-VAL then \\
\>\>    set MAX-VAL to F(EL); \\
\>\>    set WINNER to EL; \\
\>  end if; \\
end for; \\
return WINNER.
\end{tabbing}
\end{quote}

Here is the macro:
\begin{program}
(defmacro-clause (FINDING expr MAXIMIZING func \opt\ INTO var)
  (let ((max-val (gensym))
        (temp1 (gensym))
        (temp2 (gensym))
        (winner (or var iterate::*result-var*)))
    `(progn 
       (with ,max-val = nil)
       (with ,winner = nil)
       (cond 
        ((null ,max-val)
         (setq ,winner ,expr)
         (setq ,max-val (funcall ,func ,winner))
        (t
         (let* ((,temp1 ,expr)
                (,temp2 (funcall ,func ,temp1)))
           (when (> ,temp2 ,max-val)
             (setq ,max-val ,temp2)
             (setq ,winner ,temp1))))))
       (finally (leave ,winner)))))
\end{program}
Note that if no |into| variable is supplied, we use
|iterate::*result-var*|, which contains the internal variable into
which all clauses place their results.  If this variable is bound by
some clause, then \iter\ will return its value automatically;
otherwise, \nil\ will be returned.  

\subsection{Writing Drivers}

In principle, drivers can be implemented just as easily as other
\iter\ clauses.  In practice, they are a little harder to get right. 
As an example, consider writing a driver that
iterates over all the 
elements of a vector, ignoring its fill-pointer.  |for\dots in-vector|
won't work for this, because it observes the fill-pointer.  It's necessary to
use |array-dimension| instead of |length| to obtain the size of the
vector.  Here is one approach:

\begin{program}
(defmacro-clause (FOR var IN-WHOLE-VECTOR v)
  "All the elements of a vector (disregards fill-pointer)"
  (let ((vect (gensym))
        (index (gensym)))
    `(progn
       (with ,vect = ,v)
       (for ,index from 0 below (array-dimension ,vect 0))
       (for ,var = (aref ,vect ,index)))))
\end{program}
Note that we immediately put |v| in a variable, in case it is an
expression.  Again, this is just good Lisp macrology.  It also has a
subtle effect on the semantics of the driver: |v| is evaluated only
once, at the beginning of the loop, so changes to |v| in the loop have
no effect on the driver.  Similarly, the bounds for numerical iteration
e.g. the above |array-dimension| are also evaluated once only.  This is how
all of \iter's drivers work.


There is an important point concerning the |progn| in this code.  We
need the |progn|, of course, because we are returning several forms,
one of which is a driver.  But \iter\ drivers must occur at top-level.
Is this code in error?  No, because ~top-level~ is defined in \iter\
to include forms inside a |progn|.  This is just the definition of
top-level that Common Lisp uses, and for the same reason: to allow
macros to return multiple forms at top-level.

While our |for\dots in-whole-vector| clause will work, it is not
ideal.  In particular, it does not support generating.  Do do so, we
need to use |for\dots next| or |for\dots do-next|.  The job is
simplified by the |defmacro-driver| macro.

\begin{clauses}

\defmacro{defmacro-driver}{arglist |\&body| body}
Defines a driver clause in
both the |for| and |generate| forms, and provides a parameter
|generate| which ~body~ can examine to determine how it was invoked.
~arglist~ is as in |defmacro-clause|, and should begin with the symbol
|for|. 

\end{clauses}

With |defmacro-driver|, our driver looks like this:
\begin{program}
(defmacro-driver (FOR var IN-WHOLE-VECTOR v)
  "All the elements of a vector (disregards fill-pointer)"
   (let ((vect (gensym))
         (end (gensym))
         (index (gensym))
         (kwd (if generate 'generate 'for)))
     `(progn
        (with ,vect = ,v)
        (with ,end = (array-dimension ,vect 0))
        (with ,index = -1)
        (,kwd ,var next (progn (incf ,index)
                               (if (>= ,index ,end) (terminate))
                               (aref ,vect ,index))))))
\end{program}


We are still missing one thing: the |\&sequence| keywords.  
We can get them easily enough, by writing 
\begin{program}
(defmacro-driver (FOR var IN-WHOLE-VECTOR v \&sequence) 
  ...)
\end{program}
We can now refer to parameters |from|, |to|, |by|, etc. which contain
either the values for the corresponding keyword, or \nil\ if the
keyword was not supplied.  Implementing the right code for these
keywords is cumbersome but not difficult; it is left as an exercise.
But before you begin, see |defclause-sequence| below for an easier way.
       
\subsection{Extensibility Aids}
\label{aids}

This section documents assorted features that may be of use in
extending \iter.

\begin{clauses}

\defunexpvar{*result-var*}
Holds the variable that is used to return a value as a result of the
\iter\ form.  You may examine this and use it in a |with| clause, but
you should not change it.

\defmacro{defsynonym}{syn word}
 Makes ~syn~ a synonym for the existing \iter\ keyword ~word.~  Only
the first word in each clause can have synonyms. 

\defmacroxx{defclause-sequence}{element-name index-name}{|\&key|
access-fn size-fn sequence-type}{element-type
element-doc-string index-doc-string}  
Provides
a simple way to define sequence clauses.  Generates two
clauses, one for iterating over the sequence's elements, the other
for iterating over its indices.  The first symbol of both
clauses will have print-name |for|.
~element-name~  and ~index-name~ should be symbols.
~element-name~ is the second keyword of the element iterator (typically of
the form 
|in-~sequence-type~|), and ~index-name~ is the second keyword
of the index-iterator (typically of the form
|index-of-~sequence-type~|).  Either name may be 
\nil, in which case the corresponding clause is not defined.  If both
symbols are supplied, they should be in the same package.  The |for|
that begins the clauses will be in this package.

\cpar ~access-fn~ is the function to be used to
access elements of the sequence in the element iterator.  The function
should take two 
arguments, a sequence and an index, and return the appropriate element.
~size-fn~ should denote a function of one argument, a sequence, that
returns its size.  Both ~access-fn~ and ~size-fn~ are required for the
element iterator, but only ~size-fn~ is needed for the index iterator.

\cpar The ~sequence-type~ and ~element-type~ keywords are used to
suggest types for the variables 
used to hold the sequence and the
sequence elements, respectively.  The usual rules about \iter's
treatment of variable type declarations apply (see section \ref{types}).

\cpar ~element-doc-string~ and ~index-doc-string~ are
the documentation strings, for use with |display-iterate-clauses|. 

\cpar The generated element-iterator performs destructuring on the
element variable.

\cpar As an example, the above |for\dots in-whole-vector| example
could have been written:
\begin{program}
(defclause-sequence IN-WHOLE-VECTOR INDEX-OF-WHOLE-VECTOR
  :access-fn 'aref
  :size-fn \#'(lambda (v) (array-dimension v 0))
  :sequence-type 'vector
  :element-type t
  :element-doc-string 
     "Elements of a vector, disregarding fill-pointer"
  :index-doc-string 
     "Indices of vector, disregarding fill-pointer")
\end{program}

\end{clauses}

\subsection{Subtleties}

There are some subtleties to be aware of when writing \iter\ clauses.
First, the code returned by your macros may be |nconc|'ed into a list,
so you should always returned freshly consed lists, rather than
constants.  Second, \iter\ matches clauses by using |eq| on the first
symbol and |string=| on the subsequent ones, so the package of the
first symbol of a clause is relevant.  All of the clauses in this manual
have their first word in the \iter\ package.  
You can use the package system in the usual way to shadow
\iter\ clauses without replacing them.

%%% say more here, about the badness that only the first word of a
%%% clause is packagey.

\section{Non-portable Extensions to Iterate (Contribs)}
\label{contribs}

Currently, there is only one non-portable extension to iterate in the
distribution: iterate-pg. If you have made an extension that depends
on non-portable features, feel free to send them to |asf@boinkor.net|
for inclusion in the iterate distribution.

\subsection{An SQL query driver for iterate}

The pg package by Eric Marsden (see |http://cliki.net/pg|) provides an
interface to the PostgreSQL database. Using the \iterpg\ extension, it
is possible to handle the results of SQL queries using \iter.

This usage example should give you an idea of how to use it:

\begin{program}
(pg:with-pg-connection (c "somedb" "someuser")
  (iter (for (impl version date) in-relation "select * from version"
                                 on-connection *dbconn*)
        (collect version)))
\end{program}

To use the extension via |ASDF|, simply make your system depend on the
|iterate-pg| system instead of the |iterate| system. To load it
manually, use:

\begin{program}
  (asdf:oos 'asdf:load-op :iterate-pg)
\end{program}

%%%%%%%%%%%%%%%%%%%%%%%%%%%%%%%%%%%%%%%%%%%%%%%%%%%%%%%%%%%%%%%%
\section{Obtaining {\tt Iterate}}

\begin{sloppypar}
The information in this chapter is obsolete but included for
completeness's sake; Currently, the most up-to-date information on
\iter\ can be found at |http://boinkor.net/iterate.html|.
\end{sloppypar}

\begin{sloppypar}
\iter\ currently runs on Lisp Machines, and on
HP's, Sun3's and Sparcstations under Lucid.  
\iter\ source and binaries are available at the MIT AI Lab in the
subdirectories of |/src/local/lisplib/|.  The source file,
|iterate.lisp|, is also available for anonymous FTP in the directory
|/com/fpt/pub/| on the machine |TRIX.AI.MIT.EDU| (Internet number
128.52.32.6).  If you are unable to obtain |iterate| in one of these
ways, send mail to |jba@ai.mit.edu| and I will send you the source
file. 
\end{sloppypar}

\begin{sloppypar}
\iter\ resides in the |iterate| package (nickname |iter|).  Just say
\linebreak |(use-package :iterate)| to make all the necessary symbols
available. 
If a symbol is not exported, it appears in this manual with an
``|iterate::|'' prefix.
\end{sloppypar}

Send bug reports to |bug-iterate@ai.mit.edu|.  The |info-iterate|
mailing list will have notices of changes and problems; to have
yourself added, send mail to |info-iterate-request@ai.mit.edu|.


\medskip

\begin{flushleft}
 \bf Acknowledgements
\end{flushleft}
\smallskip

Richard Waters provided invaluable criticism which spurred me to improve
\iter\ greatly.  As early users, David Clemens, Oren Etzioni and Jeff
Siskind helped ferret out many bugs.

%\begin{theindex}
% The index files must be generated with the genindex program
% or from makeindex iterate-manual.idx which creates iterate-manual.ind.

%\baselineskip\normalbaselineskip
%\advance\baselineskip by -2pt

\input{iterate-manual.ind}

%\underline{Clauses}
%\input{iterate-manual.clindex}
%
%\indexspace
%\underline{Lisp}
%\input{iterate-manual.lispindex}

%\end{theindex}


\end{document}


% arch-tag: "c0f871a6-313c-11d8-abb9-000c76244c24"


%\underline{Clauses}
%% Fix ``.'' spacing in lists
\documentstyle[12pt,aimemo]{article}

\setlength{\textwidth}{6.0in}
\setlength{\oddsidemargin}{0.25in}
\setlength{\evensidemargin}{0.25in}
\setlength{\topmargin}{0pt}
\setlength{\headsep}{24pt}
\setlength{\textheight}{8.5in}

\let\origpar=\par

% somehow, LaTex fucks with things so that the plain Tex \obeyspaces
% doesn't work.  This is the same, except we use an mbox containing
% a space instead of just a space.

\newcommand{\spc}{\mbox{ }}
\newcommand{\tAb}{\mbox{        }}
\def\obspaces{\catcode`\ =\active\catcode`\^^I=\active}
{\obspaces \global\let =\spc\global\let^^I=\tAb}

\newenvironment{program}{\medskip\samepage\tt\obeylines\obspaces}{\medskip}

% use | for \tt
\catcode`\|=13
\newif\ifvbar
\def|{\ifvbar
         \endgroup\vbarfalse
      \else 
         \begingroup\tt\vbartrue 
      \fi}



\newcommand{\lisp}{\tt}
\newcommand{\iter}{|iterate|}
\newcommand{\iterpg}{|iterate-pg|}
\newcommand{\setf}{|setf|}
\newcommand{\nil}{|nil|}
\newcommand{\nonnil}{non-\nil}
\newcommand{\opt}{{\lisp \&optional}}
\newcommand{\yields}{$\Rightarrow$}



\newbox\Dots
\setbox\Dots=\hbox{\small ...}
\renewcommand{\dots}{\copy\Dots}

% Here's a way to define an environment which will treat all
% paragraphs inside it in the same manner.  It's, as near as I can
% figure, the way that Latex does things.  Using \everypar won't work;
% apparently Latex resets that a lot.  What you have to do is: at the
% top of the file do \let\origpar=\par.  In the
% begin part of the environment, do the parshape (or whatever), then say
% \def\par{{\origpar}}.  The extra braces are necessary.   That's all
% you have to do; at the end of the environment things will return to
% normal automatically.  Why does this all work?  I have no idea.
% Another important thing: there should be a blank line (i.e. a \par)
% between the end of the last paragraph and the \end{environment}.

\newlength{\clindent}
\setlength{\clindent}{0.5in}
\newlength{\clparindent}
\setlength{\clparindent}{\parindent}

\newenvironment{clauses}{\advance\linewidth -\parindent
                        \hangindent\clindent\relax
                         \hangafter=0
                          \parindent=0pt
                          \def\par{{\origpar}}}{}

\newcommand{\cpar}{\hspace \clparindent}

\newcommand{\startitem}{\bigskip\pagebreak[3]}
\newcommand{\finishitem}{\smallskip}

\newcommand{\clause}[1]{\startitem\Clause{#1}\finishitem}
\newcommand{\Clause}[1]{\hbox{#1}}
\newcommand{\clindex}[1]{\index{{\lisp #1}}}
\newcommand{\clindexx}[2]{\index{{\lisp #1\dots #2}}}
\newcommand{\clindexxx}[3]{\index{{\lisp #1\dots #2\dots #3}}}
\newcommand{\clausex}[2]{\startitem\Clausex{#1}{#2}\finishitem}
\newcommand{\Clausex}[2]{\Clause{|#1| #2}\clindex{#1}}

\newcommand{\defvar}[1]{\startitem\deff{#1}{}{Variable}\finishitem}
\newcommand{\defunexpvar}[1]{\startitem\deffnoindex{iterate::#1}{}{Variable}
                                        \lispindex{#1}\finishitem}
\newcommand{\defun}[2]{\startitem\deff{#1}{#2}{Function}\finishitem}
\newcommand{\defmacro}[2]{\startitem\deff{#1}{#2}{Macro}\finishitem}
\newcommand{\defunx}[3]{\startitem\deffx{#1}{#2}{#3}{Function}\finishitem}
\newcommand{\defmacrox}[3]{\startitem\deffx{#1}{#2}{#3}{Macro}\finishitem}
\newcommand{\defunxx}[4]{\startitem\deffxx{#1}{#2}{#3}{#4}{Function}\finishitem}
\newcommand{\defmacroxx}[4]{\startitem\deffxx{#1}{#2}{#3}{#4}{Macro}\finishitem}

\newcommand{\deff}[3]{\deffnoindex{#1}{#2}{#3}\lispindex{#1}}
\newcommand{\deffnoindex}[3]{\hbox to \hsize{{\tt #1} {\it #2}\hfill [{\it #3}]}}
\newcommand{\deffx}[4]{\deff{#1}{#2}{#4}\moreargs{#1}{#3}}
\newcommand{\deffxx}[5]{\deffx{#1}{#2}{#3}{#5}\moreargs{#1}{#4}}

\newcommand{\moreargs}[2]{\hbox to \hsize{\phantom{\lisp #1} {\it #2}\hfill}}

\newcommand{\lispindex}[1]{\index{{\lisp #1}}}

 \newenvironment{note}[1]{\pagebreak[2]\bigskip
                          \hrule\smallskip\small
                         {\setlength{\parindent}{0pt}
                           \par{\bf #1:}}}{\smallskip\hrule\bigskip}



\makeindex

%%%%%%%%%%%%%%%%%%%%%%%%%%%%%%%%%%%%%%%%%%%%%%%%%%%%%%%%%%%%%%%%

%\thispagestyle{empty}

%\vspace*{2in}

%\title{The |Iterate| Manual}
%\author{Jonathan Amsterdam}

%\maketitle

%\thispagestyle{empty}

%\begin{center}
%\Large *** DRAFT ***
%\end{center}
%This manual is in beta-test. It will become an AI Memo in a couple of
%months.  I would appreciate any comments or
%corrections.  You can reach me at email address |jba@ai.ai.mit.edu|.

\begin{document}

\def\writtenby{Jonathan Amsterdam}
\def\contractno{This report describes research done at the Artificial
Intelligence Laboratory of the Massachusetts Institute of Technology.
Support for the laboratory's artificial intelligence research is
provided in part by the Advanced Research Projects Agency of the
Department of Defense under Office
of Naval Research contract N00014-85-K-0124.}

\AIMEMO{1236}{\Large The Iterate Manual \\[1in]}
{This is the manual for version 1.4 of \iter, a powerful iteration
macro for Common  
Lisp. \iter\ is similar to {\tt loop} but provides numerous additional
features, is well integrated with Lisp, and is extensible. 
\vfill }


% use ~ for \em
\catcode`\~=13
\newif\ifem
\def~{\ifem
         \/\endgroup\emfalse
      \else 
         \begingroup\em\emtrue 
      \fi}



%\pagebreak
%\setcounter{page}{1}

%\thispagestyle{empty}


\tableofcontents

\pagebreak
\pagestyle{headings}

\section{Introduction}

\begin{sloppypar}
This manual describes
\iter, a powerful iteration facility for Common Lisp.
\iter\ provides abstractions for many common iteration
patterns and allows for the definition of additional patterns.  \iter\ is
a macro that expands into ordinary Lisp at
compile-time, so it is more efficient than higher-order
functions like |map| and |reduce.|  
While it is similar to |loop|, \iter\ offers a more Lisp-like syntax
and enhanced extensibility.
(For a more complete
comparison of \iter\ with other iteration constructs, see MIT AI Lab Working
Paper No. 324, ~Don't Loop, Iterate.~)
\end{sloppypar}

An \iter\ form consists of the symbol |iter|\footnote{You can also use
|iterate|, but |iter| is preferred because it avoids potential conflicts with
possible future additions to Common Lisp, and because it saves
horizontal space when writing code.}
followed by one or more
forms, some of which may be \iter\ 
~clauses.~  Here is a simple example of \iter\ which collects the
numbers from 1 to 10 into a list, and returns the list.  The return
value is shown following the arrow.
\begin{program}
(iter (for i from 1 to 10)
      (collect i))          \yields (1 2 3 4 5 6 7 8 9 10)
\end{program}
This form contains two clauses: a |for| clause that steps the
variable |i| over the integers from 1 to 10, and a |collect|
clause that accumulates its argument into a list.  With a few
exceptions, all \iter\ clauses have the same format: alternating
symbols (called ~keywords~) and expressions (called
~arguments~).  The syntax and terminology are those of Common
Lisp's keyword lambda lists.  One difference is that \iter's keywords
do not have to begin with a colon---though they may, except for the
first symbol of a clause.  So you can also write  |(for i :from
1 :to 10)| if you prefer.

Any Lisp form can appear in the body of an \iter, where it will have
its usual meaning.  \iter\ walks the entire body, expanding macros,
and recognizing clauses at any level.  This example collects all the
odd numbers in a list:
\begin{program}
(iter (for el in list)
      (if (and (numberp el) (oddp el))
          (collect el)))
\end{program}

There are clauses for iterating over numbers, lists, arrays and other
objects, and for
collecting, summing, counting, maximizing and other useful operations.
\iter\ also supports the creation of new variable bindings, stepping
over multiple sequences at once, destructuring, and compiler
declarations of variable types.  The following example illustrates
some of these features: 
\begin{program}
(iter (for (key . item) in alist)
         (for i from 0)
         (declare (fixnum i))
         (collect (cons i key)))
\end{program}

This loop takes the keys of an alist and returns a new alist
associating the keys with their positions in the original list. The
compiler declaration for |i| will appear in the generated code
in the appropriate place.

%%%%%%%%%%%%%%%%%%%%%%%%%%%%%%%%%%%%%%%%%%%%%%%%%%%%%%%%%%%%%%%%
\section{Clauses}

Most of \iter's clauses will be familiar to |loop| programmers.  
(|loop| is an iteration macro that has been incorporated into Common
Lisp.  Seen Guy Steele's {\em Common Lisp, 2nd Edition}.)
In
nearly all cases they behave the same as their |loop| counterparts, so
a |loop| user can switch to \iter\ with little pain (and much gain).

All clauses with the standard keyword-argument syntax consist of two
parts: a ~required~ part, 
containing keywords that must be present and in the right order; and
an ~optional~ part, containing keywords that may be omitted and,
if present, may occur in any order.  In the descriptions below, the
parts are separated by the Lisp lambda-list keyword \opt.


\subsection{Drivers}

An iteration-driving clause
conceptually causes the iteration to go forward.  Driver clauses in
\iter\ allow iteration over numbers, lists, vectors, hashtables, packages,
files and streams.  Iteration-driving clauses must
appear at the top level of an \iter\ form; they cannot be nested
inside another clause.  The driver variable is updated at the point
where the driver clause occurs.  Before the clause is executed for the
first time, the value of the variable is undefined.

%Also, regardless of where the driver clause
%appears in the body, the driver variable is stepped at the top of the
%loop; hence it is stylistically preferable, though
%not required, to place driver clauses at the beginning of the \iter.

Multiple drivers may appear in a single \iter\ form, in which case all
of the driver variables are updated each time through the loop, in the
order in which the clauses appear.  The first driver to terminate will
terminate the entire loop.

In all cases, the value of the driver variable on exit from the loop,
including within the epilogue code (see the |finally| clause), is
undefined.

All the parameters of a driver clause are evaluated once, before the
loop begins.  Hence it is not possible to change the bounds or other
properties of an iteration by side-effect from within the loop.

With one exception, driver clauses begin with the word |for| (or
the synonym |as|) and mention an iteration variable, which is
given a binding within the \iter\ form.
The exception is |repeat|, which just executes a loop a
specified number of times:

\begin{clauses}

\clausex{repeat}{~n~}
Repeats the loop ~n~ times.  For example:
\begin{program}
(iter (repeat 100)
      (print "I will not talk in class."))
\end{program}
\cpar If $n \leq 0$, the
loop will never be executed.  If ~n~ is not an integer, the
actual number of executions will be $\lceil n \rceil$.

\end{clauses}

\pagebreak[3]

\subsubsection{Numerical Iteration}

\begin{clauses}
\clausex{for}{~var~ |\&sequence|} 
The general form for iterating over a sequence of numbers requires a
variable and, optionally, one or more keywords that provide the bounds
and step size of the iteration. The |\&sequence|
lambda-list keyword is a shorthand for these sequence keywords.  They are:
|from|, |upfrom|, |downfrom|, |to|, |downto|, |above|, |below| and |by|.
|from| provides the starting value for ~var~
and defaults to zero. |to| provides a final value and implies that the
successive values of ~var~ will be increasing; |downto| implies that
they will be decreasing.  The loop terminates when ~var~ passes
the final value (i.e. becomes smaller or larger than it, depending on
the direction of iteration); in other  words, the loop body
will never be 
executed for values of ~var~ past the final value.
|below| and |above| are similar to |to| and
|downto|, except that the loop terminates when ~var~ equals or passes
the final value.

\cpar If no final value is specified, the variable will be stepped
forever.  Using |from| or |upfrom| will result in increasing
values, while |downfrom| will give decreasing values.

\cpar On each iteration,
~var~ is incremented or decremented by the value of the sequence
keyword |by|, which defaults to 1.  It should always be a positive
number, even for downward iterations.

\cpar In the following examples, the sequence of numbers generated is shown
next to the clause.

\begin{program}
(for i upfrom 0) \yields\ 0 1 2 \ldots
(for i from 5)   \yields\ 5 6 7 \ldots    ; either from or upfrom is okay
(for i downfrom 0) \yields\ 0 -1 -2 \ldots
(for i from 1 to 3) \yields\ 1 2 3
(for i from 1 below 3) \yields\ 1 2
(for i from 1 to 3 by 2) \yields\ 1 3
(for i from 1 below 3 by 2) \yields\ 1
(for i from 5 downto 3) \yields\ 5 4 3
\end{program}

\end{clauses}

\subsubsection{Sequence Iteration}

There are a number of clauses for iterating over sequences.  In all of
them, the argument following |for| may be a list instead of a
symbol, in which case destructuring is performed.  See section
\ref{destructuring}. 

\begin{clauses}

\clause{|for| ~var~ |in| ~list~ \opt\ |by| ~step-function~}
\clindexx{for}{in}
~var~ is set to successive elements of list.  
~step-function,~ which 
defaults to |cdr|, is used to obtain the next sublist. 

\clause{|for| ~var~ |on| ~list~ \opt\ |by| ~step-function~}
\clindexx{for}{on}
~var~ is set to successive sublists of list. 
~step-function~ (default |cdr|) is used as in |for\dots in|.

\end{clauses}

\medskip

These two clauses use |atom| to test for the end
of a list.  Hence, given a list whose final |cdr| is not \nil,
they will silently ignore the last |cdr|.  Other choices are
|endp|,  which would signal an error, and |null|, which
would probably result in an error somewhere else.  If you wish to use
an end-test other than |atom|, set the variable
|iterate::*list-end-test*|\lispindex{*list-end-test*} to the name of
the desired function. 

\begin{clauses}
\clause{|for| ~var~ |in-vector| ~vector~ |\&sequence|}
\clindexx{for}{in-vector}
~var~ takes on successive elements from ~vector.~   The vector's
fill-pointer is observed.  Here and in subsequent clauses, the |\&sequence|
keywords include |with-index|, which takes a symbol as argument
and uses it for the index variable instead of an internally generated
symbol.  The other |\&sequence| keywords behave as in numerical
iteration, except that the default iteration bounds are the bounds of
the vector.  E.g. in |(for i in-vector v downto 3)|,
|i| will start off being bound to the last element in |v|,
and will  
be set to preceding elements down to and including the element with
index 3.

\clause{|for| ~var~ |in-sequence| ~seq~ |\&sequence|}
\clindexx{for}{in-sequence}
This uses Common Lisp's generalized sequence functions, |elt|
and |length|, to obtain elements and determine the length of
~seq.~   Hence it will work for any sequence, including lists, and will
observe the fill-pointers of vectors.

\clause{|for| ~var~ |in-string| ~string~ |\&sequence|}
\clindexx{for}{in-string}
~var~ is set to successive characters of ~string.~

\startitem
\Clause{|for| ~var~ |index-of-vector| ~vector~ |\&sequence|}
\clindexx{for}{index-of-vector}
\Clause{|for| ~var~ |index-of-sequence| ~sequence~ |\&sequence|}
\clindexx{for}{index-of-sequence}
\Clause{|for| ~var~ |index-of-string| ~string~ |\&sequence|}
\clindexx{for}{index-of-string}
\finishitem
~var~ is set to successive indices of the sequence.
These clauses avoid the overhead of accessing the sequence elements
for those applications where they do not need to be examined, or are
examined rarely.  They admit all the optional keywords of the other
sequence drivers except the (redundant) |with-index| keyword.

\clause{|for| ~(key value)~ |in-hashtable| ~table~}
\clindexx{for}{in-hashtable}

~key~ and ~value,~ which must appear as shown in a list
and may be destructuring templates, are set to the keys and values of
~table~.  If ~key~ is |nil|, then the hashtable's keys will be ignored;
similarly for ~value~.
The order in which 
elements of ~table~ will be retrieved is unpredictable.

\clause{|for| ~var~ |in-package| ~package~ \opt\ |external-only| ~ext~}
\clindexx{for}{in-package}
Iterates over all the symbols in ~package,~ or over only the
external symbols if ~ext~ is specified and non-|nil|.  ~ext~ is not
evaluated.  The same symbol may appear more than once.

\clause{|for| ~var~ |in-packages| ~packages~ \opt\ |having-access| ~symbol-types~}
\clindexx{for}{in-packages}
Iterates over all the symbols from the list of packages denoted by the
descriptor ~packages~ and having accessibility (or visibility) given by
~symbol-types~.  This defaults to the list |(:external :internal :inherited)|
and is not evaluated.  ~var~ must be a list of up to three variables: in each
iteration, these will be set to a symbol, its access-type and package (as per
|with-package-iterator| in ANSI-CL).  The same symbol may appear more than
once.

\end{clauses}

\begin{clauses}

\clause{|for| ~var~ |in-file| ~name~ \opt\ |using| ~reader~}
\clindexx{for}{in-file}
Opens the file ~name~ (which may be a string or pathname) for input,
and iterates 
over its contents. ~reader~ defaults to |read|, so by default ~var~
will be bound to the successive forms in the file.  The \iter\ body is
wrapped in an unwind-protect to ensure that the file is closed no
matter how the \iter\ is exited.

\clause{|for| ~var~ |in-stream| ~stream~ \opt\ |using| ~reader~}
\clindexx{for}{in-stream}
Like |for\dots in-file|, except that ~stream~ should be an existing
stream object that supports input operations.

\end{clauses}

\subsubsection{Generalized Drivers}

These are primarily useful for writing drivers that can also be used
as generators (see section \ref{generators}, below).
 
\begin{clauses}

\clause{|for| ~var~ |next| ~expr~}
\clindexx{for}{next}

~var~ is set to ~expr~ each time through the loop.  Destructuring is
performed.  When the clause is used as a generator, ~expr~ is the code
that is executed when |(next ~var~)| is 
encountered (see section \ref{generators}, below).
~expr~ should compute the first value for ~var~, as well
as all subsequent values, and is responsible for terminating the loop.
For compatibility with future versions of \iter, this termination
should be done with |terminate|\clindex{terminate}, which can be
considered a synonym for |finish| (see section \ref{control-flow}).

\cpar As an example, the following clauses are equivalent to |(for
i from 1 to 10)|:
\begin{program}
(initially (setq i 0))
(for i next (if (> i 10) (terminate) (incf i)))
\end{program}

\clause{|for| ~var~ |do-next| ~form~}
\clindexx{for}{do-next}
~form~ is evaluated each time through the loop.  Its value is ~not~
set to ~var~; that is ~form's~ job. ~var~ is only present so that
\iter\ knows it is a driver variable.  \linebreak |(for ~var~ next
~expr~)| is equivalent to |(for ~var~ do-next (dsetq ~var~ ~expr~))|.
(See section \ref{destructuring} for an explanation of |dsetq|.) 

\end{clauses}

\subsubsection{Generators}
\label{generators}

In all of the above clauses, the driver variable is updated on each
iteration.
Sometimes it is desirable to have greater control over updating.  
For instance, consider the problem of associating numbers, in
increasing order and with no gaps, with the
\nonnil\ elements of a list.  One obvious first pass at writing this is:

\begin{program}
(iter (for el in list)
      (for i upfrom 1)
      (if el (collect (cons el i))))
\end{program}
But on the list |(a b nil c)| this produces |((a . 1) (b . 2) (c .
4))| instead of the desired |((a . 1) (b . 2) (c . 3))|.  The problem
is that |i| is incremented each time through the loop, even when |el|
is |nil|.  

The problem could be solved elegantly if we could step |i| only when
we wished to.  This
can be accomplished for any \iter\ driver by writing |generate| (or
its synonym |generating|)
instead of |for|.  Doing so produces a ~generator~---a driver whose
values are yielded explicitly.  To obtain the next value of a
generator variable ~v~, write \linebreak |(next ~v~)|.  The value of a |next|
form is the next value of ~v~, as determined by its associated driver
clause.  |next| also has the side-effect of updating ~v~ to that
value.  If there is no next value, |next| will terminate the loop,
just as with a normal driver.


Using generators, we can now write our example like this:
\begin{program}
(iter (for el in list)
      (generate i upfrom 1)
      (if el (collect (cons el (next i)))))
\end{program}
Now |i| is updated only when |(next i)| is executed, and this occurs
only when |el| is \nonnil.

To better understand the relationship between ordinary drivers and
generators, observe that we can rewrite an ordinary driver using its
generator form immediately followed by |next|, as this example shows:
\begin{program}
(iter (generating i from 1 to 10)
      (next i)
      ...)
\end{program}
Provided that the loop body contains no |(next i)| forms, this will
behave just as if we had written |(for i from 1 to 10)|.

We can still refer to a driver variable ~v~ without using |next|; in
this case, its value is that given to it by the last evaluation of
|(next ~v~)|.  Before |(next ~v~)| has been called the first time, the
value of ~v~ is undefined.  

This semantics is more flexible than
one in which ~v~ begins the loop bound to its first value and calls
of |next| supply subsequent values, because it means the loop will not
terminate too soon if the generator's sequence is empty.  For
instance, consider the following code, which tags \nonnil\ elements of
a list using a list of tags, and also counts the null elements.
(We assume there are at least as many tags as \nonnil\ elements.)
\begin{program}
(let* ((counter 0)
       (tagged-list (iter (for el in list)
                          (generating tag in tag-list) 
                          (if (null el)
                              (incf counter)
                              (collect (cons el (next tag)))))))
  ...)
\end{program}
It may be that there are just as many tags as non-null elements of
|list|.  If all the elements of |list| are null, we still want the
counting to proceed, even though |tag-list| is \nil.  If |tag| had to be
assigned its first value before the loop begins, we would have had to
terminate the loop before the first iteration, since when |tag-list|
is \nil, |tag| 
has no first value.  With the existing semantics, however, |(next
tag)| will never execute, so the iteration will cover all the elements of
|list|. 
        
When the ``variable'' of a driver clause is actually a destructuring
template containing several variables, all the variables are eligible
for use with |next|.  As before, |(next ~v~)| evaluates to ~v's~ next
value; but the effect is to update all of the template's variables.
For instance, the following code will return the list |(a 2 c)|.
\begin{program}
(iter (generating (key . item) in '((a . 1) (b . 2) (c . 3)))
      (collect (next key))
      (collect (next item)))
\end{program}

Only driver clauses with variables can be made into generators.  This
includes all clauses mentioned so far except for |repeat|.  It does
~not~ include |for\dots previous|, |for\dots =|,
|for\dots initially\dots then| or |for\dots first\dots then| (see
below). 

\subsubsection{Previous Values of Driver Variables}

Often one would like to access the value of a variable on a previous
iteration.  \iter\ provides a special clause for accomplishing this.

\begin{clauses}

\clause{|for| ~pvar~ |previous| ~var~ \opt\ |initially| ~init~
|back| ~n~}
\clindexx{for}{previous}
Sets ~pvar~ to the previous value of ~var,~ which should be a driver
variable, a variable from another |for\dots previous| clause, or a
variable established by a |for\dots =|, 
|for\dots initially\dots then| or |for\dots first\dots then| clause
(see section \ref{setting}).
Initially, ~pvar~ is given the value ~init~ (which defaults to \nil).  
The ~init~ expression will be moved outside the loop body, so it
should not depend on anything computed within the loop.
~pvar~ retains the value of ~init~ until ~var~ is set to its second
value, at which point ~pvar~ is set to ~var's~ first value; and so on.  

\cpar The
argument ~n~ to |back| must be a constant, positive integer, and
defaults to 1.  It determines how many iterations back ~pvar~ should
track ~var~.  For example, when ~n~ is 2, then ~pvar~ will be assigned
~var's~ first value when ~var~ is set to its third value.

\cpar A |for\dots previous| clause may occur after or before its
associated driver clause. |for\dots previous| works with generators as
well as ordinary drivers.

\pagebreak[3]

\cpar Example:
\begin{program}
(iter (for el in '(1 2 3 4))
      (for p-el previous el)
      (for pp-el previous p-el initially 0)
      (collect pp-el))
\end{program}
This evaluates to |(0 0 1 2)|.  It could have been written more
economically as 
\begin{program}
(iter (for el in '(1 2 3 4))
      (for pp-el previous el back 2 initially 0)
      (collect pp-el))
\end{program}


\end{clauses}

\subsection{Variable Binding and Setting}
\label{setting}

Several clauses exist for establishing new variable bindings or for
setting variables in the loop.  They all support destructuring.

\begin{clauses}

\clausex{with}{~var~ \opt\ |=| ~value~}
Causes ~var~ to be bound to value before the loop body is entered.
If ~value~ is not supplied, ~var~ assumes a default
binding, which will be 
\nil\ in the absence of declarations.  Also, if ~value~ is not
supplied, no destructuring is performed; instead, ~var~ may be a list of
symbols, all of which are given default 
bindings.  If ~value~ is supplied, ~var~ is bound to it, with
destructuring.  

\cpar Because |with|
creates bindings whose scope includes the entire \iter\ form, it is
good style to put all |with| clauses at the beginning.

\cpar Successive occurrences of |with| result in sequential
bindings (as with 
|let*|).  There is no way to obtain parallel bindings; see
section \ref{bindings} for a rationale.  


\clause{|for| ~var~ |=| ~expr~}
\clindexx{for}{=}
On each iteration, ~expr~ is evaluated and ~var~ is set
to its value. 

\cpar This clause may appear to do the same thing as |for\dots next|.
In fact, they are quite different.  |for\dots =| provides only three
services: it sets up a binding for ~var~, sets it to ~expr~ on each
iteration, and makes it possible to use |for\dots previous| with
~var~.  |for\dots next| provides these services in addition to the
ability to turn the driver into a generator.
%Also, the code which sets ~var~ appears in the loop body in the same
%place as the |for\dots =| clause; the code for |for\dots next| appears
%at the top of the loop, as with other drivers (except when being used
%as a generator).

\clause{|for| ~var~ |initially| ~init-expr~ |then| ~then-expr~}
\clindexxx{for}{initially}{then}
Before the loop begins, ~var~ is set to ~init-expr;~ on all
iterations after the first it is set to ~then-expr.~  This clause
must occur at top-level.  ~init-expr~ will be moved outside the loop
body and ~then-expr~ will be moved to the end of the loop body, so
they are subject to code motion problems (see section
\ref{code-movement}). 

\cpar This clause may appear to be similar to |for\dots next|, but in
fact they differ significantly.  |for\dots initially\dots then| is
typically used to give ~var~ its first value before the loop begins,
and subsequent values on following iterations.  This is incompatible
with generators, whose first value and subsequent values must all be
computed by |(next ~var~)|.  Also, the update of ~var~ in 
|for\dots initially\dots then| does not occur at the location of the
clause. 
Use |for\dots initially\dots then| for
one-shot computations where its idiom is more convenient, but use
|for\dots next| for extending \iter\ with new drivers (see section
\ref{extend}).

\clause{|for| ~var~ |first| ~first-expr~ |then| ~then-expr~}
\clindexxx{for}{first}{then}
The first time through the loop, ~var~ is set to ~first-expr;~ on
subsequent iterations, it is set to ~then-expr.~  This differs from
|for\dots initially| in that ~var~ is set to ~first-expr~
inside the loop body, 
so ~first-expr~ may depend on the results of other clauses.  For
instance, 
\begin{program}
(iter (for num in list)
      (for i first num then (1+ i))
      ...)
\end{program}
will set |i| to the first element of |list| on the first
iteration, whereas 
\begin{program}
(iter (for num in list)
      (for i initially num then (1+ i))
      ...)
\end{program}
is probably erroneous; |i| will be bound to |num|'s
default binding (usually \nil) for the first iteration.

\end{clauses}

\begin{note}{Compatibility note}
|loop|'s |for\dots =| works like \iter's, but |loop| used the syntax
|for\dots =\dots then| to mean |for\dots initially\dots then|.  It was
felt that these two operations were sufficiently different to warrant
different keywords.

Also, the |for| in the above three clauses is misleading,
since none is true driver (e.g. none has a corresponding |generate|
form).  |setting| would have been a better choice, but |for| was used
to retain some compatibility with |loop|.  
\end{note}


%%%%%%%%%%%%%%%%%%%%%%%%%%%%%%%%
\subsection{Gathering Clauses}
Many of \iter's clauses accumulate values into a variable, or set a
variable under certain conditions.  At the end of the loop, this
variable contains the desired result.  All these clauses have an
optional |into| keyword, whose argument should be a symbol.  If the
|into| keyword is not supplied, the accumulation variable will be
internally generated and its value will be returned at the end of the
loop; if a variable is specified, that variable is used for the
accumulation, and is not returned as a result---it is up to the user
to return it explicitly, in the loop's epilogue code (see |finally|).
It is safe to examine the accumulation variable during the loop, but
it should not be modified.

These clauses all begin with a verb.  When the verb does
not conflict with an existing Common Lisp function, then it may be
used in either its infinitival or present-participle form (e.g. |sum|,
|summing|).  However, when there is a conflict with Common Lisp, only
the present-participle form may be used (e.g. |unioning|).  This is to
prevent \iter\ clauses from clashing with Common Lisp
functions. 

%although these clauses are described as ``producing a value,'' it is a
%mistake to think of the lisp list representing the clause as a
%value-producing form in the usual way.  clauses may legally be written
%where a value is expected, e.g. |(setq x (sum i))|, but the
%lisp value of a clause in such a context is undefined.

\subsubsection{Reductions}

~Reduction~ is an extremely common iteration pattern in which
the results of successive applications of a
binary operation are accumulated.
For example, a loop that computes the sum of the
elements of a list is performing a reduction with the addition
operation.  This could be written in Common Lisp as  |(reduce \#'+
list)| or with \iter\ as
\begin{program}
(iter (for el in list)
      (sum el))
\end{program}

\begin{clauses}

\clausex{sum}{~expr~ \opt\ |into| ~var~}
Each time through the loop, ~expr~ is evaluated and added to a
variable, which is bound initially to zero.  If ~expr~ has a type,
it is ~not~ used as the type of the sum variable, which is always
|number|.  To get the result variable to be of a more specific
type, use an explicit variable, as in
\begin{program}
(iter (for el in number-list)
      (sum el into x)
      (declare (fixnum x))
      (finally (return x)))
\end{program}

\clausex{multiply}{~expr~ \opt\ |into| ~var~}
Like |sum|, but the initial value of the result variable is $1$,
and the variable is updated by multiplying ~expr~ into it.

\clausex{counting}{~expr~ \opt\ |into| ~var~}
~expr~ is evaluated on each iteration.  If it is \nonnil, the
accumulation variable, initially zero, is incremented.

\startitem
\Clausex{maximize}{~expr~ \opt\ |into| ~var~}
\Clausex{minimize}{~expr~ \opt\ |into| ~var~}
\finishitem
~expr~ is evaluated on each iteration and its extremum (maximum or
minimum) is stored in the accumulation variable.  If ~expr~ is never
evaluated, then the result is \nil\ (if the accumulation variable is
untyped) or $0$ (if it has a numeric type).

\clausex{reducing}{~expr~ |by| ~func~ \opt\ |initial-value| ~init-val~ |into| ~var~}
This is a general way to perform reductions. ~func~ should be a
function of two arguments, the first of which will be the value
computed so far and
the second of which will be the value of ~expr~.  It should return the
new value.
|reducing| is roughly equivalent to the
Common Lisp |(reduce ~func~ ~sequence~ :key ~expr-function~)|, where
~expr-function~ is used to derive values from the successive elements
of ~sequence~.

\cpar If the |reducing| clause is never executed, the result is
undefined.

\cpar It is not necessary to provide an
initial value, but better code can be generated if one is supplied.
Regardless of its location in the \iter\ body, ~init-val~ will be
evaluated before the loop is entered, so it should not depend on any
value computed inside the \iter\ form.  

%\cpar if a ~var~ is not specified, you can get \iter\ to declare the
%type of the internal variable by putting a |the| expression around
%~func.~  see section \ref{types}.

\end{clauses}

%\begin{note}{implementation note}
%in principle, |maximize| and |minimize| can be thought of
%as reductions where the initial value is the smallest (or largest)
%value that the accumulation variable can assume.  because lisp's
%bignums can represent arbitrary integers, these clauses cannot be
%implemented as reductions in general.  if, however, the type of 
%~expr~ or ~var~ can be determined to be a fixnum
%or a float, \iter\ will implement the clause as a true reduction,
%using one of the constants |most-negative-fixnum|, |%most-positive-fixnum|, 
%|most-negative-short-float|, etc. as appropriate.
%\end{note}

\subsubsection{Accumulations}

All the predefined accumulation clauses add values to a sequence.  If
the sequence is a list, they all 
have the property that the partial list is kept in the correct order
and available for inspection at any point in the loop.

\begin{clauses}

\clausex{collect}{~expr~ \opt\ |into| ~var~ |at| ~place~ |result-type| ~type~}
Produces a sequence of the values of ~expr~ on each iteration. ~place~
indicates where the next value of ~expr~ is added to the list and may
be one of the symbols |start|, |beginning| (a synonym for |start|) or
|end|.  The symbol may be quoted, but need not be.  The default is
|end|.  For example,
\begin{program}
(iter (for i from 1 to 5)
      (collect i))
\end{program}
produces |(1 2 3 4 5)|, whereas
\begin{program}
(iter (for i from 1 to 5)
      (collect i at beginning))
\end{program}
produces |(5 4 3 2 1)| (and is likely to be faster in most Common Lisp
implementations).

\cpar If ~type~ is provided, it should be a subtype of |sequence|.
The default is |list|.  Specifying a type other than |list| will
result in |collect| returning a sequence of that type.  ~However,~ the
type of the sequence being constructed when inside the loop body is
undefined when a non-|list| type is specified.  (As with ~place~,
quoting ~type~ is optional.)

\clausex{adjoining}{~expr~ \opt\ |into| ~var~ |test| ~test~ |at| ~place~ |result-type| ~type~}
Like |collect|, but only adds the value of ~expr~ if it is not
already present.  ~test,~ which defaults to |\#'eql|, is
the test to be used with |member|.

\startitem
\Clausex{appending}{~expr~ \opt\ |into| ~var~ |at| ~place~}
\Clausex{nconcing}{\ ~expr~ \opt\ |into| ~var~ |at| ~place~}
\Clausex{unioning}{\ ~expr~ \opt\ |into| ~var~ |test| ~test~ |at| ~place~}
\Clausex{nunioning}{~expr~ \opt\ |into| ~var~ |test| ~test~ |at| ~place~}
\finishitem
These are like |collect|, but behave like the Common Lisp functions
|append|, |nconc|, |union| or |nunion|.  
As in Common Lisp, they work only on lists.  Also as in Common Lisp,
|unioning| and |nunioning| assume that the value of ~expr~ contains no
duplicates. 

\clausex{accumulate}{~expr~ |by| ~func~ \opt\ |initial-value| ~init-val~ |into| ~var~}
This is a general-purpose accumulation clause. ~func~ should be a
function of two arguments, the value of ~expr~ and the value
accumulated so far in the iteration, and it should return the updated
value.  If no initial value is supplied, \nil\ is used.  

%\cpar If a ~var~ is not specified, you can get \iter\ to declare the
%type of the internal variable by putting a |the| expression around
%~func.~  see section \ref{types}.

\cpar The differences between |accumulate| and |reducing| are slight.
One difference is that the functions take their arguments in a
different order.  Another is that in the absence of ~init-val~,
|accumulate| will use \nil, whereas |reducing| will generate different
code that avoids any dependence on the initial value.
The reason for having both clauses is that one usually
thinks of reductions (like |sum|) and accumulations (like |collect|)
as different beasts.

\end{clauses}


\subsubsection{Finders}

A ~finder~ is a clause whose value is an expression that meets some
condition.

\begin{clauses}

\clause{|finding| ~expr~ |such-that| ~test~ \opt\ |into| ~var~ |on-failure| ~failure-value~}
\clindexx{finding}{such-that}
If ~test~ (which is an expression) ever evaluates to \nonnil, the loop
is terminated, the 
epilogue code is run and the value of ~expr~ is returned.  Otherwise,
\nil\ (or ~failure-value,~ if provided) is returned.  If ~var~ is
provided, it will have either the \nonnil\ value of ~expr~ or
~failure-value~ when the epilogue code is run.

\cpar As a special case, if the ~test~ expression is a sharp-quoted
function, then it is applied to ~expr~ instead of being simply
evaluated.  E.g. |(finding x such-that \#'evenp)| is equivalent to
|(finding x such-that (evenp x))|.  

%\cpar although ~test~ need have nothing to do with ~%expr~ as in
%|(finding j such-that (> i 3))|, it usually 
%will: |(finding (length el) such-that (oddp (length el)))|.  to
%avoid performing the |length| computation twice, you could write
%|(finding (length el) such-that \#'oddp)| or |(finding (length
%el) such-that 'oddp)|; for these cases, \iter\ generates code that
%executes ~expr~ only once.  the code for |\#'oddp|
%is slightly different from that for {\lisp 'oddp}; see the discussion
%under {\lisp for\dots in} and {\lisp for\dots on}.

\cpar |On-failure| is a misnomer. Because it is always evaluated, it behaves
more like the default third argument to the |gethash| function. As a result,
|on-failure (error "Not found")| makes no sense. Instead, the clauses |leave|
or |thereis| can be used in conjunction with |finally| as follows:
\begin{program}
(iter (for x in '(1 2 3))
      (if (evenp x) (leave x))
      (finally (error "not found")))
\end{program}

\cpar This clause may appear multiple times when all defaults are
identical. It can also be used together with either |always|/|never| or
|thereis| if their defaults match. More specifically, |on-failure nil| is
compatible with |thereis|, while |on-failure t| is compatible with |always|
and |never| clauses.
\begin{program}
(iter (for i in '(7 -4 2 -3))
      (if (plusp i)
	  (finding    i  such-that (evenp i))
          (finding (- i) such-that (oddp i))))
\end{program}

\startitem
\Clause{|finding| ~expr~ |maximizing| ~m-expr~ \opt\ |into| ~var~}
\clindexx{finding}{maximizing}
\Clause{|finding| ~expr~ |minimizing| ~m-expr~ \opt\ |into| ~var~}
\clindexx{finding}{minimizing}
\finishitem
Computes the extremum (maximum or minimum) value of ~m-expr~ over all
iterations, and returns the value of ~expr~ corresponding to the
extremum.  ~expr~ is evaluated inside the loop at the time the new
extremum is established.  If ~m-expr~ is never evaluated (due to, for
example, being embedded in a conditional clause), then the returned
value depends on the type, if any, of ~expr~ (or ~var,~ if
one is supplied).  If there is no type, the returned
value will be \nil; if the type is numeric, the returned value will be
zero.

\cpar For these two clauses, ~var~ may be a list of two
symbols; in that case, the first is used to record ~expr~ and
the second, ~m-expr.~  

\cpar As with |finding\dots such-that|, if ~m-expr~ is a sharp-quoted
function, then it is called on ~expr~ instead of being evaluated.

\end{clauses}

\subsubsection{Boolean Tests}

\begin{clauses}

\clausex{first-iteration-p}{}
Returns |t| in the first cycle of the loop, otherwise \nil.

\clausex{first-time-p}{}
Returns |t| the first time the expression is evaluated, and then \nil\ forever.
This clause comes handy when printing (optional) elements separated
by a comma:

\begin{program}
(iter (for el in '(nil 1 2 nil 3))
      (when el
        (unless (first-time-p)
          (princ ", "))
        (princ el)))
\end{program}
produces |"1, 2, 3"|.

\end{clauses}

\subsubsection{Aggregated Boolean Tests}

\begin{clauses}

\clausex{always}{~expr~}
If ~expr~ ever evaluates to
\nil, then \nil\ is immediately returned; the epilogue code is not
executed.  If ~expr~ never evaluates to \nil, the epilogue code
is executed and the last value of ~expr~ (or |t| if ~expr~ was never
evaluated) is returned (whereas |loop| would constantly return |t|).

% mention last evaluated clause when multiple always clauses?

\clausex{never}{~expr~}
Like |(always (not ~expr~))|, except it does not influence the last
value returned by a possible other |always| clause. That is,
\begin{program}
(iter (repeat 2)
      (always 2)
      (never nil)) \yields 2 ; not t
\end{program}

\clausex{thereis}{~expr~}
If ~expr~ is ever \nonnil,
its value is immediately returned without running epilogue code.
Otherwise, the epilogue code is performed and \nil\ is returned.

This clause cannot be used together with |always| or |never|, because their
defaults are opposed (similarly, |(loop always 3 thereis nil)| refuses to
compile in some implementations of |loop|).

\end{clauses}


\subsection{Control Flow}
\label{control-flow}
Several clauses can be used to alter the usual flow of control in a loop.

Note: the clauses of this and subsequent sections don't adhere to \iter's
usual syntax, but instead use standard Common Lisp syntax.  Hence the
format for describing syntax subsequently is
like the standard format used in the Common Lisp manual, not like the
descriptions of clauses above.

\begin{clauses}

\clausex{finish}{}
Stops the loop and runs the epilogue code.

%for example:
%
%\begin{program}
%(iter (with answer = nil)
%         (initially (make-a-mess))
%         (for i from 1 to 10)
%         (when (correct? i)
%           (setq answer i)
%           (finish))
%         (finally (cleanup)))
%\end{program}
%
%this code will execute |cleanup| whether or not the test |(correct?
%i)| ever succeeds. 
%the (more elegant) formulation,
%\begin{program}
%(iter (initially (make-a-mess))
%         (for i from 1 to 10)
%         (finding i such-that (correct? i))
%         (finally (cleanup)))
%\end{program}
%would not execute |cleanup| if |(correct? i)| succeeded; it
%would do an immediate return.

\clausex{leave}{\opt\ ~value~}
Immediately returns ~value~ (default \nil) from the current \iter\
form, skipping the epilogue code.  Equivalent to using |return-from|.

\clausex{next-iteration}{}
Skips the remainder of the loop body and begins the next iteration of
the loop.

\clausex{while}{~expr~}
If ~expr~ ever evaluates to \nil, the loop is terminated and the
epilogue code executed.  Equivalent to |(if (not ~expr~) (finish))|.

\clausex{until}{~expr~}
Equivalent to |(if ~expr~ (finish))|.

\clausex{if-first-time}{~then~ \opt\ ~else~}
If this clause is being executed for the first time in this invocation
of the \iter\ form, then the ~then~ code is evaluated; otherwise the
~else~ code is evaluated.

\cpar |(for ~var~ first ~expr1~ then ~expr2~)| is almost equivalent to
\begin{program}
(if-first-time (dsetq ~var~ ~expr1~)
               (dsetq ~var~ ~expr2~))
\end{program}
The only difference is that the |for| version makes ~var~ available
for use with |for\dots previous|.

\end{clauses}

\subsection{Code Placement}
When fine control is desired over where code appears in a loop
generated by \iter, the following special clauses may be useful.
They are all subject to code-motion problems (see section
\ref{code-movement}).  

\begin{clauses}

\clausex{initially}{|\&rest| ~forms~}
The lisp ~forms~ are placed in the prologue section of the loop, where
they are executed once, before the loop body is entered.

\clausex{after-each}{|\&rest| ~forms~}
The ~forms~ are placed at the end of the loop body, where they
are executed after each iteration.  Unlike the other clauses in this
section, ~forms~ may contain \iter\ clauses.

\clausex{else}{|\&rest| ~forms~}
The lisp ~forms~ are placed in the epilogue section of the loop, where they
are executed if this |else| clause is never met during execution of the
loop and the loop terminates normally.

\clausex{finally}{|\&rest| ~forms~}
The lisp ~forms~ are placed in the epilogue section of the loop, where
they are executed after the loop has terminated normally.

\clausex{finally-protected}{|\&rest| ~forms~}
The lisp ~forms~ are placed in the second form of an unwind-protect
outside the loop.  They are always executed after the loop has
terminated, regardless of how the termination occurred.

\end{clauses}


%%%%%%%%%%%%%%%%%%%%%%%%%%%%%%%%%%%%%%%%%%%%%%%%%%%%%%%%%%%%%%%%
\section{Other Features}

\subsection{Multiple Accumulations}
\label{multiple}

\begin{sloppypar}
It is permitted to have more than one clause accumulate into the same
variable, as in the following:
\begin{program}
(iter (for i from 1 to 10)
      (collect i into nums)
      (collect (sqrt i) into nums)
      (finally (return nums)))
\end{program}
Clauses can only accumulate into the same variable if they are
compatible.  |collect|, |adjoining|, |appending|, |nconcing|,
|unioning| and |nunioning| are compatible with each other; |sum|,
|multiply| and |counting| are compatible; |always| and |never| are compatible;
|finding| \dots |such-that| is compatible with either |thereis| or |always|
and |never| when their defaults match; and |maximize| and |minimize| clauses
are compatible only with other |maximize| and |minimize| clauses,
respectively.

%note that the same variable ~cannot~ be both an accumulation
%variable and an ordinary variable; there can be only one variable with
%a given name within an \iter\ form. 
\end{sloppypar}

\subsection{Named Blocks}

Like Common Lisp |block|s, \iter\ forms can be given names.  The
name should be a single symbol, and it must be the first form in the
\iter.  The generated code behaves exactly like a named block; in
particular, |(return-from ~name~)| can be used to exit it:
\begin{program}
(iter fred
      (for i from 1 to 10)
      (iter barney
            (for j from i to 10)
            (if (> (* i j) 17)
                (return-from fred j))))
\end{program}
An \iter\ form that is not given a name is implicitly named \nil.

Sometimes one would like to write an expression in an inner \iter\ form,
but have it processed by an outer \iter\ form.  This is possible with
the |in| clause.  

\begin{clauses}

\clausex{in}{~name~ |\&rest| ~forms~}
Evaluates ~forms~ as if they were part of the \iter\ form named
~name~.  In other words, \iter\ clauses are processed by the \iter\
form named ~name,~ and not by any \iter\ forms that occur inside ~name.~

\cpar As an example, consider the problem of collecting a list of the
elements in a two-dimensional array.  The naive solution,
\begin{program}
(iter (for i below (array-dimension ar 0))
      (iter (for j below (array-dimension ar 1))
            (collect (aref ar i j))))
\end{program}
\noindent is wrong because the list created by the inner \iter\ is simply
ignored by the outer one.  But using |in| we can write:
\begin{program}
(iter outer (for i below (array-dimension ar 0))
      (iter (for j below (array-dimension ar 1))
            (in outer (collect (aref ar i j)))))
\end{program}
\noindent which has the desired result.

\end{clauses}

\subsection{Destructuring}
\label{destructuring}

In many places within \iter\ clauses where a variable is expected, a
list can be written instead.  In these cases, the value to be assigned
is ~destructured~ according to the pattern
described by the list.  As a simple example, the clause
\begin{program}
(for (key . item) in alist)
\end{program}
\noindent will result in |key| being set to the |car| of
each element in |alist|, and |item| being set to the |cdr|.  The
pattern list may be nested to arbitrary depth, and (as the example
shows) need not be terminated with \nil; the only requirement is that
each leaf be a bindable symbol (or \nil, in which case no binding is
generated for that piece of the structure).

Sometimes, you might like to do the equivalent of a
|multiple-value-setq| in a clause.  This 
``multiple-value destructuring'' can be expressed by writing 
\linebreak |(values $pat_1$ $pat_2 \ldots$)| for a destructuring
pattern, as in 
\begin{program}
(for (values (a . b) c d) = (three-valued-function ...))
\end{program}
\begin{sloppypar}
Note that the $pat_i$ can themselves be destructuring patterns (though
not multiple-value destructuring patterns).  You can't do multiple-value
destructuring in a |with| clause; instead wrap the whole \iter\
form in a |multiple-value-bind|.
\end{sloppypar}

\begin{note}{Rationale}
There are subtle interactions between variable declarations and
evaluation order that make the correct implementation of
multiple-value destructuring in a |with| somewhat tricky.
\end{note}

The destructuring feature of \iter\ is available as a separate
mechanism, using the |dsetq| macro:

\begin{clauses}
\defmacro{dsetq}{template expr}
Performs destructuring of ~expr~ using ~template~.   May be used
outside of an \iter\ form. Yields the primary value of ~expr~.

\end{clauses}

\subsection{On-line Help}

\begin{sloppypar}
There is a limited facility for on-line help, in the form of the 
|display-iterate-clauses| function.
\end{sloppypar}

\begin{clauses}

\defun{display-iterate-clauses}{\opt\ clause-spec}
Displays a list of \iter\ clauses.  If ~clause-spec~ is not
provided, all clauses are shown; if it is a symbol, all clauses
beginning with that symbol are shown; and if it is a list of symbols,
all clauses for which ~clause-spec~ is a prefix are shown.

\end{clauses}

\subsection{Parallel Binding and Stepping}
\label{bindings}

The parallel binding and stepping of variables is a feature that
\iter\ does ~not~ have.  This section attempts to provide a rationale.

We say that two variables are bound ~in parallel~ if neither
binding shadows the other.  This is the usual semantics of |let|
(as opposed to |let*|).  Similarly, we can say that iteration
variables are stepped in parallel if neither variable is updated
before the other, conceptually speaking; in other words, if the code
to update each variable can reference the old values of both variables.

|loop| allows parallel binding of variables and parallel stepping of
driver variables.  My view is that if you are depending on the 
serial/parallel distinction, you are doing something obscure.  If you
need to bind 
variables in parallel using |with|, then you must be using a
variable name that shadows a name in the existing lexical environment.
Don't do that.  The most common use for parallel stepping is to track
the values of variables on the previous iteration, but in fact this
does not require parallel stepping at all; the following will work:
\begin{program}
(iter (for current in list)
      (for prev previous current)
      ...)
\end{program}


%%%%%%%%%%%%%%%%%%%%%%%%%%%%%%%%%%%%%%%%%%%%%%%%%%%%%%%%%%%%%%%%
\section{Types and Declarations}
\label{types}

\subsection{Discussion}

Sometimes efficiency dictates that the types of variables be declared.
This type information needs to be communicated to \iter\ so it can
bind variables to appropriate values.  Furthermore, \iter\ must often
generate internal variables invisible to the user; there needs to be a
way for these to be declared.

As an example, consider this code, which will return the number of
odd elements in |number-list|:
\begin{program}
(iter (for el in number-list)
      (count (oddp el)))
\end{program}
In processing this form,
\iter\ will create an internal variable, let us call it |list17|, to
hold the successive |cdr|s of |number-list|, and  
will bind the variable to |number-list|.  It will also generate a
default binding for |el|; only inside the body of the loop will |el|
be set to the |car| of |list17|.  Finally, \iter\ will generate a
variable, call it |result|, to hold the result of the count, and will
bind it to zero.

When dealing with type declarations, \iter\ observes one simple rule:
~it will never generate a declaration unless requested to do so.~  The
reason is that such declarations might mask errors in compiled code by
avoiding error-checks; the resulting problems would be doubly hard to
track down because the declarations would be hidden from the
programmer.  Of course, a compiler might omit error-checks even in the
absence of declarations, though this behavior can usually be avoided,
e.g. by saying |(declaim (optimize (safety 3)))|.

So, the above \iter\ form will generate code with no declarations.
But say we wish to declare the types of |el| and the internal
variables |list17| and |result|.  How is this done?

Declaring the type of |el| is easy, since the programmer knows
the variable's name:
\begin{program}
(iter (for el in number-list)
      (declare (fixnum el))
      (counting (oddp el)))
\end{program}
\iter\ can read variable type declarations like this one.  Before
processing any clauses, it scans the entire top-level form for type
declarations 
and records the types, so that variable bindings can be performed
correctly.  In this case, |el| will be bound to zero
instead of \nil.  Also, \iter\ collects all the top-level declarations
and puts them at the begining of the generated code, so it is not
necessary to place all declarations at the beginning of an \iter\
form; instead, they can be written near the variables whose types they
declare.  

Since \iter\ is not part of the compiler, it will not know
about declarations that occur outside an \iter\ form; these
declarations must be repeated inside the form.

Here is another way we could have declared the type of |el|:
\begin{program}
(iter (for (the fixnum el) in number-list)
      (counting (oddp el)))
\end{program}
\iter\ extends the Common Lisp |the|\lispindex{the}
form to apply to variables as well as value-producing forms; anywhere
a variable is allowed---in a |with| clause, as the iteration
variable in a driver clause, as the |into| argument of an
accumulation clause, even inside a destructuring template---you can
write |(the ~type~ ~symbol~)| instead.

There is one crucial difference between using a |the| form and
actually declaring the variable: explicit declarations are always
placed in the generated code, but type information from a |the|
form is not turned into an actual declaration unless you tell \iter\
to do so using |iterate:declare-variables|.  See below.

\begin{sloppypar}
Declaring the types of internal variables is harder than declaring the
types of explicitly mentioned variables, since their names
are unknown.  You do it by declaring |iterate:declare-variables|
somewhere inside the top level of the \iter\ form.  (This will also
generate declarations for variables declared using |the|.)
\iter\ does not provide much selectivity here: it's all or none. 
And unfortunately, since \iter\ is not privy to compiler information
but instead reads declarations itself, it will not hear if you
|(declaim (iterate:declare-variables))|.  Instead, set the variable 
|iterate::*always-declare-variables*| to |t| at
compile-time, using |eval-when|.
\end{sloppypar}

To determine the appropriate types for internal variables, \iter\ uses
three sources of information:

\begin{itemize}

\item Often, the particular clause dictates a certain type for a
variable; \iter\ will use this information when available.  In the
current example, the variable |list17| will be given the type
|list|, since that is the only type that makes sense; and the
variable |result| will be given the type |fixnum|, on the
assumption that you will not be counting high enough to need bignums.
You can override this assumption only by using and explicitly declaring a
variable: 
\begin{verbatim}
(iter (declare (iterate:declare-variables))
      (for el in number-list)
      (count (oddp el) into my-result)
      (declare (integer my-result))
      (finally (return my-result)))
\end{verbatim}

\begin{sloppypar}
Other examples of the type assumptions that \iter\ makes are: type
|list| for |into| variables of collection clauses; type |list| for
expressions that are to be destructured; type |vector| for the
variable holding the vector in a |for\dots in-vector| clause, and
similarly for |string| and the |for\dots in-string| clause;
and the implementation-dependent type
|(type-of array-dimension-limit)| for the index and limit
variables generated by sequence iteration drivers like |for\dots
in-vector| and |for\dots in-string| (but not |for\dots in-sequence|,
because it may be used to iterate over a list). 
\end{sloppypar}

\item Sometimes, \iter\ will examine expressions and try to determine
their types in a simple-minded way.  If the expression is
self-evaluating (like a number, for instance), \iter\ knows that the
expression's type is the same as the type of the value it denotes, so
it can use that type.  If the expression is of the form |(the ~type~
~expr~)|, \iter\ is smart enough to extract ~type~ and use it.
However, the current version of \iter\ does not  
examine declarations of function result types or do any type
inference.  It will not determine, for 
example, that the type of |(+ 3 4)| is |fixnum|, or even
|number|.

\item In some cases, the type of an internal variable should match the
type of some other variable.  For instance, \iter\ generates an
internal variable for |(f x)| in the
clause |(for i from 1 to (f x))|, and in the absence of other
information will give it the same type as |i|.  If, however, the
expression had been written |(the fixnum (f x))|, then \iter\
would have given the internal variable the type |fixnum|
regardless of |i|'s type.  The type incompatibility errors that
could arise in this situation are not checked for.

\end{itemize}

Note that if you do declare |iterate:declare-variables|, then
\iter\ may declare user variables as well as internal ones if they do
not already have declarations, though only for variables that it
binds.  For instance, in this code:

\begin{program}
(iter (declare (iterate:declare-variables))
      (for i from 1 to 10)
      (collect i into var))
\end{program}
the variable |var| will be declared to be of type |list|.


\subsection{Summary}

\iter\ understands standard Common Lisp variable type declarations
that occur within an \iter\ form and
will pass them through to the generated code.  If the declaration
|(iterate:declare-variables)|\lispindex{declare-variables}
appears at the top level of an 
\iter\ form, or if 
|iterate::*always-declare-variables*|\lispindex{*always-declare-variables*} 
is \nonnil, then \iter\ will use the type information gleaned from user
declarations, self-evaluating expressions and |the| expressions,
combined with reasonable assumptions, to determine variable
types and declare them.


%%%%%%%%%%%%%%%%%%%%%%%%%%%%%%%%%%%%%%%%%%%%%%%%%%%%%%%%%%%%%%%%
\section{Problems with Code Movement}
\label{code-movement}

Some \iter\ clauses, or parts of clauses, result in code being
moved from the location of the clause to other parts of the loop.
Drivers behave this way, as do code-placement clauses like |initially|
and |finally|.  When using these clauses, there is a danger of writing
an expression that makes sense in its apparent location but will be
invalid or have a different meaning in another location.  For example:
\begin{program}
(iter (for i from 1 to 10)
      (let ((x 3))
        (initially (setq x 4))))
\end{program}
While it may appear that the |x| of |(initially (setq x 4))| is the
same as the |x| of |(let ((x 3)) \dots|, in fact they are not:
|initially| moves its code outside the loop body, so |x| would refer
to a global variable.  Here is another example of the same problem:
\begin{program}
(iter (for i from 1 to 10)
      (let ((x 3))
        (collect i into x)))
\end{program}
If this code were executed, |collect| would create a binding for its
|x| at the top level of the \iter\ form that the |let| will shadow.

Happily, \iter\ is smart enough to catch these errors; it
walks all problematical code to ensure that free variables are not
bound inside the loop body, and checks all variables it binds for the
same problem.

However, some errors cannot be caught:

\begin{program}
(iter (with x = 3)
      (for el in list)
      (setq x 1)
      (reducing el by \#'+ initial-value x))
\end{program}
|reducing| moves its |initial-value| argument to the initialization
part of the loop in order to produce more efficient code.  Since
\iter\ does not perform data-flow analysis, it cannot determine that
|x| is changed inside the loop; all it can establish is that |x| is
not bound internally.  Hence this code will not signal an
error and will use $3$ as the initial value of the reduction.

The following list summarizes all cases that are subject to these code
motion and variable-shadowing problems.
\begin{itemize}
\item Any variable for which \iter\ creates a binding, including those
used in |with| and the |into| keyword of many clauses.

\begin{sloppypar}
\item The special clauses which place code: |initially|, |after-each|, |else|,
|finally| and |finally-protected|. 
\end{sloppypar}

\item The variables of a |next| or |do-next| form.

\item The |initially| arguments of |for\dots initially\dots then| and
|for\dots previous|. 

\item The |then| argument of |for\dots initially\dots then|.

\item The |initial-value| arguments of |reducing| and |accumulate|.

\item The |on-failure| argument of |finding\dots such-that|.
\end{itemize}

%%%%%%%%%%%%%%%%%%%%%%%%%%%%%%%%%%%%%%%%%%%%%%%%%%%%%%%%%%%%%%%%
\section{Differences Between {\tt Iterate} and {\tt Loop}}
|loop| contains a great deal of complexity which \iter\ tries to
avoid.  Hence many esoteric features of |loop| don't exist in \iter.
Other features have been carried over, but in a cleaned-up form.  
And of course, many new features have been added; they are not
mentioned in this list.

\begin{itemize}

\item \iter's syntax is more Lisp-like than |loop|'s, having a higher
density of parens.

\item The current implementation of \iter, unlike the current version
of  |loop| (as documented in {\em Common Lisp, 2nd Ed.\/}), is
extensible (see section \ref{extend}).  

\item |loop| puts the updates of all driver variables at the top of
the loop; \iter\ leaves them where the driver clauses appear.

\item While for the most part \iter\ clauses that resemble |loop| clauses
behave similarly, there are some differences.  For instance, there is
no |for\dots =\dots then| in \iter; instead use 
|for\dots initially\dots then|.  

\item |loop| binds the variable |it| at certain times
to allow pseudo-English expressions like |when ~expr~ return it|. 
In \iter, you must bind ~expr~ to a variable yourself.  Note that
|when ~expr~ return it| is like |thereis ~expr~| except that the latter is an
accumulation clause and therefore competes with other accumulations
(remember section \ref{multiple} above).

% repeat different behaviour of |always| clause here?

\item |loop| has a special |return| clause, illustrated in the
previous item.  \iter\ doesn't need one,  since an ordinary Lisp
|return| has the same effect.

\item |loop| allows for parallel binding and stepping of iteration
variables.  \iter\ does not.  (See section \ref{bindings}.)

\item |loop| and \iter\ handle variable type declarations very
differently.  |loop| provides a special syntax for declaring variable
types, and does not examine declarations.  Moreover, the standard
implementation of |loop| will
generate declarations when none are requested.
\iter\ parses standard Common Lisp type declarations, and will never
declare a variable itself unless declarations are 
specifically requested.

\end{itemize}

%%%%%%%%%%%%%%%%%%%%%%%%%%%%%%%%%%%%%%%%%%%%%%%%%%%%%%%%%%%%%%%%
\section{Rolling Your Own}
\label{extend}

\subsection{Introduction}

\iter\ is extensible---you can write new clauses that embody new
iteration patterns.  You might want to write a new driver clause for a
data structure of your own, or you might want to write a clause that
collects or manipulates elements in a way not provided by \iter. 

This section describes how to write clauses for \iter.  Writing a
clause is like writing a macro.  In fact, writing a clause ~is~
writing a macro: since \iter\ code-walks its body and macroexpands,
you can add new abstractions to \iter\ with good old |defmacro|.

Actually, there are two extensions you can make to \iter\ that are
even easier than writing a macro.  They are adding a synonym for an
existing clause and defining a driver clause for an indexable
sequence.  These can be done with |defsynonym| and
|defclause-sequence|, respectively.  See section \ref{aids}, below.

The rest of this section explains how to write macros that expand into
\iter\ clauses.
Here's how you could add a simplified version of \iter's |multiply|
clause, if \iter\ didn't already have one:
\begin{program}
(defmacro multiply (expr)
  `(reducing ,expr by \#'* initial-value 1))
\end{program}

If you found yourself summing the square of an expression often, you
might want to write a macro for that.  A first cut might be
\begin{program}
(defmacro sum-of-squares (expr)
  `(sum (* ,expr ,expr)))
\end{program}
but if you are an experienced macro writer, you will realize that this
code will evaluate ~expr~ twice, which is probably a bad idea.  A
better version would use a temporary:
\begin{program}
(defmacro sum-of-squares (expr)
  (let ((temp (gensym)))
    `(let ((,temp ,expr))
       (sum (* ,temp ,temp)))))
\end{program}
Although this may seem complex, it is just the sort of thing you'd
have to go through to write any macro, which illustrates the point of
this section: if you can write macros, you can extend \iter.

Our macros don't use \iter's keyword-argument
syntax.  We could just use keywords with |defmacro|, but we would
still not be using \iter's clause indexing mechanism.  Unlike Lisp,
which uses just the first symbol of a form to determine what function
to call, \iter\ individuates clauses by the list of required keywords.
For instance, |for\dots in| and |for\dots in-vector| are different clauses
implemented by distinct Lisp functions.  

To buy into this indexing scheme, as well as the keyword-argument
syntax, use |defmacro-clause|:

\begin{clauses}

\defmacro{defmacro-clause}{arglist |\&body| body}
Defines a new \iter\ clause.  ~arglist~ is a list of symbols which are
alternating keywords and arguments.  \opt\ may be used, and the list
may be terminated by |\&sequence|.  ~body~ is an ordinary macro body,
as with |defmacro|.  If the first form of ~body~ is a string, it is
considered a documentation string and will be shown by
|display-iterate-clauses|. |defmacro-clause| will signal an error if 
defining the clause would result in an ambiguity.  E.g. you cannot
define the clause |for\dots from| because there would be no way to
distinguish it from a use of the |for| clause with optional keyword |from|.

\end{clauses}

\medskip

Here is |multiply| using |defmacro-clause|.  The keywords are capitalized
for readability.
\begin{program}
(defmacro-clause (MULTIPLY expr \opt\ INTO var)
  `(reducing ,expr by \#'* into ,var initial-value 1))
\end{program}
You don't have to worry about the case when |var| is not supplied; for
any clause with an |into| keyword, saying |into nil| is equivalent to
omitting the |into| entirely.

As another, more extended example, consider the fairly common
iteration pattern that 
involves finding the sequence element that maximizes (or minimizes) some
function.  \iter\ provides this as |finding\dots maximizing|, but it's
instructive to see how to write it.
Here, in pseudocode, is how you might write such a loop for
maximizing a function F:

\begin{quote}
\begin{tabbing}
set variable MAX-VAL to NIL; \\
set variable WINNER to NIL; \\
for\=\ each element EL in the sequence \\
\>  if\=\ MAX-VAL\=\ is NIL or F(EL) $>$ MAX-VAL then \\
\>\>    set MAX-VAL to F(EL); \\
\>\>    set WINNER to EL; \\
\>  end if; \\
end for; \\
return WINNER.
\end{tabbing}
\end{quote}

Here is the macro:
\begin{program}
(defmacro-clause (FINDING expr MAXIMIZING func \opt\ INTO var)
  (let ((max-val (gensym))
        (temp1 (gensym))
        (temp2 (gensym))
        (winner (or var iterate::*result-var*)))
    `(progn 
       (with ,max-val = nil)
       (with ,winner = nil)
       (cond 
        ((null ,max-val)
         (setq ,winner ,expr)
         (setq ,max-val (funcall ,func ,winner))
        (t
         (let* ((,temp1 ,expr)
                (,temp2 (funcall ,func ,temp1)))
           (when (> ,temp2 ,max-val)
             (setq ,max-val ,temp2)
             (setq ,winner ,temp1))))))
       (finally (leave ,winner)))))
\end{program}
Note that if no |into| variable is supplied, we use
|iterate::*result-var*|, which contains the internal variable into
which all clauses place their results.  If this variable is bound by
some clause, then \iter\ will return its value automatically;
otherwise, \nil\ will be returned.  

\subsection{Writing Drivers}

In principle, drivers can be implemented just as easily as other
\iter\ clauses.  In practice, they are a little harder to get right. 
As an example, consider writing a driver that
iterates over all the 
elements of a vector, ignoring its fill-pointer.  |for\dots in-vector|
won't work for this, because it observes the fill-pointer.  It's necessary to
use |array-dimension| instead of |length| to obtain the size of the
vector.  Here is one approach:

\begin{program}
(defmacro-clause (FOR var IN-WHOLE-VECTOR v)
  "All the elements of a vector (disregards fill-pointer)"
  (let ((vect (gensym))
        (index (gensym)))
    `(progn
       (with ,vect = ,v)
       (for ,index from 0 below (array-dimension ,vect 0))
       (for ,var = (aref ,vect ,index)))))
\end{program}
Note that we immediately put |v| in a variable, in case it is an
expression.  Again, this is just good Lisp macrology.  It also has a
subtle effect on the semantics of the driver: |v| is evaluated only
once, at the beginning of the loop, so changes to |v| in the loop have
no effect on the driver.  Similarly, the bounds for numerical iteration
e.g. the above |array-dimension| are also evaluated once only.  This is how
all of \iter's drivers work.


There is an important point concerning the |progn| in this code.  We
need the |progn|, of course, because we are returning several forms,
one of which is a driver.  But \iter\ drivers must occur at top-level.
Is this code in error?  No, because ~top-level~ is defined in \iter\
to include forms inside a |progn|.  This is just the definition of
top-level that Common Lisp uses, and for the same reason: to allow
macros to return multiple forms at top-level.

While our |for\dots in-whole-vector| clause will work, it is not
ideal.  In particular, it does not support generating.  Do do so, we
need to use |for\dots next| or |for\dots do-next|.  The job is
simplified by the |defmacro-driver| macro.

\begin{clauses}

\defmacro{defmacro-driver}{arglist |\&body| body}
Defines a driver clause in
both the |for| and |generate| forms, and provides a parameter
|generate| which ~body~ can examine to determine how it was invoked.
~arglist~ is as in |defmacro-clause|, and should begin with the symbol
|for|. 

\end{clauses}

With |defmacro-driver|, our driver looks like this:
\begin{program}
(defmacro-driver (FOR var IN-WHOLE-VECTOR v)
  "All the elements of a vector (disregards fill-pointer)"
   (let ((vect (gensym))
         (end (gensym))
         (index (gensym))
         (kwd (if generate 'generate 'for)))
     `(progn
        (with ,vect = ,v)
        (with ,end = (array-dimension ,vect 0))
        (with ,index = -1)
        (,kwd ,var next (progn (incf ,index)
                               (if (>= ,index ,end) (terminate))
                               (aref ,vect ,index))))))
\end{program}


We are still missing one thing: the |\&sequence| keywords.  
We can get them easily enough, by writing 
\begin{program}
(defmacro-driver (FOR var IN-WHOLE-VECTOR v \&sequence) 
  ...)
\end{program}
We can now refer to parameters |from|, |to|, |by|, etc. which contain
either the values for the corresponding keyword, or \nil\ if the
keyword was not supplied.  Implementing the right code for these
keywords is cumbersome but not difficult; it is left as an exercise.
But before you begin, see |defclause-sequence| below for an easier way.
       
\subsection{Extensibility Aids}
\label{aids}

This section documents assorted features that may be of use in
extending \iter.

\begin{clauses}

\defunexpvar{*result-var*}
Holds the variable that is used to return a value as a result of the
\iter\ form.  You may examine this and use it in a |with| clause, but
you should not change it.

\defmacro{defsynonym}{syn word}
 Makes ~syn~ a synonym for the existing \iter\ keyword ~word.~  Only
the first word in each clause can have synonyms. 

\defmacroxx{defclause-sequence}{element-name index-name}{|\&key|
access-fn size-fn sequence-type}{element-type
element-doc-string index-doc-string}  
Provides
a simple way to define sequence clauses.  Generates two
clauses, one for iterating over the sequence's elements, the other
for iterating over its indices.  The first symbol of both
clauses will have print-name |for|.
~element-name~  and ~index-name~ should be symbols.
~element-name~ is the second keyword of the element iterator (typically of
the form 
|in-~sequence-type~|), and ~index-name~ is the second keyword
of the index-iterator (typically of the form
|index-of-~sequence-type~|).  Either name may be 
\nil, in which case the corresponding clause is not defined.  If both
symbols are supplied, they should be in the same package.  The |for|
that begins the clauses will be in this package.

\cpar ~access-fn~ is the function to be used to
access elements of the sequence in the element iterator.  The function
should take two 
arguments, a sequence and an index, and return the appropriate element.
~size-fn~ should denote a function of one argument, a sequence, that
returns its size.  Both ~access-fn~ and ~size-fn~ are required for the
element iterator, but only ~size-fn~ is needed for the index iterator.

\cpar The ~sequence-type~ and ~element-type~ keywords are used to
suggest types for the variables 
used to hold the sequence and the
sequence elements, respectively.  The usual rules about \iter's
treatment of variable type declarations apply (see section \ref{types}).

\cpar ~element-doc-string~ and ~index-doc-string~ are
the documentation strings, for use with |display-iterate-clauses|. 

\cpar The generated element-iterator performs destructuring on the
element variable.

\cpar As an example, the above |for\dots in-whole-vector| example
could have been written:
\begin{program}
(defclause-sequence IN-WHOLE-VECTOR INDEX-OF-WHOLE-VECTOR
  :access-fn 'aref
  :size-fn \#'(lambda (v) (array-dimension v 0))
  :sequence-type 'vector
  :element-type t
  :element-doc-string 
     "Elements of a vector, disregarding fill-pointer"
  :index-doc-string 
     "Indices of vector, disregarding fill-pointer")
\end{program}

\end{clauses}

\subsection{Subtleties}

There are some subtleties to be aware of when writing \iter\ clauses.
First, the code returned by your macros may be |nconc|'ed into a list,
so you should always returned freshly consed lists, rather than
constants.  Second, \iter\ matches clauses by using |eq| on the first
symbol and |string=| on the subsequent ones, so the package of the
first symbol of a clause is relevant.  All of the clauses in this manual
have their first word in the \iter\ package.  
You can use the package system in the usual way to shadow
\iter\ clauses without replacing them.

%%% say more here, about the badness that only the first word of a
%%% clause is packagey.

\section{Non-portable Extensions to Iterate (Contribs)}
\label{contribs}

Currently, there is only one non-portable extension to iterate in the
distribution: iterate-pg. If you have made an extension that depends
on non-portable features, feel free to send them to |asf@boinkor.net|
for inclusion in the iterate distribution.

\subsection{An SQL query driver for iterate}

The pg package by Eric Marsden (see |http://cliki.net/pg|) provides an
interface to the PostgreSQL database. Using the \iterpg\ extension, it
is possible to handle the results of SQL queries using \iter.

This usage example should give you an idea of how to use it:

\begin{program}
(pg:with-pg-connection (c "somedb" "someuser")
  (iter (for (impl version date) in-relation "select * from version"
                                 on-connection *dbconn*)
        (collect version)))
\end{program}

To use the extension via |ASDF|, simply make your system depend on the
|iterate-pg| system instead of the |iterate| system. To load it
manually, use:

\begin{program}
  (asdf:oos 'asdf:load-op :iterate-pg)
\end{program}

%%%%%%%%%%%%%%%%%%%%%%%%%%%%%%%%%%%%%%%%%%%%%%%%%%%%%%%%%%%%%%%%
\section{Obtaining {\tt Iterate}}

\begin{sloppypar}
The information in this chapter is obsolete but included for
completeness's sake; Currently, the most up-to-date information on
\iter\ can be found at |http://boinkor.net/iterate.html|.
\end{sloppypar}

\begin{sloppypar}
\iter\ currently runs on Lisp Machines, and on
HP's, Sun3's and Sparcstations under Lucid.  
\iter\ source and binaries are available at the MIT AI Lab in the
subdirectories of |/src/local/lisplib/|.  The source file,
|iterate.lisp|, is also available for anonymous FTP in the directory
|/com/fpt/pub/| on the machine |TRIX.AI.MIT.EDU| (Internet number
128.52.32.6).  If you are unable to obtain |iterate| in one of these
ways, send mail to |jba@ai.mit.edu| and I will send you the source
file. 
\end{sloppypar}

\begin{sloppypar}
\iter\ resides in the |iterate| package (nickname |iter|).  Just say
\linebreak |(use-package :iterate)| to make all the necessary symbols
available. 
If a symbol is not exported, it appears in this manual with an
``|iterate::|'' prefix.
\end{sloppypar}

Send bug reports to |bug-iterate@ai.mit.edu|.  The |info-iterate|
mailing list will have notices of changes and problems; to have
yourself added, send mail to |info-iterate-request@ai.mit.edu|.


\medskip

\begin{flushleft}
 \bf Acknowledgements
\end{flushleft}
\smallskip

Richard Waters provided invaluable criticism which spurred me to improve
\iter\ greatly.  As early users, David Clemens, Oren Etzioni and Jeff
Siskind helped ferret out many bugs.

%\begin{theindex}
% The index files must be generated with the genindex program
% or from makeindex iterate-manual.idx which creates iterate-manual.ind.

%\baselineskip\normalbaselineskip
%\advance\baselineskip by -2pt

\input{iterate-manual.ind}

%\underline{Clauses}
%\input{iterate-manual.clindex}
%
%\indexspace
%\underline{Lisp}
%\input{iterate-manual.lispindex}

%\end{theindex}


\end{document}


% arch-tag: "c0f871a6-313c-11d8-abb9-000c76244c24"

%
%\indexspace
%\underline{Lisp}
%% Fix ``.'' spacing in lists
\documentstyle[12pt,aimemo]{article}

\setlength{\textwidth}{6.0in}
\setlength{\oddsidemargin}{0.25in}
\setlength{\evensidemargin}{0.25in}
\setlength{\topmargin}{0pt}
\setlength{\headsep}{24pt}
\setlength{\textheight}{8.5in}

\let\origpar=\par

% somehow, LaTex fucks with things so that the plain Tex \obeyspaces
% doesn't work.  This is the same, except we use an mbox containing
% a space instead of just a space.

\newcommand{\spc}{\mbox{ }}
\newcommand{\tAb}{\mbox{        }}
\def\obspaces{\catcode`\ =\active\catcode`\^^I=\active}
{\obspaces \global\let =\spc\global\let^^I=\tAb}

\newenvironment{program}{\medskip\samepage\tt\obeylines\obspaces}{\medskip}

% use | for \tt
\catcode`\|=13
\newif\ifvbar
\def|{\ifvbar
         \endgroup\vbarfalse
      \else 
         \begingroup\tt\vbartrue 
      \fi}



\newcommand{\lisp}{\tt}
\newcommand{\iter}{|iterate|}
\newcommand{\iterpg}{|iterate-pg|}
\newcommand{\setf}{|setf|}
\newcommand{\nil}{|nil|}
\newcommand{\nonnil}{non-\nil}
\newcommand{\opt}{{\lisp \&optional}}
\newcommand{\yields}{$\Rightarrow$}



\newbox\Dots
\setbox\Dots=\hbox{\small ...}
\renewcommand{\dots}{\copy\Dots}

% Here's a way to define an environment which will treat all
% paragraphs inside it in the same manner.  It's, as near as I can
% figure, the way that Latex does things.  Using \everypar won't work;
% apparently Latex resets that a lot.  What you have to do is: at the
% top of the file do \let\origpar=\par.  In the
% begin part of the environment, do the parshape (or whatever), then say
% \def\par{{\origpar}}.  The extra braces are necessary.   That's all
% you have to do; at the end of the environment things will return to
% normal automatically.  Why does this all work?  I have no idea.
% Another important thing: there should be a blank line (i.e. a \par)
% between the end of the last paragraph and the \end{environment}.

\newlength{\clindent}
\setlength{\clindent}{0.5in}
\newlength{\clparindent}
\setlength{\clparindent}{\parindent}

\newenvironment{clauses}{\advance\linewidth -\parindent
                        \hangindent\clindent\relax
                         \hangafter=0
                          \parindent=0pt
                          \def\par{{\origpar}}}{}

\newcommand{\cpar}{\hspace \clparindent}

\newcommand{\startitem}{\bigskip\pagebreak[3]}
\newcommand{\finishitem}{\smallskip}

\newcommand{\clause}[1]{\startitem\Clause{#1}\finishitem}
\newcommand{\Clause}[1]{\hbox{#1}}
\newcommand{\clindex}[1]{\index{{\lisp #1}}}
\newcommand{\clindexx}[2]{\index{{\lisp #1\dots #2}}}
\newcommand{\clindexxx}[3]{\index{{\lisp #1\dots #2\dots #3}}}
\newcommand{\clausex}[2]{\startitem\Clausex{#1}{#2}\finishitem}
\newcommand{\Clausex}[2]{\Clause{|#1| #2}\clindex{#1}}

\newcommand{\defvar}[1]{\startitem\deff{#1}{}{Variable}\finishitem}
\newcommand{\defunexpvar}[1]{\startitem\deffnoindex{iterate::#1}{}{Variable}
                                        \lispindex{#1}\finishitem}
\newcommand{\defun}[2]{\startitem\deff{#1}{#2}{Function}\finishitem}
\newcommand{\defmacro}[2]{\startitem\deff{#1}{#2}{Macro}\finishitem}
\newcommand{\defunx}[3]{\startitem\deffx{#1}{#2}{#3}{Function}\finishitem}
\newcommand{\defmacrox}[3]{\startitem\deffx{#1}{#2}{#3}{Macro}\finishitem}
\newcommand{\defunxx}[4]{\startitem\deffxx{#1}{#2}{#3}{#4}{Function}\finishitem}
\newcommand{\defmacroxx}[4]{\startitem\deffxx{#1}{#2}{#3}{#4}{Macro}\finishitem}

\newcommand{\deff}[3]{\deffnoindex{#1}{#2}{#3}\lispindex{#1}}
\newcommand{\deffnoindex}[3]{\hbox to \hsize{{\tt #1} {\it #2}\hfill [{\it #3}]}}
\newcommand{\deffx}[4]{\deff{#1}{#2}{#4}\moreargs{#1}{#3}}
\newcommand{\deffxx}[5]{\deffx{#1}{#2}{#3}{#5}\moreargs{#1}{#4}}

\newcommand{\moreargs}[2]{\hbox to \hsize{\phantom{\lisp #1} {\it #2}\hfill}}

\newcommand{\lispindex}[1]{\index{{\lisp #1}}}

 \newenvironment{note}[1]{\pagebreak[2]\bigskip
                          \hrule\smallskip\small
                         {\setlength{\parindent}{0pt}
                           \par{\bf #1:}}}{\smallskip\hrule\bigskip}



\makeindex

%%%%%%%%%%%%%%%%%%%%%%%%%%%%%%%%%%%%%%%%%%%%%%%%%%%%%%%%%%%%%%%%

%\thispagestyle{empty}

%\vspace*{2in}

%\title{The |Iterate| Manual}
%\author{Jonathan Amsterdam}

%\maketitle

%\thispagestyle{empty}

%\begin{center}
%\Large *** DRAFT ***
%\end{center}
%This manual is in beta-test. It will become an AI Memo in a couple of
%months.  I would appreciate any comments or
%corrections.  You can reach me at email address |jba@ai.ai.mit.edu|.

\begin{document}

\def\writtenby{Jonathan Amsterdam}
\def\contractno{This report describes research done at the Artificial
Intelligence Laboratory of the Massachusetts Institute of Technology.
Support for the laboratory's artificial intelligence research is
provided in part by the Advanced Research Projects Agency of the
Department of Defense under Office
of Naval Research contract N00014-85-K-0124.}

\AIMEMO{1236}{\Large The Iterate Manual \\[1in]}
{This is the manual for version 1.4 of \iter, a powerful iteration
macro for Common  
Lisp. \iter\ is similar to {\tt loop} but provides numerous additional
features, is well integrated with Lisp, and is extensible. 
\vfill }


% use ~ for \em
\catcode`\~=13
\newif\ifem
\def~{\ifem
         \/\endgroup\emfalse
      \else 
         \begingroup\em\emtrue 
      \fi}



%\pagebreak
%\setcounter{page}{1}

%\thispagestyle{empty}


\tableofcontents

\pagebreak
\pagestyle{headings}

\section{Introduction}

\begin{sloppypar}
This manual describes
\iter, a powerful iteration facility for Common Lisp.
\iter\ provides abstractions for many common iteration
patterns and allows for the definition of additional patterns.  \iter\ is
a macro that expands into ordinary Lisp at
compile-time, so it is more efficient than higher-order
functions like |map| and |reduce.|  
While it is similar to |loop|, \iter\ offers a more Lisp-like syntax
and enhanced extensibility.
(For a more complete
comparison of \iter\ with other iteration constructs, see MIT AI Lab Working
Paper No. 324, ~Don't Loop, Iterate.~)
\end{sloppypar}

An \iter\ form consists of the symbol |iter|\footnote{You can also use
|iterate|, but |iter| is preferred because it avoids potential conflicts with
possible future additions to Common Lisp, and because it saves
horizontal space when writing code.}
followed by one or more
forms, some of which may be \iter\ 
~clauses.~  Here is a simple example of \iter\ which collects the
numbers from 1 to 10 into a list, and returns the list.  The return
value is shown following the arrow.
\begin{program}
(iter (for i from 1 to 10)
      (collect i))          \yields (1 2 3 4 5 6 7 8 9 10)
\end{program}
This form contains two clauses: a |for| clause that steps the
variable |i| over the integers from 1 to 10, and a |collect|
clause that accumulates its argument into a list.  With a few
exceptions, all \iter\ clauses have the same format: alternating
symbols (called ~keywords~) and expressions (called
~arguments~).  The syntax and terminology are those of Common
Lisp's keyword lambda lists.  One difference is that \iter's keywords
do not have to begin with a colon---though they may, except for the
first symbol of a clause.  So you can also write  |(for i :from
1 :to 10)| if you prefer.

Any Lisp form can appear in the body of an \iter, where it will have
its usual meaning.  \iter\ walks the entire body, expanding macros,
and recognizing clauses at any level.  This example collects all the
odd numbers in a list:
\begin{program}
(iter (for el in list)
      (if (and (numberp el) (oddp el))
          (collect el)))
\end{program}

There are clauses for iterating over numbers, lists, arrays and other
objects, and for
collecting, summing, counting, maximizing and other useful operations.
\iter\ also supports the creation of new variable bindings, stepping
over multiple sequences at once, destructuring, and compiler
declarations of variable types.  The following example illustrates
some of these features: 
\begin{program}
(iter (for (key . item) in alist)
         (for i from 0)
         (declare (fixnum i))
         (collect (cons i key)))
\end{program}

This loop takes the keys of an alist and returns a new alist
associating the keys with their positions in the original list. The
compiler declaration for |i| will appear in the generated code
in the appropriate place.

%%%%%%%%%%%%%%%%%%%%%%%%%%%%%%%%%%%%%%%%%%%%%%%%%%%%%%%%%%%%%%%%
\section{Clauses}

Most of \iter's clauses will be familiar to |loop| programmers.  
(|loop| is an iteration macro that has been incorporated into Common
Lisp.  Seen Guy Steele's {\em Common Lisp, 2nd Edition}.)
In
nearly all cases they behave the same as their |loop| counterparts, so
a |loop| user can switch to \iter\ with little pain (and much gain).

All clauses with the standard keyword-argument syntax consist of two
parts: a ~required~ part, 
containing keywords that must be present and in the right order; and
an ~optional~ part, containing keywords that may be omitted and,
if present, may occur in any order.  In the descriptions below, the
parts are separated by the Lisp lambda-list keyword \opt.


\subsection{Drivers}

An iteration-driving clause
conceptually causes the iteration to go forward.  Driver clauses in
\iter\ allow iteration over numbers, lists, vectors, hashtables, packages,
files and streams.  Iteration-driving clauses must
appear at the top level of an \iter\ form; they cannot be nested
inside another clause.  The driver variable is updated at the point
where the driver clause occurs.  Before the clause is executed for the
first time, the value of the variable is undefined.

%Also, regardless of where the driver clause
%appears in the body, the driver variable is stepped at the top of the
%loop; hence it is stylistically preferable, though
%not required, to place driver clauses at the beginning of the \iter.

Multiple drivers may appear in a single \iter\ form, in which case all
of the driver variables are updated each time through the loop, in the
order in which the clauses appear.  The first driver to terminate will
terminate the entire loop.

In all cases, the value of the driver variable on exit from the loop,
including within the epilogue code (see the |finally| clause), is
undefined.

All the parameters of a driver clause are evaluated once, before the
loop begins.  Hence it is not possible to change the bounds or other
properties of an iteration by side-effect from within the loop.

With one exception, driver clauses begin with the word |for| (or
the synonym |as|) and mention an iteration variable, which is
given a binding within the \iter\ form.
The exception is |repeat|, which just executes a loop a
specified number of times:

\begin{clauses}

\clausex{repeat}{~n~}
Repeats the loop ~n~ times.  For example:
\begin{program}
(iter (repeat 100)
      (print "I will not talk in class."))
\end{program}
\cpar If $n \leq 0$, the
loop will never be executed.  If ~n~ is not an integer, the
actual number of executions will be $\lceil n \rceil$.

\end{clauses}

\pagebreak[3]

\subsubsection{Numerical Iteration}

\begin{clauses}
\clausex{for}{~var~ |\&sequence|} 
The general form for iterating over a sequence of numbers requires a
variable and, optionally, one or more keywords that provide the bounds
and step size of the iteration. The |\&sequence|
lambda-list keyword is a shorthand for these sequence keywords.  They are:
|from|, |upfrom|, |downfrom|, |to|, |downto|, |above|, |below| and |by|.
|from| provides the starting value for ~var~
and defaults to zero. |to| provides a final value and implies that the
successive values of ~var~ will be increasing; |downto| implies that
they will be decreasing.  The loop terminates when ~var~ passes
the final value (i.e. becomes smaller or larger than it, depending on
the direction of iteration); in other  words, the loop body
will never be 
executed for values of ~var~ past the final value.
|below| and |above| are similar to |to| and
|downto|, except that the loop terminates when ~var~ equals or passes
the final value.

\cpar If no final value is specified, the variable will be stepped
forever.  Using |from| or |upfrom| will result in increasing
values, while |downfrom| will give decreasing values.

\cpar On each iteration,
~var~ is incremented or decremented by the value of the sequence
keyword |by|, which defaults to 1.  It should always be a positive
number, even for downward iterations.

\cpar In the following examples, the sequence of numbers generated is shown
next to the clause.

\begin{program}
(for i upfrom 0) \yields\ 0 1 2 \ldots
(for i from 5)   \yields\ 5 6 7 \ldots    ; either from or upfrom is okay
(for i downfrom 0) \yields\ 0 -1 -2 \ldots
(for i from 1 to 3) \yields\ 1 2 3
(for i from 1 below 3) \yields\ 1 2
(for i from 1 to 3 by 2) \yields\ 1 3
(for i from 1 below 3 by 2) \yields\ 1
(for i from 5 downto 3) \yields\ 5 4 3
\end{program}

\end{clauses}

\subsubsection{Sequence Iteration}

There are a number of clauses for iterating over sequences.  In all of
them, the argument following |for| may be a list instead of a
symbol, in which case destructuring is performed.  See section
\ref{destructuring}. 

\begin{clauses}

\clause{|for| ~var~ |in| ~list~ \opt\ |by| ~step-function~}
\clindexx{for}{in}
~var~ is set to successive elements of list.  
~step-function,~ which 
defaults to |cdr|, is used to obtain the next sublist. 

\clause{|for| ~var~ |on| ~list~ \opt\ |by| ~step-function~}
\clindexx{for}{on}
~var~ is set to successive sublists of list. 
~step-function~ (default |cdr|) is used as in |for\dots in|.

\end{clauses}

\medskip

These two clauses use |atom| to test for the end
of a list.  Hence, given a list whose final |cdr| is not \nil,
they will silently ignore the last |cdr|.  Other choices are
|endp|,  which would signal an error, and |null|, which
would probably result in an error somewhere else.  If you wish to use
an end-test other than |atom|, set the variable
|iterate::*list-end-test*|\lispindex{*list-end-test*} to the name of
the desired function. 

\begin{clauses}
\clause{|for| ~var~ |in-vector| ~vector~ |\&sequence|}
\clindexx{for}{in-vector}
~var~ takes on successive elements from ~vector.~   The vector's
fill-pointer is observed.  Here and in subsequent clauses, the |\&sequence|
keywords include |with-index|, which takes a symbol as argument
and uses it for the index variable instead of an internally generated
symbol.  The other |\&sequence| keywords behave as in numerical
iteration, except that the default iteration bounds are the bounds of
the vector.  E.g. in |(for i in-vector v downto 3)|,
|i| will start off being bound to the last element in |v|,
and will  
be set to preceding elements down to and including the element with
index 3.

\clause{|for| ~var~ |in-sequence| ~seq~ |\&sequence|}
\clindexx{for}{in-sequence}
This uses Common Lisp's generalized sequence functions, |elt|
and |length|, to obtain elements and determine the length of
~seq.~   Hence it will work for any sequence, including lists, and will
observe the fill-pointers of vectors.

\clause{|for| ~var~ |in-string| ~string~ |\&sequence|}
\clindexx{for}{in-string}
~var~ is set to successive characters of ~string.~

\startitem
\Clause{|for| ~var~ |index-of-vector| ~vector~ |\&sequence|}
\clindexx{for}{index-of-vector}
\Clause{|for| ~var~ |index-of-sequence| ~sequence~ |\&sequence|}
\clindexx{for}{index-of-sequence}
\Clause{|for| ~var~ |index-of-string| ~string~ |\&sequence|}
\clindexx{for}{index-of-string}
\finishitem
~var~ is set to successive indices of the sequence.
These clauses avoid the overhead of accessing the sequence elements
for those applications where they do not need to be examined, or are
examined rarely.  They admit all the optional keywords of the other
sequence drivers except the (redundant) |with-index| keyword.

\clause{|for| ~(key value)~ |in-hashtable| ~table~}
\clindexx{for}{in-hashtable}

~key~ and ~value,~ which must appear as shown in a list
and may be destructuring templates, are set to the keys and values of
~table~.  If ~key~ is |nil|, then the hashtable's keys will be ignored;
similarly for ~value~.
The order in which 
elements of ~table~ will be retrieved is unpredictable.

\clause{|for| ~var~ |in-package| ~package~ \opt\ |external-only| ~ext~}
\clindexx{for}{in-package}
Iterates over all the symbols in ~package,~ or over only the
external symbols if ~ext~ is specified and non-|nil|.  ~ext~ is not
evaluated.  The same symbol may appear more than once.

\clause{|for| ~var~ |in-packages| ~packages~ \opt\ |having-access| ~symbol-types~}
\clindexx{for}{in-packages}
Iterates over all the symbols from the list of packages denoted by the
descriptor ~packages~ and having accessibility (or visibility) given by
~symbol-types~.  This defaults to the list |(:external :internal :inherited)|
and is not evaluated.  ~var~ must be a list of up to three variables: in each
iteration, these will be set to a symbol, its access-type and package (as per
|with-package-iterator| in ANSI-CL).  The same symbol may appear more than
once.

\end{clauses}

\begin{clauses}

\clause{|for| ~var~ |in-file| ~name~ \opt\ |using| ~reader~}
\clindexx{for}{in-file}
Opens the file ~name~ (which may be a string or pathname) for input,
and iterates 
over its contents. ~reader~ defaults to |read|, so by default ~var~
will be bound to the successive forms in the file.  The \iter\ body is
wrapped in an unwind-protect to ensure that the file is closed no
matter how the \iter\ is exited.

\clause{|for| ~var~ |in-stream| ~stream~ \opt\ |using| ~reader~}
\clindexx{for}{in-stream}
Like |for\dots in-file|, except that ~stream~ should be an existing
stream object that supports input operations.

\end{clauses}

\subsubsection{Generalized Drivers}

These are primarily useful for writing drivers that can also be used
as generators (see section \ref{generators}, below).
 
\begin{clauses}

\clause{|for| ~var~ |next| ~expr~}
\clindexx{for}{next}

~var~ is set to ~expr~ each time through the loop.  Destructuring is
performed.  When the clause is used as a generator, ~expr~ is the code
that is executed when |(next ~var~)| is 
encountered (see section \ref{generators}, below).
~expr~ should compute the first value for ~var~, as well
as all subsequent values, and is responsible for terminating the loop.
For compatibility with future versions of \iter, this termination
should be done with |terminate|\clindex{terminate}, which can be
considered a synonym for |finish| (see section \ref{control-flow}).

\cpar As an example, the following clauses are equivalent to |(for
i from 1 to 10)|:
\begin{program}
(initially (setq i 0))
(for i next (if (> i 10) (terminate) (incf i)))
\end{program}

\clause{|for| ~var~ |do-next| ~form~}
\clindexx{for}{do-next}
~form~ is evaluated each time through the loop.  Its value is ~not~
set to ~var~; that is ~form's~ job. ~var~ is only present so that
\iter\ knows it is a driver variable.  \linebreak |(for ~var~ next
~expr~)| is equivalent to |(for ~var~ do-next (dsetq ~var~ ~expr~))|.
(See section \ref{destructuring} for an explanation of |dsetq|.) 

\end{clauses}

\subsubsection{Generators}
\label{generators}

In all of the above clauses, the driver variable is updated on each
iteration.
Sometimes it is desirable to have greater control over updating.  
For instance, consider the problem of associating numbers, in
increasing order and with no gaps, with the
\nonnil\ elements of a list.  One obvious first pass at writing this is:

\begin{program}
(iter (for el in list)
      (for i upfrom 1)
      (if el (collect (cons el i))))
\end{program}
But on the list |(a b nil c)| this produces |((a . 1) (b . 2) (c .
4))| instead of the desired |((a . 1) (b . 2) (c . 3))|.  The problem
is that |i| is incremented each time through the loop, even when |el|
is |nil|.  

The problem could be solved elegantly if we could step |i| only when
we wished to.  This
can be accomplished for any \iter\ driver by writing |generate| (or
its synonym |generating|)
instead of |for|.  Doing so produces a ~generator~---a driver whose
values are yielded explicitly.  To obtain the next value of a
generator variable ~v~, write \linebreak |(next ~v~)|.  The value of a |next|
form is the next value of ~v~, as determined by its associated driver
clause.  |next| also has the side-effect of updating ~v~ to that
value.  If there is no next value, |next| will terminate the loop,
just as with a normal driver.


Using generators, we can now write our example like this:
\begin{program}
(iter (for el in list)
      (generate i upfrom 1)
      (if el (collect (cons el (next i)))))
\end{program}
Now |i| is updated only when |(next i)| is executed, and this occurs
only when |el| is \nonnil.

To better understand the relationship between ordinary drivers and
generators, observe that we can rewrite an ordinary driver using its
generator form immediately followed by |next|, as this example shows:
\begin{program}
(iter (generating i from 1 to 10)
      (next i)
      ...)
\end{program}
Provided that the loop body contains no |(next i)| forms, this will
behave just as if we had written |(for i from 1 to 10)|.

We can still refer to a driver variable ~v~ without using |next|; in
this case, its value is that given to it by the last evaluation of
|(next ~v~)|.  Before |(next ~v~)| has been called the first time, the
value of ~v~ is undefined.  

This semantics is more flexible than
one in which ~v~ begins the loop bound to its first value and calls
of |next| supply subsequent values, because it means the loop will not
terminate too soon if the generator's sequence is empty.  For
instance, consider the following code, which tags \nonnil\ elements of
a list using a list of tags, and also counts the null elements.
(We assume there are at least as many tags as \nonnil\ elements.)
\begin{program}
(let* ((counter 0)
       (tagged-list (iter (for el in list)
                          (generating tag in tag-list) 
                          (if (null el)
                              (incf counter)
                              (collect (cons el (next tag)))))))
  ...)
\end{program}
It may be that there are just as many tags as non-null elements of
|list|.  If all the elements of |list| are null, we still want the
counting to proceed, even though |tag-list| is \nil.  If |tag| had to be
assigned its first value before the loop begins, we would have had to
terminate the loop before the first iteration, since when |tag-list|
is \nil, |tag| 
has no first value.  With the existing semantics, however, |(next
tag)| will never execute, so the iteration will cover all the elements of
|list|. 
        
When the ``variable'' of a driver clause is actually a destructuring
template containing several variables, all the variables are eligible
for use with |next|.  As before, |(next ~v~)| evaluates to ~v's~ next
value; but the effect is to update all of the template's variables.
For instance, the following code will return the list |(a 2 c)|.
\begin{program}
(iter (generating (key . item) in '((a . 1) (b . 2) (c . 3)))
      (collect (next key))
      (collect (next item)))
\end{program}

Only driver clauses with variables can be made into generators.  This
includes all clauses mentioned so far except for |repeat|.  It does
~not~ include |for\dots previous|, |for\dots =|,
|for\dots initially\dots then| or |for\dots first\dots then| (see
below). 

\subsubsection{Previous Values of Driver Variables}

Often one would like to access the value of a variable on a previous
iteration.  \iter\ provides a special clause for accomplishing this.

\begin{clauses}

\clause{|for| ~pvar~ |previous| ~var~ \opt\ |initially| ~init~
|back| ~n~}
\clindexx{for}{previous}
Sets ~pvar~ to the previous value of ~var,~ which should be a driver
variable, a variable from another |for\dots previous| clause, or a
variable established by a |for\dots =|, 
|for\dots initially\dots then| or |for\dots first\dots then| clause
(see section \ref{setting}).
Initially, ~pvar~ is given the value ~init~ (which defaults to \nil).  
The ~init~ expression will be moved outside the loop body, so it
should not depend on anything computed within the loop.
~pvar~ retains the value of ~init~ until ~var~ is set to its second
value, at which point ~pvar~ is set to ~var's~ first value; and so on.  

\cpar The
argument ~n~ to |back| must be a constant, positive integer, and
defaults to 1.  It determines how many iterations back ~pvar~ should
track ~var~.  For example, when ~n~ is 2, then ~pvar~ will be assigned
~var's~ first value when ~var~ is set to its third value.

\cpar A |for\dots previous| clause may occur after or before its
associated driver clause. |for\dots previous| works with generators as
well as ordinary drivers.

\pagebreak[3]

\cpar Example:
\begin{program}
(iter (for el in '(1 2 3 4))
      (for p-el previous el)
      (for pp-el previous p-el initially 0)
      (collect pp-el))
\end{program}
This evaluates to |(0 0 1 2)|.  It could have been written more
economically as 
\begin{program}
(iter (for el in '(1 2 3 4))
      (for pp-el previous el back 2 initially 0)
      (collect pp-el))
\end{program}


\end{clauses}

\subsection{Variable Binding and Setting}
\label{setting}

Several clauses exist for establishing new variable bindings or for
setting variables in the loop.  They all support destructuring.

\begin{clauses}

\clausex{with}{~var~ \opt\ |=| ~value~}
Causes ~var~ to be bound to value before the loop body is entered.
If ~value~ is not supplied, ~var~ assumes a default
binding, which will be 
\nil\ in the absence of declarations.  Also, if ~value~ is not
supplied, no destructuring is performed; instead, ~var~ may be a list of
symbols, all of which are given default 
bindings.  If ~value~ is supplied, ~var~ is bound to it, with
destructuring.  

\cpar Because |with|
creates bindings whose scope includes the entire \iter\ form, it is
good style to put all |with| clauses at the beginning.

\cpar Successive occurrences of |with| result in sequential
bindings (as with 
|let*|).  There is no way to obtain parallel bindings; see
section \ref{bindings} for a rationale.  


\clause{|for| ~var~ |=| ~expr~}
\clindexx{for}{=}
On each iteration, ~expr~ is evaluated and ~var~ is set
to its value. 

\cpar This clause may appear to do the same thing as |for\dots next|.
In fact, they are quite different.  |for\dots =| provides only three
services: it sets up a binding for ~var~, sets it to ~expr~ on each
iteration, and makes it possible to use |for\dots previous| with
~var~.  |for\dots next| provides these services in addition to the
ability to turn the driver into a generator.
%Also, the code which sets ~var~ appears in the loop body in the same
%place as the |for\dots =| clause; the code for |for\dots next| appears
%at the top of the loop, as with other drivers (except when being used
%as a generator).

\clause{|for| ~var~ |initially| ~init-expr~ |then| ~then-expr~}
\clindexxx{for}{initially}{then}
Before the loop begins, ~var~ is set to ~init-expr;~ on all
iterations after the first it is set to ~then-expr.~  This clause
must occur at top-level.  ~init-expr~ will be moved outside the loop
body and ~then-expr~ will be moved to the end of the loop body, so
they are subject to code motion problems (see section
\ref{code-movement}). 

\cpar This clause may appear to be similar to |for\dots next|, but in
fact they differ significantly.  |for\dots initially\dots then| is
typically used to give ~var~ its first value before the loop begins,
and subsequent values on following iterations.  This is incompatible
with generators, whose first value and subsequent values must all be
computed by |(next ~var~)|.  Also, the update of ~var~ in 
|for\dots initially\dots then| does not occur at the location of the
clause. 
Use |for\dots initially\dots then| for
one-shot computations where its idiom is more convenient, but use
|for\dots next| for extending \iter\ with new drivers (see section
\ref{extend}).

\clause{|for| ~var~ |first| ~first-expr~ |then| ~then-expr~}
\clindexxx{for}{first}{then}
The first time through the loop, ~var~ is set to ~first-expr;~ on
subsequent iterations, it is set to ~then-expr.~  This differs from
|for\dots initially| in that ~var~ is set to ~first-expr~
inside the loop body, 
so ~first-expr~ may depend on the results of other clauses.  For
instance, 
\begin{program}
(iter (for num in list)
      (for i first num then (1+ i))
      ...)
\end{program}
will set |i| to the first element of |list| on the first
iteration, whereas 
\begin{program}
(iter (for num in list)
      (for i initially num then (1+ i))
      ...)
\end{program}
is probably erroneous; |i| will be bound to |num|'s
default binding (usually \nil) for the first iteration.

\end{clauses}

\begin{note}{Compatibility note}
|loop|'s |for\dots =| works like \iter's, but |loop| used the syntax
|for\dots =\dots then| to mean |for\dots initially\dots then|.  It was
felt that these two operations were sufficiently different to warrant
different keywords.

Also, the |for| in the above three clauses is misleading,
since none is true driver (e.g. none has a corresponding |generate|
form).  |setting| would have been a better choice, but |for| was used
to retain some compatibility with |loop|.  
\end{note}


%%%%%%%%%%%%%%%%%%%%%%%%%%%%%%%%
\subsection{Gathering Clauses}
Many of \iter's clauses accumulate values into a variable, or set a
variable under certain conditions.  At the end of the loop, this
variable contains the desired result.  All these clauses have an
optional |into| keyword, whose argument should be a symbol.  If the
|into| keyword is not supplied, the accumulation variable will be
internally generated and its value will be returned at the end of the
loop; if a variable is specified, that variable is used for the
accumulation, and is not returned as a result---it is up to the user
to return it explicitly, in the loop's epilogue code (see |finally|).
It is safe to examine the accumulation variable during the loop, but
it should not be modified.

These clauses all begin with a verb.  When the verb does
not conflict with an existing Common Lisp function, then it may be
used in either its infinitival or present-participle form (e.g. |sum|,
|summing|).  However, when there is a conflict with Common Lisp, only
the present-participle form may be used (e.g. |unioning|).  This is to
prevent \iter\ clauses from clashing with Common Lisp
functions. 

%although these clauses are described as ``producing a value,'' it is a
%mistake to think of the lisp list representing the clause as a
%value-producing form in the usual way.  clauses may legally be written
%where a value is expected, e.g. |(setq x (sum i))|, but the
%lisp value of a clause in such a context is undefined.

\subsubsection{Reductions}

~Reduction~ is an extremely common iteration pattern in which
the results of successive applications of a
binary operation are accumulated.
For example, a loop that computes the sum of the
elements of a list is performing a reduction with the addition
operation.  This could be written in Common Lisp as  |(reduce \#'+
list)| or with \iter\ as
\begin{program}
(iter (for el in list)
      (sum el))
\end{program}

\begin{clauses}

\clausex{sum}{~expr~ \opt\ |into| ~var~}
Each time through the loop, ~expr~ is evaluated and added to a
variable, which is bound initially to zero.  If ~expr~ has a type,
it is ~not~ used as the type of the sum variable, which is always
|number|.  To get the result variable to be of a more specific
type, use an explicit variable, as in
\begin{program}
(iter (for el in number-list)
      (sum el into x)
      (declare (fixnum x))
      (finally (return x)))
\end{program}

\clausex{multiply}{~expr~ \opt\ |into| ~var~}
Like |sum|, but the initial value of the result variable is $1$,
and the variable is updated by multiplying ~expr~ into it.

\clausex{counting}{~expr~ \opt\ |into| ~var~}
~expr~ is evaluated on each iteration.  If it is \nonnil, the
accumulation variable, initially zero, is incremented.

\startitem
\Clausex{maximize}{~expr~ \opt\ |into| ~var~}
\Clausex{minimize}{~expr~ \opt\ |into| ~var~}
\finishitem
~expr~ is evaluated on each iteration and its extremum (maximum or
minimum) is stored in the accumulation variable.  If ~expr~ is never
evaluated, then the result is \nil\ (if the accumulation variable is
untyped) or $0$ (if it has a numeric type).

\clausex{reducing}{~expr~ |by| ~func~ \opt\ |initial-value| ~init-val~ |into| ~var~}
This is a general way to perform reductions. ~func~ should be a
function of two arguments, the first of which will be the value
computed so far and
the second of which will be the value of ~expr~.  It should return the
new value.
|reducing| is roughly equivalent to the
Common Lisp |(reduce ~func~ ~sequence~ :key ~expr-function~)|, where
~expr-function~ is used to derive values from the successive elements
of ~sequence~.

\cpar If the |reducing| clause is never executed, the result is
undefined.

\cpar It is not necessary to provide an
initial value, but better code can be generated if one is supplied.
Regardless of its location in the \iter\ body, ~init-val~ will be
evaluated before the loop is entered, so it should not depend on any
value computed inside the \iter\ form.  

%\cpar if a ~var~ is not specified, you can get \iter\ to declare the
%type of the internal variable by putting a |the| expression around
%~func.~  see section \ref{types}.

\end{clauses}

%\begin{note}{implementation note}
%in principle, |maximize| and |minimize| can be thought of
%as reductions where the initial value is the smallest (or largest)
%value that the accumulation variable can assume.  because lisp's
%bignums can represent arbitrary integers, these clauses cannot be
%implemented as reductions in general.  if, however, the type of 
%~expr~ or ~var~ can be determined to be a fixnum
%or a float, \iter\ will implement the clause as a true reduction,
%using one of the constants |most-negative-fixnum|, |%most-positive-fixnum|, 
%|most-negative-short-float|, etc. as appropriate.
%\end{note}

\subsubsection{Accumulations}

All the predefined accumulation clauses add values to a sequence.  If
the sequence is a list, they all 
have the property that the partial list is kept in the correct order
and available for inspection at any point in the loop.

\begin{clauses}

\clausex{collect}{~expr~ \opt\ |into| ~var~ |at| ~place~ |result-type| ~type~}
Produces a sequence of the values of ~expr~ on each iteration. ~place~
indicates where the next value of ~expr~ is added to the list and may
be one of the symbols |start|, |beginning| (a synonym for |start|) or
|end|.  The symbol may be quoted, but need not be.  The default is
|end|.  For example,
\begin{program}
(iter (for i from 1 to 5)
      (collect i))
\end{program}
produces |(1 2 3 4 5)|, whereas
\begin{program}
(iter (for i from 1 to 5)
      (collect i at beginning))
\end{program}
produces |(5 4 3 2 1)| (and is likely to be faster in most Common Lisp
implementations).

\cpar If ~type~ is provided, it should be a subtype of |sequence|.
The default is |list|.  Specifying a type other than |list| will
result in |collect| returning a sequence of that type.  ~However,~ the
type of the sequence being constructed when inside the loop body is
undefined when a non-|list| type is specified.  (As with ~place~,
quoting ~type~ is optional.)

\clausex{adjoining}{~expr~ \opt\ |into| ~var~ |test| ~test~ |at| ~place~ |result-type| ~type~}
Like |collect|, but only adds the value of ~expr~ if it is not
already present.  ~test,~ which defaults to |\#'eql|, is
the test to be used with |member|.

\startitem
\Clausex{appending}{~expr~ \opt\ |into| ~var~ |at| ~place~}
\Clausex{nconcing}{\ ~expr~ \opt\ |into| ~var~ |at| ~place~}
\Clausex{unioning}{\ ~expr~ \opt\ |into| ~var~ |test| ~test~ |at| ~place~}
\Clausex{nunioning}{~expr~ \opt\ |into| ~var~ |test| ~test~ |at| ~place~}
\finishitem
These are like |collect|, but behave like the Common Lisp functions
|append|, |nconc|, |union| or |nunion|.  
As in Common Lisp, they work only on lists.  Also as in Common Lisp,
|unioning| and |nunioning| assume that the value of ~expr~ contains no
duplicates. 

\clausex{accumulate}{~expr~ |by| ~func~ \opt\ |initial-value| ~init-val~ |into| ~var~}
This is a general-purpose accumulation clause. ~func~ should be a
function of two arguments, the value of ~expr~ and the value
accumulated so far in the iteration, and it should return the updated
value.  If no initial value is supplied, \nil\ is used.  

%\cpar If a ~var~ is not specified, you can get \iter\ to declare the
%type of the internal variable by putting a |the| expression around
%~func.~  see section \ref{types}.

\cpar The differences between |accumulate| and |reducing| are slight.
One difference is that the functions take their arguments in a
different order.  Another is that in the absence of ~init-val~,
|accumulate| will use \nil, whereas |reducing| will generate different
code that avoids any dependence on the initial value.
The reason for having both clauses is that one usually
thinks of reductions (like |sum|) and accumulations (like |collect|)
as different beasts.

\end{clauses}


\subsubsection{Finders}

A ~finder~ is a clause whose value is an expression that meets some
condition.

\begin{clauses}

\clause{|finding| ~expr~ |such-that| ~test~ \opt\ |into| ~var~ |on-failure| ~failure-value~}
\clindexx{finding}{such-that}
If ~test~ (which is an expression) ever evaluates to \nonnil, the loop
is terminated, the 
epilogue code is run and the value of ~expr~ is returned.  Otherwise,
\nil\ (or ~failure-value,~ if provided) is returned.  If ~var~ is
provided, it will have either the \nonnil\ value of ~expr~ or
~failure-value~ when the epilogue code is run.

\cpar As a special case, if the ~test~ expression is a sharp-quoted
function, then it is applied to ~expr~ instead of being simply
evaluated.  E.g. |(finding x such-that \#'evenp)| is equivalent to
|(finding x such-that (evenp x))|.  

%\cpar although ~test~ need have nothing to do with ~%expr~ as in
%|(finding j such-that (> i 3))|, it usually 
%will: |(finding (length el) such-that (oddp (length el)))|.  to
%avoid performing the |length| computation twice, you could write
%|(finding (length el) such-that \#'oddp)| or |(finding (length
%el) such-that 'oddp)|; for these cases, \iter\ generates code that
%executes ~expr~ only once.  the code for |\#'oddp|
%is slightly different from that for {\lisp 'oddp}; see the discussion
%under {\lisp for\dots in} and {\lisp for\dots on}.

\cpar |On-failure| is a misnomer. Because it is always evaluated, it behaves
more like the default third argument to the |gethash| function. As a result,
|on-failure (error "Not found")| makes no sense. Instead, the clauses |leave|
or |thereis| can be used in conjunction with |finally| as follows:
\begin{program}
(iter (for x in '(1 2 3))
      (if (evenp x) (leave x))
      (finally (error "not found")))
\end{program}

\cpar This clause may appear multiple times when all defaults are
identical. It can also be used together with either |always|/|never| or
|thereis| if their defaults match. More specifically, |on-failure nil| is
compatible with |thereis|, while |on-failure t| is compatible with |always|
and |never| clauses.
\begin{program}
(iter (for i in '(7 -4 2 -3))
      (if (plusp i)
	  (finding    i  such-that (evenp i))
          (finding (- i) such-that (oddp i))))
\end{program}

\startitem
\Clause{|finding| ~expr~ |maximizing| ~m-expr~ \opt\ |into| ~var~}
\clindexx{finding}{maximizing}
\Clause{|finding| ~expr~ |minimizing| ~m-expr~ \opt\ |into| ~var~}
\clindexx{finding}{minimizing}
\finishitem
Computes the extremum (maximum or minimum) value of ~m-expr~ over all
iterations, and returns the value of ~expr~ corresponding to the
extremum.  ~expr~ is evaluated inside the loop at the time the new
extremum is established.  If ~m-expr~ is never evaluated (due to, for
example, being embedded in a conditional clause), then the returned
value depends on the type, if any, of ~expr~ (or ~var,~ if
one is supplied).  If there is no type, the returned
value will be \nil; if the type is numeric, the returned value will be
zero.

\cpar For these two clauses, ~var~ may be a list of two
symbols; in that case, the first is used to record ~expr~ and
the second, ~m-expr.~  

\cpar As with |finding\dots such-that|, if ~m-expr~ is a sharp-quoted
function, then it is called on ~expr~ instead of being evaluated.

\end{clauses}

\subsubsection{Boolean Tests}

\begin{clauses}

\clausex{first-iteration-p}{}
Returns |t| in the first cycle of the loop, otherwise \nil.

\clausex{first-time-p}{}
Returns |t| the first time the expression is evaluated, and then \nil\ forever.
This clause comes handy when printing (optional) elements separated
by a comma:

\begin{program}
(iter (for el in '(nil 1 2 nil 3))
      (when el
        (unless (first-time-p)
          (princ ", "))
        (princ el)))
\end{program}
produces |"1, 2, 3"|.

\end{clauses}

\subsubsection{Aggregated Boolean Tests}

\begin{clauses}

\clausex{always}{~expr~}
If ~expr~ ever evaluates to
\nil, then \nil\ is immediately returned; the epilogue code is not
executed.  If ~expr~ never evaluates to \nil, the epilogue code
is executed and the last value of ~expr~ (or |t| if ~expr~ was never
evaluated) is returned (whereas |loop| would constantly return |t|).

% mention last evaluated clause when multiple always clauses?

\clausex{never}{~expr~}
Like |(always (not ~expr~))|, except it does not influence the last
value returned by a possible other |always| clause. That is,
\begin{program}
(iter (repeat 2)
      (always 2)
      (never nil)) \yields 2 ; not t
\end{program}

\clausex{thereis}{~expr~}
If ~expr~ is ever \nonnil,
its value is immediately returned without running epilogue code.
Otherwise, the epilogue code is performed and \nil\ is returned.

This clause cannot be used together with |always| or |never|, because their
defaults are opposed (similarly, |(loop always 3 thereis nil)| refuses to
compile in some implementations of |loop|).

\end{clauses}


\subsection{Control Flow}
\label{control-flow}
Several clauses can be used to alter the usual flow of control in a loop.

Note: the clauses of this and subsequent sections don't adhere to \iter's
usual syntax, but instead use standard Common Lisp syntax.  Hence the
format for describing syntax subsequently is
like the standard format used in the Common Lisp manual, not like the
descriptions of clauses above.

\begin{clauses}

\clausex{finish}{}
Stops the loop and runs the epilogue code.

%for example:
%
%\begin{program}
%(iter (with answer = nil)
%         (initially (make-a-mess))
%         (for i from 1 to 10)
%         (when (correct? i)
%           (setq answer i)
%           (finish))
%         (finally (cleanup)))
%\end{program}
%
%this code will execute |cleanup| whether or not the test |(correct?
%i)| ever succeeds. 
%the (more elegant) formulation,
%\begin{program}
%(iter (initially (make-a-mess))
%         (for i from 1 to 10)
%         (finding i such-that (correct? i))
%         (finally (cleanup)))
%\end{program}
%would not execute |cleanup| if |(correct? i)| succeeded; it
%would do an immediate return.

\clausex{leave}{\opt\ ~value~}
Immediately returns ~value~ (default \nil) from the current \iter\
form, skipping the epilogue code.  Equivalent to using |return-from|.

\clausex{next-iteration}{}
Skips the remainder of the loop body and begins the next iteration of
the loop.

\clausex{while}{~expr~}
If ~expr~ ever evaluates to \nil, the loop is terminated and the
epilogue code executed.  Equivalent to |(if (not ~expr~) (finish))|.

\clausex{until}{~expr~}
Equivalent to |(if ~expr~ (finish))|.

\clausex{if-first-time}{~then~ \opt\ ~else~}
If this clause is being executed for the first time in this invocation
of the \iter\ form, then the ~then~ code is evaluated; otherwise the
~else~ code is evaluated.

\cpar |(for ~var~ first ~expr1~ then ~expr2~)| is almost equivalent to
\begin{program}
(if-first-time (dsetq ~var~ ~expr1~)
               (dsetq ~var~ ~expr2~))
\end{program}
The only difference is that the |for| version makes ~var~ available
for use with |for\dots previous|.

\end{clauses}

\subsection{Code Placement}
When fine control is desired over where code appears in a loop
generated by \iter, the following special clauses may be useful.
They are all subject to code-motion problems (see section
\ref{code-movement}).  

\begin{clauses}

\clausex{initially}{|\&rest| ~forms~}
The lisp ~forms~ are placed in the prologue section of the loop, where
they are executed once, before the loop body is entered.

\clausex{after-each}{|\&rest| ~forms~}
The ~forms~ are placed at the end of the loop body, where they
are executed after each iteration.  Unlike the other clauses in this
section, ~forms~ may contain \iter\ clauses.

\clausex{else}{|\&rest| ~forms~}
The lisp ~forms~ are placed in the epilogue section of the loop, where they
are executed if this |else| clause is never met during execution of the
loop and the loop terminates normally.

\clausex{finally}{|\&rest| ~forms~}
The lisp ~forms~ are placed in the epilogue section of the loop, where
they are executed after the loop has terminated normally.

\clausex{finally-protected}{|\&rest| ~forms~}
The lisp ~forms~ are placed in the second form of an unwind-protect
outside the loop.  They are always executed after the loop has
terminated, regardless of how the termination occurred.

\end{clauses}


%%%%%%%%%%%%%%%%%%%%%%%%%%%%%%%%%%%%%%%%%%%%%%%%%%%%%%%%%%%%%%%%
\section{Other Features}

\subsection{Multiple Accumulations}
\label{multiple}

\begin{sloppypar}
It is permitted to have more than one clause accumulate into the same
variable, as in the following:
\begin{program}
(iter (for i from 1 to 10)
      (collect i into nums)
      (collect (sqrt i) into nums)
      (finally (return nums)))
\end{program}
Clauses can only accumulate into the same variable if they are
compatible.  |collect|, |adjoining|, |appending|, |nconcing|,
|unioning| and |nunioning| are compatible with each other; |sum|,
|multiply| and |counting| are compatible; |always| and |never| are compatible;
|finding| \dots |such-that| is compatible with either |thereis| or |always|
and |never| when their defaults match; and |maximize| and |minimize| clauses
are compatible only with other |maximize| and |minimize| clauses,
respectively.

%note that the same variable ~cannot~ be both an accumulation
%variable and an ordinary variable; there can be only one variable with
%a given name within an \iter\ form. 
\end{sloppypar}

\subsection{Named Blocks}

Like Common Lisp |block|s, \iter\ forms can be given names.  The
name should be a single symbol, and it must be the first form in the
\iter.  The generated code behaves exactly like a named block; in
particular, |(return-from ~name~)| can be used to exit it:
\begin{program}
(iter fred
      (for i from 1 to 10)
      (iter barney
            (for j from i to 10)
            (if (> (* i j) 17)
                (return-from fred j))))
\end{program}
An \iter\ form that is not given a name is implicitly named \nil.

Sometimes one would like to write an expression in an inner \iter\ form,
but have it processed by an outer \iter\ form.  This is possible with
the |in| clause.  

\begin{clauses}

\clausex{in}{~name~ |\&rest| ~forms~}
Evaluates ~forms~ as if they were part of the \iter\ form named
~name~.  In other words, \iter\ clauses are processed by the \iter\
form named ~name,~ and not by any \iter\ forms that occur inside ~name.~

\cpar As an example, consider the problem of collecting a list of the
elements in a two-dimensional array.  The naive solution,
\begin{program}
(iter (for i below (array-dimension ar 0))
      (iter (for j below (array-dimension ar 1))
            (collect (aref ar i j))))
\end{program}
\noindent is wrong because the list created by the inner \iter\ is simply
ignored by the outer one.  But using |in| we can write:
\begin{program}
(iter outer (for i below (array-dimension ar 0))
      (iter (for j below (array-dimension ar 1))
            (in outer (collect (aref ar i j)))))
\end{program}
\noindent which has the desired result.

\end{clauses}

\subsection{Destructuring}
\label{destructuring}

In many places within \iter\ clauses where a variable is expected, a
list can be written instead.  In these cases, the value to be assigned
is ~destructured~ according to the pattern
described by the list.  As a simple example, the clause
\begin{program}
(for (key . item) in alist)
\end{program}
\noindent will result in |key| being set to the |car| of
each element in |alist|, and |item| being set to the |cdr|.  The
pattern list may be nested to arbitrary depth, and (as the example
shows) need not be terminated with \nil; the only requirement is that
each leaf be a bindable symbol (or \nil, in which case no binding is
generated for that piece of the structure).

Sometimes, you might like to do the equivalent of a
|multiple-value-setq| in a clause.  This 
``multiple-value destructuring'' can be expressed by writing 
\linebreak |(values $pat_1$ $pat_2 \ldots$)| for a destructuring
pattern, as in 
\begin{program}
(for (values (a . b) c d) = (three-valued-function ...))
\end{program}
\begin{sloppypar}
Note that the $pat_i$ can themselves be destructuring patterns (though
not multiple-value destructuring patterns).  You can't do multiple-value
destructuring in a |with| clause; instead wrap the whole \iter\
form in a |multiple-value-bind|.
\end{sloppypar}

\begin{note}{Rationale}
There are subtle interactions between variable declarations and
evaluation order that make the correct implementation of
multiple-value destructuring in a |with| somewhat tricky.
\end{note}

The destructuring feature of \iter\ is available as a separate
mechanism, using the |dsetq| macro:

\begin{clauses}
\defmacro{dsetq}{template expr}
Performs destructuring of ~expr~ using ~template~.   May be used
outside of an \iter\ form. Yields the primary value of ~expr~.

\end{clauses}

\subsection{On-line Help}

\begin{sloppypar}
There is a limited facility for on-line help, in the form of the 
|display-iterate-clauses| function.
\end{sloppypar}

\begin{clauses}

\defun{display-iterate-clauses}{\opt\ clause-spec}
Displays a list of \iter\ clauses.  If ~clause-spec~ is not
provided, all clauses are shown; if it is a symbol, all clauses
beginning with that symbol are shown; and if it is a list of symbols,
all clauses for which ~clause-spec~ is a prefix are shown.

\end{clauses}

\subsection{Parallel Binding and Stepping}
\label{bindings}

The parallel binding and stepping of variables is a feature that
\iter\ does ~not~ have.  This section attempts to provide a rationale.

We say that two variables are bound ~in parallel~ if neither
binding shadows the other.  This is the usual semantics of |let|
(as opposed to |let*|).  Similarly, we can say that iteration
variables are stepped in parallel if neither variable is updated
before the other, conceptually speaking; in other words, if the code
to update each variable can reference the old values of both variables.

|loop| allows parallel binding of variables and parallel stepping of
driver variables.  My view is that if you are depending on the 
serial/parallel distinction, you are doing something obscure.  If you
need to bind 
variables in parallel using |with|, then you must be using a
variable name that shadows a name in the existing lexical environment.
Don't do that.  The most common use for parallel stepping is to track
the values of variables on the previous iteration, but in fact this
does not require parallel stepping at all; the following will work:
\begin{program}
(iter (for current in list)
      (for prev previous current)
      ...)
\end{program}


%%%%%%%%%%%%%%%%%%%%%%%%%%%%%%%%%%%%%%%%%%%%%%%%%%%%%%%%%%%%%%%%
\section{Types and Declarations}
\label{types}

\subsection{Discussion}

Sometimes efficiency dictates that the types of variables be declared.
This type information needs to be communicated to \iter\ so it can
bind variables to appropriate values.  Furthermore, \iter\ must often
generate internal variables invisible to the user; there needs to be a
way for these to be declared.

As an example, consider this code, which will return the number of
odd elements in |number-list|:
\begin{program}
(iter (for el in number-list)
      (count (oddp el)))
\end{program}
In processing this form,
\iter\ will create an internal variable, let us call it |list17|, to
hold the successive |cdr|s of |number-list|, and  
will bind the variable to |number-list|.  It will also generate a
default binding for |el|; only inside the body of the loop will |el|
be set to the |car| of |list17|.  Finally, \iter\ will generate a
variable, call it |result|, to hold the result of the count, and will
bind it to zero.

When dealing with type declarations, \iter\ observes one simple rule:
~it will never generate a declaration unless requested to do so.~  The
reason is that such declarations might mask errors in compiled code by
avoiding error-checks; the resulting problems would be doubly hard to
track down because the declarations would be hidden from the
programmer.  Of course, a compiler might omit error-checks even in the
absence of declarations, though this behavior can usually be avoided,
e.g. by saying |(declaim (optimize (safety 3)))|.

So, the above \iter\ form will generate code with no declarations.
But say we wish to declare the types of |el| and the internal
variables |list17| and |result|.  How is this done?

Declaring the type of |el| is easy, since the programmer knows
the variable's name:
\begin{program}
(iter (for el in number-list)
      (declare (fixnum el))
      (counting (oddp el)))
\end{program}
\iter\ can read variable type declarations like this one.  Before
processing any clauses, it scans the entire top-level form for type
declarations 
and records the types, so that variable bindings can be performed
correctly.  In this case, |el| will be bound to zero
instead of \nil.  Also, \iter\ collects all the top-level declarations
and puts them at the begining of the generated code, so it is not
necessary to place all declarations at the beginning of an \iter\
form; instead, they can be written near the variables whose types they
declare.  

Since \iter\ is not part of the compiler, it will not know
about declarations that occur outside an \iter\ form; these
declarations must be repeated inside the form.

Here is another way we could have declared the type of |el|:
\begin{program}
(iter (for (the fixnum el) in number-list)
      (counting (oddp el)))
\end{program}
\iter\ extends the Common Lisp |the|\lispindex{the}
form to apply to variables as well as value-producing forms; anywhere
a variable is allowed---in a |with| clause, as the iteration
variable in a driver clause, as the |into| argument of an
accumulation clause, even inside a destructuring template---you can
write |(the ~type~ ~symbol~)| instead.

There is one crucial difference between using a |the| form and
actually declaring the variable: explicit declarations are always
placed in the generated code, but type information from a |the|
form is not turned into an actual declaration unless you tell \iter\
to do so using |iterate:declare-variables|.  See below.

\begin{sloppypar}
Declaring the types of internal variables is harder than declaring the
types of explicitly mentioned variables, since their names
are unknown.  You do it by declaring |iterate:declare-variables|
somewhere inside the top level of the \iter\ form.  (This will also
generate declarations for variables declared using |the|.)
\iter\ does not provide much selectivity here: it's all or none. 
And unfortunately, since \iter\ is not privy to compiler information
but instead reads declarations itself, it will not hear if you
|(declaim (iterate:declare-variables))|.  Instead, set the variable 
|iterate::*always-declare-variables*| to |t| at
compile-time, using |eval-when|.
\end{sloppypar}

To determine the appropriate types for internal variables, \iter\ uses
three sources of information:

\begin{itemize}

\item Often, the particular clause dictates a certain type for a
variable; \iter\ will use this information when available.  In the
current example, the variable |list17| will be given the type
|list|, since that is the only type that makes sense; and the
variable |result| will be given the type |fixnum|, on the
assumption that you will not be counting high enough to need bignums.
You can override this assumption only by using and explicitly declaring a
variable: 
\begin{verbatim}
(iter (declare (iterate:declare-variables))
      (for el in number-list)
      (count (oddp el) into my-result)
      (declare (integer my-result))
      (finally (return my-result)))
\end{verbatim}

\begin{sloppypar}
Other examples of the type assumptions that \iter\ makes are: type
|list| for |into| variables of collection clauses; type |list| for
expressions that are to be destructured; type |vector| for the
variable holding the vector in a |for\dots in-vector| clause, and
similarly for |string| and the |for\dots in-string| clause;
and the implementation-dependent type
|(type-of array-dimension-limit)| for the index and limit
variables generated by sequence iteration drivers like |for\dots
in-vector| and |for\dots in-string| (but not |for\dots in-sequence|,
because it may be used to iterate over a list). 
\end{sloppypar}

\item Sometimes, \iter\ will examine expressions and try to determine
their types in a simple-minded way.  If the expression is
self-evaluating (like a number, for instance), \iter\ knows that the
expression's type is the same as the type of the value it denotes, so
it can use that type.  If the expression is of the form |(the ~type~
~expr~)|, \iter\ is smart enough to extract ~type~ and use it.
However, the current version of \iter\ does not  
examine declarations of function result types or do any type
inference.  It will not determine, for 
example, that the type of |(+ 3 4)| is |fixnum|, or even
|number|.

\item In some cases, the type of an internal variable should match the
type of some other variable.  For instance, \iter\ generates an
internal variable for |(f x)| in the
clause |(for i from 1 to (f x))|, and in the absence of other
information will give it the same type as |i|.  If, however, the
expression had been written |(the fixnum (f x))|, then \iter\
would have given the internal variable the type |fixnum|
regardless of |i|'s type.  The type incompatibility errors that
could arise in this situation are not checked for.

\end{itemize}

Note that if you do declare |iterate:declare-variables|, then
\iter\ may declare user variables as well as internal ones if they do
not already have declarations, though only for variables that it
binds.  For instance, in this code:

\begin{program}
(iter (declare (iterate:declare-variables))
      (for i from 1 to 10)
      (collect i into var))
\end{program}
the variable |var| will be declared to be of type |list|.


\subsection{Summary}

\iter\ understands standard Common Lisp variable type declarations
that occur within an \iter\ form and
will pass them through to the generated code.  If the declaration
|(iterate:declare-variables)|\lispindex{declare-variables}
appears at the top level of an 
\iter\ form, or if 
|iterate::*always-declare-variables*|\lispindex{*always-declare-variables*} 
is \nonnil, then \iter\ will use the type information gleaned from user
declarations, self-evaluating expressions and |the| expressions,
combined with reasonable assumptions, to determine variable
types and declare them.


%%%%%%%%%%%%%%%%%%%%%%%%%%%%%%%%%%%%%%%%%%%%%%%%%%%%%%%%%%%%%%%%
\section{Problems with Code Movement}
\label{code-movement}

Some \iter\ clauses, or parts of clauses, result in code being
moved from the location of the clause to other parts of the loop.
Drivers behave this way, as do code-placement clauses like |initially|
and |finally|.  When using these clauses, there is a danger of writing
an expression that makes sense in its apparent location but will be
invalid or have a different meaning in another location.  For example:
\begin{program}
(iter (for i from 1 to 10)
      (let ((x 3))
        (initially (setq x 4))))
\end{program}
While it may appear that the |x| of |(initially (setq x 4))| is the
same as the |x| of |(let ((x 3)) \dots|, in fact they are not:
|initially| moves its code outside the loop body, so |x| would refer
to a global variable.  Here is another example of the same problem:
\begin{program}
(iter (for i from 1 to 10)
      (let ((x 3))
        (collect i into x)))
\end{program}
If this code were executed, |collect| would create a binding for its
|x| at the top level of the \iter\ form that the |let| will shadow.

Happily, \iter\ is smart enough to catch these errors; it
walks all problematical code to ensure that free variables are not
bound inside the loop body, and checks all variables it binds for the
same problem.

However, some errors cannot be caught:

\begin{program}
(iter (with x = 3)
      (for el in list)
      (setq x 1)
      (reducing el by \#'+ initial-value x))
\end{program}
|reducing| moves its |initial-value| argument to the initialization
part of the loop in order to produce more efficient code.  Since
\iter\ does not perform data-flow analysis, it cannot determine that
|x| is changed inside the loop; all it can establish is that |x| is
not bound internally.  Hence this code will not signal an
error and will use $3$ as the initial value of the reduction.

The following list summarizes all cases that are subject to these code
motion and variable-shadowing problems.
\begin{itemize}
\item Any variable for which \iter\ creates a binding, including those
used in |with| and the |into| keyword of many clauses.

\begin{sloppypar}
\item The special clauses which place code: |initially|, |after-each|, |else|,
|finally| and |finally-protected|. 
\end{sloppypar}

\item The variables of a |next| or |do-next| form.

\item The |initially| arguments of |for\dots initially\dots then| and
|for\dots previous|. 

\item The |then| argument of |for\dots initially\dots then|.

\item The |initial-value| arguments of |reducing| and |accumulate|.

\item The |on-failure| argument of |finding\dots such-that|.
\end{itemize}

%%%%%%%%%%%%%%%%%%%%%%%%%%%%%%%%%%%%%%%%%%%%%%%%%%%%%%%%%%%%%%%%
\section{Differences Between {\tt Iterate} and {\tt Loop}}
|loop| contains a great deal of complexity which \iter\ tries to
avoid.  Hence many esoteric features of |loop| don't exist in \iter.
Other features have been carried over, but in a cleaned-up form.  
And of course, many new features have been added; they are not
mentioned in this list.

\begin{itemize}

\item \iter's syntax is more Lisp-like than |loop|'s, having a higher
density of parens.

\item The current implementation of \iter, unlike the current version
of  |loop| (as documented in {\em Common Lisp, 2nd Ed.\/}), is
extensible (see section \ref{extend}).  

\item |loop| puts the updates of all driver variables at the top of
the loop; \iter\ leaves them where the driver clauses appear.

\item While for the most part \iter\ clauses that resemble |loop| clauses
behave similarly, there are some differences.  For instance, there is
no |for\dots =\dots then| in \iter; instead use 
|for\dots initially\dots then|.  

\item |loop| binds the variable |it| at certain times
to allow pseudo-English expressions like |when ~expr~ return it|. 
In \iter, you must bind ~expr~ to a variable yourself.  Note that
|when ~expr~ return it| is like |thereis ~expr~| except that the latter is an
accumulation clause and therefore competes with other accumulations
(remember section \ref{multiple} above).

% repeat different behaviour of |always| clause here?

\item |loop| has a special |return| clause, illustrated in the
previous item.  \iter\ doesn't need one,  since an ordinary Lisp
|return| has the same effect.

\item |loop| allows for parallel binding and stepping of iteration
variables.  \iter\ does not.  (See section \ref{bindings}.)

\item |loop| and \iter\ handle variable type declarations very
differently.  |loop| provides a special syntax for declaring variable
types, and does not examine declarations.  Moreover, the standard
implementation of |loop| will
generate declarations when none are requested.
\iter\ parses standard Common Lisp type declarations, and will never
declare a variable itself unless declarations are 
specifically requested.

\end{itemize}

%%%%%%%%%%%%%%%%%%%%%%%%%%%%%%%%%%%%%%%%%%%%%%%%%%%%%%%%%%%%%%%%
\section{Rolling Your Own}
\label{extend}

\subsection{Introduction}

\iter\ is extensible---you can write new clauses that embody new
iteration patterns.  You might want to write a new driver clause for a
data structure of your own, or you might want to write a clause that
collects or manipulates elements in a way not provided by \iter. 

This section describes how to write clauses for \iter.  Writing a
clause is like writing a macro.  In fact, writing a clause ~is~
writing a macro: since \iter\ code-walks its body and macroexpands,
you can add new abstractions to \iter\ with good old |defmacro|.

Actually, there are two extensions you can make to \iter\ that are
even easier than writing a macro.  They are adding a synonym for an
existing clause and defining a driver clause for an indexable
sequence.  These can be done with |defsynonym| and
|defclause-sequence|, respectively.  See section \ref{aids}, below.

The rest of this section explains how to write macros that expand into
\iter\ clauses.
Here's how you could add a simplified version of \iter's |multiply|
clause, if \iter\ didn't already have one:
\begin{program}
(defmacro multiply (expr)
  `(reducing ,expr by \#'* initial-value 1))
\end{program}

If you found yourself summing the square of an expression often, you
might want to write a macro for that.  A first cut might be
\begin{program}
(defmacro sum-of-squares (expr)
  `(sum (* ,expr ,expr)))
\end{program}
but if you are an experienced macro writer, you will realize that this
code will evaluate ~expr~ twice, which is probably a bad idea.  A
better version would use a temporary:
\begin{program}
(defmacro sum-of-squares (expr)
  (let ((temp (gensym)))
    `(let ((,temp ,expr))
       (sum (* ,temp ,temp)))))
\end{program}
Although this may seem complex, it is just the sort of thing you'd
have to go through to write any macro, which illustrates the point of
this section: if you can write macros, you can extend \iter.

Our macros don't use \iter's keyword-argument
syntax.  We could just use keywords with |defmacro|, but we would
still not be using \iter's clause indexing mechanism.  Unlike Lisp,
which uses just the first symbol of a form to determine what function
to call, \iter\ individuates clauses by the list of required keywords.
For instance, |for\dots in| and |for\dots in-vector| are different clauses
implemented by distinct Lisp functions.  

To buy into this indexing scheme, as well as the keyword-argument
syntax, use |defmacro-clause|:

\begin{clauses}

\defmacro{defmacro-clause}{arglist |\&body| body}
Defines a new \iter\ clause.  ~arglist~ is a list of symbols which are
alternating keywords and arguments.  \opt\ may be used, and the list
may be terminated by |\&sequence|.  ~body~ is an ordinary macro body,
as with |defmacro|.  If the first form of ~body~ is a string, it is
considered a documentation string and will be shown by
|display-iterate-clauses|. |defmacro-clause| will signal an error if 
defining the clause would result in an ambiguity.  E.g. you cannot
define the clause |for\dots from| because there would be no way to
distinguish it from a use of the |for| clause with optional keyword |from|.

\end{clauses}

\medskip

Here is |multiply| using |defmacro-clause|.  The keywords are capitalized
for readability.
\begin{program}
(defmacro-clause (MULTIPLY expr \opt\ INTO var)
  `(reducing ,expr by \#'* into ,var initial-value 1))
\end{program}
You don't have to worry about the case when |var| is not supplied; for
any clause with an |into| keyword, saying |into nil| is equivalent to
omitting the |into| entirely.

As another, more extended example, consider the fairly common
iteration pattern that 
involves finding the sequence element that maximizes (or minimizes) some
function.  \iter\ provides this as |finding\dots maximizing|, but it's
instructive to see how to write it.
Here, in pseudocode, is how you might write such a loop for
maximizing a function F:

\begin{quote}
\begin{tabbing}
set variable MAX-VAL to NIL; \\
set variable WINNER to NIL; \\
for\=\ each element EL in the sequence \\
\>  if\=\ MAX-VAL\=\ is NIL or F(EL) $>$ MAX-VAL then \\
\>\>    set MAX-VAL to F(EL); \\
\>\>    set WINNER to EL; \\
\>  end if; \\
end for; \\
return WINNER.
\end{tabbing}
\end{quote}

Here is the macro:
\begin{program}
(defmacro-clause (FINDING expr MAXIMIZING func \opt\ INTO var)
  (let ((max-val (gensym))
        (temp1 (gensym))
        (temp2 (gensym))
        (winner (or var iterate::*result-var*)))
    `(progn 
       (with ,max-val = nil)
       (with ,winner = nil)
       (cond 
        ((null ,max-val)
         (setq ,winner ,expr)
         (setq ,max-val (funcall ,func ,winner))
        (t
         (let* ((,temp1 ,expr)
                (,temp2 (funcall ,func ,temp1)))
           (when (> ,temp2 ,max-val)
             (setq ,max-val ,temp2)
             (setq ,winner ,temp1))))))
       (finally (leave ,winner)))))
\end{program}
Note that if no |into| variable is supplied, we use
|iterate::*result-var*|, which contains the internal variable into
which all clauses place their results.  If this variable is bound by
some clause, then \iter\ will return its value automatically;
otherwise, \nil\ will be returned.  

\subsection{Writing Drivers}

In principle, drivers can be implemented just as easily as other
\iter\ clauses.  In practice, they are a little harder to get right. 
As an example, consider writing a driver that
iterates over all the 
elements of a vector, ignoring its fill-pointer.  |for\dots in-vector|
won't work for this, because it observes the fill-pointer.  It's necessary to
use |array-dimension| instead of |length| to obtain the size of the
vector.  Here is one approach:

\begin{program}
(defmacro-clause (FOR var IN-WHOLE-VECTOR v)
  "All the elements of a vector (disregards fill-pointer)"
  (let ((vect (gensym))
        (index (gensym)))
    `(progn
       (with ,vect = ,v)
       (for ,index from 0 below (array-dimension ,vect 0))
       (for ,var = (aref ,vect ,index)))))
\end{program}
Note that we immediately put |v| in a variable, in case it is an
expression.  Again, this is just good Lisp macrology.  It also has a
subtle effect on the semantics of the driver: |v| is evaluated only
once, at the beginning of the loop, so changes to |v| in the loop have
no effect on the driver.  Similarly, the bounds for numerical iteration
e.g. the above |array-dimension| are also evaluated once only.  This is how
all of \iter's drivers work.


There is an important point concerning the |progn| in this code.  We
need the |progn|, of course, because we are returning several forms,
one of which is a driver.  But \iter\ drivers must occur at top-level.
Is this code in error?  No, because ~top-level~ is defined in \iter\
to include forms inside a |progn|.  This is just the definition of
top-level that Common Lisp uses, and for the same reason: to allow
macros to return multiple forms at top-level.

While our |for\dots in-whole-vector| clause will work, it is not
ideal.  In particular, it does not support generating.  Do do so, we
need to use |for\dots next| or |for\dots do-next|.  The job is
simplified by the |defmacro-driver| macro.

\begin{clauses}

\defmacro{defmacro-driver}{arglist |\&body| body}
Defines a driver clause in
both the |for| and |generate| forms, and provides a parameter
|generate| which ~body~ can examine to determine how it was invoked.
~arglist~ is as in |defmacro-clause|, and should begin with the symbol
|for|. 

\end{clauses}

With |defmacro-driver|, our driver looks like this:
\begin{program}
(defmacro-driver (FOR var IN-WHOLE-VECTOR v)
  "All the elements of a vector (disregards fill-pointer)"
   (let ((vect (gensym))
         (end (gensym))
         (index (gensym))
         (kwd (if generate 'generate 'for)))
     `(progn
        (with ,vect = ,v)
        (with ,end = (array-dimension ,vect 0))
        (with ,index = -1)
        (,kwd ,var next (progn (incf ,index)
                               (if (>= ,index ,end) (terminate))
                               (aref ,vect ,index))))))
\end{program}


We are still missing one thing: the |\&sequence| keywords.  
We can get them easily enough, by writing 
\begin{program}
(defmacro-driver (FOR var IN-WHOLE-VECTOR v \&sequence) 
  ...)
\end{program}
We can now refer to parameters |from|, |to|, |by|, etc. which contain
either the values for the corresponding keyword, or \nil\ if the
keyword was not supplied.  Implementing the right code for these
keywords is cumbersome but not difficult; it is left as an exercise.
But before you begin, see |defclause-sequence| below for an easier way.
       
\subsection{Extensibility Aids}
\label{aids}

This section documents assorted features that may be of use in
extending \iter.

\begin{clauses}

\defunexpvar{*result-var*}
Holds the variable that is used to return a value as a result of the
\iter\ form.  You may examine this and use it in a |with| clause, but
you should not change it.

\defmacro{defsynonym}{syn word}
 Makes ~syn~ a synonym for the existing \iter\ keyword ~word.~  Only
the first word in each clause can have synonyms. 

\defmacroxx{defclause-sequence}{element-name index-name}{|\&key|
access-fn size-fn sequence-type}{element-type
element-doc-string index-doc-string}  
Provides
a simple way to define sequence clauses.  Generates two
clauses, one for iterating over the sequence's elements, the other
for iterating over its indices.  The first symbol of both
clauses will have print-name |for|.
~element-name~  and ~index-name~ should be symbols.
~element-name~ is the second keyword of the element iterator (typically of
the form 
|in-~sequence-type~|), and ~index-name~ is the second keyword
of the index-iterator (typically of the form
|index-of-~sequence-type~|).  Either name may be 
\nil, in which case the corresponding clause is not defined.  If both
symbols are supplied, they should be in the same package.  The |for|
that begins the clauses will be in this package.

\cpar ~access-fn~ is the function to be used to
access elements of the sequence in the element iterator.  The function
should take two 
arguments, a sequence and an index, and return the appropriate element.
~size-fn~ should denote a function of one argument, a sequence, that
returns its size.  Both ~access-fn~ and ~size-fn~ are required for the
element iterator, but only ~size-fn~ is needed for the index iterator.

\cpar The ~sequence-type~ and ~element-type~ keywords are used to
suggest types for the variables 
used to hold the sequence and the
sequence elements, respectively.  The usual rules about \iter's
treatment of variable type declarations apply (see section \ref{types}).

\cpar ~element-doc-string~ and ~index-doc-string~ are
the documentation strings, for use with |display-iterate-clauses|. 

\cpar The generated element-iterator performs destructuring on the
element variable.

\cpar As an example, the above |for\dots in-whole-vector| example
could have been written:
\begin{program}
(defclause-sequence IN-WHOLE-VECTOR INDEX-OF-WHOLE-VECTOR
  :access-fn 'aref
  :size-fn \#'(lambda (v) (array-dimension v 0))
  :sequence-type 'vector
  :element-type t
  :element-doc-string 
     "Elements of a vector, disregarding fill-pointer"
  :index-doc-string 
     "Indices of vector, disregarding fill-pointer")
\end{program}

\end{clauses}

\subsection{Subtleties}

There are some subtleties to be aware of when writing \iter\ clauses.
First, the code returned by your macros may be |nconc|'ed into a list,
so you should always returned freshly consed lists, rather than
constants.  Second, \iter\ matches clauses by using |eq| on the first
symbol and |string=| on the subsequent ones, so the package of the
first symbol of a clause is relevant.  All of the clauses in this manual
have their first word in the \iter\ package.  
You can use the package system in the usual way to shadow
\iter\ clauses without replacing them.

%%% say more here, about the badness that only the first word of a
%%% clause is packagey.

\section{Non-portable Extensions to Iterate (Contribs)}
\label{contribs}

Currently, there is only one non-portable extension to iterate in the
distribution: iterate-pg. If you have made an extension that depends
on non-portable features, feel free to send them to |asf@boinkor.net|
for inclusion in the iterate distribution.

\subsection{An SQL query driver for iterate}

The pg package by Eric Marsden (see |http://cliki.net/pg|) provides an
interface to the PostgreSQL database. Using the \iterpg\ extension, it
is possible to handle the results of SQL queries using \iter.

This usage example should give you an idea of how to use it:

\begin{program}
(pg:with-pg-connection (c "somedb" "someuser")
  (iter (for (impl version date) in-relation "select * from version"
                                 on-connection *dbconn*)
        (collect version)))
\end{program}

To use the extension via |ASDF|, simply make your system depend on the
|iterate-pg| system instead of the |iterate| system. To load it
manually, use:

\begin{program}
  (asdf:oos 'asdf:load-op :iterate-pg)
\end{program}

%%%%%%%%%%%%%%%%%%%%%%%%%%%%%%%%%%%%%%%%%%%%%%%%%%%%%%%%%%%%%%%%
\section{Obtaining {\tt Iterate}}

\begin{sloppypar}
The information in this chapter is obsolete but included for
completeness's sake; Currently, the most up-to-date information on
\iter\ can be found at |http://boinkor.net/iterate.html|.
\end{sloppypar}

\begin{sloppypar}
\iter\ currently runs on Lisp Machines, and on
HP's, Sun3's and Sparcstations under Lucid.  
\iter\ source and binaries are available at the MIT AI Lab in the
subdirectories of |/src/local/lisplib/|.  The source file,
|iterate.lisp|, is also available for anonymous FTP in the directory
|/com/fpt/pub/| on the machine |TRIX.AI.MIT.EDU| (Internet number
128.52.32.6).  If you are unable to obtain |iterate| in one of these
ways, send mail to |jba@ai.mit.edu| and I will send you the source
file. 
\end{sloppypar}

\begin{sloppypar}
\iter\ resides in the |iterate| package (nickname |iter|).  Just say
\linebreak |(use-package :iterate)| to make all the necessary symbols
available. 
If a symbol is not exported, it appears in this manual with an
``|iterate::|'' prefix.
\end{sloppypar}

Send bug reports to |bug-iterate@ai.mit.edu|.  The |info-iterate|
mailing list will have notices of changes and problems; to have
yourself added, send mail to |info-iterate-request@ai.mit.edu|.


\medskip

\begin{flushleft}
 \bf Acknowledgements
\end{flushleft}
\smallskip

Richard Waters provided invaluable criticism which spurred me to improve
\iter\ greatly.  As early users, David Clemens, Oren Etzioni and Jeff
Siskind helped ferret out many bugs.

%\begin{theindex}
% The index files must be generated with the genindex program
% or from makeindex iterate-manual.idx which creates iterate-manual.ind.

%\baselineskip\normalbaselineskip
%\advance\baselineskip by -2pt

\input{iterate-manual.ind}

%\underline{Clauses}
%\input{iterate-manual.clindex}
%
%\indexspace
%\underline{Lisp}
%\input{iterate-manual.lispindex}

%\end{theindex}


\end{document}


% arch-tag: "c0f871a6-313c-11d8-abb9-000c76244c24"


%\end{theindex}


\end{document}


% arch-tag: "c0f871a6-313c-11d8-abb9-000c76244c24"

%
%\indexspace
%\underline{Lisp}
%% Fix ``.'' spacing in lists
\documentstyle[12pt,aimemo]{article}

\setlength{\textwidth}{6.0in}
\setlength{\oddsidemargin}{0.25in}
\setlength{\evensidemargin}{0.25in}
\setlength{\topmargin}{0pt}
\setlength{\headsep}{24pt}
\setlength{\textheight}{8.5in}

\let\origpar=\par

% somehow, LaTex fucks with things so that the plain Tex \obeyspaces
% doesn't work.  This is the same, except we use an mbox containing
% a space instead of just a space.

\newcommand{\spc}{\mbox{ }}
\newcommand{\tAb}{\mbox{        }}
\def\obspaces{\catcode`\ =\active\catcode`\^^I=\active}
{\obspaces \global\let =\spc\global\let^^I=\tAb}

\newenvironment{program}{\medskip\samepage\tt\obeylines\obspaces}{\medskip}

% use | for \tt
\catcode`\|=13
\newif\ifvbar
\def|{\ifvbar
         \endgroup\vbarfalse
      \else 
         \begingroup\tt\vbartrue 
      \fi}



\newcommand{\lisp}{\tt}
\newcommand{\iter}{|iterate|}
\newcommand{\iterpg}{|iterate-pg|}
\newcommand{\setf}{|setf|}
\newcommand{\nil}{|nil|}
\newcommand{\nonnil}{non-\nil}
\newcommand{\opt}{{\lisp \&optional}}
\newcommand{\yields}{$\Rightarrow$}



\newbox\Dots
\setbox\Dots=\hbox{\small ...}
\renewcommand{\dots}{\copy\Dots}

% Here's a way to define an environment which will treat all
% paragraphs inside it in the same manner.  It's, as near as I can
% figure, the way that Latex does things.  Using \everypar won't work;
% apparently Latex resets that a lot.  What you have to do is: at the
% top of the file do \let\origpar=\par.  In the
% begin part of the environment, do the parshape (or whatever), then say
% \def\par{{\origpar}}.  The extra braces are necessary.   That's all
% you have to do; at the end of the environment things will return to
% normal automatically.  Why does this all work?  I have no idea.
% Another important thing: there should be a blank line (i.e. a \par)
% between the end of the last paragraph and the \end{environment}.

\newlength{\clindent}
\setlength{\clindent}{0.5in}
\newlength{\clparindent}
\setlength{\clparindent}{\parindent}

\newenvironment{clauses}{\advance\linewidth -\parindent
                        \hangindent\clindent\relax
                         \hangafter=0
                          \parindent=0pt
                          \def\par{{\origpar}}}{}

\newcommand{\cpar}{\hspace \clparindent}

\newcommand{\startitem}{\bigskip\pagebreak[3]}
\newcommand{\finishitem}{\smallskip}

\newcommand{\clause}[1]{\startitem\Clause{#1}\finishitem}
\newcommand{\Clause}[1]{\hbox{#1}}
\newcommand{\clindex}[1]{\index{{\lisp #1}}}
\newcommand{\clindexx}[2]{\index{{\lisp #1\dots #2}}}
\newcommand{\clindexxx}[3]{\index{{\lisp #1\dots #2\dots #3}}}
\newcommand{\clausex}[2]{\startitem\Clausex{#1}{#2}\finishitem}
\newcommand{\Clausex}[2]{\Clause{|#1| #2}\clindex{#1}}

\newcommand{\defvar}[1]{\startitem\deff{#1}{}{Variable}\finishitem}
\newcommand{\defunexpvar}[1]{\startitem\deffnoindex{iterate::#1}{}{Variable}
                                        \lispindex{#1}\finishitem}
\newcommand{\defun}[2]{\startitem\deff{#1}{#2}{Function}\finishitem}
\newcommand{\defmacro}[2]{\startitem\deff{#1}{#2}{Macro}\finishitem}
\newcommand{\defunx}[3]{\startitem\deffx{#1}{#2}{#3}{Function}\finishitem}
\newcommand{\defmacrox}[3]{\startitem\deffx{#1}{#2}{#3}{Macro}\finishitem}
\newcommand{\defunxx}[4]{\startitem\deffxx{#1}{#2}{#3}{#4}{Function}\finishitem}
\newcommand{\defmacroxx}[4]{\startitem\deffxx{#1}{#2}{#3}{#4}{Macro}\finishitem}

\newcommand{\deff}[3]{\deffnoindex{#1}{#2}{#3}\lispindex{#1}}
\newcommand{\deffnoindex}[3]{\hbox to \hsize{{\tt #1} {\it #2}\hfill [{\it #3}]}}
\newcommand{\deffx}[4]{\deff{#1}{#2}{#4}\moreargs{#1}{#3}}
\newcommand{\deffxx}[5]{\deffx{#1}{#2}{#3}{#5}\moreargs{#1}{#4}}

\newcommand{\moreargs}[2]{\hbox to \hsize{\phantom{\lisp #1} {\it #2}\hfill}}

\newcommand{\lispindex}[1]{\index{{\lisp #1}}}

 \newenvironment{note}[1]{\pagebreak[2]\bigskip
                          \hrule\smallskip\small
                         {\setlength{\parindent}{0pt}
                           \par{\bf #1:}}}{\smallskip\hrule\bigskip}



\makeindex

%%%%%%%%%%%%%%%%%%%%%%%%%%%%%%%%%%%%%%%%%%%%%%%%%%%%%%%%%%%%%%%%

%\thispagestyle{empty}

%\vspace*{2in}

%\title{The |Iterate| Manual}
%\author{Jonathan Amsterdam}

%\maketitle

%\thispagestyle{empty}

%\begin{center}
%\Large *** DRAFT ***
%\end{center}
%This manual is in beta-test. It will become an AI Memo in a couple of
%months.  I would appreciate any comments or
%corrections.  You can reach me at email address |jba@ai.ai.mit.edu|.

\begin{document}

\def\writtenby{Jonathan Amsterdam}
\def\contractno{This report describes research done at the Artificial
Intelligence Laboratory of the Massachusetts Institute of Technology.
Support for the laboratory's artificial intelligence research is
provided in part by the Advanced Research Projects Agency of the
Department of Defense under Office
of Naval Research contract N00014-85-K-0124.}

\AIMEMO{1236}{\Large The Iterate Manual \\[1in]}
{This is the manual for version 1.4 of \iter, a powerful iteration
macro for Common  
Lisp. \iter\ is similar to {\tt loop} but provides numerous additional
features, is well integrated with Lisp, and is extensible. 
\vfill }


% use ~ for \em
\catcode`\~=13
\newif\ifem
\def~{\ifem
         \/\endgroup\emfalse
      \else 
         \begingroup\em\emtrue 
      \fi}



%\pagebreak
%\setcounter{page}{1}

%\thispagestyle{empty}


\tableofcontents

\pagebreak
\pagestyle{headings}

\section{Introduction}

\begin{sloppypar}
This manual describes
\iter, a powerful iteration facility for Common Lisp.
\iter\ provides abstractions for many common iteration
patterns and allows for the definition of additional patterns.  \iter\ is
a macro that expands into ordinary Lisp at
compile-time, so it is more efficient than higher-order
functions like |map| and |reduce.|  
While it is similar to |loop|, \iter\ offers a more Lisp-like syntax
and enhanced extensibility.
(For a more complete
comparison of \iter\ with other iteration constructs, see MIT AI Lab Working
Paper No. 324, ~Don't Loop, Iterate.~)
\end{sloppypar}

An \iter\ form consists of the symbol |iter|\footnote{You can also use
|iterate|, but |iter| is preferred because it avoids potential conflicts with
possible future additions to Common Lisp, and because it saves
horizontal space when writing code.}
followed by one or more
forms, some of which may be \iter\ 
~clauses.~  Here is a simple example of \iter\ which collects the
numbers from 1 to 10 into a list, and returns the list.  The return
value is shown following the arrow.
\begin{program}
(iter (for i from 1 to 10)
      (collect i))          \yields (1 2 3 4 5 6 7 8 9 10)
\end{program}
This form contains two clauses: a |for| clause that steps the
variable |i| over the integers from 1 to 10, and a |collect|
clause that accumulates its argument into a list.  With a few
exceptions, all \iter\ clauses have the same format: alternating
symbols (called ~keywords~) and expressions (called
~arguments~).  The syntax and terminology are those of Common
Lisp's keyword lambda lists.  One difference is that \iter's keywords
do not have to begin with a colon---though they may, except for the
first symbol of a clause.  So you can also write  |(for i :from
1 :to 10)| if you prefer.

Any Lisp form can appear in the body of an \iter, where it will have
its usual meaning.  \iter\ walks the entire body, expanding macros,
and recognizing clauses at any level.  This example collects all the
odd numbers in a list:
\begin{program}
(iter (for el in list)
      (if (and (numberp el) (oddp el))
          (collect el)))
\end{program}

There are clauses for iterating over numbers, lists, arrays and other
objects, and for
collecting, summing, counting, maximizing and other useful operations.
\iter\ also supports the creation of new variable bindings, stepping
over multiple sequences at once, destructuring, and compiler
declarations of variable types.  The following example illustrates
some of these features: 
\begin{program}
(iter (for (key . item) in alist)
         (for i from 0)
         (declare (fixnum i))
         (collect (cons i key)))
\end{program}

This loop takes the keys of an alist and returns a new alist
associating the keys with their positions in the original list. The
compiler declaration for |i| will appear in the generated code
in the appropriate place.

%%%%%%%%%%%%%%%%%%%%%%%%%%%%%%%%%%%%%%%%%%%%%%%%%%%%%%%%%%%%%%%%
\section{Clauses}

Most of \iter's clauses will be familiar to |loop| programmers.  
(|loop| is an iteration macro that has been incorporated into Common
Lisp.  Seen Guy Steele's {\em Common Lisp, 2nd Edition}.)
In
nearly all cases they behave the same as their |loop| counterparts, so
a |loop| user can switch to \iter\ with little pain (and much gain).

All clauses with the standard keyword-argument syntax consist of two
parts: a ~required~ part, 
containing keywords that must be present and in the right order; and
an ~optional~ part, containing keywords that may be omitted and,
if present, may occur in any order.  In the descriptions below, the
parts are separated by the Lisp lambda-list keyword \opt.


\subsection{Drivers}

An iteration-driving clause
conceptually causes the iteration to go forward.  Driver clauses in
\iter\ allow iteration over numbers, lists, vectors, hashtables, packages,
files and streams.  Iteration-driving clauses must
appear at the top level of an \iter\ form; they cannot be nested
inside another clause.  The driver variable is updated at the point
where the driver clause occurs.  Before the clause is executed for the
first time, the value of the variable is undefined.

%Also, regardless of where the driver clause
%appears in the body, the driver variable is stepped at the top of the
%loop; hence it is stylistically preferable, though
%not required, to place driver clauses at the beginning of the \iter.

Multiple drivers may appear in a single \iter\ form, in which case all
of the driver variables are updated each time through the loop, in the
order in which the clauses appear.  The first driver to terminate will
terminate the entire loop.

In all cases, the value of the driver variable on exit from the loop,
including within the epilogue code (see the |finally| clause), is
undefined.

All the parameters of a driver clause are evaluated once, before the
loop begins.  Hence it is not possible to change the bounds or other
properties of an iteration by side-effect from within the loop.

With one exception, driver clauses begin with the word |for| (or
the synonym |as|) and mention an iteration variable, which is
given a binding within the \iter\ form.
The exception is |repeat|, which just executes a loop a
specified number of times:

\begin{clauses}

\clausex{repeat}{~n~}
Repeats the loop ~n~ times.  For example:
\begin{program}
(iter (repeat 100)
      (print "I will not talk in class."))
\end{program}
\cpar If $n \leq 0$, the
loop will never be executed.  If ~n~ is not an integer, the
actual number of executions will be $\lceil n \rceil$.

\end{clauses}

\pagebreak[3]

\subsubsection{Numerical Iteration}

\begin{clauses}
\clausex{for}{~var~ |\&sequence|} 
The general form for iterating over a sequence of numbers requires a
variable and, optionally, one or more keywords that provide the bounds
and step size of the iteration. The |\&sequence|
lambda-list keyword is a shorthand for these sequence keywords.  They are:
|from|, |upfrom|, |downfrom|, |to|, |downto|, |above|, |below| and |by|.
|from| provides the starting value for ~var~
and defaults to zero. |to| provides a final value and implies that the
successive values of ~var~ will be increasing; |downto| implies that
they will be decreasing.  The loop terminates when ~var~ passes
the final value (i.e. becomes smaller or larger than it, depending on
the direction of iteration); in other  words, the loop body
will never be 
executed for values of ~var~ past the final value.
|below| and |above| are similar to |to| and
|downto|, except that the loop terminates when ~var~ equals or passes
the final value.

\cpar If no final value is specified, the variable will be stepped
forever.  Using |from| or |upfrom| will result in increasing
values, while |downfrom| will give decreasing values.

\cpar On each iteration,
~var~ is incremented or decremented by the value of the sequence
keyword |by|, which defaults to 1.  It should always be a positive
number, even for downward iterations.

\cpar In the following examples, the sequence of numbers generated is shown
next to the clause.

\begin{program}
(for i upfrom 0) \yields\ 0 1 2 \ldots
(for i from 5)   \yields\ 5 6 7 \ldots    ; either from or upfrom is okay
(for i downfrom 0) \yields\ 0 -1 -2 \ldots
(for i from 1 to 3) \yields\ 1 2 3
(for i from 1 below 3) \yields\ 1 2
(for i from 1 to 3 by 2) \yields\ 1 3
(for i from 1 below 3 by 2) \yields\ 1
(for i from 5 downto 3) \yields\ 5 4 3
\end{program}

\end{clauses}

\subsubsection{Sequence Iteration}

There are a number of clauses for iterating over sequences.  In all of
them, the argument following |for| may be a list instead of a
symbol, in which case destructuring is performed.  See section
\ref{destructuring}. 

\begin{clauses}

\clause{|for| ~var~ |in| ~list~ \opt\ |by| ~step-function~}
\clindexx{for}{in}
~var~ is set to successive elements of list.  
~step-function,~ which 
defaults to |cdr|, is used to obtain the next sublist. 

\clause{|for| ~var~ |on| ~list~ \opt\ |by| ~step-function~}
\clindexx{for}{on}
~var~ is set to successive sublists of list. 
~step-function~ (default |cdr|) is used as in |for\dots in|.

\end{clauses}

\medskip

These two clauses use |atom| to test for the end
of a list.  Hence, given a list whose final |cdr| is not \nil,
they will silently ignore the last |cdr|.  Other choices are
|endp|,  which would signal an error, and |null|, which
would probably result in an error somewhere else.  If you wish to use
an end-test other than |atom|, set the variable
|iterate::*list-end-test*|\lispindex{*list-end-test*} to the name of
the desired function. 

\begin{clauses}
\clause{|for| ~var~ |in-vector| ~vector~ |\&sequence|}
\clindexx{for}{in-vector}
~var~ takes on successive elements from ~vector.~   The vector's
fill-pointer is observed.  Here and in subsequent clauses, the |\&sequence|
keywords include |with-index|, which takes a symbol as argument
and uses it for the index variable instead of an internally generated
symbol.  The other |\&sequence| keywords behave as in numerical
iteration, except that the default iteration bounds are the bounds of
the vector.  E.g. in |(for i in-vector v downto 3)|,
|i| will start off being bound to the last element in |v|,
and will  
be set to preceding elements down to and including the element with
index 3.

\clause{|for| ~var~ |in-sequence| ~seq~ |\&sequence|}
\clindexx{for}{in-sequence}
This uses Common Lisp's generalized sequence functions, |elt|
and |length|, to obtain elements and determine the length of
~seq.~   Hence it will work for any sequence, including lists, and will
observe the fill-pointers of vectors.

\clause{|for| ~var~ |in-string| ~string~ |\&sequence|}
\clindexx{for}{in-string}
~var~ is set to successive characters of ~string.~

\startitem
\Clause{|for| ~var~ |index-of-vector| ~vector~ |\&sequence|}
\clindexx{for}{index-of-vector}
\Clause{|for| ~var~ |index-of-sequence| ~sequence~ |\&sequence|}
\clindexx{for}{index-of-sequence}
\Clause{|for| ~var~ |index-of-string| ~string~ |\&sequence|}
\clindexx{for}{index-of-string}
\finishitem
~var~ is set to successive indices of the sequence.
These clauses avoid the overhead of accessing the sequence elements
for those applications where they do not need to be examined, or are
examined rarely.  They admit all the optional keywords of the other
sequence drivers except the (redundant) |with-index| keyword.

\clause{|for| ~(key value)~ |in-hashtable| ~table~}
\clindexx{for}{in-hashtable}

~key~ and ~value,~ which must appear as shown in a list
and may be destructuring templates, are set to the keys and values of
~table~.  If ~key~ is |nil|, then the hashtable's keys will be ignored;
similarly for ~value~.
The order in which 
elements of ~table~ will be retrieved is unpredictable.

\clause{|for| ~var~ |in-package| ~package~ \opt\ |external-only| ~ext~}
\clindexx{for}{in-package}
Iterates over all the symbols in ~package,~ or over only the
external symbols if ~ext~ is specified and non-|nil|.  ~ext~ is not
evaluated.  The same symbol may appear more than once.

\clause{|for| ~var~ |in-packages| ~packages~ \opt\ |having-access| ~symbol-types~}
\clindexx{for}{in-packages}
Iterates over all the symbols from the list of packages denoted by the
descriptor ~packages~ and having accessibility (or visibility) given by
~symbol-types~.  This defaults to the list |(:external :internal :inherited)|
and is not evaluated.  ~var~ must be a list of up to three variables: in each
iteration, these will be set to a symbol, its access-type and package (as per
|with-package-iterator| in ANSI-CL).  The same symbol may appear more than
once.

\end{clauses}

\begin{clauses}

\clause{|for| ~var~ |in-file| ~name~ \opt\ |using| ~reader~}
\clindexx{for}{in-file}
Opens the file ~name~ (which may be a string or pathname) for input,
and iterates 
over its contents. ~reader~ defaults to |read|, so by default ~var~
will be bound to the successive forms in the file.  The \iter\ body is
wrapped in an unwind-protect to ensure that the file is closed no
matter how the \iter\ is exited.

\clause{|for| ~var~ |in-stream| ~stream~ \opt\ |using| ~reader~}
\clindexx{for}{in-stream}
Like |for\dots in-file|, except that ~stream~ should be an existing
stream object that supports input operations.

\end{clauses}

\subsubsection{Generalized Drivers}

These are primarily useful for writing drivers that can also be used
as generators (see section \ref{generators}, below).
 
\begin{clauses}

\clause{|for| ~var~ |next| ~expr~}
\clindexx{for}{next}

~var~ is set to ~expr~ each time through the loop.  Destructuring is
performed.  When the clause is used as a generator, ~expr~ is the code
that is executed when |(next ~var~)| is 
encountered (see section \ref{generators}, below).
~expr~ should compute the first value for ~var~, as well
as all subsequent values, and is responsible for terminating the loop.
For compatibility with future versions of \iter, this termination
should be done with |terminate|\clindex{terminate}, which can be
considered a synonym for |finish| (see section \ref{control-flow}).

\cpar As an example, the following clauses are equivalent to |(for
i from 1 to 10)|:
\begin{program}
(initially (setq i 0))
(for i next (if (> i 10) (terminate) (incf i)))
\end{program}

\clause{|for| ~var~ |do-next| ~form~}
\clindexx{for}{do-next}
~form~ is evaluated each time through the loop.  Its value is ~not~
set to ~var~; that is ~form's~ job. ~var~ is only present so that
\iter\ knows it is a driver variable.  \linebreak |(for ~var~ next
~expr~)| is equivalent to |(for ~var~ do-next (dsetq ~var~ ~expr~))|.
(See section \ref{destructuring} for an explanation of |dsetq|.) 

\end{clauses}

\subsubsection{Generators}
\label{generators}

In all of the above clauses, the driver variable is updated on each
iteration.
Sometimes it is desirable to have greater control over updating.  
For instance, consider the problem of associating numbers, in
increasing order and with no gaps, with the
\nonnil\ elements of a list.  One obvious first pass at writing this is:

\begin{program}
(iter (for el in list)
      (for i upfrom 1)
      (if el (collect (cons el i))))
\end{program}
But on the list |(a b nil c)| this produces |((a . 1) (b . 2) (c .
4))| instead of the desired |((a . 1) (b . 2) (c . 3))|.  The problem
is that |i| is incremented each time through the loop, even when |el|
is |nil|.  

The problem could be solved elegantly if we could step |i| only when
we wished to.  This
can be accomplished for any \iter\ driver by writing |generate| (or
its synonym |generating|)
instead of |for|.  Doing so produces a ~generator~---a driver whose
values are yielded explicitly.  To obtain the next value of a
generator variable ~v~, write \linebreak |(next ~v~)|.  The value of a |next|
form is the next value of ~v~, as determined by its associated driver
clause.  |next| also has the side-effect of updating ~v~ to that
value.  If there is no next value, |next| will terminate the loop,
just as with a normal driver.


Using generators, we can now write our example like this:
\begin{program}
(iter (for el in list)
      (generate i upfrom 1)
      (if el (collect (cons el (next i)))))
\end{program}
Now |i| is updated only when |(next i)| is executed, and this occurs
only when |el| is \nonnil.

To better understand the relationship between ordinary drivers and
generators, observe that we can rewrite an ordinary driver using its
generator form immediately followed by |next|, as this example shows:
\begin{program}
(iter (generating i from 1 to 10)
      (next i)
      ...)
\end{program}
Provided that the loop body contains no |(next i)| forms, this will
behave just as if we had written |(for i from 1 to 10)|.

We can still refer to a driver variable ~v~ without using |next|; in
this case, its value is that given to it by the last evaluation of
|(next ~v~)|.  Before |(next ~v~)| has been called the first time, the
value of ~v~ is undefined.  

This semantics is more flexible than
one in which ~v~ begins the loop bound to its first value and calls
of |next| supply subsequent values, because it means the loop will not
terminate too soon if the generator's sequence is empty.  For
instance, consider the following code, which tags \nonnil\ elements of
a list using a list of tags, and also counts the null elements.
(We assume there are at least as many tags as \nonnil\ elements.)
\begin{program}
(let* ((counter 0)
       (tagged-list (iter (for el in list)
                          (generating tag in tag-list) 
                          (if (null el)
                              (incf counter)
                              (collect (cons el (next tag)))))))
  ...)
\end{program}
It may be that there are just as many tags as non-null elements of
|list|.  If all the elements of |list| are null, we still want the
counting to proceed, even though |tag-list| is \nil.  If |tag| had to be
assigned its first value before the loop begins, we would have had to
terminate the loop before the first iteration, since when |tag-list|
is \nil, |tag| 
has no first value.  With the existing semantics, however, |(next
tag)| will never execute, so the iteration will cover all the elements of
|list|. 
        
When the ``variable'' of a driver clause is actually a destructuring
template containing several variables, all the variables are eligible
for use with |next|.  As before, |(next ~v~)| evaluates to ~v's~ next
value; but the effect is to update all of the template's variables.
For instance, the following code will return the list |(a 2 c)|.
\begin{program}
(iter (generating (key . item) in '((a . 1) (b . 2) (c . 3)))
      (collect (next key))
      (collect (next item)))
\end{program}

Only driver clauses with variables can be made into generators.  This
includes all clauses mentioned so far except for |repeat|.  It does
~not~ include |for\dots previous|, |for\dots =|,
|for\dots initially\dots then| or |for\dots first\dots then| (see
below). 

\subsubsection{Previous Values of Driver Variables}

Often one would like to access the value of a variable on a previous
iteration.  \iter\ provides a special clause for accomplishing this.

\begin{clauses}

\clause{|for| ~pvar~ |previous| ~var~ \opt\ |initially| ~init~
|back| ~n~}
\clindexx{for}{previous}
Sets ~pvar~ to the previous value of ~var,~ which should be a driver
variable, a variable from another |for\dots previous| clause, or a
variable established by a |for\dots =|, 
|for\dots initially\dots then| or |for\dots first\dots then| clause
(see section \ref{setting}).
Initially, ~pvar~ is given the value ~init~ (which defaults to \nil).  
The ~init~ expression will be moved outside the loop body, so it
should not depend on anything computed within the loop.
~pvar~ retains the value of ~init~ until ~var~ is set to its second
value, at which point ~pvar~ is set to ~var's~ first value; and so on.  

\cpar The
argument ~n~ to |back| must be a constant, positive integer, and
defaults to 1.  It determines how many iterations back ~pvar~ should
track ~var~.  For example, when ~n~ is 2, then ~pvar~ will be assigned
~var's~ first value when ~var~ is set to its third value.

\cpar A |for\dots previous| clause may occur after or before its
associated driver clause. |for\dots previous| works with generators as
well as ordinary drivers.

\pagebreak[3]

\cpar Example:
\begin{program}
(iter (for el in '(1 2 3 4))
      (for p-el previous el)
      (for pp-el previous p-el initially 0)
      (collect pp-el))
\end{program}
This evaluates to |(0 0 1 2)|.  It could have been written more
economically as 
\begin{program}
(iter (for el in '(1 2 3 4))
      (for pp-el previous el back 2 initially 0)
      (collect pp-el))
\end{program}


\end{clauses}

\subsection{Variable Binding and Setting}
\label{setting}

Several clauses exist for establishing new variable bindings or for
setting variables in the loop.  They all support destructuring.

\begin{clauses}

\clausex{with}{~var~ \opt\ |=| ~value~}
Causes ~var~ to be bound to value before the loop body is entered.
If ~value~ is not supplied, ~var~ assumes a default
binding, which will be 
\nil\ in the absence of declarations.  Also, if ~value~ is not
supplied, no destructuring is performed; instead, ~var~ may be a list of
symbols, all of which are given default 
bindings.  If ~value~ is supplied, ~var~ is bound to it, with
destructuring.  

\cpar Because |with|
creates bindings whose scope includes the entire \iter\ form, it is
good style to put all |with| clauses at the beginning.

\cpar Successive occurrences of |with| result in sequential
bindings (as with 
|let*|).  There is no way to obtain parallel bindings; see
section \ref{bindings} for a rationale.  


\clause{|for| ~var~ |=| ~expr~}
\clindexx{for}{=}
On each iteration, ~expr~ is evaluated and ~var~ is set
to its value. 

\cpar This clause may appear to do the same thing as |for\dots next|.
In fact, they are quite different.  |for\dots =| provides only three
services: it sets up a binding for ~var~, sets it to ~expr~ on each
iteration, and makes it possible to use |for\dots previous| with
~var~.  |for\dots next| provides these services in addition to the
ability to turn the driver into a generator.
%Also, the code which sets ~var~ appears in the loop body in the same
%place as the |for\dots =| clause; the code for |for\dots next| appears
%at the top of the loop, as with other drivers (except when being used
%as a generator).

\clause{|for| ~var~ |initially| ~init-expr~ |then| ~then-expr~}
\clindexxx{for}{initially}{then}
Before the loop begins, ~var~ is set to ~init-expr;~ on all
iterations after the first it is set to ~then-expr.~  This clause
must occur at top-level.  ~init-expr~ will be moved outside the loop
body and ~then-expr~ will be moved to the end of the loop body, so
they are subject to code motion problems (see section
\ref{code-movement}). 

\cpar This clause may appear to be similar to |for\dots next|, but in
fact they differ significantly.  |for\dots initially\dots then| is
typically used to give ~var~ its first value before the loop begins,
and subsequent values on following iterations.  This is incompatible
with generators, whose first value and subsequent values must all be
computed by |(next ~var~)|.  Also, the update of ~var~ in 
|for\dots initially\dots then| does not occur at the location of the
clause. 
Use |for\dots initially\dots then| for
one-shot computations where its idiom is more convenient, but use
|for\dots next| for extending \iter\ with new drivers (see section
\ref{extend}).

\clause{|for| ~var~ |first| ~first-expr~ |then| ~then-expr~}
\clindexxx{for}{first}{then}
The first time through the loop, ~var~ is set to ~first-expr;~ on
subsequent iterations, it is set to ~then-expr.~  This differs from
|for\dots initially| in that ~var~ is set to ~first-expr~
inside the loop body, 
so ~first-expr~ may depend on the results of other clauses.  For
instance, 
\begin{program}
(iter (for num in list)
      (for i first num then (1+ i))
      ...)
\end{program}
will set |i| to the first element of |list| on the first
iteration, whereas 
\begin{program}
(iter (for num in list)
      (for i initially num then (1+ i))
      ...)
\end{program}
is probably erroneous; |i| will be bound to |num|'s
default binding (usually \nil) for the first iteration.

\end{clauses}

\begin{note}{Compatibility note}
|loop|'s |for\dots =| works like \iter's, but |loop| used the syntax
|for\dots =\dots then| to mean |for\dots initially\dots then|.  It was
felt that these two operations were sufficiently different to warrant
different keywords.

Also, the |for| in the above three clauses is misleading,
since none is true driver (e.g. none has a corresponding |generate|
form).  |setting| would have been a better choice, but |for| was used
to retain some compatibility with |loop|.  
\end{note}


%%%%%%%%%%%%%%%%%%%%%%%%%%%%%%%%
\subsection{Gathering Clauses}
Many of \iter's clauses accumulate values into a variable, or set a
variable under certain conditions.  At the end of the loop, this
variable contains the desired result.  All these clauses have an
optional |into| keyword, whose argument should be a symbol.  If the
|into| keyword is not supplied, the accumulation variable will be
internally generated and its value will be returned at the end of the
loop; if a variable is specified, that variable is used for the
accumulation, and is not returned as a result---it is up to the user
to return it explicitly, in the loop's epilogue code (see |finally|).
It is safe to examine the accumulation variable during the loop, but
it should not be modified.

These clauses all begin with a verb.  When the verb does
not conflict with an existing Common Lisp function, then it may be
used in either its infinitival or present-participle form (e.g. |sum|,
|summing|).  However, when there is a conflict with Common Lisp, only
the present-participle form may be used (e.g. |unioning|).  This is to
prevent \iter\ clauses from clashing with Common Lisp
functions. 

%although these clauses are described as ``producing a value,'' it is a
%mistake to think of the lisp list representing the clause as a
%value-producing form in the usual way.  clauses may legally be written
%where a value is expected, e.g. |(setq x (sum i))|, but the
%lisp value of a clause in such a context is undefined.

\subsubsection{Reductions}

~Reduction~ is an extremely common iteration pattern in which
the results of successive applications of a
binary operation are accumulated.
For example, a loop that computes the sum of the
elements of a list is performing a reduction with the addition
operation.  This could be written in Common Lisp as  |(reduce \#'+
list)| or with \iter\ as
\begin{program}
(iter (for el in list)
      (sum el))
\end{program}

\begin{clauses}

\clausex{sum}{~expr~ \opt\ |into| ~var~}
Each time through the loop, ~expr~ is evaluated and added to a
variable, which is bound initially to zero.  If ~expr~ has a type,
it is ~not~ used as the type of the sum variable, which is always
|number|.  To get the result variable to be of a more specific
type, use an explicit variable, as in
\begin{program}
(iter (for el in number-list)
      (sum el into x)
      (declare (fixnum x))
      (finally (return x)))
\end{program}

\clausex{multiply}{~expr~ \opt\ |into| ~var~}
Like |sum|, but the initial value of the result variable is $1$,
and the variable is updated by multiplying ~expr~ into it.

\clausex{counting}{~expr~ \opt\ |into| ~var~}
~expr~ is evaluated on each iteration.  If it is \nonnil, the
accumulation variable, initially zero, is incremented.

\startitem
\Clausex{maximize}{~expr~ \opt\ |into| ~var~}
\Clausex{minimize}{~expr~ \opt\ |into| ~var~}
\finishitem
~expr~ is evaluated on each iteration and its extremum (maximum or
minimum) is stored in the accumulation variable.  If ~expr~ is never
evaluated, then the result is \nil\ (if the accumulation variable is
untyped) or $0$ (if it has a numeric type).

\clausex{reducing}{~expr~ |by| ~func~ \opt\ |initial-value| ~init-val~ |into| ~var~}
This is a general way to perform reductions. ~func~ should be a
function of two arguments, the first of which will be the value
computed so far and
the second of which will be the value of ~expr~.  It should return the
new value.
|reducing| is roughly equivalent to the
Common Lisp |(reduce ~func~ ~sequence~ :key ~expr-function~)|, where
~expr-function~ is used to derive values from the successive elements
of ~sequence~.

\cpar If the |reducing| clause is never executed, the result is
undefined.

\cpar It is not necessary to provide an
initial value, but better code can be generated if one is supplied.
Regardless of its location in the \iter\ body, ~init-val~ will be
evaluated before the loop is entered, so it should not depend on any
value computed inside the \iter\ form.  

%\cpar if a ~var~ is not specified, you can get \iter\ to declare the
%type of the internal variable by putting a |the| expression around
%~func.~  see section \ref{types}.

\end{clauses}

%\begin{note}{implementation note}
%in principle, |maximize| and |minimize| can be thought of
%as reductions where the initial value is the smallest (or largest)
%value that the accumulation variable can assume.  because lisp's
%bignums can represent arbitrary integers, these clauses cannot be
%implemented as reductions in general.  if, however, the type of 
%~expr~ or ~var~ can be determined to be a fixnum
%or a float, \iter\ will implement the clause as a true reduction,
%using one of the constants |most-negative-fixnum|, |%most-positive-fixnum|, 
%|most-negative-short-float|, etc. as appropriate.
%\end{note}

\subsubsection{Accumulations}

All the predefined accumulation clauses add values to a sequence.  If
the sequence is a list, they all 
have the property that the partial list is kept in the correct order
and available for inspection at any point in the loop.

\begin{clauses}

\clausex{collect}{~expr~ \opt\ |into| ~var~ |at| ~place~ |result-type| ~type~}
Produces a sequence of the values of ~expr~ on each iteration. ~place~
indicates where the next value of ~expr~ is added to the list and may
be one of the symbols |start|, |beginning| (a synonym for |start|) or
|end|.  The symbol may be quoted, but need not be.  The default is
|end|.  For example,
\begin{program}
(iter (for i from 1 to 5)
      (collect i))
\end{program}
produces |(1 2 3 4 5)|, whereas
\begin{program}
(iter (for i from 1 to 5)
      (collect i at beginning))
\end{program}
produces |(5 4 3 2 1)| (and is likely to be faster in most Common Lisp
implementations).

\cpar If ~type~ is provided, it should be a subtype of |sequence|.
The default is |list|.  Specifying a type other than |list| will
result in |collect| returning a sequence of that type.  ~However,~ the
type of the sequence being constructed when inside the loop body is
undefined when a non-|list| type is specified.  (As with ~place~,
quoting ~type~ is optional.)

\clausex{adjoining}{~expr~ \opt\ |into| ~var~ |test| ~test~ |at| ~place~ |result-type| ~type~}
Like |collect|, but only adds the value of ~expr~ if it is not
already present.  ~test,~ which defaults to |\#'eql|, is
the test to be used with |member|.

\startitem
\Clausex{appending}{~expr~ \opt\ |into| ~var~ |at| ~place~}
\Clausex{nconcing}{\ ~expr~ \opt\ |into| ~var~ |at| ~place~}
\Clausex{unioning}{\ ~expr~ \opt\ |into| ~var~ |test| ~test~ |at| ~place~}
\Clausex{nunioning}{~expr~ \opt\ |into| ~var~ |test| ~test~ |at| ~place~}
\finishitem
These are like |collect|, but behave like the Common Lisp functions
|append|, |nconc|, |union| or |nunion|.  
As in Common Lisp, they work only on lists.  Also as in Common Lisp,
|unioning| and |nunioning| assume that the value of ~expr~ contains no
duplicates. 

\clausex{accumulate}{~expr~ |by| ~func~ \opt\ |initial-value| ~init-val~ |into| ~var~}
This is a general-purpose accumulation clause. ~func~ should be a
function of two arguments, the value of ~expr~ and the value
accumulated so far in the iteration, and it should return the updated
value.  If no initial value is supplied, \nil\ is used.  

%\cpar If a ~var~ is not specified, you can get \iter\ to declare the
%type of the internal variable by putting a |the| expression around
%~func.~  see section \ref{types}.

\cpar The differences between |accumulate| and |reducing| are slight.
One difference is that the functions take their arguments in a
different order.  Another is that in the absence of ~init-val~,
|accumulate| will use \nil, whereas |reducing| will generate different
code that avoids any dependence on the initial value.
The reason for having both clauses is that one usually
thinks of reductions (like |sum|) and accumulations (like |collect|)
as different beasts.

\end{clauses}


\subsubsection{Finders}

A ~finder~ is a clause whose value is an expression that meets some
condition.

\begin{clauses}

\clause{|finding| ~expr~ |such-that| ~test~ \opt\ |into| ~var~ |on-failure| ~failure-value~}
\clindexx{finding}{such-that}
If ~test~ (which is an expression) ever evaluates to \nonnil, the loop
is terminated, the 
epilogue code is run and the value of ~expr~ is returned.  Otherwise,
\nil\ (or ~failure-value,~ if provided) is returned.  If ~var~ is
provided, it will have either the \nonnil\ value of ~expr~ or
~failure-value~ when the epilogue code is run.

\cpar As a special case, if the ~test~ expression is a sharp-quoted
function, then it is applied to ~expr~ instead of being simply
evaluated.  E.g. |(finding x such-that \#'evenp)| is equivalent to
|(finding x such-that (evenp x))|.  

%\cpar although ~test~ need have nothing to do with ~%expr~ as in
%|(finding j such-that (> i 3))|, it usually 
%will: |(finding (length el) such-that (oddp (length el)))|.  to
%avoid performing the |length| computation twice, you could write
%|(finding (length el) such-that \#'oddp)| or |(finding (length
%el) such-that 'oddp)|; for these cases, \iter\ generates code that
%executes ~expr~ only once.  the code for |\#'oddp|
%is slightly different from that for {\lisp 'oddp}; see the discussion
%under {\lisp for\dots in} and {\lisp for\dots on}.

\cpar |On-failure| is a misnomer. Because it is always evaluated, it behaves
more like the default third argument to the |gethash| function. As a result,
|on-failure (error "Not found")| makes no sense. Instead, the clauses |leave|
or |thereis| can be used in conjunction with |finally| as follows:
\begin{program}
(iter (for x in '(1 2 3))
      (if (evenp x) (leave x))
      (finally (error "not found")))
\end{program}

\cpar This clause may appear multiple times when all defaults are
identical. It can also be used together with either |always|/|never| or
|thereis| if their defaults match. More specifically, |on-failure nil| is
compatible with |thereis|, while |on-failure t| is compatible with |always|
and |never| clauses.
\begin{program}
(iter (for i in '(7 -4 2 -3))
      (if (plusp i)
	  (finding    i  such-that (evenp i))
          (finding (- i) such-that (oddp i))))
\end{program}

\startitem
\Clause{|finding| ~expr~ |maximizing| ~m-expr~ \opt\ |into| ~var~}
\clindexx{finding}{maximizing}
\Clause{|finding| ~expr~ |minimizing| ~m-expr~ \opt\ |into| ~var~}
\clindexx{finding}{minimizing}
\finishitem
Computes the extremum (maximum or minimum) value of ~m-expr~ over all
iterations, and returns the value of ~expr~ corresponding to the
extremum.  ~expr~ is evaluated inside the loop at the time the new
extremum is established.  If ~m-expr~ is never evaluated (due to, for
example, being embedded in a conditional clause), then the returned
value depends on the type, if any, of ~expr~ (or ~var,~ if
one is supplied).  If there is no type, the returned
value will be \nil; if the type is numeric, the returned value will be
zero.

\cpar For these two clauses, ~var~ may be a list of two
symbols; in that case, the first is used to record ~expr~ and
the second, ~m-expr.~  

\cpar As with |finding\dots such-that|, if ~m-expr~ is a sharp-quoted
function, then it is called on ~expr~ instead of being evaluated.

\end{clauses}

\subsubsection{Boolean Tests}

\begin{clauses}

\clausex{first-iteration-p}{}
Returns |t| in the first cycle of the loop, otherwise \nil.

\clausex{first-time-p}{}
Returns |t| the first time the expression is evaluated, and then \nil\ forever.
This clause comes handy when printing (optional) elements separated
by a comma:

\begin{program}
(iter (for el in '(nil 1 2 nil 3))
      (when el
        (unless (first-time-p)
          (princ ", "))
        (princ el)))
\end{program}
produces |"1, 2, 3"|.

\end{clauses}

\subsubsection{Aggregated Boolean Tests}

\begin{clauses}

\clausex{always}{~expr~}
If ~expr~ ever evaluates to
\nil, then \nil\ is immediately returned; the epilogue code is not
executed.  If ~expr~ never evaluates to \nil, the epilogue code
is executed and the last value of ~expr~ (or |t| if ~expr~ was never
evaluated) is returned (whereas |loop| would constantly return |t|).

% mention last evaluated clause when multiple always clauses?

\clausex{never}{~expr~}
Like |(always (not ~expr~))|, except it does not influence the last
value returned by a possible other |always| clause. That is,
\begin{program}
(iter (repeat 2)
      (always 2)
      (never nil)) \yields 2 ; not t
\end{program}

\clausex{thereis}{~expr~}
If ~expr~ is ever \nonnil,
its value is immediately returned without running epilogue code.
Otherwise, the epilogue code is performed and \nil\ is returned.

This clause cannot be used together with |always| or |never|, because their
defaults are opposed (similarly, |(loop always 3 thereis nil)| refuses to
compile in some implementations of |loop|).

\end{clauses}


\subsection{Control Flow}
\label{control-flow}
Several clauses can be used to alter the usual flow of control in a loop.

Note: the clauses of this and subsequent sections don't adhere to \iter's
usual syntax, but instead use standard Common Lisp syntax.  Hence the
format for describing syntax subsequently is
like the standard format used in the Common Lisp manual, not like the
descriptions of clauses above.

\begin{clauses}

\clausex{finish}{}
Stops the loop and runs the epilogue code.

%for example:
%
%\begin{program}
%(iter (with answer = nil)
%         (initially (make-a-mess))
%         (for i from 1 to 10)
%         (when (correct? i)
%           (setq answer i)
%           (finish))
%         (finally (cleanup)))
%\end{program}
%
%this code will execute |cleanup| whether or not the test |(correct?
%i)| ever succeeds. 
%the (more elegant) formulation,
%\begin{program}
%(iter (initially (make-a-mess))
%         (for i from 1 to 10)
%         (finding i such-that (correct? i))
%         (finally (cleanup)))
%\end{program}
%would not execute |cleanup| if |(correct? i)| succeeded; it
%would do an immediate return.

\clausex{leave}{\opt\ ~value~}
Immediately returns ~value~ (default \nil) from the current \iter\
form, skipping the epilogue code.  Equivalent to using |return-from|.

\clausex{next-iteration}{}
Skips the remainder of the loop body and begins the next iteration of
the loop.

\clausex{while}{~expr~}
If ~expr~ ever evaluates to \nil, the loop is terminated and the
epilogue code executed.  Equivalent to |(if (not ~expr~) (finish))|.

\clausex{until}{~expr~}
Equivalent to |(if ~expr~ (finish))|.

\clausex{if-first-time}{~then~ \opt\ ~else~}
If this clause is being executed for the first time in this invocation
of the \iter\ form, then the ~then~ code is evaluated; otherwise the
~else~ code is evaluated.

\cpar |(for ~var~ first ~expr1~ then ~expr2~)| is almost equivalent to
\begin{program}
(if-first-time (dsetq ~var~ ~expr1~)
               (dsetq ~var~ ~expr2~))
\end{program}
The only difference is that the |for| version makes ~var~ available
for use with |for\dots previous|.

\end{clauses}

\subsection{Code Placement}
When fine control is desired over where code appears in a loop
generated by \iter, the following special clauses may be useful.
They are all subject to code-motion problems (see section
\ref{code-movement}).  

\begin{clauses}

\clausex{initially}{|\&rest| ~forms~}
The lisp ~forms~ are placed in the prologue section of the loop, where
they are executed once, before the loop body is entered.

\clausex{after-each}{|\&rest| ~forms~}
The ~forms~ are placed at the end of the loop body, where they
are executed after each iteration.  Unlike the other clauses in this
section, ~forms~ may contain \iter\ clauses.

\clausex{else}{|\&rest| ~forms~}
The lisp ~forms~ are placed in the epilogue section of the loop, where they
are executed if this |else| clause is never met during execution of the
loop and the loop terminates normally.

\clausex{finally}{|\&rest| ~forms~}
The lisp ~forms~ are placed in the epilogue section of the loop, where
they are executed after the loop has terminated normally.

\clausex{finally-protected}{|\&rest| ~forms~}
The lisp ~forms~ are placed in the second form of an unwind-protect
outside the loop.  They are always executed after the loop has
terminated, regardless of how the termination occurred.

\end{clauses}


%%%%%%%%%%%%%%%%%%%%%%%%%%%%%%%%%%%%%%%%%%%%%%%%%%%%%%%%%%%%%%%%
\section{Other Features}

\subsection{Multiple Accumulations}
\label{multiple}

\begin{sloppypar}
It is permitted to have more than one clause accumulate into the same
variable, as in the following:
\begin{program}
(iter (for i from 1 to 10)
      (collect i into nums)
      (collect (sqrt i) into nums)
      (finally (return nums)))
\end{program}
Clauses can only accumulate into the same variable if they are
compatible.  |collect|, |adjoining|, |appending|, |nconcing|,
|unioning| and |nunioning| are compatible with each other; |sum|,
|multiply| and |counting| are compatible; |always| and |never| are compatible;
|finding| \dots |such-that| is compatible with either |thereis| or |always|
and |never| when their defaults match; and |maximize| and |minimize| clauses
are compatible only with other |maximize| and |minimize| clauses,
respectively.

%note that the same variable ~cannot~ be both an accumulation
%variable and an ordinary variable; there can be only one variable with
%a given name within an \iter\ form. 
\end{sloppypar}

\subsection{Named Blocks}

Like Common Lisp |block|s, \iter\ forms can be given names.  The
name should be a single symbol, and it must be the first form in the
\iter.  The generated code behaves exactly like a named block; in
particular, |(return-from ~name~)| can be used to exit it:
\begin{program}
(iter fred
      (for i from 1 to 10)
      (iter barney
            (for j from i to 10)
            (if (> (* i j) 17)
                (return-from fred j))))
\end{program}
An \iter\ form that is not given a name is implicitly named \nil.

Sometimes one would like to write an expression in an inner \iter\ form,
but have it processed by an outer \iter\ form.  This is possible with
the |in| clause.  

\begin{clauses}

\clausex{in}{~name~ |\&rest| ~forms~}
Evaluates ~forms~ as if they were part of the \iter\ form named
~name~.  In other words, \iter\ clauses are processed by the \iter\
form named ~name,~ and not by any \iter\ forms that occur inside ~name.~

\cpar As an example, consider the problem of collecting a list of the
elements in a two-dimensional array.  The naive solution,
\begin{program}
(iter (for i below (array-dimension ar 0))
      (iter (for j below (array-dimension ar 1))
            (collect (aref ar i j))))
\end{program}
\noindent is wrong because the list created by the inner \iter\ is simply
ignored by the outer one.  But using |in| we can write:
\begin{program}
(iter outer (for i below (array-dimension ar 0))
      (iter (for j below (array-dimension ar 1))
            (in outer (collect (aref ar i j)))))
\end{program}
\noindent which has the desired result.

\end{clauses}

\subsection{Destructuring}
\label{destructuring}

In many places within \iter\ clauses where a variable is expected, a
list can be written instead.  In these cases, the value to be assigned
is ~destructured~ according to the pattern
described by the list.  As a simple example, the clause
\begin{program}
(for (key . item) in alist)
\end{program}
\noindent will result in |key| being set to the |car| of
each element in |alist|, and |item| being set to the |cdr|.  The
pattern list may be nested to arbitrary depth, and (as the example
shows) need not be terminated with \nil; the only requirement is that
each leaf be a bindable symbol (or \nil, in which case no binding is
generated for that piece of the structure).

Sometimes, you might like to do the equivalent of a
|multiple-value-setq| in a clause.  This 
``multiple-value destructuring'' can be expressed by writing 
\linebreak |(values $pat_1$ $pat_2 \ldots$)| for a destructuring
pattern, as in 
\begin{program}
(for (values (a . b) c d) = (three-valued-function ...))
\end{program}
\begin{sloppypar}
Note that the $pat_i$ can themselves be destructuring patterns (though
not multiple-value destructuring patterns).  You can't do multiple-value
destructuring in a |with| clause; instead wrap the whole \iter\
form in a |multiple-value-bind|.
\end{sloppypar}

\begin{note}{Rationale}
There are subtle interactions between variable declarations and
evaluation order that make the correct implementation of
multiple-value destructuring in a |with| somewhat tricky.
\end{note}

The destructuring feature of \iter\ is available as a separate
mechanism, using the |dsetq| macro:

\begin{clauses}
\defmacro{dsetq}{template expr}
Performs destructuring of ~expr~ using ~template~.   May be used
outside of an \iter\ form. Yields the primary value of ~expr~.

\end{clauses}

\subsection{On-line Help}

\begin{sloppypar}
There is a limited facility for on-line help, in the form of the 
|display-iterate-clauses| function.
\end{sloppypar}

\begin{clauses}

\defun{display-iterate-clauses}{\opt\ clause-spec}
Displays a list of \iter\ clauses.  If ~clause-spec~ is not
provided, all clauses are shown; if it is a symbol, all clauses
beginning with that symbol are shown; and if it is a list of symbols,
all clauses for which ~clause-spec~ is a prefix are shown.

\end{clauses}

\subsection{Parallel Binding and Stepping}
\label{bindings}

The parallel binding and stepping of variables is a feature that
\iter\ does ~not~ have.  This section attempts to provide a rationale.

We say that two variables are bound ~in parallel~ if neither
binding shadows the other.  This is the usual semantics of |let|
(as opposed to |let*|).  Similarly, we can say that iteration
variables are stepped in parallel if neither variable is updated
before the other, conceptually speaking; in other words, if the code
to update each variable can reference the old values of both variables.

|loop| allows parallel binding of variables and parallel stepping of
driver variables.  My view is that if you are depending on the 
serial/parallel distinction, you are doing something obscure.  If you
need to bind 
variables in parallel using |with|, then you must be using a
variable name that shadows a name in the existing lexical environment.
Don't do that.  The most common use for parallel stepping is to track
the values of variables on the previous iteration, but in fact this
does not require parallel stepping at all; the following will work:
\begin{program}
(iter (for current in list)
      (for prev previous current)
      ...)
\end{program}


%%%%%%%%%%%%%%%%%%%%%%%%%%%%%%%%%%%%%%%%%%%%%%%%%%%%%%%%%%%%%%%%
\section{Types and Declarations}
\label{types}

\subsection{Discussion}

Sometimes efficiency dictates that the types of variables be declared.
This type information needs to be communicated to \iter\ so it can
bind variables to appropriate values.  Furthermore, \iter\ must often
generate internal variables invisible to the user; there needs to be a
way for these to be declared.

As an example, consider this code, which will return the number of
odd elements in |number-list|:
\begin{program}
(iter (for el in number-list)
      (count (oddp el)))
\end{program}
In processing this form,
\iter\ will create an internal variable, let us call it |list17|, to
hold the successive |cdr|s of |number-list|, and  
will bind the variable to |number-list|.  It will also generate a
default binding for |el|; only inside the body of the loop will |el|
be set to the |car| of |list17|.  Finally, \iter\ will generate a
variable, call it |result|, to hold the result of the count, and will
bind it to zero.

When dealing with type declarations, \iter\ observes one simple rule:
~it will never generate a declaration unless requested to do so.~  The
reason is that such declarations might mask errors in compiled code by
avoiding error-checks; the resulting problems would be doubly hard to
track down because the declarations would be hidden from the
programmer.  Of course, a compiler might omit error-checks even in the
absence of declarations, though this behavior can usually be avoided,
e.g. by saying |(declaim (optimize (safety 3)))|.

So, the above \iter\ form will generate code with no declarations.
But say we wish to declare the types of |el| and the internal
variables |list17| and |result|.  How is this done?

Declaring the type of |el| is easy, since the programmer knows
the variable's name:
\begin{program}
(iter (for el in number-list)
      (declare (fixnum el))
      (counting (oddp el)))
\end{program}
\iter\ can read variable type declarations like this one.  Before
processing any clauses, it scans the entire top-level form for type
declarations 
and records the types, so that variable bindings can be performed
correctly.  In this case, |el| will be bound to zero
instead of \nil.  Also, \iter\ collects all the top-level declarations
and puts them at the begining of the generated code, so it is not
necessary to place all declarations at the beginning of an \iter\
form; instead, they can be written near the variables whose types they
declare.  

Since \iter\ is not part of the compiler, it will not know
about declarations that occur outside an \iter\ form; these
declarations must be repeated inside the form.

Here is another way we could have declared the type of |el|:
\begin{program}
(iter (for (the fixnum el) in number-list)
      (counting (oddp el)))
\end{program}
\iter\ extends the Common Lisp |the|\lispindex{the}
form to apply to variables as well as value-producing forms; anywhere
a variable is allowed---in a |with| clause, as the iteration
variable in a driver clause, as the |into| argument of an
accumulation clause, even inside a destructuring template---you can
write |(the ~type~ ~symbol~)| instead.

There is one crucial difference between using a |the| form and
actually declaring the variable: explicit declarations are always
placed in the generated code, but type information from a |the|
form is not turned into an actual declaration unless you tell \iter\
to do so using |iterate:declare-variables|.  See below.

\begin{sloppypar}
Declaring the types of internal variables is harder than declaring the
types of explicitly mentioned variables, since their names
are unknown.  You do it by declaring |iterate:declare-variables|
somewhere inside the top level of the \iter\ form.  (This will also
generate declarations for variables declared using |the|.)
\iter\ does not provide much selectivity here: it's all or none. 
And unfortunately, since \iter\ is not privy to compiler information
but instead reads declarations itself, it will not hear if you
|(declaim (iterate:declare-variables))|.  Instead, set the variable 
|iterate::*always-declare-variables*| to |t| at
compile-time, using |eval-when|.
\end{sloppypar}

To determine the appropriate types for internal variables, \iter\ uses
three sources of information:

\begin{itemize}

\item Often, the particular clause dictates a certain type for a
variable; \iter\ will use this information when available.  In the
current example, the variable |list17| will be given the type
|list|, since that is the only type that makes sense; and the
variable |result| will be given the type |fixnum|, on the
assumption that you will not be counting high enough to need bignums.
You can override this assumption only by using and explicitly declaring a
variable: 
\begin{verbatim}
(iter (declare (iterate:declare-variables))
      (for el in number-list)
      (count (oddp el) into my-result)
      (declare (integer my-result))
      (finally (return my-result)))
\end{verbatim}

\begin{sloppypar}
Other examples of the type assumptions that \iter\ makes are: type
|list| for |into| variables of collection clauses; type |list| for
expressions that are to be destructured; type |vector| for the
variable holding the vector in a |for\dots in-vector| clause, and
similarly for |string| and the |for\dots in-string| clause;
and the implementation-dependent type
|(type-of array-dimension-limit)| for the index and limit
variables generated by sequence iteration drivers like |for\dots
in-vector| and |for\dots in-string| (but not |for\dots in-sequence|,
because it may be used to iterate over a list). 
\end{sloppypar}

\item Sometimes, \iter\ will examine expressions and try to determine
their types in a simple-minded way.  If the expression is
self-evaluating (like a number, for instance), \iter\ knows that the
expression's type is the same as the type of the value it denotes, so
it can use that type.  If the expression is of the form |(the ~type~
~expr~)|, \iter\ is smart enough to extract ~type~ and use it.
However, the current version of \iter\ does not  
examine declarations of function result types or do any type
inference.  It will not determine, for 
example, that the type of |(+ 3 4)| is |fixnum|, or even
|number|.

\item In some cases, the type of an internal variable should match the
type of some other variable.  For instance, \iter\ generates an
internal variable for |(f x)| in the
clause |(for i from 1 to (f x))|, and in the absence of other
information will give it the same type as |i|.  If, however, the
expression had been written |(the fixnum (f x))|, then \iter\
would have given the internal variable the type |fixnum|
regardless of |i|'s type.  The type incompatibility errors that
could arise in this situation are not checked for.

\end{itemize}

Note that if you do declare |iterate:declare-variables|, then
\iter\ may declare user variables as well as internal ones if they do
not already have declarations, though only for variables that it
binds.  For instance, in this code:

\begin{program}
(iter (declare (iterate:declare-variables))
      (for i from 1 to 10)
      (collect i into var))
\end{program}
the variable |var| will be declared to be of type |list|.


\subsection{Summary}

\iter\ understands standard Common Lisp variable type declarations
that occur within an \iter\ form and
will pass them through to the generated code.  If the declaration
|(iterate:declare-variables)|\lispindex{declare-variables}
appears at the top level of an 
\iter\ form, or if 
|iterate::*always-declare-variables*|\lispindex{*always-declare-variables*} 
is \nonnil, then \iter\ will use the type information gleaned from user
declarations, self-evaluating expressions and |the| expressions,
combined with reasonable assumptions, to determine variable
types and declare them.


%%%%%%%%%%%%%%%%%%%%%%%%%%%%%%%%%%%%%%%%%%%%%%%%%%%%%%%%%%%%%%%%
\section{Problems with Code Movement}
\label{code-movement}

Some \iter\ clauses, or parts of clauses, result in code being
moved from the location of the clause to other parts of the loop.
Drivers behave this way, as do code-placement clauses like |initially|
and |finally|.  When using these clauses, there is a danger of writing
an expression that makes sense in its apparent location but will be
invalid or have a different meaning in another location.  For example:
\begin{program}
(iter (for i from 1 to 10)
      (let ((x 3))
        (initially (setq x 4))))
\end{program}
While it may appear that the |x| of |(initially (setq x 4))| is the
same as the |x| of |(let ((x 3)) \dots|, in fact they are not:
|initially| moves its code outside the loop body, so |x| would refer
to a global variable.  Here is another example of the same problem:
\begin{program}
(iter (for i from 1 to 10)
      (let ((x 3))
        (collect i into x)))
\end{program}
If this code were executed, |collect| would create a binding for its
|x| at the top level of the \iter\ form that the |let| will shadow.

Happily, \iter\ is smart enough to catch these errors; it
walks all problematical code to ensure that free variables are not
bound inside the loop body, and checks all variables it binds for the
same problem.

However, some errors cannot be caught:

\begin{program}
(iter (with x = 3)
      (for el in list)
      (setq x 1)
      (reducing el by \#'+ initial-value x))
\end{program}
|reducing| moves its |initial-value| argument to the initialization
part of the loop in order to produce more efficient code.  Since
\iter\ does not perform data-flow analysis, it cannot determine that
|x| is changed inside the loop; all it can establish is that |x| is
not bound internally.  Hence this code will not signal an
error and will use $3$ as the initial value of the reduction.

The following list summarizes all cases that are subject to these code
motion and variable-shadowing problems.
\begin{itemize}
\item Any variable for which \iter\ creates a binding, including those
used in |with| and the |into| keyword of many clauses.

\begin{sloppypar}
\item The special clauses which place code: |initially|, |after-each|, |else|,
|finally| and |finally-protected|. 
\end{sloppypar}

\item The variables of a |next| or |do-next| form.

\item The |initially| arguments of |for\dots initially\dots then| and
|for\dots previous|. 

\item The |then| argument of |for\dots initially\dots then|.

\item The |initial-value| arguments of |reducing| and |accumulate|.

\item The |on-failure| argument of |finding\dots such-that|.
\end{itemize}

%%%%%%%%%%%%%%%%%%%%%%%%%%%%%%%%%%%%%%%%%%%%%%%%%%%%%%%%%%%%%%%%
\section{Differences Between {\tt Iterate} and {\tt Loop}}
|loop| contains a great deal of complexity which \iter\ tries to
avoid.  Hence many esoteric features of |loop| don't exist in \iter.
Other features have been carried over, but in a cleaned-up form.  
And of course, many new features have been added; they are not
mentioned in this list.

\begin{itemize}

\item \iter's syntax is more Lisp-like than |loop|'s, having a higher
density of parens.

\item The current implementation of \iter, unlike the current version
of  |loop| (as documented in {\em Common Lisp, 2nd Ed.\/}), is
extensible (see section \ref{extend}).  

\item |loop| puts the updates of all driver variables at the top of
the loop; \iter\ leaves them where the driver clauses appear.

\item While for the most part \iter\ clauses that resemble |loop| clauses
behave similarly, there are some differences.  For instance, there is
no |for\dots =\dots then| in \iter; instead use 
|for\dots initially\dots then|.  

\item |loop| binds the variable |it| at certain times
to allow pseudo-English expressions like |when ~expr~ return it|. 
In \iter, you must bind ~expr~ to a variable yourself.  Note that
|when ~expr~ return it| is like |thereis ~expr~| except that the latter is an
accumulation clause and therefore competes with other accumulations
(remember section \ref{multiple} above).

% repeat different behaviour of |always| clause here?

\item |loop| has a special |return| clause, illustrated in the
previous item.  \iter\ doesn't need one,  since an ordinary Lisp
|return| has the same effect.

\item |loop| allows for parallel binding and stepping of iteration
variables.  \iter\ does not.  (See section \ref{bindings}.)

\item |loop| and \iter\ handle variable type declarations very
differently.  |loop| provides a special syntax for declaring variable
types, and does not examine declarations.  Moreover, the standard
implementation of |loop| will
generate declarations when none are requested.
\iter\ parses standard Common Lisp type declarations, and will never
declare a variable itself unless declarations are 
specifically requested.

\end{itemize}

%%%%%%%%%%%%%%%%%%%%%%%%%%%%%%%%%%%%%%%%%%%%%%%%%%%%%%%%%%%%%%%%
\section{Rolling Your Own}
\label{extend}

\subsection{Introduction}

\iter\ is extensible---you can write new clauses that embody new
iteration patterns.  You might want to write a new driver clause for a
data structure of your own, or you might want to write a clause that
collects or manipulates elements in a way not provided by \iter. 

This section describes how to write clauses for \iter.  Writing a
clause is like writing a macro.  In fact, writing a clause ~is~
writing a macro: since \iter\ code-walks its body and macroexpands,
you can add new abstractions to \iter\ with good old |defmacro|.

Actually, there are two extensions you can make to \iter\ that are
even easier than writing a macro.  They are adding a synonym for an
existing clause and defining a driver clause for an indexable
sequence.  These can be done with |defsynonym| and
|defclause-sequence|, respectively.  See section \ref{aids}, below.

The rest of this section explains how to write macros that expand into
\iter\ clauses.
Here's how you could add a simplified version of \iter's |multiply|
clause, if \iter\ didn't already have one:
\begin{program}
(defmacro multiply (expr)
  `(reducing ,expr by \#'* initial-value 1))
\end{program}

If you found yourself summing the square of an expression often, you
might want to write a macro for that.  A first cut might be
\begin{program}
(defmacro sum-of-squares (expr)
  `(sum (* ,expr ,expr)))
\end{program}
but if you are an experienced macro writer, you will realize that this
code will evaluate ~expr~ twice, which is probably a bad idea.  A
better version would use a temporary:
\begin{program}
(defmacro sum-of-squares (expr)
  (let ((temp (gensym)))
    `(let ((,temp ,expr))
       (sum (* ,temp ,temp)))))
\end{program}
Although this may seem complex, it is just the sort of thing you'd
have to go through to write any macro, which illustrates the point of
this section: if you can write macros, you can extend \iter.

Our macros don't use \iter's keyword-argument
syntax.  We could just use keywords with |defmacro|, but we would
still not be using \iter's clause indexing mechanism.  Unlike Lisp,
which uses just the first symbol of a form to determine what function
to call, \iter\ individuates clauses by the list of required keywords.
For instance, |for\dots in| and |for\dots in-vector| are different clauses
implemented by distinct Lisp functions.  

To buy into this indexing scheme, as well as the keyword-argument
syntax, use |defmacro-clause|:

\begin{clauses}

\defmacro{defmacro-clause}{arglist |\&body| body}
Defines a new \iter\ clause.  ~arglist~ is a list of symbols which are
alternating keywords and arguments.  \opt\ may be used, and the list
may be terminated by |\&sequence|.  ~body~ is an ordinary macro body,
as with |defmacro|.  If the first form of ~body~ is a string, it is
considered a documentation string and will be shown by
|display-iterate-clauses|. |defmacro-clause| will signal an error if 
defining the clause would result in an ambiguity.  E.g. you cannot
define the clause |for\dots from| because there would be no way to
distinguish it from a use of the |for| clause with optional keyword |from|.

\end{clauses}

\medskip

Here is |multiply| using |defmacro-clause|.  The keywords are capitalized
for readability.
\begin{program}
(defmacro-clause (MULTIPLY expr \opt\ INTO var)
  `(reducing ,expr by \#'* into ,var initial-value 1))
\end{program}
You don't have to worry about the case when |var| is not supplied; for
any clause with an |into| keyword, saying |into nil| is equivalent to
omitting the |into| entirely.

As another, more extended example, consider the fairly common
iteration pattern that 
involves finding the sequence element that maximizes (or minimizes) some
function.  \iter\ provides this as |finding\dots maximizing|, but it's
instructive to see how to write it.
Here, in pseudocode, is how you might write such a loop for
maximizing a function F:

\begin{quote}
\begin{tabbing}
set variable MAX-VAL to NIL; \\
set variable WINNER to NIL; \\
for\=\ each element EL in the sequence \\
\>  if\=\ MAX-VAL\=\ is NIL or F(EL) $>$ MAX-VAL then \\
\>\>    set MAX-VAL to F(EL); \\
\>\>    set WINNER to EL; \\
\>  end if; \\
end for; \\
return WINNER.
\end{tabbing}
\end{quote}

Here is the macro:
\begin{program}
(defmacro-clause (FINDING expr MAXIMIZING func \opt\ INTO var)
  (let ((max-val (gensym))
        (temp1 (gensym))
        (temp2 (gensym))
        (winner (or var iterate::*result-var*)))
    `(progn 
       (with ,max-val = nil)
       (with ,winner = nil)
       (cond 
        ((null ,max-val)
         (setq ,winner ,expr)
         (setq ,max-val (funcall ,func ,winner))
        (t
         (let* ((,temp1 ,expr)
                (,temp2 (funcall ,func ,temp1)))
           (when (> ,temp2 ,max-val)
             (setq ,max-val ,temp2)
             (setq ,winner ,temp1))))))
       (finally (leave ,winner)))))
\end{program}
Note that if no |into| variable is supplied, we use
|iterate::*result-var*|, which contains the internal variable into
which all clauses place their results.  If this variable is bound by
some clause, then \iter\ will return its value automatically;
otherwise, \nil\ will be returned.  

\subsection{Writing Drivers}

In principle, drivers can be implemented just as easily as other
\iter\ clauses.  In practice, they are a little harder to get right. 
As an example, consider writing a driver that
iterates over all the 
elements of a vector, ignoring its fill-pointer.  |for\dots in-vector|
won't work for this, because it observes the fill-pointer.  It's necessary to
use |array-dimension| instead of |length| to obtain the size of the
vector.  Here is one approach:

\begin{program}
(defmacro-clause (FOR var IN-WHOLE-VECTOR v)
  "All the elements of a vector (disregards fill-pointer)"
  (let ((vect (gensym))
        (index (gensym)))
    `(progn
       (with ,vect = ,v)
       (for ,index from 0 below (array-dimension ,vect 0))
       (for ,var = (aref ,vect ,index)))))
\end{program}
Note that we immediately put |v| in a variable, in case it is an
expression.  Again, this is just good Lisp macrology.  It also has a
subtle effect on the semantics of the driver: |v| is evaluated only
once, at the beginning of the loop, so changes to |v| in the loop have
no effect on the driver.  Similarly, the bounds for numerical iteration
e.g. the above |array-dimension| are also evaluated once only.  This is how
all of \iter's drivers work.


There is an important point concerning the |progn| in this code.  We
need the |progn|, of course, because we are returning several forms,
one of which is a driver.  But \iter\ drivers must occur at top-level.
Is this code in error?  No, because ~top-level~ is defined in \iter\
to include forms inside a |progn|.  This is just the definition of
top-level that Common Lisp uses, and for the same reason: to allow
macros to return multiple forms at top-level.

While our |for\dots in-whole-vector| clause will work, it is not
ideal.  In particular, it does not support generating.  Do do so, we
need to use |for\dots next| or |for\dots do-next|.  The job is
simplified by the |defmacro-driver| macro.

\begin{clauses}

\defmacro{defmacro-driver}{arglist |\&body| body}
Defines a driver clause in
both the |for| and |generate| forms, and provides a parameter
|generate| which ~body~ can examine to determine how it was invoked.
~arglist~ is as in |defmacro-clause|, and should begin with the symbol
|for|. 

\end{clauses}

With |defmacro-driver|, our driver looks like this:
\begin{program}
(defmacro-driver (FOR var IN-WHOLE-VECTOR v)
  "All the elements of a vector (disregards fill-pointer)"
   (let ((vect (gensym))
         (end (gensym))
         (index (gensym))
         (kwd (if generate 'generate 'for)))
     `(progn
        (with ,vect = ,v)
        (with ,end = (array-dimension ,vect 0))
        (with ,index = -1)
        (,kwd ,var next (progn (incf ,index)
                               (if (>= ,index ,end) (terminate))
                               (aref ,vect ,index))))))
\end{program}


We are still missing one thing: the |\&sequence| keywords.  
We can get them easily enough, by writing 
\begin{program}
(defmacro-driver (FOR var IN-WHOLE-VECTOR v \&sequence) 
  ...)
\end{program}
We can now refer to parameters |from|, |to|, |by|, etc. which contain
either the values for the corresponding keyword, or \nil\ if the
keyword was not supplied.  Implementing the right code for these
keywords is cumbersome but not difficult; it is left as an exercise.
But before you begin, see |defclause-sequence| below for an easier way.
       
\subsection{Extensibility Aids}
\label{aids}

This section documents assorted features that may be of use in
extending \iter.

\begin{clauses}

\defunexpvar{*result-var*}
Holds the variable that is used to return a value as a result of the
\iter\ form.  You may examine this and use it in a |with| clause, but
you should not change it.

\defmacro{defsynonym}{syn word}
 Makes ~syn~ a synonym for the existing \iter\ keyword ~word.~  Only
the first word in each clause can have synonyms. 

\defmacroxx{defclause-sequence}{element-name index-name}{|\&key|
access-fn size-fn sequence-type}{element-type
element-doc-string index-doc-string}  
Provides
a simple way to define sequence clauses.  Generates two
clauses, one for iterating over the sequence's elements, the other
for iterating over its indices.  The first symbol of both
clauses will have print-name |for|.
~element-name~  and ~index-name~ should be symbols.
~element-name~ is the second keyword of the element iterator (typically of
the form 
|in-~sequence-type~|), and ~index-name~ is the second keyword
of the index-iterator (typically of the form
|index-of-~sequence-type~|).  Either name may be 
\nil, in which case the corresponding clause is not defined.  If both
symbols are supplied, they should be in the same package.  The |for|
that begins the clauses will be in this package.

\cpar ~access-fn~ is the function to be used to
access elements of the sequence in the element iterator.  The function
should take two 
arguments, a sequence and an index, and return the appropriate element.
~size-fn~ should denote a function of one argument, a sequence, that
returns its size.  Both ~access-fn~ and ~size-fn~ are required for the
element iterator, but only ~size-fn~ is needed for the index iterator.

\cpar The ~sequence-type~ and ~element-type~ keywords are used to
suggest types for the variables 
used to hold the sequence and the
sequence elements, respectively.  The usual rules about \iter's
treatment of variable type declarations apply (see section \ref{types}).

\cpar ~element-doc-string~ and ~index-doc-string~ are
the documentation strings, for use with |display-iterate-clauses|. 

\cpar The generated element-iterator performs destructuring on the
element variable.

\cpar As an example, the above |for\dots in-whole-vector| example
could have been written:
\begin{program}
(defclause-sequence IN-WHOLE-VECTOR INDEX-OF-WHOLE-VECTOR
  :access-fn 'aref
  :size-fn \#'(lambda (v) (array-dimension v 0))
  :sequence-type 'vector
  :element-type t
  :element-doc-string 
     "Elements of a vector, disregarding fill-pointer"
  :index-doc-string 
     "Indices of vector, disregarding fill-pointer")
\end{program}

\end{clauses}

\subsection{Subtleties}

There are some subtleties to be aware of when writing \iter\ clauses.
First, the code returned by your macros may be |nconc|'ed into a list,
so you should always returned freshly consed lists, rather than
constants.  Second, \iter\ matches clauses by using |eq| on the first
symbol and |string=| on the subsequent ones, so the package of the
first symbol of a clause is relevant.  All of the clauses in this manual
have their first word in the \iter\ package.  
You can use the package system in the usual way to shadow
\iter\ clauses without replacing them.

%%% say more here, about the badness that only the first word of a
%%% clause is packagey.

\section{Non-portable Extensions to Iterate (Contribs)}
\label{contribs}

Currently, there is only one non-portable extension to iterate in the
distribution: iterate-pg. If you have made an extension that depends
on non-portable features, feel free to send them to |asf@boinkor.net|
for inclusion in the iterate distribution.

\subsection{An SQL query driver for iterate}

The pg package by Eric Marsden (see |http://cliki.net/pg|) provides an
interface to the PostgreSQL database. Using the \iterpg\ extension, it
is possible to handle the results of SQL queries using \iter.

This usage example should give you an idea of how to use it:

\begin{program}
(pg:with-pg-connection (c "somedb" "someuser")
  (iter (for (impl version date) in-relation "select * from version"
                                 on-connection *dbconn*)
        (collect version)))
\end{program}

To use the extension via |ASDF|, simply make your system depend on the
|iterate-pg| system instead of the |iterate| system. To load it
manually, use:

\begin{program}
  (asdf:oos 'asdf:load-op :iterate-pg)
\end{program}

%%%%%%%%%%%%%%%%%%%%%%%%%%%%%%%%%%%%%%%%%%%%%%%%%%%%%%%%%%%%%%%%
\section{Obtaining {\tt Iterate}}

\begin{sloppypar}
The information in this chapter is obsolete but included for
completeness's sake; Currently, the most up-to-date information on
\iter\ can be found at |http://boinkor.net/iterate.html|.
\end{sloppypar}

\begin{sloppypar}
\iter\ currently runs on Lisp Machines, and on
HP's, Sun3's and Sparcstations under Lucid.  
\iter\ source and binaries are available at the MIT AI Lab in the
subdirectories of |/src/local/lisplib/|.  The source file,
|iterate.lisp|, is also available for anonymous FTP in the directory
|/com/fpt/pub/| on the machine |TRIX.AI.MIT.EDU| (Internet number
128.52.32.6).  If you are unable to obtain |iterate| in one of these
ways, send mail to |jba@ai.mit.edu| and I will send you the source
file. 
\end{sloppypar}

\begin{sloppypar}
\iter\ resides in the |iterate| package (nickname |iter|).  Just say
\linebreak |(use-package :iterate)| to make all the necessary symbols
available. 
If a symbol is not exported, it appears in this manual with an
``|iterate::|'' prefix.
\end{sloppypar}

Send bug reports to |bug-iterate@ai.mit.edu|.  The |info-iterate|
mailing list will have notices of changes and problems; to have
yourself added, send mail to |info-iterate-request@ai.mit.edu|.


\medskip

\begin{flushleft}
 \bf Acknowledgements
\end{flushleft}
\smallskip

Richard Waters provided invaluable criticism which spurred me to improve
\iter\ greatly.  As early users, David Clemens, Oren Etzioni and Jeff
Siskind helped ferret out many bugs.

%\begin{theindex}
% The index files must be generated with the genindex program
% or from makeindex iterate-manual.idx which creates iterate-manual.ind.

%\baselineskip\normalbaselineskip
%\advance\baselineskip by -2pt

% Fix ``.'' spacing in lists
\documentstyle[12pt,aimemo]{article}

\setlength{\textwidth}{6.0in}
\setlength{\oddsidemargin}{0.25in}
\setlength{\evensidemargin}{0.25in}
\setlength{\topmargin}{0pt}
\setlength{\headsep}{24pt}
\setlength{\textheight}{8.5in}

\let\origpar=\par

% somehow, LaTex fucks with things so that the plain Tex \obeyspaces
% doesn't work.  This is the same, except we use an mbox containing
% a space instead of just a space.

\newcommand{\spc}{\mbox{ }}
\newcommand{\tAb}{\mbox{        }}
\def\obspaces{\catcode`\ =\active\catcode`\^^I=\active}
{\obspaces \global\let =\spc\global\let^^I=\tAb}

\newenvironment{program}{\medskip\samepage\tt\obeylines\obspaces}{\medskip}

% use | for \tt
\catcode`\|=13
\newif\ifvbar
\def|{\ifvbar
         \endgroup\vbarfalse
      \else 
         \begingroup\tt\vbartrue 
      \fi}



\newcommand{\lisp}{\tt}
\newcommand{\iter}{|iterate|}
\newcommand{\iterpg}{|iterate-pg|}
\newcommand{\setf}{|setf|}
\newcommand{\nil}{|nil|}
\newcommand{\nonnil}{non-\nil}
\newcommand{\opt}{{\lisp \&optional}}
\newcommand{\yields}{$\Rightarrow$}



\newbox\Dots
\setbox\Dots=\hbox{\small ...}
\renewcommand{\dots}{\copy\Dots}

% Here's a way to define an environment which will treat all
% paragraphs inside it in the same manner.  It's, as near as I can
% figure, the way that Latex does things.  Using \everypar won't work;
% apparently Latex resets that a lot.  What you have to do is: at the
% top of the file do \let\origpar=\par.  In the
% begin part of the environment, do the parshape (or whatever), then say
% \def\par{{\origpar}}.  The extra braces are necessary.   That's all
% you have to do; at the end of the environment things will return to
% normal automatically.  Why does this all work?  I have no idea.
% Another important thing: there should be a blank line (i.e. a \par)
% between the end of the last paragraph and the \end{environment}.

\newlength{\clindent}
\setlength{\clindent}{0.5in}
\newlength{\clparindent}
\setlength{\clparindent}{\parindent}

\newenvironment{clauses}{\advance\linewidth -\parindent
                        \hangindent\clindent\relax
                         \hangafter=0
                          \parindent=0pt
                          \def\par{{\origpar}}}{}

\newcommand{\cpar}{\hspace \clparindent}

\newcommand{\startitem}{\bigskip\pagebreak[3]}
\newcommand{\finishitem}{\smallskip}

\newcommand{\clause}[1]{\startitem\Clause{#1}\finishitem}
\newcommand{\Clause}[1]{\hbox{#1}}
\newcommand{\clindex}[1]{\index{{\lisp #1}}}
\newcommand{\clindexx}[2]{\index{{\lisp #1\dots #2}}}
\newcommand{\clindexxx}[3]{\index{{\lisp #1\dots #2\dots #3}}}
\newcommand{\clausex}[2]{\startitem\Clausex{#1}{#2}\finishitem}
\newcommand{\Clausex}[2]{\Clause{|#1| #2}\clindex{#1}}

\newcommand{\defvar}[1]{\startitem\deff{#1}{}{Variable}\finishitem}
\newcommand{\defunexpvar}[1]{\startitem\deffnoindex{iterate::#1}{}{Variable}
                                        \lispindex{#1}\finishitem}
\newcommand{\defun}[2]{\startitem\deff{#1}{#2}{Function}\finishitem}
\newcommand{\defmacro}[2]{\startitem\deff{#1}{#2}{Macro}\finishitem}
\newcommand{\defunx}[3]{\startitem\deffx{#1}{#2}{#3}{Function}\finishitem}
\newcommand{\defmacrox}[3]{\startitem\deffx{#1}{#2}{#3}{Macro}\finishitem}
\newcommand{\defunxx}[4]{\startitem\deffxx{#1}{#2}{#3}{#4}{Function}\finishitem}
\newcommand{\defmacroxx}[4]{\startitem\deffxx{#1}{#2}{#3}{#4}{Macro}\finishitem}

\newcommand{\deff}[3]{\deffnoindex{#1}{#2}{#3}\lispindex{#1}}
\newcommand{\deffnoindex}[3]{\hbox to \hsize{{\tt #1} {\it #2}\hfill [{\it #3}]}}
\newcommand{\deffx}[4]{\deff{#1}{#2}{#4}\moreargs{#1}{#3}}
\newcommand{\deffxx}[5]{\deffx{#1}{#2}{#3}{#5}\moreargs{#1}{#4}}

\newcommand{\moreargs}[2]{\hbox to \hsize{\phantom{\lisp #1} {\it #2}\hfill}}

\newcommand{\lispindex}[1]{\index{{\lisp #1}}}

 \newenvironment{note}[1]{\pagebreak[2]\bigskip
                          \hrule\smallskip\small
                         {\setlength{\parindent}{0pt}
                           \par{\bf #1:}}}{\smallskip\hrule\bigskip}



\makeindex

%%%%%%%%%%%%%%%%%%%%%%%%%%%%%%%%%%%%%%%%%%%%%%%%%%%%%%%%%%%%%%%%

%\thispagestyle{empty}

%\vspace*{2in}

%\title{The |Iterate| Manual}
%\author{Jonathan Amsterdam}

%\maketitle

%\thispagestyle{empty}

%\begin{center}
%\Large *** DRAFT ***
%\end{center}
%This manual is in beta-test. It will become an AI Memo in a couple of
%months.  I would appreciate any comments or
%corrections.  You can reach me at email address |jba@ai.ai.mit.edu|.

\begin{document}

\def\writtenby{Jonathan Amsterdam}
\def\contractno{This report describes research done at the Artificial
Intelligence Laboratory of the Massachusetts Institute of Technology.
Support for the laboratory's artificial intelligence research is
provided in part by the Advanced Research Projects Agency of the
Department of Defense under Office
of Naval Research contract N00014-85-K-0124.}

\AIMEMO{1236}{\Large The Iterate Manual \\[1in]}
{This is the manual for version 1.4 of \iter, a powerful iteration
macro for Common  
Lisp. \iter\ is similar to {\tt loop} but provides numerous additional
features, is well integrated with Lisp, and is extensible. 
\vfill }


% use ~ for \em
\catcode`\~=13
\newif\ifem
\def~{\ifem
         \/\endgroup\emfalse
      \else 
         \begingroup\em\emtrue 
      \fi}



%\pagebreak
%\setcounter{page}{1}

%\thispagestyle{empty}


\tableofcontents

\pagebreak
\pagestyle{headings}

\section{Introduction}

\begin{sloppypar}
This manual describes
\iter, a powerful iteration facility for Common Lisp.
\iter\ provides abstractions for many common iteration
patterns and allows for the definition of additional patterns.  \iter\ is
a macro that expands into ordinary Lisp at
compile-time, so it is more efficient than higher-order
functions like |map| and |reduce.|  
While it is similar to |loop|, \iter\ offers a more Lisp-like syntax
and enhanced extensibility.
(For a more complete
comparison of \iter\ with other iteration constructs, see MIT AI Lab Working
Paper No. 324, ~Don't Loop, Iterate.~)
\end{sloppypar}

An \iter\ form consists of the symbol |iter|\footnote{You can also use
|iterate|, but |iter| is preferred because it avoids potential conflicts with
possible future additions to Common Lisp, and because it saves
horizontal space when writing code.}
followed by one or more
forms, some of which may be \iter\ 
~clauses.~  Here is a simple example of \iter\ which collects the
numbers from 1 to 10 into a list, and returns the list.  The return
value is shown following the arrow.
\begin{program}
(iter (for i from 1 to 10)
      (collect i))          \yields (1 2 3 4 5 6 7 8 9 10)
\end{program}
This form contains two clauses: a |for| clause that steps the
variable |i| over the integers from 1 to 10, and a |collect|
clause that accumulates its argument into a list.  With a few
exceptions, all \iter\ clauses have the same format: alternating
symbols (called ~keywords~) and expressions (called
~arguments~).  The syntax and terminology are those of Common
Lisp's keyword lambda lists.  One difference is that \iter's keywords
do not have to begin with a colon---though they may, except for the
first symbol of a clause.  So you can also write  |(for i :from
1 :to 10)| if you prefer.

Any Lisp form can appear in the body of an \iter, where it will have
its usual meaning.  \iter\ walks the entire body, expanding macros,
and recognizing clauses at any level.  This example collects all the
odd numbers in a list:
\begin{program}
(iter (for el in list)
      (if (and (numberp el) (oddp el))
          (collect el)))
\end{program}

There are clauses for iterating over numbers, lists, arrays and other
objects, and for
collecting, summing, counting, maximizing and other useful operations.
\iter\ also supports the creation of new variable bindings, stepping
over multiple sequences at once, destructuring, and compiler
declarations of variable types.  The following example illustrates
some of these features: 
\begin{program}
(iter (for (key . item) in alist)
         (for i from 0)
         (declare (fixnum i))
         (collect (cons i key)))
\end{program}

This loop takes the keys of an alist and returns a new alist
associating the keys with their positions in the original list. The
compiler declaration for |i| will appear in the generated code
in the appropriate place.

%%%%%%%%%%%%%%%%%%%%%%%%%%%%%%%%%%%%%%%%%%%%%%%%%%%%%%%%%%%%%%%%
\section{Clauses}

Most of \iter's clauses will be familiar to |loop| programmers.  
(|loop| is an iteration macro that has been incorporated into Common
Lisp.  Seen Guy Steele's {\em Common Lisp, 2nd Edition}.)
In
nearly all cases they behave the same as their |loop| counterparts, so
a |loop| user can switch to \iter\ with little pain (and much gain).

All clauses with the standard keyword-argument syntax consist of two
parts: a ~required~ part, 
containing keywords that must be present and in the right order; and
an ~optional~ part, containing keywords that may be omitted and,
if present, may occur in any order.  In the descriptions below, the
parts are separated by the Lisp lambda-list keyword \opt.


\subsection{Drivers}

An iteration-driving clause
conceptually causes the iteration to go forward.  Driver clauses in
\iter\ allow iteration over numbers, lists, vectors, hashtables, packages,
files and streams.  Iteration-driving clauses must
appear at the top level of an \iter\ form; they cannot be nested
inside another clause.  The driver variable is updated at the point
where the driver clause occurs.  Before the clause is executed for the
first time, the value of the variable is undefined.

%Also, regardless of where the driver clause
%appears in the body, the driver variable is stepped at the top of the
%loop; hence it is stylistically preferable, though
%not required, to place driver clauses at the beginning of the \iter.

Multiple drivers may appear in a single \iter\ form, in which case all
of the driver variables are updated each time through the loop, in the
order in which the clauses appear.  The first driver to terminate will
terminate the entire loop.

In all cases, the value of the driver variable on exit from the loop,
including within the epilogue code (see the |finally| clause), is
undefined.

All the parameters of a driver clause are evaluated once, before the
loop begins.  Hence it is not possible to change the bounds or other
properties of an iteration by side-effect from within the loop.

With one exception, driver clauses begin with the word |for| (or
the synonym |as|) and mention an iteration variable, which is
given a binding within the \iter\ form.
The exception is |repeat|, which just executes a loop a
specified number of times:

\begin{clauses}

\clausex{repeat}{~n~}
Repeats the loop ~n~ times.  For example:
\begin{program}
(iter (repeat 100)
      (print "I will not talk in class."))
\end{program}
\cpar If $n \leq 0$, the
loop will never be executed.  If ~n~ is not an integer, the
actual number of executions will be $\lceil n \rceil$.

\end{clauses}

\pagebreak[3]

\subsubsection{Numerical Iteration}

\begin{clauses}
\clausex{for}{~var~ |\&sequence|} 
The general form for iterating over a sequence of numbers requires a
variable and, optionally, one or more keywords that provide the bounds
and step size of the iteration. The |\&sequence|
lambda-list keyword is a shorthand for these sequence keywords.  They are:
|from|, |upfrom|, |downfrom|, |to|, |downto|, |above|, |below| and |by|.
|from| provides the starting value for ~var~
and defaults to zero. |to| provides a final value and implies that the
successive values of ~var~ will be increasing; |downto| implies that
they will be decreasing.  The loop terminates when ~var~ passes
the final value (i.e. becomes smaller or larger than it, depending on
the direction of iteration); in other  words, the loop body
will never be 
executed for values of ~var~ past the final value.
|below| and |above| are similar to |to| and
|downto|, except that the loop terminates when ~var~ equals or passes
the final value.

\cpar If no final value is specified, the variable will be stepped
forever.  Using |from| or |upfrom| will result in increasing
values, while |downfrom| will give decreasing values.

\cpar On each iteration,
~var~ is incremented or decremented by the value of the sequence
keyword |by|, which defaults to 1.  It should always be a positive
number, even for downward iterations.

\cpar In the following examples, the sequence of numbers generated is shown
next to the clause.

\begin{program}
(for i upfrom 0) \yields\ 0 1 2 \ldots
(for i from 5)   \yields\ 5 6 7 \ldots    ; either from or upfrom is okay
(for i downfrom 0) \yields\ 0 -1 -2 \ldots
(for i from 1 to 3) \yields\ 1 2 3
(for i from 1 below 3) \yields\ 1 2
(for i from 1 to 3 by 2) \yields\ 1 3
(for i from 1 below 3 by 2) \yields\ 1
(for i from 5 downto 3) \yields\ 5 4 3
\end{program}

\end{clauses}

\subsubsection{Sequence Iteration}

There are a number of clauses for iterating over sequences.  In all of
them, the argument following |for| may be a list instead of a
symbol, in which case destructuring is performed.  See section
\ref{destructuring}. 

\begin{clauses}

\clause{|for| ~var~ |in| ~list~ \opt\ |by| ~step-function~}
\clindexx{for}{in}
~var~ is set to successive elements of list.  
~step-function,~ which 
defaults to |cdr|, is used to obtain the next sublist. 

\clause{|for| ~var~ |on| ~list~ \opt\ |by| ~step-function~}
\clindexx{for}{on}
~var~ is set to successive sublists of list. 
~step-function~ (default |cdr|) is used as in |for\dots in|.

\end{clauses}

\medskip

These two clauses use |atom| to test for the end
of a list.  Hence, given a list whose final |cdr| is not \nil,
they will silently ignore the last |cdr|.  Other choices are
|endp|,  which would signal an error, and |null|, which
would probably result in an error somewhere else.  If you wish to use
an end-test other than |atom|, set the variable
|iterate::*list-end-test*|\lispindex{*list-end-test*} to the name of
the desired function. 

\begin{clauses}
\clause{|for| ~var~ |in-vector| ~vector~ |\&sequence|}
\clindexx{for}{in-vector}
~var~ takes on successive elements from ~vector.~   The vector's
fill-pointer is observed.  Here and in subsequent clauses, the |\&sequence|
keywords include |with-index|, which takes a symbol as argument
and uses it for the index variable instead of an internally generated
symbol.  The other |\&sequence| keywords behave as in numerical
iteration, except that the default iteration bounds are the bounds of
the vector.  E.g. in |(for i in-vector v downto 3)|,
|i| will start off being bound to the last element in |v|,
and will  
be set to preceding elements down to and including the element with
index 3.

\clause{|for| ~var~ |in-sequence| ~seq~ |\&sequence|}
\clindexx{for}{in-sequence}
This uses Common Lisp's generalized sequence functions, |elt|
and |length|, to obtain elements and determine the length of
~seq.~   Hence it will work for any sequence, including lists, and will
observe the fill-pointers of vectors.

\clause{|for| ~var~ |in-string| ~string~ |\&sequence|}
\clindexx{for}{in-string}
~var~ is set to successive characters of ~string.~

\startitem
\Clause{|for| ~var~ |index-of-vector| ~vector~ |\&sequence|}
\clindexx{for}{index-of-vector}
\Clause{|for| ~var~ |index-of-sequence| ~sequence~ |\&sequence|}
\clindexx{for}{index-of-sequence}
\Clause{|for| ~var~ |index-of-string| ~string~ |\&sequence|}
\clindexx{for}{index-of-string}
\finishitem
~var~ is set to successive indices of the sequence.
These clauses avoid the overhead of accessing the sequence elements
for those applications where they do not need to be examined, or are
examined rarely.  They admit all the optional keywords of the other
sequence drivers except the (redundant) |with-index| keyword.

\clause{|for| ~(key value)~ |in-hashtable| ~table~}
\clindexx{for}{in-hashtable}

~key~ and ~value,~ which must appear as shown in a list
and may be destructuring templates, are set to the keys and values of
~table~.  If ~key~ is |nil|, then the hashtable's keys will be ignored;
similarly for ~value~.
The order in which 
elements of ~table~ will be retrieved is unpredictable.

\clause{|for| ~var~ |in-package| ~package~ \opt\ |external-only| ~ext~}
\clindexx{for}{in-package}
Iterates over all the symbols in ~package,~ or over only the
external symbols if ~ext~ is specified and non-|nil|.  ~ext~ is not
evaluated.  The same symbol may appear more than once.

\clause{|for| ~var~ |in-packages| ~packages~ \opt\ |having-access| ~symbol-types~}
\clindexx{for}{in-packages}
Iterates over all the symbols from the list of packages denoted by the
descriptor ~packages~ and having accessibility (or visibility) given by
~symbol-types~.  This defaults to the list |(:external :internal :inherited)|
and is not evaluated.  ~var~ must be a list of up to three variables: in each
iteration, these will be set to a symbol, its access-type and package (as per
|with-package-iterator| in ANSI-CL).  The same symbol may appear more than
once.

\end{clauses}

\begin{clauses}

\clause{|for| ~var~ |in-file| ~name~ \opt\ |using| ~reader~}
\clindexx{for}{in-file}
Opens the file ~name~ (which may be a string or pathname) for input,
and iterates 
over its contents. ~reader~ defaults to |read|, so by default ~var~
will be bound to the successive forms in the file.  The \iter\ body is
wrapped in an unwind-protect to ensure that the file is closed no
matter how the \iter\ is exited.

\clause{|for| ~var~ |in-stream| ~stream~ \opt\ |using| ~reader~}
\clindexx{for}{in-stream}
Like |for\dots in-file|, except that ~stream~ should be an existing
stream object that supports input operations.

\end{clauses}

\subsubsection{Generalized Drivers}

These are primarily useful for writing drivers that can also be used
as generators (see section \ref{generators}, below).
 
\begin{clauses}

\clause{|for| ~var~ |next| ~expr~}
\clindexx{for}{next}

~var~ is set to ~expr~ each time through the loop.  Destructuring is
performed.  When the clause is used as a generator, ~expr~ is the code
that is executed when |(next ~var~)| is 
encountered (see section \ref{generators}, below).
~expr~ should compute the first value for ~var~, as well
as all subsequent values, and is responsible for terminating the loop.
For compatibility with future versions of \iter, this termination
should be done with |terminate|\clindex{terminate}, which can be
considered a synonym for |finish| (see section \ref{control-flow}).

\cpar As an example, the following clauses are equivalent to |(for
i from 1 to 10)|:
\begin{program}
(initially (setq i 0))
(for i next (if (> i 10) (terminate) (incf i)))
\end{program}

\clause{|for| ~var~ |do-next| ~form~}
\clindexx{for}{do-next}
~form~ is evaluated each time through the loop.  Its value is ~not~
set to ~var~; that is ~form's~ job. ~var~ is only present so that
\iter\ knows it is a driver variable.  \linebreak |(for ~var~ next
~expr~)| is equivalent to |(for ~var~ do-next (dsetq ~var~ ~expr~))|.
(See section \ref{destructuring} for an explanation of |dsetq|.) 

\end{clauses}

\subsubsection{Generators}
\label{generators}

In all of the above clauses, the driver variable is updated on each
iteration.
Sometimes it is desirable to have greater control over updating.  
For instance, consider the problem of associating numbers, in
increasing order and with no gaps, with the
\nonnil\ elements of a list.  One obvious first pass at writing this is:

\begin{program}
(iter (for el in list)
      (for i upfrom 1)
      (if el (collect (cons el i))))
\end{program}
But on the list |(a b nil c)| this produces |((a . 1) (b . 2) (c .
4))| instead of the desired |((a . 1) (b . 2) (c . 3))|.  The problem
is that |i| is incremented each time through the loop, even when |el|
is |nil|.  

The problem could be solved elegantly if we could step |i| only when
we wished to.  This
can be accomplished for any \iter\ driver by writing |generate| (or
its synonym |generating|)
instead of |for|.  Doing so produces a ~generator~---a driver whose
values are yielded explicitly.  To obtain the next value of a
generator variable ~v~, write \linebreak |(next ~v~)|.  The value of a |next|
form is the next value of ~v~, as determined by its associated driver
clause.  |next| also has the side-effect of updating ~v~ to that
value.  If there is no next value, |next| will terminate the loop,
just as with a normal driver.


Using generators, we can now write our example like this:
\begin{program}
(iter (for el in list)
      (generate i upfrom 1)
      (if el (collect (cons el (next i)))))
\end{program}
Now |i| is updated only when |(next i)| is executed, and this occurs
only when |el| is \nonnil.

To better understand the relationship between ordinary drivers and
generators, observe that we can rewrite an ordinary driver using its
generator form immediately followed by |next|, as this example shows:
\begin{program}
(iter (generating i from 1 to 10)
      (next i)
      ...)
\end{program}
Provided that the loop body contains no |(next i)| forms, this will
behave just as if we had written |(for i from 1 to 10)|.

We can still refer to a driver variable ~v~ without using |next|; in
this case, its value is that given to it by the last evaluation of
|(next ~v~)|.  Before |(next ~v~)| has been called the first time, the
value of ~v~ is undefined.  

This semantics is more flexible than
one in which ~v~ begins the loop bound to its first value and calls
of |next| supply subsequent values, because it means the loop will not
terminate too soon if the generator's sequence is empty.  For
instance, consider the following code, which tags \nonnil\ elements of
a list using a list of tags, and also counts the null elements.
(We assume there are at least as many tags as \nonnil\ elements.)
\begin{program}
(let* ((counter 0)
       (tagged-list (iter (for el in list)
                          (generating tag in tag-list) 
                          (if (null el)
                              (incf counter)
                              (collect (cons el (next tag)))))))
  ...)
\end{program}
It may be that there are just as many tags as non-null elements of
|list|.  If all the elements of |list| are null, we still want the
counting to proceed, even though |tag-list| is \nil.  If |tag| had to be
assigned its first value before the loop begins, we would have had to
terminate the loop before the first iteration, since when |tag-list|
is \nil, |tag| 
has no first value.  With the existing semantics, however, |(next
tag)| will never execute, so the iteration will cover all the elements of
|list|. 
        
When the ``variable'' of a driver clause is actually a destructuring
template containing several variables, all the variables are eligible
for use with |next|.  As before, |(next ~v~)| evaluates to ~v's~ next
value; but the effect is to update all of the template's variables.
For instance, the following code will return the list |(a 2 c)|.
\begin{program}
(iter (generating (key . item) in '((a . 1) (b . 2) (c . 3)))
      (collect (next key))
      (collect (next item)))
\end{program}

Only driver clauses with variables can be made into generators.  This
includes all clauses mentioned so far except for |repeat|.  It does
~not~ include |for\dots previous|, |for\dots =|,
|for\dots initially\dots then| or |for\dots first\dots then| (see
below). 

\subsubsection{Previous Values of Driver Variables}

Often one would like to access the value of a variable on a previous
iteration.  \iter\ provides a special clause for accomplishing this.

\begin{clauses}

\clause{|for| ~pvar~ |previous| ~var~ \opt\ |initially| ~init~
|back| ~n~}
\clindexx{for}{previous}
Sets ~pvar~ to the previous value of ~var,~ which should be a driver
variable, a variable from another |for\dots previous| clause, or a
variable established by a |for\dots =|, 
|for\dots initially\dots then| or |for\dots first\dots then| clause
(see section \ref{setting}).
Initially, ~pvar~ is given the value ~init~ (which defaults to \nil).  
The ~init~ expression will be moved outside the loop body, so it
should not depend on anything computed within the loop.
~pvar~ retains the value of ~init~ until ~var~ is set to its second
value, at which point ~pvar~ is set to ~var's~ first value; and so on.  

\cpar The
argument ~n~ to |back| must be a constant, positive integer, and
defaults to 1.  It determines how many iterations back ~pvar~ should
track ~var~.  For example, when ~n~ is 2, then ~pvar~ will be assigned
~var's~ first value when ~var~ is set to its third value.

\cpar A |for\dots previous| clause may occur after or before its
associated driver clause. |for\dots previous| works with generators as
well as ordinary drivers.

\pagebreak[3]

\cpar Example:
\begin{program}
(iter (for el in '(1 2 3 4))
      (for p-el previous el)
      (for pp-el previous p-el initially 0)
      (collect pp-el))
\end{program}
This evaluates to |(0 0 1 2)|.  It could have been written more
economically as 
\begin{program}
(iter (for el in '(1 2 3 4))
      (for pp-el previous el back 2 initially 0)
      (collect pp-el))
\end{program}


\end{clauses}

\subsection{Variable Binding and Setting}
\label{setting}

Several clauses exist for establishing new variable bindings or for
setting variables in the loop.  They all support destructuring.

\begin{clauses}

\clausex{with}{~var~ \opt\ |=| ~value~}
Causes ~var~ to be bound to value before the loop body is entered.
If ~value~ is not supplied, ~var~ assumes a default
binding, which will be 
\nil\ in the absence of declarations.  Also, if ~value~ is not
supplied, no destructuring is performed; instead, ~var~ may be a list of
symbols, all of which are given default 
bindings.  If ~value~ is supplied, ~var~ is bound to it, with
destructuring.  

\cpar Because |with|
creates bindings whose scope includes the entire \iter\ form, it is
good style to put all |with| clauses at the beginning.

\cpar Successive occurrences of |with| result in sequential
bindings (as with 
|let*|).  There is no way to obtain parallel bindings; see
section \ref{bindings} for a rationale.  


\clause{|for| ~var~ |=| ~expr~}
\clindexx{for}{=}
On each iteration, ~expr~ is evaluated and ~var~ is set
to its value. 

\cpar This clause may appear to do the same thing as |for\dots next|.
In fact, they are quite different.  |for\dots =| provides only three
services: it sets up a binding for ~var~, sets it to ~expr~ on each
iteration, and makes it possible to use |for\dots previous| with
~var~.  |for\dots next| provides these services in addition to the
ability to turn the driver into a generator.
%Also, the code which sets ~var~ appears in the loop body in the same
%place as the |for\dots =| clause; the code for |for\dots next| appears
%at the top of the loop, as with other drivers (except when being used
%as a generator).

\clause{|for| ~var~ |initially| ~init-expr~ |then| ~then-expr~}
\clindexxx{for}{initially}{then}
Before the loop begins, ~var~ is set to ~init-expr;~ on all
iterations after the first it is set to ~then-expr.~  This clause
must occur at top-level.  ~init-expr~ will be moved outside the loop
body and ~then-expr~ will be moved to the end of the loop body, so
they are subject to code motion problems (see section
\ref{code-movement}). 

\cpar This clause may appear to be similar to |for\dots next|, but in
fact they differ significantly.  |for\dots initially\dots then| is
typically used to give ~var~ its first value before the loop begins,
and subsequent values on following iterations.  This is incompatible
with generators, whose first value and subsequent values must all be
computed by |(next ~var~)|.  Also, the update of ~var~ in 
|for\dots initially\dots then| does not occur at the location of the
clause. 
Use |for\dots initially\dots then| for
one-shot computations where its idiom is more convenient, but use
|for\dots next| for extending \iter\ with new drivers (see section
\ref{extend}).

\clause{|for| ~var~ |first| ~first-expr~ |then| ~then-expr~}
\clindexxx{for}{first}{then}
The first time through the loop, ~var~ is set to ~first-expr;~ on
subsequent iterations, it is set to ~then-expr.~  This differs from
|for\dots initially| in that ~var~ is set to ~first-expr~
inside the loop body, 
so ~first-expr~ may depend on the results of other clauses.  For
instance, 
\begin{program}
(iter (for num in list)
      (for i first num then (1+ i))
      ...)
\end{program}
will set |i| to the first element of |list| on the first
iteration, whereas 
\begin{program}
(iter (for num in list)
      (for i initially num then (1+ i))
      ...)
\end{program}
is probably erroneous; |i| will be bound to |num|'s
default binding (usually \nil) for the first iteration.

\end{clauses}

\begin{note}{Compatibility note}
|loop|'s |for\dots =| works like \iter's, but |loop| used the syntax
|for\dots =\dots then| to mean |for\dots initially\dots then|.  It was
felt that these two operations were sufficiently different to warrant
different keywords.

Also, the |for| in the above three clauses is misleading,
since none is true driver (e.g. none has a corresponding |generate|
form).  |setting| would have been a better choice, but |for| was used
to retain some compatibility with |loop|.  
\end{note}


%%%%%%%%%%%%%%%%%%%%%%%%%%%%%%%%
\subsection{Gathering Clauses}
Many of \iter's clauses accumulate values into a variable, or set a
variable under certain conditions.  At the end of the loop, this
variable contains the desired result.  All these clauses have an
optional |into| keyword, whose argument should be a symbol.  If the
|into| keyword is not supplied, the accumulation variable will be
internally generated and its value will be returned at the end of the
loop; if a variable is specified, that variable is used for the
accumulation, and is not returned as a result---it is up to the user
to return it explicitly, in the loop's epilogue code (see |finally|).
It is safe to examine the accumulation variable during the loop, but
it should not be modified.

These clauses all begin with a verb.  When the verb does
not conflict with an existing Common Lisp function, then it may be
used in either its infinitival or present-participle form (e.g. |sum|,
|summing|).  However, when there is a conflict with Common Lisp, only
the present-participle form may be used (e.g. |unioning|).  This is to
prevent \iter\ clauses from clashing with Common Lisp
functions. 

%although these clauses are described as ``producing a value,'' it is a
%mistake to think of the lisp list representing the clause as a
%value-producing form in the usual way.  clauses may legally be written
%where a value is expected, e.g. |(setq x (sum i))|, but the
%lisp value of a clause in such a context is undefined.

\subsubsection{Reductions}

~Reduction~ is an extremely common iteration pattern in which
the results of successive applications of a
binary operation are accumulated.
For example, a loop that computes the sum of the
elements of a list is performing a reduction with the addition
operation.  This could be written in Common Lisp as  |(reduce \#'+
list)| or with \iter\ as
\begin{program}
(iter (for el in list)
      (sum el))
\end{program}

\begin{clauses}

\clausex{sum}{~expr~ \opt\ |into| ~var~}
Each time through the loop, ~expr~ is evaluated and added to a
variable, which is bound initially to zero.  If ~expr~ has a type,
it is ~not~ used as the type of the sum variable, which is always
|number|.  To get the result variable to be of a more specific
type, use an explicit variable, as in
\begin{program}
(iter (for el in number-list)
      (sum el into x)
      (declare (fixnum x))
      (finally (return x)))
\end{program}

\clausex{multiply}{~expr~ \opt\ |into| ~var~}
Like |sum|, but the initial value of the result variable is $1$,
and the variable is updated by multiplying ~expr~ into it.

\clausex{counting}{~expr~ \opt\ |into| ~var~}
~expr~ is evaluated on each iteration.  If it is \nonnil, the
accumulation variable, initially zero, is incremented.

\startitem
\Clausex{maximize}{~expr~ \opt\ |into| ~var~}
\Clausex{minimize}{~expr~ \opt\ |into| ~var~}
\finishitem
~expr~ is evaluated on each iteration and its extremum (maximum or
minimum) is stored in the accumulation variable.  If ~expr~ is never
evaluated, then the result is \nil\ (if the accumulation variable is
untyped) or $0$ (if it has a numeric type).

\clausex{reducing}{~expr~ |by| ~func~ \opt\ |initial-value| ~init-val~ |into| ~var~}
This is a general way to perform reductions. ~func~ should be a
function of two arguments, the first of which will be the value
computed so far and
the second of which will be the value of ~expr~.  It should return the
new value.
|reducing| is roughly equivalent to the
Common Lisp |(reduce ~func~ ~sequence~ :key ~expr-function~)|, where
~expr-function~ is used to derive values from the successive elements
of ~sequence~.

\cpar If the |reducing| clause is never executed, the result is
undefined.

\cpar It is not necessary to provide an
initial value, but better code can be generated if one is supplied.
Regardless of its location in the \iter\ body, ~init-val~ will be
evaluated before the loop is entered, so it should not depend on any
value computed inside the \iter\ form.  

%\cpar if a ~var~ is not specified, you can get \iter\ to declare the
%type of the internal variable by putting a |the| expression around
%~func.~  see section \ref{types}.

\end{clauses}

%\begin{note}{implementation note}
%in principle, |maximize| and |minimize| can be thought of
%as reductions where the initial value is the smallest (or largest)
%value that the accumulation variable can assume.  because lisp's
%bignums can represent arbitrary integers, these clauses cannot be
%implemented as reductions in general.  if, however, the type of 
%~expr~ or ~var~ can be determined to be a fixnum
%or a float, \iter\ will implement the clause as a true reduction,
%using one of the constants |most-negative-fixnum|, |%most-positive-fixnum|, 
%|most-negative-short-float|, etc. as appropriate.
%\end{note}

\subsubsection{Accumulations}

All the predefined accumulation clauses add values to a sequence.  If
the sequence is a list, they all 
have the property that the partial list is kept in the correct order
and available for inspection at any point in the loop.

\begin{clauses}

\clausex{collect}{~expr~ \opt\ |into| ~var~ |at| ~place~ |result-type| ~type~}
Produces a sequence of the values of ~expr~ on each iteration. ~place~
indicates where the next value of ~expr~ is added to the list and may
be one of the symbols |start|, |beginning| (a synonym for |start|) or
|end|.  The symbol may be quoted, but need not be.  The default is
|end|.  For example,
\begin{program}
(iter (for i from 1 to 5)
      (collect i))
\end{program}
produces |(1 2 3 4 5)|, whereas
\begin{program}
(iter (for i from 1 to 5)
      (collect i at beginning))
\end{program}
produces |(5 4 3 2 1)| (and is likely to be faster in most Common Lisp
implementations).

\cpar If ~type~ is provided, it should be a subtype of |sequence|.
The default is |list|.  Specifying a type other than |list| will
result in |collect| returning a sequence of that type.  ~However,~ the
type of the sequence being constructed when inside the loop body is
undefined when a non-|list| type is specified.  (As with ~place~,
quoting ~type~ is optional.)

\clausex{adjoining}{~expr~ \opt\ |into| ~var~ |test| ~test~ |at| ~place~ |result-type| ~type~}
Like |collect|, but only adds the value of ~expr~ if it is not
already present.  ~test,~ which defaults to |\#'eql|, is
the test to be used with |member|.

\startitem
\Clausex{appending}{~expr~ \opt\ |into| ~var~ |at| ~place~}
\Clausex{nconcing}{\ ~expr~ \opt\ |into| ~var~ |at| ~place~}
\Clausex{unioning}{\ ~expr~ \opt\ |into| ~var~ |test| ~test~ |at| ~place~}
\Clausex{nunioning}{~expr~ \opt\ |into| ~var~ |test| ~test~ |at| ~place~}
\finishitem
These are like |collect|, but behave like the Common Lisp functions
|append|, |nconc|, |union| or |nunion|.  
As in Common Lisp, they work only on lists.  Also as in Common Lisp,
|unioning| and |nunioning| assume that the value of ~expr~ contains no
duplicates. 

\clausex{accumulate}{~expr~ |by| ~func~ \opt\ |initial-value| ~init-val~ |into| ~var~}
This is a general-purpose accumulation clause. ~func~ should be a
function of two arguments, the value of ~expr~ and the value
accumulated so far in the iteration, and it should return the updated
value.  If no initial value is supplied, \nil\ is used.  

%\cpar If a ~var~ is not specified, you can get \iter\ to declare the
%type of the internal variable by putting a |the| expression around
%~func.~  see section \ref{types}.

\cpar The differences between |accumulate| and |reducing| are slight.
One difference is that the functions take their arguments in a
different order.  Another is that in the absence of ~init-val~,
|accumulate| will use \nil, whereas |reducing| will generate different
code that avoids any dependence on the initial value.
The reason for having both clauses is that one usually
thinks of reductions (like |sum|) and accumulations (like |collect|)
as different beasts.

\end{clauses}


\subsubsection{Finders}

A ~finder~ is a clause whose value is an expression that meets some
condition.

\begin{clauses}

\clause{|finding| ~expr~ |such-that| ~test~ \opt\ |into| ~var~ |on-failure| ~failure-value~}
\clindexx{finding}{such-that}
If ~test~ (which is an expression) ever evaluates to \nonnil, the loop
is terminated, the 
epilogue code is run and the value of ~expr~ is returned.  Otherwise,
\nil\ (or ~failure-value,~ if provided) is returned.  If ~var~ is
provided, it will have either the \nonnil\ value of ~expr~ or
~failure-value~ when the epilogue code is run.

\cpar As a special case, if the ~test~ expression is a sharp-quoted
function, then it is applied to ~expr~ instead of being simply
evaluated.  E.g. |(finding x such-that \#'evenp)| is equivalent to
|(finding x such-that (evenp x))|.  

%\cpar although ~test~ need have nothing to do with ~%expr~ as in
%|(finding j such-that (> i 3))|, it usually 
%will: |(finding (length el) such-that (oddp (length el)))|.  to
%avoid performing the |length| computation twice, you could write
%|(finding (length el) such-that \#'oddp)| or |(finding (length
%el) such-that 'oddp)|; for these cases, \iter\ generates code that
%executes ~expr~ only once.  the code for |\#'oddp|
%is slightly different from that for {\lisp 'oddp}; see the discussion
%under {\lisp for\dots in} and {\lisp for\dots on}.

\cpar |On-failure| is a misnomer. Because it is always evaluated, it behaves
more like the default third argument to the |gethash| function. As a result,
|on-failure (error "Not found")| makes no sense. Instead, the clauses |leave|
or |thereis| can be used in conjunction with |finally| as follows:
\begin{program}
(iter (for x in '(1 2 3))
      (if (evenp x) (leave x))
      (finally (error "not found")))
\end{program}

\cpar This clause may appear multiple times when all defaults are
identical. It can also be used together with either |always|/|never| or
|thereis| if their defaults match. More specifically, |on-failure nil| is
compatible with |thereis|, while |on-failure t| is compatible with |always|
and |never| clauses.
\begin{program}
(iter (for i in '(7 -4 2 -3))
      (if (plusp i)
	  (finding    i  such-that (evenp i))
          (finding (- i) such-that (oddp i))))
\end{program}

\startitem
\Clause{|finding| ~expr~ |maximizing| ~m-expr~ \opt\ |into| ~var~}
\clindexx{finding}{maximizing}
\Clause{|finding| ~expr~ |minimizing| ~m-expr~ \opt\ |into| ~var~}
\clindexx{finding}{minimizing}
\finishitem
Computes the extremum (maximum or minimum) value of ~m-expr~ over all
iterations, and returns the value of ~expr~ corresponding to the
extremum.  ~expr~ is evaluated inside the loop at the time the new
extremum is established.  If ~m-expr~ is never evaluated (due to, for
example, being embedded in a conditional clause), then the returned
value depends on the type, if any, of ~expr~ (or ~var,~ if
one is supplied).  If there is no type, the returned
value will be \nil; if the type is numeric, the returned value will be
zero.

\cpar For these two clauses, ~var~ may be a list of two
symbols; in that case, the first is used to record ~expr~ and
the second, ~m-expr.~  

\cpar As with |finding\dots such-that|, if ~m-expr~ is a sharp-quoted
function, then it is called on ~expr~ instead of being evaluated.

\end{clauses}

\subsubsection{Boolean Tests}

\begin{clauses}

\clausex{first-iteration-p}{}
Returns |t| in the first cycle of the loop, otherwise \nil.

\clausex{first-time-p}{}
Returns |t| the first time the expression is evaluated, and then \nil\ forever.
This clause comes handy when printing (optional) elements separated
by a comma:

\begin{program}
(iter (for el in '(nil 1 2 nil 3))
      (when el
        (unless (first-time-p)
          (princ ", "))
        (princ el)))
\end{program}
produces |"1, 2, 3"|.

\end{clauses}

\subsubsection{Aggregated Boolean Tests}

\begin{clauses}

\clausex{always}{~expr~}
If ~expr~ ever evaluates to
\nil, then \nil\ is immediately returned; the epilogue code is not
executed.  If ~expr~ never evaluates to \nil, the epilogue code
is executed and the last value of ~expr~ (or |t| if ~expr~ was never
evaluated) is returned (whereas |loop| would constantly return |t|).

% mention last evaluated clause when multiple always clauses?

\clausex{never}{~expr~}
Like |(always (not ~expr~))|, except it does not influence the last
value returned by a possible other |always| clause. That is,
\begin{program}
(iter (repeat 2)
      (always 2)
      (never nil)) \yields 2 ; not t
\end{program}

\clausex{thereis}{~expr~}
If ~expr~ is ever \nonnil,
its value is immediately returned without running epilogue code.
Otherwise, the epilogue code is performed and \nil\ is returned.

This clause cannot be used together with |always| or |never|, because their
defaults are opposed (similarly, |(loop always 3 thereis nil)| refuses to
compile in some implementations of |loop|).

\end{clauses}


\subsection{Control Flow}
\label{control-flow}
Several clauses can be used to alter the usual flow of control in a loop.

Note: the clauses of this and subsequent sections don't adhere to \iter's
usual syntax, but instead use standard Common Lisp syntax.  Hence the
format for describing syntax subsequently is
like the standard format used in the Common Lisp manual, not like the
descriptions of clauses above.

\begin{clauses}

\clausex{finish}{}
Stops the loop and runs the epilogue code.

%for example:
%
%\begin{program}
%(iter (with answer = nil)
%         (initially (make-a-mess))
%         (for i from 1 to 10)
%         (when (correct? i)
%           (setq answer i)
%           (finish))
%         (finally (cleanup)))
%\end{program}
%
%this code will execute |cleanup| whether or not the test |(correct?
%i)| ever succeeds. 
%the (more elegant) formulation,
%\begin{program}
%(iter (initially (make-a-mess))
%         (for i from 1 to 10)
%         (finding i such-that (correct? i))
%         (finally (cleanup)))
%\end{program}
%would not execute |cleanup| if |(correct? i)| succeeded; it
%would do an immediate return.

\clausex{leave}{\opt\ ~value~}
Immediately returns ~value~ (default \nil) from the current \iter\
form, skipping the epilogue code.  Equivalent to using |return-from|.

\clausex{next-iteration}{}
Skips the remainder of the loop body and begins the next iteration of
the loop.

\clausex{while}{~expr~}
If ~expr~ ever evaluates to \nil, the loop is terminated and the
epilogue code executed.  Equivalent to |(if (not ~expr~) (finish))|.

\clausex{until}{~expr~}
Equivalent to |(if ~expr~ (finish))|.

\clausex{if-first-time}{~then~ \opt\ ~else~}
If this clause is being executed for the first time in this invocation
of the \iter\ form, then the ~then~ code is evaluated; otherwise the
~else~ code is evaluated.

\cpar |(for ~var~ first ~expr1~ then ~expr2~)| is almost equivalent to
\begin{program}
(if-first-time (dsetq ~var~ ~expr1~)
               (dsetq ~var~ ~expr2~))
\end{program}
The only difference is that the |for| version makes ~var~ available
for use with |for\dots previous|.

\end{clauses}

\subsection{Code Placement}
When fine control is desired over where code appears in a loop
generated by \iter, the following special clauses may be useful.
They are all subject to code-motion problems (see section
\ref{code-movement}).  

\begin{clauses}

\clausex{initially}{|\&rest| ~forms~}
The lisp ~forms~ are placed in the prologue section of the loop, where
they are executed once, before the loop body is entered.

\clausex{after-each}{|\&rest| ~forms~}
The ~forms~ are placed at the end of the loop body, where they
are executed after each iteration.  Unlike the other clauses in this
section, ~forms~ may contain \iter\ clauses.

\clausex{else}{|\&rest| ~forms~}
The lisp ~forms~ are placed in the epilogue section of the loop, where they
are executed if this |else| clause is never met during execution of the
loop and the loop terminates normally.

\clausex{finally}{|\&rest| ~forms~}
The lisp ~forms~ are placed in the epilogue section of the loop, where
they are executed after the loop has terminated normally.

\clausex{finally-protected}{|\&rest| ~forms~}
The lisp ~forms~ are placed in the second form of an unwind-protect
outside the loop.  They are always executed after the loop has
terminated, regardless of how the termination occurred.

\end{clauses}


%%%%%%%%%%%%%%%%%%%%%%%%%%%%%%%%%%%%%%%%%%%%%%%%%%%%%%%%%%%%%%%%
\section{Other Features}

\subsection{Multiple Accumulations}
\label{multiple}

\begin{sloppypar}
It is permitted to have more than one clause accumulate into the same
variable, as in the following:
\begin{program}
(iter (for i from 1 to 10)
      (collect i into nums)
      (collect (sqrt i) into nums)
      (finally (return nums)))
\end{program}
Clauses can only accumulate into the same variable if they are
compatible.  |collect|, |adjoining|, |appending|, |nconcing|,
|unioning| and |nunioning| are compatible with each other; |sum|,
|multiply| and |counting| are compatible; |always| and |never| are compatible;
|finding| \dots |such-that| is compatible with either |thereis| or |always|
and |never| when their defaults match; and |maximize| and |minimize| clauses
are compatible only with other |maximize| and |minimize| clauses,
respectively.

%note that the same variable ~cannot~ be both an accumulation
%variable and an ordinary variable; there can be only one variable with
%a given name within an \iter\ form. 
\end{sloppypar}

\subsection{Named Blocks}

Like Common Lisp |block|s, \iter\ forms can be given names.  The
name should be a single symbol, and it must be the first form in the
\iter.  The generated code behaves exactly like a named block; in
particular, |(return-from ~name~)| can be used to exit it:
\begin{program}
(iter fred
      (for i from 1 to 10)
      (iter barney
            (for j from i to 10)
            (if (> (* i j) 17)
                (return-from fred j))))
\end{program}
An \iter\ form that is not given a name is implicitly named \nil.

Sometimes one would like to write an expression in an inner \iter\ form,
but have it processed by an outer \iter\ form.  This is possible with
the |in| clause.  

\begin{clauses}

\clausex{in}{~name~ |\&rest| ~forms~}
Evaluates ~forms~ as if they were part of the \iter\ form named
~name~.  In other words, \iter\ clauses are processed by the \iter\
form named ~name,~ and not by any \iter\ forms that occur inside ~name.~

\cpar As an example, consider the problem of collecting a list of the
elements in a two-dimensional array.  The naive solution,
\begin{program}
(iter (for i below (array-dimension ar 0))
      (iter (for j below (array-dimension ar 1))
            (collect (aref ar i j))))
\end{program}
\noindent is wrong because the list created by the inner \iter\ is simply
ignored by the outer one.  But using |in| we can write:
\begin{program}
(iter outer (for i below (array-dimension ar 0))
      (iter (for j below (array-dimension ar 1))
            (in outer (collect (aref ar i j)))))
\end{program}
\noindent which has the desired result.

\end{clauses}

\subsection{Destructuring}
\label{destructuring}

In many places within \iter\ clauses where a variable is expected, a
list can be written instead.  In these cases, the value to be assigned
is ~destructured~ according to the pattern
described by the list.  As a simple example, the clause
\begin{program}
(for (key . item) in alist)
\end{program}
\noindent will result in |key| being set to the |car| of
each element in |alist|, and |item| being set to the |cdr|.  The
pattern list may be nested to arbitrary depth, and (as the example
shows) need not be terminated with \nil; the only requirement is that
each leaf be a bindable symbol (or \nil, in which case no binding is
generated for that piece of the structure).

Sometimes, you might like to do the equivalent of a
|multiple-value-setq| in a clause.  This 
``multiple-value destructuring'' can be expressed by writing 
\linebreak |(values $pat_1$ $pat_2 \ldots$)| for a destructuring
pattern, as in 
\begin{program}
(for (values (a . b) c d) = (three-valued-function ...))
\end{program}
\begin{sloppypar}
Note that the $pat_i$ can themselves be destructuring patterns (though
not multiple-value destructuring patterns).  You can't do multiple-value
destructuring in a |with| clause; instead wrap the whole \iter\
form in a |multiple-value-bind|.
\end{sloppypar}

\begin{note}{Rationale}
There are subtle interactions between variable declarations and
evaluation order that make the correct implementation of
multiple-value destructuring in a |with| somewhat tricky.
\end{note}

The destructuring feature of \iter\ is available as a separate
mechanism, using the |dsetq| macro:

\begin{clauses}
\defmacro{dsetq}{template expr}
Performs destructuring of ~expr~ using ~template~.   May be used
outside of an \iter\ form. Yields the primary value of ~expr~.

\end{clauses}

\subsection{On-line Help}

\begin{sloppypar}
There is a limited facility for on-line help, in the form of the 
|display-iterate-clauses| function.
\end{sloppypar}

\begin{clauses}

\defun{display-iterate-clauses}{\opt\ clause-spec}
Displays a list of \iter\ clauses.  If ~clause-spec~ is not
provided, all clauses are shown; if it is a symbol, all clauses
beginning with that symbol are shown; and if it is a list of symbols,
all clauses for which ~clause-spec~ is a prefix are shown.

\end{clauses}

\subsection{Parallel Binding and Stepping}
\label{bindings}

The parallel binding and stepping of variables is a feature that
\iter\ does ~not~ have.  This section attempts to provide a rationale.

We say that two variables are bound ~in parallel~ if neither
binding shadows the other.  This is the usual semantics of |let|
(as opposed to |let*|).  Similarly, we can say that iteration
variables are stepped in parallel if neither variable is updated
before the other, conceptually speaking; in other words, if the code
to update each variable can reference the old values of both variables.

|loop| allows parallel binding of variables and parallel stepping of
driver variables.  My view is that if you are depending on the 
serial/parallel distinction, you are doing something obscure.  If you
need to bind 
variables in parallel using |with|, then you must be using a
variable name that shadows a name in the existing lexical environment.
Don't do that.  The most common use for parallel stepping is to track
the values of variables on the previous iteration, but in fact this
does not require parallel stepping at all; the following will work:
\begin{program}
(iter (for current in list)
      (for prev previous current)
      ...)
\end{program}


%%%%%%%%%%%%%%%%%%%%%%%%%%%%%%%%%%%%%%%%%%%%%%%%%%%%%%%%%%%%%%%%
\section{Types and Declarations}
\label{types}

\subsection{Discussion}

Sometimes efficiency dictates that the types of variables be declared.
This type information needs to be communicated to \iter\ so it can
bind variables to appropriate values.  Furthermore, \iter\ must often
generate internal variables invisible to the user; there needs to be a
way for these to be declared.

As an example, consider this code, which will return the number of
odd elements in |number-list|:
\begin{program}
(iter (for el in number-list)
      (count (oddp el)))
\end{program}
In processing this form,
\iter\ will create an internal variable, let us call it |list17|, to
hold the successive |cdr|s of |number-list|, and  
will bind the variable to |number-list|.  It will also generate a
default binding for |el|; only inside the body of the loop will |el|
be set to the |car| of |list17|.  Finally, \iter\ will generate a
variable, call it |result|, to hold the result of the count, and will
bind it to zero.

When dealing with type declarations, \iter\ observes one simple rule:
~it will never generate a declaration unless requested to do so.~  The
reason is that such declarations might mask errors in compiled code by
avoiding error-checks; the resulting problems would be doubly hard to
track down because the declarations would be hidden from the
programmer.  Of course, a compiler might omit error-checks even in the
absence of declarations, though this behavior can usually be avoided,
e.g. by saying |(declaim (optimize (safety 3)))|.

So, the above \iter\ form will generate code with no declarations.
But say we wish to declare the types of |el| and the internal
variables |list17| and |result|.  How is this done?

Declaring the type of |el| is easy, since the programmer knows
the variable's name:
\begin{program}
(iter (for el in number-list)
      (declare (fixnum el))
      (counting (oddp el)))
\end{program}
\iter\ can read variable type declarations like this one.  Before
processing any clauses, it scans the entire top-level form for type
declarations 
and records the types, so that variable bindings can be performed
correctly.  In this case, |el| will be bound to zero
instead of \nil.  Also, \iter\ collects all the top-level declarations
and puts them at the begining of the generated code, so it is not
necessary to place all declarations at the beginning of an \iter\
form; instead, they can be written near the variables whose types they
declare.  

Since \iter\ is not part of the compiler, it will not know
about declarations that occur outside an \iter\ form; these
declarations must be repeated inside the form.

Here is another way we could have declared the type of |el|:
\begin{program}
(iter (for (the fixnum el) in number-list)
      (counting (oddp el)))
\end{program}
\iter\ extends the Common Lisp |the|\lispindex{the}
form to apply to variables as well as value-producing forms; anywhere
a variable is allowed---in a |with| clause, as the iteration
variable in a driver clause, as the |into| argument of an
accumulation clause, even inside a destructuring template---you can
write |(the ~type~ ~symbol~)| instead.

There is one crucial difference between using a |the| form and
actually declaring the variable: explicit declarations are always
placed in the generated code, but type information from a |the|
form is not turned into an actual declaration unless you tell \iter\
to do so using |iterate:declare-variables|.  See below.

\begin{sloppypar}
Declaring the types of internal variables is harder than declaring the
types of explicitly mentioned variables, since their names
are unknown.  You do it by declaring |iterate:declare-variables|
somewhere inside the top level of the \iter\ form.  (This will also
generate declarations for variables declared using |the|.)
\iter\ does not provide much selectivity here: it's all or none. 
And unfortunately, since \iter\ is not privy to compiler information
but instead reads declarations itself, it will not hear if you
|(declaim (iterate:declare-variables))|.  Instead, set the variable 
|iterate::*always-declare-variables*| to |t| at
compile-time, using |eval-when|.
\end{sloppypar}

To determine the appropriate types for internal variables, \iter\ uses
three sources of information:

\begin{itemize}

\item Often, the particular clause dictates a certain type for a
variable; \iter\ will use this information when available.  In the
current example, the variable |list17| will be given the type
|list|, since that is the only type that makes sense; and the
variable |result| will be given the type |fixnum|, on the
assumption that you will not be counting high enough to need bignums.
You can override this assumption only by using and explicitly declaring a
variable: 
\begin{verbatim}
(iter (declare (iterate:declare-variables))
      (for el in number-list)
      (count (oddp el) into my-result)
      (declare (integer my-result))
      (finally (return my-result)))
\end{verbatim}

\begin{sloppypar}
Other examples of the type assumptions that \iter\ makes are: type
|list| for |into| variables of collection clauses; type |list| for
expressions that are to be destructured; type |vector| for the
variable holding the vector in a |for\dots in-vector| clause, and
similarly for |string| and the |for\dots in-string| clause;
and the implementation-dependent type
|(type-of array-dimension-limit)| for the index and limit
variables generated by sequence iteration drivers like |for\dots
in-vector| and |for\dots in-string| (but not |for\dots in-sequence|,
because it may be used to iterate over a list). 
\end{sloppypar}

\item Sometimes, \iter\ will examine expressions and try to determine
their types in a simple-minded way.  If the expression is
self-evaluating (like a number, for instance), \iter\ knows that the
expression's type is the same as the type of the value it denotes, so
it can use that type.  If the expression is of the form |(the ~type~
~expr~)|, \iter\ is smart enough to extract ~type~ and use it.
However, the current version of \iter\ does not  
examine declarations of function result types or do any type
inference.  It will not determine, for 
example, that the type of |(+ 3 4)| is |fixnum|, or even
|number|.

\item In some cases, the type of an internal variable should match the
type of some other variable.  For instance, \iter\ generates an
internal variable for |(f x)| in the
clause |(for i from 1 to (f x))|, and in the absence of other
information will give it the same type as |i|.  If, however, the
expression had been written |(the fixnum (f x))|, then \iter\
would have given the internal variable the type |fixnum|
regardless of |i|'s type.  The type incompatibility errors that
could arise in this situation are not checked for.

\end{itemize}

Note that if you do declare |iterate:declare-variables|, then
\iter\ may declare user variables as well as internal ones if they do
not already have declarations, though only for variables that it
binds.  For instance, in this code:

\begin{program}
(iter (declare (iterate:declare-variables))
      (for i from 1 to 10)
      (collect i into var))
\end{program}
the variable |var| will be declared to be of type |list|.


\subsection{Summary}

\iter\ understands standard Common Lisp variable type declarations
that occur within an \iter\ form and
will pass them through to the generated code.  If the declaration
|(iterate:declare-variables)|\lispindex{declare-variables}
appears at the top level of an 
\iter\ form, or if 
|iterate::*always-declare-variables*|\lispindex{*always-declare-variables*} 
is \nonnil, then \iter\ will use the type information gleaned from user
declarations, self-evaluating expressions and |the| expressions,
combined with reasonable assumptions, to determine variable
types and declare them.


%%%%%%%%%%%%%%%%%%%%%%%%%%%%%%%%%%%%%%%%%%%%%%%%%%%%%%%%%%%%%%%%
\section{Problems with Code Movement}
\label{code-movement}

Some \iter\ clauses, or parts of clauses, result in code being
moved from the location of the clause to other parts of the loop.
Drivers behave this way, as do code-placement clauses like |initially|
and |finally|.  When using these clauses, there is a danger of writing
an expression that makes sense in its apparent location but will be
invalid or have a different meaning in another location.  For example:
\begin{program}
(iter (for i from 1 to 10)
      (let ((x 3))
        (initially (setq x 4))))
\end{program}
While it may appear that the |x| of |(initially (setq x 4))| is the
same as the |x| of |(let ((x 3)) \dots|, in fact they are not:
|initially| moves its code outside the loop body, so |x| would refer
to a global variable.  Here is another example of the same problem:
\begin{program}
(iter (for i from 1 to 10)
      (let ((x 3))
        (collect i into x)))
\end{program}
If this code were executed, |collect| would create a binding for its
|x| at the top level of the \iter\ form that the |let| will shadow.

Happily, \iter\ is smart enough to catch these errors; it
walks all problematical code to ensure that free variables are not
bound inside the loop body, and checks all variables it binds for the
same problem.

However, some errors cannot be caught:

\begin{program}
(iter (with x = 3)
      (for el in list)
      (setq x 1)
      (reducing el by \#'+ initial-value x))
\end{program}
|reducing| moves its |initial-value| argument to the initialization
part of the loop in order to produce more efficient code.  Since
\iter\ does not perform data-flow analysis, it cannot determine that
|x| is changed inside the loop; all it can establish is that |x| is
not bound internally.  Hence this code will not signal an
error and will use $3$ as the initial value of the reduction.

The following list summarizes all cases that are subject to these code
motion and variable-shadowing problems.
\begin{itemize}
\item Any variable for which \iter\ creates a binding, including those
used in |with| and the |into| keyword of many clauses.

\begin{sloppypar}
\item The special clauses which place code: |initially|, |after-each|, |else|,
|finally| and |finally-protected|. 
\end{sloppypar}

\item The variables of a |next| or |do-next| form.

\item The |initially| arguments of |for\dots initially\dots then| and
|for\dots previous|. 

\item The |then| argument of |for\dots initially\dots then|.

\item The |initial-value| arguments of |reducing| and |accumulate|.

\item The |on-failure| argument of |finding\dots such-that|.
\end{itemize}

%%%%%%%%%%%%%%%%%%%%%%%%%%%%%%%%%%%%%%%%%%%%%%%%%%%%%%%%%%%%%%%%
\section{Differences Between {\tt Iterate} and {\tt Loop}}
|loop| contains a great deal of complexity which \iter\ tries to
avoid.  Hence many esoteric features of |loop| don't exist in \iter.
Other features have been carried over, but in a cleaned-up form.  
And of course, many new features have been added; they are not
mentioned in this list.

\begin{itemize}

\item \iter's syntax is more Lisp-like than |loop|'s, having a higher
density of parens.

\item The current implementation of \iter, unlike the current version
of  |loop| (as documented in {\em Common Lisp, 2nd Ed.\/}), is
extensible (see section \ref{extend}).  

\item |loop| puts the updates of all driver variables at the top of
the loop; \iter\ leaves them where the driver clauses appear.

\item While for the most part \iter\ clauses that resemble |loop| clauses
behave similarly, there are some differences.  For instance, there is
no |for\dots =\dots then| in \iter; instead use 
|for\dots initially\dots then|.  

\item |loop| binds the variable |it| at certain times
to allow pseudo-English expressions like |when ~expr~ return it|. 
In \iter, you must bind ~expr~ to a variable yourself.  Note that
|when ~expr~ return it| is like |thereis ~expr~| except that the latter is an
accumulation clause and therefore competes with other accumulations
(remember section \ref{multiple} above).

% repeat different behaviour of |always| clause here?

\item |loop| has a special |return| clause, illustrated in the
previous item.  \iter\ doesn't need one,  since an ordinary Lisp
|return| has the same effect.

\item |loop| allows for parallel binding and stepping of iteration
variables.  \iter\ does not.  (See section \ref{bindings}.)

\item |loop| and \iter\ handle variable type declarations very
differently.  |loop| provides a special syntax for declaring variable
types, and does not examine declarations.  Moreover, the standard
implementation of |loop| will
generate declarations when none are requested.
\iter\ parses standard Common Lisp type declarations, and will never
declare a variable itself unless declarations are 
specifically requested.

\end{itemize}

%%%%%%%%%%%%%%%%%%%%%%%%%%%%%%%%%%%%%%%%%%%%%%%%%%%%%%%%%%%%%%%%
\section{Rolling Your Own}
\label{extend}

\subsection{Introduction}

\iter\ is extensible---you can write new clauses that embody new
iteration patterns.  You might want to write a new driver clause for a
data structure of your own, or you might want to write a clause that
collects or manipulates elements in a way not provided by \iter. 

This section describes how to write clauses for \iter.  Writing a
clause is like writing a macro.  In fact, writing a clause ~is~
writing a macro: since \iter\ code-walks its body and macroexpands,
you can add new abstractions to \iter\ with good old |defmacro|.

Actually, there are two extensions you can make to \iter\ that are
even easier than writing a macro.  They are adding a synonym for an
existing clause and defining a driver clause for an indexable
sequence.  These can be done with |defsynonym| and
|defclause-sequence|, respectively.  See section \ref{aids}, below.

The rest of this section explains how to write macros that expand into
\iter\ clauses.
Here's how you could add a simplified version of \iter's |multiply|
clause, if \iter\ didn't already have one:
\begin{program}
(defmacro multiply (expr)
  `(reducing ,expr by \#'* initial-value 1))
\end{program}

If you found yourself summing the square of an expression often, you
might want to write a macro for that.  A first cut might be
\begin{program}
(defmacro sum-of-squares (expr)
  `(sum (* ,expr ,expr)))
\end{program}
but if you are an experienced macro writer, you will realize that this
code will evaluate ~expr~ twice, which is probably a bad idea.  A
better version would use a temporary:
\begin{program}
(defmacro sum-of-squares (expr)
  (let ((temp (gensym)))
    `(let ((,temp ,expr))
       (sum (* ,temp ,temp)))))
\end{program}
Although this may seem complex, it is just the sort of thing you'd
have to go through to write any macro, which illustrates the point of
this section: if you can write macros, you can extend \iter.

Our macros don't use \iter's keyword-argument
syntax.  We could just use keywords with |defmacro|, but we would
still not be using \iter's clause indexing mechanism.  Unlike Lisp,
which uses just the first symbol of a form to determine what function
to call, \iter\ individuates clauses by the list of required keywords.
For instance, |for\dots in| and |for\dots in-vector| are different clauses
implemented by distinct Lisp functions.  

To buy into this indexing scheme, as well as the keyword-argument
syntax, use |defmacro-clause|:

\begin{clauses}

\defmacro{defmacro-clause}{arglist |\&body| body}
Defines a new \iter\ clause.  ~arglist~ is a list of symbols which are
alternating keywords and arguments.  \opt\ may be used, and the list
may be terminated by |\&sequence|.  ~body~ is an ordinary macro body,
as with |defmacro|.  If the first form of ~body~ is a string, it is
considered a documentation string and will be shown by
|display-iterate-clauses|. |defmacro-clause| will signal an error if 
defining the clause would result in an ambiguity.  E.g. you cannot
define the clause |for\dots from| because there would be no way to
distinguish it from a use of the |for| clause with optional keyword |from|.

\end{clauses}

\medskip

Here is |multiply| using |defmacro-clause|.  The keywords are capitalized
for readability.
\begin{program}
(defmacro-clause (MULTIPLY expr \opt\ INTO var)
  `(reducing ,expr by \#'* into ,var initial-value 1))
\end{program}
You don't have to worry about the case when |var| is not supplied; for
any clause with an |into| keyword, saying |into nil| is equivalent to
omitting the |into| entirely.

As another, more extended example, consider the fairly common
iteration pattern that 
involves finding the sequence element that maximizes (or minimizes) some
function.  \iter\ provides this as |finding\dots maximizing|, but it's
instructive to see how to write it.
Here, in pseudocode, is how you might write such a loop for
maximizing a function F:

\begin{quote}
\begin{tabbing}
set variable MAX-VAL to NIL; \\
set variable WINNER to NIL; \\
for\=\ each element EL in the sequence \\
\>  if\=\ MAX-VAL\=\ is NIL or F(EL) $>$ MAX-VAL then \\
\>\>    set MAX-VAL to F(EL); \\
\>\>    set WINNER to EL; \\
\>  end if; \\
end for; \\
return WINNER.
\end{tabbing}
\end{quote}

Here is the macro:
\begin{program}
(defmacro-clause (FINDING expr MAXIMIZING func \opt\ INTO var)
  (let ((max-val (gensym))
        (temp1 (gensym))
        (temp2 (gensym))
        (winner (or var iterate::*result-var*)))
    `(progn 
       (with ,max-val = nil)
       (with ,winner = nil)
       (cond 
        ((null ,max-val)
         (setq ,winner ,expr)
         (setq ,max-val (funcall ,func ,winner))
        (t
         (let* ((,temp1 ,expr)
                (,temp2 (funcall ,func ,temp1)))
           (when (> ,temp2 ,max-val)
             (setq ,max-val ,temp2)
             (setq ,winner ,temp1))))))
       (finally (leave ,winner)))))
\end{program}
Note that if no |into| variable is supplied, we use
|iterate::*result-var*|, which contains the internal variable into
which all clauses place their results.  If this variable is bound by
some clause, then \iter\ will return its value automatically;
otherwise, \nil\ will be returned.  

\subsection{Writing Drivers}

In principle, drivers can be implemented just as easily as other
\iter\ clauses.  In practice, they are a little harder to get right. 
As an example, consider writing a driver that
iterates over all the 
elements of a vector, ignoring its fill-pointer.  |for\dots in-vector|
won't work for this, because it observes the fill-pointer.  It's necessary to
use |array-dimension| instead of |length| to obtain the size of the
vector.  Here is one approach:

\begin{program}
(defmacro-clause (FOR var IN-WHOLE-VECTOR v)
  "All the elements of a vector (disregards fill-pointer)"
  (let ((vect (gensym))
        (index (gensym)))
    `(progn
       (with ,vect = ,v)
       (for ,index from 0 below (array-dimension ,vect 0))
       (for ,var = (aref ,vect ,index)))))
\end{program}
Note that we immediately put |v| in a variable, in case it is an
expression.  Again, this is just good Lisp macrology.  It also has a
subtle effect on the semantics of the driver: |v| is evaluated only
once, at the beginning of the loop, so changes to |v| in the loop have
no effect on the driver.  Similarly, the bounds for numerical iteration
e.g. the above |array-dimension| are also evaluated once only.  This is how
all of \iter's drivers work.


There is an important point concerning the |progn| in this code.  We
need the |progn|, of course, because we are returning several forms,
one of which is a driver.  But \iter\ drivers must occur at top-level.
Is this code in error?  No, because ~top-level~ is defined in \iter\
to include forms inside a |progn|.  This is just the definition of
top-level that Common Lisp uses, and for the same reason: to allow
macros to return multiple forms at top-level.

While our |for\dots in-whole-vector| clause will work, it is not
ideal.  In particular, it does not support generating.  Do do so, we
need to use |for\dots next| or |for\dots do-next|.  The job is
simplified by the |defmacro-driver| macro.

\begin{clauses}

\defmacro{defmacro-driver}{arglist |\&body| body}
Defines a driver clause in
both the |for| and |generate| forms, and provides a parameter
|generate| which ~body~ can examine to determine how it was invoked.
~arglist~ is as in |defmacro-clause|, and should begin with the symbol
|for|. 

\end{clauses}

With |defmacro-driver|, our driver looks like this:
\begin{program}
(defmacro-driver (FOR var IN-WHOLE-VECTOR v)
  "All the elements of a vector (disregards fill-pointer)"
   (let ((vect (gensym))
         (end (gensym))
         (index (gensym))
         (kwd (if generate 'generate 'for)))
     `(progn
        (with ,vect = ,v)
        (with ,end = (array-dimension ,vect 0))
        (with ,index = -1)
        (,kwd ,var next (progn (incf ,index)
                               (if (>= ,index ,end) (terminate))
                               (aref ,vect ,index))))))
\end{program}


We are still missing one thing: the |\&sequence| keywords.  
We can get them easily enough, by writing 
\begin{program}
(defmacro-driver (FOR var IN-WHOLE-VECTOR v \&sequence) 
  ...)
\end{program}
We can now refer to parameters |from|, |to|, |by|, etc. which contain
either the values for the corresponding keyword, or \nil\ if the
keyword was not supplied.  Implementing the right code for these
keywords is cumbersome but not difficult; it is left as an exercise.
But before you begin, see |defclause-sequence| below for an easier way.
       
\subsection{Extensibility Aids}
\label{aids}

This section documents assorted features that may be of use in
extending \iter.

\begin{clauses}

\defunexpvar{*result-var*}
Holds the variable that is used to return a value as a result of the
\iter\ form.  You may examine this and use it in a |with| clause, but
you should not change it.

\defmacro{defsynonym}{syn word}
 Makes ~syn~ a synonym for the existing \iter\ keyword ~word.~  Only
the first word in each clause can have synonyms. 

\defmacroxx{defclause-sequence}{element-name index-name}{|\&key|
access-fn size-fn sequence-type}{element-type
element-doc-string index-doc-string}  
Provides
a simple way to define sequence clauses.  Generates two
clauses, one for iterating over the sequence's elements, the other
for iterating over its indices.  The first symbol of both
clauses will have print-name |for|.
~element-name~  and ~index-name~ should be symbols.
~element-name~ is the second keyword of the element iterator (typically of
the form 
|in-~sequence-type~|), and ~index-name~ is the second keyword
of the index-iterator (typically of the form
|index-of-~sequence-type~|).  Either name may be 
\nil, in which case the corresponding clause is not defined.  If both
symbols are supplied, they should be in the same package.  The |for|
that begins the clauses will be in this package.

\cpar ~access-fn~ is the function to be used to
access elements of the sequence in the element iterator.  The function
should take two 
arguments, a sequence and an index, and return the appropriate element.
~size-fn~ should denote a function of one argument, a sequence, that
returns its size.  Both ~access-fn~ and ~size-fn~ are required for the
element iterator, but only ~size-fn~ is needed for the index iterator.

\cpar The ~sequence-type~ and ~element-type~ keywords are used to
suggest types for the variables 
used to hold the sequence and the
sequence elements, respectively.  The usual rules about \iter's
treatment of variable type declarations apply (see section \ref{types}).

\cpar ~element-doc-string~ and ~index-doc-string~ are
the documentation strings, for use with |display-iterate-clauses|. 

\cpar The generated element-iterator performs destructuring on the
element variable.

\cpar As an example, the above |for\dots in-whole-vector| example
could have been written:
\begin{program}
(defclause-sequence IN-WHOLE-VECTOR INDEX-OF-WHOLE-VECTOR
  :access-fn 'aref
  :size-fn \#'(lambda (v) (array-dimension v 0))
  :sequence-type 'vector
  :element-type t
  :element-doc-string 
     "Elements of a vector, disregarding fill-pointer"
  :index-doc-string 
     "Indices of vector, disregarding fill-pointer")
\end{program}

\end{clauses}

\subsection{Subtleties}

There are some subtleties to be aware of when writing \iter\ clauses.
First, the code returned by your macros may be |nconc|'ed into a list,
so you should always returned freshly consed lists, rather than
constants.  Second, \iter\ matches clauses by using |eq| on the first
symbol and |string=| on the subsequent ones, so the package of the
first symbol of a clause is relevant.  All of the clauses in this manual
have their first word in the \iter\ package.  
You can use the package system in the usual way to shadow
\iter\ clauses without replacing them.

%%% say more here, about the badness that only the first word of a
%%% clause is packagey.

\section{Non-portable Extensions to Iterate (Contribs)}
\label{contribs}

Currently, there is only one non-portable extension to iterate in the
distribution: iterate-pg. If you have made an extension that depends
on non-portable features, feel free to send them to |asf@boinkor.net|
for inclusion in the iterate distribution.

\subsection{An SQL query driver for iterate}

The pg package by Eric Marsden (see |http://cliki.net/pg|) provides an
interface to the PostgreSQL database. Using the \iterpg\ extension, it
is possible to handle the results of SQL queries using \iter.

This usage example should give you an idea of how to use it:

\begin{program}
(pg:with-pg-connection (c "somedb" "someuser")
  (iter (for (impl version date) in-relation "select * from version"
                                 on-connection *dbconn*)
        (collect version)))
\end{program}

To use the extension via |ASDF|, simply make your system depend on the
|iterate-pg| system instead of the |iterate| system. To load it
manually, use:

\begin{program}
  (asdf:oos 'asdf:load-op :iterate-pg)
\end{program}

%%%%%%%%%%%%%%%%%%%%%%%%%%%%%%%%%%%%%%%%%%%%%%%%%%%%%%%%%%%%%%%%
\section{Obtaining {\tt Iterate}}

\begin{sloppypar}
The information in this chapter is obsolete but included for
completeness's sake; Currently, the most up-to-date information on
\iter\ can be found at |http://boinkor.net/iterate.html|.
\end{sloppypar}

\begin{sloppypar}
\iter\ currently runs on Lisp Machines, and on
HP's, Sun3's and Sparcstations under Lucid.  
\iter\ source and binaries are available at the MIT AI Lab in the
subdirectories of |/src/local/lisplib/|.  The source file,
|iterate.lisp|, is also available for anonymous FTP in the directory
|/com/fpt/pub/| on the machine |TRIX.AI.MIT.EDU| (Internet number
128.52.32.6).  If you are unable to obtain |iterate| in one of these
ways, send mail to |jba@ai.mit.edu| and I will send you the source
file. 
\end{sloppypar}

\begin{sloppypar}
\iter\ resides in the |iterate| package (nickname |iter|).  Just say
\linebreak |(use-package :iterate)| to make all the necessary symbols
available. 
If a symbol is not exported, it appears in this manual with an
``|iterate::|'' prefix.
\end{sloppypar}

Send bug reports to |bug-iterate@ai.mit.edu|.  The |info-iterate|
mailing list will have notices of changes and problems; to have
yourself added, send mail to |info-iterate-request@ai.mit.edu|.


\medskip

\begin{flushleft}
 \bf Acknowledgements
\end{flushleft}
\smallskip

Richard Waters provided invaluable criticism which spurred me to improve
\iter\ greatly.  As early users, David Clemens, Oren Etzioni and Jeff
Siskind helped ferret out many bugs.

%\begin{theindex}
% The index files must be generated with the genindex program
% or from makeindex iterate-manual.idx which creates iterate-manual.ind.

%\baselineskip\normalbaselineskip
%\advance\baselineskip by -2pt

\input{iterate-manual.ind}

%\underline{Clauses}
%\input{iterate-manual.clindex}
%
%\indexspace
%\underline{Lisp}
%\input{iterate-manual.lispindex}

%\end{theindex}


\end{document}


% arch-tag: "c0f871a6-313c-11d8-abb9-000c76244c24"


%\underline{Clauses}
%% Fix ``.'' spacing in lists
\documentstyle[12pt,aimemo]{article}

\setlength{\textwidth}{6.0in}
\setlength{\oddsidemargin}{0.25in}
\setlength{\evensidemargin}{0.25in}
\setlength{\topmargin}{0pt}
\setlength{\headsep}{24pt}
\setlength{\textheight}{8.5in}

\let\origpar=\par

% somehow, LaTex fucks with things so that the plain Tex \obeyspaces
% doesn't work.  This is the same, except we use an mbox containing
% a space instead of just a space.

\newcommand{\spc}{\mbox{ }}
\newcommand{\tAb}{\mbox{        }}
\def\obspaces{\catcode`\ =\active\catcode`\^^I=\active}
{\obspaces \global\let =\spc\global\let^^I=\tAb}

\newenvironment{program}{\medskip\samepage\tt\obeylines\obspaces}{\medskip}

% use | for \tt
\catcode`\|=13
\newif\ifvbar
\def|{\ifvbar
         \endgroup\vbarfalse
      \else 
         \begingroup\tt\vbartrue 
      \fi}



\newcommand{\lisp}{\tt}
\newcommand{\iter}{|iterate|}
\newcommand{\iterpg}{|iterate-pg|}
\newcommand{\setf}{|setf|}
\newcommand{\nil}{|nil|}
\newcommand{\nonnil}{non-\nil}
\newcommand{\opt}{{\lisp \&optional}}
\newcommand{\yields}{$\Rightarrow$}



\newbox\Dots
\setbox\Dots=\hbox{\small ...}
\renewcommand{\dots}{\copy\Dots}

% Here's a way to define an environment which will treat all
% paragraphs inside it in the same manner.  It's, as near as I can
% figure, the way that Latex does things.  Using \everypar won't work;
% apparently Latex resets that a lot.  What you have to do is: at the
% top of the file do \let\origpar=\par.  In the
% begin part of the environment, do the parshape (or whatever), then say
% \def\par{{\origpar}}.  The extra braces are necessary.   That's all
% you have to do; at the end of the environment things will return to
% normal automatically.  Why does this all work?  I have no idea.
% Another important thing: there should be a blank line (i.e. a \par)
% between the end of the last paragraph and the \end{environment}.

\newlength{\clindent}
\setlength{\clindent}{0.5in}
\newlength{\clparindent}
\setlength{\clparindent}{\parindent}

\newenvironment{clauses}{\advance\linewidth -\parindent
                        \hangindent\clindent\relax
                         \hangafter=0
                          \parindent=0pt
                          \def\par{{\origpar}}}{}

\newcommand{\cpar}{\hspace \clparindent}

\newcommand{\startitem}{\bigskip\pagebreak[3]}
\newcommand{\finishitem}{\smallskip}

\newcommand{\clause}[1]{\startitem\Clause{#1}\finishitem}
\newcommand{\Clause}[1]{\hbox{#1}}
\newcommand{\clindex}[1]{\index{{\lisp #1}}}
\newcommand{\clindexx}[2]{\index{{\lisp #1\dots #2}}}
\newcommand{\clindexxx}[3]{\index{{\lisp #1\dots #2\dots #3}}}
\newcommand{\clausex}[2]{\startitem\Clausex{#1}{#2}\finishitem}
\newcommand{\Clausex}[2]{\Clause{|#1| #2}\clindex{#1}}

\newcommand{\defvar}[1]{\startitem\deff{#1}{}{Variable}\finishitem}
\newcommand{\defunexpvar}[1]{\startitem\deffnoindex{iterate::#1}{}{Variable}
                                        \lispindex{#1}\finishitem}
\newcommand{\defun}[2]{\startitem\deff{#1}{#2}{Function}\finishitem}
\newcommand{\defmacro}[2]{\startitem\deff{#1}{#2}{Macro}\finishitem}
\newcommand{\defunx}[3]{\startitem\deffx{#1}{#2}{#3}{Function}\finishitem}
\newcommand{\defmacrox}[3]{\startitem\deffx{#1}{#2}{#3}{Macro}\finishitem}
\newcommand{\defunxx}[4]{\startitem\deffxx{#1}{#2}{#3}{#4}{Function}\finishitem}
\newcommand{\defmacroxx}[4]{\startitem\deffxx{#1}{#2}{#3}{#4}{Macro}\finishitem}

\newcommand{\deff}[3]{\deffnoindex{#1}{#2}{#3}\lispindex{#1}}
\newcommand{\deffnoindex}[3]{\hbox to \hsize{{\tt #1} {\it #2}\hfill [{\it #3}]}}
\newcommand{\deffx}[4]{\deff{#1}{#2}{#4}\moreargs{#1}{#3}}
\newcommand{\deffxx}[5]{\deffx{#1}{#2}{#3}{#5}\moreargs{#1}{#4}}

\newcommand{\moreargs}[2]{\hbox to \hsize{\phantom{\lisp #1} {\it #2}\hfill}}

\newcommand{\lispindex}[1]{\index{{\lisp #1}}}

 \newenvironment{note}[1]{\pagebreak[2]\bigskip
                          \hrule\smallskip\small
                         {\setlength{\parindent}{0pt}
                           \par{\bf #1:}}}{\smallskip\hrule\bigskip}



\makeindex

%%%%%%%%%%%%%%%%%%%%%%%%%%%%%%%%%%%%%%%%%%%%%%%%%%%%%%%%%%%%%%%%

%\thispagestyle{empty}

%\vspace*{2in}

%\title{The |Iterate| Manual}
%\author{Jonathan Amsterdam}

%\maketitle

%\thispagestyle{empty}

%\begin{center}
%\Large *** DRAFT ***
%\end{center}
%This manual is in beta-test. It will become an AI Memo in a couple of
%months.  I would appreciate any comments or
%corrections.  You can reach me at email address |jba@ai.ai.mit.edu|.

\begin{document}

\def\writtenby{Jonathan Amsterdam}
\def\contractno{This report describes research done at the Artificial
Intelligence Laboratory of the Massachusetts Institute of Technology.
Support for the laboratory's artificial intelligence research is
provided in part by the Advanced Research Projects Agency of the
Department of Defense under Office
of Naval Research contract N00014-85-K-0124.}

\AIMEMO{1236}{\Large The Iterate Manual \\[1in]}
{This is the manual for version 1.4 of \iter, a powerful iteration
macro for Common  
Lisp. \iter\ is similar to {\tt loop} but provides numerous additional
features, is well integrated with Lisp, and is extensible. 
\vfill }


% use ~ for \em
\catcode`\~=13
\newif\ifem
\def~{\ifem
         \/\endgroup\emfalse
      \else 
         \begingroup\em\emtrue 
      \fi}



%\pagebreak
%\setcounter{page}{1}

%\thispagestyle{empty}


\tableofcontents

\pagebreak
\pagestyle{headings}

\section{Introduction}

\begin{sloppypar}
This manual describes
\iter, a powerful iteration facility for Common Lisp.
\iter\ provides abstractions for many common iteration
patterns and allows for the definition of additional patterns.  \iter\ is
a macro that expands into ordinary Lisp at
compile-time, so it is more efficient than higher-order
functions like |map| and |reduce.|  
While it is similar to |loop|, \iter\ offers a more Lisp-like syntax
and enhanced extensibility.
(For a more complete
comparison of \iter\ with other iteration constructs, see MIT AI Lab Working
Paper No. 324, ~Don't Loop, Iterate.~)
\end{sloppypar}

An \iter\ form consists of the symbol |iter|\footnote{You can also use
|iterate|, but |iter| is preferred because it avoids potential conflicts with
possible future additions to Common Lisp, and because it saves
horizontal space when writing code.}
followed by one or more
forms, some of which may be \iter\ 
~clauses.~  Here is a simple example of \iter\ which collects the
numbers from 1 to 10 into a list, and returns the list.  The return
value is shown following the arrow.
\begin{program}
(iter (for i from 1 to 10)
      (collect i))          \yields (1 2 3 4 5 6 7 8 9 10)
\end{program}
This form contains two clauses: a |for| clause that steps the
variable |i| over the integers from 1 to 10, and a |collect|
clause that accumulates its argument into a list.  With a few
exceptions, all \iter\ clauses have the same format: alternating
symbols (called ~keywords~) and expressions (called
~arguments~).  The syntax and terminology are those of Common
Lisp's keyword lambda lists.  One difference is that \iter's keywords
do not have to begin with a colon---though they may, except for the
first symbol of a clause.  So you can also write  |(for i :from
1 :to 10)| if you prefer.

Any Lisp form can appear in the body of an \iter, where it will have
its usual meaning.  \iter\ walks the entire body, expanding macros,
and recognizing clauses at any level.  This example collects all the
odd numbers in a list:
\begin{program}
(iter (for el in list)
      (if (and (numberp el) (oddp el))
          (collect el)))
\end{program}

There are clauses for iterating over numbers, lists, arrays and other
objects, and for
collecting, summing, counting, maximizing and other useful operations.
\iter\ also supports the creation of new variable bindings, stepping
over multiple sequences at once, destructuring, and compiler
declarations of variable types.  The following example illustrates
some of these features: 
\begin{program}
(iter (for (key . item) in alist)
         (for i from 0)
         (declare (fixnum i))
         (collect (cons i key)))
\end{program}

This loop takes the keys of an alist and returns a new alist
associating the keys with their positions in the original list. The
compiler declaration for |i| will appear in the generated code
in the appropriate place.

%%%%%%%%%%%%%%%%%%%%%%%%%%%%%%%%%%%%%%%%%%%%%%%%%%%%%%%%%%%%%%%%
\section{Clauses}

Most of \iter's clauses will be familiar to |loop| programmers.  
(|loop| is an iteration macro that has been incorporated into Common
Lisp.  Seen Guy Steele's {\em Common Lisp, 2nd Edition}.)
In
nearly all cases they behave the same as their |loop| counterparts, so
a |loop| user can switch to \iter\ with little pain (and much gain).

All clauses with the standard keyword-argument syntax consist of two
parts: a ~required~ part, 
containing keywords that must be present and in the right order; and
an ~optional~ part, containing keywords that may be omitted and,
if present, may occur in any order.  In the descriptions below, the
parts are separated by the Lisp lambda-list keyword \opt.


\subsection{Drivers}

An iteration-driving clause
conceptually causes the iteration to go forward.  Driver clauses in
\iter\ allow iteration over numbers, lists, vectors, hashtables, packages,
files and streams.  Iteration-driving clauses must
appear at the top level of an \iter\ form; they cannot be nested
inside another clause.  The driver variable is updated at the point
where the driver clause occurs.  Before the clause is executed for the
first time, the value of the variable is undefined.

%Also, regardless of where the driver clause
%appears in the body, the driver variable is stepped at the top of the
%loop; hence it is stylistically preferable, though
%not required, to place driver clauses at the beginning of the \iter.

Multiple drivers may appear in a single \iter\ form, in which case all
of the driver variables are updated each time through the loop, in the
order in which the clauses appear.  The first driver to terminate will
terminate the entire loop.

In all cases, the value of the driver variable on exit from the loop,
including within the epilogue code (see the |finally| clause), is
undefined.

All the parameters of a driver clause are evaluated once, before the
loop begins.  Hence it is not possible to change the bounds or other
properties of an iteration by side-effect from within the loop.

With one exception, driver clauses begin with the word |for| (or
the synonym |as|) and mention an iteration variable, which is
given a binding within the \iter\ form.
The exception is |repeat|, which just executes a loop a
specified number of times:

\begin{clauses}

\clausex{repeat}{~n~}
Repeats the loop ~n~ times.  For example:
\begin{program}
(iter (repeat 100)
      (print "I will not talk in class."))
\end{program}
\cpar If $n \leq 0$, the
loop will never be executed.  If ~n~ is not an integer, the
actual number of executions will be $\lceil n \rceil$.

\end{clauses}

\pagebreak[3]

\subsubsection{Numerical Iteration}

\begin{clauses}
\clausex{for}{~var~ |\&sequence|} 
The general form for iterating over a sequence of numbers requires a
variable and, optionally, one or more keywords that provide the bounds
and step size of the iteration. The |\&sequence|
lambda-list keyword is a shorthand for these sequence keywords.  They are:
|from|, |upfrom|, |downfrom|, |to|, |downto|, |above|, |below| and |by|.
|from| provides the starting value for ~var~
and defaults to zero. |to| provides a final value and implies that the
successive values of ~var~ will be increasing; |downto| implies that
they will be decreasing.  The loop terminates when ~var~ passes
the final value (i.e. becomes smaller or larger than it, depending on
the direction of iteration); in other  words, the loop body
will never be 
executed for values of ~var~ past the final value.
|below| and |above| are similar to |to| and
|downto|, except that the loop terminates when ~var~ equals or passes
the final value.

\cpar If no final value is specified, the variable will be stepped
forever.  Using |from| or |upfrom| will result in increasing
values, while |downfrom| will give decreasing values.

\cpar On each iteration,
~var~ is incremented or decremented by the value of the sequence
keyword |by|, which defaults to 1.  It should always be a positive
number, even for downward iterations.

\cpar In the following examples, the sequence of numbers generated is shown
next to the clause.

\begin{program}
(for i upfrom 0) \yields\ 0 1 2 \ldots
(for i from 5)   \yields\ 5 6 7 \ldots    ; either from or upfrom is okay
(for i downfrom 0) \yields\ 0 -1 -2 \ldots
(for i from 1 to 3) \yields\ 1 2 3
(for i from 1 below 3) \yields\ 1 2
(for i from 1 to 3 by 2) \yields\ 1 3
(for i from 1 below 3 by 2) \yields\ 1
(for i from 5 downto 3) \yields\ 5 4 3
\end{program}

\end{clauses}

\subsubsection{Sequence Iteration}

There are a number of clauses for iterating over sequences.  In all of
them, the argument following |for| may be a list instead of a
symbol, in which case destructuring is performed.  See section
\ref{destructuring}. 

\begin{clauses}

\clause{|for| ~var~ |in| ~list~ \opt\ |by| ~step-function~}
\clindexx{for}{in}
~var~ is set to successive elements of list.  
~step-function,~ which 
defaults to |cdr|, is used to obtain the next sublist. 

\clause{|for| ~var~ |on| ~list~ \opt\ |by| ~step-function~}
\clindexx{for}{on}
~var~ is set to successive sublists of list. 
~step-function~ (default |cdr|) is used as in |for\dots in|.

\end{clauses}

\medskip

These two clauses use |atom| to test for the end
of a list.  Hence, given a list whose final |cdr| is not \nil,
they will silently ignore the last |cdr|.  Other choices are
|endp|,  which would signal an error, and |null|, which
would probably result in an error somewhere else.  If you wish to use
an end-test other than |atom|, set the variable
|iterate::*list-end-test*|\lispindex{*list-end-test*} to the name of
the desired function. 

\begin{clauses}
\clause{|for| ~var~ |in-vector| ~vector~ |\&sequence|}
\clindexx{for}{in-vector}
~var~ takes on successive elements from ~vector.~   The vector's
fill-pointer is observed.  Here and in subsequent clauses, the |\&sequence|
keywords include |with-index|, which takes a symbol as argument
and uses it for the index variable instead of an internally generated
symbol.  The other |\&sequence| keywords behave as in numerical
iteration, except that the default iteration bounds are the bounds of
the vector.  E.g. in |(for i in-vector v downto 3)|,
|i| will start off being bound to the last element in |v|,
and will  
be set to preceding elements down to and including the element with
index 3.

\clause{|for| ~var~ |in-sequence| ~seq~ |\&sequence|}
\clindexx{for}{in-sequence}
This uses Common Lisp's generalized sequence functions, |elt|
and |length|, to obtain elements and determine the length of
~seq.~   Hence it will work for any sequence, including lists, and will
observe the fill-pointers of vectors.

\clause{|for| ~var~ |in-string| ~string~ |\&sequence|}
\clindexx{for}{in-string}
~var~ is set to successive characters of ~string.~

\startitem
\Clause{|for| ~var~ |index-of-vector| ~vector~ |\&sequence|}
\clindexx{for}{index-of-vector}
\Clause{|for| ~var~ |index-of-sequence| ~sequence~ |\&sequence|}
\clindexx{for}{index-of-sequence}
\Clause{|for| ~var~ |index-of-string| ~string~ |\&sequence|}
\clindexx{for}{index-of-string}
\finishitem
~var~ is set to successive indices of the sequence.
These clauses avoid the overhead of accessing the sequence elements
for those applications where they do not need to be examined, or are
examined rarely.  They admit all the optional keywords of the other
sequence drivers except the (redundant) |with-index| keyword.

\clause{|for| ~(key value)~ |in-hashtable| ~table~}
\clindexx{for}{in-hashtable}

~key~ and ~value,~ which must appear as shown in a list
and may be destructuring templates, are set to the keys and values of
~table~.  If ~key~ is |nil|, then the hashtable's keys will be ignored;
similarly for ~value~.
The order in which 
elements of ~table~ will be retrieved is unpredictable.

\clause{|for| ~var~ |in-package| ~package~ \opt\ |external-only| ~ext~}
\clindexx{for}{in-package}
Iterates over all the symbols in ~package,~ or over only the
external symbols if ~ext~ is specified and non-|nil|.  ~ext~ is not
evaluated.  The same symbol may appear more than once.

\clause{|for| ~var~ |in-packages| ~packages~ \opt\ |having-access| ~symbol-types~}
\clindexx{for}{in-packages}
Iterates over all the symbols from the list of packages denoted by the
descriptor ~packages~ and having accessibility (or visibility) given by
~symbol-types~.  This defaults to the list |(:external :internal :inherited)|
and is not evaluated.  ~var~ must be a list of up to three variables: in each
iteration, these will be set to a symbol, its access-type and package (as per
|with-package-iterator| in ANSI-CL).  The same symbol may appear more than
once.

\end{clauses}

\begin{clauses}

\clause{|for| ~var~ |in-file| ~name~ \opt\ |using| ~reader~}
\clindexx{for}{in-file}
Opens the file ~name~ (which may be a string or pathname) for input,
and iterates 
over its contents. ~reader~ defaults to |read|, so by default ~var~
will be bound to the successive forms in the file.  The \iter\ body is
wrapped in an unwind-protect to ensure that the file is closed no
matter how the \iter\ is exited.

\clause{|for| ~var~ |in-stream| ~stream~ \opt\ |using| ~reader~}
\clindexx{for}{in-stream}
Like |for\dots in-file|, except that ~stream~ should be an existing
stream object that supports input operations.

\end{clauses}

\subsubsection{Generalized Drivers}

These are primarily useful for writing drivers that can also be used
as generators (see section \ref{generators}, below).
 
\begin{clauses}

\clause{|for| ~var~ |next| ~expr~}
\clindexx{for}{next}

~var~ is set to ~expr~ each time through the loop.  Destructuring is
performed.  When the clause is used as a generator, ~expr~ is the code
that is executed when |(next ~var~)| is 
encountered (see section \ref{generators}, below).
~expr~ should compute the first value for ~var~, as well
as all subsequent values, and is responsible for terminating the loop.
For compatibility with future versions of \iter, this termination
should be done with |terminate|\clindex{terminate}, which can be
considered a synonym for |finish| (see section \ref{control-flow}).

\cpar As an example, the following clauses are equivalent to |(for
i from 1 to 10)|:
\begin{program}
(initially (setq i 0))
(for i next (if (> i 10) (terminate) (incf i)))
\end{program}

\clause{|for| ~var~ |do-next| ~form~}
\clindexx{for}{do-next}
~form~ is evaluated each time through the loop.  Its value is ~not~
set to ~var~; that is ~form's~ job. ~var~ is only present so that
\iter\ knows it is a driver variable.  \linebreak |(for ~var~ next
~expr~)| is equivalent to |(for ~var~ do-next (dsetq ~var~ ~expr~))|.
(See section \ref{destructuring} for an explanation of |dsetq|.) 

\end{clauses}

\subsubsection{Generators}
\label{generators}

In all of the above clauses, the driver variable is updated on each
iteration.
Sometimes it is desirable to have greater control over updating.  
For instance, consider the problem of associating numbers, in
increasing order and with no gaps, with the
\nonnil\ elements of a list.  One obvious first pass at writing this is:

\begin{program}
(iter (for el in list)
      (for i upfrom 1)
      (if el (collect (cons el i))))
\end{program}
But on the list |(a b nil c)| this produces |((a . 1) (b . 2) (c .
4))| instead of the desired |((a . 1) (b . 2) (c . 3))|.  The problem
is that |i| is incremented each time through the loop, even when |el|
is |nil|.  

The problem could be solved elegantly if we could step |i| only when
we wished to.  This
can be accomplished for any \iter\ driver by writing |generate| (or
its synonym |generating|)
instead of |for|.  Doing so produces a ~generator~---a driver whose
values are yielded explicitly.  To obtain the next value of a
generator variable ~v~, write \linebreak |(next ~v~)|.  The value of a |next|
form is the next value of ~v~, as determined by its associated driver
clause.  |next| also has the side-effect of updating ~v~ to that
value.  If there is no next value, |next| will terminate the loop,
just as with a normal driver.


Using generators, we can now write our example like this:
\begin{program}
(iter (for el in list)
      (generate i upfrom 1)
      (if el (collect (cons el (next i)))))
\end{program}
Now |i| is updated only when |(next i)| is executed, and this occurs
only when |el| is \nonnil.

To better understand the relationship between ordinary drivers and
generators, observe that we can rewrite an ordinary driver using its
generator form immediately followed by |next|, as this example shows:
\begin{program}
(iter (generating i from 1 to 10)
      (next i)
      ...)
\end{program}
Provided that the loop body contains no |(next i)| forms, this will
behave just as if we had written |(for i from 1 to 10)|.

We can still refer to a driver variable ~v~ without using |next|; in
this case, its value is that given to it by the last evaluation of
|(next ~v~)|.  Before |(next ~v~)| has been called the first time, the
value of ~v~ is undefined.  

This semantics is more flexible than
one in which ~v~ begins the loop bound to its first value and calls
of |next| supply subsequent values, because it means the loop will not
terminate too soon if the generator's sequence is empty.  For
instance, consider the following code, which tags \nonnil\ elements of
a list using a list of tags, and also counts the null elements.
(We assume there are at least as many tags as \nonnil\ elements.)
\begin{program}
(let* ((counter 0)
       (tagged-list (iter (for el in list)
                          (generating tag in tag-list) 
                          (if (null el)
                              (incf counter)
                              (collect (cons el (next tag)))))))
  ...)
\end{program}
It may be that there are just as many tags as non-null elements of
|list|.  If all the elements of |list| are null, we still want the
counting to proceed, even though |tag-list| is \nil.  If |tag| had to be
assigned its first value before the loop begins, we would have had to
terminate the loop before the first iteration, since when |tag-list|
is \nil, |tag| 
has no first value.  With the existing semantics, however, |(next
tag)| will never execute, so the iteration will cover all the elements of
|list|. 
        
When the ``variable'' of a driver clause is actually a destructuring
template containing several variables, all the variables are eligible
for use with |next|.  As before, |(next ~v~)| evaluates to ~v's~ next
value; but the effect is to update all of the template's variables.
For instance, the following code will return the list |(a 2 c)|.
\begin{program}
(iter (generating (key . item) in '((a . 1) (b . 2) (c . 3)))
      (collect (next key))
      (collect (next item)))
\end{program}

Only driver clauses with variables can be made into generators.  This
includes all clauses mentioned so far except for |repeat|.  It does
~not~ include |for\dots previous|, |for\dots =|,
|for\dots initially\dots then| or |for\dots first\dots then| (see
below). 

\subsubsection{Previous Values of Driver Variables}

Often one would like to access the value of a variable on a previous
iteration.  \iter\ provides a special clause for accomplishing this.

\begin{clauses}

\clause{|for| ~pvar~ |previous| ~var~ \opt\ |initially| ~init~
|back| ~n~}
\clindexx{for}{previous}
Sets ~pvar~ to the previous value of ~var,~ which should be a driver
variable, a variable from another |for\dots previous| clause, or a
variable established by a |for\dots =|, 
|for\dots initially\dots then| or |for\dots first\dots then| clause
(see section \ref{setting}).
Initially, ~pvar~ is given the value ~init~ (which defaults to \nil).  
The ~init~ expression will be moved outside the loop body, so it
should not depend on anything computed within the loop.
~pvar~ retains the value of ~init~ until ~var~ is set to its second
value, at which point ~pvar~ is set to ~var's~ first value; and so on.  

\cpar The
argument ~n~ to |back| must be a constant, positive integer, and
defaults to 1.  It determines how many iterations back ~pvar~ should
track ~var~.  For example, when ~n~ is 2, then ~pvar~ will be assigned
~var's~ first value when ~var~ is set to its third value.

\cpar A |for\dots previous| clause may occur after or before its
associated driver clause. |for\dots previous| works with generators as
well as ordinary drivers.

\pagebreak[3]

\cpar Example:
\begin{program}
(iter (for el in '(1 2 3 4))
      (for p-el previous el)
      (for pp-el previous p-el initially 0)
      (collect pp-el))
\end{program}
This evaluates to |(0 0 1 2)|.  It could have been written more
economically as 
\begin{program}
(iter (for el in '(1 2 3 4))
      (for pp-el previous el back 2 initially 0)
      (collect pp-el))
\end{program}


\end{clauses}

\subsection{Variable Binding and Setting}
\label{setting}

Several clauses exist for establishing new variable bindings or for
setting variables in the loop.  They all support destructuring.

\begin{clauses}

\clausex{with}{~var~ \opt\ |=| ~value~}
Causes ~var~ to be bound to value before the loop body is entered.
If ~value~ is not supplied, ~var~ assumes a default
binding, which will be 
\nil\ in the absence of declarations.  Also, if ~value~ is not
supplied, no destructuring is performed; instead, ~var~ may be a list of
symbols, all of which are given default 
bindings.  If ~value~ is supplied, ~var~ is bound to it, with
destructuring.  

\cpar Because |with|
creates bindings whose scope includes the entire \iter\ form, it is
good style to put all |with| clauses at the beginning.

\cpar Successive occurrences of |with| result in sequential
bindings (as with 
|let*|).  There is no way to obtain parallel bindings; see
section \ref{bindings} for a rationale.  


\clause{|for| ~var~ |=| ~expr~}
\clindexx{for}{=}
On each iteration, ~expr~ is evaluated and ~var~ is set
to its value. 

\cpar This clause may appear to do the same thing as |for\dots next|.
In fact, they are quite different.  |for\dots =| provides only three
services: it sets up a binding for ~var~, sets it to ~expr~ on each
iteration, and makes it possible to use |for\dots previous| with
~var~.  |for\dots next| provides these services in addition to the
ability to turn the driver into a generator.
%Also, the code which sets ~var~ appears in the loop body in the same
%place as the |for\dots =| clause; the code for |for\dots next| appears
%at the top of the loop, as with other drivers (except when being used
%as a generator).

\clause{|for| ~var~ |initially| ~init-expr~ |then| ~then-expr~}
\clindexxx{for}{initially}{then}
Before the loop begins, ~var~ is set to ~init-expr;~ on all
iterations after the first it is set to ~then-expr.~  This clause
must occur at top-level.  ~init-expr~ will be moved outside the loop
body and ~then-expr~ will be moved to the end of the loop body, so
they are subject to code motion problems (see section
\ref{code-movement}). 

\cpar This clause may appear to be similar to |for\dots next|, but in
fact they differ significantly.  |for\dots initially\dots then| is
typically used to give ~var~ its first value before the loop begins,
and subsequent values on following iterations.  This is incompatible
with generators, whose first value and subsequent values must all be
computed by |(next ~var~)|.  Also, the update of ~var~ in 
|for\dots initially\dots then| does not occur at the location of the
clause. 
Use |for\dots initially\dots then| for
one-shot computations where its idiom is more convenient, but use
|for\dots next| for extending \iter\ with new drivers (see section
\ref{extend}).

\clause{|for| ~var~ |first| ~first-expr~ |then| ~then-expr~}
\clindexxx{for}{first}{then}
The first time through the loop, ~var~ is set to ~first-expr;~ on
subsequent iterations, it is set to ~then-expr.~  This differs from
|for\dots initially| in that ~var~ is set to ~first-expr~
inside the loop body, 
so ~first-expr~ may depend on the results of other clauses.  For
instance, 
\begin{program}
(iter (for num in list)
      (for i first num then (1+ i))
      ...)
\end{program}
will set |i| to the first element of |list| on the first
iteration, whereas 
\begin{program}
(iter (for num in list)
      (for i initially num then (1+ i))
      ...)
\end{program}
is probably erroneous; |i| will be bound to |num|'s
default binding (usually \nil) for the first iteration.

\end{clauses}

\begin{note}{Compatibility note}
|loop|'s |for\dots =| works like \iter's, but |loop| used the syntax
|for\dots =\dots then| to mean |for\dots initially\dots then|.  It was
felt that these two operations were sufficiently different to warrant
different keywords.

Also, the |for| in the above three clauses is misleading,
since none is true driver (e.g. none has a corresponding |generate|
form).  |setting| would have been a better choice, but |for| was used
to retain some compatibility with |loop|.  
\end{note}


%%%%%%%%%%%%%%%%%%%%%%%%%%%%%%%%
\subsection{Gathering Clauses}
Many of \iter's clauses accumulate values into a variable, or set a
variable under certain conditions.  At the end of the loop, this
variable contains the desired result.  All these clauses have an
optional |into| keyword, whose argument should be a symbol.  If the
|into| keyword is not supplied, the accumulation variable will be
internally generated and its value will be returned at the end of the
loop; if a variable is specified, that variable is used for the
accumulation, and is not returned as a result---it is up to the user
to return it explicitly, in the loop's epilogue code (see |finally|).
It is safe to examine the accumulation variable during the loop, but
it should not be modified.

These clauses all begin with a verb.  When the verb does
not conflict with an existing Common Lisp function, then it may be
used in either its infinitival or present-participle form (e.g. |sum|,
|summing|).  However, when there is a conflict with Common Lisp, only
the present-participle form may be used (e.g. |unioning|).  This is to
prevent \iter\ clauses from clashing with Common Lisp
functions. 

%although these clauses are described as ``producing a value,'' it is a
%mistake to think of the lisp list representing the clause as a
%value-producing form in the usual way.  clauses may legally be written
%where a value is expected, e.g. |(setq x (sum i))|, but the
%lisp value of a clause in such a context is undefined.

\subsubsection{Reductions}

~Reduction~ is an extremely common iteration pattern in which
the results of successive applications of a
binary operation are accumulated.
For example, a loop that computes the sum of the
elements of a list is performing a reduction with the addition
operation.  This could be written in Common Lisp as  |(reduce \#'+
list)| or with \iter\ as
\begin{program}
(iter (for el in list)
      (sum el))
\end{program}

\begin{clauses}

\clausex{sum}{~expr~ \opt\ |into| ~var~}
Each time through the loop, ~expr~ is evaluated and added to a
variable, which is bound initially to zero.  If ~expr~ has a type,
it is ~not~ used as the type of the sum variable, which is always
|number|.  To get the result variable to be of a more specific
type, use an explicit variable, as in
\begin{program}
(iter (for el in number-list)
      (sum el into x)
      (declare (fixnum x))
      (finally (return x)))
\end{program}

\clausex{multiply}{~expr~ \opt\ |into| ~var~}
Like |sum|, but the initial value of the result variable is $1$,
and the variable is updated by multiplying ~expr~ into it.

\clausex{counting}{~expr~ \opt\ |into| ~var~}
~expr~ is evaluated on each iteration.  If it is \nonnil, the
accumulation variable, initially zero, is incremented.

\startitem
\Clausex{maximize}{~expr~ \opt\ |into| ~var~}
\Clausex{minimize}{~expr~ \opt\ |into| ~var~}
\finishitem
~expr~ is evaluated on each iteration and its extremum (maximum or
minimum) is stored in the accumulation variable.  If ~expr~ is never
evaluated, then the result is \nil\ (if the accumulation variable is
untyped) or $0$ (if it has a numeric type).

\clausex{reducing}{~expr~ |by| ~func~ \opt\ |initial-value| ~init-val~ |into| ~var~}
This is a general way to perform reductions. ~func~ should be a
function of two arguments, the first of which will be the value
computed so far and
the second of which will be the value of ~expr~.  It should return the
new value.
|reducing| is roughly equivalent to the
Common Lisp |(reduce ~func~ ~sequence~ :key ~expr-function~)|, where
~expr-function~ is used to derive values from the successive elements
of ~sequence~.

\cpar If the |reducing| clause is never executed, the result is
undefined.

\cpar It is not necessary to provide an
initial value, but better code can be generated if one is supplied.
Regardless of its location in the \iter\ body, ~init-val~ will be
evaluated before the loop is entered, so it should not depend on any
value computed inside the \iter\ form.  

%\cpar if a ~var~ is not specified, you can get \iter\ to declare the
%type of the internal variable by putting a |the| expression around
%~func.~  see section \ref{types}.

\end{clauses}

%\begin{note}{implementation note}
%in principle, |maximize| and |minimize| can be thought of
%as reductions where the initial value is the smallest (or largest)
%value that the accumulation variable can assume.  because lisp's
%bignums can represent arbitrary integers, these clauses cannot be
%implemented as reductions in general.  if, however, the type of 
%~expr~ or ~var~ can be determined to be a fixnum
%or a float, \iter\ will implement the clause as a true reduction,
%using one of the constants |most-negative-fixnum|, |%most-positive-fixnum|, 
%|most-negative-short-float|, etc. as appropriate.
%\end{note}

\subsubsection{Accumulations}

All the predefined accumulation clauses add values to a sequence.  If
the sequence is a list, they all 
have the property that the partial list is kept in the correct order
and available for inspection at any point in the loop.

\begin{clauses}

\clausex{collect}{~expr~ \opt\ |into| ~var~ |at| ~place~ |result-type| ~type~}
Produces a sequence of the values of ~expr~ on each iteration. ~place~
indicates where the next value of ~expr~ is added to the list and may
be one of the symbols |start|, |beginning| (a synonym for |start|) or
|end|.  The symbol may be quoted, but need not be.  The default is
|end|.  For example,
\begin{program}
(iter (for i from 1 to 5)
      (collect i))
\end{program}
produces |(1 2 3 4 5)|, whereas
\begin{program}
(iter (for i from 1 to 5)
      (collect i at beginning))
\end{program}
produces |(5 4 3 2 1)| (and is likely to be faster in most Common Lisp
implementations).

\cpar If ~type~ is provided, it should be a subtype of |sequence|.
The default is |list|.  Specifying a type other than |list| will
result in |collect| returning a sequence of that type.  ~However,~ the
type of the sequence being constructed when inside the loop body is
undefined when a non-|list| type is specified.  (As with ~place~,
quoting ~type~ is optional.)

\clausex{adjoining}{~expr~ \opt\ |into| ~var~ |test| ~test~ |at| ~place~ |result-type| ~type~}
Like |collect|, but only adds the value of ~expr~ if it is not
already present.  ~test,~ which defaults to |\#'eql|, is
the test to be used with |member|.

\startitem
\Clausex{appending}{~expr~ \opt\ |into| ~var~ |at| ~place~}
\Clausex{nconcing}{\ ~expr~ \opt\ |into| ~var~ |at| ~place~}
\Clausex{unioning}{\ ~expr~ \opt\ |into| ~var~ |test| ~test~ |at| ~place~}
\Clausex{nunioning}{~expr~ \opt\ |into| ~var~ |test| ~test~ |at| ~place~}
\finishitem
These are like |collect|, but behave like the Common Lisp functions
|append|, |nconc|, |union| or |nunion|.  
As in Common Lisp, they work only on lists.  Also as in Common Lisp,
|unioning| and |nunioning| assume that the value of ~expr~ contains no
duplicates. 

\clausex{accumulate}{~expr~ |by| ~func~ \opt\ |initial-value| ~init-val~ |into| ~var~}
This is a general-purpose accumulation clause. ~func~ should be a
function of two arguments, the value of ~expr~ and the value
accumulated so far in the iteration, and it should return the updated
value.  If no initial value is supplied, \nil\ is used.  

%\cpar If a ~var~ is not specified, you can get \iter\ to declare the
%type of the internal variable by putting a |the| expression around
%~func.~  see section \ref{types}.

\cpar The differences between |accumulate| and |reducing| are slight.
One difference is that the functions take their arguments in a
different order.  Another is that in the absence of ~init-val~,
|accumulate| will use \nil, whereas |reducing| will generate different
code that avoids any dependence on the initial value.
The reason for having both clauses is that one usually
thinks of reductions (like |sum|) and accumulations (like |collect|)
as different beasts.

\end{clauses}


\subsubsection{Finders}

A ~finder~ is a clause whose value is an expression that meets some
condition.

\begin{clauses}

\clause{|finding| ~expr~ |such-that| ~test~ \opt\ |into| ~var~ |on-failure| ~failure-value~}
\clindexx{finding}{such-that}
If ~test~ (which is an expression) ever evaluates to \nonnil, the loop
is terminated, the 
epilogue code is run and the value of ~expr~ is returned.  Otherwise,
\nil\ (or ~failure-value,~ if provided) is returned.  If ~var~ is
provided, it will have either the \nonnil\ value of ~expr~ or
~failure-value~ when the epilogue code is run.

\cpar As a special case, if the ~test~ expression is a sharp-quoted
function, then it is applied to ~expr~ instead of being simply
evaluated.  E.g. |(finding x such-that \#'evenp)| is equivalent to
|(finding x such-that (evenp x))|.  

%\cpar although ~test~ need have nothing to do with ~%expr~ as in
%|(finding j such-that (> i 3))|, it usually 
%will: |(finding (length el) such-that (oddp (length el)))|.  to
%avoid performing the |length| computation twice, you could write
%|(finding (length el) such-that \#'oddp)| or |(finding (length
%el) such-that 'oddp)|; for these cases, \iter\ generates code that
%executes ~expr~ only once.  the code for |\#'oddp|
%is slightly different from that for {\lisp 'oddp}; see the discussion
%under {\lisp for\dots in} and {\lisp for\dots on}.

\cpar |On-failure| is a misnomer. Because it is always evaluated, it behaves
more like the default third argument to the |gethash| function. As a result,
|on-failure (error "Not found")| makes no sense. Instead, the clauses |leave|
or |thereis| can be used in conjunction with |finally| as follows:
\begin{program}
(iter (for x in '(1 2 3))
      (if (evenp x) (leave x))
      (finally (error "not found")))
\end{program}

\cpar This clause may appear multiple times when all defaults are
identical. It can also be used together with either |always|/|never| or
|thereis| if their defaults match. More specifically, |on-failure nil| is
compatible with |thereis|, while |on-failure t| is compatible with |always|
and |never| clauses.
\begin{program}
(iter (for i in '(7 -4 2 -3))
      (if (plusp i)
	  (finding    i  such-that (evenp i))
          (finding (- i) such-that (oddp i))))
\end{program}

\startitem
\Clause{|finding| ~expr~ |maximizing| ~m-expr~ \opt\ |into| ~var~}
\clindexx{finding}{maximizing}
\Clause{|finding| ~expr~ |minimizing| ~m-expr~ \opt\ |into| ~var~}
\clindexx{finding}{minimizing}
\finishitem
Computes the extremum (maximum or minimum) value of ~m-expr~ over all
iterations, and returns the value of ~expr~ corresponding to the
extremum.  ~expr~ is evaluated inside the loop at the time the new
extremum is established.  If ~m-expr~ is never evaluated (due to, for
example, being embedded in a conditional clause), then the returned
value depends on the type, if any, of ~expr~ (or ~var,~ if
one is supplied).  If there is no type, the returned
value will be \nil; if the type is numeric, the returned value will be
zero.

\cpar For these two clauses, ~var~ may be a list of two
symbols; in that case, the first is used to record ~expr~ and
the second, ~m-expr.~  

\cpar As with |finding\dots such-that|, if ~m-expr~ is a sharp-quoted
function, then it is called on ~expr~ instead of being evaluated.

\end{clauses}

\subsubsection{Boolean Tests}

\begin{clauses}

\clausex{first-iteration-p}{}
Returns |t| in the first cycle of the loop, otherwise \nil.

\clausex{first-time-p}{}
Returns |t| the first time the expression is evaluated, and then \nil\ forever.
This clause comes handy when printing (optional) elements separated
by a comma:

\begin{program}
(iter (for el in '(nil 1 2 nil 3))
      (when el
        (unless (first-time-p)
          (princ ", "))
        (princ el)))
\end{program}
produces |"1, 2, 3"|.

\end{clauses}

\subsubsection{Aggregated Boolean Tests}

\begin{clauses}

\clausex{always}{~expr~}
If ~expr~ ever evaluates to
\nil, then \nil\ is immediately returned; the epilogue code is not
executed.  If ~expr~ never evaluates to \nil, the epilogue code
is executed and the last value of ~expr~ (or |t| if ~expr~ was never
evaluated) is returned (whereas |loop| would constantly return |t|).

% mention last evaluated clause when multiple always clauses?

\clausex{never}{~expr~}
Like |(always (not ~expr~))|, except it does not influence the last
value returned by a possible other |always| clause. That is,
\begin{program}
(iter (repeat 2)
      (always 2)
      (never nil)) \yields 2 ; not t
\end{program}

\clausex{thereis}{~expr~}
If ~expr~ is ever \nonnil,
its value is immediately returned without running epilogue code.
Otherwise, the epilogue code is performed and \nil\ is returned.

This clause cannot be used together with |always| or |never|, because their
defaults are opposed (similarly, |(loop always 3 thereis nil)| refuses to
compile in some implementations of |loop|).

\end{clauses}


\subsection{Control Flow}
\label{control-flow}
Several clauses can be used to alter the usual flow of control in a loop.

Note: the clauses of this and subsequent sections don't adhere to \iter's
usual syntax, but instead use standard Common Lisp syntax.  Hence the
format for describing syntax subsequently is
like the standard format used in the Common Lisp manual, not like the
descriptions of clauses above.

\begin{clauses}

\clausex{finish}{}
Stops the loop and runs the epilogue code.

%for example:
%
%\begin{program}
%(iter (with answer = nil)
%         (initially (make-a-mess))
%         (for i from 1 to 10)
%         (when (correct? i)
%           (setq answer i)
%           (finish))
%         (finally (cleanup)))
%\end{program}
%
%this code will execute |cleanup| whether or not the test |(correct?
%i)| ever succeeds. 
%the (more elegant) formulation,
%\begin{program}
%(iter (initially (make-a-mess))
%         (for i from 1 to 10)
%         (finding i such-that (correct? i))
%         (finally (cleanup)))
%\end{program}
%would not execute |cleanup| if |(correct? i)| succeeded; it
%would do an immediate return.

\clausex{leave}{\opt\ ~value~}
Immediately returns ~value~ (default \nil) from the current \iter\
form, skipping the epilogue code.  Equivalent to using |return-from|.

\clausex{next-iteration}{}
Skips the remainder of the loop body and begins the next iteration of
the loop.

\clausex{while}{~expr~}
If ~expr~ ever evaluates to \nil, the loop is terminated and the
epilogue code executed.  Equivalent to |(if (not ~expr~) (finish))|.

\clausex{until}{~expr~}
Equivalent to |(if ~expr~ (finish))|.

\clausex{if-first-time}{~then~ \opt\ ~else~}
If this clause is being executed for the first time in this invocation
of the \iter\ form, then the ~then~ code is evaluated; otherwise the
~else~ code is evaluated.

\cpar |(for ~var~ first ~expr1~ then ~expr2~)| is almost equivalent to
\begin{program}
(if-first-time (dsetq ~var~ ~expr1~)
               (dsetq ~var~ ~expr2~))
\end{program}
The only difference is that the |for| version makes ~var~ available
for use with |for\dots previous|.

\end{clauses}

\subsection{Code Placement}
When fine control is desired over where code appears in a loop
generated by \iter, the following special clauses may be useful.
They are all subject to code-motion problems (see section
\ref{code-movement}).  

\begin{clauses}

\clausex{initially}{|\&rest| ~forms~}
The lisp ~forms~ are placed in the prologue section of the loop, where
they are executed once, before the loop body is entered.

\clausex{after-each}{|\&rest| ~forms~}
The ~forms~ are placed at the end of the loop body, where they
are executed after each iteration.  Unlike the other clauses in this
section, ~forms~ may contain \iter\ clauses.

\clausex{else}{|\&rest| ~forms~}
The lisp ~forms~ are placed in the epilogue section of the loop, where they
are executed if this |else| clause is never met during execution of the
loop and the loop terminates normally.

\clausex{finally}{|\&rest| ~forms~}
The lisp ~forms~ are placed in the epilogue section of the loop, where
they are executed after the loop has terminated normally.

\clausex{finally-protected}{|\&rest| ~forms~}
The lisp ~forms~ are placed in the second form of an unwind-protect
outside the loop.  They are always executed after the loop has
terminated, regardless of how the termination occurred.

\end{clauses}


%%%%%%%%%%%%%%%%%%%%%%%%%%%%%%%%%%%%%%%%%%%%%%%%%%%%%%%%%%%%%%%%
\section{Other Features}

\subsection{Multiple Accumulations}
\label{multiple}

\begin{sloppypar}
It is permitted to have more than one clause accumulate into the same
variable, as in the following:
\begin{program}
(iter (for i from 1 to 10)
      (collect i into nums)
      (collect (sqrt i) into nums)
      (finally (return nums)))
\end{program}
Clauses can only accumulate into the same variable if they are
compatible.  |collect|, |adjoining|, |appending|, |nconcing|,
|unioning| and |nunioning| are compatible with each other; |sum|,
|multiply| and |counting| are compatible; |always| and |never| are compatible;
|finding| \dots |such-that| is compatible with either |thereis| or |always|
and |never| when their defaults match; and |maximize| and |minimize| clauses
are compatible only with other |maximize| and |minimize| clauses,
respectively.

%note that the same variable ~cannot~ be both an accumulation
%variable and an ordinary variable; there can be only one variable with
%a given name within an \iter\ form. 
\end{sloppypar}

\subsection{Named Blocks}

Like Common Lisp |block|s, \iter\ forms can be given names.  The
name should be a single symbol, and it must be the first form in the
\iter.  The generated code behaves exactly like a named block; in
particular, |(return-from ~name~)| can be used to exit it:
\begin{program}
(iter fred
      (for i from 1 to 10)
      (iter barney
            (for j from i to 10)
            (if (> (* i j) 17)
                (return-from fred j))))
\end{program}
An \iter\ form that is not given a name is implicitly named \nil.

Sometimes one would like to write an expression in an inner \iter\ form,
but have it processed by an outer \iter\ form.  This is possible with
the |in| clause.  

\begin{clauses}

\clausex{in}{~name~ |\&rest| ~forms~}
Evaluates ~forms~ as if they were part of the \iter\ form named
~name~.  In other words, \iter\ clauses are processed by the \iter\
form named ~name,~ and not by any \iter\ forms that occur inside ~name.~

\cpar As an example, consider the problem of collecting a list of the
elements in a two-dimensional array.  The naive solution,
\begin{program}
(iter (for i below (array-dimension ar 0))
      (iter (for j below (array-dimension ar 1))
            (collect (aref ar i j))))
\end{program}
\noindent is wrong because the list created by the inner \iter\ is simply
ignored by the outer one.  But using |in| we can write:
\begin{program}
(iter outer (for i below (array-dimension ar 0))
      (iter (for j below (array-dimension ar 1))
            (in outer (collect (aref ar i j)))))
\end{program}
\noindent which has the desired result.

\end{clauses}

\subsection{Destructuring}
\label{destructuring}

In many places within \iter\ clauses where a variable is expected, a
list can be written instead.  In these cases, the value to be assigned
is ~destructured~ according to the pattern
described by the list.  As a simple example, the clause
\begin{program}
(for (key . item) in alist)
\end{program}
\noindent will result in |key| being set to the |car| of
each element in |alist|, and |item| being set to the |cdr|.  The
pattern list may be nested to arbitrary depth, and (as the example
shows) need not be terminated with \nil; the only requirement is that
each leaf be a bindable symbol (or \nil, in which case no binding is
generated for that piece of the structure).

Sometimes, you might like to do the equivalent of a
|multiple-value-setq| in a clause.  This 
``multiple-value destructuring'' can be expressed by writing 
\linebreak |(values $pat_1$ $pat_2 \ldots$)| for a destructuring
pattern, as in 
\begin{program}
(for (values (a . b) c d) = (three-valued-function ...))
\end{program}
\begin{sloppypar}
Note that the $pat_i$ can themselves be destructuring patterns (though
not multiple-value destructuring patterns).  You can't do multiple-value
destructuring in a |with| clause; instead wrap the whole \iter\
form in a |multiple-value-bind|.
\end{sloppypar}

\begin{note}{Rationale}
There are subtle interactions between variable declarations and
evaluation order that make the correct implementation of
multiple-value destructuring in a |with| somewhat tricky.
\end{note}

The destructuring feature of \iter\ is available as a separate
mechanism, using the |dsetq| macro:

\begin{clauses}
\defmacro{dsetq}{template expr}
Performs destructuring of ~expr~ using ~template~.   May be used
outside of an \iter\ form. Yields the primary value of ~expr~.

\end{clauses}

\subsection{On-line Help}

\begin{sloppypar}
There is a limited facility for on-line help, in the form of the 
|display-iterate-clauses| function.
\end{sloppypar}

\begin{clauses}

\defun{display-iterate-clauses}{\opt\ clause-spec}
Displays a list of \iter\ clauses.  If ~clause-spec~ is not
provided, all clauses are shown; if it is a symbol, all clauses
beginning with that symbol are shown; and if it is a list of symbols,
all clauses for which ~clause-spec~ is a prefix are shown.

\end{clauses}

\subsection{Parallel Binding and Stepping}
\label{bindings}

The parallel binding and stepping of variables is a feature that
\iter\ does ~not~ have.  This section attempts to provide a rationale.

We say that two variables are bound ~in parallel~ if neither
binding shadows the other.  This is the usual semantics of |let|
(as opposed to |let*|).  Similarly, we can say that iteration
variables are stepped in parallel if neither variable is updated
before the other, conceptually speaking; in other words, if the code
to update each variable can reference the old values of both variables.

|loop| allows parallel binding of variables and parallel stepping of
driver variables.  My view is that if you are depending on the 
serial/parallel distinction, you are doing something obscure.  If you
need to bind 
variables in parallel using |with|, then you must be using a
variable name that shadows a name in the existing lexical environment.
Don't do that.  The most common use for parallel stepping is to track
the values of variables on the previous iteration, but in fact this
does not require parallel stepping at all; the following will work:
\begin{program}
(iter (for current in list)
      (for prev previous current)
      ...)
\end{program}


%%%%%%%%%%%%%%%%%%%%%%%%%%%%%%%%%%%%%%%%%%%%%%%%%%%%%%%%%%%%%%%%
\section{Types and Declarations}
\label{types}

\subsection{Discussion}

Sometimes efficiency dictates that the types of variables be declared.
This type information needs to be communicated to \iter\ so it can
bind variables to appropriate values.  Furthermore, \iter\ must often
generate internal variables invisible to the user; there needs to be a
way for these to be declared.

As an example, consider this code, which will return the number of
odd elements in |number-list|:
\begin{program}
(iter (for el in number-list)
      (count (oddp el)))
\end{program}
In processing this form,
\iter\ will create an internal variable, let us call it |list17|, to
hold the successive |cdr|s of |number-list|, and  
will bind the variable to |number-list|.  It will also generate a
default binding for |el|; only inside the body of the loop will |el|
be set to the |car| of |list17|.  Finally, \iter\ will generate a
variable, call it |result|, to hold the result of the count, and will
bind it to zero.

When dealing with type declarations, \iter\ observes one simple rule:
~it will never generate a declaration unless requested to do so.~  The
reason is that such declarations might mask errors in compiled code by
avoiding error-checks; the resulting problems would be doubly hard to
track down because the declarations would be hidden from the
programmer.  Of course, a compiler might omit error-checks even in the
absence of declarations, though this behavior can usually be avoided,
e.g. by saying |(declaim (optimize (safety 3)))|.

So, the above \iter\ form will generate code with no declarations.
But say we wish to declare the types of |el| and the internal
variables |list17| and |result|.  How is this done?

Declaring the type of |el| is easy, since the programmer knows
the variable's name:
\begin{program}
(iter (for el in number-list)
      (declare (fixnum el))
      (counting (oddp el)))
\end{program}
\iter\ can read variable type declarations like this one.  Before
processing any clauses, it scans the entire top-level form for type
declarations 
and records the types, so that variable bindings can be performed
correctly.  In this case, |el| will be bound to zero
instead of \nil.  Also, \iter\ collects all the top-level declarations
and puts them at the begining of the generated code, so it is not
necessary to place all declarations at the beginning of an \iter\
form; instead, they can be written near the variables whose types they
declare.  

Since \iter\ is not part of the compiler, it will not know
about declarations that occur outside an \iter\ form; these
declarations must be repeated inside the form.

Here is another way we could have declared the type of |el|:
\begin{program}
(iter (for (the fixnum el) in number-list)
      (counting (oddp el)))
\end{program}
\iter\ extends the Common Lisp |the|\lispindex{the}
form to apply to variables as well as value-producing forms; anywhere
a variable is allowed---in a |with| clause, as the iteration
variable in a driver clause, as the |into| argument of an
accumulation clause, even inside a destructuring template---you can
write |(the ~type~ ~symbol~)| instead.

There is one crucial difference between using a |the| form and
actually declaring the variable: explicit declarations are always
placed in the generated code, but type information from a |the|
form is not turned into an actual declaration unless you tell \iter\
to do so using |iterate:declare-variables|.  See below.

\begin{sloppypar}
Declaring the types of internal variables is harder than declaring the
types of explicitly mentioned variables, since their names
are unknown.  You do it by declaring |iterate:declare-variables|
somewhere inside the top level of the \iter\ form.  (This will also
generate declarations for variables declared using |the|.)
\iter\ does not provide much selectivity here: it's all or none. 
And unfortunately, since \iter\ is not privy to compiler information
but instead reads declarations itself, it will not hear if you
|(declaim (iterate:declare-variables))|.  Instead, set the variable 
|iterate::*always-declare-variables*| to |t| at
compile-time, using |eval-when|.
\end{sloppypar}

To determine the appropriate types for internal variables, \iter\ uses
three sources of information:

\begin{itemize}

\item Often, the particular clause dictates a certain type for a
variable; \iter\ will use this information when available.  In the
current example, the variable |list17| will be given the type
|list|, since that is the only type that makes sense; and the
variable |result| will be given the type |fixnum|, on the
assumption that you will not be counting high enough to need bignums.
You can override this assumption only by using and explicitly declaring a
variable: 
\begin{verbatim}
(iter (declare (iterate:declare-variables))
      (for el in number-list)
      (count (oddp el) into my-result)
      (declare (integer my-result))
      (finally (return my-result)))
\end{verbatim}

\begin{sloppypar}
Other examples of the type assumptions that \iter\ makes are: type
|list| for |into| variables of collection clauses; type |list| for
expressions that are to be destructured; type |vector| for the
variable holding the vector in a |for\dots in-vector| clause, and
similarly for |string| and the |for\dots in-string| clause;
and the implementation-dependent type
|(type-of array-dimension-limit)| for the index and limit
variables generated by sequence iteration drivers like |for\dots
in-vector| and |for\dots in-string| (but not |for\dots in-sequence|,
because it may be used to iterate over a list). 
\end{sloppypar}

\item Sometimes, \iter\ will examine expressions and try to determine
their types in a simple-minded way.  If the expression is
self-evaluating (like a number, for instance), \iter\ knows that the
expression's type is the same as the type of the value it denotes, so
it can use that type.  If the expression is of the form |(the ~type~
~expr~)|, \iter\ is smart enough to extract ~type~ and use it.
However, the current version of \iter\ does not  
examine declarations of function result types or do any type
inference.  It will not determine, for 
example, that the type of |(+ 3 4)| is |fixnum|, or even
|number|.

\item In some cases, the type of an internal variable should match the
type of some other variable.  For instance, \iter\ generates an
internal variable for |(f x)| in the
clause |(for i from 1 to (f x))|, and in the absence of other
information will give it the same type as |i|.  If, however, the
expression had been written |(the fixnum (f x))|, then \iter\
would have given the internal variable the type |fixnum|
regardless of |i|'s type.  The type incompatibility errors that
could arise in this situation are not checked for.

\end{itemize}

Note that if you do declare |iterate:declare-variables|, then
\iter\ may declare user variables as well as internal ones if they do
not already have declarations, though only for variables that it
binds.  For instance, in this code:

\begin{program}
(iter (declare (iterate:declare-variables))
      (for i from 1 to 10)
      (collect i into var))
\end{program}
the variable |var| will be declared to be of type |list|.


\subsection{Summary}

\iter\ understands standard Common Lisp variable type declarations
that occur within an \iter\ form and
will pass them through to the generated code.  If the declaration
|(iterate:declare-variables)|\lispindex{declare-variables}
appears at the top level of an 
\iter\ form, or if 
|iterate::*always-declare-variables*|\lispindex{*always-declare-variables*} 
is \nonnil, then \iter\ will use the type information gleaned from user
declarations, self-evaluating expressions and |the| expressions,
combined with reasonable assumptions, to determine variable
types and declare them.


%%%%%%%%%%%%%%%%%%%%%%%%%%%%%%%%%%%%%%%%%%%%%%%%%%%%%%%%%%%%%%%%
\section{Problems with Code Movement}
\label{code-movement}

Some \iter\ clauses, or parts of clauses, result in code being
moved from the location of the clause to other parts of the loop.
Drivers behave this way, as do code-placement clauses like |initially|
and |finally|.  When using these clauses, there is a danger of writing
an expression that makes sense in its apparent location but will be
invalid or have a different meaning in another location.  For example:
\begin{program}
(iter (for i from 1 to 10)
      (let ((x 3))
        (initially (setq x 4))))
\end{program}
While it may appear that the |x| of |(initially (setq x 4))| is the
same as the |x| of |(let ((x 3)) \dots|, in fact they are not:
|initially| moves its code outside the loop body, so |x| would refer
to a global variable.  Here is another example of the same problem:
\begin{program}
(iter (for i from 1 to 10)
      (let ((x 3))
        (collect i into x)))
\end{program}
If this code were executed, |collect| would create a binding for its
|x| at the top level of the \iter\ form that the |let| will shadow.

Happily, \iter\ is smart enough to catch these errors; it
walks all problematical code to ensure that free variables are not
bound inside the loop body, and checks all variables it binds for the
same problem.

However, some errors cannot be caught:

\begin{program}
(iter (with x = 3)
      (for el in list)
      (setq x 1)
      (reducing el by \#'+ initial-value x))
\end{program}
|reducing| moves its |initial-value| argument to the initialization
part of the loop in order to produce more efficient code.  Since
\iter\ does not perform data-flow analysis, it cannot determine that
|x| is changed inside the loop; all it can establish is that |x| is
not bound internally.  Hence this code will not signal an
error and will use $3$ as the initial value of the reduction.

The following list summarizes all cases that are subject to these code
motion and variable-shadowing problems.
\begin{itemize}
\item Any variable for which \iter\ creates a binding, including those
used in |with| and the |into| keyword of many clauses.

\begin{sloppypar}
\item The special clauses which place code: |initially|, |after-each|, |else|,
|finally| and |finally-protected|. 
\end{sloppypar}

\item The variables of a |next| or |do-next| form.

\item The |initially| arguments of |for\dots initially\dots then| and
|for\dots previous|. 

\item The |then| argument of |for\dots initially\dots then|.

\item The |initial-value| arguments of |reducing| and |accumulate|.

\item The |on-failure| argument of |finding\dots such-that|.
\end{itemize}

%%%%%%%%%%%%%%%%%%%%%%%%%%%%%%%%%%%%%%%%%%%%%%%%%%%%%%%%%%%%%%%%
\section{Differences Between {\tt Iterate} and {\tt Loop}}
|loop| contains a great deal of complexity which \iter\ tries to
avoid.  Hence many esoteric features of |loop| don't exist in \iter.
Other features have been carried over, but in a cleaned-up form.  
And of course, many new features have been added; they are not
mentioned in this list.

\begin{itemize}

\item \iter's syntax is more Lisp-like than |loop|'s, having a higher
density of parens.

\item The current implementation of \iter, unlike the current version
of  |loop| (as documented in {\em Common Lisp, 2nd Ed.\/}), is
extensible (see section \ref{extend}).  

\item |loop| puts the updates of all driver variables at the top of
the loop; \iter\ leaves them where the driver clauses appear.

\item While for the most part \iter\ clauses that resemble |loop| clauses
behave similarly, there are some differences.  For instance, there is
no |for\dots =\dots then| in \iter; instead use 
|for\dots initially\dots then|.  

\item |loop| binds the variable |it| at certain times
to allow pseudo-English expressions like |when ~expr~ return it|. 
In \iter, you must bind ~expr~ to a variable yourself.  Note that
|when ~expr~ return it| is like |thereis ~expr~| except that the latter is an
accumulation clause and therefore competes with other accumulations
(remember section \ref{multiple} above).

% repeat different behaviour of |always| clause here?

\item |loop| has a special |return| clause, illustrated in the
previous item.  \iter\ doesn't need one,  since an ordinary Lisp
|return| has the same effect.

\item |loop| allows for parallel binding and stepping of iteration
variables.  \iter\ does not.  (See section \ref{bindings}.)

\item |loop| and \iter\ handle variable type declarations very
differently.  |loop| provides a special syntax for declaring variable
types, and does not examine declarations.  Moreover, the standard
implementation of |loop| will
generate declarations when none are requested.
\iter\ parses standard Common Lisp type declarations, and will never
declare a variable itself unless declarations are 
specifically requested.

\end{itemize}

%%%%%%%%%%%%%%%%%%%%%%%%%%%%%%%%%%%%%%%%%%%%%%%%%%%%%%%%%%%%%%%%
\section{Rolling Your Own}
\label{extend}

\subsection{Introduction}

\iter\ is extensible---you can write new clauses that embody new
iteration patterns.  You might want to write a new driver clause for a
data structure of your own, or you might want to write a clause that
collects or manipulates elements in a way not provided by \iter. 

This section describes how to write clauses for \iter.  Writing a
clause is like writing a macro.  In fact, writing a clause ~is~
writing a macro: since \iter\ code-walks its body and macroexpands,
you can add new abstractions to \iter\ with good old |defmacro|.

Actually, there are two extensions you can make to \iter\ that are
even easier than writing a macro.  They are adding a synonym for an
existing clause and defining a driver clause for an indexable
sequence.  These can be done with |defsynonym| and
|defclause-sequence|, respectively.  See section \ref{aids}, below.

The rest of this section explains how to write macros that expand into
\iter\ clauses.
Here's how you could add a simplified version of \iter's |multiply|
clause, if \iter\ didn't already have one:
\begin{program}
(defmacro multiply (expr)
  `(reducing ,expr by \#'* initial-value 1))
\end{program}

If you found yourself summing the square of an expression often, you
might want to write a macro for that.  A first cut might be
\begin{program}
(defmacro sum-of-squares (expr)
  `(sum (* ,expr ,expr)))
\end{program}
but if you are an experienced macro writer, you will realize that this
code will evaluate ~expr~ twice, which is probably a bad idea.  A
better version would use a temporary:
\begin{program}
(defmacro sum-of-squares (expr)
  (let ((temp (gensym)))
    `(let ((,temp ,expr))
       (sum (* ,temp ,temp)))))
\end{program}
Although this may seem complex, it is just the sort of thing you'd
have to go through to write any macro, which illustrates the point of
this section: if you can write macros, you can extend \iter.

Our macros don't use \iter's keyword-argument
syntax.  We could just use keywords with |defmacro|, but we would
still not be using \iter's clause indexing mechanism.  Unlike Lisp,
which uses just the first symbol of a form to determine what function
to call, \iter\ individuates clauses by the list of required keywords.
For instance, |for\dots in| and |for\dots in-vector| are different clauses
implemented by distinct Lisp functions.  

To buy into this indexing scheme, as well as the keyword-argument
syntax, use |defmacro-clause|:

\begin{clauses}

\defmacro{defmacro-clause}{arglist |\&body| body}
Defines a new \iter\ clause.  ~arglist~ is a list of symbols which are
alternating keywords and arguments.  \opt\ may be used, and the list
may be terminated by |\&sequence|.  ~body~ is an ordinary macro body,
as with |defmacro|.  If the first form of ~body~ is a string, it is
considered a documentation string and will be shown by
|display-iterate-clauses|. |defmacro-clause| will signal an error if 
defining the clause would result in an ambiguity.  E.g. you cannot
define the clause |for\dots from| because there would be no way to
distinguish it from a use of the |for| clause with optional keyword |from|.

\end{clauses}

\medskip

Here is |multiply| using |defmacro-clause|.  The keywords are capitalized
for readability.
\begin{program}
(defmacro-clause (MULTIPLY expr \opt\ INTO var)
  `(reducing ,expr by \#'* into ,var initial-value 1))
\end{program}
You don't have to worry about the case when |var| is not supplied; for
any clause with an |into| keyword, saying |into nil| is equivalent to
omitting the |into| entirely.

As another, more extended example, consider the fairly common
iteration pattern that 
involves finding the sequence element that maximizes (or minimizes) some
function.  \iter\ provides this as |finding\dots maximizing|, but it's
instructive to see how to write it.
Here, in pseudocode, is how you might write such a loop for
maximizing a function F:

\begin{quote}
\begin{tabbing}
set variable MAX-VAL to NIL; \\
set variable WINNER to NIL; \\
for\=\ each element EL in the sequence \\
\>  if\=\ MAX-VAL\=\ is NIL or F(EL) $>$ MAX-VAL then \\
\>\>    set MAX-VAL to F(EL); \\
\>\>    set WINNER to EL; \\
\>  end if; \\
end for; \\
return WINNER.
\end{tabbing}
\end{quote}

Here is the macro:
\begin{program}
(defmacro-clause (FINDING expr MAXIMIZING func \opt\ INTO var)
  (let ((max-val (gensym))
        (temp1 (gensym))
        (temp2 (gensym))
        (winner (or var iterate::*result-var*)))
    `(progn 
       (with ,max-val = nil)
       (with ,winner = nil)
       (cond 
        ((null ,max-val)
         (setq ,winner ,expr)
         (setq ,max-val (funcall ,func ,winner))
        (t
         (let* ((,temp1 ,expr)
                (,temp2 (funcall ,func ,temp1)))
           (when (> ,temp2 ,max-val)
             (setq ,max-val ,temp2)
             (setq ,winner ,temp1))))))
       (finally (leave ,winner)))))
\end{program}
Note that if no |into| variable is supplied, we use
|iterate::*result-var*|, which contains the internal variable into
which all clauses place their results.  If this variable is bound by
some clause, then \iter\ will return its value automatically;
otherwise, \nil\ will be returned.  

\subsection{Writing Drivers}

In principle, drivers can be implemented just as easily as other
\iter\ clauses.  In practice, they are a little harder to get right. 
As an example, consider writing a driver that
iterates over all the 
elements of a vector, ignoring its fill-pointer.  |for\dots in-vector|
won't work for this, because it observes the fill-pointer.  It's necessary to
use |array-dimension| instead of |length| to obtain the size of the
vector.  Here is one approach:

\begin{program}
(defmacro-clause (FOR var IN-WHOLE-VECTOR v)
  "All the elements of a vector (disregards fill-pointer)"
  (let ((vect (gensym))
        (index (gensym)))
    `(progn
       (with ,vect = ,v)
       (for ,index from 0 below (array-dimension ,vect 0))
       (for ,var = (aref ,vect ,index)))))
\end{program}
Note that we immediately put |v| in a variable, in case it is an
expression.  Again, this is just good Lisp macrology.  It also has a
subtle effect on the semantics of the driver: |v| is evaluated only
once, at the beginning of the loop, so changes to |v| in the loop have
no effect on the driver.  Similarly, the bounds for numerical iteration
e.g. the above |array-dimension| are also evaluated once only.  This is how
all of \iter's drivers work.


There is an important point concerning the |progn| in this code.  We
need the |progn|, of course, because we are returning several forms,
one of which is a driver.  But \iter\ drivers must occur at top-level.
Is this code in error?  No, because ~top-level~ is defined in \iter\
to include forms inside a |progn|.  This is just the definition of
top-level that Common Lisp uses, and for the same reason: to allow
macros to return multiple forms at top-level.

While our |for\dots in-whole-vector| clause will work, it is not
ideal.  In particular, it does not support generating.  Do do so, we
need to use |for\dots next| or |for\dots do-next|.  The job is
simplified by the |defmacro-driver| macro.

\begin{clauses}

\defmacro{defmacro-driver}{arglist |\&body| body}
Defines a driver clause in
both the |for| and |generate| forms, and provides a parameter
|generate| which ~body~ can examine to determine how it was invoked.
~arglist~ is as in |defmacro-clause|, and should begin with the symbol
|for|. 

\end{clauses}

With |defmacro-driver|, our driver looks like this:
\begin{program}
(defmacro-driver (FOR var IN-WHOLE-VECTOR v)
  "All the elements of a vector (disregards fill-pointer)"
   (let ((vect (gensym))
         (end (gensym))
         (index (gensym))
         (kwd (if generate 'generate 'for)))
     `(progn
        (with ,vect = ,v)
        (with ,end = (array-dimension ,vect 0))
        (with ,index = -1)
        (,kwd ,var next (progn (incf ,index)
                               (if (>= ,index ,end) (terminate))
                               (aref ,vect ,index))))))
\end{program}


We are still missing one thing: the |\&sequence| keywords.  
We can get them easily enough, by writing 
\begin{program}
(defmacro-driver (FOR var IN-WHOLE-VECTOR v \&sequence) 
  ...)
\end{program}
We can now refer to parameters |from|, |to|, |by|, etc. which contain
either the values for the corresponding keyword, or \nil\ if the
keyword was not supplied.  Implementing the right code for these
keywords is cumbersome but not difficult; it is left as an exercise.
But before you begin, see |defclause-sequence| below for an easier way.
       
\subsection{Extensibility Aids}
\label{aids}

This section documents assorted features that may be of use in
extending \iter.

\begin{clauses}

\defunexpvar{*result-var*}
Holds the variable that is used to return a value as a result of the
\iter\ form.  You may examine this and use it in a |with| clause, but
you should not change it.

\defmacro{defsynonym}{syn word}
 Makes ~syn~ a synonym for the existing \iter\ keyword ~word.~  Only
the first word in each clause can have synonyms. 

\defmacroxx{defclause-sequence}{element-name index-name}{|\&key|
access-fn size-fn sequence-type}{element-type
element-doc-string index-doc-string}  
Provides
a simple way to define sequence clauses.  Generates two
clauses, one for iterating over the sequence's elements, the other
for iterating over its indices.  The first symbol of both
clauses will have print-name |for|.
~element-name~  and ~index-name~ should be symbols.
~element-name~ is the second keyword of the element iterator (typically of
the form 
|in-~sequence-type~|), and ~index-name~ is the second keyword
of the index-iterator (typically of the form
|index-of-~sequence-type~|).  Either name may be 
\nil, in which case the corresponding clause is not defined.  If both
symbols are supplied, they should be in the same package.  The |for|
that begins the clauses will be in this package.

\cpar ~access-fn~ is the function to be used to
access elements of the sequence in the element iterator.  The function
should take two 
arguments, a sequence and an index, and return the appropriate element.
~size-fn~ should denote a function of one argument, a sequence, that
returns its size.  Both ~access-fn~ and ~size-fn~ are required for the
element iterator, but only ~size-fn~ is needed for the index iterator.

\cpar The ~sequence-type~ and ~element-type~ keywords are used to
suggest types for the variables 
used to hold the sequence and the
sequence elements, respectively.  The usual rules about \iter's
treatment of variable type declarations apply (see section \ref{types}).

\cpar ~element-doc-string~ and ~index-doc-string~ are
the documentation strings, for use with |display-iterate-clauses|. 

\cpar The generated element-iterator performs destructuring on the
element variable.

\cpar As an example, the above |for\dots in-whole-vector| example
could have been written:
\begin{program}
(defclause-sequence IN-WHOLE-VECTOR INDEX-OF-WHOLE-VECTOR
  :access-fn 'aref
  :size-fn \#'(lambda (v) (array-dimension v 0))
  :sequence-type 'vector
  :element-type t
  :element-doc-string 
     "Elements of a vector, disregarding fill-pointer"
  :index-doc-string 
     "Indices of vector, disregarding fill-pointer")
\end{program}

\end{clauses}

\subsection{Subtleties}

There are some subtleties to be aware of when writing \iter\ clauses.
First, the code returned by your macros may be |nconc|'ed into a list,
so you should always returned freshly consed lists, rather than
constants.  Second, \iter\ matches clauses by using |eq| on the first
symbol and |string=| on the subsequent ones, so the package of the
first symbol of a clause is relevant.  All of the clauses in this manual
have their first word in the \iter\ package.  
You can use the package system in the usual way to shadow
\iter\ clauses without replacing them.

%%% say more here, about the badness that only the first word of a
%%% clause is packagey.

\section{Non-portable Extensions to Iterate (Contribs)}
\label{contribs}

Currently, there is only one non-portable extension to iterate in the
distribution: iterate-pg. If you have made an extension that depends
on non-portable features, feel free to send them to |asf@boinkor.net|
for inclusion in the iterate distribution.

\subsection{An SQL query driver for iterate}

The pg package by Eric Marsden (see |http://cliki.net/pg|) provides an
interface to the PostgreSQL database. Using the \iterpg\ extension, it
is possible to handle the results of SQL queries using \iter.

This usage example should give you an idea of how to use it:

\begin{program}
(pg:with-pg-connection (c "somedb" "someuser")
  (iter (for (impl version date) in-relation "select * from version"
                                 on-connection *dbconn*)
        (collect version)))
\end{program}

To use the extension via |ASDF|, simply make your system depend on the
|iterate-pg| system instead of the |iterate| system. To load it
manually, use:

\begin{program}
  (asdf:oos 'asdf:load-op :iterate-pg)
\end{program}

%%%%%%%%%%%%%%%%%%%%%%%%%%%%%%%%%%%%%%%%%%%%%%%%%%%%%%%%%%%%%%%%
\section{Obtaining {\tt Iterate}}

\begin{sloppypar}
The information in this chapter is obsolete but included for
completeness's sake; Currently, the most up-to-date information on
\iter\ can be found at |http://boinkor.net/iterate.html|.
\end{sloppypar}

\begin{sloppypar}
\iter\ currently runs on Lisp Machines, and on
HP's, Sun3's and Sparcstations under Lucid.  
\iter\ source and binaries are available at the MIT AI Lab in the
subdirectories of |/src/local/lisplib/|.  The source file,
|iterate.lisp|, is also available for anonymous FTP in the directory
|/com/fpt/pub/| on the machine |TRIX.AI.MIT.EDU| (Internet number
128.52.32.6).  If you are unable to obtain |iterate| in one of these
ways, send mail to |jba@ai.mit.edu| and I will send you the source
file. 
\end{sloppypar}

\begin{sloppypar}
\iter\ resides in the |iterate| package (nickname |iter|).  Just say
\linebreak |(use-package :iterate)| to make all the necessary symbols
available. 
If a symbol is not exported, it appears in this manual with an
``|iterate::|'' prefix.
\end{sloppypar}

Send bug reports to |bug-iterate@ai.mit.edu|.  The |info-iterate|
mailing list will have notices of changes and problems; to have
yourself added, send mail to |info-iterate-request@ai.mit.edu|.


\medskip

\begin{flushleft}
 \bf Acknowledgements
\end{flushleft}
\smallskip

Richard Waters provided invaluable criticism which spurred me to improve
\iter\ greatly.  As early users, David Clemens, Oren Etzioni and Jeff
Siskind helped ferret out many bugs.

%\begin{theindex}
% The index files must be generated with the genindex program
% or from makeindex iterate-manual.idx which creates iterate-manual.ind.

%\baselineskip\normalbaselineskip
%\advance\baselineskip by -2pt

\input{iterate-manual.ind}

%\underline{Clauses}
%\input{iterate-manual.clindex}
%
%\indexspace
%\underline{Lisp}
%\input{iterate-manual.lispindex}

%\end{theindex}


\end{document}


% arch-tag: "c0f871a6-313c-11d8-abb9-000c76244c24"

%
%\indexspace
%\underline{Lisp}
%% Fix ``.'' spacing in lists
\documentstyle[12pt,aimemo]{article}

\setlength{\textwidth}{6.0in}
\setlength{\oddsidemargin}{0.25in}
\setlength{\evensidemargin}{0.25in}
\setlength{\topmargin}{0pt}
\setlength{\headsep}{24pt}
\setlength{\textheight}{8.5in}

\let\origpar=\par

% somehow, LaTex fucks with things so that the plain Tex \obeyspaces
% doesn't work.  This is the same, except we use an mbox containing
% a space instead of just a space.

\newcommand{\spc}{\mbox{ }}
\newcommand{\tAb}{\mbox{        }}
\def\obspaces{\catcode`\ =\active\catcode`\^^I=\active}
{\obspaces \global\let =\spc\global\let^^I=\tAb}

\newenvironment{program}{\medskip\samepage\tt\obeylines\obspaces}{\medskip}

% use | for \tt
\catcode`\|=13
\newif\ifvbar
\def|{\ifvbar
         \endgroup\vbarfalse
      \else 
         \begingroup\tt\vbartrue 
      \fi}



\newcommand{\lisp}{\tt}
\newcommand{\iter}{|iterate|}
\newcommand{\iterpg}{|iterate-pg|}
\newcommand{\setf}{|setf|}
\newcommand{\nil}{|nil|}
\newcommand{\nonnil}{non-\nil}
\newcommand{\opt}{{\lisp \&optional}}
\newcommand{\yields}{$\Rightarrow$}



\newbox\Dots
\setbox\Dots=\hbox{\small ...}
\renewcommand{\dots}{\copy\Dots}

% Here's a way to define an environment which will treat all
% paragraphs inside it in the same manner.  It's, as near as I can
% figure, the way that Latex does things.  Using \everypar won't work;
% apparently Latex resets that a lot.  What you have to do is: at the
% top of the file do \let\origpar=\par.  In the
% begin part of the environment, do the parshape (or whatever), then say
% \def\par{{\origpar}}.  The extra braces are necessary.   That's all
% you have to do; at the end of the environment things will return to
% normal automatically.  Why does this all work?  I have no idea.
% Another important thing: there should be a blank line (i.e. a \par)
% between the end of the last paragraph and the \end{environment}.

\newlength{\clindent}
\setlength{\clindent}{0.5in}
\newlength{\clparindent}
\setlength{\clparindent}{\parindent}

\newenvironment{clauses}{\advance\linewidth -\parindent
                        \hangindent\clindent\relax
                         \hangafter=0
                          \parindent=0pt
                          \def\par{{\origpar}}}{}

\newcommand{\cpar}{\hspace \clparindent}

\newcommand{\startitem}{\bigskip\pagebreak[3]}
\newcommand{\finishitem}{\smallskip}

\newcommand{\clause}[1]{\startitem\Clause{#1}\finishitem}
\newcommand{\Clause}[1]{\hbox{#1}}
\newcommand{\clindex}[1]{\index{{\lisp #1}}}
\newcommand{\clindexx}[2]{\index{{\lisp #1\dots #2}}}
\newcommand{\clindexxx}[3]{\index{{\lisp #1\dots #2\dots #3}}}
\newcommand{\clausex}[2]{\startitem\Clausex{#1}{#2}\finishitem}
\newcommand{\Clausex}[2]{\Clause{|#1| #2}\clindex{#1}}

\newcommand{\defvar}[1]{\startitem\deff{#1}{}{Variable}\finishitem}
\newcommand{\defunexpvar}[1]{\startitem\deffnoindex{iterate::#1}{}{Variable}
                                        \lispindex{#1}\finishitem}
\newcommand{\defun}[2]{\startitem\deff{#1}{#2}{Function}\finishitem}
\newcommand{\defmacro}[2]{\startitem\deff{#1}{#2}{Macro}\finishitem}
\newcommand{\defunx}[3]{\startitem\deffx{#1}{#2}{#3}{Function}\finishitem}
\newcommand{\defmacrox}[3]{\startitem\deffx{#1}{#2}{#3}{Macro}\finishitem}
\newcommand{\defunxx}[4]{\startitem\deffxx{#1}{#2}{#3}{#4}{Function}\finishitem}
\newcommand{\defmacroxx}[4]{\startitem\deffxx{#1}{#2}{#3}{#4}{Macro}\finishitem}

\newcommand{\deff}[3]{\deffnoindex{#1}{#2}{#3}\lispindex{#1}}
\newcommand{\deffnoindex}[3]{\hbox to \hsize{{\tt #1} {\it #2}\hfill [{\it #3}]}}
\newcommand{\deffx}[4]{\deff{#1}{#2}{#4}\moreargs{#1}{#3}}
\newcommand{\deffxx}[5]{\deffx{#1}{#2}{#3}{#5}\moreargs{#1}{#4}}

\newcommand{\moreargs}[2]{\hbox to \hsize{\phantom{\lisp #1} {\it #2}\hfill}}

\newcommand{\lispindex}[1]{\index{{\lisp #1}}}

 \newenvironment{note}[1]{\pagebreak[2]\bigskip
                          \hrule\smallskip\small
                         {\setlength{\parindent}{0pt}
                           \par{\bf #1:}}}{\smallskip\hrule\bigskip}



\makeindex

%%%%%%%%%%%%%%%%%%%%%%%%%%%%%%%%%%%%%%%%%%%%%%%%%%%%%%%%%%%%%%%%

%\thispagestyle{empty}

%\vspace*{2in}

%\title{The |Iterate| Manual}
%\author{Jonathan Amsterdam}

%\maketitle

%\thispagestyle{empty}

%\begin{center}
%\Large *** DRAFT ***
%\end{center}
%This manual is in beta-test. It will become an AI Memo in a couple of
%months.  I would appreciate any comments or
%corrections.  You can reach me at email address |jba@ai.ai.mit.edu|.

\begin{document}

\def\writtenby{Jonathan Amsterdam}
\def\contractno{This report describes research done at the Artificial
Intelligence Laboratory of the Massachusetts Institute of Technology.
Support for the laboratory's artificial intelligence research is
provided in part by the Advanced Research Projects Agency of the
Department of Defense under Office
of Naval Research contract N00014-85-K-0124.}

\AIMEMO{1236}{\Large The Iterate Manual \\[1in]}
{This is the manual for version 1.4 of \iter, a powerful iteration
macro for Common  
Lisp. \iter\ is similar to {\tt loop} but provides numerous additional
features, is well integrated with Lisp, and is extensible. 
\vfill }


% use ~ for \em
\catcode`\~=13
\newif\ifem
\def~{\ifem
         \/\endgroup\emfalse
      \else 
         \begingroup\em\emtrue 
      \fi}



%\pagebreak
%\setcounter{page}{1}

%\thispagestyle{empty}


\tableofcontents

\pagebreak
\pagestyle{headings}

\section{Introduction}

\begin{sloppypar}
This manual describes
\iter, a powerful iteration facility for Common Lisp.
\iter\ provides abstractions for many common iteration
patterns and allows for the definition of additional patterns.  \iter\ is
a macro that expands into ordinary Lisp at
compile-time, so it is more efficient than higher-order
functions like |map| and |reduce.|  
While it is similar to |loop|, \iter\ offers a more Lisp-like syntax
and enhanced extensibility.
(For a more complete
comparison of \iter\ with other iteration constructs, see MIT AI Lab Working
Paper No. 324, ~Don't Loop, Iterate.~)
\end{sloppypar}

An \iter\ form consists of the symbol |iter|\footnote{You can also use
|iterate|, but |iter| is preferred because it avoids potential conflicts with
possible future additions to Common Lisp, and because it saves
horizontal space when writing code.}
followed by one or more
forms, some of which may be \iter\ 
~clauses.~  Here is a simple example of \iter\ which collects the
numbers from 1 to 10 into a list, and returns the list.  The return
value is shown following the arrow.
\begin{program}
(iter (for i from 1 to 10)
      (collect i))          \yields (1 2 3 4 5 6 7 8 9 10)
\end{program}
This form contains two clauses: a |for| clause that steps the
variable |i| over the integers from 1 to 10, and a |collect|
clause that accumulates its argument into a list.  With a few
exceptions, all \iter\ clauses have the same format: alternating
symbols (called ~keywords~) and expressions (called
~arguments~).  The syntax and terminology are those of Common
Lisp's keyword lambda lists.  One difference is that \iter's keywords
do not have to begin with a colon---though they may, except for the
first symbol of a clause.  So you can also write  |(for i :from
1 :to 10)| if you prefer.

Any Lisp form can appear in the body of an \iter, where it will have
its usual meaning.  \iter\ walks the entire body, expanding macros,
and recognizing clauses at any level.  This example collects all the
odd numbers in a list:
\begin{program}
(iter (for el in list)
      (if (and (numberp el) (oddp el))
          (collect el)))
\end{program}

There are clauses for iterating over numbers, lists, arrays and other
objects, and for
collecting, summing, counting, maximizing and other useful operations.
\iter\ also supports the creation of new variable bindings, stepping
over multiple sequences at once, destructuring, and compiler
declarations of variable types.  The following example illustrates
some of these features: 
\begin{program}
(iter (for (key . item) in alist)
         (for i from 0)
         (declare (fixnum i))
         (collect (cons i key)))
\end{program}

This loop takes the keys of an alist and returns a new alist
associating the keys with their positions in the original list. The
compiler declaration for |i| will appear in the generated code
in the appropriate place.

%%%%%%%%%%%%%%%%%%%%%%%%%%%%%%%%%%%%%%%%%%%%%%%%%%%%%%%%%%%%%%%%
\section{Clauses}

Most of \iter's clauses will be familiar to |loop| programmers.  
(|loop| is an iteration macro that has been incorporated into Common
Lisp.  Seen Guy Steele's {\em Common Lisp, 2nd Edition}.)
In
nearly all cases they behave the same as their |loop| counterparts, so
a |loop| user can switch to \iter\ with little pain (and much gain).

All clauses with the standard keyword-argument syntax consist of two
parts: a ~required~ part, 
containing keywords that must be present and in the right order; and
an ~optional~ part, containing keywords that may be omitted and,
if present, may occur in any order.  In the descriptions below, the
parts are separated by the Lisp lambda-list keyword \opt.


\subsection{Drivers}

An iteration-driving clause
conceptually causes the iteration to go forward.  Driver clauses in
\iter\ allow iteration over numbers, lists, vectors, hashtables, packages,
files and streams.  Iteration-driving clauses must
appear at the top level of an \iter\ form; they cannot be nested
inside another clause.  The driver variable is updated at the point
where the driver clause occurs.  Before the clause is executed for the
first time, the value of the variable is undefined.

%Also, regardless of where the driver clause
%appears in the body, the driver variable is stepped at the top of the
%loop; hence it is stylistically preferable, though
%not required, to place driver clauses at the beginning of the \iter.

Multiple drivers may appear in a single \iter\ form, in which case all
of the driver variables are updated each time through the loop, in the
order in which the clauses appear.  The first driver to terminate will
terminate the entire loop.

In all cases, the value of the driver variable on exit from the loop,
including within the epilogue code (see the |finally| clause), is
undefined.

All the parameters of a driver clause are evaluated once, before the
loop begins.  Hence it is not possible to change the bounds or other
properties of an iteration by side-effect from within the loop.

With one exception, driver clauses begin with the word |for| (or
the synonym |as|) and mention an iteration variable, which is
given a binding within the \iter\ form.
The exception is |repeat|, which just executes a loop a
specified number of times:

\begin{clauses}

\clausex{repeat}{~n~}
Repeats the loop ~n~ times.  For example:
\begin{program}
(iter (repeat 100)
      (print "I will not talk in class."))
\end{program}
\cpar If $n \leq 0$, the
loop will never be executed.  If ~n~ is not an integer, the
actual number of executions will be $\lceil n \rceil$.

\end{clauses}

\pagebreak[3]

\subsubsection{Numerical Iteration}

\begin{clauses}
\clausex{for}{~var~ |\&sequence|} 
The general form for iterating over a sequence of numbers requires a
variable and, optionally, one or more keywords that provide the bounds
and step size of the iteration. The |\&sequence|
lambda-list keyword is a shorthand for these sequence keywords.  They are:
|from|, |upfrom|, |downfrom|, |to|, |downto|, |above|, |below| and |by|.
|from| provides the starting value for ~var~
and defaults to zero. |to| provides a final value and implies that the
successive values of ~var~ will be increasing; |downto| implies that
they will be decreasing.  The loop terminates when ~var~ passes
the final value (i.e. becomes smaller or larger than it, depending on
the direction of iteration); in other  words, the loop body
will never be 
executed for values of ~var~ past the final value.
|below| and |above| are similar to |to| and
|downto|, except that the loop terminates when ~var~ equals or passes
the final value.

\cpar If no final value is specified, the variable will be stepped
forever.  Using |from| or |upfrom| will result in increasing
values, while |downfrom| will give decreasing values.

\cpar On each iteration,
~var~ is incremented or decremented by the value of the sequence
keyword |by|, which defaults to 1.  It should always be a positive
number, even for downward iterations.

\cpar In the following examples, the sequence of numbers generated is shown
next to the clause.

\begin{program}
(for i upfrom 0) \yields\ 0 1 2 \ldots
(for i from 5)   \yields\ 5 6 7 \ldots    ; either from or upfrom is okay
(for i downfrom 0) \yields\ 0 -1 -2 \ldots
(for i from 1 to 3) \yields\ 1 2 3
(for i from 1 below 3) \yields\ 1 2
(for i from 1 to 3 by 2) \yields\ 1 3
(for i from 1 below 3 by 2) \yields\ 1
(for i from 5 downto 3) \yields\ 5 4 3
\end{program}

\end{clauses}

\subsubsection{Sequence Iteration}

There are a number of clauses for iterating over sequences.  In all of
them, the argument following |for| may be a list instead of a
symbol, in which case destructuring is performed.  See section
\ref{destructuring}. 

\begin{clauses}

\clause{|for| ~var~ |in| ~list~ \opt\ |by| ~step-function~}
\clindexx{for}{in}
~var~ is set to successive elements of list.  
~step-function,~ which 
defaults to |cdr|, is used to obtain the next sublist. 

\clause{|for| ~var~ |on| ~list~ \opt\ |by| ~step-function~}
\clindexx{for}{on}
~var~ is set to successive sublists of list. 
~step-function~ (default |cdr|) is used as in |for\dots in|.

\end{clauses}

\medskip

These two clauses use |atom| to test for the end
of a list.  Hence, given a list whose final |cdr| is not \nil,
they will silently ignore the last |cdr|.  Other choices are
|endp|,  which would signal an error, and |null|, which
would probably result in an error somewhere else.  If you wish to use
an end-test other than |atom|, set the variable
|iterate::*list-end-test*|\lispindex{*list-end-test*} to the name of
the desired function. 

\begin{clauses}
\clause{|for| ~var~ |in-vector| ~vector~ |\&sequence|}
\clindexx{for}{in-vector}
~var~ takes on successive elements from ~vector.~   The vector's
fill-pointer is observed.  Here and in subsequent clauses, the |\&sequence|
keywords include |with-index|, which takes a symbol as argument
and uses it for the index variable instead of an internally generated
symbol.  The other |\&sequence| keywords behave as in numerical
iteration, except that the default iteration bounds are the bounds of
the vector.  E.g. in |(for i in-vector v downto 3)|,
|i| will start off being bound to the last element in |v|,
and will  
be set to preceding elements down to and including the element with
index 3.

\clause{|for| ~var~ |in-sequence| ~seq~ |\&sequence|}
\clindexx{for}{in-sequence}
This uses Common Lisp's generalized sequence functions, |elt|
and |length|, to obtain elements and determine the length of
~seq.~   Hence it will work for any sequence, including lists, and will
observe the fill-pointers of vectors.

\clause{|for| ~var~ |in-string| ~string~ |\&sequence|}
\clindexx{for}{in-string}
~var~ is set to successive characters of ~string.~

\startitem
\Clause{|for| ~var~ |index-of-vector| ~vector~ |\&sequence|}
\clindexx{for}{index-of-vector}
\Clause{|for| ~var~ |index-of-sequence| ~sequence~ |\&sequence|}
\clindexx{for}{index-of-sequence}
\Clause{|for| ~var~ |index-of-string| ~string~ |\&sequence|}
\clindexx{for}{index-of-string}
\finishitem
~var~ is set to successive indices of the sequence.
These clauses avoid the overhead of accessing the sequence elements
for those applications where they do not need to be examined, or are
examined rarely.  They admit all the optional keywords of the other
sequence drivers except the (redundant) |with-index| keyword.

\clause{|for| ~(key value)~ |in-hashtable| ~table~}
\clindexx{for}{in-hashtable}

~key~ and ~value,~ which must appear as shown in a list
and may be destructuring templates, are set to the keys and values of
~table~.  If ~key~ is |nil|, then the hashtable's keys will be ignored;
similarly for ~value~.
The order in which 
elements of ~table~ will be retrieved is unpredictable.

\clause{|for| ~var~ |in-package| ~package~ \opt\ |external-only| ~ext~}
\clindexx{for}{in-package}
Iterates over all the symbols in ~package,~ or over only the
external symbols if ~ext~ is specified and non-|nil|.  ~ext~ is not
evaluated.  The same symbol may appear more than once.

\clause{|for| ~var~ |in-packages| ~packages~ \opt\ |having-access| ~symbol-types~}
\clindexx{for}{in-packages}
Iterates over all the symbols from the list of packages denoted by the
descriptor ~packages~ and having accessibility (or visibility) given by
~symbol-types~.  This defaults to the list |(:external :internal :inherited)|
and is not evaluated.  ~var~ must be a list of up to three variables: in each
iteration, these will be set to a symbol, its access-type and package (as per
|with-package-iterator| in ANSI-CL).  The same symbol may appear more than
once.

\end{clauses}

\begin{clauses}

\clause{|for| ~var~ |in-file| ~name~ \opt\ |using| ~reader~}
\clindexx{for}{in-file}
Opens the file ~name~ (which may be a string or pathname) for input,
and iterates 
over its contents. ~reader~ defaults to |read|, so by default ~var~
will be bound to the successive forms in the file.  The \iter\ body is
wrapped in an unwind-protect to ensure that the file is closed no
matter how the \iter\ is exited.

\clause{|for| ~var~ |in-stream| ~stream~ \opt\ |using| ~reader~}
\clindexx{for}{in-stream}
Like |for\dots in-file|, except that ~stream~ should be an existing
stream object that supports input operations.

\end{clauses}

\subsubsection{Generalized Drivers}

These are primarily useful for writing drivers that can also be used
as generators (see section \ref{generators}, below).
 
\begin{clauses}

\clause{|for| ~var~ |next| ~expr~}
\clindexx{for}{next}

~var~ is set to ~expr~ each time through the loop.  Destructuring is
performed.  When the clause is used as a generator, ~expr~ is the code
that is executed when |(next ~var~)| is 
encountered (see section \ref{generators}, below).
~expr~ should compute the first value for ~var~, as well
as all subsequent values, and is responsible for terminating the loop.
For compatibility with future versions of \iter, this termination
should be done with |terminate|\clindex{terminate}, which can be
considered a synonym for |finish| (see section \ref{control-flow}).

\cpar As an example, the following clauses are equivalent to |(for
i from 1 to 10)|:
\begin{program}
(initially (setq i 0))
(for i next (if (> i 10) (terminate) (incf i)))
\end{program}

\clause{|for| ~var~ |do-next| ~form~}
\clindexx{for}{do-next}
~form~ is evaluated each time through the loop.  Its value is ~not~
set to ~var~; that is ~form's~ job. ~var~ is only present so that
\iter\ knows it is a driver variable.  \linebreak |(for ~var~ next
~expr~)| is equivalent to |(for ~var~ do-next (dsetq ~var~ ~expr~))|.
(See section \ref{destructuring} for an explanation of |dsetq|.) 

\end{clauses}

\subsubsection{Generators}
\label{generators}

In all of the above clauses, the driver variable is updated on each
iteration.
Sometimes it is desirable to have greater control over updating.  
For instance, consider the problem of associating numbers, in
increasing order and with no gaps, with the
\nonnil\ elements of a list.  One obvious first pass at writing this is:

\begin{program}
(iter (for el in list)
      (for i upfrom 1)
      (if el (collect (cons el i))))
\end{program}
But on the list |(a b nil c)| this produces |((a . 1) (b . 2) (c .
4))| instead of the desired |((a . 1) (b . 2) (c . 3))|.  The problem
is that |i| is incremented each time through the loop, even when |el|
is |nil|.  

The problem could be solved elegantly if we could step |i| only when
we wished to.  This
can be accomplished for any \iter\ driver by writing |generate| (or
its synonym |generating|)
instead of |for|.  Doing so produces a ~generator~---a driver whose
values are yielded explicitly.  To obtain the next value of a
generator variable ~v~, write \linebreak |(next ~v~)|.  The value of a |next|
form is the next value of ~v~, as determined by its associated driver
clause.  |next| also has the side-effect of updating ~v~ to that
value.  If there is no next value, |next| will terminate the loop,
just as with a normal driver.


Using generators, we can now write our example like this:
\begin{program}
(iter (for el in list)
      (generate i upfrom 1)
      (if el (collect (cons el (next i)))))
\end{program}
Now |i| is updated only when |(next i)| is executed, and this occurs
only when |el| is \nonnil.

To better understand the relationship between ordinary drivers and
generators, observe that we can rewrite an ordinary driver using its
generator form immediately followed by |next|, as this example shows:
\begin{program}
(iter (generating i from 1 to 10)
      (next i)
      ...)
\end{program}
Provided that the loop body contains no |(next i)| forms, this will
behave just as if we had written |(for i from 1 to 10)|.

We can still refer to a driver variable ~v~ without using |next|; in
this case, its value is that given to it by the last evaluation of
|(next ~v~)|.  Before |(next ~v~)| has been called the first time, the
value of ~v~ is undefined.  

This semantics is more flexible than
one in which ~v~ begins the loop bound to its first value and calls
of |next| supply subsequent values, because it means the loop will not
terminate too soon if the generator's sequence is empty.  For
instance, consider the following code, which tags \nonnil\ elements of
a list using a list of tags, and also counts the null elements.
(We assume there are at least as many tags as \nonnil\ elements.)
\begin{program}
(let* ((counter 0)
       (tagged-list (iter (for el in list)
                          (generating tag in tag-list) 
                          (if (null el)
                              (incf counter)
                              (collect (cons el (next tag)))))))
  ...)
\end{program}
It may be that there are just as many tags as non-null elements of
|list|.  If all the elements of |list| are null, we still want the
counting to proceed, even though |tag-list| is \nil.  If |tag| had to be
assigned its first value before the loop begins, we would have had to
terminate the loop before the first iteration, since when |tag-list|
is \nil, |tag| 
has no first value.  With the existing semantics, however, |(next
tag)| will never execute, so the iteration will cover all the elements of
|list|. 
        
When the ``variable'' of a driver clause is actually a destructuring
template containing several variables, all the variables are eligible
for use with |next|.  As before, |(next ~v~)| evaluates to ~v's~ next
value; but the effect is to update all of the template's variables.
For instance, the following code will return the list |(a 2 c)|.
\begin{program}
(iter (generating (key . item) in '((a . 1) (b . 2) (c . 3)))
      (collect (next key))
      (collect (next item)))
\end{program}

Only driver clauses with variables can be made into generators.  This
includes all clauses mentioned so far except for |repeat|.  It does
~not~ include |for\dots previous|, |for\dots =|,
|for\dots initially\dots then| or |for\dots first\dots then| (see
below). 

\subsubsection{Previous Values of Driver Variables}

Often one would like to access the value of a variable on a previous
iteration.  \iter\ provides a special clause for accomplishing this.

\begin{clauses}

\clause{|for| ~pvar~ |previous| ~var~ \opt\ |initially| ~init~
|back| ~n~}
\clindexx{for}{previous}
Sets ~pvar~ to the previous value of ~var,~ which should be a driver
variable, a variable from another |for\dots previous| clause, or a
variable established by a |for\dots =|, 
|for\dots initially\dots then| or |for\dots first\dots then| clause
(see section \ref{setting}).
Initially, ~pvar~ is given the value ~init~ (which defaults to \nil).  
The ~init~ expression will be moved outside the loop body, so it
should not depend on anything computed within the loop.
~pvar~ retains the value of ~init~ until ~var~ is set to its second
value, at which point ~pvar~ is set to ~var's~ first value; and so on.  

\cpar The
argument ~n~ to |back| must be a constant, positive integer, and
defaults to 1.  It determines how many iterations back ~pvar~ should
track ~var~.  For example, when ~n~ is 2, then ~pvar~ will be assigned
~var's~ first value when ~var~ is set to its third value.

\cpar A |for\dots previous| clause may occur after or before its
associated driver clause. |for\dots previous| works with generators as
well as ordinary drivers.

\pagebreak[3]

\cpar Example:
\begin{program}
(iter (for el in '(1 2 3 4))
      (for p-el previous el)
      (for pp-el previous p-el initially 0)
      (collect pp-el))
\end{program}
This evaluates to |(0 0 1 2)|.  It could have been written more
economically as 
\begin{program}
(iter (for el in '(1 2 3 4))
      (for pp-el previous el back 2 initially 0)
      (collect pp-el))
\end{program}


\end{clauses}

\subsection{Variable Binding and Setting}
\label{setting}

Several clauses exist for establishing new variable bindings or for
setting variables in the loop.  They all support destructuring.

\begin{clauses}

\clausex{with}{~var~ \opt\ |=| ~value~}
Causes ~var~ to be bound to value before the loop body is entered.
If ~value~ is not supplied, ~var~ assumes a default
binding, which will be 
\nil\ in the absence of declarations.  Also, if ~value~ is not
supplied, no destructuring is performed; instead, ~var~ may be a list of
symbols, all of which are given default 
bindings.  If ~value~ is supplied, ~var~ is bound to it, with
destructuring.  

\cpar Because |with|
creates bindings whose scope includes the entire \iter\ form, it is
good style to put all |with| clauses at the beginning.

\cpar Successive occurrences of |with| result in sequential
bindings (as with 
|let*|).  There is no way to obtain parallel bindings; see
section \ref{bindings} for a rationale.  


\clause{|for| ~var~ |=| ~expr~}
\clindexx{for}{=}
On each iteration, ~expr~ is evaluated and ~var~ is set
to its value. 

\cpar This clause may appear to do the same thing as |for\dots next|.
In fact, they are quite different.  |for\dots =| provides only three
services: it sets up a binding for ~var~, sets it to ~expr~ on each
iteration, and makes it possible to use |for\dots previous| with
~var~.  |for\dots next| provides these services in addition to the
ability to turn the driver into a generator.
%Also, the code which sets ~var~ appears in the loop body in the same
%place as the |for\dots =| clause; the code for |for\dots next| appears
%at the top of the loop, as with other drivers (except when being used
%as a generator).

\clause{|for| ~var~ |initially| ~init-expr~ |then| ~then-expr~}
\clindexxx{for}{initially}{then}
Before the loop begins, ~var~ is set to ~init-expr;~ on all
iterations after the first it is set to ~then-expr.~  This clause
must occur at top-level.  ~init-expr~ will be moved outside the loop
body and ~then-expr~ will be moved to the end of the loop body, so
they are subject to code motion problems (see section
\ref{code-movement}). 

\cpar This clause may appear to be similar to |for\dots next|, but in
fact they differ significantly.  |for\dots initially\dots then| is
typically used to give ~var~ its first value before the loop begins,
and subsequent values on following iterations.  This is incompatible
with generators, whose first value and subsequent values must all be
computed by |(next ~var~)|.  Also, the update of ~var~ in 
|for\dots initially\dots then| does not occur at the location of the
clause. 
Use |for\dots initially\dots then| for
one-shot computations where its idiom is more convenient, but use
|for\dots next| for extending \iter\ with new drivers (see section
\ref{extend}).

\clause{|for| ~var~ |first| ~first-expr~ |then| ~then-expr~}
\clindexxx{for}{first}{then}
The first time through the loop, ~var~ is set to ~first-expr;~ on
subsequent iterations, it is set to ~then-expr.~  This differs from
|for\dots initially| in that ~var~ is set to ~first-expr~
inside the loop body, 
so ~first-expr~ may depend on the results of other clauses.  For
instance, 
\begin{program}
(iter (for num in list)
      (for i first num then (1+ i))
      ...)
\end{program}
will set |i| to the first element of |list| on the first
iteration, whereas 
\begin{program}
(iter (for num in list)
      (for i initially num then (1+ i))
      ...)
\end{program}
is probably erroneous; |i| will be bound to |num|'s
default binding (usually \nil) for the first iteration.

\end{clauses}

\begin{note}{Compatibility note}
|loop|'s |for\dots =| works like \iter's, but |loop| used the syntax
|for\dots =\dots then| to mean |for\dots initially\dots then|.  It was
felt that these two operations were sufficiently different to warrant
different keywords.

Also, the |for| in the above three clauses is misleading,
since none is true driver (e.g. none has a corresponding |generate|
form).  |setting| would have been a better choice, but |for| was used
to retain some compatibility with |loop|.  
\end{note}


%%%%%%%%%%%%%%%%%%%%%%%%%%%%%%%%
\subsection{Gathering Clauses}
Many of \iter's clauses accumulate values into a variable, or set a
variable under certain conditions.  At the end of the loop, this
variable contains the desired result.  All these clauses have an
optional |into| keyword, whose argument should be a symbol.  If the
|into| keyword is not supplied, the accumulation variable will be
internally generated and its value will be returned at the end of the
loop; if a variable is specified, that variable is used for the
accumulation, and is not returned as a result---it is up to the user
to return it explicitly, in the loop's epilogue code (see |finally|).
It is safe to examine the accumulation variable during the loop, but
it should not be modified.

These clauses all begin with a verb.  When the verb does
not conflict with an existing Common Lisp function, then it may be
used in either its infinitival or present-participle form (e.g. |sum|,
|summing|).  However, when there is a conflict with Common Lisp, only
the present-participle form may be used (e.g. |unioning|).  This is to
prevent \iter\ clauses from clashing with Common Lisp
functions. 

%although these clauses are described as ``producing a value,'' it is a
%mistake to think of the lisp list representing the clause as a
%value-producing form in the usual way.  clauses may legally be written
%where a value is expected, e.g. |(setq x (sum i))|, but the
%lisp value of a clause in such a context is undefined.

\subsubsection{Reductions}

~Reduction~ is an extremely common iteration pattern in which
the results of successive applications of a
binary operation are accumulated.
For example, a loop that computes the sum of the
elements of a list is performing a reduction with the addition
operation.  This could be written in Common Lisp as  |(reduce \#'+
list)| or with \iter\ as
\begin{program}
(iter (for el in list)
      (sum el))
\end{program}

\begin{clauses}

\clausex{sum}{~expr~ \opt\ |into| ~var~}
Each time through the loop, ~expr~ is evaluated and added to a
variable, which is bound initially to zero.  If ~expr~ has a type,
it is ~not~ used as the type of the sum variable, which is always
|number|.  To get the result variable to be of a more specific
type, use an explicit variable, as in
\begin{program}
(iter (for el in number-list)
      (sum el into x)
      (declare (fixnum x))
      (finally (return x)))
\end{program}

\clausex{multiply}{~expr~ \opt\ |into| ~var~}
Like |sum|, but the initial value of the result variable is $1$,
and the variable is updated by multiplying ~expr~ into it.

\clausex{counting}{~expr~ \opt\ |into| ~var~}
~expr~ is evaluated on each iteration.  If it is \nonnil, the
accumulation variable, initially zero, is incremented.

\startitem
\Clausex{maximize}{~expr~ \opt\ |into| ~var~}
\Clausex{minimize}{~expr~ \opt\ |into| ~var~}
\finishitem
~expr~ is evaluated on each iteration and its extremum (maximum or
minimum) is stored in the accumulation variable.  If ~expr~ is never
evaluated, then the result is \nil\ (if the accumulation variable is
untyped) or $0$ (if it has a numeric type).

\clausex{reducing}{~expr~ |by| ~func~ \opt\ |initial-value| ~init-val~ |into| ~var~}
This is a general way to perform reductions. ~func~ should be a
function of two arguments, the first of which will be the value
computed so far and
the second of which will be the value of ~expr~.  It should return the
new value.
|reducing| is roughly equivalent to the
Common Lisp |(reduce ~func~ ~sequence~ :key ~expr-function~)|, where
~expr-function~ is used to derive values from the successive elements
of ~sequence~.

\cpar If the |reducing| clause is never executed, the result is
undefined.

\cpar It is not necessary to provide an
initial value, but better code can be generated if one is supplied.
Regardless of its location in the \iter\ body, ~init-val~ will be
evaluated before the loop is entered, so it should not depend on any
value computed inside the \iter\ form.  

%\cpar if a ~var~ is not specified, you can get \iter\ to declare the
%type of the internal variable by putting a |the| expression around
%~func.~  see section \ref{types}.

\end{clauses}

%\begin{note}{implementation note}
%in principle, |maximize| and |minimize| can be thought of
%as reductions where the initial value is the smallest (or largest)
%value that the accumulation variable can assume.  because lisp's
%bignums can represent arbitrary integers, these clauses cannot be
%implemented as reductions in general.  if, however, the type of 
%~expr~ or ~var~ can be determined to be a fixnum
%or a float, \iter\ will implement the clause as a true reduction,
%using one of the constants |most-negative-fixnum|, |%most-positive-fixnum|, 
%|most-negative-short-float|, etc. as appropriate.
%\end{note}

\subsubsection{Accumulations}

All the predefined accumulation clauses add values to a sequence.  If
the sequence is a list, they all 
have the property that the partial list is kept in the correct order
and available for inspection at any point in the loop.

\begin{clauses}

\clausex{collect}{~expr~ \opt\ |into| ~var~ |at| ~place~ |result-type| ~type~}
Produces a sequence of the values of ~expr~ on each iteration. ~place~
indicates where the next value of ~expr~ is added to the list and may
be one of the symbols |start|, |beginning| (a synonym for |start|) or
|end|.  The symbol may be quoted, but need not be.  The default is
|end|.  For example,
\begin{program}
(iter (for i from 1 to 5)
      (collect i))
\end{program}
produces |(1 2 3 4 5)|, whereas
\begin{program}
(iter (for i from 1 to 5)
      (collect i at beginning))
\end{program}
produces |(5 4 3 2 1)| (and is likely to be faster in most Common Lisp
implementations).

\cpar If ~type~ is provided, it should be a subtype of |sequence|.
The default is |list|.  Specifying a type other than |list| will
result in |collect| returning a sequence of that type.  ~However,~ the
type of the sequence being constructed when inside the loop body is
undefined when a non-|list| type is specified.  (As with ~place~,
quoting ~type~ is optional.)

\clausex{adjoining}{~expr~ \opt\ |into| ~var~ |test| ~test~ |at| ~place~ |result-type| ~type~}
Like |collect|, but only adds the value of ~expr~ if it is not
already present.  ~test,~ which defaults to |\#'eql|, is
the test to be used with |member|.

\startitem
\Clausex{appending}{~expr~ \opt\ |into| ~var~ |at| ~place~}
\Clausex{nconcing}{\ ~expr~ \opt\ |into| ~var~ |at| ~place~}
\Clausex{unioning}{\ ~expr~ \opt\ |into| ~var~ |test| ~test~ |at| ~place~}
\Clausex{nunioning}{~expr~ \opt\ |into| ~var~ |test| ~test~ |at| ~place~}
\finishitem
These are like |collect|, but behave like the Common Lisp functions
|append|, |nconc|, |union| or |nunion|.  
As in Common Lisp, they work only on lists.  Also as in Common Lisp,
|unioning| and |nunioning| assume that the value of ~expr~ contains no
duplicates. 

\clausex{accumulate}{~expr~ |by| ~func~ \opt\ |initial-value| ~init-val~ |into| ~var~}
This is a general-purpose accumulation clause. ~func~ should be a
function of two arguments, the value of ~expr~ and the value
accumulated so far in the iteration, and it should return the updated
value.  If no initial value is supplied, \nil\ is used.  

%\cpar If a ~var~ is not specified, you can get \iter\ to declare the
%type of the internal variable by putting a |the| expression around
%~func.~  see section \ref{types}.

\cpar The differences between |accumulate| and |reducing| are slight.
One difference is that the functions take their arguments in a
different order.  Another is that in the absence of ~init-val~,
|accumulate| will use \nil, whereas |reducing| will generate different
code that avoids any dependence on the initial value.
The reason for having both clauses is that one usually
thinks of reductions (like |sum|) and accumulations (like |collect|)
as different beasts.

\end{clauses}


\subsubsection{Finders}

A ~finder~ is a clause whose value is an expression that meets some
condition.

\begin{clauses}

\clause{|finding| ~expr~ |such-that| ~test~ \opt\ |into| ~var~ |on-failure| ~failure-value~}
\clindexx{finding}{such-that}
If ~test~ (which is an expression) ever evaluates to \nonnil, the loop
is terminated, the 
epilogue code is run and the value of ~expr~ is returned.  Otherwise,
\nil\ (or ~failure-value,~ if provided) is returned.  If ~var~ is
provided, it will have either the \nonnil\ value of ~expr~ or
~failure-value~ when the epilogue code is run.

\cpar As a special case, if the ~test~ expression is a sharp-quoted
function, then it is applied to ~expr~ instead of being simply
evaluated.  E.g. |(finding x such-that \#'evenp)| is equivalent to
|(finding x such-that (evenp x))|.  

%\cpar although ~test~ need have nothing to do with ~%expr~ as in
%|(finding j such-that (> i 3))|, it usually 
%will: |(finding (length el) such-that (oddp (length el)))|.  to
%avoid performing the |length| computation twice, you could write
%|(finding (length el) such-that \#'oddp)| or |(finding (length
%el) such-that 'oddp)|; for these cases, \iter\ generates code that
%executes ~expr~ only once.  the code for |\#'oddp|
%is slightly different from that for {\lisp 'oddp}; see the discussion
%under {\lisp for\dots in} and {\lisp for\dots on}.

\cpar |On-failure| is a misnomer. Because it is always evaluated, it behaves
more like the default third argument to the |gethash| function. As a result,
|on-failure (error "Not found")| makes no sense. Instead, the clauses |leave|
or |thereis| can be used in conjunction with |finally| as follows:
\begin{program}
(iter (for x in '(1 2 3))
      (if (evenp x) (leave x))
      (finally (error "not found")))
\end{program}

\cpar This clause may appear multiple times when all defaults are
identical. It can also be used together with either |always|/|never| or
|thereis| if their defaults match. More specifically, |on-failure nil| is
compatible with |thereis|, while |on-failure t| is compatible with |always|
and |never| clauses.
\begin{program}
(iter (for i in '(7 -4 2 -3))
      (if (plusp i)
	  (finding    i  such-that (evenp i))
          (finding (- i) such-that (oddp i))))
\end{program}

\startitem
\Clause{|finding| ~expr~ |maximizing| ~m-expr~ \opt\ |into| ~var~}
\clindexx{finding}{maximizing}
\Clause{|finding| ~expr~ |minimizing| ~m-expr~ \opt\ |into| ~var~}
\clindexx{finding}{minimizing}
\finishitem
Computes the extremum (maximum or minimum) value of ~m-expr~ over all
iterations, and returns the value of ~expr~ corresponding to the
extremum.  ~expr~ is evaluated inside the loop at the time the new
extremum is established.  If ~m-expr~ is never evaluated (due to, for
example, being embedded in a conditional clause), then the returned
value depends on the type, if any, of ~expr~ (or ~var,~ if
one is supplied).  If there is no type, the returned
value will be \nil; if the type is numeric, the returned value will be
zero.

\cpar For these two clauses, ~var~ may be a list of two
symbols; in that case, the first is used to record ~expr~ and
the second, ~m-expr.~  

\cpar As with |finding\dots such-that|, if ~m-expr~ is a sharp-quoted
function, then it is called on ~expr~ instead of being evaluated.

\end{clauses}

\subsubsection{Boolean Tests}

\begin{clauses}

\clausex{first-iteration-p}{}
Returns |t| in the first cycle of the loop, otherwise \nil.

\clausex{first-time-p}{}
Returns |t| the first time the expression is evaluated, and then \nil\ forever.
This clause comes handy when printing (optional) elements separated
by a comma:

\begin{program}
(iter (for el in '(nil 1 2 nil 3))
      (when el
        (unless (first-time-p)
          (princ ", "))
        (princ el)))
\end{program}
produces |"1, 2, 3"|.

\end{clauses}

\subsubsection{Aggregated Boolean Tests}

\begin{clauses}

\clausex{always}{~expr~}
If ~expr~ ever evaluates to
\nil, then \nil\ is immediately returned; the epilogue code is not
executed.  If ~expr~ never evaluates to \nil, the epilogue code
is executed and the last value of ~expr~ (or |t| if ~expr~ was never
evaluated) is returned (whereas |loop| would constantly return |t|).

% mention last evaluated clause when multiple always clauses?

\clausex{never}{~expr~}
Like |(always (not ~expr~))|, except it does not influence the last
value returned by a possible other |always| clause. That is,
\begin{program}
(iter (repeat 2)
      (always 2)
      (never nil)) \yields 2 ; not t
\end{program}

\clausex{thereis}{~expr~}
If ~expr~ is ever \nonnil,
its value is immediately returned without running epilogue code.
Otherwise, the epilogue code is performed and \nil\ is returned.

This clause cannot be used together with |always| or |never|, because their
defaults are opposed (similarly, |(loop always 3 thereis nil)| refuses to
compile in some implementations of |loop|).

\end{clauses}


\subsection{Control Flow}
\label{control-flow}
Several clauses can be used to alter the usual flow of control in a loop.

Note: the clauses of this and subsequent sections don't adhere to \iter's
usual syntax, but instead use standard Common Lisp syntax.  Hence the
format for describing syntax subsequently is
like the standard format used in the Common Lisp manual, not like the
descriptions of clauses above.

\begin{clauses}

\clausex{finish}{}
Stops the loop and runs the epilogue code.

%for example:
%
%\begin{program}
%(iter (with answer = nil)
%         (initially (make-a-mess))
%         (for i from 1 to 10)
%         (when (correct? i)
%           (setq answer i)
%           (finish))
%         (finally (cleanup)))
%\end{program}
%
%this code will execute |cleanup| whether or not the test |(correct?
%i)| ever succeeds. 
%the (more elegant) formulation,
%\begin{program}
%(iter (initially (make-a-mess))
%         (for i from 1 to 10)
%         (finding i such-that (correct? i))
%         (finally (cleanup)))
%\end{program}
%would not execute |cleanup| if |(correct? i)| succeeded; it
%would do an immediate return.

\clausex{leave}{\opt\ ~value~}
Immediately returns ~value~ (default \nil) from the current \iter\
form, skipping the epilogue code.  Equivalent to using |return-from|.

\clausex{next-iteration}{}
Skips the remainder of the loop body and begins the next iteration of
the loop.

\clausex{while}{~expr~}
If ~expr~ ever evaluates to \nil, the loop is terminated and the
epilogue code executed.  Equivalent to |(if (not ~expr~) (finish))|.

\clausex{until}{~expr~}
Equivalent to |(if ~expr~ (finish))|.

\clausex{if-first-time}{~then~ \opt\ ~else~}
If this clause is being executed for the first time in this invocation
of the \iter\ form, then the ~then~ code is evaluated; otherwise the
~else~ code is evaluated.

\cpar |(for ~var~ first ~expr1~ then ~expr2~)| is almost equivalent to
\begin{program}
(if-first-time (dsetq ~var~ ~expr1~)
               (dsetq ~var~ ~expr2~))
\end{program}
The only difference is that the |for| version makes ~var~ available
for use with |for\dots previous|.

\end{clauses}

\subsection{Code Placement}
When fine control is desired over where code appears in a loop
generated by \iter, the following special clauses may be useful.
They are all subject to code-motion problems (see section
\ref{code-movement}).  

\begin{clauses}

\clausex{initially}{|\&rest| ~forms~}
The lisp ~forms~ are placed in the prologue section of the loop, where
they are executed once, before the loop body is entered.

\clausex{after-each}{|\&rest| ~forms~}
The ~forms~ are placed at the end of the loop body, where they
are executed after each iteration.  Unlike the other clauses in this
section, ~forms~ may contain \iter\ clauses.

\clausex{else}{|\&rest| ~forms~}
The lisp ~forms~ are placed in the epilogue section of the loop, where they
are executed if this |else| clause is never met during execution of the
loop and the loop terminates normally.

\clausex{finally}{|\&rest| ~forms~}
The lisp ~forms~ are placed in the epilogue section of the loop, where
they are executed after the loop has terminated normally.

\clausex{finally-protected}{|\&rest| ~forms~}
The lisp ~forms~ are placed in the second form of an unwind-protect
outside the loop.  They are always executed after the loop has
terminated, regardless of how the termination occurred.

\end{clauses}


%%%%%%%%%%%%%%%%%%%%%%%%%%%%%%%%%%%%%%%%%%%%%%%%%%%%%%%%%%%%%%%%
\section{Other Features}

\subsection{Multiple Accumulations}
\label{multiple}

\begin{sloppypar}
It is permitted to have more than one clause accumulate into the same
variable, as in the following:
\begin{program}
(iter (for i from 1 to 10)
      (collect i into nums)
      (collect (sqrt i) into nums)
      (finally (return nums)))
\end{program}
Clauses can only accumulate into the same variable if they are
compatible.  |collect|, |adjoining|, |appending|, |nconcing|,
|unioning| and |nunioning| are compatible with each other; |sum|,
|multiply| and |counting| are compatible; |always| and |never| are compatible;
|finding| \dots |such-that| is compatible with either |thereis| or |always|
and |never| when their defaults match; and |maximize| and |minimize| clauses
are compatible only with other |maximize| and |minimize| clauses,
respectively.

%note that the same variable ~cannot~ be both an accumulation
%variable and an ordinary variable; there can be only one variable with
%a given name within an \iter\ form. 
\end{sloppypar}

\subsection{Named Blocks}

Like Common Lisp |block|s, \iter\ forms can be given names.  The
name should be a single symbol, and it must be the first form in the
\iter.  The generated code behaves exactly like a named block; in
particular, |(return-from ~name~)| can be used to exit it:
\begin{program}
(iter fred
      (for i from 1 to 10)
      (iter barney
            (for j from i to 10)
            (if (> (* i j) 17)
                (return-from fred j))))
\end{program}
An \iter\ form that is not given a name is implicitly named \nil.

Sometimes one would like to write an expression in an inner \iter\ form,
but have it processed by an outer \iter\ form.  This is possible with
the |in| clause.  

\begin{clauses}

\clausex{in}{~name~ |\&rest| ~forms~}
Evaluates ~forms~ as if they were part of the \iter\ form named
~name~.  In other words, \iter\ clauses are processed by the \iter\
form named ~name,~ and not by any \iter\ forms that occur inside ~name.~

\cpar As an example, consider the problem of collecting a list of the
elements in a two-dimensional array.  The naive solution,
\begin{program}
(iter (for i below (array-dimension ar 0))
      (iter (for j below (array-dimension ar 1))
            (collect (aref ar i j))))
\end{program}
\noindent is wrong because the list created by the inner \iter\ is simply
ignored by the outer one.  But using |in| we can write:
\begin{program}
(iter outer (for i below (array-dimension ar 0))
      (iter (for j below (array-dimension ar 1))
            (in outer (collect (aref ar i j)))))
\end{program}
\noindent which has the desired result.

\end{clauses}

\subsection{Destructuring}
\label{destructuring}

In many places within \iter\ clauses where a variable is expected, a
list can be written instead.  In these cases, the value to be assigned
is ~destructured~ according to the pattern
described by the list.  As a simple example, the clause
\begin{program}
(for (key . item) in alist)
\end{program}
\noindent will result in |key| being set to the |car| of
each element in |alist|, and |item| being set to the |cdr|.  The
pattern list may be nested to arbitrary depth, and (as the example
shows) need not be terminated with \nil; the only requirement is that
each leaf be a bindable symbol (or \nil, in which case no binding is
generated for that piece of the structure).

Sometimes, you might like to do the equivalent of a
|multiple-value-setq| in a clause.  This 
``multiple-value destructuring'' can be expressed by writing 
\linebreak |(values $pat_1$ $pat_2 \ldots$)| for a destructuring
pattern, as in 
\begin{program}
(for (values (a . b) c d) = (three-valued-function ...))
\end{program}
\begin{sloppypar}
Note that the $pat_i$ can themselves be destructuring patterns (though
not multiple-value destructuring patterns).  You can't do multiple-value
destructuring in a |with| clause; instead wrap the whole \iter\
form in a |multiple-value-bind|.
\end{sloppypar}

\begin{note}{Rationale}
There are subtle interactions between variable declarations and
evaluation order that make the correct implementation of
multiple-value destructuring in a |with| somewhat tricky.
\end{note}

The destructuring feature of \iter\ is available as a separate
mechanism, using the |dsetq| macro:

\begin{clauses}
\defmacro{dsetq}{template expr}
Performs destructuring of ~expr~ using ~template~.   May be used
outside of an \iter\ form. Yields the primary value of ~expr~.

\end{clauses}

\subsection{On-line Help}

\begin{sloppypar}
There is a limited facility for on-line help, in the form of the 
|display-iterate-clauses| function.
\end{sloppypar}

\begin{clauses}

\defun{display-iterate-clauses}{\opt\ clause-spec}
Displays a list of \iter\ clauses.  If ~clause-spec~ is not
provided, all clauses are shown; if it is a symbol, all clauses
beginning with that symbol are shown; and if it is a list of symbols,
all clauses for which ~clause-spec~ is a prefix are shown.

\end{clauses}

\subsection{Parallel Binding and Stepping}
\label{bindings}

The parallel binding and stepping of variables is a feature that
\iter\ does ~not~ have.  This section attempts to provide a rationale.

We say that two variables are bound ~in parallel~ if neither
binding shadows the other.  This is the usual semantics of |let|
(as opposed to |let*|).  Similarly, we can say that iteration
variables are stepped in parallel if neither variable is updated
before the other, conceptually speaking; in other words, if the code
to update each variable can reference the old values of both variables.

|loop| allows parallel binding of variables and parallel stepping of
driver variables.  My view is that if you are depending on the 
serial/parallel distinction, you are doing something obscure.  If you
need to bind 
variables in parallel using |with|, then you must be using a
variable name that shadows a name in the existing lexical environment.
Don't do that.  The most common use for parallel stepping is to track
the values of variables on the previous iteration, but in fact this
does not require parallel stepping at all; the following will work:
\begin{program}
(iter (for current in list)
      (for prev previous current)
      ...)
\end{program}


%%%%%%%%%%%%%%%%%%%%%%%%%%%%%%%%%%%%%%%%%%%%%%%%%%%%%%%%%%%%%%%%
\section{Types and Declarations}
\label{types}

\subsection{Discussion}

Sometimes efficiency dictates that the types of variables be declared.
This type information needs to be communicated to \iter\ so it can
bind variables to appropriate values.  Furthermore, \iter\ must often
generate internal variables invisible to the user; there needs to be a
way for these to be declared.

As an example, consider this code, which will return the number of
odd elements in |number-list|:
\begin{program}
(iter (for el in number-list)
      (count (oddp el)))
\end{program}
In processing this form,
\iter\ will create an internal variable, let us call it |list17|, to
hold the successive |cdr|s of |number-list|, and  
will bind the variable to |number-list|.  It will also generate a
default binding for |el|; only inside the body of the loop will |el|
be set to the |car| of |list17|.  Finally, \iter\ will generate a
variable, call it |result|, to hold the result of the count, and will
bind it to zero.

When dealing with type declarations, \iter\ observes one simple rule:
~it will never generate a declaration unless requested to do so.~  The
reason is that such declarations might mask errors in compiled code by
avoiding error-checks; the resulting problems would be doubly hard to
track down because the declarations would be hidden from the
programmer.  Of course, a compiler might omit error-checks even in the
absence of declarations, though this behavior can usually be avoided,
e.g. by saying |(declaim (optimize (safety 3)))|.

So, the above \iter\ form will generate code with no declarations.
But say we wish to declare the types of |el| and the internal
variables |list17| and |result|.  How is this done?

Declaring the type of |el| is easy, since the programmer knows
the variable's name:
\begin{program}
(iter (for el in number-list)
      (declare (fixnum el))
      (counting (oddp el)))
\end{program}
\iter\ can read variable type declarations like this one.  Before
processing any clauses, it scans the entire top-level form for type
declarations 
and records the types, so that variable bindings can be performed
correctly.  In this case, |el| will be bound to zero
instead of \nil.  Also, \iter\ collects all the top-level declarations
and puts them at the begining of the generated code, so it is not
necessary to place all declarations at the beginning of an \iter\
form; instead, they can be written near the variables whose types they
declare.  

Since \iter\ is not part of the compiler, it will not know
about declarations that occur outside an \iter\ form; these
declarations must be repeated inside the form.

Here is another way we could have declared the type of |el|:
\begin{program}
(iter (for (the fixnum el) in number-list)
      (counting (oddp el)))
\end{program}
\iter\ extends the Common Lisp |the|\lispindex{the}
form to apply to variables as well as value-producing forms; anywhere
a variable is allowed---in a |with| clause, as the iteration
variable in a driver clause, as the |into| argument of an
accumulation clause, even inside a destructuring template---you can
write |(the ~type~ ~symbol~)| instead.

There is one crucial difference between using a |the| form and
actually declaring the variable: explicit declarations are always
placed in the generated code, but type information from a |the|
form is not turned into an actual declaration unless you tell \iter\
to do so using |iterate:declare-variables|.  See below.

\begin{sloppypar}
Declaring the types of internal variables is harder than declaring the
types of explicitly mentioned variables, since their names
are unknown.  You do it by declaring |iterate:declare-variables|
somewhere inside the top level of the \iter\ form.  (This will also
generate declarations for variables declared using |the|.)
\iter\ does not provide much selectivity here: it's all or none. 
And unfortunately, since \iter\ is not privy to compiler information
but instead reads declarations itself, it will not hear if you
|(declaim (iterate:declare-variables))|.  Instead, set the variable 
|iterate::*always-declare-variables*| to |t| at
compile-time, using |eval-when|.
\end{sloppypar}

To determine the appropriate types for internal variables, \iter\ uses
three sources of information:

\begin{itemize}

\item Often, the particular clause dictates a certain type for a
variable; \iter\ will use this information when available.  In the
current example, the variable |list17| will be given the type
|list|, since that is the only type that makes sense; and the
variable |result| will be given the type |fixnum|, on the
assumption that you will not be counting high enough to need bignums.
You can override this assumption only by using and explicitly declaring a
variable: 
\begin{verbatim}
(iter (declare (iterate:declare-variables))
      (for el in number-list)
      (count (oddp el) into my-result)
      (declare (integer my-result))
      (finally (return my-result)))
\end{verbatim}

\begin{sloppypar}
Other examples of the type assumptions that \iter\ makes are: type
|list| for |into| variables of collection clauses; type |list| for
expressions that are to be destructured; type |vector| for the
variable holding the vector in a |for\dots in-vector| clause, and
similarly for |string| and the |for\dots in-string| clause;
and the implementation-dependent type
|(type-of array-dimension-limit)| for the index and limit
variables generated by sequence iteration drivers like |for\dots
in-vector| and |for\dots in-string| (but not |for\dots in-sequence|,
because it may be used to iterate over a list). 
\end{sloppypar}

\item Sometimes, \iter\ will examine expressions and try to determine
their types in a simple-minded way.  If the expression is
self-evaluating (like a number, for instance), \iter\ knows that the
expression's type is the same as the type of the value it denotes, so
it can use that type.  If the expression is of the form |(the ~type~
~expr~)|, \iter\ is smart enough to extract ~type~ and use it.
However, the current version of \iter\ does not  
examine declarations of function result types or do any type
inference.  It will not determine, for 
example, that the type of |(+ 3 4)| is |fixnum|, or even
|number|.

\item In some cases, the type of an internal variable should match the
type of some other variable.  For instance, \iter\ generates an
internal variable for |(f x)| in the
clause |(for i from 1 to (f x))|, and in the absence of other
information will give it the same type as |i|.  If, however, the
expression had been written |(the fixnum (f x))|, then \iter\
would have given the internal variable the type |fixnum|
regardless of |i|'s type.  The type incompatibility errors that
could arise in this situation are not checked for.

\end{itemize}

Note that if you do declare |iterate:declare-variables|, then
\iter\ may declare user variables as well as internal ones if they do
not already have declarations, though only for variables that it
binds.  For instance, in this code:

\begin{program}
(iter (declare (iterate:declare-variables))
      (for i from 1 to 10)
      (collect i into var))
\end{program}
the variable |var| will be declared to be of type |list|.


\subsection{Summary}

\iter\ understands standard Common Lisp variable type declarations
that occur within an \iter\ form and
will pass them through to the generated code.  If the declaration
|(iterate:declare-variables)|\lispindex{declare-variables}
appears at the top level of an 
\iter\ form, or if 
|iterate::*always-declare-variables*|\lispindex{*always-declare-variables*} 
is \nonnil, then \iter\ will use the type information gleaned from user
declarations, self-evaluating expressions and |the| expressions,
combined with reasonable assumptions, to determine variable
types and declare them.


%%%%%%%%%%%%%%%%%%%%%%%%%%%%%%%%%%%%%%%%%%%%%%%%%%%%%%%%%%%%%%%%
\section{Problems with Code Movement}
\label{code-movement}

Some \iter\ clauses, or parts of clauses, result in code being
moved from the location of the clause to other parts of the loop.
Drivers behave this way, as do code-placement clauses like |initially|
and |finally|.  When using these clauses, there is a danger of writing
an expression that makes sense in its apparent location but will be
invalid or have a different meaning in another location.  For example:
\begin{program}
(iter (for i from 1 to 10)
      (let ((x 3))
        (initially (setq x 4))))
\end{program}
While it may appear that the |x| of |(initially (setq x 4))| is the
same as the |x| of |(let ((x 3)) \dots|, in fact they are not:
|initially| moves its code outside the loop body, so |x| would refer
to a global variable.  Here is another example of the same problem:
\begin{program}
(iter (for i from 1 to 10)
      (let ((x 3))
        (collect i into x)))
\end{program}
If this code were executed, |collect| would create a binding for its
|x| at the top level of the \iter\ form that the |let| will shadow.

Happily, \iter\ is smart enough to catch these errors; it
walks all problematical code to ensure that free variables are not
bound inside the loop body, and checks all variables it binds for the
same problem.

However, some errors cannot be caught:

\begin{program}
(iter (with x = 3)
      (for el in list)
      (setq x 1)
      (reducing el by \#'+ initial-value x))
\end{program}
|reducing| moves its |initial-value| argument to the initialization
part of the loop in order to produce more efficient code.  Since
\iter\ does not perform data-flow analysis, it cannot determine that
|x| is changed inside the loop; all it can establish is that |x| is
not bound internally.  Hence this code will not signal an
error and will use $3$ as the initial value of the reduction.

The following list summarizes all cases that are subject to these code
motion and variable-shadowing problems.
\begin{itemize}
\item Any variable for which \iter\ creates a binding, including those
used in |with| and the |into| keyword of many clauses.

\begin{sloppypar}
\item The special clauses which place code: |initially|, |after-each|, |else|,
|finally| and |finally-protected|. 
\end{sloppypar}

\item The variables of a |next| or |do-next| form.

\item The |initially| arguments of |for\dots initially\dots then| and
|for\dots previous|. 

\item The |then| argument of |for\dots initially\dots then|.

\item The |initial-value| arguments of |reducing| and |accumulate|.

\item The |on-failure| argument of |finding\dots such-that|.
\end{itemize}

%%%%%%%%%%%%%%%%%%%%%%%%%%%%%%%%%%%%%%%%%%%%%%%%%%%%%%%%%%%%%%%%
\section{Differences Between {\tt Iterate} and {\tt Loop}}
|loop| contains a great deal of complexity which \iter\ tries to
avoid.  Hence many esoteric features of |loop| don't exist in \iter.
Other features have been carried over, but in a cleaned-up form.  
And of course, many new features have been added; they are not
mentioned in this list.

\begin{itemize}

\item \iter's syntax is more Lisp-like than |loop|'s, having a higher
density of parens.

\item The current implementation of \iter, unlike the current version
of  |loop| (as documented in {\em Common Lisp, 2nd Ed.\/}), is
extensible (see section \ref{extend}).  

\item |loop| puts the updates of all driver variables at the top of
the loop; \iter\ leaves them where the driver clauses appear.

\item While for the most part \iter\ clauses that resemble |loop| clauses
behave similarly, there are some differences.  For instance, there is
no |for\dots =\dots then| in \iter; instead use 
|for\dots initially\dots then|.  

\item |loop| binds the variable |it| at certain times
to allow pseudo-English expressions like |when ~expr~ return it|. 
In \iter, you must bind ~expr~ to a variable yourself.  Note that
|when ~expr~ return it| is like |thereis ~expr~| except that the latter is an
accumulation clause and therefore competes with other accumulations
(remember section \ref{multiple} above).

% repeat different behaviour of |always| clause here?

\item |loop| has a special |return| clause, illustrated in the
previous item.  \iter\ doesn't need one,  since an ordinary Lisp
|return| has the same effect.

\item |loop| allows for parallel binding and stepping of iteration
variables.  \iter\ does not.  (See section \ref{bindings}.)

\item |loop| and \iter\ handle variable type declarations very
differently.  |loop| provides a special syntax for declaring variable
types, and does not examine declarations.  Moreover, the standard
implementation of |loop| will
generate declarations when none are requested.
\iter\ parses standard Common Lisp type declarations, and will never
declare a variable itself unless declarations are 
specifically requested.

\end{itemize}

%%%%%%%%%%%%%%%%%%%%%%%%%%%%%%%%%%%%%%%%%%%%%%%%%%%%%%%%%%%%%%%%
\section{Rolling Your Own}
\label{extend}

\subsection{Introduction}

\iter\ is extensible---you can write new clauses that embody new
iteration patterns.  You might want to write a new driver clause for a
data structure of your own, or you might want to write a clause that
collects or manipulates elements in a way not provided by \iter. 

This section describes how to write clauses for \iter.  Writing a
clause is like writing a macro.  In fact, writing a clause ~is~
writing a macro: since \iter\ code-walks its body and macroexpands,
you can add new abstractions to \iter\ with good old |defmacro|.

Actually, there are two extensions you can make to \iter\ that are
even easier than writing a macro.  They are adding a synonym for an
existing clause and defining a driver clause for an indexable
sequence.  These can be done with |defsynonym| and
|defclause-sequence|, respectively.  See section \ref{aids}, below.

The rest of this section explains how to write macros that expand into
\iter\ clauses.
Here's how you could add a simplified version of \iter's |multiply|
clause, if \iter\ didn't already have one:
\begin{program}
(defmacro multiply (expr)
  `(reducing ,expr by \#'* initial-value 1))
\end{program}

If you found yourself summing the square of an expression often, you
might want to write a macro for that.  A first cut might be
\begin{program}
(defmacro sum-of-squares (expr)
  `(sum (* ,expr ,expr)))
\end{program}
but if you are an experienced macro writer, you will realize that this
code will evaluate ~expr~ twice, which is probably a bad idea.  A
better version would use a temporary:
\begin{program}
(defmacro sum-of-squares (expr)
  (let ((temp (gensym)))
    `(let ((,temp ,expr))
       (sum (* ,temp ,temp)))))
\end{program}
Although this may seem complex, it is just the sort of thing you'd
have to go through to write any macro, which illustrates the point of
this section: if you can write macros, you can extend \iter.

Our macros don't use \iter's keyword-argument
syntax.  We could just use keywords with |defmacro|, but we would
still not be using \iter's clause indexing mechanism.  Unlike Lisp,
which uses just the first symbol of a form to determine what function
to call, \iter\ individuates clauses by the list of required keywords.
For instance, |for\dots in| and |for\dots in-vector| are different clauses
implemented by distinct Lisp functions.  

To buy into this indexing scheme, as well as the keyword-argument
syntax, use |defmacro-clause|:

\begin{clauses}

\defmacro{defmacro-clause}{arglist |\&body| body}
Defines a new \iter\ clause.  ~arglist~ is a list of symbols which are
alternating keywords and arguments.  \opt\ may be used, and the list
may be terminated by |\&sequence|.  ~body~ is an ordinary macro body,
as with |defmacro|.  If the first form of ~body~ is a string, it is
considered a documentation string and will be shown by
|display-iterate-clauses|. |defmacro-clause| will signal an error if 
defining the clause would result in an ambiguity.  E.g. you cannot
define the clause |for\dots from| because there would be no way to
distinguish it from a use of the |for| clause with optional keyword |from|.

\end{clauses}

\medskip

Here is |multiply| using |defmacro-clause|.  The keywords are capitalized
for readability.
\begin{program}
(defmacro-clause (MULTIPLY expr \opt\ INTO var)
  `(reducing ,expr by \#'* into ,var initial-value 1))
\end{program}
You don't have to worry about the case when |var| is not supplied; for
any clause with an |into| keyword, saying |into nil| is equivalent to
omitting the |into| entirely.

As another, more extended example, consider the fairly common
iteration pattern that 
involves finding the sequence element that maximizes (or minimizes) some
function.  \iter\ provides this as |finding\dots maximizing|, but it's
instructive to see how to write it.
Here, in pseudocode, is how you might write such a loop for
maximizing a function F:

\begin{quote}
\begin{tabbing}
set variable MAX-VAL to NIL; \\
set variable WINNER to NIL; \\
for\=\ each element EL in the sequence \\
\>  if\=\ MAX-VAL\=\ is NIL or F(EL) $>$ MAX-VAL then \\
\>\>    set MAX-VAL to F(EL); \\
\>\>    set WINNER to EL; \\
\>  end if; \\
end for; \\
return WINNER.
\end{tabbing}
\end{quote}

Here is the macro:
\begin{program}
(defmacro-clause (FINDING expr MAXIMIZING func \opt\ INTO var)
  (let ((max-val (gensym))
        (temp1 (gensym))
        (temp2 (gensym))
        (winner (or var iterate::*result-var*)))
    `(progn 
       (with ,max-val = nil)
       (with ,winner = nil)
       (cond 
        ((null ,max-val)
         (setq ,winner ,expr)
         (setq ,max-val (funcall ,func ,winner))
        (t
         (let* ((,temp1 ,expr)
                (,temp2 (funcall ,func ,temp1)))
           (when (> ,temp2 ,max-val)
             (setq ,max-val ,temp2)
             (setq ,winner ,temp1))))))
       (finally (leave ,winner)))))
\end{program}
Note that if no |into| variable is supplied, we use
|iterate::*result-var*|, which contains the internal variable into
which all clauses place their results.  If this variable is bound by
some clause, then \iter\ will return its value automatically;
otherwise, \nil\ will be returned.  

\subsection{Writing Drivers}

In principle, drivers can be implemented just as easily as other
\iter\ clauses.  In practice, they are a little harder to get right. 
As an example, consider writing a driver that
iterates over all the 
elements of a vector, ignoring its fill-pointer.  |for\dots in-vector|
won't work for this, because it observes the fill-pointer.  It's necessary to
use |array-dimension| instead of |length| to obtain the size of the
vector.  Here is one approach:

\begin{program}
(defmacro-clause (FOR var IN-WHOLE-VECTOR v)
  "All the elements of a vector (disregards fill-pointer)"
  (let ((vect (gensym))
        (index (gensym)))
    `(progn
       (with ,vect = ,v)
       (for ,index from 0 below (array-dimension ,vect 0))
       (for ,var = (aref ,vect ,index)))))
\end{program}
Note that we immediately put |v| in a variable, in case it is an
expression.  Again, this is just good Lisp macrology.  It also has a
subtle effect on the semantics of the driver: |v| is evaluated only
once, at the beginning of the loop, so changes to |v| in the loop have
no effect on the driver.  Similarly, the bounds for numerical iteration
e.g. the above |array-dimension| are also evaluated once only.  This is how
all of \iter's drivers work.


There is an important point concerning the |progn| in this code.  We
need the |progn|, of course, because we are returning several forms,
one of which is a driver.  But \iter\ drivers must occur at top-level.
Is this code in error?  No, because ~top-level~ is defined in \iter\
to include forms inside a |progn|.  This is just the definition of
top-level that Common Lisp uses, and for the same reason: to allow
macros to return multiple forms at top-level.

While our |for\dots in-whole-vector| clause will work, it is not
ideal.  In particular, it does not support generating.  Do do so, we
need to use |for\dots next| or |for\dots do-next|.  The job is
simplified by the |defmacro-driver| macro.

\begin{clauses}

\defmacro{defmacro-driver}{arglist |\&body| body}
Defines a driver clause in
both the |for| and |generate| forms, and provides a parameter
|generate| which ~body~ can examine to determine how it was invoked.
~arglist~ is as in |defmacro-clause|, and should begin with the symbol
|for|. 

\end{clauses}

With |defmacro-driver|, our driver looks like this:
\begin{program}
(defmacro-driver (FOR var IN-WHOLE-VECTOR v)
  "All the elements of a vector (disregards fill-pointer)"
   (let ((vect (gensym))
         (end (gensym))
         (index (gensym))
         (kwd (if generate 'generate 'for)))
     `(progn
        (with ,vect = ,v)
        (with ,end = (array-dimension ,vect 0))
        (with ,index = -1)
        (,kwd ,var next (progn (incf ,index)
                               (if (>= ,index ,end) (terminate))
                               (aref ,vect ,index))))))
\end{program}


We are still missing one thing: the |\&sequence| keywords.  
We can get them easily enough, by writing 
\begin{program}
(defmacro-driver (FOR var IN-WHOLE-VECTOR v \&sequence) 
  ...)
\end{program}
We can now refer to parameters |from|, |to|, |by|, etc. which contain
either the values for the corresponding keyword, or \nil\ if the
keyword was not supplied.  Implementing the right code for these
keywords is cumbersome but not difficult; it is left as an exercise.
But before you begin, see |defclause-sequence| below for an easier way.
       
\subsection{Extensibility Aids}
\label{aids}

This section documents assorted features that may be of use in
extending \iter.

\begin{clauses}

\defunexpvar{*result-var*}
Holds the variable that is used to return a value as a result of the
\iter\ form.  You may examine this and use it in a |with| clause, but
you should not change it.

\defmacro{defsynonym}{syn word}
 Makes ~syn~ a synonym for the existing \iter\ keyword ~word.~  Only
the first word in each clause can have synonyms. 

\defmacroxx{defclause-sequence}{element-name index-name}{|\&key|
access-fn size-fn sequence-type}{element-type
element-doc-string index-doc-string}  
Provides
a simple way to define sequence clauses.  Generates two
clauses, one for iterating over the sequence's elements, the other
for iterating over its indices.  The first symbol of both
clauses will have print-name |for|.
~element-name~  and ~index-name~ should be symbols.
~element-name~ is the second keyword of the element iterator (typically of
the form 
|in-~sequence-type~|), and ~index-name~ is the second keyword
of the index-iterator (typically of the form
|index-of-~sequence-type~|).  Either name may be 
\nil, in which case the corresponding clause is not defined.  If both
symbols are supplied, they should be in the same package.  The |for|
that begins the clauses will be in this package.

\cpar ~access-fn~ is the function to be used to
access elements of the sequence in the element iterator.  The function
should take two 
arguments, a sequence and an index, and return the appropriate element.
~size-fn~ should denote a function of one argument, a sequence, that
returns its size.  Both ~access-fn~ and ~size-fn~ are required for the
element iterator, but only ~size-fn~ is needed for the index iterator.

\cpar The ~sequence-type~ and ~element-type~ keywords are used to
suggest types for the variables 
used to hold the sequence and the
sequence elements, respectively.  The usual rules about \iter's
treatment of variable type declarations apply (see section \ref{types}).

\cpar ~element-doc-string~ and ~index-doc-string~ are
the documentation strings, for use with |display-iterate-clauses|. 

\cpar The generated element-iterator performs destructuring on the
element variable.

\cpar As an example, the above |for\dots in-whole-vector| example
could have been written:
\begin{program}
(defclause-sequence IN-WHOLE-VECTOR INDEX-OF-WHOLE-VECTOR
  :access-fn 'aref
  :size-fn \#'(lambda (v) (array-dimension v 0))
  :sequence-type 'vector
  :element-type t
  :element-doc-string 
     "Elements of a vector, disregarding fill-pointer"
  :index-doc-string 
     "Indices of vector, disregarding fill-pointer")
\end{program}

\end{clauses}

\subsection{Subtleties}

There are some subtleties to be aware of when writing \iter\ clauses.
First, the code returned by your macros may be |nconc|'ed into a list,
so you should always returned freshly consed lists, rather than
constants.  Second, \iter\ matches clauses by using |eq| on the first
symbol and |string=| on the subsequent ones, so the package of the
first symbol of a clause is relevant.  All of the clauses in this manual
have their first word in the \iter\ package.  
You can use the package system in the usual way to shadow
\iter\ clauses without replacing them.

%%% say more here, about the badness that only the first word of a
%%% clause is packagey.

\section{Non-portable Extensions to Iterate (Contribs)}
\label{contribs}

Currently, there is only one non-portable extension to iterate in the
distribution: iterate-pg. If you have made an extension that depends
on non-portable features, feel free to send them to |asf@boinkor.net|
for inclusion in the iterate distribution.

\subsection{An SQL query driver for iterate}

The pg package by Eric Marsden (see |http://cliki.net/pg|) provides an
interface to the PostgreSQL database. Using the \iterpg\ extension, it
is possible to handle the results of SQL queries using \iter.

This usage example should give you an idea of how to use it:

\begin{program}
(pg:with-pg-connection (c "somedb" "someuser")
  (iter (for (impl version date) in-relation "select * from version"
                                 on-connection *dbconn*)
        (collect version)))
\end{program}

To use the extension via |ASDF|, simply make your system depend on the
|iterate-pg| system instead of the |iterate| system. To load it
manually, use:

\begin{program}
  (asdf:oos 'asdf:load-op :iterate-pg)
\end{program}

%%%%%%%%%%%%%%%%%%%%%%%%%%%%%%%%%%%%%%%%%%%%%%%%%%%%%%%%%%%%%%%%
\section{Obtaining {\tt Iterate}}

\begin{sloppypar}
The information in this chapter is obsolete but included for
completeness's sake; Currently, the most up-to-date information on
\iter\ can be found at |http://boinkor.net/iterate.html|.
\end{sloppypar}

\begin{sloppypar}
\iter\ currently runs on Lisp Machines, and on
HP's, Sun3's and Sparcstations under Lucid.  
\iter\ source and binaries are available at the MIT AI Lab in the
subdirectories of |/src/local/lisplib/|.  The source file,
|iterate.lisp|, is also available for anonymous FTP in the directory
|/com/fpt/pub/| on the machine |TRIX.AI.MIT.EDU| (Internet number
128.52.32.6).  If you are unable to obtain |iterate| in one of these
ways, send mail to |jba@ai.mit.edu| and I will send you the source
file. 
\end{sloppypar}

\begin{sloppypar}
\iter\ resides in the |iterate| package (nickname |iter|).  Just say
\linebreak |(use-package :iterate)| to make all the necessary symbols
available. 
If a symbol is not exported, it appears in this manual with an
``|iterate::|'' prefix.
\end{sloppypar}

Send bug reports to |bug-iterate@ai.mit.edu|.  The |info-iterate|
mailing list will have notices of changes and problems; to have
yourself added, send mail to |info-iterate-request@ai.mit.edu|.


\medskip

\begin{flushleft}
 \bf Acknowledgements
\end{flushleft}
\smallskip

Richard Waters provided invaluable criticism which spurred me to improve
\iter\ greatly.  As early users, David Clemens, Oren Etzioni and Jeff
Siskind helped ferret out many bugs.

%\begin{theindex}
% The index files must be generated with the genindex program
% or from makeindex iterate-manual.idx which creates iterate-manual.ind.

%\baselineskip\normalbaselineskip
%\advance\baselineskip by -2pt

\input{iterate-manual.ind}

%\underline{Clauses}
%\input{iterate-manual.clindex}
%
%\indexspace
%\underline{Lisp}
%\input{iterate-manual.lispindex}

%\end{theindex}


\end{document}


% arch-tag: "c0f871a6-313c-11d8-abb9-000c76244c24"


%\end{theindex}


\end{document}


% arch-tag: "c0f871a6-313c-11d8-abb9-000c76244c24"


%\end{theindex}


\end{document}


% arch-tag: "c0f871a6-313c-11d8-abb9-000c76244c24"


%\underline{Clauses}
%% Fix ``.'' spacing in lists
\documentstyle[12pt,aimemo]{article}

\setlength{\textwidth}{6.0in}
\setlength{\oddsidemargin}{0.25in}
\setlength{\evensidemargin}{0.25in}
\setlength{\topmargin}{0pt}
\setlength{\headsep}{24pt}
\setlength{\textheight}{8.5in}

\let\origpar=\par

% somehow, LaTex fucks with things so that the plain Tex \obeyspaces
% doesn't work.  This is the same, except we use an mbox containing
% a space instead of just a space.

\newcommand{\spc}{\mbox{ }}
\newcommand{\tAb}{\mbox{        }}
\def\obspaces{\catcode`\ =\active\catcode`\^^I=\active}
{\obspaces \global\let =\spc\global\let^^I=\tAb}

\newenvironment{program}{\medskip\samepage\tt\obeylines\obspaces}{\medskip}

% use | for \tt
\catcode`\|=13
\newif\ifvbar
\def|{\ifvbar
         \endgroup\vbarfalse
      \else 
         \begingroup\tt\vbartrue 
      \fi}



\newcommand{\lisp}{\tt}
\newcommand{\iter}{|iterate|}
\newcommand{\iterpg}{|iterate-pg|}
\newcommand{\setf}{|setf|}
\newcommand{\nil}{|nil|}
\newcommand{\nonnil}{non-\nil}
\newcommand{\opt}{{\lisp \&optional}}
\newcommand{\yields}{$\Rightarrow$}



\newbox\Dots
\setbox\Dots=\hbox{\small ...}
\renewcommand{\dots}{\copy\Dots}

% Here's a way to define an environment which will treat all
% paragraphs inside it in the same manner.  It's, as near as I can
% figure, the way that Latex does things.  Using \everypar won't work;
% apparently Latex resets that a lot.  What you have to do is: at the
% top of the file do \let\origpar=\par.  In the
% begin part of the environment, do the parshape (or whatever), then say
% \def\par{{\origpar}}.  The extra braces are necessary.   That's all
% you have to do; at the end of the environment things will return to
% normal automatically.  Why does this all work?  I have no idea.
% Another important thing: there should be a blank line (i.e. a \par)
% between the end of the last paragraph and the \end{environment}.

\newlength{\clindent}
\setlength{\clindent}{0.5in}
\newlength{\clparindent}
\setlength{\clparindent}{\parindent}

\newenvironment{clauses}{\advance\linewidth -\parindent
                        \hangindent\clindent\relax
                         \hangafter=0
                          \parindent=0pt
                          \def\par{{\origpar}}}{}

\newcommand{\cpar}{\hspace \clparindent}

\newcommand{\startitem}{\bigskip\pagebreak[3]}
\newcommand{\finishitem}{\smallskip}

\newcommand{\clause}[1]{\startitem\Clause{#1}\finishitem}
\newcommand{\Clause}[1]{\hbox{#1}}
\newcommand{\clindex}[1]{\index{{\lisp #1}}}
\newcommand{\clindexx}[2]{\index{{\lisp #1\dots #2}}}
\newcommand{\clindexxx}[3]{\index{{\lisp #1\dots #2\dots #3}}}
\newcommand{\clausex}[2]{\startitem\Clausex{#1}{#2}\finishitem}
\newcommand{\Clausex}[2]{\Clause{|#1| #2}\clindex{#1}}

\newcommand{\defvar}[1]{\startitem\deff{#1}{}{Variable}\finishitem}
\newcommand{\defunexpvar}[1]{\startitem\deffnoindex{iterate::#1}{}{Variable}
                                        \lispindex{#1}\finishitem}
\newcommand{\defun}[2]{\startitem\deff{#1}{#2}{Function}\finishitem}
\newcommand{\defmacro}[2]{\startitem\deff{#1}{#2}{Macro}\finishitem}
\newcommand{\defunx}[3]{\startitem\deffx{#1}{#2}{#3}{Function}\finishitem}
\newcommand{\defmacrox}[3]{\startitem\deffx{#1}{#2}{#3}{Macro}\finishitem}
\newcommand{\defunxx}[4]{\startitem\deffxx{#1}{#2}{#3}{#4}{Function}\finishitem}
\newcommand{\defmacroxx}[4]{\startitem\deffxx{#1}{#2}{#3}{#4}{Macro}\finishitem}

\newcommand{\deff}[3]{\deffnoindex{#1}{#2}{#3}\lispindex{#1}}
\newcommand{\deffnoindex}[3]{\hbox to \hsize{{\tt #1} {\it #2}\hfill [{\it #3}]}}
\newcommand{\deffx}[4]{\deff{#1}{#2}{#4}\moreargs{#1}{#3}}
\newcommand{\deffxx}[5]{\deffx{#1}{#2}{#3}{#5}\moreargs{#1}{#4}}

\newcommand{\moreargs}[2]{\hbox to \hsize{\phantom{\lisp #1} {\it #2}\hfill}}

\newcommand{\lispindex}[1]{\index{{\lisp #1}}}

 \newenvironment{note}[1]{\pagebreak[2]\bigskip
                          \hrule\smallskip\small
                         {\setlength{\parindent}{0pt}
                           \par{\bf #1:}}}{\smallskip\hrule\bigskip}



\makeindex

%%%%%%%%%%%%%%%%%%%%%%%%%%%%%%%%%%%%%%%%%%%%%%%%%%%%%%%%%%%%%%%%

%\thispagestyle{empty}

%\vspace*{2in}

%\title{The |Iterate| Manual}
%\author{Jonathan Amsterdam}

%\maketitle

%\thispagestyle{empty}

%\begin{center}
%\Large *** DRAFT ***
%\end{center}
%This manual is in beta-test. It will become an AI Memo in a couple of
%months.  I would appreciate any comments or
%corrections.  You can reach me at email address |jba@ai.ai.mit.edu|.

\begin{document}

\def\writtenby{Jonathan Amsterdam}
\def\contractno{This report describes research done at the Artificial
Intelligence Laboratory of the Massachusetts Institute of Technology.
Support for the laboratory's artificial intelligence research is
provided in part by the Advanced Research Projects Agency of the
Department of Defense under Office
of Naval Research contract N00014-85-K-0124.}

\AIMEMO{1236}{\Large The Iterate Manual \\[1in]}
{This is the manual for version 1.4 of \iter, a powerful iteration
macro for Common  
Lisp. \iter\ is similar to {\tt loop} but provides numerous additional
features, is well integrated with Lisp, and is extensible. 
\vfill }


% use ~ for \em
\catcode`\~=13
\newif\ifem
\def~{\ifem
         \/\endgroup\emfalse
      \else 
         \begingroup\em\emtrue 
      \fi}



%\pagebreak
%\setcounter{page}{1}

%\thispagestyle{empty}


\tableofcontents

\pagebreak
\pagestyle{headings}

\section{Introduction}

\begin{sloppypar}
This manual describes
\iter, a powerful iteration facility for Common Lisp.
\iter\ provides abstractions for many common iteration
patterns and allows for the definition of additional patterns.  \iter\ is
a macro that expands into ordinary Lisp at
compile-time, so it is more efficient than higher-order
functions like |map| and |reduce.|  
While it is similar to |loop|, \iter\ offers a more Lisp-like syntax
and enhanced extensibility.
(For a more complete
comparison of \iter\ with other iteration constructs, see MIT AI Lab Working
Paper No. 324, ~Don't Loop, Iterate.~)
\end{sloppypar}

An \iter\ form consists of the symbol |iter|\footnote{You can also use
|iterate|, but |iter| is preferred because it avoids potential conflicts with
possible future additions to Common Lisp, and because it saves
horizontal space when writing code.}
followed by one or more
forms, some of which may be \iter\ 
~clauses.~  Here is a simple example of \iter\ which collects the
numbers from 1 to 10 into a list, and returns the list.  The return
value is shown following the arrow.
\begin{program}
(iter (for i from 1 to 10)
      (collect i))          \yields (1 2 3 4 5 6 7 8 9 10)
\end{program}
This form contains two clauses: a |for| clause that steps the
variable |i| over the integers from 1 to 10, and a |collect|
clause that accumulates its argument into a list.  With a few
exceptions, all \iter\ clauses have the same format: alternating
symbols (called ~keywords~) and expressions (called
~arguments~).  The syntax and terminology are those of Common
Lisp's keyword lambda lists.  One difference is that \iter's keywords
do not have to begin with a colon---though they may, except for the
first symbol of a clause.  So you can also write  |(for i :from
1 :to 10)| if you prefer.

Any Lisp form can appear in the body of an \iter, where it will have
its usual meaning.  \iter\ walks the entire body, expanding macros,
and recognizing clauses at any level.  This example collects all the
odd numbers in a list:
\begin{program}
(iter (for el in list)
      (if (and (numberp el) (oddp el))
          (collect el)))
\end{program}

There are clauses for iterating over numbers, lists, arrays and other
objects, and for
collecting, summing, counting, maximizing and other useful operations.
\iter\ also supports the creation of new variable bindings, stepping
over multiple sequences at once, destructuring, and compiler
declarations of variable types.  The following example illustrates
some of these features: 
\begin{program}
(iter (for (key . item) in alist)
         (for i from 0)
         (declare (fixnum i))
         (collect (cons i key)))
\end{program}

This loop takes the keys of an alist and returns a new alist
associating the keys with their positions in the original list. The
compiler declaration for |i| will appear in the generated code
in the appropriate place.

%%%%%%%%%%%%%%%%%%%%%%%%%%%%%%%%%%%%%%%%%%%%%%%%%%%%%%%%%%%%%%%%
\section{Clauses}

Most of \iter's clauses will be familiar to |loop| programmers.  
(|loop| is an iteration macro that has been incorporated into Common
Lisp.  Seen Guy Steele's {\em Common Lisp, 2nd Edition}.)
In
nearly all cases they behave the same as their |loop| counterparts, so
a |loop| user can switch to \iter\ with little pain (and much gain).

All clauses with the standard keyword-argument syntax consist of two
parts: a ~required~ part, 
containing keywords that must be present and in the right order; and
an ~optional~ part, containing keywords that may be omitted and,
if present, may occur in any order.  In the descriptions below, the
parts are separated by the Lisp lambda-list keyword \opt.


\subsection{Drivers}

An iteration-driving clause
conceptually causes the iteration to go forward.  Driver clauses in
\iter\ allow iteration over numbers, lists, vectors, hashtables, packages,
files and streams.  Iteration-driving clauses must
appear at the top level of an \iter\ form; they cannot be nested
inside another clause.  The driver variable is updated at the point
where the driver clause occurs.  Before the clause is executed for the
first time, the value of the variable is undefined.

%Also, regardless of where the driver clause
%appears in the body, the driver variable is stepped at the top of the
%loop; hence it is stylistically preferable, though
%not required, to place driver clauses at the beginning of the \iter.

Multiple drivers may appear in a single \iter\ form, in which case all
of the driver variables are updated each time through the loop, in the
order in which the clauses appear.  The first driver to terminate will
terminate the entire loop.

In all cases, the value of the driver variable on exit from the loop,
including within the epilogue code (see the |finally| clause), is
undefined.

All the parameters of a driver clause are evaluated once, before the
loop begins.  Hence it is not possible to change the bounds or other
properties of an iteration by side-effect from within the loop.

With one exception, driver clauses begin with the word |for| (or
the synonym |as|) and mention an iteration variable, which is
given a binding within the \iter\ form.
The exception is |repeat|, which just executes a loop a
specified number of times:

\begin{clauses}

\clausex{repeat}{~n~}
Repeats the loop ~n~ times.  For example:
\begin{program}
(iter (repeat 100)
      (print "I will not talk in class."))
\end{program}
\cpar If $n \leq 0$, the
loop will never be executed.  If ~n~ is not an integer, the
actual number of executions will be $\lceil n \rceil$.

\end{clauses}

\pagebreak[3]

\subsubsection{Numerical Iteration}

\begin{clauses}
\clausex{for}{~var~ |\&sequence|} 
The general form for iterating over a sequence of numbers requires a
variable and, optionally, one or more keywords that provide the bounds
and step size of the iteration. The |\&sequence|
lambda-list keyword is a shorthand for these sequence keywords.  They are:
|from|, |upfrom|, |downfrom|, |to|, |downto|, |above|, |below| and |by|.
|from| provides the starting value for ~var~
and defaults to zero. |to| provides a final value and implies that the
successive values of ~var~ will be increasing; |downto| implies that
they will be decreasing.  The loop terminates when ~var~ passes
the final value (i.e. becomes smaller or larger than it, depending on
the direction of iteration); in other  words, the loop body
will never be 
executed for values of ~var~ past the final value.
|below| and |above| are similar to |to| and
|downto|, except that the loop terminates when ~var~ equals or passes
the final value.

\cpar If no final value is specified, the variable will be stepped
forever.  Using |from| or |upfrom| will result in increasing
values, while |downfrom| will give decreasing values.

\cpar On each iteration,
~var~ is incremented or decremented by the value of the sequence
keyword |by|, which defaults to 1.  It should always be a positive
number, even for downward iterations.

\cpar In the following examples, the sequence of numbers generated is shown
next to the clause.

\begin{program}
(for i upfrom 0) \yields\ 0 1 2 \ldots
(for i from 5)   \yields\ 5 6 7 \ldots    ; either from or upfrom is okay
(for i downfrom 0) \yields\ 0 -1 -2 \ldots
(for i from 1 to 3) \yields\ 1 2 3
(for i from 1 below 3) \yields\ 1 2
(for i from 1 to 3 by 2) \yields\ 1 3
(for i from 1 below 3 by 2) \yields\ 1
(for i from 5 downto 3) \yields\ 5 4 3
\end{program}

\end{clauses}

\subsubsection{Sequence Iteration}

There are a number of clauses for iterating over sequences.  In all of
them, the argument following |for| may be a list instead of a
symbol, in which case destructuring is performed.  See section
\ref{destructuring}. 

\begin{clauses}

\clause{|for| ~var~ |in| ~list~ \opt\ |by| ~step-function~}
\clindexx{for}{in}
~var~ is set to successive elements of list.  
~step-function,~ which 
defaults to |cdr|, is used to obtain the next sublist. 

\clause{|for| ~var~ |on| ~list~ \opt\ |by| ~step-function~}
\clindexx{for}{on}
~var~ is set to successive sublists of list. 
~step-function~ (default |cdr|) is used as in |for\dots in|.

\end{clauses}

\medskip

These two clauses use |atom| to test for the end
of a list.  Hence, given a list whose final |cdr| is not \nil,
they will silently ignore the last |cdr|.  Other choices are
|endp|,  which would signal an error, and |null|, which
would probably result in an error somewhere else.  If you wish to use
an end-test other than |atom|, set the variable
|iterate::*list-end-test*|\lispindex{*list-end-test*} to the name of
the desired function. 

\begin{clauses}
\clause{|for| ~var~ |in-vector| ~vector~ |\&sequence|}
\clindexx{for}{in-vector}
~var~ takes on successive elements from ~vector.~   The vector's
fill-pointer is observed.  Here and in subsequent clauses, the |\&sequence|
keywords include |with-index|, which takes a symbol as argument
and uses it for the index variable instead of an internally generated
symbol.  The other |\&sequence| keywords behave as in numerical
iteration, except that the default iteration bounds are the bounds of
the vector.  E.g. in |(for i in-vector v downto 3)|,
|i| will start off being bound to the last element in |v|,
and will  
be set to preceding elements down to and including the element with
index 3.

\clause{|for| ~var~ |in-sequence| ~seq~ |\&sequence|}
\clindexx{for}{in-sequence}
This uses Common Lisp's generalized sequence functions, |elt|
and |length|, to obtain elements and determine the length of
~seq.~   Hence it will work for any sequence, including lists, and will
observe the fill-pointers of vectors.

\clause{|for| ~var~ |in-string| ~string~ |\&sequence|}
\clindexx{for}{in-string}
~var~ is set to successive characters of ~string.~

\startitem
\Clause{|for| ~var~ |index-of-vector| ~vector~ |\&sequence|}
\clindexx{for}{index-of-vector}
\Clause{|for| ~var~ |index-of-sequence| ~sequence~ |\&sequence|}
\clindexx{for}{index-of-sequence}
\Clause{|for| ~var~ |index-of-string| ~string~ |\&sequence|}
\clindexx{for}{index-of-string}
\finishitem
~var~ is set to successive indices of the sequence.
These clauses avoid the overhead of accessing the sequence elements
for those applications where they do not need to be examined, or are
examined rarely.  They admit all the optional keywords of the other
sequence drivers except the (redundant) |with-index| keyword.

\clause{|for| ~(key value)~ |in-hashtable| ~table~}
\clindexx{for}{in-hashtable}

~key~ and ~value,~ which must appear as shown in a list
and may be destructuring templates, are set to the keys and values of
~table~.  If ~key~ is |nil|, then the hashtable's keys will be ignored;
similarly for ~value~.
The order in which 
elements of ~table~ will be retrieved is unpredictable.

\clause{|for| ~var~ |in-package| ~package~ \opt\ |external-only| ~ext~}
\clindexx{for}{in-package}
Iterates over all the symbols in ~package,~ or over only the
external symbols if ~ext~ is specified and non-|nil|.  ~ext~ is not
evaluated.  The same symbol may appear more than once.

\clause{|for| ~var~ |in-packages| ~packages~ \opt\ |having-access| ~symbol-types~}
\clindexx{for}{in-packages}
Iterates over all the symbols from the list of packages denoted by the
descriptor ~packages~ and having accessibility (or visibility) given by
~symbol-types~.  This defaults to the list |(:external :internal :inherited)|
and is not evaluated.  ~var~ must be a list of up to three variables: in each
iteration, these will be set to a symbol, its access-type and package (as per
|with-package-iterator| in ANSI-CL).  The same symbol may appear more than
once.

\end{clauses}

\begin{clauses}

\clause{|for| ~var~ |in-file| ~name~ \opt\ |using| ~reader~}
\clindexx{for}{in-file}
Opens the file ~name~ (which may be a string or pathname) for input,
and iterates 
over its contents. ~reader~ defaults to |read|, so by default ~var~
will be bound to the successive forms in the file.  The \iter\ body is
wrapped in an unwind-protect to ensure that the file is closed no
matter how the \iter\ is exited.

\clause{|for| ~var~ |in-stream| ~stream~ \opt\ |using| ~reader~}
\clindexx{for}{in-stream}
Like |for\dots in-file|, except that ~stream~ should be an existing
stream object that supports input operations.

\end{clauses}

\subsubsection{Generalized Drivers}

These are primarily useful for writing drivers that can also be used
as generators (see section \ref{generators}, below).
 
\begin{clauses}

\clause{|for| ~var~ |next| ~expr~}
\clindexx{for}{next}

~var~ is set to ~expr~ each time through the loop.  Destructuring is
performed.  When the clause is used as a generator, ~expr~ is the code
that is executed when |(next ~var~)| is 
encountered (see section \ref{generators}, below).
~expr~ should compute the first value for ~var~, as well
as all subsequent values, and is responsible for terminating the loop.
For compatibility with future versions of \iter, this termination
should be done with |terminate|\clindex{terminate}, which can be
considered a synonym for |finish| (see section \ref{control-flow}).

\cpar As an example, the following clauses are equivalent to |(for
i from 1 to 10)|:
\begin{program}
(initially (setq i 0))
(for i next (if (> i 10) (terminate) (incf i)))
\end{program}

\clause{|for| ~var~ |do-next| ~form~}
\clindexx{for}{do-next}
~form~ is evaluated each time through the loop.  Its value is ~not~
set to ~var~; that is ~form's~ job. ~var~ is only present so that
\iter\ knows it is a driver variable.  \linebreak |(for ~var~ next
~expr~)| is equivalent to |(for ~var~ do-next (dsetq ~var~ ~expr~))|.
(See section \ref{destructuring} for an explanation of |dsetq|.) 

\end{clauses}

\subsubsection{Generators}
\label{generators}

In all of the above clauses, the driver variable is updated on each
iteration.
Sometimes it is desirable to have greater control over updating.  
For instance, consider the problem of associating numbers, in
increasing order and with no gaps, with the
\nonnil\ elements of a list.  One obvious first pass at writing this is:

\begin{program}
(iter (for el in list)
      (for i upfrom 1)
      (if el (collect (cons el i))))
\end{program}
But on the list |(a b nil c)| this produces |((a . 1) (b . 2) (c .
4))| instead of the desired |((a . 1) (b . 2) (c . 3))|.  The problem
is that |i| is incremented each time through the loop, even when |el|
is |nil|.  

The problem could be solved elegantly if we could step |i| only when
we wished to.  This
can be accomplished for any \iter\ driver by writing |generate| (or
its synonym |generating|)
instead of |for|.  Doing so produces a ~generator~---a driver whose
values are yielded explicitly.  To obtain the next value of a
generator variable ~v~, write \linebreak |(next ~v~)|.  The value of a |next|
form is the next value of ~v~, as determined by its associated driver
clause.  |next| also has the side-effect of updating ~v~ to that
value.  If there is no next value, |next| will terminate the loop,
just as with a normal driver.


Using generators, we can now write our example like this:
\begin{program}
(iter (for el in list)
      (generate i upfrom 1)
      (if el (collect (cons el (next i)))))
\end{program}
Now |i| is updated only when |(next i)| is executed, and this occurs
only when |el| is \nonnil.

To better understand the relationship between ordinary drivers and
generators, observe that we can rewrite an ordinary driver using its
generator form immediately followed by |next|, as this example shows:
\begin{program}
(iter (generating i from 1 to 10)
      (next i)
      ...)
\end{program}
Provided that the loop body contains no |(next i)| forms, this will
behave just as if we had written |(for i from 1 to 10)|.

We can still refer to a driver variable ~v~ without using |next|; in
this case, its value is that given to it by the last evaluation of
|(next ~v~)|.  Before |(next ~v~)| has been called the first time, the
value of ~v~ is undefined.  

This semantics is more flexible than
one in which ~v~ begins the loop bound to its first value and calls
of |next| supply subsequent values, because it means the loop will not
terminate too soon if the generator's sequence is empty.  For
instance, consider the following code, which tags \nonnil\ elements of
a list using a list of tags, and also counts the null elements.
(We assume there are at least as many tags as \nonnil\ elements.)
\begin{program}
(let* ((counter 0)
       (tagged-list (iter (for el in list)
                          (generating tag in tag-list) 
                          (if (null el)
                              (incf counter)
                              (collect (cons el (next tag)))))))
  ...)
\end{program}
It may be that there are just as many tags as non-null elements of
|list|.  If all the elements of |list| are null, we still want the
counting to proceed, even though |tag-list| is \nil.  If |tag| had to be
assigned its first value before the loop begins, we would have had to
terminate the loop before the first iteration, since when |tag-list|
is \nil, |tag| 
has no first value.  With the existing semantics, however, |(next
tag)| will never execute, so the iteration will cover all the elements of
|list|. 
        
When the ``variable'' of a driver clause is actually a destructuring
template containing several variables, all the variables are eligible
for use with |next|.  As before, |(next ~v~)| evaluates to ~v's~ next
value; but the effect is to update all of the template's variables.
For instance, the following code will return the list |(a 2 c)|.
\begin{program}
(iter (generating (key . item) in '((a . 1) (b . 2) (c . 3)))
      (collect (next key))
      (collect (next item)))
\end{program}

Only driver clauses with variables can be made into generators.  This
includes all clauses mentioned so far except for |repeat|.  It does
~not~ include |for\dots previous|, |for\dots =|,
|for\dots initially\dots then| or |for\dots first\dots then| (see
below). 

\subsubsection{Previous Values of Driver Variables}

Often one would like to access the value of a variable on a previous
iteration.  \iter\ provides a special clause for accomplishing this.

\begin{clauses}

\clause{|for| ~pvar~ |previous| ~var~ \opt\ |initially| ~init~
|back| ~n~}
\clindexx{for}{previous}
Sets ~pvar~ to the previous value of ~var,~ which should be a driver
variable, a variable from another |for\dots previous| clause, or a
variable established by a |for\dots =|, 
|for\dots initially\dots then| or |for\dots first\dots then| clause
(see section \ref{setting}).
Initially, ~pvar~ is given the value ~init~ (which defaults to \nil).  
The ~init~ expression will be moved outside the loop body, so it
should not depend on anything computed within the loop.
~pvar~ retains the value of ~init~ until ~var~ is set to its second
value, at which point ~pvar~ is set to ~var's~ first value; and so on.  

\cpar The
argument ~n~ to |back| must be a constant, positive integer, and
defaults to 1.  It determines how many iterations back ~pvar~ should
track ~var~.  For example, when ~n~ is 2, then ~pvar~ will be assigned
~var's~ first value when ~var~ is set to its third value.

\cpar A |for\dots previous| clause may occur after or before its
associated driver clause. |for\dots previous| works with generators as
well as ordinary drivers.

\pagebreak[3]

\cpar Example:
\begin{program}
(iter (for el in '(1 2 3 4))
      (for p-el previous el)
      (for pp-el previous p-el initially 0)
      (collect pp-el))
\end{program}
This evaluates to |(0 0 1 2)|.  It could have been written more
economically as 
\begin{program}
(iter (for el in '(1 2 3 4))
      (for pp-el previous el back 2 initially 0)
      (collect pp-el))
\end{program}


\end{clauses}

\subsection{Variable Binding and Setting}
\label{setting}

Several clauses exist for establishing new variable bindings or for
setting variables in the loop.  They all support destructuring.

\begin{clauses}

\clausex{with}{~var~ \opt\ |=| ~value~}
Causes ~var~ to be bound to value before the loop body is entered.
If ~value~ is not supplied, ~var~ assumes a default
binding, which will be 
\nil\ in the absence of declarations.  Also, if ~value~ is not
supplied, no destructuring is performed; instead, ~var~ may be a list of
symbols, all of which are given default 
bindings.  If ~value~ is supplied, ~var~ is bound to it, with
destructuring.  

\cpar Because |with|
creates bindings whose scope includes the entire \iter\ form, it is
good style to put all |with| clauses at the beginning.

\cpar Successive occurrences of |with| result in sequential
bindings (as with 
|let*|).  There is no way to obtain parallel bindings; see
section \ref{bindings} for a rationale.  


\clause{|for| ~var~ |=| ~expr~}
\clindexx{for}{=}
On each iteration, ~expr~ is evaluated and ~var~ is set
to its value. 

\cpar This clause may appear to do the same thing as |for\dots next|.
In fact, they are quite different.  |for\dots =| provides only three
services: it sets up a binding for ~var~, sets it to ~expr~ on each
iteration, and makes it possible to use |for\dots previous| with
~var~.  |for\dots next| provides these services in addition to the
ability to turn the driver into a generator.
%Also, the code which sets ~var~ appears in the loop body in the same
%place as the |for\dots =| clause; the code for |for\dots next| appears
%at the top of the loop, as with other drivers (except when being used
%as a generator).

\clause{|for| ~var~ |initially| ~init-expr~ |then| ~then-expr~}
\clindexxx{for}{initially}{then}
Before the loop begins, ~var~ is set to ~init-expr;~ on all
iterations after the first it is set to ~then-expr.~  This clause
must occur at top-level.  ~init-expr~ will be moved outside the loop
body and ~then-expr~ will be moved to the end of the loop body, so
they are subject to code motion problems (see section
\ref{code-movement}). 

\cpar This clause may appear to be similar to |for\dots next|, but in
fact they differ significantly.  |for\dots initially\dots then| is
typically used to give ~var~ its first value before the loop begins,
and subsequent values on following iterations.  This is incompatible
with generators, whose first value and subsequent values must all be
computed by |(next ~var~)|.  Also, the update of ~var~ in 
|for\dots initially\dots then| does not occur at the location of the
clause. 
Use |for\dots initially\dots then| for
one-shot computations where its idiom is more convenient, but use
|for\dots next| for extending \iter\ with new drivers (see section
\ref{extend}).

\clause{|for| ~var~ |first| ~first-expr~ |then| ~then-expr~}
\clindexxx{for}{first}{then}
The first time through the loop, ~var~ is set to ~first-expr;~ on
subsequent iterations, it is set to ~then-expr.~  This differs from
|for\dots initially| in that ~var~ is set to ~first-expr~
inside the loop body, 
so ~first-expr~ may depend on the results of other clauses.  For
instance, 
\begin{program}
(iter (for num in list)
      (for i first num then (1+ i))
      ...)
\end{program}
will set |i| to the first element of |list| on the first
iteration, whereas 
\begin{program}
(iter (for num in list)
      (for i initially num then (1+ i))
      ...)
\end{program}
is probably erroneous; |i| will be bound to |num|'s
default binding (usually \nil) for the first iteration.

\end{clauses}

\begin{note}{Compatibility note}
|loop|'s |for\dots =| works like \iter's, but |loop| used the syntax
|for\dots =\dots then| to mean |for\dots initially\dots then|.  It was
felt that these two operations were sufficiently different to warrant
different keywords.

Also, the |for| in the above three clauses is misleading,
since none is true driver (e.g. none has a corresponding |generate|
form).  |setting| would have been a better choice, but |for| was used
to retain some compatibility with |loop|.  
\end{note}


%%%%%%%%%%%%%%%%%%%%%%%%%%%%%%%%
\subsection{Gathering Clauses}
Many of \iter's clauses accumulate values into a variable, or set a
variable under certain conditions.  At the end of the loop, this
variable contains the desired result.  All these clauses have an
optional |into| keyword, whose argument should be a symbol.  If the
|into| keyword is not supplied, the accumulation variable will be
internally generated and its value will be returned at the end of the
loop; if a variable is specified, that variable is used for the
accumulation, and is not returned as a result---it is up to the user
to return it explicitly, in the loop's epilogue code (see |finally|).
It is safe to examine the accumulation variable during the loop, but
it should not be modified.

These clauses all begin with a verb.  When the verb does
not conflict with an existing Common Lisp function, then it may be
used in either its infinitival or present-participle form (e.g. |sum|,
|summing|).  However, when there is a conflict with Common Lisp, only
the present-participle form may be used (e.g. |unioning|).  This is to
prevent \iter\ clauses from clashing with Common Lisp
functions. 

%although these clauses are described as ``producing a value,'' it is a
%mistake to think of the lisp list representing the clause as a
%value-producing form in the usual way.  clauses may legally be written
%where a value is expected, e.g. |(setq x (sum i))|, but the
%lisp value of a clause in such a context is undefined.

\subsubsection{Reductions}

~Reduction~ is an extremely common iteration pattern in which
the results of successive applications of a
binary operation are accumulated.
For example, a loop that computes the sum of the
elements of a list is performing a reduction with the addition
operation.  This could be written in Common Lisp as  |(reduce \#'+
list)| or with \iter\ as
\begin{program}
(iter (for el in list)
      (sum el))
\end{program}

\begin{clauses}

\clausex{sum}{~expr~ \opt\ |into| ~var~}
Each time through the loop, ~expr~ is evaluated and added to a
variable, which is bound initially to zero.  If ~expr~ has a type,
it is ~not~ used as the type of the sum variable, which is always
|number|.  To get the result variable to be of a more specific
type, use an explicit variable, as in
\begin{program}
(iter (for el in number-list)
      (sum el into x)
      (declare (fixnum x))
      (finally (return x)))
\end{program}

\clausex{multiply}{~expr~ \opt\ |into| ~var~}
Like |sum|, but the initial value of the result variable is $1$,
and the variable is updated by multiplying ~expr~ into it.

\clausex{counting}{~expr~ \opt\ |into| ~var~}
~expr~ is evaluated on each iteration.  If it is \nonnil, the
accumulation variable, initially zero, is incremented.

\startitem
\Clausex{maximize}{~expr~ \opt\ |into| ~var~}
\Clausex{minimize}{~expr~ \opt\ |into| ~var~}
\finishitem
~expr~ is evaluated on each iteration and its extremum (maximum or
minimum) is stored in the accumulation variable.  If ~expr~ is never
evaluated, then the result is \nil\ (if the accumulation variable is
untyped) or $0$ (if it has a numeric type).

\clausex{reducing}{~expr~ |by| ~func~ \opt\ |initial-value| ~init-val~ |into| ~var~}
This is a general way to perform reductions. ~func~ should be a
function of two arguments, the first of which will be the value
computed so far and
the second of which will be the value of ~expr~.  It should return the
new value.
|reducing| is roughly equivalent to the
Common Lisp |(reduce ~func~ ~sequence~ :key ~expr-function~)|, where
~expr-function~ is used to derive values from the successive elements
of ~sequence~.

\cpar If the |reducing| clause is never executed, the result is
undefined.

\cpar It is not necessary to provide an
initial value, but better code can be generated if one is supplied.
Regardless of its location in the \iter\ body, ~init-val~ will be
evaluated before the loop is entered, so it should not depend on any
value computed inside the \iter\ form.  

%\cpar if a ~var~ is not specified, you can get \iter\ to declare the
%type of the internal variable by putting a |the| expression around
%~func.~  see section \ref{types}.

\end{clauses}

%\begin{note}{implementation note}
%in principle, |maximize| and |minimize| can be thought of
%as reductions where the initial value is the smallest (or largest)
%value that the accumulation variable can assume.  because lisp's
%bignums can represent arbitrary integers, these clauses cannot be
%implemented as reductions in general.  if, however, the type of 
%~expr~ or ~var~ can be determined to be a fixnum
%or a float, \iter\ will implement the clause as a true reduction,
%using one of the constants |most-negative-fixnum|, |%most-positive-fixnum|, 
%|most-negative-short-float|, etc. as appropriate.
%\end{note}

\subsubsection{Accumulations}

All the predefined accumulation clauses add values to a sequence.  If
the sequence is a list, they all 
have the property that the partial list is kept in the correct order
and available for inspection at any point in the loop.

\begin{clauses}

\clausex{collect}{~expr~ \opt\ |into| ~var~ |at| ~place~ |result-type| ~type~}
Produces a sequence of the values of ~expr~ on each iteration. ~place~
indicates where the next value of ~expr~ is added to the list and may
be one of the symbols |start|, |beginning| (a synonym for |start|) or
|end|.  The symbol may be quoted, but need not be.  The default is
|end|.  For example,
\begin{program}
(iter (for i from 1 to 5)
      (collect i))
\end{program}
produces |(1 2 3 4 5)|, whereas
\begin{program}
(iter (for i from 1 to 5)
      (collect i at beginning))
\end{program}
produces |(5 4 3 2 1)| (and is likely to be faster in most Common Lisp
implementations).

\cpar If ~type~ is provided, it should be a subtype of |sequence|.
The default is |list|.  Specifying a type other than |list| will
result in |collect| returning a sequence of that type.  ~However,~ the
type of the sequence being constructed when inside the loop body is
undefined when a non-|list| type is specified.  (As with ~place~,
quoting ~type~ is optional.)

\clausex{adjoining}{~expr~ \opt\ |into| ~var~ |test| ~test~ |at| ~place~ |result-type| ~type~}
Like |collect|, but only adds the value of ~expr~ if it is not
already present.  ~test,~ which defaults to |\#'eql|, is
the test to be used with |member|.

\startitem
\Clausex{appending}{~expr~ \opt\ |into| ~var~ |at| ~place~}
\Clausex{nconcing}{\ ~expr~ \opt\ |into| ~var~ |at| ~place~}
\Clausex{unioning}{\ ~expr~ \opt\ |into| ~var~ |test| ~test~ |at| ~place~}
\Clausex{nunioning}{~expr~ \opt\ |into| ~var~ |test| ~test~ |at| ~place~}
\finishitem
These are like |collect|, but behave like the Common Lisp functions
|append|, |nconc|, |union| or |nunion|.  
As in Common Lisp, they work only on lists.  Also as in Common Lisp,
|unioning| and |nunioning| assume that the value of ~expr~ contains no
duplicates. 

\clausex{accumulate}{~expr~ |by| ~func~ \opt\ |initial-value| ~init-val~ |into| ~var~}
This is a general-purpose accumulation clause. ~func~ should be a
function of two arguments, the value of ~expr~ and the value
accumulated so far in the iteration, and it should return the updated
value.  If no initial value is supplied, \nil\ is used.  

%\cpar If a ~var~ is not specified, you can get \iter\ to declare the
%type of the internal variable by putting a |the| expression around
%~func.~  see section \ref{types}.

\cpar The differences between |accumulate| and |reducing| are slight.
One difference is that the functions take their arguments in a
different order.  Another is that in the absence of ~init-val~,
|accumulate| will use \nil, whereas |reducing| will generate different
code that avoids any dependence on the initial value.
The reason for having both clauses is that one usually
thinks of reductions (like |sum|) and accumulations (like |collect|)
as different beasts.

\end{clauses}


\subsubsection{Finders}

A ~finder~ is a clause whose value is an expression that meets some
condition.

\begin{clauses}

\clause{|finding| ~expr~ |such-that| ~test~ \opt\ |into| ~var~ |on-failure| ~failure-value~}
\clindexx{finding}{such-that}
If ~test~ (which is an expression) ever evaluates to \nonnil, the loop
is terminated, the 
epilogue code is run and the value of ~expr~ is returned.  Otherwise,
\nil\ (or ~failure-value,~ if provided) is returned.  If ~var~ is
provided, it will have either the \nonnil\ value of ~expr~ or
~failure-value~ when the epilogue code is run.

\cpar As a special case, if the ~test~ expression is a sharp-quoted
function, then it is applied to ~expr~ instead of being simply
evaluated.  E.g. |(finding x such-that \#'evenp)| is equivalent to
|(finding x such-that (evenp x))|.  

%\cpar although ~test~ need have nothing to do with ~%expr~ as in
%|(finding j such-that (> i 3))|, it usually 
%will: |(finding (length el) such-that (oddp (length el)))|.  to
%avoid performing the |length| computation twice, you could write
%|(finding (length el) such-that \#'oddp)| or |(finding (length
%el) such-that 'oddp)|; for these cases, \iter\ generates code that
%executes ~expr~ only once.  the code for |\#'oddp|
%is slightly different from that for {\lisp 'oddp}; see the discussion
%under {\lisp for\dots in} and {\lisp for\dots on}.

\cpar |On-failure| is a misnomer. Because it is always evaluated, it behaves
more like the default third argument to the |gethash| function. As a result,
|on-failure (error "Not found")| makes no sense. Instead, the clauses |leave|
or |thereis| can be used in conjunction with |finally| as follows:
\begin{program}
(iter (for x in '(1 2 3))
      (if (evenp x) (leave x))
      (finally (error "not found")))
\end{program}

\cpar This clause may appear multiple times when all defaults are
identical. It can also be used together with either |always|/|never| or
|thereis| if their defaults match. More specifically, |on-failure nil| is
compatible with |thereis|, while |on-failure t| is compatible with |always|
and |never| clauses.
\begin{program}
(iter (for i in '(7 -4 2 -3))
      (if (plusp i)
	  (finding    i  such-that (evenp i))
          (finding (- i) such-that (oddp i))))
\end{program}

\startitem
\Clause{|finding| ~expr~ |maximizing| ~m-expr~ \opt\ |into| ~var~}
\clindexx{finding}{maximizing}
\Clause{|finding| ~expr~ |minimizing| ~m-expr~ \opt\ |into| ~var~}
\clindexx{finding}{minimizing}
\finishitem
Computes the extremum (maximum or minimum) value of ~m-expr~ over all
iterations, and returns the value of ~expr~ corresponding to the
extremum.  ~expr~ is evaluated inside the loop at the time the new
extremum is established.  If ~m-expr~ is never evaluated (due to, for
example, being embedded in a conditional clause), then the returned
value depends on the type, if any, of ~expr~ (or ~var,~ if
one is supplied).  If there is no type, the returned
value will be \nil; if the type is numeric, the returned value will be
zero.

\cpar For these two clauses, ~var~ may be a list of two
symbols; in that case, the first is used to record ~expr~ and
the second, ~m-expr.~  

\cpar As with |finding\dots such-that|, if ~m-expr~ is a sharp-quoted
function, then it is called on ~expr~ instead of being evaluated.

\end{clauses}

\subsubsection{Boolean Tests}

\begin{clauses}

\clausex{first-iteration-p}{}
Returns |t| in the first cycle of the loop, otherwise \nil.

\clausex{first-time-p}{}
Returns |t| the first time the expression is evaluated, and then \nil\ forever.
This clause comes handy when printing (optional) elements separated
by a comma:

\begin{program}
(iter (for el in '(nil 1 2 nil 3))
      (when el
        (unless (first-time-p)
          (princ ", "))
        (princ el)))
\end{program}
produces |"1, 2, 3"|.

\end{clauses}

\subsubsection{Aggregated Boolean Tests}

\begin{clauses}

\clausex{always}{~expr~}
If ~expr~ ever evaluates to
\nil, then \nil\ is immediately returned; the epilogue code is not
executed.  If ~expr~ never evaluates to \nil, the epilogue code
is executed and the last value of ~expr~ (or |t| if ~expr~ was never
evaluated) is returned (whereas |loop| would constantly return |t|).

% mention last evaluated clause when multiple always clauses?

\clausex{never}{~expr~}
Like |(always (not ~expr~))|, except it does not influence the last
value returned by a possible other |always| clause. That is,
\begin{program}
(iter (repeat 2)
      (always 2)
      (never nil)) \yields 2 ; not t
\end{program}

\clausex{thereis}{~expr~}
If ~expr~ is ever \nonnil,
its value is immediately returned without running epilogue code.
Otherwise, the epilogue code is performed and \nil\ is returned.

This clause cannot be used together with |always| or |never|, because their
defaults are opposed (similarly, |(loop always 3 thereis nil)| refuses to
compile in some implementations of |loop|).

\end{clauses}


\subsection{Control Flow}
\label{control-flow}
Several clauses can be used to alter the usual flow of control in a loop.

Note: the clauses of this and subsequent sections don't adhere to \iter's
usual syntax, but instead use standard Common Lisp syntax.  Hence the
format for describing syntax subsequently is
like the standard format used in the Common Lisp manual, not like the
descriptions of clauses above.

\begin{clauses}

\clausex{finish}{}
Stops the loop and runs the epilogue code.

%for example:
%
%\begin{program}
%(iter (with answer = nil)
%         (initially (make-a-mess))
%         (for i from 1 to 10)
%         (when (correct? i)
%           (setq answer i)
%           (finish))
%         (finally (cleanup)))
%\end{program}
%
%this code will execute |cleanup| whether or not the test |(correct?
%i)| ever succeeds. 
%the (more elegant) formulation,
%\begin{program}
%(iter (initially (make-a-mess))
%         (for i from 1 to 10)
%         (finding i such-that (correct? i))
%         (finally (cleanup)))
%\end{program}
%would not execute |cleanup| if |(correct? i)| succeeded; it
%would do an immediate return.

\clausex{leave}{\opt\ ~value~}
Immediately returns ~value~ (default \nil) from the current \iter\
form, skipping the epilogue code.  Equivalent to using |return-from|.

\clausex{next-iteration}{}
Skips the remainder of the loop body and begins the next iteration of
the loop.

\clausex{while}{~expr~}
If ~expr~ ever evaluates to \nil, the loop is terminated and the
epilogue code executed.  Equivalent to |(if (not ~expr~) (finish))|.

\clausex{until}{~expr~}
Equivalent to |(if ~expr~ (finish))|.

\clausex{if-first-time}{~then~ \opt\ ~else~}
If this clause is being executed for the first time in this invocation
of the \iter\ form, then the ~then~ code is evaluated; otherwise the
~else~ code is evaluated.

\cpar |(for ~var~ first ~expr1~ then ~expr2~)| is almost equivalent to
\begin{program}
(if-first-time (dsetq ~var~ ~expr1~)
               (dsetq ~var~ ~expr2~))
\end{program}
The only difference is that the |for| version makes ~var~ available
for use with |for\dots previous|.

\end{clauses}

\subsection{Code Placement}
When fine control is desired over where code appears in a loop
generated by \iter, the following special clauses may be useful.
They are all subject to code-motion problems (see section
\ref{code-movement}).  

\begin{clauses}

\clausex{initially}{|\&rest| ~forms~}
The lisp ~forms~ are placed in the prologue section of the loop, where
they are executed once, before the loop body is entered.

\clausex{after-each}{|\&rest| ~forms~}
The ~forms~ are placed at the end of the loop body, where they
are executed after each iteration.  Unlike the other clauses in this
section, ~forms~ may contain \iter\ clauses.

\clausex{else}{|\&rest| ~forms~}
The lisp ~forms~ are placed in the epilogue section of the loop, where they
are executed if this |else| clause is never met during execution of the
loop and the loop terminates normally.

\clausex{finally}{|\&rest| ~forms~}
The lisp ~forms~ are placed in the epilogue section of the loop, where
they are executed after the loop has terminated normally.

\clausex{finally-protected}{|\&rest| ~forms~}
The lisp ~forms~ are placed in the second form of an unwind-protect
outside the loop.  They are always executed after the loop has
terminated, regardless of how the termination occurred.

\end{clauses}


%%%%%%%%%%%%%%%%%%%%%%%%%%%%%%%%%%%%%%%%%%%%%%%%%%%%%%%%%%%%%%%%
\section{Other Features}

\subsection{Multiple Accumulations}
\label{multiple}

\begin{sloppypar}
It is permitted to have more than one clause accumulate into the same
variable, as in the following:
\begin{program}
(iter (for i from 1 to 10)
      (collect i into nums)
      (collect (sqrt i) into nums)
      (finally (return nums)))
\end{program}
Clauses can only accumulate into the same variable if they are
compatible.  |collect|, |adjoining|, |appending|, |nconcing|,
|unioning| and |nunioning| are compatible with each other; |sum|,
|multiply| and |counting| are compatible; |always| and |never| are compatible;
|finding| \dots |such-that| is compatible with either |thereis| or |always|
and |never| when their defaults match; and |maximize| and |minimize| clauses
are compatible only with other |maximize| and |minimize| clauses,
respectively.

%note that the same variable ~cannot~ be both an accumulation
%variable and an ordinary variable; there can be only one variable with
%a given name within an \iter\ form. 
\end{sloppypar}

\subsection{Named Blocks}

Like Common Lisp |block|s, \iter\ forms can be given names.  The
name should be a single symbol, and it must be the first form in the
\iter.  The generated code behaves exactly like a named block; in
particular, |(return-from ~name~)| can be used to exit it:
\begin{program}
(iter fred
      (for i from 1 to 10)
      (iter barney
            (for j from i to 10)
            (if (> (* i j) 17)
                (return-from fred j))))
\end{program}
An \iter\ form that is not given a name is implicitly named \nil.

Sometimes one would like to write an expression in an inner \iter\ form,
but have it processed by an outer \iter\ form.  This is possible with
the |in| clause.  

\begin{clauses}

\clausex{in}{~name~ |\&rest| ~forms~}
Evaluates ~forms~ as if they were part of the \iter\ form named
~name~.  In other words, \iter\ clauses are processed by the \iter\
form named ~name,~ and not by any \iter\ forms that occur inside ~name.~

\cpar As an example, consider the problem of collecting a list of the
elements in a two-dimensional array.  The naive solution,
\begin{program}
(iter (for i below (array-dimension ar 0))
      (iter (for j below (array-dimension ar 1))
            (collect (aref ar i j))))
\end{program}
\noindent is wrong because the list created by the inner \iter\ is simply
ignored by the outer one.  But using |in| we can write:
\begin{program}
(iter outer (for i below (array-dimension ar 0))
      (iter (for j below (array-dimension ar 1))
            (in outer (collect (aref ar i j)))))
\end{program}
\noindent which has the desired result.

\end{clauses}

\subsection{Destructuring}
\label{destructuring}

In many places within \iter\ clauses where a variable is expected, a
list can be written instead.  In these cases, the value to be assigned
is ~destructured~ according to the pattern
described by the list.  As a simple example, the clause
\begin{program}
(for (key . item) in alist)
\end{program}
\noindent will result in |key| being set to the |car| of
each element in |alist|, and |item| being set to the |cdr|.  The
pattern list may be nested to arbitrary depth, and (as the example
shows) need not be terminated with \nil; the only requirement is that
each leaf be a bindable symbol (or \nil, in which case no binding is
generated for that piece of the structure).

Sometimes, you might like to do the equivalent of a
|multiple-value-setq| in a clause.  This 
``multiple-value destructuring'' can be expressed by writing 
\linebreak |(values $pat_1$ $pat_2 \ldots$)| for a destructuring
pattern, as in 
\begin{program}
(for (values (a . b) c d) = (three-valued-function ...))
\end{program}
\begin{sloppypar}
Note that the $pat_i$ can themselves be destructuring patterns (though
not multiple-value destructuring patterns).  You can't do multiple-value
destructuring in a |with| clause; instead wrap the whole \iter\
form in a |multiple-value-bind|.
\end{sloppypar}

\begin{note}{Rationale}
There are subtle interactions between variable declarations and
evaluation order that make the correct implementation of
multiple-value destructuring in a |with| somewhat tricky.
\end{note}

The destructuring feature of \iter\ is available as a separate
mechanism, using the |dsetq| macro:

\begin{clauses}
\defmacro{dsetq}{template expr}
Performs destructuring of ~expr~ using ~template~.   May be used
outside of an \iter\ form. Yields the primary value of ~expr~.

\end{clauses}

\subsection{On-line Help}

\begin{sloppypar}
There is a limited facility for on-line help, in the form of the 
|display-iterate-clauses| function.
\end{sloppypar}

\begin{clauses}

\defun{display-iterate-clauses}{\opt\ clause-spec}
Displays a list of \iter\ clauses.  If ~clause-spec~ is not
provided, all clauses are shown; if it is a symbol, all clauses
beginning with that symbol are shown; and if it is a list of symbols,
all clauses for which ~clause-spec~ is a prefix are shown.

\end{clauses}

\subsection{Parallel Binding and Stepping}
\label{bindings}

The parallel binding and stepping of variables is a feature that
\iter\ does ~not~ have.  This section attempts to provide a rationale.

We say that two variables are bound ~in parallel~ if neither
binding shadows the other.  This is the usual semantics of |let|
(as opposed to |let*|).  Similarly, we can say that iteration
variables are stepped in parallel if neither variable is updated
before the other, conceptually speaking; in other words, if the code
to update each variable can reference the old values of both variables.

|loop| allows parallel binding of variables and parallel stepping of
driver variables.  My view is that if you are depending on the 
serial/parallel distinction, you are doing something obscure.  If you
need to bind 
variables in parallel using |with|, then you must be using a
variable name that shadows a name in the existing lexical environment.
Don't do that.  The most common use for parallel stepping is to track
the values of variables on the previous iteration, but in fact this
does not require parallel stepping at all; the following will work:
\begin{program}
(iter (for current in list)
      (for prev previous current)
      ...)
\end{program}


%%%%%%%%%%%%%%%%%%%%%%%%%%%%%%%%%%%%%%%%%%%%%%%%%%%%%%%%%%%%%%%%
\section{Types and Declarations}
\label{types}

\subsection{Discussion}

Sometimes efficiency dictates that the types of variables be declared.
This type information needs to be communicated to \iter\ so it can
bind variables to appropriate values.  Furthermore, \iter\ must often
generate internal variables invisible to the user; there needs to be a
way for these to be declared.

As an example, consider this code, which will return the number of
odd elements in |number-list|:
\begin{program}
(iter (for el in number-list)
      (count (oddp el)))
\end{program}
In processing this form,
\iter\ will create an internal variable, let us call it |list17|, to
hold the successive |cdr|s of |number-list|, and  
will bind the variable to |number-list|.  It will also generate a
default binding for |el|; only inside the body of the loop will |el|
be set to the |car| of |list17|.  Finally, \iter\ will generate a
variable, call it |result|, to hold the result of the count, and will
bind it to zero.

When dealing with type declarations, \iter\ observes one simple rule:
~it will never generate a declaration unless requested to do so.~  The
reason is that such declarations might mask errors in compiled code by
avoiding error-checks; the resulting problems would be doubly hard to
track down because the declarations would be hidden from the
programmer.  Of course, a compiler might omit error-checks even in the
absence of declarations, though this behavior can usually be avoided,
e.g. by saying |(declaim (optimize (safety 3)))|.

So, the above \iter\ form will generate code with no declarations.
But say we wish to declare the types of |el| and the internal
variables |list17| and |result|.  How is this done?

Declaring the type of |el| is easy, since the programmer knows
the variable's name:
\begin{program}
(iter (for el in number-list)
      (declare (fixnum el))
      (counting (oddp el)))
\end{program}
\iter\ can read variable type declarations like this one.  Before
processing any clauses, it scans the entire top-level form for type
declarations 
and records the types, so that variable bindings can be performed
correctly.  In this case, |el| will be bound to zero
instead of \nil.  Also, \iter\ collects all the top-level declarations
and puts them at the begining of the generated code, so it is not
necessary to place all declarations at the beginning of an \iter\
form; instead, they can be written near the variables whose types they
declare.  

Since \iter\ is not part of the compiler, it will not know
about declarations that occur outside an \iter\ form; these
declarations must be repeated inside the form.

Here is another way we could have declared the type of |el|:
\begin{program}
(iter (for (the fixnum el) in number-list)
      (counting (oddp el)))
\end{program}
\iter\ extends the Common Lisp |the|\lispindex{the}
form to apply to variables as well as value-producing forms; anywhere
a variable is allowed---in a |with| clause, as the iteration
variable in a driver clause, as the |into| argument of an
accumulation clause, even inside a destructuring template---you can
write |(the ~type~ ~symbol~)| instead.

There is one crucial difference between using a |the| form and
actually declaring the variable: explicit declarations are always
placed in the generated code, but type information from a |the|
form is not turned into an actual declaration unless you tell \iter\
to do so using |iterate:declare-variables|.  See below.

\begin{sloppypar}
Declaring the types of internal variables is harder than declaring the
types of explicitly mentioned variables, since their names
are unknown.  You do it by declaring |iterate:declare-variables|
somewhere inside the top level of the \iter\ form.  (This will also
generate declarations for variables declared using |the|.)
\iter\ does not provide much selectivity here: it's all or none. 
And unfortunately, since \iter\ is not privy to compiler information
but instead reads declarations itself, it will not hear if you
|(declaim (iterate:declare-variables))|.  Instead, set the variable 
|iterate::*always-declare-variables*| to |t| at
compile-time, using |eval-when|.
\end{sloppypar}

To determine the appropriate types for internal variables, \iter\ uses
three sources of information:

\begin{itemize}

\item Often, the particular clause dictates a certain type for a
variable; \iter\ will use this information when available.  In the
current example, the variable |list17| will be given the type
|list|, since that is the only type that makes sense; and the
variable |result| will be given the type |fixnum|, on the
assumption that you will not be counting high enough to need bignums.
You can override this assumption only by using and explicitly declaring a
variable: 
\begin{verbatim}
(iter (declare (iterate:declare-variables))
      (for el in number-list)
      (count (oddp el) into my-result)
      (declare (integer my-result))
      (finally (return my-result)))
\end{verbatim}

\begin{sloppypar}
Other examples of the type assumptions that \iter\ makes are: type
|list| for |into| variables of collection clauses; type |list| for
expressions that are to be destructured; type |vector| for the
variable holding the vector in a |for\dots in-vector| clause, and
similarly for |string| and the |for\dots in-string| clause;
and the implementation-dependent type
|(type-of array-dimension-limit)| for the index and limit
variables generated by sequence iteration drivers like |for\dots
in-vector| and |for\dots in-string| (but not |for\dots in-sequence|,
because it may be used to iterate over a list). 
\end{sloppypar}

\item Sometimes, \iter\ will examine expressions and try to determine
their types in a simple-minded way.  If the expression is
self-evaluating (like a number, for instance), \iter\ knows that the
expression's type is the same as the type of the value it denotes, so
it can use that type.  If the expression is of the form |(the ~type~
~expr~)|, \iter\ is smart enough to extract ~type~ and use it.
However, the current version of \iter\ does not  
examine declarations of function result types or do any type
inference.  It will not determine, for 
example, that the type of |(+ 3 4)| is |fixnum|, or even
|number|.

\item In some cases, the type of an internal variable should match the
type of some other variable.  For instance, \iter\ generates an
internal variable for |(f x)| in the
clause |(for i from 1 to (f x))|, and in the absence of other
information will give it the same type as |i|.  If, however, the
expression had been written |(the fixnum (f x))|, then \iter\
would have given the internal variable the type |fixnum|
regardless of |i|'s type.  The type incompatibility errors that
could arise in this situation are not checked for.

\end{itemize}

Note that if you do declare |iterate:declare-variables|, then
\iter\ may declare user variables as well as internal ones if they do
not already have declarations, though only for variables that it
binds.  For instance, in this code:

\begin{program}
(iter (declare (iterate:declare-variables))
      (for i from 1 to 10)
      (collect i into var))
\end{program}
the variable |var| will be declared to be of type |list|.


\subsection{Summary}

\iter\ understands standard Common Lisp variable type declarations
that occur within an \iter\ form and
will pass them through to the generated code.  If the declaration
|(iterate:declare-variables)|\lispindex{declare-variables}
appears at the top level of an 
\iter\ form, or if 
|iterate::*always-declare-variables*|\lispindex{*always-declare-variables*} 
is \nonnil, then \iter\ will use the type information gleaned from user
declarations, self-evaluating expressions and |the| expressions,
combined with reasonable assumptions, to determine variable
types and declare them.


%%%%%%%%%%%%%%%%%%%%%%%%%%%%%%%%%%%%%%%%%%%%%%%%%%%%%%%%%%%%%%%%
\section{Problems with Code Movement}
\label{code-movement}

Some \iter\ clauses, or parts of clauses, result in code being
moved from the location of the clause to other parts of the loop.
Drivers behave this way, as do code-placement clauses like |initially|
and |finally|.  When using these clauses, there is a danger of writing
an expression that makes sense in its apparent location but will be
invalid or have a different meaning in another location.  For example:
\begin{program}
(iter (for i from 1 to 10)
      (let ((x 3))
        (initially (setq x 4))))
\end{program}
While it may appear that the |x| of |(initially (setq x 4))| is the
same as the |x| of |(let ((x 3)) \dots|, in fact they are not:
|initially| moves its code outside the loop body, so |x| would refer
to a global variable.  Here is another example of the same problem:
\begin{program}
(iter (for i from 1 to 10)
      (let ((x 3))
        (collect i into x)))
\end{program}
If this code were executed, |collect| would create a binding for its
|x| at the top level of the \iter\ form that the |let| will shadow.

Happily, \iter\ is smart enough to catch these errors; it
walks all problematical code to ensure that free variables are not
bound inside the loop body, and checks all variables it binds for the
same problem.

However, some errors cannot be caught:

\begin{program}
(iter (with x = 3)
      (for el in list)
      (setq x 1)
      (reducing el by \#'+ initial-value x))
\end{program}
|reducing| moves its |initial-value| argument to the initialization
part of the loop in order to produce more efficient code.  Since
\iter\ does not perform data-flow analysis, it cannot determine that
|x| is changed inside the loop; all it can establish is that |x| is
not bound internally.  Hence this code will not signal an
error and will use $3$ as the initial value of the reduction.

The following list summarizes all cases that are subject to these code
motion and variable-shadowing problems.
\begin{itemize}
\item Any variable for which \iter\ creates a binding, including those
used in |with| and the |into| keyword of many clauses.

\begin{sloppypar}
\item The special clauses which place code: |initially|, |after-each|, |else|,
|finally| and |finally-protected|. 
\end{sloppypar}

\item The variables of a |next| or |do-next| form.

\item The |initially| arguments of |for\dots initially\dots then| and
|for\dots previous|. 

\item The |then| argument of |for\dots initially\dots then|.

\item The |initial-value| arguments of |reducing| and |accumulate|.

\item The |on-failure| argument of |finding\dots such-that|.
\end{itemize}

%%%%%%%%%%%%%%%%%%%%%%%%%%%%%%%%%%%%%%%%%%%%%%%%%%%%%%%%%%%%%%%%
\section{Differences Between {\tt Iterate} and {\tt Loop}}
|loop| contains a great deal of complexity which \iter\ tries to
avoid.  Hence many esoteric features of |loop| don't exist in \iter.
Other features have been carried over, but in a cleaned-up form.  
And of course, many new features have been added; they are not
mentioned in this list.

\begin{itemize}

\item \iter's syntax is more Lisp-like than |loop|'s, having a higher
density of parens.

\item The current implementation of \iter, unlike the current version
of  |loop| (as documented in {\em Common Lisp, 2nd Ed.\/}), is
extensible (see section \ref{extend}).  

\item |loop| puts the updates of all driver variables at the top of
the loop; \iter\ leaves them where the driver clauses appear.

\item While for the most part \iter\ clauses that resemble |loop| clauses
behave similarly, there are some differences.  For instance, there is
no |for\dots =\dots then| in \iter; instead use 
|for\dots initially\dots then|.  

\item |loop| binds the variable |it| at certain times
to allow pseudo-English expressions like |when ~expr~ return it|. 
In \iter, you must bind ~expr~ to a variable yourself.  Note that
|when ~expr~ return it| is like |thereis ~expr~| except that the latter is an
accumulation clause and therefore competes with other accumulations
(remember section \ref{multiple} above).

% repeat different behaviour of |always| clause here?

\item |loop| has a special |return| clause, illustrated in the
previous item.  \iter\ doesn't need one,  since an ordinary Lisp
|return| has the same effect.

\item |loop| allows for parallel binding and stepping of iteration
variables.  \iter\ does not.  (See section \ref{bindings}.)

\item |loop| and \iter\ handle variable type declarations very
differently.  |loop| provides a special syntax for declaring variable
types, and does not examine declarations.  Moreover, the standard
implementation of |loop| will
generate declarations when none are requested.
\iter\ parses standard Common Lisp type declarations, and will never
declare a variable itself unless declarations are 
specifically requested.

\end{itemize}

%%%%%%%%%%%%%%%%%%%%%%%%%%%%%%%%%%%%%%%%%%%%%%%%%%%%%%%%%%%%%%%%
\section{Rolling Your Own}
\label{extend}

\subsection{Introduction}

\iter\ is extensible---you can write new clauses that embody new
iteration patterns.  You might want to write a new driver clause for a
data structure of your own, or you might want to write a clause that
collects or manipulates elements in a way not provided by \iter. 

This section describes how to write clauses for \iter.  Writing a
clause is like writing a macro.  In fact, writing a clause ~is~
writing a macro: since \iter\ code-walks its body and macroexpands,
you can add new abstractions to \iter\ with good old |defmacro|.

Actually, there are two extensions you can make to \iter\ that are
even easier than writing a macro.  They are adding a synonym for an
existing clause and defining a driver clause for an indexable
sequence.  These can be done with |defsynonym| and
|defclause-sequence|, respectively.  See section \ref{aids}, below.

The rest of this section explains how to write macros that expand into
\iter\ clauses.
Here's how you could add a simplified version of \iter's |multiply|
clause, if \iter\ didn't already have one:
\begin{program}
(defmacro multiply (expr)
  `(reducing ,expr by \#'* initial-value 1))
\end{program}

If you found yourself summing the square of an expression often, you
might want to write a macro for that.  A first cut might be
\begin{program}
(defmacro sum-of-squares (expr)
  `(sum (* ,expr ,expr)))
\end{program}
but if you are an experienced macro writer, you will realize that this
code will evaluate ~expr~ twice, which is probably a bad idea.  A
better version would use a temporary:
\begin{program}
(defmacro sum-of-squares (expr)
  (let ((temp (gensym)))
    `(let ((,temp ,expr))
       (sum (* ,temp ,temp)))))
\end{program}
Although this may seem complex, it is just the sort of thing you'd
have to go through to write any macro, which illustrates the point of
this section: if you can write macros, you can extend \iter.

Our macros don't use \iter's keyword-argument
syntax.  We could just use keywords with |defmacro|, but we would
still not be using \iter's clause indexing mechanism.  Unlike Lisp,
which uses just the first symbol of a form to determine what function
to call, \iter\ individuates clauses by the list of required keywords.
For instance, |for\dots in| and |for\dots in-vector| are different clauses
implemented by distinct Lisp functions.  

To buy into this indexing scheme, as well as the keyword-argument
syntax, use |defmacro-clause|:

\begin{clauses}

\defmacro{defmacro-clause}{arglist |\&body| body}
Defines a new \iter\ clause.  ~arglist~ is a list of symbols which are
alternating keywords and arguments.  \opt\ may be used, and the list
may be terminated by |\&sequence|.  ~body~ is an ordinary macro body,
as with |defmacro|.  If the first form of ~body~ is a string, it is
considered a documentation string and will be shown by
|display-iterate-clauses|. |defmacro-clause| will signal an error if 
defining the clause would result in an ambiguity.  E.g. you cannot
define the clause |for\dots from| because there would be no way to
distinguish it from a use of the |for| clause with optional keyword |from|.

\end{clauses}

\medskip

Here is |multiply| using |defmacro-clause|.  The keywords are capitalized
for readability.
\begin{program}
(defmacro-clause (MULTIPLY expr \opt\ INTO var)
  `(reducing ,expr by \#'* into ,var initial-value 1))
\end{program}
You don't have to worry about the case when |var| is not supplied; for
any clause with an |into| keyword, saying |into nil| is equivalent to
omitting the |into| entirely.

As another, more extended example, consider the fairly common
iteration pattern that 
involves finding the sequence element that maximizes (or minimizes) some
function.  \iter\ provides this as |finding\dots maximizing|, but it's
instructive to see how to write it.
Here, in pseudocode, is how you might write such a loop for
maximizing a function F:

\begin{quote}
\begin{tabbing}
set variable MAX-VAL to NIL; \\
set variable WINNER to NIL; \\
for\=\ each element EL in the sequence \\
\>  if\=\ MAX-VAL\=\ is NIL or F(EL) $>$ MAX-VAL then \\
\>\>    set MAX-VAL to F(EL); \\
\>\>    set WINNER to EL; \\
\>  end if; \\
end for; \\
return WINNER.
\end{tabbing}
\end{quote}

Here is the macro:
\begin{program}
(defmacro-clause (FINDING expr MAXIMIZING func \opt\ INTO var)
  (let ((max-val (gensym))
        (temp1 (gensym))
        (temp2 (gensym))
        (winner (or var iterate::*result-var*)))
    `(progn 
       (with ,max-val = nil)
       (with ,winner = nil)
       (cond 
        ((null ,max-val)
         (setq ,winner ,expr)
         (setq ,max-val (funcall ,func ,winner))
        (t
         (let* ((,temp1 ,expr)
                (,temp2 (funcall ,func ,temp1)))
           (when (> ,temp2 ,max-val)
             (setq ,max-val ,temp2)
             (setq ,winner ,temp1))))))
       (finally (leave ,winner)))))
\end{program}
Note that if no |into| variable is supplied, we use
|iterate::*result-var*|, which contains the internal variable into
which all clauses place their results.  If this variable is bound by
some clause, then \iter\ will return its value automatically;
otherwise, \nil\ will be returned.  

\subsection{Writing Drivers}

In principle, drivers can be implemented just as easily as other
\iter\ clauses.  In practice, they are a little harder to get right. 
As an example, consider writing a driver that
iterates over all the 
elements of a vector, ignoring its fill-pointer.  |for\dots in-vector|
won't work for this, because it observes the fill-pointer.  It's necessary to
use |array-dimension| instead of |length| to obtain the size of the
vector.  Here is one approach:

\begin{program}
(defmacro-clause (FOR var IN-WHOLE-VECTOR v)
  "All the elements of a vector (disregards fill-pointer)"
  (let ((vect (gensym))
        (index (gensym)))
    `(progn
       (with ,vect = ,v)
       (for ,index from 0 below (array-dimension ,vect 0))
       (for ,var = (aref ,vect ,index)))))
\end{program}
Note that we immediately put |v| in a variable, in case it is an
expression.  Again, this is just good Lisp macrology.  It also has a
subtle effect on the semantics of the driver: |v| is evaluated only
once, at the beginning of the loop, so changes to |v| in the loop have
no effect on the driver.  Similarly, the bounds for numerical iteration
e.g. the above |array-dimension| are also evaluated once only.  This is how
all of \iter's drivers work.


There is an important point concerning the |progn| in this code.  We
need the |progn|, of course, because we are returning several forms,
one of which is a driver.  But \iter\ drivers must occur at top-level.
Is this code in error?  No, because ~top-level~ is defined in \iter\
to include forms inside a |progn|.  This is just the definition of
top-level that Common Lisp uses, and for the same reason: to allow
macros to return multiple forms at top-level.

While our |for\dots in-whole-vector| clause will work, it is not
ideal.  In particular, it does not support generating.  Do do so, we
need to use |for\dots next| or |for\dots do-next|.  The job is
simplified by the |defmacro-driver| macro.

\begin{clauses}

\defmacro{defmacro-driver}{arglist |\&body| body}
Defines a driver clause in
both the |for| and |generate| forms, and provides a parameter
|generate| which ~body~ can examine to determine how it was invoked.
~arglist~ is as in |defmacro-clause|, and should begin with the symbol
|for|. 

\end{clauses}

With |defmacro-driver|, our driver looks like this:
\begin{program}
(defmacro-driver (FOR var IN-WHOLE-VECTOR v)
  "All the elements of a vector (disregards fill-pointer)"
   (let ((vect (gensym))
         (end (gensym))
         (index (gensym))
         (kwd (if generate 'generate 'for)))
     `(progn
        (with ,vect = ,v)
        (with ,end = (array-dimension ,vect 0))
        (with ,index = -1)
        (,kwd ,var next (progn (incf ,index)
                               (if (>= ,index ,end) (terminate))
                               (aref ,vect ,index))))))
\end{program}


We are still missing one thing: the |\&sequence| keywords.  
We can get them easily enough, by writing 
\begin{program}
(defmacro-driver (FOR var IN-WHOLE-VECTOR v \&sequence) 
  ...)
\end{program}
We can now refer to parameters |from|, |to|, |by|, etc. which contain
either the values for the corresponding keyword, or \nil\ if the
keyword was not supplied.  Implementing the right code for these
keywords is cumbersome but not difficult; it is left as an exercise.
But before you begin, see |defclause-sequence| below for an easier way.
       
\subsection{Extensibility Aids}
\label{aids}

This section documents assorted features that may be of use in
extending \iter.

\begin{clauses}

\defunexpvar{*result-var*}
Holds the variable that is used to return a value as a result of the
\iter\ form.  You may examine this and use it in a |with| clause, but
you should not change it.

\defmacro{defsynonym}{syn word}
 Makes ~syn~ a synonym for the existing \iter\ keyword ~word.~  Only
the first word in each clause can have synonyms. 

\defmacroxx{defclause-sequence}{element-name index-name}{|\&key|
access-fn size-fn sequence-type}{element-type
element-doc-string index-doc-string}  
Provides
a simple way to define sequence clauses.  Generates two
clauses, one for iterating over the sequence's elements, the other
for iterating over its indices.  The first symbol of both
clauses will have print-name |for|.
~element-name~  and ~index-name~ should be symbols.
~element-name~ is the second keyword of the element iterator (typically of
the form 
|in-~sequence-type~|), and ~index-name~ is the second keyword
of the index-iterator (typically of the form
|index-of-~sequence-type~|).  Either name may be 
\nil, in which case the corresponding clause is not defined.  If both
symbols are supplied, they should be in the same package.  The |for|
that begins the clauses will be in this package.

\cpar ~access-fn~ is the function to be used to
access elements of the sequence in the element iterator.  The function
should take two 
arguments, a sequence and an index, and return the appropriate element.
~size-fn~ should denote a function of one argument, a sequence, that
returns its size.  Both ~access-fn~ and ~size-fn~ are required for the
element iterator, but only ~size-fn~ is needed for the index iterator.

\cpar The ~sequence-type~ and ~element-type~ keywords are used to
suggest types for the variables 
used to hold the sequence and the
sequence elements, respectively.  The usual rules about \iter's
treatment of variable type declarations apply (see section \ref{types}).

\cpar ~element-doc-string~ and ~index-doc-string~ are
the documentation strings, for use with |display-iterate-clauses|. 

\cpar The generated element-iterator performs destructuring on the
element variable.

\cpar As an example, the above |for\dots in-whole-vector| example
could have been written:
\begin{program}
(defclause-sequence IN-WHOLE-VECTOR INDEX-OF-WHOLE-VECTOR
  :access-fn 'aref
  :size-fn \#'(lambda (v) (array-dimension v 0))
  :sequence-type 'vector
  :element-type t
  :element-doc-string 
     "Elements of a vector, disregarding fill-pointer"
  :index-doc-string 
     "Indices of vector, disregarding fill-pointer")
\end{program}

\end{clauses}

\subsection{Subtleties}

There are some subtleties to be aware of when writing \iter\ clauses.
First, the code returned by your macros may be |nconc|'ed into a list,
so you should always returned freshly consed lists, rather than
constants.  Second, \iter\ matches clauses by using |eq| on the first
symbol and |string=| on the subsequent ones, so the package of the
first symbol of a clause is relevant.  All of the clauses in this manual
have their first word in the \iter\ package.  
You can use the package system in the usual way to shadow
\iter\ clauses without replacing them.

%%% say more here, about the badness that only the first word of a
%%% clause is packagey.

\section{Non-portable Extensions to Iterate (Contribs)}
\label{contribs}

Currently, there is only one non-portable extension to iterate in the
distribution: iterate-pg. If you have made an extension that depends
on non-portable features, feel free to send them to |asf@boinkor.net|
for inclusion in the iterate distribution.

\subsection{An SQL query driver for iterate}

The pg package by Eric Marsden (see |http://cliki.net/pg|) provides an
interface to the PostgreSQL database. Using the \iterpg\ extension, it
is possible to handle the results of SQL queries using \iter.

This usage example should give you an idea of how to use it:

\begin{program}
(pg:with-pg-connection (c "somedb" "someuser")
  (iter (for (impl version date) in-relation "select * from version"
                                 on-connection *dbconn*)
        (collect version)))
\end{program}

To use the extension via |ASDF|, simply make your system depend on the
|iterate-pg| system instead of the |iterate| system. To load it
manually, use:

\begin{program}
  (asdf:oos 'asdf:load-op :iterate-pg)
\end{program}

%%%%%%%%%%%%%%%%%%%%%%%%%%%%%%%%%%%%%%%%%%%%%%%%%%%%%%%%%%%%%%%%
\section{Obtaining {\tt Iterate}}

\begin{sloppypar}
The information in this chapter is obsolete but included for
completeness's sake; Currently, the most up-to-date information on
\iter\ can be found at |http://boinkor.net/iterate.html|.
\end{sloppypar}

\begin{sloppypar}
\iter\ currently runs on Lisp Machines, and on
HP's, Sun3's and Sparcstations under Lucid.  
\iter\ source and binaries are available at the MIT AI Lab in the
subdirectories of |/src/local/lisplib/|.  The source file,
|iterate.lisp|, is also available for anonymous FTP in the directory
|/com/fpt/pub/| on the machine |TRIX.AI.MIT.EDU| (Internet number
128.52.32.6).  If you are unable to obtain |iterate| in one of these
ways, send mail to |jba@ai.mit.edu| and I will send you the source
file. 
\end{sloppypar}

\begin{sloppypar}
\iter\ resides in the |iterate| package (nickname |iter|).  Just say
\linebreak |(use-package :iterate)| to make all the necessary symbols
available. 
If a symbol is not exported, it appears in this manual with an
``|iterate::|'' prefix.
\end{sloppypar}

Send bug reports to |bug-iterate@ai.mit.edu|.  The |info-iterate|
mailing list will have notices of changes and problems; to have
yourself added, send mail to |info-iterate-request@ai.mit.edu|.


\medskip

\begin{flushleft}
 \bf Acknowledgements
\end{flushleft}
\smallskip

Richard Waters provided invaluable criticism which spurred me to improve
\iter\ greatly.  As early users, David Clemens, Oren Etzioni and Jeff
Siskind helped ferret out many bugs.

%\begin{theindex}
% The index files must be generated with the genindex program
% or from makeindex iterate-manual.idx which creates iterate-manual.ind.

%\baselineskip\normalbaselineskip
%\advance\baselineskip by -2pt

% Fix ``.'' spacing in lists
\documentstyle[12pt,aimemo]{article}

\setlength{\textwidth}{6.0in}
\setlength{\oddsidemargin}{0.25in}
\setlength{\evensidemargin}{0.25in}
\setlength{\topmargin}{0pt}
\setlength{\headsep}{24pt}
\setlength{\textheight}{8.5in}

\let\origpar=\par

% somehow, LaTex fucks with things so that the plain Tex \obeyspaces
% doesn't work.  This is the same, except we use an mbox containing
% a space instead of just a space.

\newcommand{\spc}{\mbox{ }}
\newcommand{\tAb}{\mbox{        }}
\def\obspaces{\catcode`\ =\active\catcode`\^^I=\active}
{\obspaces \global\let =\spc\global\let^^I=\tAb}

\newenvironment{program}{\medskip\samepage\tt\obeylines\obspaces}{\medskip}

% use | for \tt
\catcode`\|=13
\newif\ifvbar
\def|{\ifvbar
         \endgroup\vbarfalse
      \else 
         \begingroup\tt\vbartrue 
      \fi}



\newcommand{\lisp}{\tt}
\newcommand{\iter}{|iterate|}
\newcommand{\iterpg}{|iterate-pg|}
\newcommand{\setf}{|setf|}
\newcommand{\nil}{|nil|}
\newcommand{\nonnil}{non-\nil}
\newcommand{\opt}{{\lisp \&optional}}
\newcommand{\yields}{$\Rightarrow$}



\newbox\Dots
\setbox\Dots=\hbox{\small ...}
\renewcommand{\dots}{\copy\Dots}

% Here's a way to define an environment which will treat all
% paragraphs inside it in the same manner.  It's, as near as I can
% figure, the way that Latex does things.  Using \everypar won't work;
% apparently Latex resets that a lot.  What you have to do is: at the
% top of the file do \let\origpar=\par.  In the
% begin part of the environment, do the parshape (or whatever), then say
% \def\par{{\origpar}}.  The extra braces are necessary.   That's all
% you have to do; at the end of the environment things will return to
% normal automatically.  Why does this all work?  I have no idea.
% Another important thing: there should be a blank line (i.e. a \par)
% between the end of the last paragraph and the \end{environment}.

\newlength{\clindent}
\setlength{\clindent}{0.5in}
\newlength{\clparindent}
\setlength{\clparindent}{\parindent}

\newenvironment{clauses}{\advance\linewidth -\parindent
                        \hangindent\clindent\relax
                         \hangafter=0
                          \parindent=0pt
                          \def\par{{\origpar}}}{}

\newcommand{\cpar}{\hspace \clparindent}

\newcommand{\startitem}{\bigskip\pagebreak[3]}
\newcommand{\finishitem}{\smallskip}

\newcommand{\clause}[1]{\startitem\Clause{#1}\finishitem}
\newcommand{\Clause}[1]{\hbox{#1}}
\newcommand{\clindex}[1]{\index{{\lisp #1}}}
\newcommand{\clindexx}[2]{\index{{\lisp #1\dots #2}}}
\newcommand{\clindexxx}[3]{\index{{\lisp #1\dots #2\dots #3}}}
\newcommand{\clausex}[2]{\startitem\Clausex{#1}{#2}\finishitem}
\newcommand{\Clausex}[2]{\Clause{|#1| #2}\clindex{#1}}

\newcommand{\defvar}[1]{\startitem\deff{#1}{}{Variable}\finishitem}
\newcommand{\defunexpvar}[1]{\startitem\deffnoindex{iterate::#1}{}{Variable}
                                        \lispindex{#1}\finishitem}
\newcommand{\defun}[2]{\startitem\deff{#1}{#2}{Function}\finishitem}
\newcommand{\defmacro}[2]{\startitem\deff{#1}{#2}{Macro}\finishitem}
\newcommand{\defunx}[3]{\startitem\deffx{#1}{#2}{#3}{Function}\finishitem}
\newcommand{\defmacrox}[3]{\startitem\deffx{#1}{#2}{#3}{Macro}\finishitem}
\newcommand{\defunxx}[4]{\startitem\deffxx{#1}{#2}{#3}{#4}{Function}\finishitem}
\newcommand{\defmacroxx}[4]{\startitem\deffxx{#1}{#2}{#3}{#4}{Macro}\finishitem}

\newcommand{\deff}[3]{\deffnoindex{#1}{#2}{#3}\lispindex{#1}}
\newcommand{\deffnoindex}[3]{\hbox to \hsize{{\tt #1} {\it #2}\hfill [{\it #3}]}}
\newcommand{\deffx}[4]{\deff{#1}{#2}{#4}\moreargs{#1}{#3}}
\newcommand{\deffxx}[5]{\deffx{#1}{#2}{#3}{#5}\moreargs{#1}{#4}}

\newcommand{\moreargs}[2]{\hbox to \hsize{\phantom{\lisp #1} {\it #2}\hfill}}

\newcommand{\lispindex}[1]{\index{{\lisp #1}}}

 \newenvironment{note}[1]{\pagebreak[2]\bigskip
                          \hrule\smallskip\small
                         {\setlength{\parindent}{0pt}
                           \par{\bf #1:}}}{\smallskip\hrule\bigskip}



\makeindex

%%%%%%%%%%%%%%%%%%%%%%%%%%%%%%%%%%%%%%%%%%%%%%%%%%%%%%%%%%%%%%%%

%\thispagestyle{empty}

%\vspace*{2in}

%\title{The |Iterate| Manual}
%\author{Jonathan Amsterdam}

%\maketitle

%\thispagestyle{empty}

%\begin{center}
%\Large *** DRAFT ***
%\end{center}
%This manual is in beta-test. It will become an AI Memo in a couple of
%months.  I would appreciate any comments or
%corrections.  You can reach me at email address |jba@ai.ai.mit.edu|.

\begin{document}

\def\writtenby{Jonathan Amsterdam}
\def\contractno{This report describes research done at the Artificial
Intelligence Laboratory of the Massachusetts Institute of Technology.
Support for the laboratory's artificial intelligence research is
provided in part by the Advanced Research Projects Agency of the
Department of Defense under Office
of Naval Research contract N00014-85-K-0124.}

\AIMEMO{1236}{\Large The Iterate Manual \\[1in]}
{This is the manual for version 1.4 of \iter, a powerful iteration
macro for Common  
Lisp. \iter\ is similar to {\tt loop} but provides numerous additional
features, is well integrated with Lisp, and is extensible. 
\vfill }


% use ~ for \em
\catcode`\~=13
\newif\ifem
\def~{\ifem
         \/\endgroup\emfalse
      \else 
         \begingroup\em\emtrue 
      \fi}



%\pagebreak
%\setcounter{page}{1}

%\thispagestyle{empty}


\tableofcontents

\pagebreak
\pagestyle{headings}

\section{Introduction}

\begin{sloppypar}
This manual describes
\iter, a powerful iteration facility for Common Lisp.
\iter\ provides abstractions for many common iteration
patterns and allows for the definition of additional patterns.  \iter\ is
a macro that expands into ordinary Lisp at
compile-time, so it is more efficient than higher-order
functions like |map| and |reduce.|  
While it is similar to |loop|, \iter\ offers a more Lisp-like syntax
and enhanced extensibility.
(For a more complete
comparison of \iter\ with other iteration constructs, see MIT AI Lab Working
Paper No. 324, ~Don't Loop, Iterate.~)
\end{sloppypar}

An \iter\ form consists of the symbol |iter|\footnote{You can also use
|iterate|, but |iter| is preferred because it avoids potential conflicts with
possible future additions to Common Lisp, and because it saves
horizontal space when writing code.}
followed by one or more
forms, some of which may be \iter\ 
~clauses.~  Here is a simple example of \iter\ which collects the
numbers from 1 to 10 into a list, and returns the list.  The return
value is shown following the arrow.
\begin{program}
(iter (for i from 1 to 10)
      (collect i))          \yields (1 2 3 4 5 6 7 8 9 10)
\end{program}
This form contains two clauses: a |for| clause that steps the
variable |i| over the integers from 1 to 10, and a |collect|
clause that accumulates its argument into a list.  With a few
exceptions, all \iter\ clauses have the same format: alternating
symbols (called ~keywords~) and expressions (called
~arguments~).  The syntax and terminology are those of Common
Lisp's keyword lambda lists.  One difference is that \iter's keywords
do not have to begin with a colon---though they may, except for the
first symbol of a clause.  So you can also write  |(for i :from
1 :to 10)| if you prefer.

Any Lisp form can appear in the body of an \iter, where it will have
its usual meaning.  \iter\ walks the entire body, expanding macros,
and recognizing clauses at any level.  This example collects all the
odd numbers in a list:
\begin{program}
(iter (for el in list)
      (if (and (numberp el) (oddp el))
          (collect el)))
\end{program}

There are clauses for iterating over numbers, lists, arrays and other
objects, and for
collecting, summing, counting, maximizing and other useful operations.
\iter\ also supports the creation of new variable bindings, stepping
over multiple sequences at once, destructuring, and compiler
declarations of variable types.  The following example illustrates
some of these features: 
\begin{program}
(iter (for (key . item) in alist)
         (for i from 0)
         (declare (fixnum i))
         (collect (cons i key)))
\end{program}

This loop takes the keys of an alist and returns a new alist
associating the keys with their positions in the original list. The
compiler declaration for |i| will appear in the generated code
in the appropriate place.

%%%%%%%%%%%%%%%%%%%%%%%%%%%%%%%%%%%%%%%%%%%%%%%%%%%%%%%%%%%%%%%%
\section{Clauses}

Most of \iter's clauses will be familiar to |loop| programmers.  
(|loop| is an iteration macro that has been incorporated into Common
Lisp.  Seen Guy Steele's {\em Common Lisp, 2nd Edition}.)
In
nearly all cases they behave the same as their |loop| counterparts, so
a |loop| user can switch to \iter\ with little pain (and much gain).

All clauses with the standard keyword-argument syntax consist of two
parts: a ~required~ part, 
containing keywords that must be present and in the right order; and
an ~optional~ part, containing keywords that may be omitted and,
if present, may occur in any order.  In the descriptions below, the
parts are separated by the Lisp lambda-list keyword \opt.


\subsection{Drivers}

An iteration-driving clause
conceptually causes the iteration to go forward.  Driver clauses in
\iter\ allow iteration over numbers, lists, vectors, hashtables, packages,
files and streams.  Iteration-driving clauses must
appear at the top level of an \iter\ form; they cannot be nested
inside another clause.  The driver variable is updated at the point
where the driver clause occurs.  Before the clause is executed for the
first time, the value of the variable is undefined.

%Also, regardless of where the driver clause
%appears in the body, the driver variable is stepped at the top of the
%loop; hence it is stylistically preferable, though
%not required, to place driver clauses at the beginning of the \iter.

Multiple drivers may appear in a single \iter\ form, in which case all
of the driver variables are updated each time through the loop, in the
order in which the clauses appear.  The first driver to terminate will
terminate the entire loop.

In all cases, the value of the driver variable on exit from the loop,
including within the epilogue code (see the |finally| clause), is
undefined.

All the parameters of a driver clause are evaluated once, before the
loop begins.  Hence it is not possible to change the bounds or other
properties of an iteration by side-effect from within the loop.

With one exception, driver clauses begin with the word |for| (or
the synonym |as|) and mention an iteration variable, which is
given a binding within the \iter\ form.
The exception is |repeat|, which just executes a loop a
specified number of times:

\begin{clauses}

\clausex{repeat}{~n~}
Repeats the loop ~n~ times.  For example:
\begin{program}
(iter (repeat 100)
      (print "I will not talk in class."))
\end{program}
\cpar If $n \leq 0$, the
loop will never be executed.  If ~n~ is not an integer, the
actual number of executions will be $\lceil n \rceil$.

\end{clauses}

\pagebreak[3]

\subsubsection{Numerical Iteration}

\begin{clauses}
\clausex{for}{~var~ |\&sequence|} 
The general form for iterating over a sequence of numbers requires a
variable and, optionally, one or more keywords that provide the bounds
and step size of the iteration. The |\&sequence|
lambda-list keyword is a shorthand for these sequence keywords.  They are:
|from|, |upfrom|, |downfrom|, |to|, |downto|, |above|, |below| and |by|.
|from| provides the starting value for ~var~
and defaults to zero. |to| provides a final value and implies that the
successive values of ~var~ will be increasing; |downto| implies that
they will be decreasing.  The loop terminates when ~var~ passes
the final value (i.e. becomes smaller or larger than it, depending on
the direction of iteration); in other  words, the loop body
will never be 
executed for values of ~var~ past the final value.
|below| and |above| are similar to |to| and
|downto|, except that the loop terminates when ~var~ equals or passes
the final value.

\cpar If no final value is specified, the variable will be stepped
forever.  Using |from| or |upfrom| will result in increasing
values, while |downfrom| will give decreasing values.

\cpar On each iteration,
~var~ is incremented or decremented by the value of the sequence
keyword |by|, which defaults to 1.  It should always be a positive
number, even for downward iterations.

\cpar In the following examples, the sequence of numbers generated is shown
next to the clause.

\begin{program}
(for i upfrom 0) \yields\ 0 1 2 \ldots
(for i from 5)   \yields\ 5 6 7 \ldots    ; either from or upfrom is okay
(for i downfrom 0) \yields\ 0 -1 -2 \ldots
(for i from 1 to 3) \yields\ 1 2 3
(for i from 1 below 3) \yields\ 1 2
(for i from 1 to 3 by 2) \yields\ 1 3
(for i from 1 below 3 by 2) \yields\ 1
(for i from 5 downto 3) \yields\ 5 4 3
\end{program}

\end{clauses}

\subsubsection{Sequence Iteration}

There are a number of clauses for iterating over sequences.  In all of
them, the argument following |for| may be a list instead of a
symbol, in which case destructuring is performed.  See section
\ref{destructuring}. 

\begin{clauses}

\clause{|for| ~var~ |in| ~list~ \opt\ |by| ~step-function~}
\clindexx{for}{in}
~var~ is set to successive elements of list.  
~step-function,~ which 
defaults to |cdr|, is used to obtain the next sublist. 

\clause{|for| ~var~ |on| ~list~ \opt\ |by| ~step-function~}
\clindexx{for}{on}
~var~ is set to successive sublists of list. 
~step-function~ (default |cdr|) is used as in |for\dots in|.

\end{clauses}

\medskip

These two clauses use |atom| to test for the end
of a list.  Hence, given a list whose final |cdr| is not \nil,
they will silently ignore the last |cdr|.  Other choices are
|endp|,  which would signal an error, and |null|, which
would probably result in an error somewhere else.  If you wish to use
an end-test other than |atom|, set the variable
|iterate::*list-end-test*|\lispindex{*list-end-test*} to the name of
the desired function. 

\begin{clauses}
\clause{|for| ~var~ |in-vector| ~vector~ |\&sequence|}
\clindexx{for}{in-vector}
~var~ takes on successive elements from ~vector.~   The vector's
fill-pointer is observed.  Here and in subsequent clauses, the |\&sequence|
keywords include |with-index|, which takes a symbol as argument
and uses it for the index variable instead of an internally generated
symbol.  The other |\&sequence| keywords behave as in numerical
iteration, except that the default iteration bounds are the bounds of
the vector.  E.g. in |(for i in-vector v downto 3)|,
|i| will start off being bound to the last element in |v|,
and will  
be set to preceding elements down to and including the element with
index 3.

\clause{|for| ~var~ |in-sequence| ~seq~ |\&sequence|}
\clindexx{for}{in-sequence}
This uses Common Lisp's generalized sequence functions, |elt|
and |length|, to obtain elements and determine the length of
~seq.~   Hence it will work for any sequence, including lists, and will
observe the fill-pointers of vectors.

\clause{|for| ~var~ |in-string| ~string~ |\&sequence|}
\clindexx{for}{in-string}
~var~ is set to successive characters of ~string.~

\startitem
\Clause{|for| ~var~ |index-of-vector| ~vector~ |\&sequence|}
\clindexx{for}{index-of-vector}
\Clause{|for| ~var~ |index-of-sequence| ~sequence~ |\&sequence|}
\clindexx{for}{index-of-sequence}
\Clause{|for| ~var~ |index-of-string| ~string~ |\&sequence|}
\clindexx{for}{index-of-string}
\finishitem
~var~ is set to successive indices of the sequence.
These clauses avoid the overhead of accessing the sequence elements
for those applications where they do not need to be examined, or are
examined rarely.  They admit all the optional keywords of the other
sequence drivers except the (redundant) |with-index| keyword.

\clause{|for| ~(key value)~ |in-hashtable| ~table~}
\clindexx{for}{in-hashtable}

~key~ and ~value,~ which must appear as shown in a list
and may be destructuring templates, are set to the keys and values of
~table~.  If ~key~ is |nil|, then the hashtable's keys will be ignored;
similarly for ~value~.
The order in which 
elements of ~table~ will be retrieved is unpredictable.

\clause{|for| ~var~ |in-package| ~package~ \opt\ |external-only| ~ext~}
\clindexx{for}{in-package}
Iterates over all the symbols in ~package,~ or over only the
external symbols if ~ext~ is specified and non-|nil|.  ~ext~ is not
evaluated.  The same symbol may appear more than once.

\clause{|for| ~var~ |in-packages| ~packages~ \opt\ |having-access| ~symbol-types~}
\clindexx{for}{in-packages}
Iterates over all the symbols from the list of packages denoted by the
descriptor ~packages~ and having accessibility (or visibility) given by
~symbol-types~.  This defaults to the list |(:external :internal :inherited)|
and is not evaluated.  ~var~ must be a list of up to three variables: in each
iteration, these will be set to a symbol, its access-type and package (as per
|with-package-iterator| in ANSI-CL).  The same symbol may appear more than
once.

\end{clauses}

\begin{clauses}

\clause{|for| ~var~ |in-file| ~name~ \opt\ |using| ~reader~}
\clindexx{for}{in-file}
Opens the file ~name~ (which may be a string or pathname) for input,
and iterates 
over its contents. ~reader~ defaults to |read|, so by default ~var~
will be bound to the successive forms in the file.  The \iter\ body is
wrapped in an unwind-protect to ensure that the file is closed no
matter how the \iter\ is exited.

\clause{|for| ~var~ |in-stream| ~stream~ \opt\ |using| ~reader~}
\clindexx{for}{in-stream}
Like |for\dots in-file|, except that ~stream~ should be an existing
stream object that supports input operations.

\end{clauses}

\subsubsection{Generalized Drivers}

These are primarily useful for writing drivers that can also be used
as generators (see section \ref{generators}, below).
 
\begin{clauses}

\clause{|for| ~var~ |next| ~expr~}
\clindexx{for}{next}

~var~ is set to ~expr~ each time through the loop.  Destructuring is
performed.  When the clause is used as a generator, ~expr~ is the code
that is executed when |(next ~var~)| is 
encountered (see section \ref{generators}, below).
~expr~ should compute the first value for ~var~, as well
as all subsequent values, and is responsible for terminating the loop.
For compatibility with future versions of \iter, this termination
should be done with |terminate|\clindex{terminate}, which can be
considered a synonym for |finish| (see section \ref{control-flow}).

\cpar As an example, the following clauses are equivalent to |(for
i from 1 to 10)|:
\begin{program}
(initially (setq i 0))
(for i next (if (> i 10) (terminate) (incf i)))
\end{program}

\clause{|for| ~var~ |do-next| ~form~}
\clindexx{for}{do-next}
~form~ is evaluated each time through the loop.  Its value is ~not~
set to ~var~; that is ~form's~ job. ~var~ is only present so that
\iter\ knows it is a driver variable.  \linebreak |(for ~var~ next
~expr~)| is equivalent to |(for ~var~ do-next (dsetq ~var~ ~expr~))|.
(See section \ref{destructuring} for an explanation of |dsetq|.) 

\end{clauses}

\subsubsection{Generators}
\label{generators}

In all of the above clauses, the driver variable is updated on each
iteration.
Sometimes it is desirable to have greater control over updating.  
For instance, consider the problem of associating numbers, in
increasing order and with no gaps, with the
\nonnil\ elements of a list.  One obvious first pass at writing this is:

\begin{program}
(iter (for el in list)
      (for i upfrom 1)
      (if el (collect (cons el i))))
\end{program}
But on the list |(a b nil c)| this produces |((a . 1) (b . 2) (c .
4))| instead of the desired |((a . 1) (b . 2) (c . 3))|.  The problem
is that |i| is incremented each time through the loop, even when |el|
is |nil|.  

The problem could be solved elegantly if we could step |i| only when
we wished to.  This
can be accomplished for any \iter\ driver by writing |generate| (or
its synonym |generating|)
instead of |for|.  Doing so produces a ~generator~---a driver whose
values are yielded explicitly.  To obtain the next value of a
generator variable ~v~, write \linebreak |(next ~v~)|.  The value of a |next|
form is the next value of ~v~, as determined by its associated driver
clause.  |next| also has the side-effect of updating ~v~ to that
value.  If there is no next value, |next| will terminate the loop,
just as with a normal driver.


Using generators, we can now write our example like this:
\begin{program}
(iter (for el in list)
      (generate i upfrom 1)
      (if el (collect (cons el (next i)))))
\end{program}
Now |i| is updated only when |(next i)| is executed, and this occurs
only when |el| is \nonnil.

To better understand the relationship between ordinary drivers and
generators, observe that we can rewrite an ordinary driver using its
generator form immediately followed by |next|, as this example shows:
\begin{program}
(iter (generating i from 1 to 10)
      (next i)
      ...)
\end{program}
Provided that the loop body contains no |(next i)| forms, this will
behave just as if we had written |(for i from 1 to 10)|.

We can still refer to a driver variable ~v~ without using |next|; in
this case, its value is that given to it by the last evaluation of
|(next ~v~)|.  Before |(next ~v~)| has been called the first time, the
value of ~v~ is undefined.  

This semantics is more flexible than
one in which ~v~ begins the loop bound to its first value and calls
of |next| supply subsequent values, because it means the loop will not
terminate too soon if the generator's sequence is empty.  For
instance, consider the following code, which tags \nonnil\ elements of
a list using a list of tags, and also counts the null elements.
(We assume there are at least as many tags as \nonnil\ elements.)
\begin{program}
(let* ((counter 0)
       (tagged-list (iter (for el in list)
                          (generating tag in tag-list) 
                          (if (null el)
                              (incf counter)
                              (collect (cons el (next tag)))))))
  ...)
\end{program}
It may be that there are just as many tags as non-null elements of
|list|.  If all the elements of |list| are null, we still want the
counting to proceed, even though |tag-list| is \nil.  If |tag| had to be
assigned its first value before the loop begins, we would have had to
terminate the loop before the first iteration, since when |tag-list|
is \nil, |tag| 
has no first value.  With the existing semantics, however, |(next
tag)| will never execute, so the iteration will cover all the elements of
|list|. 
        
When the ``variable'' of a driver clause is actually a destructuring
template containing several variables, all the variables are eligible
for use with |next|.  As before, |(next ~v~)| evaluates to ~v's~ next
value; but the effect is to update all of the template's variables.
For instance, the following code will return the list |(a 2 c)|.
\begin{program}
(iter (generating (key . item) in '((a . 1) (b . 2) (c . 3)))
      (collect (next key))
      (collect (next item)))
\end{program}

Only driver clauses with variables can be made into generators.  This
includes all clauses mentioned so far except for |repeat|.  It does
~not~ include |for\dots previous|, |for\dots =|,
|for\dots initially\dots then| or |for\dots first\dots then| (see
below). 

\subsubsection{Previous Values of Driver Variables}

Often one would like to access the value of a variable on a previous
iteration.  \iter\ provides a special clause for accomplishing this.

\begin{clauses}

\clause{|for| ~pvar~ |previous| ~var~ \opt\ |initially| ~init~
|back| ~n~}
\clindexx{for}{previous}
Sets ~pvar~ to the previous value of ~var,~ which should be a driver
variable, a variable from another |for\dots previous| clause, or a
variable established by a |for\dots =|, 
|for\dots initially\dots then| or |for\dots first\dots then| clause
(see section \ref{setting}).
Initially, ~pvar~ is given the value ~init~ (which defaults to \nil).  
The ~init~ expression will be moved outside the loop body, so it
should not depend on anything computed within the loop.
~pvar~ retains the value of ~init~ until ~var~ is set to its second
value, at which point ~pvar~ is set to ~var's~ first value; and so on.  

\cpar The
argument ~n~ to |back| must be a constant, positive integer, and
defaults to 1.  It determines how many iterations back ~pvar~ should
track ~var~.  For example, when ~n~ is 2, then ~pvar~ will be assigned
~var's~ first value when ~var~ is set to its third value.

\cpar A |for\dots previous| clause may occur after or before its
associated driver clause. |for\dots previous| works with generators as
well as ordinary drivers.

\pagebreak[3]

\cpar Example:
\begin{program}
(iter (for el in '(1 2 3 4))
      (for p-el previous el)
      (for pp-el previous p-el initially 0)
      (collect pp-el))
\end{program}
This evaluates to |(0 0 1 2)|.  It could have been written more
economically as 
\begin{program}
(iter (for el in '(1 2 3 4))
      (for pp-el previous el back 2 initially 0)
      (collect pp-el))
\end{program}


\end{clauses}

\subsection{Variable Binding and Setting}
\label{setting}

Several clauses exist for establishing new variable bindings or for
setting variables in the loop.  They all support destructuring.

\begin{clauses}

\clausex{with}{~var~ \opt\ |=| ~value~}
Causes ~var~ to be bound to value before the loop body is entered.
If ~value~ is not supplied, ~var~ assumes a default
binding, which will be 
\nil\ in the absence of declarations.  Also, if ~value~ is not
supplied, no destructuring is performed; instead, ~var~ may be a list of
symbols, all of which are given default 
bindings.  If ~value~ is supplied, ~var~ is bound to it, with
destructuring.  

\cpar Because |with|
creates bindings whose scope includes the entire \iter\ form, it is
good style to put all |with| clauses at the beginning.

\cpar Successive occurrences of |with| result in sequential
bindings (as with 
|let*|).  There is no way to obtain parallel bindings; see
section \ref{bindings} for a rationale.  


\clause{|for| ~var~ |=| ~expr~}
\clindexx{for}{=}
On each iteration, ~expr~ is evaluated and ~var~ is set
to its value. 

\cpar This clause may appear to do the same thing as |for\dots next|.
In fact, they are quite different.  |for\dots =| provides only three
services: it sets up a binding for ~var~, sets it to ~expr~ on each
iteration, and makes it possible to use |for\dots previous| with
~var~.  |for\dots next| provides these services in addition to the
ability to turn the driver into a generator.
%Also, the code which sets ~var~ appears in the loop body in the same
%place as the |for\dots =| clause; the code for |for\dots next| appears
%at the top of the loop, as with other drivers (except when being used
%as a generator).

\clause{|for| ~var~ |initially| ~init-expr~ |then| ~then-expr~}
\clindexxx{for}{initially}{then}
Before the loop begins, ~var~ is set to ~init-expr;~ on all
iterations after the first it is set to ~then-expr.~  This clause
must occur at top-level.  ~init-expr~ will be moved outside the loop
body and ~then-expr~ will be moved to the end of the loop body, so
they are subject to code motion problems (see section
\ref{code-movement}). 

\cpar This clause may appear to be similar to |for\dots next|, but in
fact they differ significantly.  |for\dots initially\dots then| is
typically used to give ~var~ its first value before the loop begins,
and subsequent values on following iterations.  This is incompatible
with generators, whose first value and subsequent values must all be
computed by |(next ~var~)|.  Also, the update of ~var~ in 
|for\dots initially\dots then| does not occur at the location of the
clause. 
Use |for\dots initially\dots then| for
one-shot computations where its idiom is more convenient, but use
|for\dots next| for extending \iter\ with new drivers (see section
\ref{extend}).

\clause{|for| ~var~ |first| ~first-expr~ |then| ~then-expr~}
\clindexxx{for}{first}{then}
The first time through the loop, ~var~ is set to ~first-expr;~ on
subsequent iterations, it is set to ~then-expr.~  This differs from
|for\dots initially| in that ~var~ is set to ~first-expr~
inside the loop body, 
so ~first-expr~ may depend on the results of other clauses.  For
instance, 
\begin{program}
(iter (for num in list)
      (for i first num then (1+ i))
      ...)
\end{program}
will set |i| to the first element of |list| on the first
iteration, whereas 
\begin{program}
(iter (for num in list)
      (for i initially num then (1+ i))
      ...)
\end{program}
is probably erroneous; |i| will be bound to |num|'s
default binding (usually \nil) for the first iteration.

\end{clauses}

\begin{note}{Compatibility note}
|loop|'s |for\dots =| works like \iter's, but |loop| used the syntax
|for\dots =\dots then| to mean |for\dots initially\dots then|.  It was
felt that these two operations were sufficiently different to warrant
different keywords.

Also, the |for| in the above three clauses is misleading,
since none is true driver (e.g. none has a corresponding |generate|
form).  |setting| would have been a better choice, but |for| was used
to retain some compatibility with |loop|.  
\end{note}


%%%%%%%%%%%%%%%%%%%%%%%%%%%%%%%%
\subsection{Gathering Clauses}
Many of \iter's clauses accumulate values into a variable, or set a
variable under certain conditions.  At the end of the loop, this
variable contains the desired result.  All these clauses have an
optional |into| keyword, whose argument should be a symbol.  If the
|into| keyword is not supplied, the accumulation variable will be
internally generated and its value will be returned at the end of the
loop; if a variable is specified, that variable is used for the
accumulation, and is not returned as a result---it is up to the user
to return it explicitly, in the loop's epilogue code (see |finally|).
It is safe to examine the accumulation variable during the loop, but
it should not be modified.

These clauses all begin with a verb.  When the verb does
not conflict with an existing Common Lisp function, then it may be
used in either its infinitival or present-participle form (e.g. |sum|,
|summing|).  However, when there is a conflict with Common Lisp, only
the present-participle form may be used (e.g. |unioning|).  This is to
prevent \iter\ clauses from clashing with Common Lisp
functions. 

%although these clauses are described as ``producing a value,'' it is a
%mistake to think of the lisp list representing the clause as a
%value-producing form in the usual way.  clauses may legally be written
%where a value is expected, e.g. |(setq x (sum i))|, but the
%lisp value of a clause in such a context is undefined.

\subsubsection{Reductions}

~Reduction~ is an extremely common iteration pattern in which
the results of successive applications of a
binary operation are accumulated.
For example, a loop that computes the sum of the
elements of a list is performing a reduction with the addition
operation.  This could be written in Common Lisp as  |(reduce \#'+
list)| or with \iter\ as
\begin{program}
(iter (for el in list)
      (sum el))
\end{program}

\begin{clauses}

\clausex{sum}{~expr~ \opt\ |into| ~var~}
Each time through the loop, ~expr~ is evaluated and added to a
variable, which is bound initially to zero.  If ~expr~ has a type,
it is ~not~ used as the type of the sum variable, which is always
|number|.  To get the result variable to be of a more specific
type, use an explicit variable, as in
\begin{program}
(iter (for el in number-list)
      (sum el into x)
      (declare (fixnum x))
      (finally (return x)))
\end{program}

\clausex{multiply}{~expr~ \opt\ |into| ~var~}
Like |sum|, but the initial value of the result variable is $1$,
and the variable is updated by multiplying ~expr~ into it.

\clausex{counting}{~expr~ \opt\ |into| ~var~}
~expr~ is evaluated on each iteration.  If it is \nonnil, the
accumulation variable, initially zero, is incremented.

\startitem
\Clausex{maximize}{~expr~ \opt\ |into| ~var~}
\Clausex{minimize}{~expr~ \opt\ |into| ~var~}
\finishitem
~expr~ is evaluated on each iteration and its extremum (maximum or
minimum) is stored in the accumulation variable.  If ~expr~ is never
evaluated, then the result is \nil\ (if the accumulation variable is
untyped) or $0$ (if it has a numeric type).

\clausex{reducing}{~expr~ |by| ~func~ \opt\ |initial-value| ~init-val~ |into| ~var~}
This is a general way to perform reductions. ~func~ should be a
function of two arguments, the first of which will be the value
computed so far and
the second of which will be the value of ~expr~.  It should return the
new value.
|reducing| is roughly equivalent to the
Common Lisp |(reduce ~func~ ~sequence~ :key ~expr-function~)|, where
~expr-function~ is used to derive values from the successive elements
of ~sequence~.

\cpar If the |reducing| clause is never executed, the result is
undefined.

\cpar It is not necessary to provide an
initial value, but better code can be generated if one is supplied.
Regardless of its location in the \iter\ body, ~init-val~ will be
evaluated before the loop is entered, so it should not depend on any
value computed inside the \iter\ form.  

%\cpar if a ~var~ is not specified, you can get \iter\ to declare the
%type of the internal variable by putting a |the| expression around
%~func.~  see section \ref{types}.

\end{clauses}

%\begin{note}{implementation note}
%in principle, |maximize| and |minimize| can be thought of
%as reductions where the initial value is the smallest (or largest)
%value that the accumulation variable can assume.  because lisp's
%bignums can represent arbitrary integers, these clauses cannot be
%implemented as reductions in general.  if, however, the type of 
%~expr~ or ~var~ can be determined to be a fixnum
%or a float, \iter\ will implement the clause as a true reduction,
%using one of the constants |most-negative-fixnum|, |%most-positive-fixnum|, 
%|most-negative-short-float|, etc. as appropriate.
%\end{note}

\subsubsection{Accumulations}

All the predefined accumulation clauses add values to a sequence.  If
the sequence is a list, they all 
have the property that the partial list is kept in the correct order
and available for inspection at any point in the loop.

\begin{clauses}

\clausex{collect}{~expr~ \opt\ |into| ~var~ |at| ~place~ |result-type| ~type~}
Produces a sequence of the values of ~expr~ on each iteration. ~place~
indicates where the next value of ~expr~ is added to the list and may
be one of the symbols |start|, |beginning| (a synonym for |start|) or
|end|.  The symbol may be quoted, but need not be.  The default is
|end|.  For example,
\begin{program}
(iter (for i from 1 to 5)
      (collect i))
\end{program}
produces |(1 2 3 4 5)|, whereas
\begin{program}
(iter (for i from 1 to 5)
      (collect i at beginning))
\end{program}
produces |(5 4 3 2 1)| (and is likely to be faster in most Common Lisp
implementations).

\cpar If ~type~ is provided, it should be a subtype of |sequence|.
The default is |list|.  Specifying a type other than |list| will
result in |collect| returning a sequence of that type.  ~However,~ the
type of the sequence being constructed when inside the loop body is
undefined when a non-|list| type is specified.  (As with ~place~,
quoting ~type~ is optional.)

\clausex{adjoining}{~expr~ \opt\ |into| ~var~ |test| ~test~ |at| ~place~ |result-type| ~type~}
Like |collect|, but only adds the value of ~expr~ if it is not
already present.  ~test,~ which defaults to |\#'eql|, is
the test to be used with |member|.

\startitem
\Clausex{appending}{~expr~ \opt\ |into| ~var~ |at| ~place~}
\Clausex{nconcing}{\ ~expr~ \opt\ |into| ~var~ |at| ~place~}
\Clausex{unioning}{\ ~expr~ \opt\ |into| ~var~ |test| ~test~ |at| ~place~}
\Clausex{nunioning}{~expr~ \opt\ |into| ~var~ |test| ~test~ |at| ~place~}
\finishitem
These are like |collect|, but behave like the Common Lisp functions
|append|, |nconc|, |union| or |nunion|.  
As in Common Lisp, they work only on lists.  Also as in Common Lisp,
|unioning| and |nunioning| assume that the value of ~expr~ contains no
duplicates. 

\clausex{accumulate}{~expr~ |by| ~func~ \opt\ |initial-value| ~init-val~ |into| ~var~}
This is a general-purpose accumulation clause. ~func~ should be a
function of two arguments, the value of ~expr~ and the value
accumulated so far in the iteration, and it should return the updated
value.  If no initial value is supplied, \nil\ is used.  

%\cpar If a ~var~ is not specified, you can get \iter\ to declare the
%type of the internal variable by putting a |the| expression around
%~func.~  see section \ref{types}.

\cpar The differences between |accumulate| and |reducing| are slight.
One difference is that the functions take their arguments in a
different order.  Another is that in the absence of ~init-val~,
|accumulate| will use \nil, whereas |reducing| will generate different
code that avoids any dependence on the initial value.
The reason for having both clauses is that one usually
thinks of reductions (like |sum|) and accumulations (like |collect|)
as different beasts.

\end{clauses}


\subsubsection{Finders}

A ~finder~ is a clause whose value is an expression that meets some
condition.

\begin{clauses}

\clause{|finding| ~expr~ |such-that| ~test~ \opt\ |into| ~var~ |on-failure| ~failure-value~}
\clindexx{finding}{such-that}
If ~test~ (which is an expression) ever evaluates to \nonnil, the loop
is terminated, the 
epilogue code is run and the value of ~expr~ is returned.  Otherwise,
\nil\ (or ~failure-value,~ if provided) is returned.  If ~var~ is
provided, it will have either the \nonnil\ value of ~expr~ or
~failure-value~ when the epilogue code is run.

\cpar As a special case, if the ~test~ expression is a sharp-quoted
function, then it is applied to ~expr~ instead of being simply
evaluated.  E.g. |(finding x such-that \#'evenp)| is equivalent to
|(finding x such-that (evenp x))|.  

%\cpar although ~test~ need have nothing to do with ~%expr~ as in
%|(finding j such-that (> i 3))|, it usually 
%will: |(finding (length el) such-that (oddp (length el)))|.  to
%avoid performing the |length| computation twice, you could write
%|(finding (length el) such-that \#'oddp)| or |(finding (length
%el) such-that 'oddp)|; for these cases, \iter\ generates code that
%executes ~expr~ only once.  the code for |\#'oddp|
%is slightly different from that for {\lisp 'oddp}; see the discussion
%under {\lisp for\dots in} and {\lisp for\dots on}.

\cpar |On-failure| is a misnomer. Because it is always evaluated, it behaves
more like the default third argument to the |gethash| function. As a result,
|on-failure (error "Not found")| makes no sense. Instead, the clauses |leave|
or |thereis| can be used in conjunction with |finally| as follows:
\begin{program}
(iter (for x in '(1 2 3))
      (if (evenp x) (leave x))
      (finally (error "not found")))
\end{program}

\cpar This clause may appear multiple times when all defaults are
identical. It can also be used together with either |always|/|never| or
|thereis| if their defaults match. More specifically, |on-failure nil| is
compatible with |thereis|, while |on-failure t| is compatible with |always|
and |never| clauses.
\begin{program}
(iter (for i in '(7 -4 2 -3))
      (if (plusp i)
	  (finding    i  such-that (evenp i))
          (finding (- i) such-that (oddp i))))
\end{program}

\startitem
\Clause{|finding| ~expr~ |maximizing| ~m-expr~ \opt\ |into| ~var~}
\clindexx{finding}{maximizing}
\Clause{|finding| ~expr~ |minimizing| ~m-expr~ \opt\ |into| ~var~}
\clindexx{finding}{minimizing}
\finishitem
Computes the extremum (maximum or minimum) value of ~m-expr~ over all
iterations, and returns the value of ~expr~ corresponding to the
extremum.  ~expr~ is evaluated inside the loop at the time the new
extremum is established.  If ~m-expr~ is never evaluated (due to, for
example, being embedded in a conditional clause), then the returned
value depends on the type, if any, of ~expr~ (or ~var,~ if
one is supplied).  If there is no type, the returned
value will be \nil; if the type is numeric, the returned value will be
zero.

\cpar For these two clauses, ~var~ may be a list of two
symbols; in that case, the first is used to record ~expr~ and
the second, ~m-expr.~  

\cpar As with |finding\dots such-that|, if ~m-expr~ is a sharp-quoted
function, then it is called on ~expr~ instead of being evaluated.

\end{clauses}

\subsubsection{Boolean Tests}

\begin{clauses}

\clausex{first-iteration-p}{}
Returns |t| in the first cycle of the loop, otherwise \nil.

\clausex{first-time-p}{}
Returns |t| the first time the expression is evaluated, and then \nil\ forever.
This clause comes handy when printing (optional) elements separated
by a comma:

\begin{program}
(iter (for el in '(nil 1 2 nil 3))
      (when el
        (unless (first-time-p)
          (princ ", "))
        (princ el)))
\end{program}
produces |"1, 2, 3"|.

\end{clauses}

\subsubsection{Aggregated Boolean Tests}

\begin{clauses}

\clausex{always}{~expr~}
If ~expr~ ever evaluates to
\nil, then \nil\ is immediately returned; the epilogue code is not
executed.  If ~expr~ never evaluates to \nil, the epilogue code
is executed and the last value of ~expr~ (or |t| if ~expr~ was never
evaluated) is returned (whereas |loop| would constantly return |t|).

% mention last evaluated clause when multiple always clauses?

\clausex{never}{~expr~}
Like |(always (not ~expr~))|, except it does not influence the last
value returned by a possible other |always| clause. That is,
\begin{program}
(iter (repeat 2)
      (always 2)
      (never nil)) \yields 2 ; not t
\end{program}

\clausex{thereis}{~expr~}
If ~expr~ is ever \nonnil,
its value is immediately returned without running epilogue code.
Otherwise, the epilogue code is performed and \nil\ is returned.

This clause cannot be used together with |always| or |never|, because their
defaults are opposed (similarly, |(loop always 3 thereis nil)| refuses to
compile in some implementations of |loop|).

\end{clauses}


\subsection{Control Flow}
\label{control-flow}
Several clauses can be used to alter the usual flow of control in a loop.

Note: the clauses of this and subsequent sections don't adhere to \iter's
usual syntax, but instead use standard Common Lisp syntax.  Hence the
format for describing syntax subsequently is
like the standard format used in the Common Lisp manual, not like the
descriptions of clauses above.

\begin{clauses}

\clausex{finish}{}
Stops the loop and runs the epilogue code.

%for example:
%
%\begin{program}
%(iter (with answer = nil)
%         (initially (make-a-mess))
%         (for i from 1 to 10)
%         (when (correct? i)
%           (setq answer i)
%           (finish))
%         (finally (cleanup)))
%\end{program}
%
%this code will execute |cleanup| whether or not the test |(correct?
%i)| ever succeeds. 
%the (more elegant) formulation,
%\begin{program}
%(iter (initially (make-a-mess))
%         (for i from 1 to 10)
%         (finding i such-that (correct? i))
%         (finally (cleanup)))
%\end{program}
%would not execute |cleanup| if |(correct? i)| succeeded; it
%would do an immediate return.

\clausex{leave}{\opt\ ~value~}
Immediately returns ~value~ (default \nil) from the current \iter\
form, skipping the epilogue code.  Equivalent to using |return-from|.

\clausex{next-iteration}{}
Skips the remainder of the loop body and begins the next iteration of
the loop.

\clausex{while}{~expr~}
If ~expr~ ever evaluates to \nil, the loop is terminated and the
epilogue code executed.  Equivalent to |(if (not ~expr~) (finish))|.

\clausex{until}{~expr~}
Equivalent to |(if ~expr~ (finish))|.

\clausex{if-first-time}{~then~ \opt\ ~else~}
If this clause is being executed for the first time in this invocation
of the \iter\ form, then the ~then~ code is evaluated; otherwise the
~else~ code is evaluated.

\cpar |(for ~var~ first ~expr1~ then ~expr2~)| is almost equivalent to
\begin{program}
(if-first-time (dsetq ~var~ ~expr1~)
               (dsetq ~var~ ~expr2~))
\end{program}
The only difference is that the |for| version makes ~var~ available
for use with |for\dots previous|.

\end{clauses}

\subsection{Code Placement}
When fine control is desired over where code appears in a loop
generated by \iter, the following special clauses may be useful.
They are all subject to code-motion problems (see section
\ref{code-movement}).  

\begin{clauses}

\clausex{initially}{|\&rest| ~forms~}
The lisp ~forms~ are placed in the prologue section of the loop, where
they are executed once, before the loop body is entered.

\clausex{after-each}{|\&rest| ~forms~}
The ~forms~ are placed at the end of the loop body, where they
are executed after each iteration.  Unlike the other clauses in this
section, ~forms~ may contain \iter\ clauses.

\clausex{else}{|\&rest| ~forms~}
The lisp ~forms~ are placed in the epilogue section of the loop, where they
are executed if this |else| clause is never met during execution of the
loop and the loop terminates normally.

\clausex{finally}{|\&rest| ~forms~}
The lisp ~forms~ are placed in the epilogue section of the loop, where
they are executed after the loop has terminated normally.

\clausex{finally-protected}{|\&rest| ~forms~}
The lisp ~forms~ are placed in the second form of an unwind-protect
outside the loop.  They are always executed after the loop has
terminated, regardless of how the termination occurred.

\end{clauses}


%%%%%%%%%%%%%%%%%%%%%%%%%%%%%%%%%%%%%%%%%%%%%%%%%%%%%%%%%%%%%%%%
\section{Other Features}

\subsection{Multiple Accumulations}
\label{multiple}

\begin{sloppypar}
It is permitted to have more than one clause accumulate into the same
variable, as in the following:
\begin{program}
(iter (for i from 1 to 10)
      (collect i into nums)
      (collect (sqrt i) into nums)
      (finally (return nums)))
\end{program}
Clauses can only accumulate into the same variable if they are
compatible.  |collect|, |adjoining|, |appending|, |nconcing|,
|unioning| and |nunioning| are compatible with each other; |sum|,
|multiply| and |counting| are compatible; |always| and |never| are compatible;
|finding| \dots |such-that| is compatible with either |thereis| or |always|
and |never| when their defaults match; and |maximize| and |minimize| clauses
are compatible only with other |maximize| and |minimize| clauses,
respectively.

%note that the same variable ~cannot~ be both an accumulation
%variable and an ordinary variable; there can be only one variable with
%a given name within an \iter\ form. 
\end{sloppypar}

\subsection{Named Blocks}

Like Common Lisp |block|s, \iter\ forms can be given names.  The
name should be a single symbol, and it must be the first form in the
\iter.  The generated code behaves exactly like a named block; in
particular, |(return-from ~name~)| can be used to exit it:
\begin{program}
(iter fred
      (for i from 1 to 10)
      (iter barney
            (for j from i to 10)
            (if (> (* i j) 17)
                (return-from fred j))))
\end{program}
An \iter\ form that is not given a name is implicitly named \nil.

Sometimes one would like to write an expression in an inner \iter\ form,
but have it processed by an outer \iter\ form.  This is possible with
the |in| clause.  

\begin{clauses}

\clausex{in}{~name~ |\&rest| ~forms~}
Evaluates ~forms~ as if they were part of the \iter\ form named
~name~.  In other words, \iter\ clauses are processed by the \iter\
form named ~name,~ and not by any \iter\ forms that occur inside ~name.~

\cpar As an example, consider the problem of collecting a list of the
elements in a two-dimensional array.  The naive solution,
\begin{program}
(iter (for i below (array-dimension ar 0))
      (iter (for j below (array-dimension ar 1))
            (collect (aref ar i j))))
\end{program}
\noindent is wrong because the list created by the inner \iter\ is simply
ignored by the outer one.  But using |in| we can write:
\begin{program}
(iter outer (for i below (array-dimension ar 0))
      (iter (for j below (array-dimension ar 1))
            (in outer (collect (aref ar i j)))))
\end{program}
\noindent which has the desired result.

\end{clauses}

\subsection{Destructuring}
\label{destructuring}

In many places within \iter\ clauses where a variable is expected, a
list can be written instead.  In these cases, the value to be assigned
is ~destructured~ according to the pattern
described by the list.  As a simple example, the clause
\begin{program}
(for (key . item) in alist)
\end{program}
\noindent will result in |key| being set to the |car| of
each element in |alist|, and |item| being set to the |cdr|.  The
pattern list may be nested to arbitrary depth, and (as the example
shows) need not be terminated with \nil; the only requirement is that
each leaf be a bindable symbol (or \nil, in which case no binding is
generated for that piece of the structure).

Sometimes, you might like to do the equivalent of a
|multiple-value-setq| in a clause.  This 
``multiple-value destructuring'' can be expressed by writing 
\linebreak |(values $pat_1$ $pat_2 \ldots$)| for a destructuring
pattern, as in 
\begin{program}
(for (values (a . b) c d) = (three-valued-function ...))
\end{program}
\begin{sloppypar}
Note that the $pat_i$ can themselves be destructuring patterns (though
not multiple-value destructuring patterns).  You can't do multiple-value
destructuring in a |with| clause; instead wrap the whole \iter\
form in a |multiple-value-bind|.
\end{sloppypar}

\begin{note}{Rationale}
There are subtle interactions between variable declarations and
evaluation order that make the correct implementation of
multiple-value destructuring in a |with| somewhat tricky.
\end{note}

The destructuring feature of \iter\ is available as a separate
mechanism, using the |dsetq| macro:

\begin{clauses}
\defmacro{dsetq}{template expr}
Performs destructuring of ~expr~ using ~template~.   May be used
outside of an \iter\ form. Yields the primary value of ~expr~.

\end{clauses}

\subsection{On-line Help}

\begin{sloppypar}
There is a limited facility for on-line help, in the form of the 
|display-iterate-clauses| function.
\end{sloppypar}

\begin{clauses}

\defun{display-iterate-clauses}{\opt\ clause-spec}
Displays a list of \iter\ clauses.  If ~clause-spec~ is not
provided, all clauses are shown; if it is a symbol, all clauses
beginning with that symbol are shown; and if it is a list of symbols,
all clauses for which ~clause-spec~ is a prefix are shown.

\end{clauses}

\subsection{Parallel Binding and Stepping}
\label{bindings}

The parallel binding and stepping of variables is a feature that
\iter\ does ~not~ have.  This section attempts to provide a rationale.

We say that two variables are bound ~in parallel~ if neither
binding shadows the other.  This is the usual semantics of |let|
(as opposed to |let*|).  Similarly, we can say that iteration
variables are stepped in parallel if neither variable is updated
before the other, conceptually speaking; in other words, if the code
to update each variable can reference the old values of both variables.

|loop| allows parallel binding of variables and parallel stepping of
driver variables.  My view is that if you are depending on the 
serial/parallel distinction, you are doing something obscure.  If you
need to bind 
variables in parallel using |with|, then you must be using a
variable name that shadows a name in the existing lexical environment.
Don't do that.  The most common use for parallel stepping is to track
the values of variables on the previous iteration, but in fact this
does not require parallel stepping at all; the following will work:
\begin{program}
(iter (for current in list)
      (for prev previous current)
      ...)
\end{program}


%%%%%%%%%%%%%%%%%%%%%%%%%%%%%%%%%%%%%%%%%%%%%%%%%%%%%%%%%%%%%%%%
\section{Types and Declarations}
\label{types}

\subsection{Discussion}

Sometimes efficiency dictates that the types of variables be declared.
This type information needs to be communicated to \iter\ so it can
bind variables to appropriate values.  Furthermore, \iter\ must often
generate internal variables invisible to the user; there needs to be a
way for these to be declared.

As an example, consider this code, which will return the number of
odd elements in |number-list|:
\begin{program}
(iter (for el in number-list)
      (count (oddp el)))
\end{program}
In processing this form,
\iter\ will create an internal variable, let us call it |list17|, to
hold the successive |cdr|s of |number-list|, and  
will bind the variable to |number-list|.  It will also generate a
default binding for |el|; only inside the body of the loop will |el|
be set to the |car| of |list17|.  Finally, \iter\ will generate a
variable, call it |result|, to hold the result of the count, and will
bind it to zero.

When dealing with type declarations, \iter\ observes one simple rule:
~it will never generate a declaration unless requested to do so.~  The
reason is that such declarations might mask errors in compiled code by
avoiding error-checks; the resulting problems would be doubly hard to
track down because the declarations would be hidden from the
programmer.  Of course, a compiler might omit error-checks even in the
absence of declarations, though this behavior can usually be avoided,
e.g. by saying |(declaim (optimize (safety 3)))|.

So, the above \iter\ form will generate code with no declarations.
But say we wish to declare the types of |el| and the internal
variables |list17| and |result|.  How is this done?

Declaring the type of |el| is easy, since the programmer knows
the variable's name:
\begin{program}
(iter (for el in number-list)
      (declare (fixnum el))
      (counting (oddp el)))
\end{program}
\iter\ can read variable type declarations like this one.  Before
processing any clauses, it scans the entire top-level form for type
declarations 
and records the types, so that variable bindings can be performed
correctly.  In this case, |el| will be bound to zero
instead of \nil.  Also, \iter\ collects all the top-level declarations
and puts them at the begining of the generated code, so it is not
necessary to place all declarations at the beginning of an \iter\
form; instead, they can be written near the variables whose types they
declare.  

Since \iter\ is not part of the compiler, it will not know
about declarations that occur outside an \iter\ form; these
declarations must be repeated inside the form.

Here is another way we could have declared the type of |el|:
\begin{program}
(iter (for (the fixnum el) in number-list)
      (counting (oddp el)))
\end{program}
\iter\ extends the Common Lisp |the|\lispindex{the}
form to apply to variables as well as value-producing forms; anywhere
a variable is allowed---in a |with| clause, as the iteration
variable in a driver clause, as the |into| argument of an
accumulation clause, even inside a destructuring template---you can
write |(the ~type~ ~symbol~)| instead.

There is one crucial difference between using a |the| form and
actually declaring the variable: explicit declarations are always
placed in the generated code, but type information from a |the|
form is not turned into an actual declaration unless you tell \iter\
to do so using |iterate:declare-variables|.  See below.

\begin{sloppypar}
Declaring the types of internal variables is harder than declaring the
types of explicitly mentioned variables, since their names
are unknown.  You do it by declaring |iterate:declare-variables|
somewhere inside the top level of the \iter\ form.  (This will also
generate declarations for variables declared using |the|.)
\iter\ does not provide much selectivity here: it's all or none. 
And unfortunately, since \iter\ is not privy to compiler information
but instead reads declarations itself, it will not hear if you
|(declaim (iterate:declare-variables))|.  Instead, set the variable 
|iterate::*always-declare-variables*| to |t| at
compile-time, using |eval-when|.
\end{sloppypar}

To determine the appropriate types for internal variables, \iter\ uses
three sources of information:

\begin{itemize}

\item Often, the particular clause dictates a certain type for a
variable; \iter\ will use this information when available.  In the
current example, the variable |list17| will be given the type
|list|, since that is the only type that makes sense; and the
variable |result| will be given the type |fixnum|, on the
assumption that you will not be counting high enough to need bignums.
You can override this assumption only by using and explicitly declaring a
variable: 
\begin{verbatim}
(iter (declare (iterate:declare-variables))
      (for el in number-list)
      (count (oddp el) into my-result)
      (declare (integer my-result))
      (finally (return my-result)))
\end{verbatim}

\begin{sloppypar}
Other examples of the type assumptions that \iter\ makes are: type
|list| for |into| variables of collection clauses; type |list| for
expressions that are to be destructured; type |vector| for the
variable holding the vector in a |for\dots in-vector| clause, and
similarly for |string| and the |for\dots in-string| clause;
and the implementation-dependent type
|(type-of array-dimension-limit)| for the index and limit
variables generated by sequence iteration drivers like |for\dots
in-vector| and |for\dots in-string| (but not |for\dots in-sequence|,
because it may be used to iterate over a list). 
\end{sloppypar}

\item Sometimes, \iter\ will examine expressions and try to determine
their types in a simple-minded way.  If the expression is
self-evaluating (like a number, for instance), \iter\ knows that the
expression's type is the same as the type of the value it denotes, so
it can use that type.  If the expression is of the form |(the ~type~
~expr~)|, \iter\ is smart enough to extract ~type~ and use it.
However, the current version of \iter\ does not  
examine declarations of function result types or do any type
inference.  It will not determine, for 
example, that the type of |(+ 3 4)| is |fixnum|, or even
|number|.

\item In some cases, the type of an internal variable should match the
type of some other variable.  For instance, \iter\ generates an
internal variable for |(f x)| in the
clause |(for i from 1 to (f x))|, and in the absence of other
information will give it the same type as |i|.  If, however, the
expression had been written |(the fixnum (f x))|, then \iter\
would have given the internal variable the type |fixnum|
regardless of |i|'s type.  The type incompatibility errors that
could arise in this situation are not checked for.

\end{itemize}

Note that if you do declare |iterate:declare-variables|, then
\iter\ may declare user variables as well as internal ones if they do
not already have declarations, though only for variables that it
binds.  For instance, in this code:

\begin{program}
(iter (declare (iterate:declare-variables))
      (for i from 1 to 10)
      (collect i into var))
\end{program}
the variable |var| will be declared to be of type |list|.


\subsection{Summary}

\iter\ understands standard Common Lisp variable type declarations
that occur within an \iter\ form and
will pass them through to the generated code.  If the declaration
|(iterate:declare-variables)|\lispindex{declare-variables}
appears at the top level of an 
\iter\ form, or if 
|iterate::*always-declare-variables*|\lispindex{*always-declare-variables*} 
is \nonnil, then \iter\ will use the type information gleaned from user
declarations, self-evaluating expressions and |the| expressions,
combined with reasonable assumptions, to determine variable
types and declare them.


%%%%%%%%%%%%%%%%%%%%%%%%%%%%%%%%%%%%%%%%%%%%%%%%%%%%%%%%%%%%%%%%
\section{Problems with Code Movement}
\label{code-movement}

Some \iter\ clauses, or parts of clauses, result in code being
moved from the location of the clause to other parts of the loop.
Drivers behave this way, as do code-placement clauses like |initially|
and |finally|.  When using these clauses, there is a danger of writing
an expression that makes sense in its apparent location but will be
invalid or have a different meaning in another location.  For example:
\begin{program}
(iter (for i from 1 to 10)
      (let ((x 3))
        (initially (setq x 4))))
\end{program}
While it may appear that the |x| of |(initially (setq x 4))| is the
same as the |x| of |(let ((x 3)) \dots|, in fact they are not:
|initially| moves its code outside the loop body, so |x| would refer
to a global variable.  Here is another example of the same problem:
\begin{program}
(iter (for i from 1 to 10)
      (let ((x 3))
        (collect i into x)))
\end{program}
If this code were executed, |collect| would create a binding for its
|x| at the top level of the \iter\ form that the |let| will shadow.

Happily, \iter\ is smart enough to catch these errors; it
walks all problematical code to ensure that free variables are not
bound inside the loop body, and checks all variables it binds for the
same problem.

However, some errors cannot be caught:

\begin{program}
(iter (with x = 3)
      (for el in list)
      (setq x 1)
      (reducing el by \#'+ initial-value x))
\end{program}
|reducing| moves its |initial-value| argument to the initialization
part of the loop in order to produce more efficient code.  Since
\iter\ does not perform data-flow analysis, it cannot determine that
|x| is changed inside the loop; all it can establish is that |x| is
not bound internally.  Hence this code will not signal an
error and will use $3$ as the initial value of the reduction.

The following list summarizes all cases that are subject to these code
motion and variable-shadowing problems.
\begin{itemize}
\item Any variable for which \iter\ creates a binding, including those
used in |with| and the |into| keyword of many clauses.

\begin{sloppypar}
\item The special clauses which place code: |initially|, |after-each|, |else|,
|finally| and |finally-protected|. 
\end{sloppypar}

\item The variables of a |next| or |do-next| form.

\item The |initially| arguments of |for\dots initially\dots then| and
|for\dots previous|. 

\item The |then| argument of |for\dots initially\dots then|.

\item The |initial-value| arguments of |reducing| and |accumulate|.

\item The |on-failure| argument of |finding\dots such-that|.
\end{itemize}

%%%%%%%%%%%%%%%%%%%%%%%%%%%%%%%%%%%%%%%%%%%%%%%%%%%%%%%%%%%%%%%%
\section{Differences Between {\tt Iterate} and {\tt Loop}}
|loop| contains a great deal of complexity which \iter\ tries to
avoid.  Hence many esoteric features of |loop| don't exist in \iter.
Other features have been carried over, but in a cleaned-up form.  
And of course, many new features have been added; they are not
mentioned in this list.

\begin{itemize}

\item \iter's syntax is more Lisp-like than |loop|'s, having a higher
density of parens.

\item The current implementation of \iter, unlike the current version
of  |loop| (as documented in {\em Common Lisp, 2nd Ed.\/}), is
extensible (see section \ref{extend}).  

\item |loop| puts the updates of all driver variables at the top of
the loop; \iter\ leaves them where the driver clauses appear.

\item While for the most part \iter\ clauses that resemble |loop| clauses
behave similarly, there are some differences.  For instance, there is
no |for\dots =\dots then| in \iter; instead use 
|for\dots initially\dots then|.  

\item |loop| binds the variable |it| at certain times
to allow pseudo-English expressions like |when ~expr~ return it|. 
In \iter, you must bind ~expr~ to a variable yourself.  Note that
|when ~expr~ return it| is like |thereis ~expr~| except that the latter is an
accumulation clause and therefore competes with other accumulations
(remember section \ref{multiple} above).

% repeat different behaviour of |always| clause here?

\item |loop| has a special |return| clause, illustrated in the
previous item.  \iter\ doesn't need one,  since an ordinary Lisp
|return| has the same effect.

\item |loop| allows for parallel binding and stepping of iteration
variables.  \iter\ does not.  (See section \ref{bindings}.)

\item |loop| and \iter\ handle variable type declarations very
differently.  |loop| provides a special syntax for declaring variable
types, and does not examine declarations.  Moreover, the standard
implementation of |loop| will
generate declarations when none are requested.
\iter\ parses standard Common Lisp type declarations, and will never
declare a variable itself unless declarations are 
specifically requested.

\end{itemize}

%%%%%%%%%%%%%%%%%%%%%%%%%%%%%%%%%%%%%%%%%%%%%%%%%%%%%%%%%%%%%%%%
\section{Rolling Your Own}
\label{extend}

\subsection{Introduction}

\iter\ is extensible---you can write new clauses that embody new
iteration patterns.  You might want to write a new driver clause for a
data structure of your own, or you might want to write a clause that
collects or manipulates elements in a way not provided by \iter. 

This section describes how to write clauses for \iter.  Writing a
clause is like writing a macro.  In fact, writing a clause ~is~
writing a macro: since \iter\ code-walks its body and macroexpands,
you can add new abstractions to \iter\ with good old |defmacro|.

Actually, there are two extensions you can make to \iter\ that are
even easier than writing a macro.  They are adding a synonym for an
existing clause and defining a driver clause for an indexable
sequence.  These can be done with |defsynonym| and
|defclause-sequence|, respectively.  See section \ref{aids}, below.

The rest of this section explains how to write macros that expand into
\iter\ clauses.
Here's how you could add a simplified version of \iter's |multiply|
clause, if \iter\ didn't already have one:
\begin{program}
(defmacro multiply (expr)
  `(reducing ,expr by \#'* initial-value 1))
\end{program}

If you found yourself summing the square of an expression often, you
might want to write a macro for that.  A first cut might be
\begin{program}
(defmacro sum-of-squares (expr)
  `(sum (* ,expr ,expr)))
\end{program}
but if you are an experienced macro writer, you will realize that this
code will evaluate ~expr~ twice, which is probably a bad idea.  A
better version would use a temporary:
\begin{program}
(defmacro sum-of-squares (expr)
  (let ((temp (gensym)))
    `(let ((,temp ,expr))
       (sum (* ,temp ,temp)))))
\end{program}
Although this may seem complex, it is just the sort of thing you'd
have to go through to write any macro, which illustrates the point of
this section: if you can write macros, you can extend \iter.

Our macros don't use \iter's keyword-argument
syntax.  We could just use keywords with |defmacro|, but we would
still not be using \iter's clause indexing mechanism.  Unlike Lisp,
which uses just the first symbol of a form to determine what function
to call, \iter\ individuates clauses by the list of required keywords.
For instance, |for\dots in| and |for\dots in-vector| are different clauses
implemented by distinct Lisp functions.  

To buy into this indexing scheme, as well as the keyword-argument
syntax, use |defmacro-clause|:

\begin{clauses}

\defmacro{defmacro-clause}{arglist |\&body| body}
Defines a new \iter\ clause.  ~arglist~ is a list of symbols which are
alternating keywords and arguments.  \opt\ may be used, and the list
may be terminated by |\&sequence|.  ~body~ is an ordinary macro body,
as with |defmacro|.  If the first form of ~body~ is a string, it is
considered a documentation string and will be shown by
|display-iterate-clauses|. |defmacro-clause| will signal an error if 
defining the clause would result in an ambiguity.  E.g. you cannot
define the clause |for\dots from| because there would be no way to
distinguish it from a use of the |for| clause with optional keyword |from|.

\end{clauses}

\medskip

Here is |multiply| using |defmacro-clause|.  The keywords are capitalized
for readability.
\begin{program}
(defmacro-clause (MULTIPLY expr \opt\ INTO var)
  `(reducing ,expr by \#'* into ,var initial-value 1))
\end{program}
You don't have to worry about the case when |var| is not supplied; for
any clause with an |into| keyword, saying |into nil| is equivalent to
omitting the |into| entirely.

As another, more extended example, consider the fairly common
iteration pattern that 
involves finding the sequence element that maximizes (or minimizes) some
function.  \iter\ provides this as |finding\dots maximizing|, but it's
instructive to see how to write it.
Here, in pseudocode, is how you might write such a loop for
maximizing a function F:

\begin{quote}
\begin{tabbing}
set variable MAX-VAL to NIL; \\
set variable WINNER to NIL; \\
for\=\ each element EL in the sequence \\
\>  if\=\ MAX-VAL\=\ is NIL or F(EL) $>$ MAX-VAL then \\
\>\>    set MAX-VAL to F(EL); \\
\>\>    set WINNER to EL; \\
\>  end if; \\
end for; \\
return WINNER.
\end{tabbing}
\end{quote}

Here is the macro:
\begin{program}
(defmacro-clause (FINDING expr MAXIMIZING func \opt\ INTO var)
  (let ((max-val (gensym))
        (temp1 (gensym))
        (temp2 (gensym))
        (winner (or var iterate::*result-var*)))
    `(progn 
       (with ,max-val = nil)
       (with ,winner = nil)
       (cond 
        ((null ,max-val)
         (setq ,winner ,expr)
         (setq ,max-val (funcall ,func ,winner))
        (t
         (let* ((,temp1 ,expr)
                (,temp2 (funcall ,func ,temp1)))
           (when (> ,temp2 ,max-val)
             (setq ,max-val ,temp2)
             (setq ,winner ,temp1))))))
       (finally (leave ,winner)))))
\end{program}
Note that if no |into| variable is supplied, we use
|iterate::*result-var*|, which contains the internal variable into
which all clauses place their results.  If this variable is bound by
some clause, then \iter\ will return its value automatically;
otherwise, \nil\ will be returned.  

\subsection{Writing Drivers}

In principle, drivers can be implemented just as easily as other
\iter\ clauses.  In practice, they are a little harder to get right. 
As an example, consider writing a driver that
iterates over all the 
elements of a vector, ignoring its fill-pointer.  |for\dots in-vector|
won't work for this, because it observes the fill-pointer.  It's necessary to
use |array-dimension| instead of |length| to obtain the size of the
vector.  Here is one approach:

\begin{program}
(defmacro-clause (FOR var IN-WHOLE-VECTOR v)
  "All the elements of a vector (disregards fill-pointer)"
  (let ((vect (gensym))
        (index (gensym)))
    `(progn
       (with ,vect = ,v)
       (for ,index from 0 below (array-dimension ,vect 0))
       (for ,var = (aref ,vect ,index)))))
\end{program}
Note that we immediately put |v| in a variable, in case it is an
expression.  Again, this is just good Lisp macrology.  It also has a
subtle effect on the semantics of the driver: |v| is evaluated only
once, at the beginning of the loop, so changes to |v| in the loop have
no effect on the driver.  Similarly, the bounds for numerical iteration
e.g. the above |array-dimension| are also evaluated once only.  This is how
all of \iter's drivers work.


There is an important point concerning the |progn| in this code.  We
need the |progn|, of course, because we are returning several forms,
one of which is a driver.  But \iter\ drivers must occur at top-level.
Is this code in error?  No, because ~top-level~ is defined in \iter\
to include forms inside a |progn|.  This is just the definition of
top-level that Common Lisp uses, and for the same reason: to allow
macros to return multiple forms at top-level.

While our |for\dots in-whole-vector| clause will work, it is not
ideal.  In particular, it does not support generating.  Do do so, we
need to use |for\dots next| or |for\dots do-next|.  The job is
simplified by the |defmacro-driver| macro.

\begin{clauses}

\defmacro{defmacro-driver}{arglist |\&body| body}
Defines a driver clause in
both the |for| and |generate| forms, and provides a parameter
|generate| which ~body~ can examine to determine how it was invoked.
~arglist~ is as in |defmacro-clause|, and should begin with the symbol
|for|. 

\end{clauses}

With |defmacro-driver|, our driver looks like this:
\begin{program}
(defmacro-driver (FOR var IN-WHOLE-VECTOR v)
  "All the elements of a vector (disregards fill-pointer)"
   (let ((vect (gensym))
         (end (gensym))
         (index (gensym))
         (kwd (if generate 'generate 'for)))
     `(progn
        (with ,vect = ,v)
        (with ,end = (array-dimension ,vect 0))
        (with ,index = -1)
        (,kwd ,var next (progn (incf ,index)
                               (if (>= ,index ,end) (terminate))
                               (aref ,vect ,index))))))
\end{program}


We are still missing one thing: the |\&sequence| keywords.  
We can get them easily enough, by writing 
\begin{program}
(defmacro-driver (FOR var IN-WHOLE-VECTOR v \&sequence) 
  ...)
\end{program}
We can now refer to parameters |from|, |to|, |by|, etc. which contain
either the values for the corresponding keyword, or \nil\ if the
keyword was not supplied.  Implementing the right code for these
keywords is cumbersome but not difficult; it is left as an exercise.
But before you begin, see |defclause-sequence| below for an easier way.
       
\subsection{Extensibility Aids}
\label{aids}

This section documents assorted features that may be of use in
extending \iter.

\begin{clauses}

\defunexpvar{*result-var*}
Holds the variable that is used to return a value as a result of the
\iter\ form.  You may examine this and use it in a |with| clause, but
you should not change it.

\defmacro{defsynonym}{syn word}
 Makes ~syn~ a synonym for the existing \iter\ keyword ~word.~  Only
the first word in each clause can have synonyms. 

\defmacroxx{defclause-sequence}{element-name index-name}{|\&key|
access-fn size-fn sequence-type}{element-type
element-doc-string index-doc-string}  
Provides
a simple way to define sequence clauses.  Generates two
clauses, one for iterating over the sequence's elements, the other
for iterating over its indices.  The first symbol of both
clauses will have print-name |for|.
~element-name~  and ~index-name~ should be symbols.
~element-name~ is the second keyword of the element iterator (typically of
the form 
|in-~sequence-type~|), and ~index-name~ is the second keyword
of the index-iterator (typically of the form
|index-of-~sequence-type~|).  Either name may be 
\nil, in which case the corresponding clause is not defined.  If both
symbols are supplied, they should be in the same package.  The |for|
that begins the clauses will be in this package.

\cpar ~access-fn~ is the function to be used to
access elements of the sequence in the element iterator.  The function
should take two 
arguments, a sequence and an index, and return the appropriate element.
~size-fn~ should denote a function of one argument, a sequence, that
returns its size.  Both ~access-fn~ and ~size-fn~ are required for the
element iterator, but only ~size-fn~ is needed for the index iterator.

\cpar The ~sequence-type~ and ~element-type~ keywords are used to
suggest types for the variables 
used to hold the sequence and the
sequence elements, respectively.  The usual rules about \iter's
treatment of variable type declarations apply (see section \ref{types}).

\cpar ~element-doc-string~ and ~index-doc-string~ are
the documentation strings, for use with |display-iterate-clauses|. 

\cpar The generated element-iterator performs destructuring on the
element variable.

\cpar As an example, the above |for\dots in-whole-vector| example
could have been written:
\begin{program}
(defclause-sequence IN-WHOLE-VECTOR INDEX-OF-WHOLE-VECTOR
  :access-fn 'aref
  :size-fn \#'(lambda (v) (array-dimension v 0))
  :sequence-type 'vector
  :element-type t
  :element-doc-string 
     "Elements of a vector, disregarding fill-pointer"
  :index-doc-string 
     "Indices of vector, disregarding fill-pointer")
\end{program}

\end{clauses}

\subsection{Subtleties}

There are some subtleties to be aware of when writing \iter\ clauses.
First, the code returned by your macros may be |nconc|'ed into a list,
so you should always returned freshly consed lists, rather than
constants.  Second, \iter\ matches clauses by using |eq| on the first
symbol and |string=| on the subsequent ones, so the package of the
first symbol of a clause is relevant.  All of the clauses in this manual
have their first word in the \iter\ package.  
You can use the package system in the usual way to shadow
\iter\ clauses without replacing them.

%%% say more here, about the badness that only the first word of a
%%% clause is packagey.

\section{Non-portable Extensions to Iterate (Contribs)}
\label{contribs}

Currently, there is only one non-portable extension to iterate in the
distribution: iterate-pg. If you have made an extension that depends
on non-portable features, feel free to send them to |asf@boinkor.net|
for inclusion in the iterate distribution.

\subsection{An SQL query driver for iterate}

The pg package by Eric Marsden (see |http://cliki.net/pg|) provides an
interface to the PostgreSQL database. Using the \iterpg\ extension, it
is possible to handle the results of SQL queries using \iter.

This usage example should give you an idea of how to use it:

\begin{program}
(pg:with-pg-connection (c "somedb" "someuser")
  (iter (for (impl version date) in-relation "select * from version"
                                 on-connection *dbconn*)
        (collect version)))
\end{program}

To use the extension via |ASDF|, simply make your system depend on the
|iterate-pg| system instead of the |iterate| system. To load it
manually, use:

\begin{program}
  (asdf:oos 'asdf:load-op :iterate-pg)
\end{program}

%%%%%%%%%%%%%%%%%%%%%%%%%%%%%%%%%%%%%%%%%%%%%%%%%%%%%%%%%%%%%%%%
\section{Obtaining {\tt Iterate}}

\begin{sloppypar}
The information in this chapter is obsolete but included for
completeness's sake; Currently, the most up-to-date information on
\iter\ can be found at |http://boinkor.net/iterate.html|.
\end{sloppypar}

\begin{sloppypar}
\iter\ currently runs on Lisp Machines, and on
HP's, Sun3's and Sparcstations under Lucid.  
\iter\ source and binaries are available at the MIT AI Lab in the
subdirectories of |/src/local/lisplib/|.  The source file,
|iterate.lisp|, is also available for anonymous FTP in the directory
|/com/fpt/pub/| on the machine |TRIX.AI.MIT.EDU| (Internet number
128.52.32.6).  If you are unable to obtain |iterate| in one of these
ways, send mail to |jba@ai.mit.edu| and I will send you the source
file. 
\end{sloppypar}

\begin{sloppypar}
\iter\ resides in the |iterate| package (nickname |iter|).  Just say
\linebreak |(use-package :iterate)| to make all the necessary symbols
available. 
If a symbol is not exported, it appears in this manual with an
``|iterate::|'' prefix.
\end{sloppypar}

Send bug reports to |bug-iterate@ai.mit.edu|.  The |info-iterate|
mailing list will have notices of changes and problems; to have
yourself added, send mail to |info-iterate-request@ai.mit.edu|.


\medskip

\begin{flushleft}
 \bf Acknowledgements
\end{flushleft}
\smallskip

Richard Waters provided invaluable criticism which spurred me to improve
\iter\ greatly.  As early users, David Clemens, Oren Etzioni and Jeff
Siskind helped ferret out many bugs.

%\begin{theindex}
% The index files must be generated with the genindex program
% or from makeindex iterate-manual.idx which creates iterate-manual.ind.

%\baselineskip\normalbaselineskip
%\advance\baselineskip by -2pt

% Fix ``.'' spacing in lists
\documentstyle[12pt,aimemo]{article}

\setlength{\textwidth}{6.0in}
\setlength{\oddsidemargin}{0.25in}
\setlength{\evensidemargin}{0.25in}
\setlength{\topmargin}{0pt}
\setlength{\headsep}{24pt}
\setlength{\textheight}{8.5in}

\let\origpar=\par

% somehow, LaTex fucks with things so that the plain Tex \obeyspaces
% doesn't work.  This is the same, except we use an mbox containing
% a space instead of just a space.

\newcommand{\spc}{\mbox{ }}
\newcommand{\tAb}{\mbox{        }}
\def\obspaces{\catcode`\ =\active\catcode`\^^I=\active}
{\obspaces \global\let =\spc\global\let^^I=\tAb}

\newenvironment{program}{\medskip\samepage\tt\obeylines\obspaces}{\medskip}

% use | for \tt
\catcode`\|=13
\newif\ifvbar
\def|{\ifvbar
         \endgroup\vbarfalse
      \else 
         \begingroup\tt\vbartrue 
      \fi}



\newcommand{\lisp}{\tt}
\newcommand{\iter}{|iterate|}
\newcommand{\iterpg}{|iterate-pg|}
\newcommand{\setf}{|setf|}
\newcommand{\nil}{|nil|}
\newcommand{\nonnil}{non-\nil}
\newcommand{\opt}{{\lisp \&optional}}
\newcommand{\yields}{$\Rightarrow$}



\newbox\Dots
\setbox\Dots=\hbox{\small ...}
\renewcommand{\dots}{\copy\Dots}

% Here's a way to define an environment which will treat all
% paragraphs inside it in the same manner.  It's, as near as I can
% figure, the way that Latex does things.  Using \everypar won't work;
% apparently Latex resets that a lot.  What you have to do is: at the
% top of the file do \let\origpar=\par.  In the
% begin part of the environment, do the parshape (or whatever), then say
% \def\par{{\origpar}}.  The extra braces are necessary.   That's all
% you have to do; at the end of the environment things will return to
% normal automatically.  Why does this all work?  I have no idea.
% Another important thing: there should be a blank line (i.e. a \par)
% between the end of the last paragraph and the \end{environment}.

\newlength{\clindent}
\setlength{\clindent}{0.5in}
\newlength{\clparindent}
\setlength{\clparindent}{\parindent}

\newenvironment{clauses}{\advance\linewidth -\parindent
                        \hangindent\clindent\relax
                         \hangafter=0
                          \parindent=0pt
                          \def\par{{\origpar}}}{}

\newcommand{\cpar}{\hspace \clparindent}

\newcommand{\startitem}{\bigskip\pagebreak[3]}
\newcommand{\finishitem}{\smallskip}

\newcommand{\clause}[1]{\startitem\Clause{#1}\finishitem}
\newcommand{\Clause}[1]{\hbox{#1}}
\newcommand{\clindex}[1]{\index{{\lisp #1}}}
\newcommand{\clindexx}[2]{\index{{\lisp #1\dots #2}}}
\newcommand{\clindexxx}[3]{\index{{\lisp #1\dots #2\dots #3}}}
\newcommand{\clausex}[2]{\startitem\Clausex{#1}{#2}\finishitem}
\newcommand{\Clausex}[2]{\Clause{|#1| #2}\clindex{#1}}

\newcommand{\defvar}[1]{\startitem\deff{#1}{}{Variable}\finishitem}
\newcommand{\defunexpvar}[1]{\startitem\deffnoindex{iterate::#1}{}{Variable}
                                        \lispindex{#1}\finishitem}
\newcommand{\defun}[2]{\startitem\deff{#1}{#2}{Function}\finishitem}
\newcommand{\defmacro}[2]{\startitem\deff{#1}{#2}{Macro}\finishitem}
\newcommand{\defunx}[3]{\startitem\deffx{#1}{#2}{#3}{Function}\finishitem}
\newcommand{\defmacrox}[3]{\startitem\deffx{#1}{#2}{#3}{Macro}\finishitem}
\newcommand{\defunxx}[4]{\startitem\deffxx{#1}{#2}{#3}{#4}{Function}\finishitem}
\newcommand{\defmacroxx}[4]{\startitem\deffxx{#1}{#2}{#3}{#4}{Macro}\finishitem}

\newcommand{\deff}[3]{\deffnoindex{#1}{#2}{#3}\lispindex{#1}}
\newcommand{\deffnoindex}[3]{\hbox to \hsize{{\tt #1} {\it #2}\hfill [{\it #3}]}}
\newcommand{\deffx}[4]{\deff{#1}{#2}{#4}\moreargs{#1}{#3}}
\newcommand{\deffxx}[5]{\deffx{#1}{#2}{#3}{#5}\moreargs{#1}{#4}}

\newcommand{\moreargs}[2]{\hbox to \hsize{\phantom{\lisp #1} {\it #2}\hfill}}

\newcommand{\lispindex}[1]{\index{{\lisp #1}}}

 \newenvironment{note}[1]{\pagebreak[2]\bigskip
                          \hrule\smallskip\small
                         {\setlength{\parindent}{0pt}
                           \par{\bf #1:}}}{\smallskip\hrule\bigskip}



\makeindex

%%%%%%%%%%%%%%%%%%%%%%%%%%%%%%%%%%%%%%%%%%%%%%%%%%%%%%%%%%%%%%%%

%\thispagestyle{empty}

%\vspace*{2in}

%\title{The |Iterate| Manual}
%\author{Jonathan Amsterdam}

%\maketitle

%\thispagestyle{empty}

%\begin{center}
%\Large *** DRAFT ***
%\end{center}
%This manual is in beta-test. It will become an AI Memo in a couple of
%months.  I would appreciate any comments or
%corrections.  You can reach me at email address |jba@ai.ai.mit.edu|.

\begin{document}

\def\writtenby{Jonathan Amsterdam}
\def\contractno{This report describes research done at the Artificial
Intelligence Laboratory of the Massachusetts Institute of Technology.
Support for the laboratory's artificial intelligence research is
provided in part by the Advanced Research Projects Agency of the
Department of Defense under Office
of Naval Research contract N00014-85-K-0124.}

\AIMEMO{1236}{\Large The Iterate Manual \\[1in]}
{This is the manual for version 1.4 of \iter, a powerful iteration
macro for Common  
Lisp. \iter\ is similar to {\tt loop} but provides numerous additional
features, is well integrated with Lisp, and is extensible. 
\vfill }


% use ~ for \em
\catcode`\~=13
\newif\ifem
\def~{\ifem
         \/\endgroup\emfalse
      \else 
         \begingroup\em\emtrue 
      \fi}



%\pagebreak
%\setcounter{page}{1}

%\thispagestyle{empty}


\tableofcontents

\pagebreak
\pagestyle{headings}

\section{Introduction}

\begin{sloppypar}
This manual describes
\iter, a powerful iteration facility for Common Lisp.
\iter\ provides abstractions for many common iteration
patterns and allows for the definition of additional patterns.  \iter\ is
a macro that expands into ordinary Lisp at
compile-time, so it is more efficient than higher-order
functions like |map| and |reduce.|  
While it is similar to |loop|, \iter\ offers a more Lisp-like syntax
and enhanced extensibility.
(For a more complete
comparison of \iter\ with other iteration constructs, see MIT AI Lab Working
Paper No. 324, ~Don't Loop, Iterate.~)
\end{sloppypar}

An \iter\ form consists of the symbol |iter|\footnote{You can also use
|iterate|, but |iter| is preferred because it avoids potential conflicts with
possible future additions to Common Lisp, and because it saves
horizontal space when writing code.}
followed by one or more
forms, some of which may be \iter\ 
~clauses.~  Here is a simple example of \iter\ which collects the
numbers from 1 to 10 into a list, and returns the list.  The return
value is shown following the arrow.
\begin{program}
(iter (for i from 1 to 10)
      (collect i))          \yields (1 2 3 4 5 6 7 8 9 10)
\end{program}
This form contains two clauses: a |for| clause that steps the
variable |i| over the integers from 1 to 10, and a |collect|
clause that accumulates its argument into a list.  With a few
exceptions, all \iter\ clauses have the same format: alternating
symbols (called ~keywords~) and expressions (called
~arguments~).  The syntax and terminology are those of Common
Lisp's keyword lambda lists.  One difference is that \iter's keywords
do not have to begin with a colon---though they may, except for the
first symbol of a clause.  So you can also write  |(for i :from
1 :to 10)| if you prefer.

Any Lisp form can appear in the body of an \iter, where it will have
its usual meaning.  \iter\ walks the entire body, expanding macros,
and recognizing clauses at any level.  This example collects all the
odd numbers in a list:
\begin{program}
(iter (for el in list)
      (if (and (numberp el) (oddp el))
          (collect el)))
\end{program}

There are clauses for iterating over numbers, lists, arrays and other
objects, and for
collecting, summing, counting, maximizing and other useful operations.
\iter\ also supports the creation of new variable bindings, stepping
over multiple sequences at once, destructuring, and compiler
declarations of variable types.  The following example illustrates
some of these features: 
\begin{program}
(iter (for (key . item) in alist)
         (for i from 0)
         (declare (fixnum i))
         (collect (cons i key)))
\end{program}

This loop takes the keys of an alist and returns a new alist
associating the keys with their positions in the original list. The
compiler declaration for |i| will appear in the generated code
in the appropriate place.

%%%%%%%%%%%%%%%%%%%%%%%%%%%%%%%%%%%%%%%%%%%%%%%%%%%%%%%%%%%%%%%%
\section{Clauses}

Most of \iter's clauses will be familiar to |loop| programmers.  
(|loop| is an iteration macro that has been incorporated into Common
Lisp.  Seen Guy Steele's {\em Common Lisp, 2nd Edition}.)
In
nearly all cases they behave the same as their |loop| counterparts, so
a |loop| user can switch to \iter\ with little pain (and much gain).

All clauses with the standard keyword-argument syntax consist of two
parts: a ~required~ part, 
containing keywords that must be present and in the right order; and
an ~optional~ part, containing keywords that may be omitted and,
if present, may occur in any order.  In the descriptions below, the
parts are separated by the Lisp lambda-list keyword \opt.


\subsection{Drivers}

An iteration-driving clause
conceptually causes the iteration to go forward.  Driver clauses in
\iter\ allow iteration over numbers, lists, vectors, hashtables, packages,
files and streams.  Iteration-driving clauses must
appear at the top level of an \iter\ form; they cannot be nested
inside another clause.  The driver variable is updated at the point
where the driver clause occurs.  Before the clause is executed for the
first time, the value of the variable is undefined.

%Also, regardless of where the driver clause
%appears in the body, the driver variable is stepped at the top of the
%loop; hence it is stylistically preferable, though
%not required, to place driver clauses at the beginning of the \iter.

Multiple drivers may appear in a single \iter\ form, in which case all
of the driver variables are updated each time through the loop, in the
order in which the clauses appear.  The first driver to terminate will
terminate the entire loop.

In all cases, the value of the driver variable on exit from the loop,
including within the epilogue code (see the |finally| clause), is
undefined.

All the parameters of a driver clause are evaluated once, before the
loop begins.  Hence it is not possible to change the bounds or other
properties of an iteration by side-effect from within the loop.

With one exception, driver clauses begin with the word |for| (or
the synonym |as|) and mention an iteration variable, which is
given a binding within the \iter\ form.
The exception is |repeat|, which just executes a loop a
specified number of times:

\begin{clauses}

\clausex{repeat}{~n~}
Repeats the loop ~n~ times.  For example:
\begin{program}
(iter (repeat 100)
      (print "I will not talk in class."))
\end{program}
\cpar If $n \leq 0$, the
loop will never be executed.  If ~n~ is not an integer, the
actual number of executions will be $\lceil n \rceil$.

\end{clauses}

\pagebreak[3]

\subsubsection{Numerical Iteration}

\begin{clauses}
\clausex{for}{~var~ |\&sequence|} 
The general form for iterating over a sequence of numbers requires a
variable and, optionally, one or more keywords that provide the bounds
and step size of the iteration. The |\&sequence|
lambda-list keyword is a shorthand for these sequence keywords.  They are:
|from|, |upfrom|, |downfrom|, |to|, |downto|, |above|, |below| and |by|.
|from| provides the starting value for ~var~
and defaults to zero. |to| provides a final value and implies that the
successive values of ~var~ will be increasing; |downto| implies that
they will be decreasing.  The loop terminates when ~var~ passes
the final value (i.e. becomes smaller or larger than it, depending on
the direction of iteration); in other  words, the loop body
will never be 
executed for values of ~var~ past the final value.
|below| and |above| are similar to |to| and
|downto|, except that the loop terminates when ~var~ equals or passes
the final value.

\cpar If no final value is specified, the variable will be stepped
forever.  Using |from| or |upfrom| will result in increasing
values, while |downfrom| will give decreasing values.

\cpar On each iteration,
~var~ is incremented or decremented by the value of the sequence
keyword |by|, which defaults to 1.  It should always be a positive
number, even for downward iterations.

\cpar In the following examples, the sequence of numbers generated is shown
next to the clause.

\begin{program}
(for i upfrom 0) \yields\ 0 1 2 \ldots
(for i from 5)   \yields\ 5 6 7 \ldots    ; either from or upfrom is okay
(for i downfrom 0) \yields\ 0 -1 -2 \ldots
(for i from 1 to 3) \yields\ 1 2 3
(for i from 1 below 3) \yields\ 1 2
(for i from 1 to 3 by 2) \yields\ 1 3
(for i from 1 below 3 by 2) \yields\ 1
(for i from 5 downto 3) \yields\ 5 4 3
\end{program}

\end{clauses}

\subsubsection{Sequence Iteration}

There are a number of clauses for iterating over sequences.  In all of
them, the argument following |for| may be a list instead of a
symbol, in which case destructuring is performed.  See section
\ref{destructuring}. 

\begin{clauses}

\clause{|for| ~var~ |in| ~list~ \opt\ |by| ~step-function~}
\clindexx{for}{in}
~var~ is set to successive elements of list.  
~step-function,~ which 
defaults to |cdr|, is used to obtain the next sublist. 

\clause{|for| ~var~ |on| ~list~ \opt\ |by| ~step-function~}
\clindexx{for}{on}
~var~ is set to successive sublists of list. 
~step-function~ (default |cdr|) is used as in |for\dots in|.

\end{clauses}

\medskip

These two clauses use |atom| to test for the end
of a list.  Hence, given a list whose final |cdr| is not \nil,
they will silently ignore the last |cdr|.  Other choices are
|endp|,  which would signal an error, and |null|, which
would probably result in an error somewhere else.  If you wish to use
an end-test other than |atom|, set the variable
|iterate::*list-end-test*|\lispindex{*list-end-test*} to the name of
the desired function. 

\begin{clauses}
\clause{|for| ~var~ |in-vector| ~vector~ |\&sequence|}
\clindexx{for}{in-vector}
~var~ takes on successive elements from ~vector.~   The vector's
fill-pointer is observed.  Here and in subsequent clauses, the |\&sequence|
keywords include |with-index|, which takes a symbol as argument
and uses it for the index variable instead of an internally generated
symbol.  The other |\&sequence| keywords behave as in numerical
iteration, except that the default iteration bounds are the bounds of
the vector.  E.g. in |(for i in-vector v downto 3)|,
|i| will start off being bound to the last element in |v|,
and will  
be set to preceding elements down to and including the element with
index 3.

\clause{|for| ~var~ |in-sequence| ~seq~ |\&sequence|}
\clindexx{for}{in-sequence}
This uses Common Lisp's generalized sequence functions, |elt|
and |length|, to obtain elements and determine the length of
~seq.~   Hence it will work for any sequence, including lists, and will
observe the fill-pointers of vectors.

\clause{|for| ~var~ |in-string| ~string~ |\&sequence|}
\clindexx{for}{in-string}
~var~ is set to successive characters of ~string.~

\startitem
\Clause{|for| ~var~ |index-of-vector| ~vector~ |\&sequence|}
\clindexx{for}{index-of-vector}
\Clause{|for| ~var~ |index-of-sequence| ~sequence~ |\&sequence|}
\clindexx{for}{index-of-sequence}
\Clause{|for| ~var~ |index-of-string| ~string~ |\&sequence|}
\clindexx{for}{index-of-string}
\finishitem
~var~ is set to successive indices of the sequence.
These clauses avoid the overhead of accessing the sequence elements
for those applications where they do not need to be examined, or are
examined rarely.  They admit all the optional keywords of the other
sequence drivers except the (redundant) |with-index| keyword.

\clause{|for| ~(key value)~ |in-hashtable| ~table~}
\clindexx{for}{in-hashtable}

~key~ and ~value,~ which must appear as shown in a list
and may be destructuring templates, are set to the keys and values of
~table~.  If ~key~ is |nil|, then the hashtable's keys will be ignored;
similarly for ~value~.
The order in which 
elements of ~table~ will be retrieved is unpredictable.

\clause{|for| ~var~ |in-package| ~package~ \opt\ |external-only| ~ext~}
\clindexx{for}{in-package}
Iterates over all the symbols in ~package,~ or over only the
external symbols if ~ext~ is specified and non-|nil|.  ~ext~ is not
evaluated.  The same symbol may appear more than once.

\clause{|for| ~var~ |in-packages| ~packages~ \opt\ |having-access| ~symbol-types~}
\clindexx{for}{in-packages}
Iterates over all the symbols from the list of packages denoted by the
descriptor ~packages~ and having accessibility (or visibility) given by
~symbol-types~.  This defaults to the list |(:external :internal :inherited)|
and is not evaluated.  ~var~ must be a list of up to three variables: in each
iteration, these will be set to a symbol, its access-type and package (as per
|with-package-iterator| in ANSI-CL).  The same symbol may appear more than
once.

\end{clauses}

\begin{clauses}

\clause{|for| ~var~ |in-file| ~name~ \opt\ |using| ~reader~}
\clindexx{for}{in-file}
Opens the file ~name~ (which may be a string or pathname) for input,
and iterates 
over its contents. ~reader~ defaults to |read|, so by default ~var~
will be bound to the successive forms in the file.  The \iter\ body is
wrapped in an unwind-protect to ensure that the file is closed no
matter how the \iter\ is exited.

\clause{|for| ~var~ |in-stream| ~stream~ \opt\ |using| ~reader~}
\clindexx{for}{in-stream}
Like |for\dots in-file|, except that ~stream~ should be an existing
stream object that supports input operations.

\end{clauses}

\subsubsection{Generalized Drivers}

These are primarily useful for writing drivers that can also be used
as generators (see section \ref{generators}, below).
 
\begin{clauses}

\clause{|for| ~var~ |next| ~expr~}
\clindexx{for}{next}

~var~ is set to ~expr~ each time through the loop.  Destructuring is
performed.  When the clause is used as a generator, ~expr~ is the code
that is executed when |(next ~var~)| is 
encountered (see section \ref{generators}, below).
~expr~ should compute the first value for ~var~, as well
as all subsequent values, and is responsible for terminating the loop.
For compatibility with future versions of \iter, this termination
should be done with |terminate|\clindex{terminate}, which can be
considered a synonym for |finish| (see section \ref{control-flow}).

\cpar As an example, the following clauses are equivalent to |(for
i from 1 to 10)|:
\begin{program}
(initially (setq i 0))
(for i next (if (> i 10) (terminate) (incf i)))
\end{program}

\clause{|for| ~var~ |do-next| ~form~}
\clindexx{for}{do-next}
~form~ is evaluated each time through the loop.  Its value is ~not~
set to ~var~; that is ~form's~ job. ~var~ is only present so that
\iter\ knows it is a driver variable.  \linebreak |(for ~var~ next
~expr~)| is equivalent to |(for ~var~ do-next (dsetq ~var~ ~expr~))|.
(See section \ref{destructuring} for an explanation of |dsetq|.) 

\end{clauses}

\subsubsection{Generators}
\label{generators}

In all of the above clauses, the driver variable is updated on each
iteration.
Sometimes it is desirable to have greater control over updating.  
For instance, consider the problem of associating numbers, in
increasing order and with no gaps, with the
\nonnil\ elements of a list.  One obvious first pass at writing this is:

\begin{program}
(iter (for el in list)
      (for i upfrom 1)
      (if el (collect (cons el i))))
\end{program}
But on the list |(a b nil c)| this produces |((a . 1) (b . 2) (c .
4))| instead of the desired |((a . 1) (b . 2) (c . 3))|.  The problem
is that |i| is incremented each time through the loop, even when |el|
is |nil|.  

The problem could be solved elegantly if we could step |i| only when
we wished to.  This
can be accomplished for any \iter\ driver by writing |generate| (or
its synonym |generating|)
instead of |for|.  Doing so produces a ~generator~---a driver whose
values are yielded explicitly.  To obtain the next value of a
generator variable ~v~, write \linebreak |(next ~v~)|.  The value of a |next|
form is the next value of ~v~, as determined by its associated driver
clause.  |next| also has the side-effect of updating ~v~ to that
value.  If there is no next value, |next| will terminate the loop,
just as with a normal driver.


Using generators, we can now write our example like this:
\begin{program}
(iter (for el in list)
      (generate i upfrom 1)
      (if el (collect (cons el (next i)))))
\end{program}
Now |i| is updated only when |(next i)| is executed, and this occurs
only when |el| is \nonnil.

To better understand the relationship between ordinary drivers and
generators, observe that we can rewrite an ordinary driver using its
generator form immediately followed by |next|, as this example shows:
\begin{program}
(iter (generating i from 1 to 10)
      (next i)
      ...)
\end{program}
Provided that the loop body contains no |(next i)| forms, this will
behave just as if we had written |(for i from 1 to 10)|.

We can still refer to a driver variable ~v~ without using |next|; in
this case, its value is that given to it by the last evaluation of
|(next ~v~)|.  Before |(next ~v~)| has been called the first time, the
value of ~v~ is undefined.  

This semantics is more flexible than
one in which ~v~ begins the loop bound to its first value and calls
of |next| supply subsequent values, because it means the loop will not
terminate too soon if the generator's sequence is empty.  For
instance, consider the following code, which tags \nonnil\ elements of
a list using a list of tags, and also counts the null elements.
(We assume there are at least as many tags as \nonnil\ elements.)
\begin{program}
(let* ((counter 0)
       (tagged-list (iter (for el in list)
                          (generating tag in tag-list) 
                          (if (null el)
                              (incf counter)
                              (collect (cons el (next tag)))))))
  ...)
\end{program}
It may be that there are just as many tags as non-null elements of
|list|.  If all the elements of |list| are null, we still want the
counting to proceed, even though |tag-list| is \nil.  If |tag| had to be
assigned its first value before the loop begins, we would have had to
terminate the loop before the first iteration, since when |tag-list|
is \nil, |tag| 
has no first value.  With the existing semantics, however, |(next
tag)| will never execute, so the iteration will cover all the elements of
|list|. 
        
When the ``variable'' of a driver clause is actually a destructuring
template containing several variables, all the variables are eligible
for use with |next|.  As before, |(next ~v~)| evaluates to ~v's~ next
value; but the effect is to update all of the template's variables.
For instance, the following code will return the list |(a 2 c)|.
\begin{program}
(iter (generating (key . item) in '((a . 1) (b . 2) (c . 3)))
      (collect (next key))
      (collect (next item)))
\end{program}

Only driver clauses with variables can be made into generators.  This
includes all clauses mentioned so far except for |repeat|.  It does
~not~ include |for\dots previous|, |for\dots =|,
|for\dots initially\dots then| or |for\dots first\dots then| (see
below). 

\subsubsection{Previous Values of Driver Variables}

Often one would like to access the value of a variable on a previous
iteration.  \iter\ provides a special clause for accomplishing this.

\begin{clauses}

\clause{|for| ~pvar~ |previous| ~var~ \opt\ |initially| ~init~
|back| ~n~}
\clindexx{for}{previous}
Sets ~pvar~ to the previous value of ~var,~ which should be a driver
variable, a variable from another |for\dots previous| clause, or a
variable established by a |for\dots =|, 
|for\dots initially\dots then| or |for\dots first\dots then| clause
(see section \ref{setting}).
Initially, ~pvar~ is given the value ~init~ (which defaults to \nil).  
The ~init~ expression will be moved outside the loop body, so it
should not depend on anything computed within the loop.
~pvar~ retains the value of ~init~ until ~var~ is set to its second
value, at which point ~pvar~ is set to ~var's~ first value; and so on.  

\cpar The
argument ~n~ to |back| must be a constant, positive integer, and
defaults to 1.  It determines how many iterations back ~pvar~ should
track ~var~.  For example, when ~n~ is 2, then ~pvar~ will be assigned
~var's~ first value when ~var~ is set to its third value.

\cpar A |for\dots previous| clause may occur after or before its
associated driver clause. |for\dots previous| works with generators as
well as ordinary drivers.

\pagebreak[3]

\cpar Example:
\begin{program}
(iter (for el in '(1 2 3 4))
      (for p-el previous el)
      (for pp-el previous p-el initially 0)
      (collect pp-el))
\end{program}
This evaluates to |(0 0 1 2)|.  It could have been written more
economically as 
\begin{program}
(iter (for el in '(1 2 3 4))
      (for pp-el previous el back 2 initially 0)
      (collect pp-el))
\end{program}


\end{clauses}

\subsection{Variable Binding and Setting}
\label{setting}

Several clauses exist for establishing new variable bindings or for
setting variables in the loop.  They all support destructuring.

\begin{clauses}

\clausex{with}{~var~ \opt\ |=| ~value~}
Causes ~var~ to be bound to value before the loop body is entered.
If ~value~ is not supplied, ~var~ assumes a default
binding, which will be 
\nil\ in the absence of declarations.  Also, if ~value~ is not
supplied, no destructuring is performed; instead, ~var~ may be a list of
symbols, all of which are given default 
bindings.  If ~value~ is supplied, ~var~ is bound to it, with
destructuring.  

\cpar Because |with|
creates bindings whose scope includes the entire \iter\ form, it is
good style to put all |with| clauses at the beginning.

\cpar Successive occurrences of |with| result in sequential
bindings (as with 
|let*|).  There is no way to obtain parallel bindings; see
section \ref{bindings} for a rationale.  


\clause{|for| ~var~ |=| ~expr~}
\clindexx{for}{=}
On each iteration, ~expr~ is evaluated and ~var~ is set
to its value. 

\cpar This clause may appear to do the same thing as |for\dots next|.
In fact, they are quite different.  |for\dots =| provides only three
services: it sets up a binding for ~var~, sets it to ~expr~ on each
iteration, and makes it possible to use |for\dots previous| with
~var~.  |for\dots next| provides these services in addition to the
ability to turn the driver into a generator.
%Also, the code which sets ~var~ appears in the loop body in the same
%place as the |for\dots =| clause; the code for |for\dots next| appears
%at the top of the loop, as with other drivers (except when being used
%as a generator).

\clause{|for| ~var~ |initially| ~init-expr~ |then| ~then-expr~}
\clindexxx{for}{initially}{then}
Before the loop begins, ~var~ is set to ~init-expr;~ on all
iterations after the first it is set to ~then-expr.~  This clause
must occur at top-level.  ~init-expr~ will be moved outside the loop
body and ~then-expr~ will be moved to the end of the loop body, so
they are subject to code motion problems (see section
\ref{code-movement}). 

\cpar This clause may appear to be similar to |for\dots next|, but in
fact they differ significantly.  |for\dots initially\dots then| is
typically used to give ~var~ its first value before the loop begins,
and subsequent values on following iterations.  This is incompatible
with generators, whose first value and subsequent values must all be
computed by |(next ~var~)|.  Also, the update of ~var~ in 
|for\dots initially\dots then| does not occur at the location of the
clause. 
Use |for\dots initially\dots then| for
one-shot computations where its idiom is more convenient, but use
|for\dots next| for extending \iter\ with new drivers (see section
\ref{extend}).

\clause{|for| ~var~ |first| ~first-expr~ |then| ~then-expr~}
\clindexxx{for}{first}{then}
The first time through the loop, ~var~ is set to ~first-expr;~ on
subsequent iterations, it is set to ~then-expr.~  This differs from
|for\dots initially| in that ~var~ is set to ~first-expr~
inside the loop body, 
so ~first-expr~ may depend on the results of other clauses.  For
instance, 
\begin{program}
(iter (for num in list)
      (for i first num then (1+ i))
      ...)
\end{program}
will set |i| to the first element of |list| on the first
iteration, whereas 
\begin{program}
(iter (for num in list)
      (for i initially num then (1+ i))
      ...)
\end{program}
is probably erroneous; |i| will be bound to |num|'s
default binding (usually \nil) for the first iteration.

\end{clauses}

\begin{note}{Compatibility note}
|loop|'s |for\dots =| works like \iter's, but |loop| used the syntax
|for\dots =\dots then| to mean |for\dots initially\dots then|.  It was
felt that these two operations were sufficiently different to warrant
different keywords.

Also, the |for| in the above three clauses is misleading,
since none is true driver (e.g. none has a corresponding |generate|
form).  |setting| would have been a better choice, but |for| was used
to retain some compatibility with |loop|.  
\end{note}


%%%%%%%%%%%%%%%%%%%%%%%%%%%%%%%%
\subsection{Gathering Clauses}
Many of \iter's clauses accumulate values into a variable, or set a
variable under certain conditions.  At the end of the loop, this
variable contains the desired result.  All these clauses have an
optional |into| keyword, whose argument should be a symbol.  If the
|into| keyword is not supplied, the accumulation variable will be
internally generated and its value will be returned at the end of the
loop; if a variable is specified, that variable is used for the
accumulation, and is not returned as a result---it is up to the user
to return it explicitly, in the loop's epilogue code (see |finally|).
It is safe to examine the accumulation variable during the loop, but
it should not be modified.

These clauses all begin with a verb.  When the verb does
not conflict with an existing Common Lisp function, then it may be
used in either its infinitival or present-participle form (e.g. |sum|,
|summing|).  However, when there is a conflict with Common Lisp, only
the present-participle form may be used (e.g. |unioning|).  This is to
prevent \iter\ clauses from clashing with Common Lisp
functions. 

%although these clauses are described as ``producing a value,'' it is a
%mistake to think of the lisp list representing the clause as a
%value-producing form in the usual way.  clauses may legally be written
%where a value is expected, e.g. |(setq x (sum i))|, but the
%lisp value of a clause in such a context is undefined.

\subsubsection{Reductions}

~Reduction~ is an extremely common iteration pattern in which
the results of successive applications of a
binary operation are accumulated.
For example, a loop that computes the sum of the
elements of a list is performing a reduction with the addition
operation.  This could be written in Common Lisp as  |(reduce \#'+
list)| or with \iter\ as
\begin{program}
(iter (for el in list)
      (sum el))
\end{program}

\begin{clauses}

\clausex{sum}{~expr~ \opt\ |into| ~var~}
Each time through the loop, ~expr~ is evaluated and added to a
variable, which is bound initially to zero.  If ~expr~ has a type,
it is ~not~ used as the type of the sum variable, which is always
|number|.  To get the result variable to be of a more specific
type, use an explicit variable, as in
\begin{program}
(iter (for el in number-list)
      (sum el into x)
      (declare (fixnum x))
      (finally (return x)))
\end{program}

\clausex{multiply}{~expr~ \opt\ |into| ~var~}
Like |sum|, but the initial value of the result variable is $1$,
and the variable is updated by multiplying ~expr~ into it.

\clausex{counting}{~expr~ \opt\ |into| ~var~}
~expr~ is evaluated on each iteration.  If it is \nonnil, the
accumulation variable, initially zero, is incremented.

\startitem
\Clausex{maximize}{~expr~ \opt\ |into| ~var~}
\Clausex{minimize}{~expr~ \opt\ |into| ~var~}
\finishitem
~expr~ is evaluated on each iteration and its extremum (maximum or
minimum) is stored in the accumulation variable.  If ~expr~ is never
evaluated, then the result is \nil\ (if the accumulation variable is
untyped) or $0$ (if it has a numeric type).

\clausex{reducing}{~expr~ |by| ~func~ \opt\ |initial-value| ~init-val~ |into| ~var~}
This is a general way to perform reductions. ~func~ should be a
function of two arguments, the first of which will be the value
computed so far and
the second of which will be the value of ~expr~.  It should return the
new value.
|reducing| is roughly equivalent to the
Common Lisp |(reduce ~func~ ~sequence~ :key ~expr-function~)|, where
~expr-function~ is used to derive values from the successive elements
of ~sequence~.

\cpar If the |reducing| clause is never executed, the result is
undefined.

\cpar It is not necessary to provide an
initial value, but better code can be generated if one is supplied.
Regardless of its location in the \iter\ body, ~init-val~ will be
evaluated before the loop is entered, so it should not depend on any
value computed inside the \iter\ form.  

%\cpar if a ~var~ is not specified, you can get \iter\ to declare the
%type of the internal variable by putting a |the| expression around
%~func.~  see section \ref{types}.

\end{clauses}

%\begin{note}{implementation note}
%in principle, |maximize| and |minimize| can be thought of
%as reductions where the initial value is the smallest (or largest)
%value that the accumulation variable can assume.  because lisp's
%bignums can represent arbitrary integers, these clauses cannot be
%implemented as reductions in general.  if, however, the type of 
%~expr~ or ~var~ can be determined to be a fixnum
%or a float, \iter\ will implement the clause as a true reduction,
%using one of the constants |most-negative-fixnum|, |%most-positive-fixnum|, 
%|most-negative-short-float|, etc. as appropriate.
%\end{note}

\subsubsection{Accumulations}

All the predefined accumulation clauses add values to a sequence.  If
the sequence is a list, they all 
have the property that the partial list is kept in the correct order
and available for inspection at any point in the loop.

\begin{clauses}

\clausex{collect}{~expr~ \opt\ |into| ~var~ |at| ~place~ |result-type| ~type~}
Produces a sequence of the values of ~expr~ on each iteration. ~place~
indicates where the next value of ~expr~ is added to the list and may
be one of the symbols |start|, |beginning| (a synonym for |start|) or
|end|.  The symbol may be quoted, but need not be.  The default is
|end|.  For example,
\begin{program}
(iter (for i from 1 to 5)
      (collect i))
\end{program}
produces |(1 2 3 4 5)|, whereas
\begin{program}
(iter (for i from 1 to 5)
      (collect i at beginning))
\end{program}
produces |(5 4 3 2 1)| (and is likely to be faster in most Common Lisp
implementations).

\cpar If ~type~ is provided, it should be a subtype of |sequence|.
The default is |list|.  Specifying a type other than |list| will
result in |collect| returning a sequence of that type.  ~However,~ the
type of the sequence being constructed when inside the loop body is
undefined when a non-|list| type is specified.  (As with ~place~,
quoting ~type~ is optional.)

\clausex{adjoining}{~expr~ \opt\ |into| ~var~ |test| ~test~ |at| ~place~ |result-type| ~type~}
Like |collect|, but only adds the value of ~expr~ if it is not
already present.  ~test,~ which defaults to |\#'eql|, is
the test to be used with |member|.

\startitem
\Clausex{appending}{~expr~ \opt\ |into| ~var~ |at| ~place~}
\Clausex{nconcing}{\ ~expr~ \opt\ |into| ~var~ |at| ~place~}
\Clausex{unioning}{\ ~expr~ \opt\ |into| ~var~ |test| ~test~ |at| ~place~}
\Clausex{nunioning}{~expr~ \opt\ |into| ~var~ |test| ~test~ |at| ~place~}
\finishitem
These are like |collect|, but behave like the Common Lisp functions
|append|, |nconc|, |union| or |nunion|.  
As in Common Lisp, they work only on lists.  Also as in Common Lisp,
|unioning| and |nunioning| assume that the value of ~expr~ contains no
duplicates. 

\clausex{accumulate}{~expr~ |by| ~func~ \opt\ |initial-value| ~init-val~ |into| ~var~}
This is a general-purpose accumulation clause. ~func~ should be a
function of two arguments, the value of ~expr~ and the value
accumulated so far in the iteration, and it should return the updated
value.  If no initial value is supplied, \nil\ is used.  

%\cpar If a ~var~ is not specified, you can get \iter\ to declare the
%type of the internal variable by putting a |the| expression around
%~func.~  see section \ref{types}.

\cpar The differences between |accumulate| and |reducing| are slight.
One difference is that the functions take their arguments in a
different order.  Another is that in the absence of ~init-val~,
|accumulate| will use \nil, whereas |reducing| will generate different
code that avoids any dependence on the initial value.
The reason for having both clauses is that one usually
thinks of reductions (like |sum|) and accumulations (like |collect|)
as different beasts.

\end{clauses}


\subsubsection{Finders}

A ~finder~ is a clause whose value is an expression that meets some
condition.

\begin{clauses}

\clause{|finding| ~expr~ |such-that| ~test~ \opt\ |into| ~var~ |on-failure| ~failure-value~}
\clindexx{finding}{such-that}
If ~test~ (which is an expression) ever evaluates to \nonnil, the loop
is terminated, the 
epilogue code is run and the value of ~expr~ is returned.  Otherwise,
\nil\ (or ~failure-value,~ if provided) is returned.  If ~var~ is
provided, it will have either the \nonnil\ value of ~expr~ or
~failure-value~ when the epilogue code is run.

\cpar As a special case, if the ~test~ expression is a sharp-quoted
function, then it is applied to ~expr~ instead of being simply
evaluated.  E.g. |(finding x such-that \#'evenp)| is equivalent to
|(finding x such-that (evenp x))|.  

%\cpar although ~test~ need have nothing to do with ~%expr~ as in
%|(finding j such-that (> i 3))|, it usually 
%will: |(finding (length el) such-that (oddp (length el)))|.  to
%avoid performing the |length| computation twice, you could write
%|(finding (length el) such-that \#'oddp)| or |(finding (length
%el) such-that 'oddp)|; for these cases, \iter\ generates code that
%executes ~expr~ only once.  the code for |\#'oddp|
%is slightly different from that for {\lisp 'oddp}; see the discussion
%under {\lisp for\dots in} and {\lisp for\dots on}.

\cpar |On-failure| is a misnomer. Because it is always evaluated, it behaves
more like the default third argument to the |gethash| function. As a result,
|on-failure (error "Not found")| makes no sense. Instead, the clauses |leave|
or |thereis| can be used in conjunction with |finally| as follows:
\begin{program}
(iter (for x in '(1 2 3))
      (if (evenp x) (leave x))
      (finally (error "not found")))
\end{program}

\cpar This clause may appear multiple times when all defaults are
identical. It can also be used together with either |always|/|never| or
|thereis| if their defaults match. More specifically, |on-failure nil| is
compatible with |thereis|, while |on-failure t| is compatible with |always|
and |never| clauses.
\begin{program}
(iter (for i in '(7 -4 2 -3))
      (if (plusp i)
	  (finding    i  such-that (evenp i))
          (finding (- i) such-that (oddp i))))
\end{program}

\startitem
\Clause{|finding| ~expr~ |maximizing| ~m-expr~ \opt\ |into| ~var~}
\clindexx{finding}{maximizing}
\Clause{|finding| ~expr~ |minimizing| ~m-expr~ \opt\ |into| ~var~}
\clindexx{finding}{minimizing}
\finishitem
Computes the extremum (maximum or minimum) value of ~m-expr~ over all
iterations, and returns the value of ~expr~ corresponding to the
extremum.  ~expr~ is evaluated inside the loop at the time the new
extremum is established.  If ~m-expr~ is never evaluated (due to, for
example, being embedded in a conditional clause), then the returned
value depends on the type, if any, of ~expr~ (or ~var,~ if
one is supplied).  If there is no type, the returned
value will be \nil; if the type is numeric, the returned value will be
zero.

\cpar For these two clauses, ~var~ may be a list of two
symbols; in that case, the first is used to record ~expr~ and
the second, ~m-expr.~  

\cpar As with |finding\dots such-that|, if ~m-expr~ is a sharp-quoted
function, then it is called on ~expr~ instead of being evaluated.

\end{clauses}

\subsubsection{Boolean Tests}

\begin{clauses}

\clausex{first-iteration-p}{}
Returns |t| in the first cycle of the loop, otherwise \nil.

\clausex{first-time-p}{}
Returns |t| the first time the expression is evaluated, and then \nil\ forever.
This clause comes handy when printing (optional) elements separated
by a comma:

\begin{program}
(iter (for el in '(nil 1 2 nil 3))
      (when el
        (unless (first-time-p)
          (princ ", "))
        (princ el)))
\end{program}
produces |"1, 2, 3"|.

\end{clauses}

\subsubsection{Aggregated Boolean Tests}

\begin{clauses}

\clausex{always}{~expr~}
If ~expr~ ever evaluates to
\nil, then \nil\ is immediately returned; the epilogue code is not
executed.  If ~expr~ never evaluates to \nil, the epilogue code
is executed and the last value of ~expr~ (or |t| if ~expr~ was never
evaluated) is returned (whereas |loop| would constantly return |t|).

% mention last evaluated clause when multiple always clauses?

\clausex{never}{~expr~}
Like |(always (not ~expr~))|, except it does not influence the last
value returned by a possible other |always| clause. That is,
\begin{program}
(iter (repeat 2)
      (always 2)
      (never nil)) \yields 2 ; not t
\end{program}

\clausex{thereis}{~expr~}
If ~expr~ is ever \nonnil,
its value is immediately returned without running epilogue code.
Otherwise, the epilogue code is performed and \nil\ is returned.

This clause cannot be used together with |always| or |never|, because their
defaults are opposed (similarly, |(loop always 3 thereis nil)| refuses to
compile in some implementations of |loop|).

\end{clauses}


\subsection{Control Flow}
\label{control-flow}
Several clauses can be used to alter the usual flow of control in a loop.

Note: the clauses of this and subsequent sections don't adhere to \iter's
usual syntax, but instead use standard Common Lisp syntax.  Hence the
format for describing syntax subsequently is
like the standard format used in the Common Lisp manual, not like the
descriptions of clauses above.

\begin{clauses}

\clausex{finish}{}
Stops the loop and runs the epilogue code.

%for example:
%
%\begin{program}
%(iter (with answer = nil)
%         (initially (make-a-mess))
%         (for i from 1 to 10)
%         (when (correct? i)
%           (setq answer i)
%           (finish))
%         (finally (cleanup)))
%\end{program}
%
%this code will execute |cleanup| whether or not the test |(correct?
%i)| ever succeeds. 
%the (more elegant) formulation,
%\begin{program}
%(iter (initially (make-a-mess))
%         (for i from 1 to 10)
%         (finding i such-that (correct? i))
%         (finally (cleanup)))
%\end{program}
%would not execute |cleanup| if |(correct? i)| succeeded; it
%would do an immediate return.

\clausex{leave}{\opt\ ~value~}
Immediately returns ~value~ (default \nil) from the current \iter\
form, skipping the epilogue code.  Equivalent to using |return-from|.

\clausex{next-iteration}{}
Skips the remainder of the loop body and begins the next iteration of
the loop.

\clausex{while}{~expr~}
If ~expr~ ever evaluates to \nil, the loop is terminated and the
epilogue code executed.  Equivalent to |(if (not ~expr~) (finish))|.

\clausex{until}{~expr~}
Equivalent to |(if ~expr~ (finish))|.

\clausex{if-first-time}{~then~ \opt\ ~else~}
If this clause is being executed for the first time in this invocation
of the \iter\ form, then the ~then~ code is evaluated; otherwise the
~else~ code is evaluated.

\cpar |(for ~var~ first ~expr1~ then ~expr2~)| is almost equivalent to
\begin{program}
(if-first-time (dsetq ~var~ ~expr1~)
               (dsetq ~var~ ~expr2~))
\end{program}
The only difference is that the |for| version makes ~var~ available
for use with |for\dots previous|.

\end{clauses}

\subsection{Code Placement}
When fine control is desired over where code appears in a loop
generated by \iter, the following special clauses may be useful.
They are all subject to code-motion problems (see section
\ref{code-movement}).  

\begin{clauses}

\clausex{initially}{|\&rest| ~forms~}
The lisp ~forms~ are placed in the prologue section of the loop, where
they are executed once, before the loop body is entered.

\clausex{after-each}{|\&rest| ~forms~}
The ~forms~ are placed at the end of the loop body, where they
are executed after each iteration.  Unlike the other clauses in this
section, ~forms~ may contain \iter\ clauses.

\clausex{else}{|\&rest| ~forms~}
The lisp ~forms~ are placed in the epilogue section of the loop, where they
are executed if this |else| clause is never met during execution of the
loop and the loop terminates normally.

\clausex{finally}{|\&rest| ~forms~}
The lisp ~forms~ are placed in the epilogue section of the loop, where
they are executed after the loop has terminated normally.

\clausex{finally-protected}{|\&rest| ~forms~}
The lisp ~forms~ are placed in the second form of an unwind-protect
outside the loop.  They are always executed after the loop has
terminated, regardless of how the termination occurred.

\end{clauses}


%%%%%%%%%%%%%%%%%%%%%%%%%%%%%%%%%%%%%%%%%%%%%%%%%%%%%%%%%%%%%%%%
\section{Other Features}

\subsection{Multiple Accumulations}
\label{multiple}

\begin{sloppypar}
It is permitted to have more than one clause accumulate into the same
variable, as in the following:
\begin{program}
(iter (for i from 1 to 10)
      (collect i into nums)
      (collect (sqrt i) into nums)
      (finally (return nums)))
\end{program}
Clauses can only accumulate into the same variable if they are
compatible.  |collect|, |adjoining|, |appending|, |nconcing|,
|unioning| and |nunioning| are compatible with each other; |sum|,
|multiply| and |counting| are compatible; |always| and |never| are compatible;
|finding| \dots |such-that| is compatible with either |thereis| or |always|
and |never| when their defaults match; and |maximize| and |minimize| clauses
are compatible only with other |maximize| and |minimize| clauses,
respectively.

%note that the same variable ~cannot~ be both an accumulation
%variable and an ordinary variable; there can be only one variable with
%a given name within an \iter\ form. 
\end{sloppypar}

\subsection{Named Blocks}

Like Common Lisp |block|s, \iter\ forms can be given names.  The
name should be a single symbol, and it must be the first form in the
\iter.  The generated code behaves exactly like a named block; in
particular, |(return-from ~name~)| can be used to exit it:
\begin{program}
(iter fred
      (for i from 1 to 10)
      (iter barney
            (for j from i to 10)
            (if (> (* i j) 17)
                (return-from fred j))))
\end{program}
An \iter\ form that is not given a name is implicitly named \nil.

Sometimes one would like to write an expression in an inner \iter\ form,
but have it processed by an outer \iter\ form.  This is possible with
the |in| clause.  

\begin{clauses}

\clausex{in}{~name~ |\&rest| ~forms~}
Evaluates ~forms~ as if they were part of the \iter\ form named
~name~.  In other words, \iter\ clauses are processed by the \iter\
form named ~name,~ and not by any \iter\ forms that occur inside ~name.~

\cpar As an example, consider the problem of collecting a list of the
elements in a two-dimensional array.  The naive solution,
\begin{program}
(iter (for i below (array-dimension ar 0))
      (iter (for j below (array-dimension ar 1))
            (collect (aref ar i j))))
\end{program}
\noindent is wrong because the list created by the inner \iter\ is simply
ignored by the outer one.  But using |in| we can write:
\begin{program}
(iter outer (for i below (array-dimension ar 0))
      (iter (for j below (array-dimension ar 1))
            (in outer (collect (aref ar i j)))))
\end{program}
\noindent which has the desired result.

\end{clauses}

\subsection{Destructuring}
\label{destructuring}

In many places within \iter\ clauses where a variable is expected, a
list can be written instead.  In these cases, the value to be assigned
is ~destructured~ according to the pattern
described by the list.  As a simple example, the clause
\begin{program}
(for (key . item) in alist)
\end{program}
\noindent will result in |key| being set to the |car| of
each element in |alist|, and |item| being set to the |cdr|.  The
pattern list may be nested to arbitrary depth, and (as the example
shows) need not be terminated with \nil; the only requirement is that
each leaf be a bindable symbol (or \nil, in which case no binding is
generated for that piece of the structure).

Sometimes, you might like to do the equivalent of a
|multiple-value-setq| in a clause.  This 
``multiple-value destructuring'' can be expressed by writing 
\linebreak |(values $pat_1$ $pat_2 \ldots$)| for a destructuring
pattern, as in 
\begin{program}
(for (values (a . b) c d) = (three-valued-function ...))
\end{program}
\begin{sloppypar}
Note that the $pat_i$ can themselves be destructuring patterns (though
not multiple-value destructuring patterns).  You can't do multiple-value
destructuring in a |with| clause; instead wrap the whole \iter\
form in a |multiple-value-bind|.
\end{sloppypar}

\begin{note}{Rationale}
There are subtle interactions between variable declarations and
evaluation order that make the correct implementation of
multiple-value destructuring in a |with| somewhat tricky.
\end{note}

The destructuring feature of \iter\ is available as a separate
mechanism, using the |dsetq| macro:

\begin{clauses}
\defmacro{dsetq}{template expr}
Performs destructuring of ~expr~ using ~template~.   May be used
outside of an \iter\ form. Yields the primary value of ~expr~.

\end{clauses}

\subsection{On-line Help}

\begin{sloppypar}
There is a limited facility for on-line help, in the form of the 
|display-iterate-clauses| function.
\end{sloppypar}

\begin{clauses}

\defun{display-iterate-clauses}{\opt\ clause-spec}
Displays a list of \iter\ clauses.  If ~clause-spec~ is not
provided, all clauses are shown; if it is a symbol, all clauses
beginning with that symbol are shown; and if it is a list of symbols,
all clauses for which ~clause-spec~ is a prefix are shown.

\end{clauses}

\subsection{Parallel Binding and Stepping}
\label{bindings}

The parallel binding and stepping of variables is a feature that
\iter\ does ~not~ have.  This section attempts to provide a rationale.

We say that two variables are bound ~in parallel~ if neither
binding shadows the other.  This is the usual semantics of |let|
(as opposed to |let*|).  Similarly, we can say that iteration
variables are stepped in parallel if neither variable is updated
before the other, conceptually speaking; in other words, if the code
to update each variable can reference the old values of both variables.

|loop| allows parallel binding of variables and parallel stepping of
driver variables.  My view is that if you are depending on the 
serial/parallel distinction, you are doing something obscure.  If you
need to bind 
variables in parallel using |with|, then you must be using a
variable name that shadows a name in the existing lexical environment.
Don't do that.  The most common use for parallel stepping is to track
the values of variables on the previous iteration, but in fact this
does not require parallel stepping at all; the following will work:
\begin{program}
(iter (for current in list)
      (for prev previous current)
      ...)
\end{program}


%%%%%%%%%%%%%%%%%%%%%%%%%%%%%%%%%%%%%%%%%%%%%%%%%%%%%%%%%%%%%%%%
\section{Types and Declarations}
\label{types}

\subsection{Discussion}

Sometimes efficiency dictates that the types of variables be declared.
This type information needs to be communicated to \iter\ so it can
bind variables to appropriate values.  Furthermore, \iter\ must often
generate internal variables invisible to the user; there needs to be a
way for these to be declared.

As an example, consider this code, which will return the number of
odd elements in |number-list|:
\begin{program}
(iter (for el in number-list)
      (count (oddp el)))
\end{program}
In processing this form,
\iter\ will create an internal variable, let us call it |list17|, to
hold the successive |cdr|s of |number-list|, and  
will bind the variable to |number-list|.  It will also generate a
default binding for |el|; only inside the body of the loop will |el|
be set to the |car| of |list17|.  Finally, \iter\ will generate a
variable, call it |result|, to hold the result of the count, and will
bind it to zero.

When dealing with type declarations, \iter\ observes one simple rule:
~it will never generate a declaration unless requested to do so.~  The
reason is that such declarations might mask errors in compiled code by
avoiding error-checks; the resulting problems would be doubly hard to
track down because the declarations would be hidden from the
programmer.  Of course, a compiler might omit error-checks even in the
absence of declarations, though this behavior can usually be avoided,
e.g. by saying |(declaim (optimize (safety 3)))|.

So, the above \iter\ form will generate code with no declarations.
But say we wish to declare the types of |el| and the internal
variables |list17| and |result|.  How is this done?

Declaring the type of |el| is easy, since the programmer knows
the variable's name:
\begin{program}
(iter (for el in number-list)
      (declare (fixnum el))
      (counting (oddp el)))
\end{program}
\iter\ can read variable type declarations like this one.  Before
processing any clauses, it scans the entire top-level form for type
declarations 
and records the types, so that variable bindings can be performed
correctly.  In this case, |el| will be bound to zero
instead of \nil.  Also, \iter\ collects all the top-level declarations
and puts them at the begining of the generated code, so it is not
necessary to place all declarations at the beginning of an \iter\
form; instead, they can be written near the variables whose types they
declare.  

Since \iter\ is not part of the compiler, it will not know
about declarations that occur outside an \iter\ form; these
declarations must be repeated inside the form.

Here is another way we could have declared the type of |el|:
\begin{program}
(iter (for (the fixnum el) in number-list)
      (counting (oddp el)))
\end{program}
\iter\ extends the Common Lisp |the|\lispindex{the}
form to apply to variables as well as value-producing forms; anywhere
a variable is allowed---in a |with| clause, as the iteration
variable in a driver clause, as the |into| argument of an
accumulation clause, even inside a destructuring template---you can
write |(the ~type~ ~symbol~)| instead.

There is one crucial difference between using a |the| form and
actually declaring the variable: explicit declarations are always
placed in the generated code, but type information from a |the|
form is not turned into an actual declaration unless you tell \iter\
to do so using |iterate:declare-variables|.  See below.

\begin{sloppypar}
Declaring the types of internal variables is harder than declaring the
types of explicitly mentioned variables, since their names
are unknown.  You do it by declaring |iterate:declare-variables|
somewhere inside the top level of the \iter\ form.  (This will also
generate declarations for variables declared using |the|.)
\iter\ does not provide much selectivity here: it's all or none. 
And unfortunately, since \iter\ is not privy to compiler information
but instead reads declarations itself, it will not hear if you
|(declaim (iterate:declare-variables))|.  Instead, set the variable 
|iterate::*always-declare-variables*| to |t| at
compile-time, using |eval-when|.
\end{sloppypar}

To determine the appropriate types for internal variables, \iter\ uses
three sources of information:

\begin{itemize}

\item Often, the particular clause dictates a certain type for a
variable; \iter\ will use this information when available.  In the
current example, the variable |list17| will be given the type
|list|, since that is the only type that makes sense; and the
variable |result| will be given the type |fixnum|, on the
assumption that you will not be counting high enough to need bignums.
You can override this assumption only by using and explicitly declaring a
variable: 
\begin{verbatim}
(iter (declare (iterate:declare-variables))
      (for el in number-list)
      (count (oddp el) into my-result)
      (declare (integer my-result))
      (finally (return my-result)))
\end{verbatim}

\begin{sloppypar}
Other examples of the type assumptions that \iter\ makes are: type
|list| for |into| variables of collection clauses; type |list| for
expressions that are to be destructured; type |vector| for the
variable holding the vector in a |for\dots in-vector| clause, and
similarly for |string| and the |for\dots in-string| clause;
and the implementation-dependent type
|(type-of array-dimension-limit)| for the index and limit
variables generated by sequence iteration drivers like |for\dots
in-vector| and |for\dots in-string| (but not |for\dots in-sequence|,
because it may be used to iterate over a list). 
\end{sloppypar}

\item Sometimes, \iter\ will examine expressions and try to determine
their types in a simple-minded way.  If the expression is
self-evaluating (like a number, for instance), \iter\ knows that the
expression's type is the same as the type of the value it denotes, so
it can use that type.  If the expression is of the form |(the ~type~
~expr~)|, \iter\ is smart enough to extract ~type~ and use it.
However, the current version of \iter\ does not  
examine declarations of function result types or do any type
inference.  It will not determine, for 
example, that the type of |(+ 3 4)| is |fixnum|, or even
|number|.

\item In some cases, the type of an internal variable should match the
type of some other variable.  For instance, \iter\ generates an
internal variable for |(f x)| in the
clause |(for i from 1 to (f x))|, and in the absence of other
information will give it the same type as |i|.  If, however, the
expression had been written |(the fixnum (f x))|, then \iter\
would have given the internal variable the type |fixnum|
regardless of |i|'s type.  The type incompatibility errors that
could arise in this situation are not checked for.

\end{itemize}

Note that if you do declare |iterate:declare-variables|, then
\iter\ may declare user variables as well as internal ones if they do
not already have declarations, though only for variables that it
binds.  For instance, in this code:

\begin{program}
(iter (declare (iterate:declare-variables))
      (for i from 1 to 10)
      (collect i into var))
\end{program}
the variable |var| will be declared to be of type |list|.


\subsection{Summary}

\iter\ understands standard Common Lisp variable type declarations
that occur within an \iter\ form and
will pass them through to the generated code.  If the declaration
|(iterate:declare-variables)|\lispindex{declare-variables}
appears at the top level of an 
\iter\ form, or if 
|iterate::*always-declare-variables*|\lispindex{*always-declare-variables*} 
is \nonnil, then \iter\ will use the type information gleaned from user
declarations, self-evaluating expressions and |the| expressions,
combined with reasonable assumptions, to determine variable
types and declare them.


%%%%%%%%%%%%%%%%%%%%%%%%%%%%%%%%%%%%%%%%%%%%%%%%%%%%%%%%%%%%%%%%
\section{Problems with Code Movement}
\label{code-movement}

Some \iter\ clauses, or parts of clauses, result in code being
moved from the location of the clause to other parts of the loop.
Drivers behave this way, as do code-placement clauses like |initially|
and |finally|.  When using these clauses, there is a danger of writing
an expression that makes sense in its apparent location but will be
invalid or have a different meaning in another location.  For example:
\begin{program}
(iter (for i from 1 to 10)
      (let ((x 3))
        (initially (setq x 4))))
\end{program}
While it may appear that the |x| of |(initially (setq x 4))| is the
same as the |x| of |(let ((x 3)) \dots|, in fact they are not:
|initially| moves its code outside the loop body, so |x| would refer
to a global variable.  Here is another example of the same problem:
\begin{program}
(iter (for i from 1 to 10)
      (let ((x 3))
        (collect i into x)))
\end{program}
If this code were executed, |collect| would create a binding for its
|x| at the top level of the \iter\ form that the |let| will shadow.

Happily, \iter\ is smart enough to catch these errors; it
walks all problematical code to ensure that free variables are not
bound inside the loop body, and checks all variables it binds for the
same problem.

However, some errors cannot be caught:

\begin{program}
(iter (with x = 3)
      (for el in list)
      (setq x 1)
      (reducing el by \#'+ initial-value x))
\end{program}
|reducing| moves its |initial-value| argument to the initialization
part of the loop in order to produce more efficient code.  Since
\iter\ does not perform data-flow analysis, it cannot determine that
|x| is changed inside the loop; all it can establish is that |x| is
not bound internally.  Hence this code will not signal an
error and will use $3$ as the initial value of the reduction.

The following list summarizes all cases that are subject to these code
motion and variable-shadowing problems.
\begin{itemize}
\item Any variable for which \iter\ creates a binding, including those
used in |with| and the |into| keyword of many clauses.

\begin{sloppypar}
\item The special clauses which place code: |initially|, |after-each|, |else|,
|finally| and |finally-protected|. 
\end{sloppypar}

\item The variables of a |next| or |do-next| form.

\item The |initially| arguments of |for\dots initially\dots then| and
|for\dots previous|. 

\item The |then| argument of |for\dots initially\dots then|.

\item The |initial-value| arguments of |reducing| and |accumulate|.

\item The |on-failure| argument of |finding\dots such-that|.
\end{itemize}

%%%%%%%%%%%%%%%%%%%%%%%%%%%%%%%%%%%%%%%%%%%%%%%%%%%%%%%%%%%%%%%%
\section{Differences Between {\tt Iterate} and {\tt Loop}}
|loop| contains a great deal of complexity which \iter\ tries to
avoid.  Hence many esoteric features of |loop| don't exist in \iter.
Other features have been carried over, but in a cleaned-up form.  
And of course, many new features have been added; they are not
mentioned in this list.

\begin{itemize}

\item \iter's syntax is more Lisp-like than |loop|'s, having a higher
density of parens.

\item The current implementation of \iter, unlike the current version
of  |loop| (as documented in {\em Common Lisp, 2nd Ed.\/}), is
extensible (see section \ref{extend}).  

\item |loop| puts the updates of all driver variables at the top of
the loop; \iter\ leaves them where the driver clauses appear.

\item While for the most part \iter\ clauses that resemble |loop| clauses
behave similarly, there are some differences.  For instance, there is
no |for\dots =\dots then| in \iter; instead use 
|for\dots initially\dots then|.  

\item |loop| binds the variable |it| at certain times
to allow pseudo-English expressions like |when ~expr~ return it|. 
In \iter, you must bind ~expr~ to a variable yourself.  Note that
|when ~expr~ return it| is like |thereis ~expr~| except that the latter is an
accumulation clause and therefore competes with other accumulations
(remember section \ref{multiple} above).

% repeat different behaviour of |always| clause here?

\item |loop| has a special |return| clause, illustrated in the
previous item.  \iter\ doesn't need one,  since an ordinary Lisp
|return| has the same effect.

\item |loop| allows for parallel binding and stepping of iteration
variables.  \iter\ does not.  (See section \ref{bindings}.)

\item |loop| and \iter\ handle variable type declarations very
differently.  |loop| provides a special syntax for declaring variable
types, and does not examine declarations.  Moreover, the standard
implementation of |loop| will
generate declarations when none are requested.
\iter\ parses standard Common Lisp type declarations, and will never
declare a variable itself unless declarations are 
specifically requested.

\end{itemize}

%%%%%%%%%%%%%%%%%%%%%%%%%%%%%%%%%%%%%%%%%%%%%%%%%%%%%%%%%%%%%%%%
\section{Rolling Your Own}
\label{extend}

\subsection{Introduction}

\iter\ is extensible---you can write new clauses that embody new
iteration patterns.  You might want to write a new driver clause for a
data structure of your own, or you might want to write a clause that
collects or manipulates elements in a way not provided by \iter. 

This section describes how to write clauses for \iter.  Writing a
clause is like writing a macro.  In fact, writing a clause ~is~
writing a macro: since \iter\ code-walks its body and macroexpands,
you can add new abstractions to \iter\ with good old |defmacro|.

Actually, there are two extensions you can make to \iter\ that are
even easier than writing a macro.  They are adding a synonym for an
existing clause and defining a driver clause for an indexable
sequence.  These can be done with |defsynonym| and
|defclause-sequence|, respectively.  See section \ref{aids}, below.

The rest of this section explains how to write macros that expand into
\iter\ clauses.
Here's how you could add a simplified version of \iter's |multiply|
clause, if \iter\ didn't already have one:
\begin{program}
(defmacro multiply (expr)
  `(reducing ,expr by \#'* initial-value 1))
\end{program}

If you found yourself summing the square of an expression often, you
might want to write a macro for that.  A first cut might be
\begin{program}
(defmacro sum-of-squares (expr)
  `(sum (* ,expr ,expr)))
\end{program}
but if you are an experienced macro writer, you will realize that this
code will evaluate ~expr~ twice, which is probably a bad idea.  A
better version would use a temporary:
\begin{program}
(defmacro sum-of-squares (expr)
  (let ((temp (gensym)))
    `(let ((,temp ,expr))
       (sum (* ,temp ,temp)))))
\end{program}
Although this may seem complex, it is just the sort of thing you'd
have to go through to write any macro, which illustrates the point of
this section: if you can write macros, you can extend \iter.

Our macros don't use \iter's keyword-argument
syntax.  We could just use keywords with |defmacro|, but we would
still not be using \iter's clause indexing mechanism.  Unlike Lisp,
which uses just the first symbol of a form to determine what function
to call, \iter\ individuates clauses by the list of required keywords.
For instance, |for\dots in| and |for\dots in-vector| are different clauses
implemented by distinct Lisp functions.  

To buy into this indexing scheme, as well as the keyword-argument
syntax, use |defmacro-clause|:

\begin{clauses}

\defmacro{defmacro-clause}{arglist |\&body| body}
Defines a new \iter\ clause.  ~arglist~ is a list of symbols which are
alternating keywords and arguments.  \opt\ may be used, and the list
may be terminated by |\&sequence|.  ~body~ is an ordinary macro body,
as with |defmacro|.  If the first form of ~body~ is a string, it is
considered a documentation string and will be shown by
|display-iterate-clauses|. |defmacro-clause| will signal an error if 
defining the clause would result in an ambiguity.  E.g. you cannot
define the clause |for\dots from| because there would be no way to
distinguish it from a use of the |for| clause with optional keyword |from|.

\end{clauses}

\medskip

Here is |multiply| using |defmacro-clause|.  The keywords are capitalized
for readability.
\begin{program}
(defmacro-clause (MULTIPLY expr \opt\ INTO var)
  `(reducing ,expr by \#'* into ,var initial-value 1))
\end{program}
You don't have to worry about the case when |var| is not supplied; for
any clause with an |into| keyword, saying |into nil| is equivalent to
omitting the |into| entirely.

As another, more extended example, consider the fairly common
iteration pattern that 
involves finding the sequence element that maximizes (or minimizes) some
function.  \iter\ provides this as |finding\dots maximizing|, but it's
instructive to see how to write it.
Here, in pseudocode, is how you might write such a loop for
maximizing a function F:

\begin{quote}
\begin{tabbing}
set variable MAX-VAL to NIL; \\
set variable WINNER to NIL; \\
for\=\ each element EL in the sequence \\
\>  if\=\ MAX-VAL\=\ is NIL or F(EL) $>$ MAX-VAL then \\
\>\>    set MAX-VAL to F(EL); \\
\>\>    set WINNER to EL; \\
\>  end if; \\
end for; \\
return WINNER.
\end{tabbing}
\end{quote}

Here is the macro:
\begin{program}
(defmacro-clause (FINDING expr MAXIMIZING func \opt\ INTO var)
  (let ((max-val (gensym))
        (temp1 (gensym))
        (temp2 (gensym))
        (winner (or var iterate::*result-var*)))
    `(progn 
       (with ,max-val = nil)
       (with ,winner = nil)
       (cond 
        ((null ,max-val)
         (setq ,winner ,expr)
         (setq ,max-val (funcall ,func ,winner))
        (t
         (let* ((,temp1 ,expr)
                (,temp2 (funcall ,func ,temp1)))
           (when (> ,temp2 ,max-val)
             (setq ,max-val ,temp2)
             (setq ,winner ,temp1))))))
       (finally (leave ,winner)))))
\end{program}
Note that if no |into| variable is supplied, we use
|iterate::*result-var*|, which contains the internal variable into
which all clauses place their results.  If this variable is bound by
some clause, then \iter\ will return its value automatically;
otherwise, \nil\ will be returned.  

\subsection{Writing Drivers}

In principle, drivers can be implemented just as easily as other
\iter\ clauses.  In practice, they are a little harder to get right. 
As an example, consider writing a driver that
iterates over all the 
elements of a vector, ignoring its fill-pointer.  |for\dots in-vector|
won't work for this, because it observes the fill-pointer.  It's necessary to
use |array-dimension| instead of |length| to obtain the size of the
vector.  Here is one approach:

\begin{program}
(defmacro-clause (FOR var IN-WHOLE-VECTOR v)
  "All the elements of a vector (disregards fill-pointer)"
  (let ((vect (gensym))
        (index (gensym)))
    `(progn
       (with ,vect = ,v)
       (for ,index from 0 below (array-dimension ,vect 0))
       (for ,var = (aref ,vect ,index)))))
\end{program}
Note that we immediately put |v| in a variable, in case it is an
expression.  Again, this is just good Lisp macrology.  It also has a
subtle effect on the semantics of the driver: |v| is evaluated only
once, at the beginning of the loop, so changes to |v| in the loop have
no effect on the driver.  Similarly, the bounds for numerical iteration
e.g. the above |array-dimension| are also evaluated once only.  This is how
all of \iter's drivers work.


There is an important point concerning the |progn| in this code.  We
need the |progn|, of course, because we are returning several forms,
one of which is a driver.  But \iter\ drivers must occur at top-level.
Is this code in error?  No, because ~top-level~ is defined in \iter\
to include forms inside a |progn|.  This is just the definition of
top-level that Common Lisp uses, and for the same reason: to allow
macros to return multiple forms at top-level.

While our |for\dots in-whole-vector| clause will work, it is not
ideal.  In particular, it does not support generating.  Do do so, we
need to use |for\dots next| or |for\dots do-next|.  The job is
simplified by the |defmacro-driver| macro.

\begin{clauses}

\defmacro{defmacro-driver}{arglist |\&body| body}
Defines a driver clause in
both the |for| and |generate| forms, and provides a parameter
|generate| which ~body~ can examine to determine how it was invoked.
~arglist~ is as in |defmacro-clause|, and should begin with the symbol
|for|. 

\end{clauses}

With |defmacro-driver|, our driver looks like this:
\begin{program}
(defmacro-driver (FOR var IN-WHOLE-VECTOR v)
  "All the elements of a vector (disregards fill-pointer)"
   (let ((vect (gensym))
         (end (gensym))
         (index (gensym))
         (kwd (if generate 'generate 'for)))
     `(progn
        (with ,vect = ,v)
        (with ,end = (array-dimension ,vect 0))
        (with ,index = -1)
        (,kwd ,var next (progn (incf ,index)
                               (if (>= ,index ,end) (terminate))
                               (aref ,vect ,index))))))
\end{program}


We are still missing one thing: the |\&sequence| keywords.  
We can get them easily enough, by writing 
\begin{program}
(defmacro-driver (FOR var IN-WHOLE-VECTOR v \&sequence) 
  ...)
\end{program}
We can now refer to parameters |from|, |to|, |by|, etc. which contain
either the values for the corresponding keyword, or \nil\ if the
keyword was not supplied.  Implementing the right code for these
keywords is cumbersome but not difficult; it is left as an exercise.
But before you begin, see |defclause-sequence| below for an easier way.
       
\subsection{Extensibility Aids}
\label{aids}

This section documents assorted features that may be of use in
extending \iter.

\begin{clauses}

\defunexpvar{*result-var*}
Holds the variable that is used to return a value as a result of the
\iter\ form.  You may examine this and use it in a |with| clause, but
you should not change it.

\defmacro{defsynonym}{syn word}
 Makes ~syn~ a synonym for the existing \iter\ keyword ~word.~  Only
the first word in each clause can have synonyms. 

\defmacroxx{defclause-sequence}{element-name index-name}{|\&key|
access-fn size-fn sequence-type}{element-type
element-doc-string index-doc-string}  
Provides
a simple way to define sequence clauses.  Generates two
clauses, one for iterating over the sequence's elements, the other
for iterating over its indices.  The first symbol of both
clauses will have print-name |for|.
~element-name~  and ~index-name~ should be symbols.
~element-name~ is the second keyword of the element iterator (typically of
the form 
|in-~sequence-type~|), and ~index-name~ is the second keyword
of the index-iterator (typically of the form
|index-of-~sequence-type~|).  Either name may be 
\nil, in which case the corresponding clause is not defined.  If both
symbols are supplied, they should be in the same package.  The |for|
that begins the clauses will be in this package.

\cpar ~access-fn~ is the function to be used to
access elements of the sequence in the element iterator.  The function
should take two 
arguments, a sequence and an index, and return the appropriate element.
~size-fn~ should denote a function of one argument, a sequence, that
returns its size.  Both ~access-fn~ and ~size-fn~ are required for the
element iterator, but only ~size-fn~ is needed for the index iterator.

\cpar The ~sequence-type~ and ~element-type~ keywords are used to
suggest types for the variables 
used to hold the sequence and the
sequence elements, respectively.  The usual rules about \iter's
treatment of variable type declarations apply (see section \ref{types}).

\cpar ~element-doc-string~ and ~index-doc-string~ are
the documentation strings, for use with |display-iterate-clauses|. 

\cpar The generated element-iterator performs destructuring on the
element variable.

\cpar As an example, the above |for\dots in-whole-vector| example
could have been written:
\begin{program}
(defclause-sequence IN-WHOLE-VECTOR INDEX-OF-WHOLE-VECTOR
  :access-fn 'aref
  :size-fn \#'(lambda (v) (array-dimension v 0))
  :sequence-type 'vector
  :element-type t
  :element-doc-string 
     "Elements of a vector, disregarding fill-pointer"
  :index-doc-string 
     "Indices of vector, disregarding fill-pointer")
\end{program}

\end{clauses}

\subsection{Subtleties}

There are some subtleties to be aware of when writing \iter\ clauses.
First, the code returned by your macros may be |nconc|'ed into a list,
so you should always returned freshly consed lists, rather than
constants.  Second, \iter\ matches clauses by using |eq| on the first
symbol and |string=| on the subsequent ones, so the package of the
first symbol of a clause is relevant.  All of the clauses in this manual
have their first word in the \iter\ package.  
You can use the package system in the usual way to shadow
\iter\ clauses without replacing them.

%%% say more here, about the badness that only the first word of a
%%% clause is packagey.

\section{Non-portable Extensions to Iterate (Contribs)}
\label{contribs}

Currently, there is only one non-portable extension to iterate in the
distribution: iterate-pg. If you have made an extension that depends
on non-portable features, feel free to send them to |asf@boinkor.net|
for inclusion in the iterate distribution.

\subsection{An SQL query driver for iterate}

The pg package by Eric Marsden (see |http://cliki.net/pg|) provides an
interface to the PostgreSQL database. Using the \iterpg\ extension, it
is possible to handle the results of SQL queries using \iter.

This usage example should give you an idea of how to use it:

\begin{program}
(pg:with-pg-connection (c "somedb" "someuser")
  (iter (for (impl version date) in-relation "select * from version"
                                 on-connection *dbconn*)
        (collect version)))
\end{program}

To use the extension via |ASDF|, simply make your system depend on the
|iterate-pg| system instead of the |iterate| system. To load it
manually, use:

\begin{program}
  (asdf:oos 'asdf:load-op :iterate-pg)
\end{program}

%%%%%%%%%%%%%%%%%%%%%%%%%%%%%%%%%%%%%%%%%%%%%%%%%%%%%%%%%%%%%%%%
\section{Obtaining {\tt Iterate}}

\begin{sloppypar}
The information in this chapter is obsolete but included for
completeness's sake; Currently, the most up-to-date information on
\iter\ can be found at |http://boinkor.net/iterate.html|.
\end{sloppypar}

\begin{sloppypar}
\iter\ currently runs on Lisp Machines, and on
HP's, Sun3's and Sparcstations under Lucid.  
\iter\ source and binaries are available at the MIT AI Lab in the
subdirectories of |/src/local/lisplib/|.  The source file,
|iterate.lisp|, is also available for anonymous FTP in the directory
|/com/fpt/pub/| on the machine |TRIX.AI.MIT.EDU| (Internet number
128.52.32.6).  If you are unable to obtain |iterate| in one of these
ways, send mail to |jba@ai.mit.edu| and I will send you the source
file. 
\end{sloppypar}

\begin{sloppypar}
\iter\ resides in the |iterate| package (nickname |iter|).  Just say
\linebreak |(use-package :iterate)| to make all the necessary symbols
available. 
If a symbol is not exported, it appears in this manual with an
``|iterate::|'' prefix.
\end{sloppypar}

Send bug reports to |bug-iterate@ai.mit.edu|.  The |info-iterate|
mailing list will have notices of changes and problems; to have
yourself added, send mail to |info-iterate-request@ai.mit.edu|.


\medskip

\begin{flushleft}
 \bf Acknowledgements
\end{flushleft}
\smallskip

Richard Waters provided invaluable criticism which spurred me to improve
\iter\ greatly.  As early users, David Clemens, Oren Etzioni and Jeff
Siskind helped ferret out many bugs.

%\begin{theindex}
% The index files must be generated with the genindex program
% or from makeindex iterate-manual.idx which creates iterate-manual.ind.

%\baselineskip\normalbaselineskip
%\advance\baselineskip by -2pt

\input{iterate-manual.ind}

%\underline{Clauses}
%\input{iterate-manual.clindex}
%
%\indexspace
%\underline{Lisp}
%\input{iterate-manual.lispindex}

%\end{theindex}


\end{document}


% arch-tag: "c0f871a6-313c-11d8-abb9-000c76244c24"


%\underline{Clauses}
%% Fix ``.'' spacing in lists
\documentstyle[12pt,aimemo]{article}

\setlength{\textwidth}{6.0in}
\setlength{\oddsidemargin}{0.25in}
\setlength{\evensidemargin}{0.25in}
\setlength{\topmargin}{0pt}
\setlength{\headsep}{24pt}
\setlength{\textheight}{8.5in}

\let\origpar=\par

% somehow, LaTex fucks with things so that the plain Tex \obeyspaces
% doesn't work.  This is the same, except we use an mbox containing
% a space instead of just a space.

\newcommand{\spc}{\mbox{ }}
\newcommand{\tAb}{\mbox{        }}
\def\obspaces{\catcode`\ =\active\catcode`\^^I=\active}
{\obspaces \global\let =\spc\global\let^^I=\tAb}

\newenvironment{program}{\medskip\samepage\tt\obeylines\obspaces}{\medskip}

% use | for \tt
\catcode`\|=13
\newif\ifvbar
\def|{\ifvbar
         \endgroup\vbarfalse
      \else 
         \begingroup\tt\vbartrue 
      \fi}



\newcommand{\lisp}{\tt}
\newcommand{\iter}{|iterate|}
\newcommand{\iterpg}{|iterate-pg|}
\newcommand{\setf}{|setf|}
\newcommand{\nil}{|nil|}
\newcommand{\nonnil}{non-\nil}
\newcommand{\opt}{{\lisp \&optional}}
\newcommand{\yields}{$\Rightarrow$}



\newbox\Dots
\setbox\Dots=\hbox{\small ...}
\renewcommand{\dots}{\copy\Dots}

% Here's a way to define an environment which will treat all
% paragraphs inside it in the same manner.  It's, as near as I can
% figure, the way that Latex does things.  Using \everypar won't work;
% apparently Latex resets that a lot.  What you have to do is: at the
% top of the file do \let\origpar=\par.  In the
% begin part of the environment, do the parshape (or whatever), then say
% \def\par{{\origpar}}.  The extra braces are necessary.   That's all
% you have to do; at the end of the environment things will return to
% normal automatically.  Why does this all work?  I have no idea.
% Another important thing: there should be a blank line (i.e. a \par)
% between the end of the last paragraph and the \end{environment}.

\newlength{\clindent}
\setlength{\clindent}{0.5in}
\newlength{\clparindent}
\setlength{\clparindent}{\parindent}

\newenvironment{clauses}{\advance\linewidth -\parindent
                        \hangindent\clindent\relax
                         \hangafter=0
                          \parindent=0pt
                          \def\par{{\origpar}}}{}

\newcommand{\cpar}{\hspace \clparindent}

\newcommand{\startitem}{\bigskip\pagebreak[3]}
\newcommand{\finishitem}{\smallskip}

\newcommand{\clause}[1]{\startitem\Clause{#1}\finishitem}
\newcommand{\Clause}[1]{\hbox{#1}}
\newcommand{\clindex}[1]{\index{{\lisp #1}}}
\newcommand{\clindexx}[2]{\index{{\lisp #1\dots #2}}}
\newcommand{\clindexxx}[3]{\index{{\lisp #1\dots #2\dots #3}}}
\newcommand{\clausex}[2]{\startitem\Clausex{#1}{#2}\finishitem}
\newcommand{\Clausex}[2]{\Clause{|#1| #2}\clindex{#1}}

\newcommand{\defvar}[1]{\startitem\deff{#1}{}{Variable}\finishitem}
\newcommand{\defunexpvar}[1]{\startitem\deffnoindex{iterate::#1}{}{Variable}
                                        \lispindex{#1}\finishitem}
\newcommand{\defun}[2]{\startitem\deff{#1}{#2}{Function}\finishitem}
\newcommand{\defmacro}[2]{\startitem\deff{#1}{#2}{Macro}\finishitem}
\newcommand{\defunx}[3]{\startitem\deffx{#1}{#2}{#3}{Function}\finishitem}
\newcommand{\defmacrox}[3]{\startitem\deffx{#1}{#2}{#3}{Macro}\finishitem}
\newcommand{\defunxx}[4]{\startitem\deffxx{#1}{#2}{#3}{#4}{Function}\finishitem}
\newcommand{\defmacroxx}[4]{\startitem\deffxx{#1}{#2}{#3}{#4}{Macro}\finishitem}

\newcommand{\deff}[3]{\deffnoindex{#1}{#2}{#3}\lispindex{#1}}
\newcommand{\deffnoindex}[3]{\hbox to \hsize{{\tt #1} {\it #2}\hfill [{\it #3}]}}
\newcommand{\deffx}[4]{\deff{#1}{#2}{#4}\moreargs{#1}{#3}}
\newcommand{\deffxx}[5]{\deffx{#1}{#2}{#3}{#5}\moreargs{#1}{#4}}

\newcommand{\moreargs}[2]{\hbox to \hsize{\phantom{\lisp #1} {\it #2}\hfill}}

\newcommand{\lispindex}[1]{\index{{\lisp #1}}}

 \newenvironment{note}[1]{\pagebreak[2]\bigskip
                          \hrule\smallskip\small
                         {\setlength{\parindent}{0pt}
                           \par{\bf #1:}}}{\smallskip\hrule\bigskip}



\makeindex

%%%%%%%%%%%%%%%%%%%%%%%%%%%%%%%%%%%%%%%%%%%%%%%%%%%%%%%%%%%%%%%%

%\thispagestyle{empty}

%\vspace*{2in}

%\title{The |Iterate| Manual}
%\author{Jonathan Amsterdam}

%\maketitle

%\thispagestyle{empty}

%\begin{center}
%\Large *** DRAFT ***
%\end{center}
%This manual is in beta-test. It will become an AI Memo in a couple of
%months.  I would appreciate any comments or
%corrections.  You can reach me at email address |jba@ai.ai.mit.edu|.

\begin{document}

\def\writtenby{Jonathan Amsterdam}
\def\contractno{This report describes research done at the Artificial
Intelligence Laboratory of the Massachusetts Institute of Technology.
Support for the laboratory's artificial intelligence research is
provided in part by the Advanced Research Projects Agency of the
Department of Defense under Office
of Naval Research contract N00014-85-K-0124.}

\AIMEMO{1236}{\Large The Iterate Manual \\[1in]}
{This is the manual for version 1.4 of \iter, a powerful iteration
macro for Common  
Lisp. \iter\ is similar to {\tt loop} but provides numerous additional
features, is well integrated with Lisp, and is extensible. 
\vfill }


% use ~ for \em
\catcode`\~=13
\newif\ifem
\def~{\ifem
         \/\endgroup\emfalse
      \else 
         \begingroup\em\emtrue 
      \fi}



%\pagebreak
%\setcounter{page}{1}

%\thispagestyle{empty}


\tableofcontents

\pagebreak
\pagestyle{headings}

\section{Introduction}

\begin{sloppypar}
This manual describes
\iter, a powerful iteration facility for Common Lisp.
\iter\ provides abstractions for many common iteration
patterns and allows for the definition of additional patterns.  \iter\ is
a macro that expands into ordinary Lisp at
compile-time, so it is more efficient than higher-order
functions like |map| and |reduce.|  
While it is similar to |loop|, \iter\ offers a more Lisp-like syntax
and enhanced extensibility.
(For a more complete
comparison of \iter\ with other iteration constructs, see MIT AI Lab Working
Paper No. 324, ~Don't Loop, Iterate.~)
\end{sloppypar}

An \iter\ form consists of the symbol |iter|\footnote{You can also use
|iterate|, but |iter| is preferred because it avoids potential conflicts with
possible future additions to Common Lisp, and because it saves
horizontal space when writing code.}
followed by one or more
forms, some of which may be \iter\ 
~clauses.~  Here is a simple example of \iter\ which collects the
numbers from 1 to 10 into a list, and returns the list.  The return
value is shown following the arrow.
\begin{program}
(iter (for i from 1 to 10)
      (collect i))          \yields (1 2 3 4 5 6 7 8 9 10)
\end{program}
This form contains two clauses: a |for| clause that steps the
variable |i| over the integers from 1 to 10, and a |collect|
clause that accumulates its argument into a list.  With a few
exceptions, all \iter\ clauses have the same format: alternating
symbols (called ~keywords~) and expressions (called
~arguments~).  The syntax and terminology are those of Common
Lisp's keyword lambda lists.  One difference is that \iter's keywords
do not have to begin with a colon---though they may, except for the
first symbol of a clause.  So you can also write  |(for i :from
1 :to 10)| if you prefer.

Any Lisp form can appear in the body of an \iter, where it will have
its usual meaning.  \iter\ walks the entire body, expanding macros,
and recognizing clauses at any level.  This example collects all the
odd numbers in a list:
\begin{program}
(iter (for el in list)
      (if (and (numberp el) (oddp el))
          (collect el)))
\end{program}

There are clauses for iterating over numbers, lists, arrays and other
objects, and for
collecting, summing, counting, maximizing and other useful operations.
\iter\ also supports the creation of new variable bindings, stepping
over multiple sequences at once, destructuring, and compiler
declarations of variable types.  The following example illustrates
some of these features: 
\begin{program}
(iter (for (key . item) in alist)
         (for i from 0)
         (declare (fixnum i))
         (collect (cons i key)))
\end{program}

This loop takes the keys of an alist and returns a new alist
associating the keys with their positions in the original list. The
compiler declaration for |i| will appear in the generated code
in the appropriate place.

%%%%%%%%%%%%%%%%%%%%%%%%%%%%%%%%%%%%%%%%%%%%%%%%%%%%%%%%%%%%%%%%
\section{Clauses}

Most of \iter's clauses will be familiar to |loop| programmers.  
(|loop| is an iteration macro that has been incorporated into Common
Lisp.  Seen Guy Steele's {\em Common Lisp, 2nd Edition}.)
In
nearly all cases they behave the same as their |loop| counterparts, so
a |loop| user can switch to \iter\ with little pain (and much gain).

All clauses with the standard keyword-argument syntax consist of two
parts: a ~required~ part, 
containing keywords that must be present and in the right order; and
an ~optional~ part, containing keywords that may be omitted and,
if present, may occur in any order.  In the descriptions below, the
parts are separated by the Lisp lambda-list keyword \opt.


\subsection{Drivers}

An iteration-driving clause
conceptually causes the iteration to go forward.  Driver clauses in
\iter\ allow iteration over numbers, lists, vectors, hashtables, packages,
files and streams.  Iteration-driving clauses must
appear at the top level of an \iter\ form; they cannot be nested
inside another clause.  The driver variable is updated at the point
where the driver clause occurs.  Before the clause is executed for the
first time, the value of the variable is undefined.

%Also, regardless of where the driver clause
%appears in the body, the driver variable is stepped at the top of the
%loop; hence it is stylistically preferable, though
%not required, to place driver clauses at the beginning of the \iter.

Multiple drivers may appear in a single \iter\ form, in which case all
of the driver variables are updated each time through the loop, in the
order in which the clauses appear.  The first driver to terminate will
terminate the entire loop.

In all cases, the value of the driver variable on exit from the loop,
including within the epilogue code (see the |finally| clause), is
undefined.

All the parameters of a driver clause are evaluated once, before the
loop begins.  Hence it is not possible to change the bounds or other
properties of an iteration by side-effect from within the loop.

With one exception, driver clauses begin with the word |for| (or
the synonym |as|) and mention an iteration variable, which is
given a binding within the \iter\ form.
The exception is |repeat|, which just executes a loop a
specified number of times:

\begin{clauses}

\clausex{repeat}{~n~}
Repeats the loop ~n~ times.  For example:
\begin{program}
(iter (repeat 100)
      (print "I will not talk in class."))
\end{program}
\cpar If $n \leq 0$, the
loop will never be executed.  If ~n~ is not an integer, the
actual number of executions will be $\lceil n \rceil$.

\end{clauses}

\pagebreak[3]

\subsubsection{Numerical Iteration}

\begin{clauses}
\clausex{for}{~var~ |\&sequence|} 
The general form for iterating over a sequence of numbers requires a
variable and, optionally, one or more keywords that provide the bounds
and step size of the iteration. The |\&sequence|
lambda-list keyword is a shorthand for these sequence keywords.  They are:
|from|, |upfrom|, |downfrom|, |to|, |downto|, |above|, |below| and |by|.
|from| provides the starting value for ~var~
and defaults to zero. |to| provides a final value and implies that the
successive values of ~var~ will be increasing; |downto| implies that
they will be decreasing.  The loop terminates when ~var~ passes
the final value (i.e. becomes smaller or larger than it, depending on
the direction of iteration); in other  words, the loop body
will never be 
executed for values of ~var~ past the final value.
|below| and |above| are similar to |to| and
|downto|, except that the loop terminates when ~var~ equals or passes
the final value.

\cpar If no final value is specified, the variable will be stepped
forever.  Using |from| or |upfrom| will result in increasing
values, while |downfrom| will give decreasing values.

\cpar On each iteration,
~var~ is incremented or decremented by the value of the sequence
keyword |by|, which defaults to 1.  It should always be a positive
number, even for downward iterations.

\cpar In the following examples, the sequence of numbers generated is shown
next to the clause.

\begin{program}
(for i upfrom 0) \yields\ 0 1 2 \ldots
(for i from 5)   \yields\ 5 6 7 \ldots    ; either from or upfrom is okay
(for i downfrom 0) \yields\ 0 -1 -2 \ldots
(for i from 1 to 3) \yields\ 1 2 3
(for i from 1 below 3) \yields\ 1 2
(for i from 1 to 3 by 2) \yields\ 1 3
(for i from 1 below 3 by 2) \yields\ 1
(for i from 5 downto 3) \yields\ 5 4 3
\end{program}

\end{clauses}

\subsubsection{Sequence Iteration}

There are a number of clauses for iterating over sequences.  In all of
them, the argument following |for| may be a list instead of a
symbol, in which case destructuring is performed.  See section
\ref{destructuring}. 

\begin{clauses}

\clause{|for| ~var~ |in| ~list~ \opt\ |by| ~step-function~}
\clindexx{for}{in}
~var~ is set to successive elements of list.  
~step-function,~ which 
defaults to |cdr|, is used to obtain the next sublist. 

\clause{|for| ~var~ |on| ~list~ \opt\ |by| ~step-function~}
\clindexx{for}{on}
~var~ is set to successive sublists of list. 
~step-function~ (default |cdr|) is used as in |for\dots in|.

\end{clauses}

\medskip

These two clauses use |atom| to test for the end
of a list.  Hence, given a list whose final |cdr| is not \nil,
they will silently ignore the last |cdr|.  Other choices are
|endp|,  which would signal an error, and |null|, which
would probably result in an error somewhere else.  If you wish to use
an end-test other than |atom|, set the variable
|iterate::*list-end-test*|\lispindex{*list-end-test*} to the name of
the desired function. 

\begin{clauses}
\clause{|for| ~var~ |in-vector| ~vector~ |\&sequence|}
\clindexx{for}{in-vector}
~var~ takes on successive elements from ~vector.~   The vector's
fill-pointer is observed.  Here and in subsequent clauses, the |\&sequence|
keywords include |with-index|, which takes a symbol as argument
and uses it for the index variable instead of an internally generated
symbol.  The other |\&sequence| keywords behave as in numerical
iteration, except that the default iteration bounds are the bounds of
the vector.  E.g. in |(for i in-vector v downto 3)|,
|i| will start off being bound to the last element in |v|,
and will  
be set to preceding elements down to and including the element with
index 3.

\clause{|for| ~var~ |in-sequence| ~seq~ |\&sequence|}
\clindexx{for}{in-sequence}
This uses Common Lisp's generalized sequence functions, |elt|
and |length|, to obtain elements and determine the length of
~seq.~   Hence it will work for any sequence, including lists, and will
observe the fill-pointers of vectors.

\clause{|for| ~var~ |in-string| ~string~ |\&sequence|}
\clindexx{for}{in-string}
~var~ is set to successive characters of ~string.~

\startitem
\Clause{|for| ~var~ |index-of-vector| ~vector~ |\&sequence|}
\clindexx{for}{index-of-vector}
\Clause{|for| ~var~ |index-of-sequence| ~sequence~ |\&sequence|}
\clindexx{for}{index-of-sequence}
\Clause{|for| ~var~ |index-of-string| ~string~ |\&sequence|}
\clindexx{for}{index-of-string}
\finishitem
~var~ is set to successive indices of the sequence.
These clauses avoid the overhead of accessing the sequence elements
for those applications where they do not need to be examined, or are
examined rarely.  They admit all the optional keywords of the other
sequence drivers except the (redundant) |with-index| keyword.

\clause{|for| ~(key value)~ |in-hashtable| ~table~}
\clindexx{for}{in-hashtable}

~key~ and ~value,~ which must appear as shown in a list
and may be destructuring templates, are set to the keys and values of
~table~.  If ~key~ is |nil|, then the hashtable's keys will be ignored;
similarly for ~value~.
The order in which 
elements of ~table~ will be retrieved is unpredictable.

\clause{|for| ~var~ |in-package| ~package~ \opt\ |external-only| ~ext~}
\clindexx{for}{in-package}
Iterates over all the symbols in ~package,~ or over only the
external symbols if ~ext~ is specified and non-|nil|.  ~ext~ is not
evaluated.  The same symbol may appear more than once.

\clause{|for| ~var~ |in-packages| ~packages~ \opt\ |having-access| ~symbol-types~}
\clindexx{for}{in-packages}
Iterates over all the symbols from the list of packages denoted by the
descriptor ~packages~ and having accessibility (or visibility) given by
~symbol-types~.  This defaults to the list |(:external :internal :inherited)|
and is not evaluated.  ~var~ must be a list of up to three variables: in each
iteration, these will be set to a symbol, its access-type and package (as per
|with-package-iterator| in ANSI-CL).  The same symbol may appear more than
once.

\end{clauses}

\begin{clauses}

\clause{|for| ~var~ |in-file| ~name~ \opt\ |using| ~reader~}
\clindexx{for}{in-file}
Opens the file ~name~ (which may be a string or pathname) for input,
and iterates 
over its contents. ~reader~ defaults to |read|, so by default ~var~
will be bound to the successive forms in the file.  The \iter\ body is
wrapped in an unwind-protect to ensure that the file is closed no
matter how the \iter\ is exited.

\clause{|for| ~var~ |in-stream| ~stream~ \opt\ |using| ~reader~}
\clindexx{for}{in-stream}
Like |for\dots in-file|, except that ~stream~ should be an existing
stream object that supports input operations.

\end{clauses}

\subsubsection{Generalized Drivers}

These are primarily useful for writing drivers that can also be used
as generators (see section \ref{generators}, below).
 
\begin{clauses}

\clause{|for| ~var~ |next| ~expr~}
\clindexx{for}{next}

~var~ is set to ~expr~ each time through the loop.  Destructuring is
performed.  When the clause is used as a generator, ~expr~ is the code
that is executed when |(next ~var~)| is 
encountered (see section \ref{generators}, below).
~expr~ should compute the first value for ~var~, as well
as all subsequent values, and is responsible for terminating the loop.
For compatibility with future versions of \iter, this termination
should be done with |terminate|\clindex{terminate}, which can be
considered a synonym for |finish| (see section \ref{control-flow}).

\cpar As an example, the following clauses are equivalent to |(for
i from 1 to 10)|:
\begin{program}
(initially (setq i 0))
(for i next (if (> i 10) (terminate) (incf i)))
\end{program}

\clause{|for| ~var~ |do-next| ~form~}
\clindexx{for}{do-next}
~form~ is evaluated each time through the loop.  Its value is ~not~
set to ~var~; that is ~form's~ job. ~var~ is only present so that
\iter\ knows it is a driver variable.  \linebreak |(for ~var~ next
~expr~)| is equivalent to |(for ~var~ do-next (dsetq ~var~ ~expr~))|.
(See section \ref{destructuring} for an explanation of |dsetq|.) 

\end{clauses}

\subsubsection{Generators}
\label{generators}

In all of the above clauses, the driver variable is updated on each
iteration.
Sometimes it is desirable to have greater control over updating.  
For instance, consider the problem of associating numbers, in
increasing order and with no gaps, with the
\nonnil\ elements of a list.  One obvious first pass at writing this is:

\begin{program}
(iter (for el in list)
      (for i upfrom 1)
      (if el (collect (cons el i))))
\end{program}
But on the list |(a b nil c)| this produces |((a . 1) (b . 2) (c .
4))| instead of the desired |((a . 1) (b . 2) (c . 3))|.  The problem
is that |i| is incremented each time through the loop, even when |el|
is |nil|.  

The problem could be solved elegantly if we could step |i| only when
we wished to.  This
can be accomplished for any \iter\ driver by writing |generate| (or
its synonym |generating|)
instead of |for|.  Doing so produces a ~generator~---a driver whose
values are yielded explicitly.  To obtain the next value of a
generator variable ~v~, write \linebreak |(next ~v~)|.  The value of a |next|
form is the next value of ~v~, as determined by its associated driver
clause.  |next| also has the side-effect of updating ~v~ to that
value.  If there is no next value, |next| will terminate the loop,
just as with a normal driver.


Using generators, we can now write our example like this:
\begin{program}
(iter (for el in list)
      (generate i upfrom 1)
      (if el (collect (cons el (next i)))))
\end{program}
Now |i| is updated only when |(next i)| is executed, and this occurs
only when |el| is \nonnil.

To better understand the relationship between ordinary drivers and
generators, observe that we can rewrite an ordinary driver using its
generator form immediately followed by |next|, as this example shows:
\begin{program}
(iter (generating i from 1 to 10)
      (next i)
      ...)
\end{program}
Provided that the loop body contains no |(next i)| forms, this will
behave just as if we had written |(for i from 1 to 10)|.

We can still refer to a driver variable ~v~ without using |next|; in
this case, its value is that given to it by the last evaluation of
|(next ~v~)|.  Before |(next ~v~)| has been called the first time, the
value of ~v~ is undefined.  

This semantics is more flexible than
one in which ~v~ begins the loop bound to its first value and calls
of |next| supply subsequent values, because it means the loop will not
terminate too soon if the generator's sequence is empty.  For
instance, consider the following code, which tags \nonnil\ elements of
a list using a list of tags, and also counts the null elements.
(We assume there are at least as many tags as \nonnil\ elements.)
\begin{program}
(let* ((counter 0)
       (tagged-list (iter (for el in list)
                          (generating tag in tag-list) 
                          (if (null el)
                              (incf counter)
                              (collect (cons el (next tag)))))))
  ...)
\end{program}
It may be that there are just as many tags as non-null elements of
|list|.  If all the elements of |list| are null, we still want the
counting to proceed, even though |tag-list| is \nil.  If |tag| had to be
assigned its first value before the loop begins, we would have had to
terminate the loop before the first iteration, since when |tag-list|
is \nil, |tag| 
has no first value.  With the existing semantics, however, |(next
tag)| will never execute, so the iteration will cover all the elements of
|list|. 
        
When the ``variable'' of a driver clause is actually a destructuring
template containing several variables, all the variables are eligible
for use with |next|.  As before, |(next ~v~)| evaluates to ~v's~ next
value; but the effect is to update all of the template's variables.
For instance, the following code will return the list |(a 2 c)|.
\begin{program}
(iter (generating (key . item) in '((a . 1) (b . 2) (c . 3)))
      (collect (next key))
      (collect (next item)))
\end{program}

Only driver clauses with variables can be made into generators.  This
includes all clauses mentioned so far except for |repeat|.  It does
~not~ include |for\dots previous|, |for\dots =|,
|for\dots initially\dots then| or |for\dots first\dots then| (see
below). 

\subsubsection{Previous Values of Driver Variables}

Often one would like to access the value of a variable on a previous
iteration.  \iter\ provides a special clause for accomplishing this.

\begin{clauses}

\clause{|for| ~pvar~ |previous| ~var~ \opt\ |initially| ~init~
|back| ~n~}
\clindexx{for}{previous}
Sets ~pvar~ to the previous value of ~var,~ which should be a driver
variable, a variable from another |for\dots previous| clause, or a
variable established by a |for\dots =|, 
|for\dots initially\dots then| or |for\dots first\dots then| clause
(see section \ref{setting}).
Initially, ~pvar~ is given the value ~init~ (which defaults to \nil).  
The ~init~ expression will be moved outside the loop body, so it
should not depend on anything computed within the loop.
~pvar~ retains the value of ~init~ until ~var~ is set to its second
value, at which point ~pvar~ is set to ~var's~ first value; and so on.  

\cpar The
argument ~n~ to |back| must be a constant, positive integer, and
defaults to 1.  It determines how many iterations back ~pvar~ should
track ~var~.  For example, when ~n~ is 2, then ~pvar~ will be assigned
~var's~ first value when ~var~ is set to its third value.

\cpar A |for\dots previous| clause may occur after or before its
associated driver clause. |for\dots previous| works with generators as
well as ordinary drivers.

\pagebreak[3]

\cpar Example:
\begin{program}
(iter (for el in '(1 2 3 4))
      (for p-el previous el)
      (for pp-el previous p-el initially 0)
      (collect pp-el))
\end{program}
This evaluates to |(0 0 1 2)|.  It could have been written more
economically as 
\begin{program}
(iter (for el in '(1 2 3 4))
      (for pp-el previous el back 2 initially 0)
      (collect pp-el))
\end{program}


\end{clauses}

\subsection{Variable Binding and Setting}
\label{setting}

Several clauses exist for establishing new variable bindings or for
setting variables in the loop.  They all support destructuring.

\begin{clauses}

\clausex{with}{~var~ \opt\ |=| ~value~}
Causes ~var~ to be bound to value before the loop body is entered.
If ~value~ is not supplied, ~var~ assumes a default
binding, which will be 
\nil\ in the absence of declarations.  Also, if ~value~ is not
supplied, no destructuring is performed; instead, ~var~ may be a list of
symbols, all of which are given default 
bindings.  If ~value~ is supplied, ~var~ is bound to it, with
destructuring.  

\cpar Because |with|
creates bindings whose scope includes the entire \iter\ form, it is
good style to put all |with| clauses at the beginning.

\cpar Successive occurrences of |with| result in sequential
bindings (as with 
|let*|).  There is no way to obtain parallel bindings; see
section \ref{bindings} for a rationale.  


\clause{|for| ~var~ |=| ~expr~}
\clindexx{for}{=}
On each iteration, ~expr~ is evaluated and ~var~ is set
to its value. 

\cpar This clause may appear to do the same thing as |for\dots next|.
In fact, they are quite different.  |for\dots =| provides only three
services: it sets up a binding for ~var~, sets it to ~expr~ on each
iteration, and makes it possible to use |for\dots previous| with
~var~.  |for\dots next| provides these services in addition to the
ability to turn the driver into a generator.
%Also, the code which sets ~var~ appears in the loop body in the same
%place as the |for\dots =| clause; the code for |for\dots next| appears
%at the top of the loop, as with other drivers (except when being used
%as a generator).

\clause{|for| ~var~ |initially| ~init-expr~ |then| ~then-expr~}
\clindexxx{for}{initially}{then}
Before the loop begins, ~var~ is set to ~init-expr;~ on all
iterations after the first it is set to ~then-expr.~  This clause
must occur at top-level.  ~init-expr~ will be moved outside the loop
body and ~then-expr~ will be moved to the end of the loop body, so
they are subject to code motion problems (see section
\ref{code-movement}). 

\cpar This clause may appear to be similar to |for\dots next|, but in
fact they differ significantly.  |for\dots initially\dots then| is
typically used to give ~var~ its first value before the loop begins,
and subsequent values on following iterations.  This is incompatible
with generators, whose first value and subsequent values must all be
computed by |(next ~var~)|.  Also, the update of ~var~ in 
|for\dots initially\dots then| does not occur at the location of the
clause. 
Use |for\dots initially\dots then| for
one-shot computations where its idiom is more convenient, but use
|for\dots next| for extending \iter\ with new drivers (see section
\ref{extend}).

\clause{|for| ~var~ |first| ~first-expr~ |then| ~then-expr~}
\clindexxx{for}{first}{then}
The first time through the loop, ~var~ is set to ~first-expr;~ on
subsequent iterations, it is set to ~then-expr.~  This differs from
|for\dots initially| in that ~var~ is set to ~first-expr~
inside the loop body, 
so ~first-expr~ may depend on the results of other clauses.  For
instance, 
\begin{program}
(iter (for num in list)
      (for i first num then (1+ i))
      ...)
\end{program}
will set |i| to the first element of |list| on the first
iteration, whereas 
\begin{program}
(iter (for num in list)
      (for i initially num then (1+ i))
      ...)
\end{program}
is probably erroneous; |i| will be bound to |num|'s
default binding (usually \nil) for the first iteration.

\end{clauses}

\begin{note}{Compatibility note}
|loop|'s |for\dots =| works like \iter's, but |loop| used the syntax
|for\dots =\dots then| to mean |for\dots initially\dots then|.  It was
felt that these two operations were sufficiently different to warrant
different keywords.

Also, the |for| in the above three clauses is misleading,
since none is true driver (e.g. none has a corresponding |generate|
form).  |setting| would have been a better choice, but |for| was used
to retain some compatibility with |loop|.  
\end{note}


%%%%%%%%%%%%%%%%%%%%%%%%%%%%%%%%
\subsection{Gathering Clauses}
Many of \iter's clauses accumulate values into a variable, or set a
variable under certain conditions.  At the end of the loop, this
variable contains the desired result.  All these clauses have an
optional |into| keyword, whose argument should be a symbol.  If the
|into| keyword is not supplied, the accumulation variable will be
internally generated and its value will be returned at the end of the
loop; if a variable is specified, that variable is used for the
accumulation, and is not returned as a result---it is up to the user
to return it explicitly, in the loop's epilogue code (see |finally|).
It is safe to examine the accumulation variable during the loop, but
it should not be modified.

These clauses all begin with a verb.  When the verb does
not conflict with an existing Common Lisp function, then it may be
used in either its infinitival or present-participle form (e.g. |sum|,
|summing|).  However, when there is a conflict with Common Lisp, only
the present-participle form may be used (e.g. |unioning|).  This is to
prevent \iter\ clauses from clashing with Common Lisp
functions. 

%although these clauses are described as ``producing a value,'' it is a
%mistake to think of the lisp list representing the clause as a
%value-producing form in the usual way.  clauses may legally be written
%where a value is expected, e.g. |(setq x (sum i))|, but the
%lisp value of a clause in such a context is undefined.

\subsubsection{Reductions}

~Reduction~ is an extremely common iteration pattern in which
the results of successive applications of a
binary operation are accumulated.
For example, a loop that computes the sum of the
elements of a list is performing a reduction with the addition
operation.  This could be written in Common Lisp as  |(reduce \#'+
list)| or with \iter\ as
\begin{program}
(iter (for el in list)
      (sum el))
\end{program}

\begin{clauses}

\clausex{sum}{~expr~ \opt\ |into| ~var~}
Each time through the loop, ~expr~ is evaluated and added to a
variable, which is bound initially to zero.  If ~expr~ has a type,
it is ~not~ used as the type of the sum variable, which is always
|number|.  To get the result variable to be of a more specific
type, use an explicit variable, as in
\begin{program}
(iter (for el in number-list)
      (sum el into x)
      (declare (fixnum x))
      (finally (return x)))
\end{program}

\clausex{multiply}{~expr~ \opt\ |into| ~var~}
Like |sum|, but the initial value of the result variable is $1$,
and the variable is updated by multiplying ~expr~ into it.

\clausex{counting}{~expr~ \opt\ |into| ~var~}
~expr~ is evaluated on each iteration.  If it is \nonnil, the
accumulation variable, initially zero, is incremented.

\startitem
\Clausex{maximize}{~expr~ \opt\ |into| ~var~}
\Clausex{minimize}{~expr~ \opt\ |into| ~var~}
\finishitem
~expr~ is evaluated on each iteration and its extremum (maximum or
minimum) is stored in the accumulation variable.  If ~expr~ is never
evaluated, then the result is \nil\ (if the accumulation variable is
untyped) or $0$ (if it has a numeric type).

\clausex{reducing}{~expr~ |by| ~func~ \opt\ |initial-value| ~init-val~ |into| ~var~}
This is a general way to perform reductions. ~func~ should be a
function of two arguments, the first of which will be the value
computed so far and
the second of which will be the value of ~expr~.  It should return the
new value.
|reducing| is roughly equivalent to the
Common Lisp |(reduce ~func~ ~sequence~ :key ~expr-function~)|, where
~expr-function~ is used to derive values from the successive elements
of ~sequence~.

\cpar If the |reducing| clause is never executed, the result is
undefined.

\cpar It is not necessary to provide an
initial value, but better code can be generated if one is supplied.
Regardless of its location in the \iter\ body, ~init-val~ will be
evaluated before the loop is entered, so it should not depend on any
value computed inside the \iter\ form.  

%\cpar if a ~var~ is not specified, you can get \iter\ to declare the
%type of the internal variable by putting a |the| expression around
%~func.~  see section \ref{types}.

\end{clauses}

%\begin{note}{implementation note}
%in principle, |maximize| and |minimize| can be thought of
%as reductions where the initial value is the smallest (or largest)
%value that the accumulation variable can assume.  because lisp's
%bignums can represent arbitrary integers, these clauses cannot be
%implemented as reductions in general.  if, however, the type of 
%~expr~ or ~var~ can be determined to be a fixnum
%or a float, \iter\ will implement the clause as a true reduction,
%using one of the constants |most-negative-fixnum|, |%most-positive-fixnum|, 
%|most-negative-short-float|, etc. as appropriate.
%\end{note}

\subsubsection{Accumulations}

All the predefined accumulation clauses add values to a sequence.  If
the sequence is a list, they all 
have the property that the partial list is kept in the correct order
and available for inspection at any point in the loop.

\begin{clauses}

\clausex{collect}{~expr~ \opt\ |into| ~var~ |at| ~place~ |result-type| ~type~}
Produces a sequence of the values of ~expr~ on each iteration. ~place~
indicates where the next value of ~expr~ is added to the list and may
be one of the symbols |start|, |beginning| (a synonym for |start|) or
|end|.  The symbol may be quoted, but need not be.  The default is
|end|.  For example,
\begin{program}
(iter (for i from 1 to 5)
      (collect i))
\end{program}
produces |(1 2 3 4 5)|, whereas
\begin{program}
(iter (for i from 1 to 5)
      (collect i at beginning))
\end{program}
produces |(5 4 3 2 1)| (and is likely to be faster in most Common Lisp
implementations).

\cpar If ~type~ is provided, it should be a subtype of |sequence|.
The default is |list|.  Specifying a type other than |list| will
result in |collect| returning a sequence of that type.  ~However,~ the
type of the sequence being constructed when inside the loop body is
undefined when a non-|list| type is specified.  (As with ~place~,
quoting ~type~ is optional.)

\clausex{adjoining}{~expr~ \opt\ |into| ~var~ |test| ~test~ |at| ~place~ |result-type| ~type~}
Like |collect|, but only adds the value of ~expr~ if it is not
already present.  ~test,~ which defaults to |\#'eql|, is
the test to be used with |member|.

\startitem
\Clausex{appending}{~expr~ \opt\ |into| ~var~ |at| ~place~}
\Clausex{nconcing}{\ ~expr~ \opt\ |into| ~var~ |at| ~place~}
\Clausex{unioning}{\ ~expr~ \opt\ |into| ~var~ |test| ~test~ |at| ~place~}
\Clausex{nunioning}{~expr~ \opt\ |into| ~var~ |test| ~test~ |at| ~place~}
\finishitem
These are like |collect|, but behave like the Common Lisp functions
|append|, |nconc|, |union| or |nunion|.  
As in Common Lisp, they work only on lists.  Also as in Common Lisp,
|unioning| and |nunioning| assume that the value of ~expr~ contains no
duplicates. 

\clausex{accumulate}{~expr~ |by| ~func~ \opt\ |initial-value| ~init-val~ |into| ~var~}
This is a general-purpose accumulation clause. ~func~ should be a
function of two arguments, the value of ~expr~ and the value
accumulated so far in the iteration, and it should return the updated
value.  If no initial value is supplied, \nil\ is used.  

%\cpar If a ~var~ is not specified, you can get \iter\ to declare the
%type of the internal variable by putting a |the| expression around
%~func.~  see section \ref{types}.

\cpar The differences between |accumulate| and |reducing| are slight.
One difference is that the functions take their arguments in a
different order.  Another is that in the absence of ~init-val~,
|accumulate| will use \nil, whereas |reducing| will generate different
code that avoids any dependence on the initial value.
The reason for having both clauses is that one usually
thinks of reductions (like |sum|) and accumulations (like |collect|)
as different beasts.

\end{clauses}


\subsubsection{Finders}

A ~finder~ is a clause whose value is an expression that meets some
condition.

\begin{clauses}

\clause{|finding| ~expr~ |such-that| ~test~ \opt\ |into| ~var~ |on-failure| ~failure-value~}
\clindexx{finding}{such-that}
If ~test~ (which is an expression) ever evaluates to \nonnil, the loop
is terminated, the 
epilogue code is run and the value of ~expr~ is returned.  Otherwise,
\nil\ (or ~failure-value,~ if provided) is returned.  If ~var~ is
provided, it will have either the \nonnil\ value of ~expr~ or
~failure-value~ when the epilogue code is run.

\cpar As a special case, if the ~test~ expression is a sharp-quoted
function, then it is applied to ~expr~ instead of being simply
evaluated.  E.g. |(finding x such-that \#'evenp)| is equivalent to
|(finding x such-that (evenp x))|.  

%\cpar although ~test~ need have nothing to do with ~%expr~ as in
%|(finding j such-that (> i 3))|, it usually 
%will: |(finding (length el) such-that (oddp (length el)))|.  to
%avoid performing the |length| computation twice, you could write
%|(finding (length el) such-that \#'oddp)| or |(finding (length
%el) such-that 'oddp)|; for these cases, \iter\ generates code that
%executes ~expr~ only once.  the code for |\#'oddp|
%is slightly different from that for {\lisp 'oddp}; see the discussion
%under {\lisp for\dots in} and {\lisp for\dots on}.

\cpar |On-failure| is a misnomer. Because it is always evaluated, it behaves
more like the default third argument to the |gethash| function. As a result,
|on-failure (error "Not found")| makes no sense. Instead, the clauses |leave|
or |thereis| can be used in conjunction with |finally| as follows:
\begin{program}
(iter (for x in '(1 2 3))
      (if (evenp x) (leave x))
      (finally (error "not found")))
\end{program}

\cpar This clause may appear multiple times when all defaults are
identical. It can also be used together with either |always|/|never| or
|thereis| if their defaults match. More specifically, |on-failure nil| is
compatible with |thereis|, while |on-failure t| is compatible with |always|
and |never| clauses.
\begin{program}
(iter (for i in '(7 -4 2 -3))
      (if (plusp i)
	  (finding    i  such-that (evenp i))
          (finding (- i) such-that (oddp i))))
\end{program}

\startitem
\Clause{|finding| ~expr~ |maximizing| ~m-expr~ \opt\ |into| ~var~}
\clindexx{finding}{maximizing}
\Clause{|finding| ~expr~ |minimizing| ~m-expr~ \opt\ |into| ~var~}
\clindexx{finding}{minimizing}
\finishitem
Computes the extremum (maximum or minimum) value of ~m-expr~ over all
iterations, and returns the value of ~expr~ corresponding to the
extremum.  ~expr~ is evaluated inside the loop at the time the new
extremum is established.  If ~m-expr~ is never evaluated (due to, for
example, being embedded in a conditional clause), then the returned
value depends on the type, if any, of ~expr~ (or ~var,~ if
one is supplied).  If there is no type, the returned
value will be \nil; if the type is numeric, the returned value will be
zero.

\cpar For these two clauses, ~var~ may be a list of two
symbols; in that case, the first is used to record ~expr~ and
the second, ~m-expr.~  

\cpar As with |finding\dots such-that|, if ~m-expr~ is a sharp-quoted
function, then it is called on ~expr~ instead of being evaluated.

\end{clauses}

\subsubsection{Boolean Tests}

\begin{clauses}

\clausex{first-iteration-p}{}
Returns |t| in the first cycle of the loop, otherwise \nil.

\clausex{first-time-p}{}
Returns |t| the first time the expression is evaluated, and then \nil\ forever.
This clause comes handy when printing (optional) elements separated
by a comma:

\begin{program}
(iter (for el in '(nil 1 2 nil 3))
      (when el
        (unless (first-time-p)
          (princ ", "))
        (princ el)))
\end{program}
produces |"1, 2, 3"|.

\end{clauses}

\subsubsection{Aggregated Boolean Tests}

\begin{clauses}

\clausex{always}{~expr~}
If ~expr~ ever evaluates to
\nil, then \nil\ is immediately returned; the epilogue code is not
executed.  If ~expr~ never evaluates to \nil, the epilogue code
is executed and the last value of ~expr~ (or |t| if ~expr~ was never
evaluated) is returned (whereas |loop| would constantly return |t|).

% mention last evaluated clause when multiple always clauses?

\clausex{never}{~expr~}
Like |(always (not ~expr~))|, except it does not influence the last
value returned by a possible other |always| clause. That is,
\begin{program}
(iter (repeat 2)
      (always 2)
      (never nil)) \yields 2 ; not t
\end{program}

\clausex{thereis}{~expr~}
If ~expr~ is ever \nonnil,
its value is immediately returned without running epilogue code.
Otherwise, the epilogue code is performed and \nil\ is returned.

This clause cannot be used together with |always| or |never|, because their
defaults are opposed (similarly, |(loop always 3 thereis nil)| refuses to
compile in some implementations of |loop|).

\end{clauses}


\subsection{Control Flow}
\label{control-flow}
Several clauses can be used to alter the usual flow of control in a loop.

Note: the clauses of this and subsequent sections don't adhere to \iter's
usual syntax, but instead use standard Common Lisp syntax.  Hence the
format for describing syntax subsequently is
like the standard format used in the Common Lisp manual, not like the
descriptions of clauses above.

\begin{clauses}

\clausex{finish}{}
Stops the loop and runs the epilogue code.

%for example:
%
%\begin{program}
%(iter (with answer = nil)
%         (initially (make-a-mess))
%         (for i from 1 to 10)
%         (when (correct? i)
%           (setq answer i)
%           (finish))
%         (finally (cleanup)))
%\end{program}
%
%this code will execute |cleanup| whether or not the test |(correct?
%i)| ever succeeds. 
%the (more elegant) formulation,
%\begin{program}
%(iter (initially (make-a-mess))
%         (for i from 1 to 10)
%         (finding i such-that (correct? i))
%         (finally (cleanup)))
%\end{program}
%would not execute |cleanup| if |(correct? i)| succeeded; it
%would do an immediate return.

\clausex{leave}{\opt\ ~value~}
Immediately returns ~value~ (default \nil) from the current \iter\
form, skipping the epilogue code.  Equivalent to using |return-from|.

\clausex{next-iteration}{}
Skips the remainder of the loop body and begins the next iteration of
the loop.

\clausex{while}{~expr~}
If ~expr~ ever evaluates to \nil, the loop is terminated and the
epilogue code executed.  Equivalent to |(if (not ~expr~) (finish))|.

\clausex{until}{~expr~}
Equivalent to |(if ~expr~ (finish))|.

\clausex{if-first-time}{~then~ \opt\ ~else~}
If this clause is being executed for the first time in this invocation
of the \iter\ form, then the ~then~ code is evaluated; otherwise the
~else~ code is evaluated.

\cpar |(for ~var~ first ~expr1~ then ~expr2~)| is almost equivalent to
\begin{program}
(if-first-time (dsetq ~var~ ~expr1~)
               (dsetq ~var~ ~expr2~))
\end{program}
The only difference is that the |for| version makes ~var~ available
for use with |for\dots previous|.

\end{clauses}

\subsection{Code Placement}
When fine control is desired over where code appears in a loop
generated by \iter, the following special clauses may be useful.
They are all subject to code-motion problems (see section
\ref{code-movement}).  

\begin{clauses}

\clausex{initially}{|\&rest| ~forms~}
The lisp ~forms~ are placed in the prologue section of the loop, where
they are executed once, before the loop body is entered.

\clausex{after-each}{|\&rest| ~forms~}
The ~forms~ are placed at the end of the loop body, where they
are executed after each iteration.  Unlike the other clauses in this
section, ~forms~ may contain \iter\ clauses.

\clausex{else}{|\&rest| ~forms~}
The lisp ~forms~ are placed in the epilogue section of the loop, where they
are executed if this |else| clause is never met during execution of the
loop and the loop terminates normally.

\clausex{finally}{|\&rest| ~forms~}
The lisp ~forms~ are placed in the epilogue section of the loop, where
they are executed after the loop has terminated normally.

\clausex{finally-protected}{|\&rest| ~forms~}
The lisp ~forms~ are placed in the second form of an unwind-protect
outside the loop.  They are always executed after the loop has
terminated, regardless of how the termination occurred.

\end{clauses}


%%%%%%%%%%%%%%%%%%%%%%%%%%%%%%%%%%%%%%%%%%%%%%%%%%%%%%%%%%%%%%%%
\section{Other Features}

\subsection{Multiple Accumulations}
\label{multiple}

\begin{sloppypar}
It is permitted to have more than one clause accumulate into the same
variable, as in the following:
\begin{program}
(iter (for i from 1 to 10)
      (collect i into nums)
      (collect (sqrt i) into nums)
      (finally (return nums)))
\end{program}
Clauses can only accumulate into the same variable if they are
compatible.  |collect|, |adjoining|, |appending|, |nconcing|,
|unioning| and |nunioning| are compatible with each other; |sum|,
|multiply| and |counting| are compatible; |always| and |never| are compatible;
|finding| \dots |such-that| is compatible with either |thereis| or |always|
and |never| when their defaults match; and |maximize| and |minimize| clauses
are compatible only with other |maximize| and |minimize| clauses,
respectively.

%note that the same variable ~cannot~ be both an accumulation
%variable and an ordinary variable; there can be only one variable with
%a given name within an \iter\ form. 
\end{sloppypar}

\subsection{Named Blocks}

Like Common Lisp |block|s, \iter\ forms can be given names.  The
name should be a single symbol, and it must be the first form in the
\iter.  The generated code behaves exactly like a named block; in
particular, |(return-from ~name~)| can be used to exit it:
\begin{program}
(iter fred
      (for i from 1 to 10)
      (iter barney
            (for j from i to 10)
            (if (> (* i j) 17)
                (return-from fred j))))
\end{program}
An \iter\ form that is not given a name is implicitly named \nil.

Sometimes one would like to write an expression in an inner \iter\ form,
but have it processed by an outer \iter\ form.  This is possible with
the |in| clause.  

\begin{clauses}

\clausex{in}{~name~ |\&rest| ~forms~}
Evaluates ~forms~ as if they were part of the \iter\ form named
~name~.  In other words, \iter\ clauses are processed by the \iter\
form named ~name,~ and not by any \iter\ forms that occur inside ~name.~

\cpar As an example, consider the problem of collecting a list of the
elements in a two-dimensional array.  The naive solution,
\begin{program}
(iter (for i below (array-dimension ar 0))
      (iter (for j below (array-dimension ar 1))
            (collect (aref ar i j))))
\end{program}
\noindent is wrong because the list created by the inner \iter\ is simply
ignored by the outer one.  But using |in| we can write:
\begin{program}
(iter outer (for i below (array-dimension ar 0))
      (iter (for j below (array-dimension ar 1))
            (in outer (collect (aref ar i j)))))
\end{program}
\noindent which has the desired result.

\end{clauses}

\subsection{Destructuring}
\label{destructuring}

In many places within \iter\ clauses where a variable is expected, a
list can be written instead.  In these cases, the value to be assigned
is ~destructured~ according to the pattern
described by the list.  As a simple example, the clause
\begin{program}
(for (key . item) in alist)
\end{program}
\noindent will result in |key| being set to the |car| of
each element in |alist|, and |item| being set to the |cdr|.  The
pattern list may be nested to arbitrary depth, and (as the example
shows) need not be terminated with \nil; the only requirement is that
each leaf be a bindable symbol (or \nil, in which case no binding is
generated for that piece of the structure).

Sometimes, you might like to do the equivalent of a
|multiple-value-setq| in a clause.  This 
``multiple-value destructuring'' can be expressed by writing 
\linebreak |(values $pat_1$ $pat_2 \ldots$)| for a destructuring
pattern, as in 
\begin{program}
(for (values (a . b) c d) = (three-valued-function ...))
\end{program}
\begin{sloppypar}
Note that the $pat_i$ can themselves be destructuring patterns (though
not multiple-value destructuring patterns).  You can't do multiple-value
destructuring in a |with| clause; instead wrap the whole \iter\
form in a |multiple-value-bind|.
\end{sloppypar}

\begin{note}{Rationale}
There are subtle interactions between variable declarations and
evaluation order that make the correct implementation of
multiple-value destructuring in a |with| somewhat tricky.
\end{note}

The destructuring feature of \iter\ is available as a separate
mechanism, using the |dsetq| macro:

\begin{clauses}
\defmacro{dsetq}{template expr}
Performs destructuring of ~expr~ using ~template~.   May be used
outside of an \iter\ form. Yields the primary value of ~expr~.

\end{clauses}

\subsection{On-line Help}

\begin{sloppypar}
There is a limited facility for on-line help, in the form of the 
|display-iterate-clauses| function.
\end{sloppypar}

\begin{clauses}

\defun{display-iterate-clauses}{\opt\ clause-spec}
Displays a list of \iter\ clauses.  If ~clause-spec~ is not
provided, all clauses are shown; if it is a symbol, all clauses
beginning with that symbol are shown; and if it is a list of symbols,
all clauses for which ~clause-spec~ is a prefix are shown.

\end{clauses}

\subsection{Parallel Binding and Stepping}
\label{bindings}

The parallel binding and stepping of variables is a feature that
\iter\ does ~not~ have.  This section attempts to provide a rationale.

We say that two variables are bound ~in parallel~ if neither
binding shadows the other.  This is the usual semantics of |let|
(as opposed to |let*|).  Similarly, we can say that iteration
variables are stepped in parallel if neither variable is updated
before the other, conceptually speaking; in other words, if the code
to update each variable can reference the old values of both variables.

|loop| allows parallel binding of variables and parallel stepping of
driver variables.  My view is that if you are depending on the 
serial/parallel distinction, you are doing something obscure.  If you
need to bind 
variables in parallel using |with|, then you must be using a
variable name that shadows a name in the existing lexical environment.
Don't do that.  The most common use for parallel stepping is to track
the values of variables on the previous iteration, but in fact this
does not require parallel stepping at all; the following will work:
\begin{program}
(iter (for current in list)
      (for prev previous current)
      ...)
\end{program}


%%%%%%%%%%%%%%%%%%%%%%%%%%%%%%%%%%%%%%%%%%%%%%%%%%%%%%%%%%%%%%%%
\section{Types and Declarations}
\label{types}

\subsection{Discussion}

Sometimes efficiency dictates that the types of variables be declared.
This type information needs to be communicated to \iter\ so it can
bind variables to appropriate values.  Furthermore, \iter\ must often
generate internal variables invisible to the user; there needs to be a
way for these to be declared.

As an example, consider this code, which will return the number of
odd elements in |number-list|:
\begin{program}
(iter (for el in number-list)
      (count (oddp el)))
\end{program}
In processing this form,
\iter\ will create an internal variable, let us call it |list17|, to
hold the successive |cdr|s of |number-list|, and  
will bind the variable to |number-list|.  It will also generate a
default binding for |el|; only inside the body of the loop will |el|
be set to the |car| of |list17|.  Finally, \iter\ will generate a
variable, call it |result|, to hold the result of the count, and will
bind it to zero.

When dealing with type declarations, \iter\ observes one simple rule:
~it will never generate a declaration unless requested to do so.~  The
reason is that such declarations might mask errors in compiled code by
avoiding error-checks; the resulting problems would be doubly hard to
track down because the declarations would be hidden from the
programmer.  Of course, a compiler might omit error-checks even in the
absence of declarations, though this behavior can usually be avoided,
e.g. by saying |(declaim (optimize (safety 3)))|.

So, the above \iter\ form will generate code with no declarations.
But say we wish to declare the types of |el| and the internal
variables |list17| and |result|.  How is this done?

Declaring the type of |el| is easy, since the programmer knows
the variable's name:
\begin{program}
(iter (for el in number-list)
      (declare (fixnum el))
      (counting (oddp el)))
\end{program}
\iter\ can read variable type declarations like this one.  Before
processing any clauses, it scans the entire top-level form for type
declarations 
and records the types, so that variable bindings can be performed
correctly.  In this case, |el| will be bound to zero
instead of \nil.  Also, \iter\ collects all the top-level declarations
and puts them at the begining of the generated code, so it is not
necessary to place all declarations at the beginning of an \iter\
form; instead, they can be written near the variables whose types they
declare.  

Since \iter\ is not part of the compiler, it will not know
about declarations that occur outside an \iter\ form; these
declarations must be repeated inside the form.

Here is another way we could have declared the type of |el|:
\begin{program}
(iter (for (the fixnum el) in number-list)
      (counting (oddp el)))
\end{program}
\iter\ extends the Common Lisp |the|\lispindex{the}
form to apply to variables as well as value-producing forms; anywhere
a variable is allowed---in a |with| clause, as the iteration
variable in a driver clause, as the |into| argument of an
accumulation clause, even inside a destructuring template---you can
write |(the ~type~ ~symbol~)| instead.

There is one crucial difference between using a |the| form and
actually declaring the variable: explicit declarations are always
placed in the generated code, but type information from a |the|
form is not turned into an actual declaration unless you tell \iter\
to do so using |iterate:declare-variables|.  See below.

\begin{sloppypar}
Declaring the types of internal variables is harder than declaring the
types of explicitly mentioned variables, since their names
are unknown.  You do it by declaring |iterate:declare-variables|
somewhere inside the top level of the \iter\ form.  (This will also
generate declarations for variables declared using |the|.)
\iter\ does not provide much selectivity here: it's all or none. 
And unfortunately, since \iter\ is not privy to compiler information
but instead reads declarations itself, it will not hear if you
|(declaim (iterate:declare-variables))|.  Instead, set the variable 
|iterate::*always-declare-variables*| to |t| at
compile-time, using |eval-when|.
\end{sloppypar}

To determine the appropriate types for internal variables, \iter\ uses
three sources of information:

\begin{itemize}

\item Often, the particular clause dictates a certain type for a
variable; \iter\ will use this information when available.  In the
current example, the variable |list17| will be given the type
|list|, since that is the only type that makes sense; and the
variable |result| will be given the type |fixnum|, on the
assumption that you will not be counting high enough to need bignums.
You can override this assumption only by using and explicitly declaring a
variable: 
\begin{verbatim}
(iter (declare (iterate:declare-variables))
      (for el in number-list)
      (count (oddp el) into my-result)
      (declare (integer my-result))
      (finally (return my-result)))
\end{verbatim}

\begin{sloppypar}
Other examples of the type assumptions that \iter\ makes are: type
|list| for |into| variables of collection clauses; type |list| for
expressions that are to be destructured; type |vector| for the
variable holding the vector in a |for\dots in-vector| clause, and
similarly for |string| and the |for\dots in-string| clause;
and the implementation-dependent type
|(type-of array-dimension-limit)| for the index and limit
variables generated by sequence iteration drivers like |for\dots
in-vector| and |for\dots in-string| (but not |for\dots in-sequence|,
because it may be used to iterate over a list). 
\end{sloppypar}

\item Sometimes, \iter\ will examine expressions and try to determine
their types in a simple-minded way.  If the expression is
self-evaluating (like a number, for instance), \iter\ knows that the
expression's type is the same as the type of the value it denotes, so
it can use that type.  If the expression is of the form |(the ~type~
~expr~)|, \iter\ is smart enough to extract ~type~ and use it.
However, the current version of \iter\ does not  
examine declarations of function result types or do any type
inference.  It will not determine, for 
example, that the type of |(+ 3 4)| is |fixnum|, or even
|number|.

\item In some cases, the type of an internal variable should match the
type of some other variable.  For instance, \iter\ generates an
internal variable for |(f x)| in the
clause |(for i from 1 to (f x))|, and in the absence of other
information will give it the same type as |i|.  If, however, the
expression had been written |(the fixnum (f x))|, then \iter\
would have given the internal variable the type |fixnum|
regardless of |i|'s type.  The type incompatibility errors that
could arise in this situation are not checked for.

\end{itemize}

Note that if you do declare |iterate:declare-variables|, then
\iter\ may declare user variables as well as internal ones if they do
not already have declarations, though only for variables that it
binds.  For instance, in this code:

\begin{program}
(iter (declare (iterate:declare-variables))
      (for i from 1 to 10)
      (collect i into var))
\end{program}
the variable |var| will be declared to be of type |list|.


\subsection{Summary}

\iter\ understands standard Common Lisp variable type declarations
that occur within an \iter\ form and
will pass them through to the generated code.  If the declaration
|(iterate:declare-variables)|\lispindex{declare-variables}
appears at the top level of an 
\iter\ form, or if 
|iterate::*always-declare-variables*|\lispindex{*always-declare-variables*} 
is \nonnil, then \iter\ will use the type information gleaned from user
declarations, self-evaluating expressions and |the| expressions,
combined with reasonable assumptions, to determine variable
types and declare them.


%%%%%%%%%%%%%%%%%%%%%%%%%%%%%%%%%%%%%%%%%%%%%%%%%%%%%%%%%%%%%%%%
\section{Problems with Code Movement}
\label{code-movement}

Some \iter\ clauses, or parts of clauses, result in code being
moved from the location of the clause to other parts of the loop.
Drivers behave this way, as do code-placement clauses like |initially|
and |finally|.  When using these clauses, there is a danger of writing
an expression that makes sense in its apparent location but will be
invalid or have a different meaning in another location.  For example:
\begin{program}
(iter (for i from 1 to 10)
      (let ((x 3))
        (initially (setq x 4))))
\end{program}
While it may appear that the |x| of |(initially (setq x 4))| is the
same as the |x| of |(let ((x 3)) \dots|, in fact they are not:
|initially| moves its code outside the loop body, so |x| would refer
to a global variable.  Here is another example of the same problem:
\begin{program}
(iter (for i from 1 to 10)
      (let ((x 3))
        (collect i into x)))
\end{program}
If this code were executed, |collect| would create a binding for its
|x| at the top level of the \iter\ form that the |let| will shadow.

Happily, \iter\ is smart enough to catch these errors; it
walks all problematical code to ensure that free variables are not
bound inside the loop body, and checks all variables it binds for the
same problem.

However, some errors cannot be caught:

\begin{program}
(iter (with x = 3)
      (for el in list)
      (setq x 1)
      (reducing el by \#'+ initial-value x))
\end{program}
|reducing| moves its |initial-value| argument to the initialization
part of the loop in order to produce more efficient code.  Since
\iter\ does not perform data-flow analysis, it cannot determine that
|x| is changed inside the loop; all it can establish is that |x| is
not bound internally.  Hence this code will not signal an
error and will use $3$ as the initial value of the reduction.

The following list summarizes all cases that are subject to these code
motion and variable-shadowing problems.
\begin{itemize}
\item Any variable for which \iter\ creates a binding, including those
used in |with| and the |into| keyword of many clauses.

\begin{sloppypar}
\item The special clauses which place code: |initially|, |after-each|, |else|,
|finally| and |finally-protected|. 
\end{sloppypar}

\item The variables of a |next| or |do-next| form.

\item The |initially| arguments of |for\dots initially\dots then| and
|for\dots previous|. 

\item The |then| argument of |for\dots initially\dots then|.

\item The |initial-value| arguments of |reducing| and |accumulate|.

\item The |on-failure| argument of |finding\dots such-that|.
\end{itemize}

%%%%%%%%%%%%%%%%%%%%%%%%%%%%%%%%%%%%%%%%%%%%%%%%%%%%%%%%%%%%%%%%
\section{Differences Between {\tt Iterate} and {\tt Loop}}
|loop| contains a great deal of complexity which \iter\ tries to
avoid.  Hence many esoteric features of |loop| don't exist in \iter.
Other features have been carried over, but in a cleaned-up form.  
And of course, many new features have been added; they are not
mentioned in this list.

\begin{itemize}

\item \iter's syntax is more Lisp-like than |loop|'s, having a higher
density of parens.

\item The current implementation of \iter, unlike the current version
of  |loop| (as documented in {\em Common Lisp, 2nd Ed.\/}), is
extensible (see section \ref{extend}).  

\item |loop| puts the updates of all driver variables at the top of
the loop; \iter\ leaves them where the driver clauses appear.

\item While for the most part \iter\ clauses that resemble |loop| clauses
behave similarly, there are some differences.  For instance, there is
no |for\dots =\dots then| in \iter; instead use 
|for\dots initially\dots then|.  

\item |loop| binds the variable |it| at certain times
to allow pseudo-English expressions like |when ~expr~ return it|. 
In \iter, you must bind ~expr~ to a variable yourself.  Note that
|when ~expr~ return it| is like |thereis ~expr~| except that the latter is an
accumulation clause and therefore competes with other accumulations
(remember section \ref{multiple} above).

% repeat different behaviour of |always| clause here?

\item |loop| has a special |return| clause, illustrated in the
previous item.  \iter\ doesn't need one,  since an ordinary Lisp
|return| has the same effect.

\item |loop| allows for parallel binding and stepping of iteration
variables.  \iter\ does not.  (See section \ref{bindings}.)

\item |loop| and \iter\ handle variable type declarations very
differently.  |loop| provides a special syntax for declaring variable
types, and does not examine declarations.  Moreover, the standard
implementation of |loop| will
generate declarations when none are requested.
\iter\ parses standard Common Lisp type declarations, and will never
declare a variable itself unless declarations are 
specifically requested.

\end{itemize}

%%%%%%%%%%%%%%%%%%%%%%%%%%%%%%%%%%%%%%%%%%%%%%%%%%%%%%%%%%%%%%%%
\section{Rolling Your Own}
\label{extend}

\subsection{Introduction}

\iter\ is extensible---you can write new clauses that embody new
iteration patterns.  You might want to write a new driver clause for a
data structure of your own, or you might want to write a clause that
collects or manipulates elements in a way not provided by \iter. 

This section describes how to write clauses for \iter.  Writing a
clause is like writing a macro.  In fact, writing a clause ~is~
writing a macro: since \iter\ code-walks its body and macroexpands,
you can add new abstractions to \iter\ with good old |defmacro|.

Actually, there are two extensions you can make to \iter\ that are
even easier than writing a macro.  They are adding a synonym for an
existing clause and defining a driver clause for an indexable
sequence.  These can be done with |defsynonym| and
|defclause-sequence|, respectively.  See section \ref{aids}, below.

The rest of this section explains how to write macros that expand into
\iter\ clauses.
Here's how you could add a simplified version of \iter's |multiply|
clause, if \iter\ didn't already have one:
\begin{program}
(defmacro multiply (expr)
  `(reducing ,expr by \#'* initial-value 1))
\end{program}

If you found yourself summing the square of an expression often, you
might want to write a macro for that.  A first cut might be
\begin{program}
(defmacro sum-of-squares (expr)
  `(sum (* ,expr ,expr)))
\end{program}
but if you are an experienced macro writer, you will realize that this
code will evaluate ~expr~ twice, which is probably a bad idea.  A
better version would use a temporary:
\begin{program}
(defmacro sum-of-squares (expr)
  (let ((temp (gensym)))
    `(let ((,temp ,expr))
       (sum (* ,temp ,temp)))))
\end{program}
Although this may seem complex, it is just the sort of thing you'd
have to go through to write any macro, which illustrates the point of
this section: if you can write macros, you can extend \iter.

Our macros don't use \iter's keyword-argument
syntax.  We could just use keywords with |defmacro|, but we would
still not be using \iter's clause indexing mechanism.  Unlike Lisp,
which uses just the first symbol of a form to determine what function
to call, \iter\ individuates clauses by the list of required keywords.
For instance, |for\dots in| and |for\dots in-vector| are different clauses
implemented by distinct Lisp functions.  

To buy into this indexing scheme, as well as the keyword-argument
syntax, use |defmacro-clause|:

\begin{clauses}

\defmacro{defmacro-clause}{arglist |\&body| body}
Defines a new \iter\ clause.  ~arglist~ is a list of symbols which are
alternating keywords and arguments.  \opt\ may be used, and the list
may be terminated by |\&sequence|.  ~body~ is an ordinary macro body,
as with |defmacro|.  If the first form of ~body~ is a string, it is
considered a documentation string and will be shown by
|display-iterate-clauses|. |defmacro-clause| will signal an error if 
defining the clause would result in an ambiguity.  E.g. you cannot
define the clause |for\dots from| because there would be no way to
distinguish it from a use of the |for| clause with optional keyword |from|.

\end{clauses}

\medskip

Here is |multiply| using |defmacro-clause|.  The keywords are capitalized
for readability.
\begin{program}
(defmacro-clause (MULTIPLY expr \opt\ INTO var)
  `(reducing ,expr by \#'* into ,var initial-value 1))
\end{program}
You don't have to worry about the case when |var| is not supplied; for
any clause with an |into| keyword, saying |into nil| is equivalent to
omitting the |into| entirely.

As another, more extended example, consider the fairly common
iteration pattern that 
involves finding the sequence element that maximizes (or minimizes) some
function.  \iter\ provides this as |finding\dots maximizing|, but it's
instructive to see how to write it.
Here, in pseudocode, is how you might write such a loop for
maximizing a function F:

\begin{quote}
\begin{tabbing}
set variable MAX-VAL to NIL; \\
set variable WINNER to NIL; \\
for\=\ each element EL in the sequence \\
\>  if\=\ MAX-VAL\=\ is NIL or F(EL) $>$ MAX-VAL then \\
\>\>    set MAX-VAL to F(EL); \\
\>\>    set WINNER to EL; \\
\>  end if; \\
end for; \\
return WINNER.
\end{tabbing}
\end{quote}

Here is the macro:
\begin{program}
(defmacro-clause (FINDING expr MAXIMIZING func \opt\ INTO var)
  (let ((max-val (gensym))
        (temp1 (gensym))
        (temp2 (gensym))
        (winner (or var iterate::*result-var*)))
    `(progn 
       (with ,max-val = nil)
       (with ,winner = nil)
       (cond 
        ((null ,max-val)
         (setq ,winner ,expr)
         (setq ,max-val (funcall ,func ,winner))
        (t
         (let* ((,temp1 ,expr)
                (,temp2 (funcall ,func ,temp1)))
           (when (> ,temp2 ,max-val)
             (setq ,max-val ,temp2)
             (setq ,winner ,temp1))))))
       (finally (leave ,winner)))))
\end{program}
Note that if no |into| variable is supplied, we use
|iterate::*result-var*|, which contains the internal variable into
which all clauses place their results.  If this variable is bound by
some clause, then \iter\ will return its value automatically;
otherwise, \nil\ will be returned.  

\subsection{Writing Drivers}

In principle, drivers can be implemented just as easily as other
\iter\ clauses.  In practice, they are a little harder to get right. 
As an example, consider writing a driver that
iterates over all the 
elements of a vector, ignoring its fill-pointer.  |for\dots in-vector|
won't work for this, because it observes the fill-pointer.  It's necessary to
use |array-dimension| instead of |length| to obtain the size of the
vector.  Here is one approach:

\begin{program}
(defmacro-clause (FOR var IN-WHOLE-VECTOR v)
  "All the elements of a vector (disregards fill-pointer)"
  (let ((vect (gensym))
        (index (gensym)))
    `(progn
       (with ,vect = ,v)
       (for ,index from 0 below (array-dimension ,vect 0))
       (for ,var = (aref ,vect ,index)))))
\end{program}
Note that we immediately put |v| in a variable, in case it is an
expression.  Again, this is just good Lisp macrology.  It also has a
subtle effect on the semantics of the driver: |v| is evaluated only
once, at the beginning of the loop, so changes to |v| in the loop have
no effect on the driver.  Similarly, the bounds for numerical iteration
e.g. the above |array-dimension| are also evaluated once only.  This is how
all of \iter's drivers work.


There is an important point concerning the |progn| in this code.  We
need the |progn|, of course, because we are returning several forms,
one of which is a driver.  But \iter\ drivers must occur at top-level.
Is this code in error?  No, because ~top-level~ is defined in \iter\
to include forms inside a |progn|.  This is just the definition of
top-level that Common Lisp uses, and for the same reason: to allow
macros to return multiple forms at top-level.

While our |for\dots in-whole-vector| clause will work, it is not
ideal.  In particular, it does not support generating.  Do do so, we
need to use |for\dots next| or |for\dots do-next|.  The job is
simplified by the |defmacro-driver| macro.

\begin{clauses}

\defmacro{defmacro-driver}{arglist |\&body| body}
Defines a driver clause in
both the |for| and |generate| forms, and provides a parameter
|generate| which ~body~ can examine to determine how it was invoked.
~arglist~ is as in |defmacro-clause|, and should begin with the symbol
|for|. 

\end{clauses}

With |defmacro-driver|, our driver looks like this:
\begin{program}
(defmacro-driver (FOR var IN-WHOLE-VECTOR v)
  "All the elements of a vector (disregards fill-pointer)"
   (let ((vect (gensym))
         (end (gensym))
         (index (gensym))
         (kwd (if generate 'generate 'for)))
     `(progn
        (with ,vect = ,v)
        (with ,end = (array-dimension ,vect 0))
        (with ,index = -1)
        (,kwd ,var next (progn (incf ,index)
                               (if (>= ,index ,end) (terminate))
                               (aref ,vect ,index))))))
\end{program}


We are still missing one thing: the |\&sequence| keywords.  
We can get them easily enough, by writing 
\begin{program}
(defmacro-driver (FOR var IN-WHOLE-VECTOR v \&sequence) 
  ...)
\end{program}
We can now refer to parameters |from|, |to|, |by|, etc. which contain
either the values for the corresponding keyword, or \nil\ if the
keyword was not supplied.  Implementing the right code for these
keywords is cumbersome but not difficult; it is left as an exercise.
But before you begin, see |defclause-sequence| below for an easier way.
       
\subsection{Extensibility Aids}
\label{aids}

This section documents assorted features that may be of use in
extending \iter.

\begin{clauses}

\defunexpvar{*result-var*}
Holds the variable that is used to return a value as a result of the
\iter\ form.  You may examine this and use it in a |with| clause, but
you should not change it.

\defmacro{defsynonym}{syn word}
 Makes ~syn~ a synonym for the existing \iter\ keyword ~word.~  Only
the first word in each clause can have synonyms. 

\defmacroxx{defclause-sequence}{element-name index-name}{|\&key|
access-fn size-fn sequence-type}{element-type
element-doc-string index-doc-string}  
Provides
a simple way to define sequence clauses.  Generates two
clauses, one for iterating over the sequence's elements, the other
for iterating over its indices.  The first symbol of both
clauses will have print-name |for|.
~element-name~  and ~index-name~ should be symbols.
~element-name~ is the second keyword of the element iterator (typically of
the form 
|in-~sequence-type~|), and ~index-name~ is the second keyword
of the index-iterator (typically of the form
|index-of-~sequence-type~|).  Either name may be 
\nil, in which case the corresponding clause is not defined.  If both
symbols are supplied, they should be in the same package.  The |for|
that begins the clauses will be in this package.

\cpar ~access-fn~ is the function to be used to
access elements of the sequence in the element iterator.  The function
should take two 
arguments, a sequence and an index, and return the appropriate element.
~size-fn~ should denote a function of one argument, a sequence, that
returns its size.  Both ~access-fn~ and ~size-fn~ are required for the
element iterator, but only ~size-fn~ is needed for the index iterator.

\cpar The ~sequence-type~ and ~element-type~ keywords are used to
suggest types for the variables 
used to hold the sequence and the
sequence elements, respectively.  The usual rules about \iter's
treatment of variable type declarations apply (see section \ref{types}).

\cpar ~element-doc-string~ and ~index-doc-string~ are
the documentation strings, for use with |display-iterate-clauses|. 

\cpar The generated element-iterator performs destructuring on the
element variable.

\cpar As an example, the above |for\dots in-whole-vector| example
could have been written:
\begin{program}
(defclause-sequence IN-WHOLE-VECTOR INDEX-OF-WHOLE-VECTOR
  :access-fn 'aref
  :size-fn \#'(lambda (v) (array-dimension v 0))
  :sequence-type 'vector
  :element-type t
  :element-doc-string 
     "Elements of a vector, disregarding fill-pointer"
  :index-doc-string 
     "Indices of vector, disregarding fill-pointer")
\end{program}

\end{clauses}

\subsection{Subtleties}

There are some subtleties to be aware of when writing \iter\ clauses.
First, the code returned by your macros may be |nconc|'ed into a list,
so you should always returned freshly consed lists, rather than
constants.  Second, \iter\ matches clauses by using |eq| on the first
symbol and |string=| on the subsequent ones, so the package of the
first symbol of a clause is relevant.  All of the clauses in this manual
have their first word in the \iter\ package.  
You can use the package system in the usual way to shadow
\iter\ clauses without replacing them.

%%% say more here, about the badness that only the first word of a
%%% clause is packagey.

\section{Non-portable Extensions to Iterate (Contribs)}
\label{contribs}

Currently, there is only one non-portable extension to iterate in the
distribution: iterate-pg. If you have made an extension that depends
on non-portable features, feel free to send them to |asf@boinkor.net|
for inclusion in the iterate distribution.

\subsection{An SQL query driver for iterate}

The pg package by Eric Marsden (see |http://cliki.net/pg|) provides an
interface to the PostgreSQL database. Using the \iterpg\ extension, it
is possible to handle the results of SQL queries using \iter.

This usage example should give you an idea of how to use it:

\begin{program}
(pg:with-pg-connection (c "somedb" "someuser")
  (iter (for (impl version date) in-relation "select * from version"
                                 on-connection *dbconn*)
        (collect version)))
\end{program}

To use the extension via |ASDF|, simply make your system depend on the
|iterate-pg| system instead of the |iterate| system. To load it
manually, use:

\begin{program}
  (asdf:oos 'asdf:load-op :iterate-pg)
\end{program}

%%%%%%%%%%%%%%%%%%%%%%%%%%%%%%%%%%%%%%%%%%%%%%%%%%%%%%%%%%%%%%%%
\section{Obtaining {\tt Iterate}}

\begin{sloppypar}
The information in this chapter is obsolete but included for
completeness's sake; Currently, the most up-to-date information on
\iter\ can be found at |http://boinkor.net/iterate.html|.
\end{sloppypar}

\begin{sloppypar}
\iter\ currently runs on Lisp Machines, and on
HP's, Sun3's and Sparcstations under Lucid.  
\iter\ source and binaries are available at the MIT AI Lab in the
subdirectories of |/src/local/lisplib/|.  The source file,
|iterate.lisp|, is also available for anonymous FTP in the directory
|/com/fpt/pub/| on the machine |TRIX.AI.MIT.EDU| (Internet number
128.52.32.6).  If you are unable to obtain |iterate| in one of these
ways, send mail to |jba@ai.mit.edu| and I will send you the source
file. 
\end{sloppypar}

\begin{sloppypar}
\iter\ resides in the |iterate| package (nickname |iter|).  Just say
\linebreak |(use-package :iterate)| to make all the necessary symbols
available. 
If a symbol is not exported, it appears in this manual with an
``|iterate::|'' prefix.
\end{sloppypar}

Send bug reports to |bug-iterate@ai.mit.edu|.  The |info-iterate|
mailing list will have notices of changes and problems; to have
yourself added, send mail to |info-iterate-request@ai.mit.edu|.


\medskip

\begin{flushleft}
 \bf Acknowledgements
\end{flushleft}
\smallskip

Richard Waters provided invaluable criticism which spurred me to improve
\iter\ greatly.  As early users, David Clemens, Oren Etzioni and Jeff
Siskind helped ferret out many bugs.

%\begin{theindex}
% The index files must be generated with the genindex program
% or from makeindex iterate-manual.idx which creates iterate-manual.ind.

%\baselineskip\normalbaselineskip
%\advance\baselineskip by -2pt

\input{iterate-manual.ind}

%\underline{Clauses}
%\input{iterate-manual.clindex}
%
%\indexspace
%\underline{Lisp}
%\input{iterate-manual.lispindex}

%\end{theindex}


\end{document}


% arch-tag: "c0f871a6-313c-11d8-abb9-000c76244c24"

%
%\indexspace
%\underline{Lisp}
%% Fix ``.'' spacing in lists
\documentstyle[12pt,aimemo]{article}

\setlength{\textwidth}{6.0in}
\setlength{\oddsidemargin}{0.25in}
\setlength{\evensidemargin}{0.25in}
\setlength{\topmargin}{0pt}
\setlength{\headsep}{24pt}
\setlength{\textheight}{8.5in}

\let\origpar=\par

% somehow, LaTex fucks with things so that the plain Tex \obeyspaces
% doesn't work.  This is the same, except we use an mbox containing
% a space instead of just a space.

\newcommand{\spc}{\mbox{ }}
\newcommand{\tAb}{\mbox{        }}
\def\obspaces{\catcode`\ =\active\catcode`\^^I=\active}
{\obspaces \global\let =\spc\global\let^^I=\tAb}

\newenvironment{program}{\medskip\samepage\tt\obeylines\obspaces}{\medskip}

% use | for \tt
\catcode`\|=13
\newif\ifvbar
\def|{\ifvbar
         \endgroup\vbarfalse
      \else 
         \begingroup\tt\vbartrue 
      \fi}



\newcommand{\lisp}{\tt}
\newcommand{\iter}{|iterate|}
\newcommand{\iterpg}{|iterate-pg|}
\newcommand{\setf}{|setf|}
\newcommand{\nil}{|nil|}
\newcommand{\nonnil}{non-\nil}
\newcommand{\opt}{{\lisp \&optional}}
\newcommand{\yields}{$\Rightarrow$}



\newbox\Dots
\setbox\Dots=\hbox{\small ...}
\renewcommand{\dots}{\copy\Dots}

% Here's a way to define an environment which will treat all
% paragraphs inside it in the same manner.  It's, as near as I can
% figure, the way that Latex does things.  Using \everypar won't work;
% apparently Latex resets that a lot.  What you have to do is: at the
% top of the file do \let\origpar=\par.  In the
% begin part of the environment, do the parshape (or whatever), then say
% \def\par{{\origpar}}.  The extra braces are necessary.   That's all
% you have to do; at the end of the environment things will return to
% normal automatically.  Why does this all work?  I have no idea.
% Another important thing: there should be a blank line (i.e. a \par)
% between the end of the last paragraph and the \end{environment}.

\newlength{\clindent}
\setlength{\clindent}{0.5in}
\newlength{\clparindent}
\setlength{\clparindent}{\parindent}

\newenvironment{clauses}{\advance\linewidth -\parindent
                        \hangindent\clindent\relax
                         \hangafter=0
                          \parindent=0pt
                          \def\par{{\origpar}}}{}

\newcommand{\cpar}{\hspace \clparindent}

\newcommand{\startitem}{\bigskip\pagebreak[3]}
\newcommand{\finishitem}{\smallskip}

\newcommand{\clause}[1]{\startitem\Clause{#1}\finishitem}
\newcommand{\Clause}[1]{\hbox{#1}}
\newcommand{\clindex}[1]{\index{{\lisp #1}}}
\newcommand{\clindexx}[2]{\index{{\lisp #1\dots #2}}}
\newcommand{\clindexxx}[3]{\index{{\lisp #1\dots #2\dots #3}}}
\newcommand{\clausex}[2]{\startitem\Clausex{#1}{#2}\finishitem}
\newcommand{\Clausex}[2]{\Clause{|#1| #2}\clindex{#1}}

\newcommand{\defvar}[1]{\startitem\deff{#1}{}{Variable}\finishitem}
\newcommand{\defunexpvar}[1]{\startitem\deffnoindex{iterate::#1}{}{Variable}
                                        \lispindex{#1}\finishitem}
\newcommand{\defun}[2]{\startitem\deff{#1}{#2}{Function}\finishitem}
\newcommand{\defmacro}[2]{\startitem\deff{#1}{#2}{Macro}\finishitem}
\newcommand{\defunx}[3]{\startitem\deffx{#1}{#2}{#3}{Function}\finishitem}
\newcommand{\defmacrox}[3]{\startitem\deffx{#1}{#2}{#3}{Macro}\finishitem}
\newcommand{\defunxx}[4]{\startitem\deffxx{#1}{#2}{#3}{#4}{Function}\finishitem}
\newcommand{\defmacroxx}[4]{\startitem\deffxx{#1}{#2}{#3}{#4}{Macro}\finishitem}

\newcommand{\deff}[3]{\deffnoindex{#1}{#2}{#3}\lispindex{#1}}
\newcommand{\deffnoindex}[3]{\hbox to \hsize{{\tt #1} {\it #2}\hfill [{\it #3}]}}
\newcommand{\deffx}[4]{\deff{#1}{#2}{#4}\moreargs{#1}{#3}}
\newcommand{\deffxx}[5]{\deffx{#1}{#2}{#3}{#5}\moreargs{#1}{#4}}

\newcommand{\moreargs}[2]{\hbox to \hsize{\phantom{\lisp #1} {\it #2}\hfill}}

\newcommand{\lispindex}[1]{\index{{\lisp #1}}}

 \newenvironment{note}[1]{\pagebreak[2]\bigskip
                          \hrule\smallskip\small
                         {\setlength{\parindent}{0pt}
                           \par{\bf #1:}}}{\smallskip\hrule\bigskip}



\makeindex

%%%%%%%%%%%%%%%%%%%%%%%%%%%%%%%%%%%%%%%%%%%%%%%%%%%%%%%%%%%%%%%%

%\thispagestyle{empty}

%\vspace*{2in}

%\title{The |Iterate| Manual}
%\author{Jonathan Amsterdam}

%\maketitle

%\thispagestyle{empty}

%\begin{center}
%\Large *** DRAFT ***
%\end{center}
%This manual is in beta-test. It will become an AI Memo in a couple of
%months.  I would appreciate any comments or
%corrections.  You can reach me at email address |jba@ai.ai.mit.edu|.

\begin{document}

\def\writtenby{Jonathan Amsterdam}
\def\contractno{This report describes research done at the Artificial
Intelligence Laboratory of the Massachusetts Institute of Technology.
Support for the laboratory's artificial intelligence research is
provided in part by the Advanced Research Projects Agency of the
Department of Defense under Office
of Naval Research contract N00014-85-K-0124.}

\AIMEMO{1236}{\Large The Iterate Manual \\[1in]}
{This is the manual for version 1.4 of \iter, a powerful iteration
macro for Common  
Lisp. \iter\ is similar to {\tt loop} but provides numerous additional
features, is well integrated with Lisp, and is extensible. 
\vfill }


% use ~ for \em
\catcode`\~=13
\newif\ifem
\def~{\ifem
         \/\endgroup\emfalse
      \else 
         \begingroup\em\emtrue 
      \fi}



%\pagebreak
%\setcounter{page}{1}

%\thispagestyle{empty}


\tableofcontents

\pagebreak
\pagestyle{headings}

\section{Introduction}

\begin{sloppypar}
This manual describes
\iter, a powerful iteration facility for Common Lisp.
\iter\ provides abstractions for many common iteration
patterns and allows for the definition of additional patterns.  \iter\ is
a macro that expands into ordinary Lisp at
compile-time, so it is more efficient than higher-order
functions like |map| and |reduce.|  
While it is similar to |loop|, \iter\ offers a more Lisp-like syntax
and enhanced extensibility.
(For a more complete
comparison of \iter\ with other iteration constructs, see MIT AI Lab Working
Paper No. 324, ~Don't Loop, Iterate.~)
\end{sloppypar}

An \iter\ form consists of the symbol |iter|\footnote{You can also use
|iterate|, but |iter| is preferred because it avoids potential conflicts with
possible future additions to Common Lisp, and because it saves
horizontal space when writing code.}
followed by one or more
forms, some of which may be \iter\ 
~clauses.~  Here is a simple example of \iter\ which collects the
numbers from 1 to 10 into a list, and returns the list.  The return
value is shown following the arrow.
\begin{program}
(iter (for i from 1 to 10)
      (collect i))          \yields (1 2 3 4 5 6 7 8 9 10)
\end{program}
This form contains two clauses: a |for| clause that steps the
variable |i| over the integers from 1 to 10, and a |collect|
clause that accumulates its argument into a list.  With a few
exceptions, all \iter\ clauses have the same format: alternating
symbols (called ~keywords~) and expressions (called
~arguments~).  The syntax and terminology are those of Common
Lisp's keyword lambda lists.  One difference is that \iter's keywords
do not have to begin with a colon---though they may, except for the
first symbol of a clause.  So you can also write  |(for i :from
1 :to 10)| if you prefer.

Any Lisp form can appear in the body of an \iter, where it will have
its usual meaning.  \iter\ walks the entire body, expanding macros,
and recognizing clauses at any level.  This example collects all the
odd numbers in a list:
\begin{program}
(iter (for el in list)
      (if (and (numberp el) (oddp el))
          (collect el)))
\end{program}

There are clauses for iterating over numbers, lists, arrays and other
objects, and for
collecting, summing, counting, maximizing and other useful operations.
\iter\ also supports the creation of new variable bindings, stepping
over multiple sequences at once, destructuring, and compiler
declarations of variable types.  The following example illustrates
some of these features: 
\begin{program}
(iter (for (key . item) in alist)
         (for i from 0)
         (declare (fixnum i))
         (collect (cons i key)))
\end{program}

This loop takes the keys of an alist and returns a new alist
associating the keys with their positions in the original list. The
compiler declaration for |i| will appear in the generated code
in the appropriate place.

%%%%%%%%%%%%%%%%%%%%%%%%%%%%%%%%%%%%%%%%%%%%%%%%%%%%%%%%%%%%%%%%
\section{Clauses}

Most of \iter's clauses will be familiar to |loop| programmers.  
(|loop| is an iteration macro that has been incorporated into Common
Lisp.  Seen Guy Steele's {\em Common Lisp, 2nd Edition}.)
In
nearly all cases they behave the same as their |loop| counterparts, so
a |loop| user can switch to \iter\ with little pain (and much gain).

All clauses with the standard keyword-argument syntax consist of two
parts: a ~required~ part, 
containing keywords that must be present and in the right order; and
an ~optional~ part, containing keywords that may be omitted and,
if present, may occur in any order.  In the descriptions below, the
parts are separated by the Lisp lambda-list keyword \opt.


\subsection{Drivers}

An iteration-driving clause
conceptually causes the iteration to go forward.  Driver clauses in
\iter\ allow iteration over numbers, lists, vectors, hashtables, packages,
files and streams.  Iteration-driving clauses must
appear at the top level of an \iter\ form; they cannot be nested
inside another clause.  The driver variable is updated at the point
where the driver clause occurs.  Before the clause is executed for the
first time, the value of the variable is undefined.

%Also, regardless of where the driver clause
%appears in the body, the driver variable is stepped at the top of the
%loop; hence it is stylistically preferable, though
%not required, to place driver clauses at the beginning of the \iter.

Multiple drivers may appear in a single \iter\ form, in which case all
of the driver variables are updated each time through the loop, in the
order in which the clauses appear.  The first driver to terminate will
terminate the entire loop.

In all cases, the value of the driver variable on exit from the loop,
including within the epilogue code (see the |finally| clause), is
undefined.

All the parameters of a driver clause are evaluated once, before the
loop begins.  Hence it is not possible to change the bounds or other
properties of an iteration by side-effect from within the loop.

With one exception, driver clauses begin with the word |for| (or
the synonym |as|) and mention an iteration variable, which is
given a binding within the \iter\ form.
The exception is |repeat|, which just executes a loop a
specified number of times:

\begin{clauses}

\clausex{repeat}{~n~}
Repeats the loop ~n~ times.  For example:
\begin{program}
(iter (repeat 100)
      (print "I will not talk in class."))
\end{program}
\cpar If $n \leq 0$, the
loop will never be executed.  If ~n~ is not an integer, the
actual number of executions will be $\lceil n \rceil$.

\end{clauses}

\pagebreak[3]

\subsubsection{Numerical Iteration}

\begin{clauses}
\clausex{for}{~var~ |\&sequence|} 
The general form for iterating over a sequence of numbers requires a
variable and, optionally, one or more keywords that provide the bounds
and step size of the iteration. The |\&sequence|
lambda-list keyword is a shorthand for these sequence keywords.  They are:
|from|, |upfrom|, |downfrom|, |to|, |downto|, |above|, |below| and |by|.
|from| provides the starting value for ~var~
and defaults to zero. |to| provides a final value and implies that the
successive values of ~var~ will be increasing; |downto| implies that
they will be decreasing.  The loop terminates when ~var~ passes
the final value (i.e. becomes smaller or larger than it, depending on
the direction of iteration); in other  words, the loop body
will never be 
executed for values of ~var~ past the final value.
|below| and |above| are similar to |to| and
|downto|, except that the loop terminates when ~var~ equals or passes
the final value.

\cpar If no final value is specified, the variable will be stepped
forever.  Using |from| or |upfrom| will result in increasing
values, while |downfrom| will give decreasing values.

\cpar On each iteration,
~var~ is incremented or decremented by the value of the sequence
keyword |by|, which defaults to 1.  It should always be a positive
number, even for downward iterations.

\cpar In the following examples, the sequence of numbers generated is shown
next to the clause.

\begin{program}
(for i upfrom 0) \yields\ 0 1 2 \ldots
(for i from 5)   \yields\ 5 6 7 \ldots    ; either from or upfrom is okay
(for i downfrom 0) \yields\ 0 -1 -2 \ldots
(for i from 1 to 3) \yields\ 1 2 3
(for i from 1 below 3) \yields\ 1 2
(for i from 1 to 3 by 2) \yields\ 1 3
(for i from 1 below 3 by 2) \yields\ 1
(for i from 5 downto 3) \yields\ 5 4 3
\end{program}

\end{clauses}

\subsubsection{Sequence Iteration}

There are a number of clauses for iterating over sequences.  In all of
them, the argument following |for| may be a list instead of a
symbol, in which case destructuring is performed.  See section
\ref{destructuring}. 

\begin{clauses}

\clause{|for| ~var~ |in| ~list~ \opt\ |by| ~step-function~}
\clindexx{for}{in}
~var~ is set to successive elements of list.  
~step-function,~ which 
defaults to |cdr|, is used to obtain the next sublist. 

\clause{|for| ~var~ |on| ~list~ \opt\ |by| ~step-function~}
\clindexx{for}{on}
~var~ is set to successive sublists of list. 
~step-function~ (default |cdr|) is used as in |for\dots in|.

\end{clauses}

\medskip

These two clauses use |atom| to test for the end
of a list.  Hence, given a list whose final |cdr| is not \nil,
they will silently ignore the last |cdr|.  Other choices are
|endp|,  which would signal an error, and |null|, which
would probably result in an error somewhere else.  If you wish to use
an end-test other than |atom|, set the variable
|iterate::*list-end-test*|\lispindex{*list-end-test*} to the name of
the desired function. 

\begin{clauses}
\clause{|for| ~var~ |in-vector| ~vector~ |\&sequence|}
\clindexx{for}{in-vector}
~var~ takes on successive elements from ~vector.~   The vector's
fill-pointer is observed.  Here and in subsequent clauses, the |\&sequence|
keywords include |with-index|, which takes a symbol as argument
and uses it for the index variable instead of an internally generated
symbol.  The other |\&sequence| keywords behave as in numerical
iteration, except that the default iteration bounds are the bounds of
the vector.  E.g. in |(for i in-vector v downto 3)|,
|i| will start off being bound to the last element in |v|,
and will  
be set to preceding elements down to and including the element with
index 3.

\clause{|for| ~var~ |in-sequence| ~seq~ |\&sequence|}
\clindexx{for}{in-sequence}
This uses Common Lisp's generalized sequence functions, |elt|
and |length|, to obtain elements and determine the length of
~seq.~   Hence it will work for any sequence, including lists, and will
observe the fill-pointers of vectors.

\clause{|for| ~var~ |in-string| ~string~ |\&sequence|}
\clindexx{for}{in-string}
~var~ is set to successive characters of ~string.~

\startitem
\Clause{|for| ~var~ |index-of-vector| ~vector~ |\&sequence|}
\clindexx{for}{index-of-vector}
\Clause{|for| ~var~ |index-of-sequence| ~sequence~ |\&sequence|}
\clindexx{for}{index-of-sequence}
\Clause{|for| ~var~ |index-of-string| ~string~ |\&sequence|}
\clindexx{for}{index-of-string}
\finishitem
~var~ is set to successive indices of the sequence.
These clauses avoid the overhead of accessing the sequence elements
for those applications where they do not need to be examined, or are
examined rarely.  They admit all the optional keywords of the other
sequence drivers except the (redundant) |with-index| keyword.

\clause{|for| ~(key value)~ |in-hashtable| ~table~}
\clindexx{for}{in-hashtable}

~key~ and ~value,~ which must appear as shown in a list
and may be destructuring templates, are set to the keys and values of
~table~.  If ~key~ is |nil|, then the hashtable's keys will be ignored;
similarly for ~value~.
The order in which 
elements of ~table~ will be retrieved is unpredictable.

\clause{|for| ~var~ |in-package| ~package~ \opt\ |external-only| ~ext~}
\clindexx{for}{in-package}
Iterates over all the symbols in ~package,~ or over only the
external symbols if ~ext~ is specified and non-|nil|.  ~ext~ is not
evaluated.  The same symbol may appear more than once.

\clause{|for| ~var~ |in-packages| ~packages~ \opt\ |having-access| ~symbol-types~}
\clindexx{for}{in-packages}
Iterates over all the symbols from the list of packages denoted by the
descriptor ~packages~ and having accessibility (or visibility) given by
~symbol-types~.  This defaults to the list |(:external :internal :inherited)|
and is not evaluated.  ~var~ must be a list of up to three variables: in each
iteration, these will be set to a symbol, its access-type and package (as per
|with-package-iterator| in ANSI-CL).  The same symbol may appear more than
once.

\end{clauses}

\begin{clauses}

\clause{|for| ~var~ |in-file| ~name~ \opt\ |using| ~reader~}
\clindexx{for}{in-file}
Opens the file ~name~ (which may be a string or pathname) for input,
and iterates 
over its contents. ~reader~ defaults to |read|, so by default ~var~
will be bound to the successive forms in the file.  The \iter\ body is
wrapped in an unwind-protect to ensure that the file is closed no
matter how the \iter\ is exited.

\clause{|for| ~var~ |in-stream| ~stream~ \opt\ |using| ~reader~}
\clindexx{for}{in-stream}
Like |for\dots in-file|, except that ~stream~ should be an existing
stream object that supports input operations.

\end{clauses}

\subsubsection{Generalized Drivers}

These are primarily useful for writing drivers that can also be used
as generators (see section \ref{generators}, below).
 
\begin{clauses}

\clause{|for| ~var~ |next| ~expr~}
\clindexx{for}{next}

~var~ is set to ~expr~ each time through the loop.  Destructuring is
performed.  When the clause is used as a generator, ~expr~ is the code
that is executed when |(next ~var~)| is 
encountered (see section \ref{generators}, below).
~expr~ should compute the first value for ~var~, as well
as all subsequent values, and is responsible for terminating the loop.
For compatibility with future versions of \iter, this termination
should be done with |terminate|\clindex{terminate}, which can be
considered a synonym for |finish| (see section \ref{control-flow}).

\cpar As an example, the following clauses are equivalent to |(for
i from 1 to 10)|:
\begin{program}
(initially (setq i 0))
(for i next (if (> i 10) (terminate) (incf i)))
\end{program}

\clause{|for| ~var~ |do-next| ~form~}
\clindexx{for}{do-next}
~form~ is evaluated each time through the loop.  Its value is ~not~
set to ~var~; that is ~form's~ job. ~var~ is only present so that
\iter\ knows it is a driver variable.  \linebreak |(for ~var~ next
~expr~)| is equivalent to |(for ~var~ do-next (dsetq ~var~ ~expr~))|.
(See section \ref{destructuring} for an explanation of |dsetq|.) 

\end{clauses}

\subsubsection{Generators}
\label{generators}

In all of the above clauses, the driver variable is updated on each
iteration.
Sometimes it is desirable to have greater control over updating.  
For instance, consider the problem of associating numbers, in
increasing order and with no gaps, with the
\nonnil\ elements of a list.  One obvious first pass at writing this is:

\begin{program}
(iter (for el in list)
      (for i upfrom 1)
      (if el (collect (cons el i))))
\end{program}
But on the list |(a b nil c)| this produces |((a . 1) (b . 2) (c .
4))| instead of the desired |((a . 1) (b . 2) (c . 3))|.  The problem
is that |i| is incremented each time through the loop, even when |el|
is |nil|.  

The problem could be solved elegantly if we could step |i| only when
we wished to.  This
can be accomplished for any \iter\ driver by writing |generate| (or
its synonym |generating|)
instead of |for|.  Doing so produces a ~generator~---a driver whose
values are yielded explicitly.  To obtain the next value of a
generator variable ~v~, write \linebreak |(next ~v~)|.  The value of a |next|
form is the next value of ~v~, as determined by its associated driver
clause.  |next| also has the side-effect of updating ~v~ to that
value.  If there is no next value, |next| will terminate the loop,
just as with a normal driver.


Using generators, we can now write our example like this:
\begin{program}
(iter (for el in list)
      (generate i upfrom 1)
      (if el (collect (cons el (next i)))))
\end{program}
Now |i| is updated only when |(next i)| is executed, and this occurs
only when |el| is \nonnil.

To better understand the relationship between ordinary drivers and
generators, observe that we can rewrite an ordinary driver using its
generator form immediately followed by |next|, as this example shows:
\begin{program}
(iter (generating i from 1 to 10)
      (next i)
      ...)
\end{program}
Provided that the loop body contains no |(next i)| forms, this will
behave just as if we had written |(for i from 1 to 10)|.

We can still refer to a driver variable ~v~ without using |next|; in
this case, its value is that given to it by the last evaluation of
|(next ~v~)|.  Before |(next ~v~)| has been called the first time, the
value of ~v~ is undefined.  

This semantics is more flexible than
one in which ~v~ begins the loop bound to its first value and calls
of |next| supply subsequent values, because it means the loop will not
terminate too soon if the generator's sequence is empty.  For
instance, consider the following code, which tags \nonnil\ elements of
a list using a list of tags, and also counts the null elements.
(We assume there are at least as many tags as \nonnil\ elements.)
\begin{program}
(let* ((counter 0)
       (tagged-list (iter (for el in list)
                          (generating tag in tag-list) 
                          (if (null el)
                              (incf counter)
                              (collect (cons el (next tag)))))))
  ...)
\end{program}
It may be that there are just as many tags as non-null elements of
|list|.  If all the elements of |list| are null, we still want the
counting to proceed, even though |tag-list| is \nil.  If |tag| had to be
assigned its first value before the loop begins, we would have had to
terminate the loop before the first iteration, since when |tag-list|
is \nil, |tag| 
has no first value.  With the existing semantics, however, |(next
tag)| will never execute, so the iteration will cover all the elements of
|list|. 
        
When the ``variable'' of a driver clause is actually a destructuring
template containing several variables, all the variables are eligible
for use with |next|.  As before, |(next ~v~)| evaluates to ~v's~ next
value; but the effect is to update all of the template's variables.
For instance, the following code will return the list |(a 2 c)|.
\begin{program}
(iter (generating (key . item) in '((a . 1) (b . 2) (c . 3)))
      (collect (next key))
      (collect (next item)))
\end{program}

Only driver clauses with variables can be made into generators.  This
includes all clauses mentioned so far except for |repeat|.  It does
~not~ include |for\dots previous|, |for\dots =|,
|for\dots initially\dots then| or |for\dots first\dots then| (see
below). 

\subsubsection{Previous Values of Driver Variables}

Often one would like to access the value of a variable on a previous
iteration.  \iter\ provides a special clause for accomplishing this.

\begin{clauses}

\clause{|for| ~pvar~ |previous| ~var~ \opt\ |initially| ~init~
|back| ~n~}
\clindexx{for}{previous}
Sets ~pvar~ to the previous value of ~var,~ which should be a driver
variable, a variable from another |for\dots previous| clause, or a
variable established by a |for\dots =|, 
|for\dots initially\dots then| or |for\dots first\dots then| clause
(see section \ref{setting}).
Initially, ~pvar~ is given the value ~init~ (which defaults to \nil).  
The ~init~ expression will be moved outside the loop body, so it
should not depend on anything computed within the loop.
~pvar~ retains the value of ~init~ until ~var~ is set to its second
value, at which point ~pvar~ is set to ~var's~ first value; and so on.  

\cpar The
argument ~n~ to |back| must be a constant, positive integer, and
defaults to 1.  It determines how many iterations back ~pvar~ should
track ~var~.  For example, when ~n~ is 2, then ~pvar~ will be assigned
~var's~ first value when ~var~ is set to its third value.

\cpar A |for\dots previous| clause may occur after or before its
associated driver clause. |for\dots previous| works with generators as
well as ordinary drivers.

\pagebreak[3]

\cpar Example:
\begin{program}
(iter (for el in '(1 2 3 4))
      (for p-el previous el)
      (for pp-el previous p-el initially 0)
      (collect pp-el))
\end{program}
This evaluates to |(0 0 1 2)|.  It could have been written more
economically as 
\begin{program}
(iter (for el in '(1 2 3 4))
      (for pp-el previous el back 2 initially 0)
      (collect pp-el))
\end{program}


\end{clauses}

\subsection{Variable Binding and Setting}
\label{setting}

Several clauses exist for establishing new variable bindings or for
setting variables in the loop.  They all support destructuring.

\begin{clauses}

\clausex{with}{~var~ \opt\ |=| ~value~}
Causes ~var~ to be bound to value before the loop body is entered.
If ~value~ is not supplied, ~var~ assumes a default
binding, which will be 
\nil\ in the absence of declarations.  Also, if ~value~ is not
supplied, no destructuring is performed; instead, ~var~ may be a list of
symbols, all of which are given default 
bindings.  If ~value~ is supplied, ~var~ is bound to it, with
destructuring.  

\cpar Because |with|
creates bindings whose scope includes the entire \iter\ form, it is
good style to put all |with| clauses at the beginning.

\cpar Successive occurrences of |with| result in sequential
bindings (as with 
|let*|).  There is no way to obtain parallel bindings; see
section \ref{bindings} for a rationale.  


\clause{|for| ~var~ |=| ~expr~}
\clindexx{for}{=}
On each iteration, ~expr~ is evaluated and ~var~ is set
to its value. 

\cpar This clause may appear to do the same thing as |for\dots next|.
In fact, they are quite different.  |for\dots =| provides only three
services: it sets up a binding for ~var~, sets it to ~expr~ on each
iteration, and makes it possible to use |for\dots previous| with
~var~.  |for\dots next| provides these services in addition to the
ability to turn the driver into a generator.
%Also, the code which sets ~var~ appears in the loop body in the same
%place as the |for\dots =| clause; the code for |for\dots next| appears
%at the top of the loop, as with other drivers (except when being used
%as a generator).

\clause{|for| ~var~ |initially| ~init-expr~ |then| ~then-expr~}
\clindexxx{for}{initially}{then}
Before the loop begins, ~var~ is set to ~init-expr;~ on all
iterations after the first it is set to ~then-expr.~  This clause
must occur at top-level.  ~init-expr~ will be moved outside the loop
body and ~then-expr~ will be moved to the end of the loop body, so
they are subject to code motion problems (see section
\ref{code-movement}). 

\cpar This clause may appear to be similar to |for\dots next|, but in
fact they differ significantly.  |for\dots initially\dots then| is
typically used to give ~var~ its first value before the loop begins,
and subsequent values on following iterations.  This is incompatible
with generators, whose first value and subsequent values must all be
computed by |(next ~var~)|.  Also, the update of ~var~ in 
|for\dots initially\dots then| does not occur at the location of the
clause. 
Use |for\dots initially\dots then| for
one-shot computations where its idiom is more convenient, but use
|for\dots next| for extending \iter\ with new drivers (see section
\ref{extend}).

\clause{|for| ~var~ |first| ~first-expr~ |then| ~then-expr~}
\clindexxx{for}{first}{then}
The first time through the loop, ~var~ is set to ~first-expr;~ on
subsequent iterations, it is set to ~then-expr.~  This differs from
|for\dots initially| in that ~var~ is set to ~first-expr~
inside the loop body, 
so ~first-expr~ may depend on the results of other clauses.  For
instance, 
\begin{program}
(iter (for num in list)
      (for i first num then (1+ i))
      ...)
\end{program}
will set |i| to the first element of |list| on the first
iteration, whereas 
\begin{program}
(iter (for num in list)
      (for i initially num then (1+ i))
      ...)
\end{program}
is probably erroneous; |i| will be bound to |num|'s
default binding (usually \nil) for the first iteration.

\end{clauses}

\begin{note}{Compatibility note}
|loop|'s |for\dots =| works like \iter's, but |loop| used the syntax
|for\dots =\dots then| to mean |for\dots initially\dots then|.  It was
felt that these two operations were sufficiently different to warrant
different keywords.

Also, the |for| in the above three clauses is misleading,
since none is true driver (e.g. none has a corresponding |generate|
form).  |setting| would have been a better choice, but |for| was used
to retain some compatibility with |loop|.  
\end{note}


%%%%%%%%%%%%%%%%%%%%%%%%%%%%%%%%
\subsection{Gathering Clauses}
Many of \iter's clauses accumulate values into a variable, or set a
variable under certain conditions.  At the end of the loop, this
variable contains the desired result.  All these clauses have an
optional |into| keyword, whose argument should be a symbol.  If the
|into| keyword is not supplied, the accumulation variable will be
internally generated and its value will be returned at the end of the
loop; if a variable is specified, that variable is used for the
accumulation, and is not returned as a result---it is up to the user
to return it explicitly, in the loop's epilogue code (see |finally|).
It is safe to examine the accumulation variable during the loop, but
it should not be modified.

These clauses all begin with a verb.  When the verb does
not conflict with an existing Common Lisp function, then it may be
used in either its infinitival or present-participle form (e.g. |sum|,
|summing|).  However, when there is a conflict with Common Lisp, only
the present-participle form may be used (e.g. |unioning|).  This is to
prevent \iter\ clauses from clashing with Common Lisp
functions. 

%although these clauses are described as ``producing a value,'' it is a
%mistake to think of the lisp list representing the clause as a
%value-producing form in the usual way.  clauses may legally be written
%where a value is expected, e.g. |(setq x (sum i))|, but the
%lisp value of a clause in such a context is undefined.

\subsubsection{Reductions}

~Reduction~ is an extremely common iteration pattern in which
the results of successive applications of a
binary operation are accumulated.
For example, a loop that computes the sum of the
elements of a list is performing a reduction with the addition
operation.  This could be written in Common Lisp as  |(reduce \#'+
list)| or with \iter\ as
\begin{program}
(iter (for el in list)
      (sum el))
\end{program}

\begin{clauses}

\clausex{sum}{~expr~ \opt\ |into| ~var~}
Each time through the loop, ~expr~ is evaluated and added to a
variable, which is bound initially to zero.  If ~expr~ has a type,
it is ~not~ used as the type of the sum variable, which is always
|number|.  To get the result variable to be of a more specific
type, use an explicit variable, as in
\begin{program}
(iter (for el in number-list)
      (sum el into x)
      (declare (fixnum x))
      (finally (return x)))
\end{program}

\clausex{multiply}{~expr~ \opt\ |into| ~var~}
Like |sum|, but the initial value of the result variable is $1$,
and the variable is updated by multiplying ~expr~ into it.

\clausex{counting}{~expr~ \opt\ |into| ~var~}
~expr~ is evaluated on each iteration.  If it is \nonnil, the
accumulation variable, initially zero, is incremented.

\startitem
\Clausex{maximize}{~expr~ \opt\ |into| ~var~}
\Clausex{minimize}{~expr~ \opt\ |into| ~var~}
\finishitem
~expr~ is evaluated on each iteration and its extremum (maximum or
minimum) is stored in the accumulation variable.  If ~expr~ is never
evaluated, then the result is \nil\ (if the accumulation variable is
untyped) or $0$ (if it has a numeric type).

\clausex{reducing}{~expr~ |by| ~func~ \opt\ |initial-value| ~init-val~ |into| ~var~}
This is a general way to perform reductions. ~func~ should be a
function of two arguments, the first of which will be the value
computed so far and
the second of which will be the value of ~expr~.  It should return the
new value.
|reducing| is roughly equivalent to the
Common Lisp |(reduce ~func~ ~sequence~ :key ~expr-function~)|, where
~expr-function~ is used to derive values from the successive elements
of ~sequence~.

\cpar If the |reducing| clause is never executed, the result is
undefined.

\cpar It is not necessary to provide an
initial value, but better code can be generated if one is supplied.
Regardless of its location in the \iter\ body, ~init-val~ will be
evaluated before the loop is entered, so it should not depend on any
value computed inside the \iter\ form.  

%\cpar if a ~var~ is not specified, you can get \iter\ to declare the
%type of the internal variable by putting a |the| expression around
%~func.~  see section \ref{types}.

\end{clauses}

%\begin{note}{implementation note}
%in principle, |maximize| and |minimize| can be thought of
%as reductions where the initial value is the smallest (or largest)
%value that the accumulation variable can assume.  because lisp's
%bignums can represent arbitrary integers, these clauses cannot be
%implemented as reductions in general.  if, however, the type of 
%~expr~ or ~var~ can be determined to be a fixnum
%or a float, \iter\ will implement the clause as a true reduction,
%using one of the constants |most-negative-fixnum|, |%most-positive-fixnum|, 
%|most-negative-short-float|, etc. as appropriate.
%\end{note}

\subsubsection{Accumulations}

All the predefined accumulation clauses add values to a sequence.  If
the sequence is a list, they all 
have the property that the partial list is kept in the correct order
and available for inspection at any point in the loop.

\begin{clauses}

\clausex{collect}{~expr~ \opt\ |into| ~var~ |at| ~place~ |result-type| ~type~}
Produces a sequence of the values of ~expr~ on each iteration. ~place~
indicates where the next value of ~expr~ is added to the list and may
be one of the symbols |start|, |beginning| (a synonym for |start|) or
|end|.  The symbol may be quoted, but need not be.  The default is
|end|.  For example,
\begin{program}
(iter (for i from 1 to 5)
      (collect i))
\end{program}
produces |(1 2 3 4 5)|, whereas
\begin{program}
(iter (for i from 1 to 5)
      (collect i at beginning))
\end{program}
produces |(5 4 3 2 1)| (and is likely to be faster in most Common Lisp
implementations).

\cpar If ~type~ is provided, it should be a subtype of |sequence|.
The default is |list|.  Specifying a type other than |list| will
result in |collect| returning a sequence of that type.  ~However,~ the
type of the sequence being constructed when inside the loop body is
undefined when a non-|list| type is specified.  (As with ~place~,
quoting ~type~ is optional.)

\clausex{adjoining}{~expr~ \opt\ |into| ~var~ |test| ~test~ |at| ~place~ |result-type| ~type~}
Like |collect|, but only adds the value of ~expr~ if it is not
already present.  ~test,~ which defaults to |\#'eql|, is
the test to be used with |member|.

\startitem
\Clausex{appending}{~expr~ \opt\ |into| ~var~ |at| ~place~}
\Clausex{nconcing}{\ ~expr~ \opt\ |into| ~var~ |at| ~place~}
\Clausex{unioning}{\ ~expr~ \opt\ |into| ~var~ |test| ~test~ |at| ~place~}
\Clausex{nunioning}{~expr~ \opt\ |into| ~var~ |test| ~test~ |at| ~place~}
\finishitem
These are like |collect|, but behave like the Common Lisp functions
|append|, |nconc|, |union| or |nunion|.  
As in Common Lisp, they work only on lists.  Also as in Common Lisp,
|unioning| and |nunioning| assume that the value of ~expr~ contains no
duplicates. 

\clausex{accumulate}{~expr~ |by| ~func~ \opt\ |initial-value| ~init-val~ |into| ~var~}
This is a general-purpose accumulation clause. ~func~ should be a
function of two arguments, the value of ~expr~ and the value
accumulated so far in the iteration, and it should return the updated
value.  If no initial value is supplied, \nil\ is used.  

%\cpar If a ~var~ is not specified, you can get \iter\ to declare the
%type of the internal variable by putting a |the| expression around
%~func.~  see section \ref{types}.

\cpar The differences between |accumulate| and |reducing| are slight.
One difference is that the functions take their arguments in a
different order.  Another is that in the absence of ~init-val~,
|accumulate| will use \nil, whereas |reducing| will generate different
code that avoids any dependence on the initial value.
The reason for having both clauses is that one usually
thinks of reductions (like |sum|) and accumulations (like |collect|)
as different beasts.

\end{clauses}


\subsubsection{Finders}

A ~finder~ is a clause whose value is an expression that meets some
condition.

\begin{clauses}

\clause{|finding| ~expr~ |such-that| ~test~ \opt\ |into| ~var~ |on-failure| ~failure-value~}
\clindexx{finding}{such-that}
If ~test~ (which is an expression) ever evaluates to \nonnil, the loop
is terminated, the 
epilogue code is run and the value of ~expr~ is returned.  Otherwise,
\nil\ (or ~failure-value,~ if provided) is returned.  If ~var~ is
provided, it will have either the \nonnil\ value of ~expr~ or
~failure-value~ when the epilogue code is run.

\cpar As a special case, if the ~test~ expression is a sharp-quoted
function, then it is applied to ~expr~ instead of being simply
evaluated.  E.g. |(finding x such-that \#'evenp)| is equivalent to
|(finding x such-that (evenp x))|.  

%\cpar although ~test~ need have nothing to do with ~%expr~ as in
%|(finding j such-that (> i 3))|, it usually 
%will: |(finding (length el) such-that (oddp (length el)))|.  to
%avoid performing the |length| computation twice, you could write
%|(finding (length el) such-that \#'oddp)| or |(finding (length
%el) such-that 'oddp)|; for these cases, \iter\ generates code that
%executes ~expr~ only once.  the code for |\#'oddp|
%is slightly different from that for {\lisp 'oddp}; see the discussion
%under {\lisp for\dots in} and {\lisp for\dots on}.

\cpar |On-failure| is a misnomer. Because it is always evaluated, it behaves
more like the default third argument to the |gethash| function. As a result,
|on-failure (error "Not found")| makes no sense. Instead, the clauses |leave|
or |thereis| can be used in conjunction with |finally| as follows:
\begin{program}
(iter (for x in '(1 2 3))
      (if (evenp x) (leave x))
      (finally (error "not found")))
\end{program}

\cpar This clause may appear multiple times when all defaults are
identical. It can also be used together with either |always|/|never| or
|thereis| if their defaults match. More specifically, |on-failure nil| is
compatible with |thereis|, while |on-failure t| is compatible with |always|
and |never| clauses.
\begin{program}
(iter (for i in '(7 -4 2 -3))
      (if (plusp i)
	  (finding    i  such-that (evenp i))
          (finding (- i) such-that (oddp i))))
\end{program}

\startitem
\Clause{|finding| ~expr~ |maximizing| ~m-expr~ \opt\ |into| ~var~}
\clindexx{finding}{maximizing}
\Clause{|finding| ~expr~ |minimizing| ~m-expr~ \opt\ |into| ~var~}
\clindexx{finding}{minimizing}
\finishitem
Computes the extremum (maximum or minimum) value of ~m-expr~ over all
iterations, and returns the value of ~expr~ corresponding to the
extremum.  ~expr~ is evaluated inside the loop at the time the new
extremum is established.  If ~m-expr~ is never evaluated (due to, for
example, being embedded in a conditional clause), then the returned
value depends on the type, if any, of ~expr~ (or ~var,~ if
one is supplied).  If there is no type, the returned
value will be \nil; if the type is numeric, the returned value will be
zero.

\cpar For these two clauses, ~var~ may be a list of two
symbols; in that case, the first is used to record ~expr~ and
the second, ~m-expr.~  

\cpar As with |finding\dots such-that|, if ~m-expr~ is a sharp-quoted
function, then it is called on ~expr~ instead of being evaluated.

\end{clauses}

\subsubsection{Boolean Tests}

\begin{clauses}

\clausex{first-iteration-p}{}
Returns |t| in the first cycle of the loop, otherwise \nil.

\clausex{first-time-p}{}
Returns |t| the first time the expression is evaluated, and then \nil\ forever.
This clause comes handy when printing (optional) elements separated
by a comma:

\begin{program}
(iter (for el in '(nil 1 2 nil 3))
      (when el
        (unless (first-time-p)
          (princ ", "))
        (princ el)))
\end{program}
produces |"1, 2, 3"|.

\end{clauses}

\subsubsection{Aggregated Boolean Tests}

\begin{clauses}

\clausex{always}{~expr~}
If ~expr~ ever evaluates to
\nil, then \nil\ is immediately returned; the epilogue code is not
executed.  If ~expr~ never evaluates to \nil, the epilogue code
is executed and the last value of ~expr~ (or |t| if ~expr~ was never
evaluated) is returned (whereas |loop| would constantly return |t|).

% mention last evaluated clause when multiple always clauses?

\clausex{never}{~expr~}
Like |(always (not ~expr~))|, except it does not influence the last
value returned by a possible other |always| clause. That is,
\begin{program}
(iter (repeat 2)
      (always 2)
      (never nil)) \yields 2 ; not t
\end{program}

\clausex{thereis}{~expr~}
If ~expr~ is ever \nonnil,
its value is immediately returned without running epilogue code.
Otherwise, the epilogue code is performed and \nil\ is returned.

This clause cannot be used together with |always| or |never|, because their
defaults are opposed (similarly, |(loop always 3 thereis nil)| refuses to
compile in some implementations of |loop|).

\end{clauses}


\subsection{Control Flow}
\label{control-flow}
Several clauses can be used to alter the usual flow of control in a loop.

Note: the clauses of this and subsequent sections don't adhere to \iter's
usual syntax, but instead use standard Common Lisp syntax.  Hence the
format for describing syntax subsequently is
like the standard format used in the Common Lisp manual, not like the
descriptions of clauses above.

\begin{clauses}

\clausex{finish}{}
Stops the loop and runs the epilogue code.

%for example:
%
%\begin{program}
%(iter (with answer = nil)
%         (initially (make-a-mess))
%         (for i from 1 to 10)
%         (when (correct? i)
%           (setq answer i)
%           (finish))
%         (finally (cleanup)))
%\end{program}
%
%this code will execute |cleanup| whether or not the test |(correct?
%i)| ever succeeds. 
%the (more elegant) formulation,
%\begin{program}
%(iter (initially (make-a-mess))
%         (for i from 1 to 10)
%         (finding i such-that (correct? i))
%         (finally (cleanup)))
%\end{program}
%would not execute |cleanup| if |(correct? i)| succeeded; it
%would do an immediate return.

\clausex{leave}{\opt\ ~value~}
Immediately returns ~value~ (default \nil) from the current \iter\
form, skipping the epilogue code.  Equivalent to using |return-from|.

\clausex{next-iteration}{}
Skips the remainder of the loop body and begins the next iteration of
the loop.

\clausex{while}{~expr~}
If ~expr~ ever evaluates to \nil, the loop is terminated and the
epilogue code executed.  Equivalent to |(if (not ~expr~) (finish))|.

\clausex{until}{~expr~}
Equivalent to |(if ~expr~ (finish))|.

\clausex{if-first-time}{~then~ \opt\ ~else~}
If this clause is being executed for the first time in this invocation
of the \iter\ form, then the ~then~ code is evaluated; otherwise the
~else~ code is evaluated.

\cpar |(for ~var~ first ~expr1~ then ~expr2~)| is almost equivalent to
\begin{program}
(if-first-time (dsetq ~var~ ~expr1~)
               (dsetq ~var~ ~expr2~))
\end{program}
The only difference is that the |for| version makes ~var~ available
for use with |for\dots previous|.

\end{clauses}

\subsection{Code Placement}
When fine control is desired over where code appears in a loop
generated by \iter, the following special clauses may be useful.
They are all subject to code-motion problems (see section
\ref{code-movement}).  

\begin{clauses}

\clausex{initially}{|\&rest| ~forms~}
The lisp ~forms~ are placed in the prologue section of the loop, where
they are executed once, before the loop body is entered.

\clausex{after-each}{|\&rest| ~forms~}
The ~forms~ are placed at the end of the loop body, where they
are executed after each iteration.  Unlike the other clauses in this
section, ~forms~ may contain \iter\ clauses.

\clausex{else}{|\&rest| ~forms~}
The lisp ~forms~ are placed in the epilogue section of the loop, where they
are executed if this |else| clause is never met during execution of the
loop and the loop terminates normally.

\clausex{finally}{|\&rest| ~forms~}
The lisp ~forms~ are placed in the epilogue section of the loop, where
they are executed after the loop has terminated normally.

\clausex{finally-protected}{|\&rest| ~forms~}
The lisp ~forms~ are placed in the second form of an unwind-protect
outside the loop.  They are always executed after the loop has
terminated, regardless of how the termination occurred.

\end{clauses}


%%%%%%%%%%%%%%%%%%%%%%%%%%%%%%%%%%%%%%%%%%%%%%%%%%%%%%%%%%%%%%%%
\section{Other Features}

\subsection{Multiple Accumulations}
\label{multiple}

\begin{sloppypar}
It is permitted to have more than one clause accumulate into the same
variable, as in the following:
\begin{program}
(iter (for i from 1 to 10)
      (collect i into nums)
      (collect (sqrt i) into nums)
      (finally (return nums)))
\end{program}
Clauses can only accumulate into the same variable if they are
compatible.  |collect|, |adjoining|, |appending|, |nconcing|,
|unioning| and |nunioning| are compatible with each other; |sum|,
|multiply| and |counting| are compatible; |always| and |never| are compatible;
|finding| \dots |such-that| is compatible with either |thereis| or |always|
and |never| when their defaults match; and |maximize| and |minimize| clauses
are compatible only with other |maximize| and |minimize| clauses,
respectively.

%note that the same variable ~cannot~ be both an accumulation
%variable and an ordinary variable; there can be only one variable with
%a given name within an \iter\ form. 
\end{sloppypar}

\subsection{Named Blocks}

Like Common Lisp |block|s, \iter\ forms can be given names.  The
name should be a single symbol, and it must be the first form in the
\iter.  The generated code behaves exactly like a named block; in
particular, |(return-from ~name~)| can be used to exit it:
\begin{program}
(iter fred
      (for i from 1 to 10)
      (iter barney
            (for j from i to 10)
            (if (> (* i j) 17)
                (return-from fred j))))
\end{program}
An \iter\ form that is not given a name is implicitly named \nil.

Sometimes one would like to write an expression in an inner \iter\ form,
but have it processed by an outer \iter\ form.  This is possible with
the |in| clause.  

\begin{clauses}

\clausex{in}{~name~ |\&rest| ~forms~}
Evaluates ~forms~ as if they were part of the \iter\ form named
~name~.  In other words, \iter\ clauses are processed by the \iter\
form named ~name,~ and not by any \iter\ forms that occur inside ~name.~

\cpar As an example, consider the problem of collecting a list of the
elements in a two-dimensional array.  The naive solution,
\begin{program}
(iter (for i below (array-dimension ar 0))
      (iter (for j below (array-dimension ar 1))
            (collect (aref ar i j))))
\end{program}
\noindent is wrong because the list created by the inner \iter\ is simply
ignored by the outer one.  But using |in| we can write:
\begin{program}
(iter outer (for i below (array-dimension ar 0))
      (iter (for j below (array-dimension ar 1))
            (in outer (collect (aref ar i j)))))
\end{program}
\noindent which has the desired result.

\end{clauses}

\subsection{Destructuring}
\label{destructuring}

In many places within \iter\ clauses where a variable is expected, a
list can be written instead.  In these cases, the value to be assigned
is ~destructured~ according to the pattern
described by the list.  As a simple example, the clause
\begin{program}
(for (key . item) in alist)
\end{program}
\noindent will result in |key| being set to the |car| of
each element in |alist|, and |item| being set to the |cdr|.  The
pattern list may be nested to arbitrary depth, and (as the example
shows) need not be terminated with \nil; the only requirement is that
each leaf be a bindable symbol (or \nil, in which case no binding is
generated for that piece of the structure).

Sometimes, you might like to do the equivalent of a
|multiple-value-setq| in a clause.  This 
``multiple-value destructuring'' can be expressed by writing 
\linebreak |(values $pat_1$ $pat_2 \ldots$)| for a destructuring
pattern, as in 
\begin{program}
(for (values (a . b) c d) = (three-valued-function ...))
\end{program}
\begin{sloppypar}
Note that the $pat_i$ can themselves be destructuring patterns (though
not multiple-value destructuring patterns).  You can't do multiple-value
destructuring in a |with| clause; instead wrap the whole \iter\
form in a |multiple-value-bind|.
\end{sloppypar}

\begin{note}{Rationale}
There are subtle interactions between variable declarations and
evaluation order that make the correct implementation of
multiple-value destructuring in a |with| somewhat tricky.
\end{note}

The destructuring feature of \iter\ is available as a separate
mechanism, using the |dsetq| macro:

\begin{clauses}
\defmacro{dsetq}{template expr}
Performs destructuring of ~expr~ using ~template~.   May be used
outside of an \iter\ form. Yields the primary value of ~expr~.

\end{clauses}

\subsection{On-line Help}

\begin{sloppypar}
There is a limited facility for on-line help, in the form of the 
|display-iterate-clauses| function.
\end{sloppypar}

\begin{clauses}

\defun{display-iterate-clauses}{\opt\ clause-spec}
Displays a list of \iter\ clauses.  If ~clause-spec~ is not
provided, all clauses are shown; if it is a symbol, all clauses
beginning with that symbol are shown; and if it is a list of symbols,
all clauses for which ~clause-spec~ is a prefix are shown.

\end{clauses}

\subsection{Parallel Binding and Stepping}
\label{bindings}

The parallel binding and stepping of variables is a feature that
\iter\ does ~not~ have.  This section attempts to provide a rationale.

We say that two variables are bound ~in parallel~ if neither
binding shadows the other.  This is the usual semantics of |let|
(as opposed to |let*|).  Similarly, we can say that iteration
variables are stepped in parallel if neither variable is updated
before the other, conceptually speaking; in other words, if the code
to update each variable can reference the old values of both variables.

|loop| allows parallel binding of variables and parallel stepping of
driver variables.  My view is that if you are depending on the 
serial/parallel distinction, you are doing something obscure.  If you
need to bind 
variables in parallel using |with|, then you must be using a
variable name that shadows a name in the existing lexical environment.
Don't do that.  The most common use for parallel stepping is to track
the values of variables on the previous iteration, but in fact this
does not require parallel stepping at all; the following will work:
\begin{program}
(iter (for current in list)
      (for prev previous current)
      ...)
\end{program}


%%%%%%%%%%%%%%%%%%%%%%%%%%%%%%%%%%%%%%%%%%%%%%%%%%%%%%%%%%%%%%%%
\section{Types and Declarations}
\label{types}

\subsection{Discussion}

Sometimes efficiency dictates that the types of variables be declared.
This type information needs to be communicated to \iter\ so it can
bind variables to appropriate values.  Furthermore, \iter\ must often
generate internal variables invisible to the user; there needs to be a
way for these to be declared.

As an example, consider this code, which will return the number of
odd elements in |number-list|:
\begin{program}
(iter (for el in number-list)
      (count (oddp el)))
\end{program}
In processing this form,
\iter\ will create an internal variable, let us call it |list17|, to
hold the successive |cdr|s of |number-list|, and  
will bind the variable to |number-list|.  It will also generate a
default binding for |el|; only inside the body of the loop will |el|
be set to the |car| of |list17|.  Finally, \iter\ will generate a
variable, call it |result|, to hold the result of the count, and will
bind it to zero.

When dealing with type declarations, \iter\ observes one simple rule:
~it will never generate a declaration unless requested to do so.~  The
reason is that such declarations might mask errors in compiled code by
avoiding error-checks; the resulting problems would be doubly hard to
track down because the declarations would be hidden from the
programmer.  Of course, a compiler might omit error-checks even in the
absence of declarations, though this behavior can usually be avoided,
e.g. by saying |(declaim (optimize (safety 3)))|.

So, the above \iter\ form will generate code with no declarations.
But say we wish to declare the types of |el| and the internal
variables |list17| and |result|.  How is this done?

Declaring the type of |el| is easy, since the programmer knows
the variable's name:
\begin{program}
(iter (for el in number-list)
      (declare (fixnum el))
      (counting (oddp el)))
\end{program}
\iter\ can read variable type declarations like this one.  Before
processing any clauses, it scans the entire top-level form for type
declarations 
and records the types, so that variable bindings can be performed
correctly.  In this case, |el| will be bound to zero
instead of \nil.  Also, \iter\ collects all the top-level declarations
and puts them at the begining of the generated code, so it is not
necessary to place all declarations at the beginning of an \iter\
form; instead, they can be written near the variables whose types they
declare.  

Since \iter\ is not part of the compiler, it will not know
about declarations that occur outside an \iter\ form; these
declarations must be repeated inside the form.

Here is another way we could have declared the type of |el|:
\begin{program}
(iter (for (the fixnum el) in number-list)
      (counting (oddp el)))
\end{program}
\iter\ extends the Common Lisp |the|\lispindex{the}
form to apply to variables as well as value-producing forms; anywhere
a variable is allowed---in a |with| clause, as the iteration
variable in a driver clause, as the |into| argument of an
accumulation clause, even inside a destructuring template---you can
write |(the ~type~ ~symbol~)| instead.

There is one crucial difference between using a |the| form and
actually declaring the variable: explicit declarations are always
placed in the generated code, but type information from a |the|
form is not turned into an actual declaration unless you tell \iter\
to do so using |iterate:declare-variables|.  See below.

\begin{sloppypar}
Declaring the types of internal variables is harder than declaring the
types of explicitly mentioned variables, since their names
are unknown.  You do it by declaring |iterate:declare-variables|
somewhere inside the top level of the \iter\ form.  (This will also
generate declarations for variables declared using |the|.)
\iter\ does not provide much selectivity here: it's all or none. 
And unfortunately, since \iter\ is not privy to compiler information
but instead reads declarations itself, it will not hear if you
|(declaim (iterate:declare-variables))|.  Instead, set the variable 
|iterate::*always-declare-variables*| to |t| at
compile-time, using |eval-when|.
\end{sloppypar}

To determine the appropriate types for internal variables, \iter\ uses
three sources of information:

\begin{itemize}

\item Often, the particular clause dictates a certain type for a
variable; \iter\ will use this information when available.  In the
current example, the variable |list17| will be given the type
|list|, since that is the only type that makes sense; and the
variable |result| will be given the type |fixnum|, on the
assumption that you will not be counting high enough to need bignums.
You can override this assumption only by using and explicitly declaring a
variable: 
\begin{verbatim}
(iter (declare (iterate:declare-variables))
      (for el in number-list)
      (count (oddp el) into my-result)
      (declare (integer my-result))
      (finally (return my-result)))
\end{verbatim}

\begin{sloppypar}
Other examples of the type assumptions that \iter\ makes are: type
|list| for |into| variables of collection clauses; type |list| for
expressions that are to be destructured; type |vector| for the
variable holding the vector in a |for\dots in-vector| clause, and
similarly for |string| and the |for\dots in-string| clause;
and the implementation-dependent type
|(type-of array-dimension-limit)| for the index and limit
variables generated by sequence iteration drivers like |for\dots
in-vector| and |for\dots in-string| (but not |for\dots in-sequence|,
because it may be used to iterate over a list). 
\end{sloppypar}

\item Sometimes, \iter\ will examine expressions and try to determine
their types in a simple-minded way.  If the expression is
self-evaluating (like a number, for instance), \iter\ knows that the
expression's type is the same as the type of the value it denotes, so
it can use that type.  If the expression is of the form |(the ~type~
~expr~)|, \iter\ is smart enough to extract ~type~ and use it.
However, the current version of \iter\ does not  
examine declarations of function result types or do any type
inference.  It will not determine, for 
example, that the type of |(+ 3 4)| is |fixnum|, or even
|number|.

\item In some cases, the type of an internal variable should match the
type of some other variable.  For instance, \iter\ generates an
internal variable for |(f x)| in the
clause |(for i from 1 to (f x))|, and in the absence of other
information will give it the same type as |i|.  If, however, the
expression had been written |(the fixnum (f x))|, then \iter\
would have given the internal variable the type |fixnum|
regardless of |i|'s type.  The type incompatibility errors that
could arise in this situation are not checked for.

\end{itemize}

Note that if you do declare |iterate:declare-variables|, then
\iter\ may declare user variables as well as internal ones if they do
not already have declarations, though only for variables that it
binds.  For instance, in this code:

\begin{program}
(iter (declare (iterate:declare-variables))
      (for i from 1 to 10)
      (collect i into var))
\end{program}
the variable |var| will be declared to be of type |list|.


\subsection{Summary}

\iter\ understands standard Common Lisp variable type declarations
that occur within an \iter\ form and
will pass them through to the generated code.  If the declaration
|(iterate:declare-variables)|\lispindex{declare-variables}
appears at the top level of an 
\iter\ form, or if 
|iterate::*always-declare-variables*|\lispindex{*always-declare-variables*} 
is \nonnil, then \iter\ will use the type information gleaned from user
declarations, self-evaluating expressions and |the| expressions,
combined with reasonable assumptions, to determine variable
types and declare them.


%%%%%%%%%%%%%%%%%%%%%%%%%%%%%%%%%%%%%%%%%%%%%%%%%%%%%%%%%%%%%%%%
\section{Problems with Code Movement}
\label{code-movement}

Some \iter\ clauses, or parts of clauses, result in code being
moved from the location of the clause to other parts of the loop.
Drivers behave this way, as do code-placement clauses like |initially|
and |finally|.  When using these clauses, there is a danger of writing
an expression that makes sense in its apparent location but will be
invalid or have a different meaning in another location.  For example:
\begin{program}
(iter (for i from 1 to 10)
      (let ((x 3))
        (initially (setq x 4))))
\end{program}
While it may appear that the |x| of |(initially (setq x 4))| is the
same as the |x| of |(let ((x 3)) \dots|, in fact they are not:
|initially| moves its code outside the loop body, so |x| would refer
to a global variable.  Here is another example of the same problem:
\begin{program}
(iter (for i from 1 to 10)
      (let ((x 3))
        (collect i into x)))
\end{program}
If this code were executed, |collect| would create a binding for its
|x| at the top level of the \iter\ form that the |let| will shadow.

Happily, \iter\ is smart enough to catch these errors; it
walks all problematical code to ensure that free variables are not
bound inside the loop body, and checks all variables it binds for the
same problem.

However, some errors cannot be caught:

\begin{program}
(iter (with x = 3)
      (for el in list)
      (setq x 1)
      (reducing el by \#'+ initial-value x))
\end{program}
|reducing| moves its |initial-value| argument to the initialization
part of the loop in order to produce more efficient code.  Since
\iter\ does not perform data-flow analysis, it cannot determine that
|x| is changed inside the loop; all it can establish is that |x| is
not bound internally.  Hence this code will not signal an
error and will use $3$ as the initial value of the reduction.

The following list summarizes all cases that are subject to these code
motion and variable-shadowing problems.
\begin{itemize}
\item Any variable for which \iter\ creates a binding, including those
used in |with| and the |into| keyword of many clauses.

\begin{sloppypar}
\item The special clauses which place code: |initially|, |after-each|, |else|,
|finally| and |finally-protected|. 
\end{sloppypar}

\item The variables of a |next| or |do-next| form.

\item The |initially| arguments of |for\dots initially\dots then| and
|for\dots previous|. 

\item The |then| argument of |for\dots initially\dots then|.

\item The |initial-value| arguments of |reducing| and |accumulate|.

\item The |on-failure| argument of |finding\dots such-that|.
\end{itemize}

%%%%%%%%%%%%%%%%%%%%%%%%%%%%%%%%%%%%%%%%%%%%%%%%%%%%%%%%%%%%%%%%
\section{Differences Between {\tt Iterate} and {\tt Loop}}
|loop| contains a great deal of complexity which \iter\ tries to
avoid.  Hence many esoteric features of |loop| don't exist in \iter.
Other features have been carried over, but in a cleaned-up form.  
And of course, many new features have been added; they are not
mentioned in this list.

\begin{itemize}

\item \iter's syntax is more Lisp-like than |loop|'s, having a higher
density of parens.

\item The current implementation of \iter, unlike the current version
of  |loop| (as documented in {\em Common Lisp, 2nd Ed.\/}), is
extensible (see section \ref{extend}).  

\item |loop| puts the updates of all driver variables at the top of
the loop; \iter\ leaves them where the driver clauses appear.

\item While for the most part \iter\ clauses that resemble |loop| clauses
behave similarly, there are some differences.  For instance, there is
no |for\dots =\dots then| in \iter; instead use 
|for\dots initially\dots then|.  

\item |loop| binds the variable |it| at certain times
to allow pseudo-English expressions like |when ~expr~ return it|. 
In \iter, you must bind ~expr~ to a variable yourself.  Note that
|when ~expr~ return it| is like |thereis ~expr~| except that the latter is an
accumulation clause and therefore competes with other accumulations
(remember section \ref{multiple} above).

% repeat different behaviour of |always| clause here?

\item |loop| has a special |return| clause, illustrated in the
previous item.  \iter\ doesn't need one,  since an ordinary Lisp
|return| has the same effect.

\item |loop| allows for parallel binding and stepping of iteration
variables.  \iter\ does not.  (See section \ref{bindings}.)

\item |loop| and \iter\ handle variable type declarations very
differently.  |loop| provides a special syntax for declaring variable
types, and does not examine declarations.  Moreover, the standard
implementation of |loop| will
generate declarations when none are requested.
\iter\ parses standard Common Lisp type declarations, and will never
declare a variable itself unless declarations are 
specifically requested.

\end{itemize}

%%%%%%%%%%%%%%%%%%%%%%%%%%%%%%%%%%%%%%%%%%%%%%%%%%%%%%%%%%%%%%%%
\section{Rolling Your Own}
\label{extend}

\subsection{Introduction}

\iter\ is extensible---you can write new clauses that embody new
iteration patterns.  You might want to write a new driver clause for a
data structure of your own, or you might want to write a clause that
collects or manipulates elements in a way not provided by \iter. 

This section describes how to write clauses for \iter.  Writing a
clause is like writing a macro.  In fact, writing a clause ~is~
writing a macro: since \iter\ code-walks its body and macroexpands,
you can add new abstractions to \iter\ with good old |defmacro|.

Actually, there are two extensions you can make to \iter\ that are
even easier than writing a macro.  They are adding a synonym for an
existing clause and defining a driver clause for an indexable
sequence.  These can be done with |defsynonym| and
|defclause-sequence|, respectively.  See section \ref{aids}, below.

The rest of this section explains how to write macros that expand into
\iter\ clauses.
Here's how you could add a simplified version of \iter's |multiply|
clause, if \iter\ didn't already have one:
\begin{program}
(defmacro multiply (expr)
  `(reducing ,expr by \#'* initial-value 1))
\end{program}

If you found yourself summing the square of an expression often, you
might want to write a macro for that.  A first cut might be
\begin{program}
(defmacro sum-of-squares (expr)
  `(sum (* ,expr ,expr)))
\end{program}
but if you are an experienced macro writer, you will realize that this
code will evaluate ~expr~ twice, which is probably a bad idea.  A
better version would use a temporary:
\begin{program}
(defmacro sum-of-squares (expr)
  (let ((temp (gensym)))
    `(let ((,temp ,expr))
       (sum (* ,temp ,temp)))))
\end{program}
Although this may seem complex, it is just the sort of thing you'd
have to go through to write any macro, which illustrates the point of
this section: if you can write macros, you can extend \iter.

Our macros don't use \iter's keyword-argument
syntax.  We could just use keywords with |defmacro|, but we would
still not be using \iter's clause indexing mechanism.  Unlike Lisp,
which uses just the first symbol of a form to determine what function
to call, \iter\ individuates clauses by the list of required keywords.
For instance, |for\dots in| and |for\dots in-vector| are different clauses
implemented by distinct Lisp functions.  

To buy into this indexing scheme, as well as the keyword-argument
syntax, use |defmacro-clause|:

\begin{clauses}

\defmacro{defmacro-clause}{arglist |\&body| body}
Defines a new \iter\ clause.  ~arglist~ is a list of symbols which are
alternating keywords and arguments.  \opt\ may be used, and the list
may be terminated by |\&sequence|.  ~body~ is an ordinary macro body,
as with |defmacro|.  If the first form of ~body~ is a string, it is
considered a documentation string and will be shown by
|display-iterate-clauses|. |defmacro-clause| will signal an error if 
defining the clause would result in an ambiguity.  E.g. you cannot
define the clause |for\dots from| because there would be no way to
distinguish it from a use of the |for| clause with optional keyword |from|.

\end{clauses}

\medskip

Here is |multiply| using |defmacro-clause|.  The keywords are capitalized
for readability.
\begin{program}
(defmacro-clause (MULTIPLY expr \opt\ INTO var)
  `(reducing ,expr by \#'* into ,var initial-value 1))
\end{program}
You don't have to worry about the case when |var| is not supplied; for
any clause with an |into| keyword, saying |into nil| is equivalent to
omitting the |into| entirely.

As another, more extended example, consider the fairly common
iteration pattern that 
involves finding the sequence element that maximizes (or minimizes) some
function.  \iter\ provides this as |finding\dots maximizing|, but it's
instructive to see how to write it.
Here, in pseudocode, is how you might write such a loop for
maximizing a function F:

\begin{quote}
\begin{tabbing}
set variable MAX-VAL to NIL; \\
set variable WINNER to NIL; \\
for\=\ each element EL in the sequence \\
\>  if\=\ MAX-VAL\=\ is NIL or F(EL) $>$ MAX-VAL then \\
\>\>    set MAX-VAL to F(EL); \\
\>\>    set WINNER to EL; \\
\>  end if; \\
end for; \\
return WINNER.
\end{tabbing}
\end{quote}

Here is the macro:
\begin{program}
(defmacro-clause (FINDING expr MAXIMIZING func \opt\ INTO var)
  (let ((max-val (gensym))
        (temp1 (gensym))
        (temp2 (gensym))
        (winner (or var iterate::*result-var*)))
    `(progn 
       (with ,max-val = nil)
       (with ,winner = nil)
       (cond 
        ((null ,max-val)
         (setq ,winner ,expr)
         (setq ,max-val (funcall ,func ,winner))
        (t
         (let* ((,temp1 ,expr)
                (,temp2 (funcall ,func ,temp1)))
           (when (> ,temp2 ,max-val)
             (setq ,max-val ,temp2)
             (setq ,winner ,temp1))))))
       (finally (leave ,winner)))))
\end{program}
Note that if no |into| variable is supplied, we use
|iterate::*result-var*|, which contains the internal variable into
which all clauses place their results.  If this variable is bound by
some clause, then \iter\ will return its value automatically;
otherwise, \nil\ will be returned.  

\subsection{Writing Drivers}

In principle, drivers can be implemented just as easily as other
\iter\ clauses.  In practice, they are a little harder to get right. 
As an example, consider writing a driver that
iterates over all the 
elements of a vector, ignoring its fill-pointer.  |for\dots in-vector|
won't work for this, because it observes the fill-pointer.  It's necessary to
use |array-dimension| instead of |length| to obtain the size of the
vector.  Here is one approach:

\begin{program}
(defmacro-clause (FOR var IN-WHOLE-VECTOR v)
  "All the elements of a vector (disregards fill-pointer)"
  (let ((vect (gensym))
        (index (gensym)))
    `(progn
       (with ,vect = ,v)
       (for ,index from 0 below (array-dimension ,vect 0))
       (for ,var = (aref ,vect ,index)))))
\end{program}
Note that we immediately put |v| in a variable, in case it is an
expression.  Again, this is just good Lisp macrology.  It also has a
subtle effect on the semantics of the driver: |v| is evaluated only
once, at the beginning of the loop, so changes to |v| in the loop have
no effect on the driver.  Similarly, the bounds for numerical iteration
e.g. the above |array-dimension| are also evaluated once only.  This is how
all of \iter's drivers work.


There is an important point concerning the |progn| in this code.  We
need the |progn|, of course, because we are returning several forms,
one of which is a driver.  But \iter\ drivers must occur at top-level.
Is this code in error?  No, because ~top-level~ is defined in \iter\
to include forms inside a |progn|.  This is just the definition of
top-level that Common Lisp uses, and for the same reason: to allow
macros to return multiple forms at top-level.

While our |for\dots in-whole-vector| clause will work, it is not
ideal.  In particular, it does not support generating.  Do do so, we
need to use |for\dots next| or |for\dots do-next|.  The job is
simplified by the |defmacro-driver| macro.

\begin{clauses}

\defmacro{defmacro-driver}{arglist |\&body| body}
Defines a driver clause in
both the |for| and |generate| forms, and provides a parameter
|generate| which ~body~ can examine to determine how it was invoked.
~arglist~ is as in |defmacro-clause|, and should begin with the symbol
|for|. 

\end{clauses}

With |defmacro-driver|, our driver looks like this:
\begin{program}
(defmacro-driver (FOR var IN-WHOLE-VECTOR v)
  "All the elements of a vector (disregards fill-pointer)"
   (let ((vect (gensym))
         (end (gensym))
         (index (gensym))
         (kwd (if generate 'generate 'for)))
     `(progn
        (with ,vect = ,v)
        (with ,end = (array-dimension ,vect 0))
        (with ,index = -1)
        (,kwd ,var next (progn (incf ,index)
                               (if (>= ,index ,end) (terminate))
                               (aref ,vect ,index))))))
\end{program}


We are still missing one thing: the |\&sequence| keywords.  
We can get them easily enough, by writing 
\begin{program}
(defmacro-driver (FOR var IN-WHOLE-VECTOR v \&sequence) 
  ...)
\end{program}
We can now refer to parameters |from|, |to|, |by|, etc. which contain
either the values for the corresponding keyword, or \nil\ if the
keyword was not supplied.  Implementing the right code for these
keywords is cumbersome but not difficult; it is left as an exercise.
But before you begin, see |defclause-sequence| below for an easier way.
       
\subsection{Extensibility Aids}
\label{aids}

This section documents assorted features that may be of use in
extending \iter.

\begin{clauses}

\defunexpvar{*result-var*}
Holds the variable that is used to return a value as a result of the
\iter\ form.  You may examine this and use it in a |with| clause, but
you should not change it.

\defmacro{defsynonym}{syn word}
 Makes ~syn~ a synonym for the existing \iter\ keyword ~word.~  Only
the first word in each clause can have synonyms. 

\defmacroxx{defclause-sequence}{element-name index-name}{|\&key|
access-fn size-fn sequence-type}{element-type
element-doc-string index-doc-string}  
Provides
a simple way to define sequence clauses.  Generates two
clauses, one for iterating over the sequence's elements, the other
for iterating over its indices.  The first symbol of both
clauses will have print-name |for|.
~element-name~  and ~index-name~ should be symbols.
~element-name~ is the second keyword of the element iterator (typically of
the form 
|in-~sequence-type~|), and ~index-name~ is the second keyword
of the index-iterator (typically of the form
|index-of-~sequence-type~|).  Either name may be 
\nil, in which case the corresponding clause is not defined.  If both
symbols are supplied, they should be in the same package.  The |for|
that begins the clauses will be in this package.

\cpar ~access-fn~ is the function to be used to
access elements of the sequence in the element iterator.  The function
should take two 
arguments, a sequence and an index, and return the appropriate element.
~size-fn~ should denote a function of one argument, a sequence, that
returns its size.  Both ~access-fn~ and ~size-fn~ are required for the
element iterator, but only ~size-fn~ is needed for the index iterator.

\cpar The ~sequence-type~ and ~element-type~ keywords are used to
suggest types for the variables 
used to hold the sequence and the
sequence elements, respectively.  The usual rules about \iter's
treatment of variable type declarations apply (see section \ref{types}).

\cpar ~element-doc-string~ and ~index-doc-string~ are
the documentation strings, for use with |display-iterate-clauses|. 

\cpar The generated element-iterator performs destructuring on the
element variable.

\cpar As an example, the above |for\dots in-whole-vector| example
could have been written:
\begin{program}
(defclause-sequence IN-WHOLE-VECTOR INDEX-OF-WHOLE-VECTOR
  :access-fn 'aref
  :size-fn \#'(lambda (v) (array-dimension v 0))
  :sequence-type 'vector
  :element-type t
  :element-doc-string 
     "Elements of a vector, disregarding fill-pointer"
  :index-doc-string 
     "Indices of vector, disregarding fill-pointer")
\end{program}

\end{clauses}

\subsection{Subtleties}

There are some subtleties to be aware of when writing \iter\ clauses.
First, the code returned by your macros may be |nconc|'ed into a list,
so you should always returned freshly consed lists, rather than
constants.  Second, \iter\ matches clauses by using |eq| on the first
symbol and |string=| on the subsequent ones, so the package of the
first symbol of a clause is relevant.  All of the clauses in this manual
have their first word in the \iter\ package.  
You can use the package system in the usual way to shadow
\iter\ clauses without replacing them.

%%% say more here, about the badness that only the first word of a
%%% clause is packagey.

\section{Non-portable Extensions to Iterate (Contribs)}
\label{contribs}

Currently, there is only one non-portable extension to iterate in the
distribution: iterate-pg. If you have made an extension that depends
on non-portable features, feel free to send them to |asf@boinkor.net|
for inclusion in the iterate distribution.

\subsection{An SQL query driver for iterate}

The pg package by Eric Marsden (see |http://cliki.net/pg|) provides an
interface to the PostgreSQL database. Using the \iterpg\ extension, it
is possible to handle the results of SQL queries using \iter.

This usage example should give you an idea of how to use it:

\begin{program}
(pg:with-pg-connection (c "somedb" "someuser")
  (iter (for (impl version date) in-relation "select * from version"
                                 on-connection *dbconn*)
        (collect version)))
\end{program}

To use the extension via |ASDF|, simply make your system depend on the
|iterate-pg| system instead of the |iterate| system. To load it
manually, use:

\begin{program}
  (asdf:oos 'asdf:load-op :iterate-pg)
\end{program}

%%%%%%%%%%%%%%%%%%%%%%%%%%%%%%%%%%%%%%%%%%%%%%%%%%%%%%%%%%%%%%%%
\section{Obtaining {\tt Iterate}}

\begin{sloppypar}
The information in this chapter is obsolete but included for
completeness's sake; Currently, the most up-to-date information on
\iter\ can be found at |http://boinkor.net/iterate.html|.
\end{sloppypar}

\begin{sloppypar}
\iter\ currently runs on Lisp Machines, and on
HP's, Sun3's and Sparcstations under Lucid.  
\iter\ source and binaries are available at the MIT AI Lab in the
subdirectories of |/src/local/lisplib/|.  The source file,
|iterate.lisp|, is also available for anonymous FTP in the directory
|/com/fpt/pub/| on the machine |TRIX.AI.MIT.EDU| (Internet number
128.52.32.6).  If you are unable to obtain |iterate| in one of these
ways, send mail to |jba@ai.mit.edu| and I will send you the source
file. 
\end{sloppypar}

\begin{sloppypar}
\iter\ resides in the |iterate| package (nickname |iter|).  Just say
\linebreak |(use-package :iterate)| to make all the necessary symbols
available. 
If a symbol is not exported, it appears in this manual with an
``|iterate::|'' prefix.
\end{sloppypar}

Send bug reports to |bug-iterate@ai.mit.edu|.  The |info-iterate|
mailing list will have notices of changes and problems; to have
yourself added, send mail to |info-iterate-request@ai.mit.edu|.


\medskip

\begin{flushleft}
 \bf Acknowledgements
\end{flushleft}
\smallskip

Richard Waters provided invaluable criticism which spurred me to improve
\iter\ greatly.  As early users, David Clemens, Oren Etzioni and Jeff
Siskind helped ferret out many bugs.

%\begin{theindex}
% The index files must be generated with the genindex program
% or from makeindex iterate-manual.idx which creates iterate-manual.ind.

%\baselineskip\normalbaselineskip
%\advance\baselineskip by -2pt

\input{iterate-manual.ind}

%\underline{Clauses}
%\input{iterate-manual.clindex}
%
%\indexspace
%\underline{Lisp}
%\input{iterate-manual.lispindex}

%\end{theindex}


\end{document}


% arch-tag: "c0f871a6-313c-11d8-abb9-000c76244c24"


%\end{theindex}


\end{document}


% arch-tag: "c0f871a6-313c-11d8-abb9-000c76244c24"


%\underline{Clauses}
%% Fix ``.'' spacing in lists
\documentstyle[12pt,aimemo]{article}

\setlength{\textwidth}{6.0in}
\setlength{\oddsidemargin}{0.25in}
\setlength{\evensidemargin}{0.25in}
\setlength{\topmargin}{0pt}
\setlength{\headsep}{24pt}
\setlength{\textheight}{8.5in}

\let\origpar=\par

% somehow, LaTex fucks with things so that the plain Tex \obeyspaces
% doesn't work.  This is the same, except we use an mbox containing
% a space instead of just a space.

\newcommand{\spc}{\mbox{ }}
\newcommand{\tAb}{\mbox{        }}
\def\obspaces{\catcode`\ =\active\catcode`\^^I=\active}
{\obspaces \global\let =\spc\global\let^^I=\tAb}

\newenvironment{program}{\medskip\samepage\tt\obeylines\obspaces}{\medskip}

% use | for \tt
\catcode`\|=13
\newif\ifvbar
\def|{\ifvbar
         \endgroup\vbarfalse
      \else 
         \begingroup\tt\vbartrue 
      \fi}



\newcommand{\lisp}{\tt}
\newcommand{\iter}{|iterate|}
\newcommand{\iterpg}{|iterate-pg|}
\newcommand{\setf}{|setf|}
\newcommand{\nil}{|nil|}
\newcommand{\nonnil}{non-\nil}
\newcommand{\opt}{{\lisp \&optional}}
\newcommand{\yields}{$\Rightarrow$}



\newbox\Dots
\setbox\Dots=\hbox{\small ...}
\renewcommand{\dots}{\copy\Dots}

% Here's a way to define an environment which will treat all
% paragraphs inside it in the same manner.  It's, as near as I can
% figure, the way that Latex does things.  Using \everypar won't work;
% apparently Latex resets that a lot.  What you have to do is: at the
% top of the file do \let\origpar=\par.  In the
% begin part of the environment, do the parshape (or whatever), then say
% \def\par{{\origpar}}.  The extra braces are necessary.   That's all
% you have to do; at the end of the environment things will return to
% normal automatically.  Why does this all work?  I have no idea.
% Another important thing: there should be a blank line (i.e. a \par)
% between the end of the last paragraph and the \end{environment}.

\newlength{\clindent}
\setlength{\clindent}{0.5in}
\newlength{\clparindent}
\setlength{\clparindent}{\parindent}

\newenvironment{clauses}{\advance\linewidth -\parindent
                        \hangindent\clindent\relax
                         \hangafter=0
                          \parindent=0pt
                          \def\par{{\origpar}}}{}

\newcommand{\cpar}{\hspace \clparindent}

\newcommand{\startitem}{\bigskip\pagebreak[3]}
\newcommand{\finishitem}{\smallskip}

\newcommand{\clause}[1]{\startitem\Clause{#1}\finishitem}
\newcommand{\Clause}[1]{\hbox{#1}}
\newcommand{\clindex}[1]{\index{{\lisp #1}}}
\newcommand{\clindexx}[2]{\index{{\lisp #1\dots #2}}}
\newcommand{\clindexxx}[3]{\index{{\lisp #1\dots #2\dots #3}}}
\newcommand{\clausex}[2]{\startitem\Clausex{#1}{#2}\finishitem}
\newcommand{\Clausex}[2]{\Clause{|#1| #2}\clindex{#1}}

\newcommand{\defvar}[1]{\startitem\deff{#1}{}{Variable}\finishitem}
\newcommand{\defunexpvar}[1]{\startitem\deffnoindex{iterate::#1}{}{Variable}
                                        \lispindex{#1}\finishitem}
\newcommand{\defun}[2]{\startitem\deff{#1}{#2}{Function}\finishitem}
\newcommand{\defmacro}[2]{\startitem\deff{#1}{#2}{Macro}\finishitem}
\newcommand{\defunx}[3]{\startitem\deffx{#1}{#2}{#3}{Function}\finishitem}
\newcommand{\defmacrox}[3]{\startitem\deffx{#1}{#2}{#3}{Macro}\finishitem}
\newcommand{\defunxx}[4]{\startitem\deffxx{#1}{#2}{#3}{#4}{Function}\finishitem}
\newcommand{\defmacroxx}[4]{\startitem\deffxx{#1}{#2}{#3}{#4}{Macro}\finishitem}

\newcommand{\deff}[3]{\deffnoindex{#1}{#2}{#3}\lispindex{#1}}
\newcommand{\deffnoindex}[3]{\hbox to \hsize{{\tt #1} {\it #2}\hfill [{\it #3}]}}
\newcommand{\deffx}[4]{\deff{#1}{#2}{#4}\moreargs{#1}{#3}}
\newcommand{\deffxx}[5]{\deffx{#1}{#2}{#3}{#5}\moreargs{#1}{#4}}

\newcommand{\moreargs}[2]{\hbox to \hsize{\phantom{\lisp #1} {\it #2}\hfill}}

\newcommand{\lispindex}[1]{\index{{\lisp #1}}}

 \newenvironment{note}[1]{\pagebreak[2]\bigskip
                          \hrule\smallskip\small
                         {\setlength{\parindent}{0pt}
                           \par{\bf #1:}}}{\smallskip\hrule\bigskip}



\makeindex

%%%%%%%%%%%%%%%%%%%%%%%%%%%%%%%%%%%%%%%%%%%%%%%%%%%%%%%%%%%%%%%%

%\thispagestyle{empty}

%\vspace*{2in}

%\title{The |Iterate| Manual}
%\author{Jonathan Amsterdam}

%\maketitle

%\thispagestyle{empty}

%\begin{center}
%\Large *** DRAFT ***
%\end{center}
%This manual is in beta-test. It will become an AI Memo in a couple of
%months.  I would appreciate any comments or
%corrections.  You can reach me at email address |jba@ai.ai.mit.edu|.

\begin{document}

\def\writtenby{Jonathan Amsterdam}
\def\contractno{This report describes research done at the Artificial
Intelligence Laboratory of the Massachusetts Institute of Technology.
Support for the laboratory's artificial intelligence research is
provided in part by the Advanced Research Projects Agency of the
Department of Defense under Office
of Naval Research contract N00014-85-K-0124.}

\AIMEMO{1236}{\Large The Iterate Manual \\[1in]}
{This is the manual for version 1.4 of \iter, a powerful iteration
macro for Common  
Lisp. \iter\ is similar to {\tt loop} but provides numerous additional
features, is well integrated with Lisp, and is extensible. 
\vfill }


% use ~ for \em
\catcode`\~=13
\newif\ifem
\def~{\ifem
         \/\endgroup\emfalse
      \else 
         \begingroup\em\emtrue 
      \fi}



%\pagebreak
%\setcounter{page}{1}

%\thispagestyle{empty}


\tableofcontents

\pagebreak
\pagestyle{headings}

\section{Introduction}

\begin{sloppypar}
This manual describes
\iter, a powerful iteration facility for Common Lisp.
\iter\ provides abstractions for many common iteration
patterns and allows for the definition of additional patterns.  \iter\ is
a macro that expands into ordinary Lisp at
compile-time, so it is more efficient than higher-order
functions like |map| and |reduce.|  
While it is similar to |loop|, \iter\ offers a more Lisp-like syntax
and enhanced extensibility.
(For a more complete
comparison of \iter\ with other iteration constructs, see MIT AI Lab Working
Paper No. 324, ~Don't Loop, Iterate.~)
\end{sloppypar}

An \iter\ form consists of the symbol |iter|\footnote{You can also use
|iterate|, but |iter| is preferred because it avoids potential conflicts with
possible future additions to Common Lisp, and because it saves
horizontal space when writing code.}
followed by one or more
forms, some of which may be \iter\ 
~clauses.~  Here is a simple example of \iter\ which collects the
numbers from 1 to 10 into a list, and returns the list.  The return
value is shown following the arrow.
\begin{program}
(iter (for i from 1 to 10)
      (collect i))          \yields (1 2 3 4 5 6 7 8 9 10)
\end{program}
This form contains two clauses: a |for| clause that steps the
variable |i| over the integers from 1 to 10, and a |collect|
clause that accumulates its argument into a list.  With a few
exceptions, all \iter\ clauses have the same format: alternating
symbols (called ~keywords~) and expressions (called
~arguments~).  The syntax and terminology are those of Common
Lisp's keyword lambda lists.  One difference is that \iter's keywords
do not have to begin with a colon---though they may, except for the
first symbol of a clause.  So you can also write  |(for i :from
1 :to 10)| if you prefer.

Any Lisp form can appear in the body of an \iter, where it will have
its usual meaning.  \iter\ walks the entire body, expanding macros,
and recognizing clauses at any level.  This example collects all the
odd numbers in a list:
\begin{program}
(iter (for el in list)
      (if (and (numberp el) (oddp el))
          (collect el)))
\end{program}

There are clauses for iterating over numbers, lists, arrays and other
objects, and for
collecting, summing, counting, maximizing and other useful operations.
\iter\ also supports the creation of new variable bindings, stepping
over multiple sequences at once, destructuring, and compiler
declarations of variable types.  The following example illustrates
some of these features: 
\begin{program}
(iter (for (key . item) in alist)
         (for i from 0)
         (declare (fixnum i))
         (collect (cons i key)))
\end{program}

This loop takes the keys of an alist and returns a new alist
associating the keys with their positions in the original list. The
compiler declaration for |i| will appear in the generated code
in the appropriate place.

%%%%%%%%%%%%%%%%%%%%%%%%%%%%%%%%%%%%%%%%%%%%%%%%%%%%%%%%%%%%%%%%
\section{Clauses}

Most of \iter's clauses will be familiar to |loop| programmers.  
(|loop| is an iteration macro that has been incorporated into Common
Lisp.  Seen Guy Steele's {\em Common Lisp, 2nd Edition}.)
In
nearly all cases they behave the same as their |loop| counterparts, so
a |loop| user can switch to \iter\ with little pain (and much gain).

All clauses with the standard keyword-argument syntax consist of two
parts: a ~required~ part, 
containing keywords that must be present and in the right order; and
an ~optional~ part, containing keywords that may be omitted and,
if present, may occur in any order.  In the descriptions below, the
parts are separated by the Lisp lambda-list keyword \opt.


\subsection{Drivers}

An iteration-driving clause
conceptually causes the iteration to go forward.  Driver clauses in
\iter\ allow iteration over numbers, lists, vectors, hashtables, packages,
files and streams.  Iteration-driving clauses must
appear at the top level of an \iter\ form; they cannot be nested
inside another clause.  The driver variable is updated at the point
where the driver clause occurs.  Before the clause is executed for the
first time, the value of the variable is undefined.

%Also, regardless of where the driver clause
%appears in the body, the driver variable is stepped at the top of the
%loop; hence it is stylistically preferable, though
%not required, to place driver clauses at the beginning of the \iter.

Multiple drivers may appear in a single \iter\ form, in which case all
of the driver variables are updated each time through the loop, in the
order in which the clauses appear.  The first driver to terminate will
terminate the entire loop.

In all cases, the value of the driver variable on exit from the loop,
including within the epilogue code (see the |finally| clause), is
undefined.

All the parameters of a driver clause are evaluated once, before the
loop begins.  Hence it is not possible to change the bounds or other
properties of an iteration by side-effect from within the loop.

With one exception, driver clauses begin with the word |for| (or
the synonym |as|) and mention an iteration variable, which is
given a binding within the \iter\ form.
The exception is |repeat|, which just executes a loop a
specified number of times:

\begin{clauses}

\clausex{repeat}{~n~}
Repeats the loop ~n~ times.  For example:
\begin{program}
(iter (repeat 100)
      (print "I will not talk in class."))
\end{program}
\cpar If $n \leq 0$, the
loop will never be executed.  If ~n~ is not an integer, the
actual number of executions will be $\lceil n \rceil$.

\end{clauses}

\pagebreak[3]

\subsubsection{Numerical Iteration}

\begin{clauses}
\clausex{for}{~var~ |\&sequence|} 
The general form for iterating over a sequence of numbers requires a
variable and, optionally, one or more keywords that provide the bounds
and step size of the iteration. The |\&sequence|
lambda-list keyword is a shorthand for these sequence keywords.  They are:
|from|, |upfrom|, |downfrom|, |to|, |downto|, |above|, |below| and |by|.
|from| provides the starting value for ~var~
and defaults to zero. |to| provides a final value and implies that the
successive values of ~var~ will be increasing; |downto| implies that
they will be decreasing.  The loop terminates when ~var~ passes
the final value (i.e. becomes smaller or larger than it, depending on
the direction of iteration); in other  words, the loop body
will never be 
executed for values of ~var~ past the final value.
|below| and |above| are similar to |to| and
|downto|, except that the loop terminates when ~var~ equals or passes
the final value.

\cpar If no final value is specified, the variable will be stepped
forever.  Using |from| or |upfrom| will result in increasing
values, while |downfrom| will give decreasing values.

\cpar On each iteration,
~var~ is incremented or decremented by the value of the sequence
keyword |by|, which defaults to 1.  It should always be a positive
number, even for downward iterations.

\cpar In the following examples, the sequence of numbers generated is shown
next to the clause.

\begin{program}
(for i upfrom 0) \yields\ 0 1 2 \ldots
(for i from 5)   \yields\ 5 6 7 \ldots    ; either from or upfrom is okay
(for i downfrom 0) \yields\ 0 -1 -2 \ldots
(for i from 1 to 3) \yields\ 1 2 3
(for i from 1 below 3) \yields\ 1 2
(for i from 1 to 3 by 2) \yields\ 1 3
(for i from 1 below 3 by 2) \yields\ 1
(for i from 5 downto 3) \yields\ 5 4 3
\end{program}

\end{clauses}

\subsubsection{Sequence Iteration}

There are a number of clauses for iterating over sequences.  In all of
them, the argument following |for| may be a list instead of a
symbol, in which case destructuring is performed.  See section
\ref{destructuring}. 

\begin{clauses}

\clause{|for| ~var~ |in| ~list~ \opt\ |by| ~step-function~}
\clindexx{for}{in}
~var~ is set to successive elements of list.  
~step-function,~ which 
defaults to |cdr|, is used to obtain the next sublist. 

\clause{|for| ~var~ |on| ~list~ \opt\ |by| ~step-function~}
\clindexx{for}{on}
~var~ is set to successive sublists of list. 
~step-function~ (default |cdr|) is used as in |for\dots in|.

\end{clauses}

\medskip

These two clauses use |atom| to test for the end
of a list.  Hence, given a list whose final |cdr| is not \nil,
they will silently ignore the last |cdr|.  Other choices are
|endp|,  which would signal an error, and |null|, which
would probably result in an error somewhere else.  If you wish to use
an end-test other than |atom|, set the variable
|iterate::*list-end-test*|\lispindex{*list-end-test*} to the name of
the desired function. 

\begin{clauses}
\clause{|for| ~var~ |in-vector| ~vector~ |\&sequence|}
\clindexx{for}{in-vector}
~var~ takes on successive elements from ~vector.~   The vector's
fill-pointer is observed.  Here and in subsequent clauses, the |\&sequence|
keywords include |with-index|, which takes a symbol as argument
and uses it for the index variable instead of an internally generated
symbol.  The other |\&sequence| keywords behave as in numerical
iteration, except that the default iteration bounds are the bounds of
the vector.  E.g. in |(for i in-vector v downto 3)|,
|i| will start off being bound to the last element in |v|,
and will  
be set to preceding elements down to and including the element with
index 3.

\clause{|for| ~var~ |in-sequence| ~seq~ |\&sequence|}
\clindexx{for}{in-sequence}
This uses Common Lisp's generalized sequence functions, |elt|
and |length|, to obtain elements and determine the length of
~seq.~   Hence it will work for any sequence, including lists, and will
observe the fill-pointers of vectors.

\clause{|for| ~var~ |in-string| ~string~ |\&sequence|}
\clindexx{for}{in-string}
~var~ is set to successive characters of ~string.~

\startitem
\Clause{|for| ~var~ |index-of-vector| ~vector~ |\&sequence|}
\clindexx{for}{index-of-vector}
\Clause{|for| ~var~ |index-of-sequence| ~sequence~ |\&sequence|}
\clindexx{for}{index-of-sequence}
\Clause{|for| ~var~ |index-of-string| ~string~ |\&sequence|}
\clindexx{for}{index-of-string}
\finishitem
~var~ is set to successive indices of the sequence.
These clauses avoid the overhead of accessing the sequence elements
for those applications where they do not need to be examined, or are
examined rarely.  They admit all the optional keywords of the other
sequence drivers except the (redundant) |with-index| keyword.

\clause{|for| ~(key value)~ |in-hashtable| ~table~}
\clindexx{for}{in-hashtable}

~key~ and ~value,~ which must appear as shown in a list
and may be destructuring templates, are set to the keys and values of
~table~.  If ~key~ is |nil|, then the hashtable's keys will be ignored;
similarly for ~value~.
The order in which 
elements of ~table~ will be retrieved is unpredictable.

\clause{|for| ~var~ |in-package| ~package~ \opt\ |external-only| ~ext~}
\clindexx{for}{in-package}
Iterates over all the symbols in ~package,~ or over only the
external symbols if ~ext~ is specified and non-|nil|.  ~ext~ is not
evaluated.  The same symbol may appear more than once.

\clause{|for| ~var~ |in-packages| ~packages~ \opt\ |having-access| ~symbol-types~}
\clindexx{for}{in-packages}
Iterates over all the symbols from the list of packages denoted by the
descriptor ~packages~ and having accessibility (or visibility) given by
~symbol-types~.  This defaults to the list |(:external :internal :inherited)|
and is not evaluated.  ~var~ must be a list of up to three variables: in each
iteration, these will be set to a symbol, its access-type and package (as per
|with-package-iterator| in ANSI-CL).  The same symbol may appear more than
once.

\end{clauses}

\begin{clauses}

\clause{|for| ~var~ |in-file| ~name~ \opt\ |using| ~reader~}
\clindexx{for}{in-file}
Opens the file ~name~ (which may be a string or pathname) for input,
and iterates 
over its contents. ~reader~ defaults to |read|, so by default ~var~
will be bound to the successive forms in the file.  The \iter\ body is
wrapped in an unwind-protect to ensure that the file is closed no
matter how the \iter\ is exited.

\clause{|for| ~var~ |in-stream| ~stream~ \opt\ |using| ~reader~}
\clindexx{for}{in-stream}
Like |for\dots in-file|, except that ~stream~ should be an existing
stream object that supports input operations.

\end{clauses}

\subsubsection{Generalized Drivers}

These are primarily useful for writing drivers that can also be used
as generators (see section \ref{generators}, below).
 
\begin{clauses}

\clause{|for| ~var~ |next| ~expr~}
\clindexx{for}{next}

~var~ is set to ~expr~ each time through the loop.  Destructuring is
performed.  When the clause is used as a generator, ~expr~ is the code
that is executed when |(next ~var~)| is 
encountered (see section \ref{generators}, below).
~expr~ should compute the first value for ~var~, as well
as all subsequent values, and is responsible for terminating the loop.
For compatibility with future versions of \iter, this termination
should be done with |terminate|\clindex{terminate}, which can be
considered a synonym for |finish| (see section \ref{control-flow}).

\cpar As an example, the following clauses are equivalent to |(for
i from 1 to 10)|:
\begin{program}
(initially (setq i 0))
(for i next (if (> i 10) (terminate) (incf i)))
\end{program}

\clause{|for| ~var~ |do-next| ~form~}
\clindexx{for}{do-next}
~form~ is evaluated each time through the loop.  Its value is ~not~
set to ~var~; that is ~form's~ job. ~var~ is only present so that
\iter\ knows it is a driver variable.  \linebreak |(for ~var~ next
~expr~)| is equivalent to |(for ~var~ do-next (dsetq ~var~ ~expr~))|.
(See section \ref{destructuring} for an explanation of |dsetq|.) 

\end{clauses}

\subsubsection{Generators}
\label{generators}

In all of the above clauses, the driver variable is updated on each
iteration.
Sometimes it is desirable to have greater control over updating.  
For instance, consider the problem of associating numbers, in
increasing order and with no gaps, with the
\nonnil\ elements of a list.  One obvious first pass at writing this is:

\begin{program}
(iter (for el in list)
      (for i upfrom 1)
      (if el (collect (cons el i))))
\end{program}
But on the list |(a b nil c)| this produces |((a . 1) (b . 2) (c .
4))| instead of the desired |((a . 1) (b . 2) (c . 3))|.  The problem
is that |i| is incremented each time through the loop, even when |el|
is |nil|.  

The problem could be solved elegantly if we could step |i| only when
we wished to.  This
can be accomplished for any \iter\ driver by writing |generate| (or
its synonym |generating|)
instead of |for|.  Doing so produces a ~generator~---a driver whose
values are yielded explicitly.  To obtain the next value of a
generator variable ~v~, write \linebreak |(next ~v~)|.  The value of a |next|
form is the next value of ~v~, as determined by its associated driver
clause.  |next| also has the side-effect of updating ~v~ to that
value.  If there is no next value, |next| will terminate the loop,
just as with a normal driver.


Using generators, we can now write our example like this:
\begin{program}
(iter (for el in list)
      (generate i upfrom 1)
      (if el (collect (cons el (next i)))))
\end{program}
Now |i| is updated only when |(next i)| is executed, and this occurs
only when |el| is \nonnil.

To better understand the relationship between ordinary drivers and
generators, observe that we can rewrite an ordinary driver using its
generator form immediately followed by |next|, as this example shows:
\begin{program}
(iter (generating i from 1 to 10)
      (next i)
      ...)
\end{program}
Provided that the loop body contains no |(next i)| forms, this will
behave just as if we had written |(for i from 1 to 10)|.

We can still refer to a driver variable ~v~ without using |next|; in
this case, its value is that given to it by the last evaluation of
|(next ~v~)|.  Before |(next ~v~)| has been called the first time, the
value of ~v~ is undefined.  

This semantics is more flexible than
one in which ~v~ begins the loop bound to its first value and calls
of |next| supply subsequent values, because it means the loop will not
terminate too soon if the generator's sequence is empty.  For
instance, consider the following code, which tags \nonnil\ elements of
a list using a list of tags, and also counts the null elements.
(We assume there are at least as many tags as \nonnil\ elements.)
\begin{program}
(let* ((counter 0)
       (tagged-list (iter (for el in list)
                          (generating tag in tag-list) 
                          (if (null el)
                              (incf counter)
                              (collect (cons el (next tag)))))))
  ...)
\end{program}
It may be that there are just as many tags as non-null elements of
|list|.  If all the elements of |list| are null, we still want the
counting to proceed, even though |tag-list| is \nil.  If |tag| had to be
assigned its first value before the loop begins, we would have had to
terminate the loop before the first iteration, since when |tag-list|
is \nil, |tag| 
has no first value.  With the existing semantics, however, |(next
tag)| will never execute, so the iteration will cover all the elements of
|list|. 
        
When the ``variable'' of a driver clause is actually a destructuring
template containing several variables, all the variables are eligible
for use with |next|.  As before, |(next ~v~)| evaluates to ~v's~ next
value; but the effect is to update all of the template's variables.
For instance, the following code will return the list |(a 2 c)|.
\begin{program}
(iter (generating (key . item) in '((a . 1) (b . 2) (c . 3)))
      (collect (next key))
      (collect (next item)))
\end{program}

Only driver clauses with variables can be made into generators.  This
includes all clauses mentioned so far except for |repeat|.  It does
~not~ include |for\dots previous|, |for\dots =|,
|for\dots initially\dots then| or |for\dots first\dots then| (see
below). 

\subsubsection{Previous Values of Driver Variables}

Often one would like to access the value of a variable on a previous
iteration.  \iter\ provides a special clause for accomplishing this.

\begin{clauses}

\clause{|for| ~pvar~ |previous| ~var~ \opt\ |initially| ~init~
|back| ~n~}
\clindexx{for}{previous}
Sets ~pvar~ to the previous value of ~var,~ which should be a driver
variable, a variable from another |for\dots previous| clause, or a
variable established by a |for\dots =|, 
|for\dots initially\dots then| or |for\dots first\dots then| clause
(see section \ref{setting}).
Initially, ~pvar~ is given the value ~init~ (which defaults to \nil).  
The ~init~ expression will be moved outside the loop body, so it
should not depend on anything computed within the loop.
~pvar~ retains the value of ~init~ until ~var~ is set to its second
value, at which point ~pvar~ is set to ~var's~ first value; and so on.  

\cpar The
argument ~n~ to |back| must be a constant, positive integer, and
defaults to 1.  It determines how many iterations back ~pvar~ should
track ~var~.  For example, when ~n~ is 2, then ~pvar~ will be assigned
~var's~ first value when ~var~ is set to its third value.

\cpar A |for\dots previous| clause may occur after or before its
associated driver clause. |for\dots previous| works with generators as
well as ordinary drivers.

\pagebreak[3]

\cpar Example:
\begin{program}
(iter (for el in '(1 2 3 4))
      (for p-el previous el)
      (for pp-el previous p-el initially 0)
      (collect pp-el))
\end{program}
This evaluates to |(0 0 1 2)|.  It could have been written more
economically as 
\begin{program}
(iter (for el in '(1 2 3 4))
      (for pp-el previous el back 2 initially 0)
      (collect pp-el))
\end{program}


\end{clauses}

\subsection{Variable Binding and Setting}
\label{setting}

Several clauses exist for establishing new variable bindings or for
setting variables in the loop.  They all support destructuring.

\begin{clauses}

\clausex{with}{~var~ \opt\ |=| ~value~}
Causes ~var~ to be bound to value before the loop body is entered.
If ~value~ is not supplied, ~var~ assumes a default
binding, which will be 
\nil\ in the absence of declarations.  Also, if ~value~ is not
supplied, no destructuring is performed; instead, ~var~ may be a list of
symbols, all of which are given default 
bindings.  If ~value~ is supplied, ~var~ is bound to it, with
destructuring.  

\cpar Because |with|
creates bindings whose scope includes the entire \iter\ form, it is
good style to put all |with| clauses at the beginning.

\cpar Successive occurrences of |with| result in sequential
bindings (as with 
|let*|).  There is no way to obtain parallel bindings; see
section \ref{bindings} for a rationale.  


\clause{|for| ~var~ |=| ~expr~}
\clindexx{for}{=}
On each iteration, ~expr~ is evaluated and ~var~ is set
to its value. 

\cpar This clause may appear to do the same thing as |for\dots next|.
In fact, they are quite different.  |for\dots =| provides only three
services: it sets up a binding for ~var~, sets it to ~expr~ on each
iteration, and makes it possible to use |for\dots previous| with
~var~.  |for\dots next| provides these services in addition to the
ability to turn the driver into a generator.
%Also, the code which sets ~var~ appears in the loop body in the same
%place as the |for\dots =| clause; the code for |for\dots next| appears
%at the top of the loop, as with other drivers (except when being used
%as a generator).

\clause{|for| ~var~ |initially| ~init-expr~ |then| ~then-expr~}
\clindexxx{for}{initially}{then}
Before the loop begins, ~var~ is set to ~init-expr;~ on all
iterations after the first it is set to ~then-expr.~  This clause
must occur at top-level.  ~init-expr~ will be moved outside the loop
body and ~then-expr~ will be moved to the end of the loop body, so
they are subject to code motion problems (see section
\ref{code-movement}). 

\cpar This clause may appear to be similar to |for\dots next|, but in
fact they differ significantly.  |for\dots initially\dots then| is
typically used to give ~var~ its first value before the loop begins,
and subsequent values on following iterations.  This is incompatible
with generators, whose first value and subsequent values must all be
computed by |(next ~var~)|.  Also, the update of ~var~ in 
|for\dots initially\dots then| does not occur at the location of the
clause. 
Use |for\dots initially\dots then| for
one-shot computations where its idiom is more convenient, but use
|for\dots next| for extending \iter\ with new drivers (see section
\ref{extend}).

\clause{|for| ~var~ |first| ~first-expr~ |then| ~then-expr~}
\clindexxx{for}{first}{then}
The first time through the loop, ~var~ is set to ~first-expr;~ on
subsequent iterations, it is set to ~then-expr.~  This differs from
|for\dots initially| in that ~var~ is set to ~first-expr~
inside the loop body, 
so ~first-expr~ may depend on the results of other clauses.  For
instance, 
\begin{program}
(iter (for num in list)
      (for i first num then (1+ i))
      ...)
\end{program}
will set |i| to the first element of |list| on the first
iteration, whereas 
\begin{program}
(iter (for num in list)
      (for i initially num then (1+ i))
      ...)
\end{program}
is probably erroneous; |i| will be bound to |num|'s
default binding (usually \nil) for the first iteration.

\end{clauses}

\begin{note}{Compatibility note}
|loop|'s |for\dots =| works like \iter's, but |loop| used the syntax
|for\dots =\dots then| to mean |for\dots initially\dots then|.  It was
felt that these two operations were sufficiently different to warrant
different keywords.

Also, the |for| in the above three clauses is misleading,
since none is true driver (e.g. none has a corresponding |generate|
form).  |setting| would have been a better choice, but |for| was used
to retain some compatibility with |loop|.  
\end{note}


%%%%%%%%%%%%%%%%%%%%%%%%%%%%%%%%
\subsection{Gathering Clauses}
Many of \iter's clauses accumulate values into a variable, or set a
variable under certain conditions.  At the end of the loop, this
variable contains the desired result.  All these clauses have an
optional |into| keyword, whose argument should be a symbol.  If the
|into| keyword is not supplied, the accumulation variable will be
internally generated and its value will be returned at the end of the
loop; if a variable is specified, that variable is used for the
accumulation, and is not returned as a result---it is up to the user
to return it explicitly, in the loop's epilogue code (see |finally|).
It is safe to examine the accumulation variable during the loop, but
it should not be modified.

These clauses all begin with a verb.  When the verb does
not conflict with an existing Common Lisp function, then it may be
used in either its infinitival or present-participle form (e.g. |sum|,
|summing|).  However, when there is a conflict with Common Lisp, only
the present-participle form may be used (e.g. |unioning|).  This is to
prevent \iter\ clauses from clashing with Common Lisp
functions. 

%although these clauses are described as ``producing a value,'' it is a
%mistake to think of the lisp list representing the clause as a
%value-producing form in the usual way.  clauses may legally be written
%where a value is expected, e.g. |(setq x (sum i))|, but the
%lisp value of a clause in such a context is undefined.

\subsubsection{Reductions}

~Reduction~ is an extremely common iteration pattern in which
the results of successive applications of a
binary operation are accumulated.
For example, a loop that computes the sum of the
elements of a list is performing a reduction with the addition
operation.  This could be written in Common Lisp as  |(reduce \#'+
list)| or with \iter\ as
\begin{program}
(iter (for el in list)
      (sum el))
\end{program}

\begin{clauses}

\clausex{sum}{~expr~ \opt\ |into| ~var~}
Each time through the loop, ~expr~ is evaluated and added to a
variable, which is bound initially to zero.  If ~expr~ has a type,
it is ~not~ used as the type of the sum variable, which is always
|number|.  To get the result variable to be of a more specific
type, use an explicit variable, as in
\begin{program}
(iter (for el in number-list)
      (sum el into x)
      (declare (fixnum x))
      (finally (return x)))
\end{program}

\clausex{multiply}{~expr~ \opt\ |into| ~var~}
Like |sum|, but the initial value of the result variable is $1$,
and the variable is updated by multiplying ~expr~ into it.

\clausex{counting}{~expr~ \opt\ |into| ~var~}
~expr~ is evaluated on each iteration.  If it is \nonnil, the
accumulation variable, initially zero, is incremented.

\startitem
\Clausex{maximize}{~expr~ \opt\ |into| ~var~}
\Clausex{minimize}{~expr~ \opt\ |into| ~var~}
\finishitem
~expr~ is evaluated on each iteration and its extremum (maximum or
minimum) is stored in the accumulation variable.  If ~expr~ is never
evaluated, then the result is \nil\ (if the accumulation variable is
untyped) or $0$ (if it has a numeric type).

\clausex{reducing}{~expr~ |by| ~func~ \opt\ |initial-value| ~init-val~ |into| ~var~}
This is a general way to perform reductions. ~func~ should be a
function of two arguments, the first of which will be the value
computed so far and
the second of which will be the value of ~expr~.  It should return the
new value.
|reducing| is roughly equivalent to the
Common Lisp |(reduce ~func~ ~sequence~ :key ~expr-function~)|, where
~expr-function~ is used to derive values from the successive elements
of ~sequence~.

\cpar If the |reducing| clause is never executed, the result is
undefined.

\cpar It is not necessary to provide an
initial value, but better code can be generated if one is supplied.
Regardless of its location in the \iter\ body, ~init-val~ will be
evaluated before the loop is entered, so it should not depend on any
value computed inside the \iter\ form.  

%\cpar if a ~var~ is not specified, you can get \iter\ to declare the
%type of the internal variable by putting a |the| expression around
%~func.~  see section \ref{types}.

\end{clauses}

%\begin{note}{implementation note}
%in principle, |maximize| and |minimize| can be thought of
%as reductions where the initial value is the smallest (or largest)
%value that the accumulation variable can assume.  because lisp's
%bignums can represent arbitrary integers, these clauses cannot be
%implemented as reductions in general.  if, however, the type of 
%~expr~ or ~var~ can be determined to be a fixnum
%or a float, \iter\ will implement the clause as a true reduction,
%using one of the constants |most-negative-fixnum|, |%most-positive-fixnum|, 
%|most-negative-short-float|, etc. as appropriate.
%\end{note}

\subsubsection{Accumulations}

All the predefined accumulation clauses add values to a sequence.  If
the sequence is a list, they all 
have the property that the partial list is kept in the correct order
and available for inspection at any point in the loop.

\begin{clauses}

\clausex{collect}{~expr~ \opt\ |into| ~var~ |at| ~place~ |result-type| ~type~}
Produces a sequence of the values of ~expr~ on each iteration. ~place~
indicates where the next value of ~expr~ is added to the list and may
be one of the symbols |start|, |beginning| (a synonym for |start|) or
|end|.  The symbol may be quoted, but need not be.  The default is
|end|.  For example,
\begin{program}
(iter (for i from 1 to 5)
      (collect i))
\end{program}
produces |(1 2 3 4 5)|, whereas
\begin{program}
(iter (for i from 1 to 5)
      (collect i at beginning))
\end{program}
produces |(5 4 3 2 1)| (and is likely to be faster in most Common Lisp
implementations).

\cpar If ~type~ is provided, it should be a subtype of |sequence|.
The default is |list|.  Specifying a type other than |list| will
result in |collect| returning a sequence of that type.  ~However,~ the
type of the sequence being constructed when inside the loop body is
undefined when a non-|list| type is specified.  (As with ~place~,
quoting ~type~ is optional.)

\clausex{adjoining}{~expr~ \opt\ |into| ~var~ |test| ~test~ |at| ~place~ |result-type| ~type~}
Like |collect|, but only adds the value of ~expr~ if it is not
already present.  ~test,~ which defaults to |\#'eql|, is
the test to be used with |member|.

\startitem
\Clausex{appending}{~expr~ \opt\ |into| ~var~ |at| ~place~}
\Clausex{nconcing}{\ ~expr~ \opt\ |into| ~var~ |at| ~place~}
\Clausex{unioning}{\ ~expr~ \opt\ |into| ~var~ |test| ~test~ |at| ~place~}
\Clausex{nunioning}{~expr~ \opt\ |into| ~var~ |test| ~test~ |at| ~place~}
\finishitem
These are like |collect|, but behave like the Common Lisp functions
|append|, |nconc|, |union| or |nunion|.  
As in Common Lisp, they work only on lists.  Also as in Common Lisp,
|unioning| and |nunioning| assume that the value of ~expr~ contains no
duplicates. 

\clausex{accumulate}{~expr~ |by| ~func~ \opt\ |initial-value| ~init-val~ |into| ~var~}
This is a general-purpose accumulation clause. ~func~ should be a
function of two arguments, the value of ~expr~ and the value
accumulated so far in the iteration, and it should return the updated
value.  If no initial value is supplied, \nil\ is used.  

%\cpar If a ~var~ is not specified, you can get \iter\ to declare the
%type of the internal variable by putting a |the| expression around
%~func.~  see section \ref{types}.

\cpar The differences between |accumulate| and |reducing| are slight.
One difference is that the functions take their arguments in a
different order.  Another is that in the absence of ~init-val~,
|accumulate| will use \nil, whereas |reducing| will generate different
code that avoids any dependence on the initial value.
The reason for having both clauses is that one usually
thinks of reductions (like |sum|) and accumulations (like |collect|)
as different beasts.

\end{clauses}


\subsubsection{Finders}

A ~finder~ is a clause whose value is an expression that meets some
condition.

\begin{clauses}

\clause{|finding| ~expr~ |such-that| ~test~ \opt\ |into| ~var~ |on-failure| ~failure-value~}
\clindexx{finding}{such-that}
If ~test~ (which is an expression) ever evaluates to \nonnil, the loop
is terminated, the 
epilogue code is run and the value of ~expr~ is returned.  Otherwise,
\nil\ (or ~failure-value,~ if provided) is returned.  If ~var~ is
provided, it will have either the \nonnil\ value of ~expr~ or
~failure-value~ when the epilogue code is run.

\cpar As a special case, if the ~test~ expression is a sharp-quoted
function, then it is applied to ~expr~ instead of being simply
evaluated.  E.g. |(finding x such-that \#'evenp)| is equivalent to
|(finding x such-that (evenp x))|.  

%\cpar although ~test~ need have nothing to do with ~%expr~ as in
%|(finding j such-that (> i 3))|, it usually 
%will: |(finding (length el) such-that (oddp (length el)))|.  to
%avoid performing the |length| computation twice, you could write
%|(finding (length el) such-that \#'oddp)| or |(finding (length
%el) such-that 'oddp)|; for these cases, \iter\ generates code that
%executes ~expr~ only once.  the code for |\#'oddp|
%is slightly different from that for {\lisp 'oddp}; see the discussion
%under {\lisp for\dots in} and {\lisp for\dots on}.

\cpar |On-failure| is a misnomer. Because it is always evaluated, it behaves
more like the default third argument to the |gethash| function. As a result,
|on-failure (error "Not found")| makes no sense. Instead, the clauses |leave|
or |thereis| can be used in conjunction with |finally| as follows:
\begin{program}
(iter (for x in '(1 2 3))
      (if (evenp x) (leave x))
      (finally (error "not found")))
\end{program}

\cpar This clause may appear multiple times when all defaults are
identical. It can also be used together with either |always|/|never| or
|thereis| if their defaults match. More specifically, |on-failure nil| is
compatible with |thereis|, while |on-failure t| is compatible with |always|
and |never| clauses.
\begin{program}
(iter (for i in '(7 -4 2 -3))
      (if (plusp i)
	  (finding    i  such-that (evenp i))
          (finding (- i) such-that (oddp i))))
\end{program}

\startitem
\Clause{|finding| ~expr~ |maximizing| ~m-expr~ \opt\ |into| ~var~}
\clindexx{finding}{maximizing}
\Clause{|finding| ~expr~ |minimizing| ~m-expr~ \opt\ |into| ~var~}
\clindexx{finding}{minimizing}
\finishitem
Computes the extremum (maximum or minimum) value of ~m-expr~ over all
iterations, and returns the value of ~expr~ corresponding to the
extremum.  ~expr~ is evaluated inside the loop at the time the new
extremum is established.  If ~m-expr~ is never evaluated (due to, for
example, being embedded in a conditional clause), then the returned
value depends on the type, if any, of ~expr~ (or ~var,~ if
one is supplied).  If there is no type, the returned
value will be \nil; if the type is numeric, the returned value will be
zero.

\cpar For these two clauses, ~var~ may be a list of two
symbols; in that case, the first is used to record ~expr~ and
the second, ~m-expr.~  

\cpar As with |finding\dots such-that|, if ~m-expr~ is a sharp-quoted
function, then it is called on ~expr~ instead of being evaluated.

\end{clauses}

\subsubsection{Boolean Tests}

\begin{clauses}

\clausex{first-iteration-p}{}
Returns |t| in the first cycle of the loop, otherwise \nil.

\clausex{first-time-p}{}
Returns |t| the first time the expression is evaluated, and then \nil\ forever.
This clause comes handy when printing (optional) elements separated
by a comma:

\begin{program}
(iter (for el in '(nil 1 2 nil 3))
      (when el
        (unless (first-time-p)
          (princ ", "))
        (princ el)))
\end{program}
produces |"1, 2, 3"|.

\end{clauses}

\subsubsection{Aggregated Boolean Tests}

\begin{clauses}

\clausex{always}{~expr~}
If ~expr~ ever evaluates to
\nil, then \nil\ is immediately returned; the epilogue code is not
executed.  If ~expr~ never evaluates to \nil, the epilogue code
is executed and the last value of ~expr~ (or |t| if ~expr~ was never
evaluated) is returned (whereas |loop| would constantly return |t|).

% mention last evaluated clause when multiple always clauses?

\clausex{never}{~expr~}
Like |(always (not ~expr~))|, except it does not influence the last
value returned by a possible other |always| clause. That is,
\begin{program}
(iter (repeat 2)
      (always 2)
      (never nil)) \yields 2 ; not t
\end{program}

\clausex{thereis}{~expr~}
If ~expr~ is ever \nonnil,
its value is immediately returned without running epilogue code.
Otherwise, the epilogue code is performed and \nil\ is returned.

This clause cannot be used together with |always| or |never|, because their
defaults are opposed (similarly, |(loop always 3 thereis nil)| refuses to
compile in some implementations of |loop|).

\end{clauses}


\subsection{Control Flow}
\label{control-flow}
Several clauses can be used to alter the usual flow of control in a loop.

Note: the clauses of this and subsequent sections don't adhere to \iter's
usual syntax, but instead use standard Common Lisp syntax.  Hence the
format for describing syntax subsequently is
like the standard format used in the Common Lisp manual, not like the
descriptions of clauses above.

\begin{clauses}

\clausex{finish}{}
Stops the loop and runs the epilogue code.

%for example:
%
%\begin{program}
%(iter (with answer = nil)
%         (initially (make-a-mess))
%         (for i from 1 to 10)
%         (when (correct? i)
%           (setq answer i)
%           (finish))
%         (finally (cleanup)))
%\end{program}
%
%this code will execute |cleanup| whether or not the test |(correct?
%i)| ever succeeds. 
%the (more elegant) formulation,
%\begin{program}
%(iter (initially (make-a-mess))
%         (for i from 1 to 10)
%         (finding i such-that (correct? i))
%         (finally (cleanup)))
%\end{program}
%would not execute |cleanup| if |(correct? i)| succeeded; it
%would do an immediate return.

\clausex{leave}{\opt\ ~value~}
Immediately returns ~value~ (default \nil) from the current \iter\
form, skipping the epilogue code.  Equivalent to using |return-from|.

\clausex{next-iteration}{}
Skips the remainder of the loop body and begins the next iteration of
the loop.

\clausex{while}{~expr~}
If ~expr~ ever evaluates to \nil, the loop is terminated and the
epilogue code executed.  Equivalent to |(if (not ~expr~) (finish))|.

\clausex{until}{~expr~}
Equivalent to |(if ~expr~ (finish))|.

\clausex{if-first-time}{~then~ \opt\ ~else~}
If this clause is being executed for the first time in this invocation
of the \iter\ form, then the ~then~ code is evaluated; otherwise the
~else~ code is evaluated.

\cpar |(for ~var~ first ~expr1~ then ~expr2~)| is almost equivalent to
\begin{program}
(if-first-time (dsetq ~var~ ~expr1~)
               (dsetq ~var~ ~expr2~))
\end{program}
The only difference is that the |for| version makes ~var~ available
for use with |for\dots previous|.

\end{clauses}

\subsection{Code Placement}
When fine control is desired over where code appears in a loop
generated by \iter, the following special clauses may be useful.
They are all subject to code-motion problems (see section
\ref{code-movement}).  

\begin{clauses}

\clausex{initially}{|\&rest| ~forms~}
The lisp ~forms~ are placed in the prologue section of the loop, where
they are executed once, before the loop body is entered.

\clausex{after-each}{|\&rest| ~forms~}
The ~forms~ are placed at the end of the loop body, where they
are executed after each iteration.  Unlike the other clauses in this
section, ~forms~ may contain \iter\ clauses.

\clausex{else}{|\&rest| ~forms~}
The lisp ~forms~ are placed in the epilogue section of the loop, where they
are executed if this |else| clause is never met during execution of the
loop and the loop terminates normally.

\clausex{finally}{|\&rest| ~forms~}
The lisp ~forms~ are placed in the epilogue section of the loop, where
they are executed after the loop has terminated normally.

\clausex{finally-protected}{|\&rest| ~forms~}
The lisp ~forms~ are placed in the second form of an unwind-protect
outside the loop.  They are always executed after the loop has
terminated, regardless of how the termination occurred.

\end{clauses}


%%%%%%%%%%%%%%%%%%%%%%%%%%%%%%%%%%%%%%%%%%%%%%%%%%%%%%%%%%%%%%%%
\section{Other Features}

\subsection{Multiple Accumulations}
\label{multiple}

\begin{sloppypar}
It is permitted to have more than one clause accumulate into the same
variable, as in the following:
\begin{program}
(iter (for i from 1 to 10)
      (collect i into nums)
      (collect (sqrt i) into nums)
      (finally (return nums)))
\end{program}
Clauses can only accumulate into the same variable if they are
compatible.  |collect|, |adjoining|, |appending|, |nconcing|,
|unioning| and |nunioning| are compatible with each other; |sum|,
|multiply| and |counting| are compatible; |always| and |never| are compatible;
|finding| \dots |such-that| is compatible with either |thereis| or |always|
and |never| when their defaults match; and |maximize| and |minimize| clauses
are compatible only with other |maximize| and |minimize| clauses,
respectively.

%note that the same variable ~cannot~ be both an accumulation
%variable and an ordinary variable; there can be only one variable with
%a given name within an \iter\ form. 
\end{sloppypar}

\subsection{Named Blocks}

Like Common Lisp |block|s, \iter\ forms can be given names.  The
name should be a single symbol, and it must be the first form in the
\iter.  The generated code behaves exactly like a named block; in
particular, |(return-from ~name~)| can be used to exit it:
\begin{program}
(iter fred
      (for i from 1 to 10)
      (iter barney
            (for j from i to 10)
            (if (> (* i j) 17)
                (return-from fred j))))
\end{program}
An \iter\ form that is not given a name is implicitly named \nil.

Sometimes one would like to write an expression in an inner \iter\ form,
but have it processed by an outer \iter\ form.  This is possible with
the |in| clause.  

\begin{clauses}

\clausex{in}{~name~ |\&rest| ~forms~}
Evaluates ~forms~ as if they were part of the \iter\ form named
~name~.  In other words, \iter\ clauses are processed by the \iter\
form named ~name,~ and not by any \iter\ forms that occur inside ~name.~

\cpar As an example, consider the problem of collecting a list of the
elements in a two-dimensional array.  The naive solution,
\begin{program}
(iter (for i below (array-dimension ar 0))
      (iter (for j below (array-dimension ar 1))
            (collect (aref ar i j))))
\end{program}
\noindent is wrong because the list created by the inner \iter\ is simply
ignored by the outer one.  But using |in| we can write:
\begin{program}
(iter outer (for i below (array-dimension ar 0))
      (iter (for j below (array-dimension ar 1))
            (in outer (collect (aref ar i j)))))
\end{program}
\noindent which has the desired result.

\end{clauses}

\subsection{Destructuring}
\label{destructuring}

In many places within \iter\ clauses where a variable is expected, a
list can be written instead.  In these cases, the value to be assigned
is ~destructured~ according to the pattern
described by the list.  As a simple example, the clause
\begin{program}
(for (key . item) in alist)
\end{program}
\noindent will result in |key| being set to the |car| of
each element in |alist|, and |item| being set to the |cdr|.  The
pattern list may be nested to arbitrary depth, and (as the example
shows) need not be terminated with \nil; the only requirement is that
each leaf be a bindable symbol (or \nil, in which case no binding is
generated for that piece of the structure).

Sometimes, you might like to do the equivalent of a
|multiple-value-setq| in a clause.  This 
``multiple-value destructuring'' can be expressed by writing 
\linebreak |(values $pat_1$ $pat_2 \ldots$)| for a destructuring
pattern, as in 
\begin{program}
(for (values (a . b) c d) = (three-valued-function ...))
\end{program}
\begin{sloppypar}
Note that the $pat_i$ can themselves be destructuring patterns (though
not multiple-value destructuring patterns).  You can't do multiple-value
destructuring in a |with| clause; instead wrap the whole \iter\
form in a |multiple-value-bind|.
\end{sloppypar}

\begin{note}{Rationale}
There are subtle interactions between variable declarations and
evaluation order that make the correct implementation of
multiple-value destructuring in a |with| somewhat tricky.
\end{note}

The destructuring feature of \iter\ is available as a separate
mechanism, using the |dsetq| macro:

\begin{clauses}
\defmacro{dsetq}{template expr}
Performs destructuring of ~expr~ using ~template~.   May be used
outside of an \iter\ form. Yields the primary value of ~expr~.

\end{clauses}

\subsection{On-line Help}

\begin{sloppypar}
There is a limited facility for on-line help, in the form of the 
|display-iterate-clauses| function.
\end{sloppypar}

\begin{clauses}

\defun{display-iterate-clauses}{\opt\ clause-spec}
Displays a list of \iter\ clauses.  If ~clause-spec~ is not
provided, all clauses are shown; if it is a symbol, all clauses
beginning with that symbol are shown; and if it is a list of symbols,
all clauses for which ~clause-spec~ is a prefix are shown.

\end{clauses}

\subsection{Parallel Binding and Stepping}
\label{bindings}

The parallel binding and stepping of variables is a feature that
\iter\ does ~not~ have.  This section attempts to provide a rationale.

We say that two variables are bound ~in parallel~ if neither
binding shadows the other.  This is the usual semantics of |let|
(as opposed to |let*|).  Similarly, we can say that iteration
variables are stepped in parallel if neither variable is updated
before the other, conceptually speaking; in other words, if the code
to update each variable can reference the old values of both variables.

|loop| allows parallel binding of variables and parallel stepping of
driver variables.  My view is that if you are depending on the 
serial/parallel distinction, you are doing something obscure.  If you
need to bind 
variables in parallel using |with|, then you must be using a
variable name that shadows a name in the existing lexical environment.
Don't do that.  The most common use for parallel stepping is to track
the values of variables on the previous iteration, but in fact this
does not require parallel stepping at all; the following will work:
\begin{program}
(iter (for current in list)
      (for prev previous current)
      ...)
\end{program}


%%%%%%%%%%%%%%%%%%%%%%%%%%%%%%%%%%%%%%%%%%%%%%%%%%%%%%%%%%%%%%%%
\section{Types and Declarations}
\label{types}

\subsection{Discussion}

Sometimes efficiency dictates that the types of variables be declared.
This type information needs to be communicated to \iter\ so it can
bind variables to appropriate values.  Furthermore, \iter\ must often
generate internal variables invisible to the user; there needs to be a
way for these to be declared.

As an example, consider this code, which will return the number of
odd elements in |number-list|:
\begin{program}
(iter (for el in number-list)
      (count (oddp el)))
\end{program}
In processing this form,
\iter\ will create an internal variable, let us call it |list17|, to
hold the successive |cdr|s of |number-list|, and  
will bind the variable to |number-list|.  It will also generate a
default binding for |el|; only inside the body of the loop will |el|
be set to the |car| of |list17|.  Finally, \iter\ will generate a
variable, call it |result|, to hold the result of the count, and will
bind it to zero.

When dealing with type declarations, \iter\ observes one simple rule:
~it will never generate a declaration unless requested to do so.~  The
reason is that such declarations might mask errors in compiled code by
avoiding error-checks; the resulting problems would be doubly hard to
track down because the declarations would be hidden from the
programmer.  Of course, a compiler might omit error-checks even in the
absence of declarations, though this behavior can usually be avoided,
e.g. by saying |(declaim (optimize (safety 3)))|.

So, the above \iter\ form will generate code with no declarations.
But say we wish to declare the types of |el| and the internal
variables |list17| and |result|.  How is this done?

Declaring the type of |el| is easy, since the programmer knows
the variable's name:
\begin{program}
(iter (for el in number-list)
      (declare (fixnum el))
      (counting (oddp el)))
\end{program}
\iter\ can read variable type declarations like this one.  Before
processing any clauses, it scans the entire top-level form for type
declarations 
and records the types, so that variable bindings can be performed
correctly.  In this case, |el| will be bound to zero
instead of \nil.  Also, \iter\ collects all the top-level declarations
and puts them at the begining of the generated code, so it is not
necessary to place all declarations at the beginning of an \iter\
form; instead, they can be written near the variables whose types they
declare.  

Since \iter\ is not part of the compiler, it will not know
about declarations that occur outside an \iter\ form; these
declarations must be repeated inside the form.

Here is another way we could have declared the type of |el|:
\begin{program}
(iter (for (the fixnum el) in number-list)
      (counting (oddp el)))
\end{program}
\iter\ extends the Common Lisp |the|\lispindex{the}
form to apply to variables as well as value-producing forms; anywhere
a variable is allowed---in a |with| clause, as the iteration
variable in a driver clause, as the |into| argument of an
accumulation clause, even inside a destructuring template---you can
write |(the ~type~ ~symbol~)| instead.

There is one crucial difference between using a |the| form and
actually declaring the variable: explicit declarations are always
placed in the generated code, but type information from a |the|
form is not turned into an actual declaration unless you tell \iter\
to do so using |iterate:declare-variables|.  See below.

\begin{sloppypar}
Declaring the types of internal variables is harder than declaring the
types of explicitly mentioned variables, since their names
are unknown.  You do it by declaring |iterate:declare-variables|
somewhere inside the top level of the \iter\ form.  (This will also
generate declarations for variables declared using |the|.)
\iter\ does not provide much selectivity here: it's all or none. 
And unfortunately, since \iter\ is not privy to compiler information
but instead reads declarations itself, it will not hear if you
|(declaim (iterate:declare-variables))|.  Instead, set the variable 
|iterate::*always-declare-variables*| to |t| at
compile-time, using |eval-when|.
\end{sloppypar}

To determine the appropriate types for internal variables, \iter\ uses
three sources of information:

\begin{itemize}

\item Often, the particular clause dictates a certain type for a
variable; \iter\ will use this information when available.  In the
current example, the variable |list17| will be given the type
|list|, since that is the only type that makes sense; and the
variable |result| will be given the type |fixnum|, on the
assumption that you will not be counting high enough to need bignums.
You can override this assumption only by using and explicitly declaring a
variable: 
\begin{verbatim}
(iter (declare (iterate:declare-variables))
      (for el in number-list)
      (count (oddp el) into my-result)
      (declare (integer my-result))
      (finally (return my-result)))
\end{verbatim}

\begin{sloppypar}
Other examples of the type assumptions that \iter\ makes are: type
|list| for |into| variables of collection clauses; type |list| for
expressions that are to be destructured; type |vector| for the
variable holding the vector in a |for\dots in-vector| clause, and
similarly for |string| and the |for\dots in-string| clause;
and the implementation-dependent type
|(type-of array-dimension-limit)| for the index and limit
variables generated by sequence iteration drivers like |for\dots
in-vector| and |for\dots in-string| (but not |for\dots in-sequence|,
because it may be used to iterate over a list). 
\end{sloppypar}

\item Sometimes, \iter\ will examine expressions and try to determine
their types in a simple-minded way.  If the expression is
self-evaluating (like a number, for instance), \iter\ knows that the
expression's type is the same as the type of the value it denotes, so
it can use that type.  If the expression is of the form |(the ~type~
~expr~)|, \iter\ is smart enough to extract ~type~ and use it.
However, the current version of \iter\ does not  
examine declarations of function result types or do any type
inference.  It will not determine, for 
example, that the type of |(+ 3 4)| is |fixnum|, or even
|number|.

\item In some cases, the type of an internal variable should match the
type of some other variable.  For instance, \iter\ generates an
internal variable for |(f x)| in the
clause |(for i from 1 to (f x))|, and in the absence of other
information will give it the same type as |i|.  If, however, the
expression had been written |(the fixnum (f x))|, then \iter\
would have given the internal variable the type |fixnum|
regardless of |i|'s type.  The type incompatibility errors that
could arise in this situation are not checked for.

\end{itemize}

Note that if you do declare |iterate:declare-variables|, then
\iter\ may declare user variables as well as internal ones if they do
not already have declarations, though only for variables that it
binds.  For instance, in this code:

\begin{program}
(iter (declare (iterate:declare-variables))
      (for i from 1 to 10)
      (collect i into var))
\end{program}
the variable |var| will be declared to be of type |list|.


\subsection{Summary}

\iter\ understands standard Common Lisp variable type declarations
that occur within an \iter\ form and
will pass them through to the generated code.  If the declaration
|(iterate:declare-variables)|\lispindex{declare-variables}
appears at the top level of an 
\iter\ form, or if 
|iterate::*always-declare-variables*|\lispindex{*always-declare-variables*} 
is \nonnil, then \iter\ will use the type information gleaned from user
declarations, self-evaluating expressions and |the| expressions,
combined with reasonable assumptions, to determine variable
types and declare them.


%%%%%%%%%%%%%%%%%%%%%%%%%%%%%%%%%%%%%%%%%%%%%%%%%%%%%%%%%%%%%%%%
\section{Problems with Code Movement}
\label{code-movement}

Some \iter\ clauses, or parts of clauses, result in code being
moved from the location of the clause to other parts of the loop.
Drivers behave this way, as do code-placement clauses like |initially|
and |finally|.  When using these clauses, there is a danger of writing
an expression that makes sense in its apparent location but will be
invalid or have a different meaning in another location.  For example:
\begin{program}
(iter (for i from 1 to 10)
      (let ((x 3))
        (initially (setq x 4))))
\end{program}
While it may appear that the |x| of |(initially (setq x 4))| is the
same as the |x| of |(let ((x 3)) \dots|, in fact they are not:
|initially| moves its code outside the loop body, so |x| would refer
to a global variable.  Here is another example of the same problem:
\begin{program}
(iter (for i from 1 to 10)
      (let ((x 3))
        (collect i into x)))
\end{program}
If this code were executed, |collect| would create a binding for its
|x| at the top level of the \iter\ form that the |let| will shadow.

Happily, \iter\ is smart enough to catch these errors; it
walks all problematical code to ensure that free variables are not
bound inside the loop body, and checks all variables it binds for the
same problem.

However, some errors cannot be caught:

\begin{program}
(iter (with x = 3)
      (for el in list)
      (setq x 1)
      (reducing el by \#'+ initial-value x))
\end{program}
|reducing| moves its |initial-value| argument to the initialization
part of the loop in order to produce more efficient code.  Since
\iter\ does not perform data-flow analysis, it cannot determine that
|x| is changed inside the loop; all it can establish is that |x| is
not bound internally.  Hence this code will not signal an
error and will use $3$ as the initial value of the reduction.

The following list summarizes all cases that are subject to these code
motion and variable-shadowing problems.
\begin{itemize}
\item Any variable for which \iter\ creates a binding, including those
used in |with| and the |into| keyword of many clauses.

\begin{sloppypar}
\item The special clauses which place code: |initially|, |after-each|, |else|,
|finally| and |finally-protected|. 
\end{sloppypar}

\item The variables of a |next| or |do-next| form.

\item The |initially| arguments of |for\dots initially\dots then| and
|for\dots previous|. 

\item The |then| argument of |for\dots initially\dots then|.

\item The |initial-value| arguments of |reducing| and |accumulate|.

\item The |on-failure| argument of |finding\dots such-that|.
\end{itemize}

%%%%%%%%%%%%%%%%%%%%%%%%%%%%%%%%%%%%%%%%%%%%%%%%%%%%%%%%%%%%%%%%
\section{Differences Between {\tt Iterate} and {\tt Loop}}
|loop| contains a great deal of complexity which \iter\ tries to
avoid.  Hence many esoteric features of |loop| don't exist in \iter.
Other features have been carried over, but in a cleaned-up form.  
And of course, many new features have been added; they are not
mentioned in this list.

\begin{itemize}

\item \iter's syntax is more Lisp-like than |loop|'s, having a higher
density of parens.

\item The current implementation of \iter, unlike the current version
of  |loop| (as documented in {\em Common Lisp, 2nd Ed.\/}), is
extensible (see section \ref{extend}).  

\item |loop| puts the updates of all driver variables at the top of
the loop; \iter\ leaves them where the driver clauses appear.

\item While for the most part \iter\ clauses that resemble |loop| clauses
behave similarly, there are some differences.  For instance, there is
no |for\dots =\dots then| in \iter; instead use 
|for\dots initially\dots then|.  

\item |loop| binds the variable |it| at certain times
to allow pseudo-English expressions like |when ~expr~ return it|. 
In \iter, you must bind ~expr~ to a variable yourself.  Note that
|when ~expr~ return it| is like |thereis ~expr~| except that the latter is an
accumulation clause and therefore competes with other accumulations
(remember section \ref{multiple} above).

% repeat different behaviour of |always| clause here?

\item |loop| has a special |return| clause, illustrated in the
previous item.  \iter\ doesn't need one,  since an ordinary Lisp
|return| has the same effect.

\item |loop| allows for parallel binding and stepping of iteration
variables.  \iter\ does not.  (See section \ref{bindings}.)

\item |loop| and \iter\ handle variable type declarations very
differently.  |loop| provides a special syntax for declaring variable
types, and does not examine declarations.  Moreover, the standard
implementation of |loop| will
generate declarations when none are requested.
\iter\ parses standard Common Lisp type declarations, and will never
declare a variable itself unless declarations are 
specifically requested.

\end{itemize}

%%%%%%%%%%%%%%%%%%%%%%%%%%%%%%%%%%%%%%%%%%%%%%%%%%%%%%%%%%%%%%%%
\section{Rolling Your Own}
\label{extend}

\subsection{Introduction}

\iter\ is extensible---you can write new clauses that embody new
iteration patterns.  You might want to write a new driver clause for a
data structure of your own, or you might want to write a clause that
collects or manipulates elements in a way not provided by \iter. 

This section describes how to write clauses for \iter.  Writing a
clause is like writing a macro.  In fact, writing a clause ~is~
writing a macro: since \iter\ code-walks its body and macroexpands,
you can add new abstractions to \iter\ with good old |defmacro|.

Actually, there are two extensions you can make to \iter\ that are
even easier than writing a macro.  They are adding a synonym for an
existing clause and defining a driver clause for an indexable
sequence.  These can be done with |defsynonym| and
|defclause-sequence|, respectively.  See section \ref{aids}, below.

The rest of this section explains how to write macros that expand into
\iter\ clauses.
Here's how you could add a simplified version of \iter's |multiply|
clause, if \iter\ didn't already have one:
\begin{program}
(defmacro multiply (expr)
  `(reducing ,expr by \#'* initial-value 1))
\end{program}

If you found yourself summing the square of an expression often, you
might want to write a macro for that.  A first cut might be
\begin{program}
(defmacro sum-of-squares (expr)
  `(sum (* ,expr ,expr)))
\end{program}
but if you are an experienced macro writer, you will realize that this
code will evaluate ~expr~ twice, which is probably a bad idea.  A
better version would use a temporary:
\begin{program}
(defmacro sum-of-squares (expr)
  (let ((temp (gensym)))
    `(let ((,temp ,expr))
       (sum (* ,temp ,temp)))))
\end{program}
Although this may seem complex, it is just the sort of thing you'd
have to go through to write any macro, which illustrates the point of
this section: if you can write macros, you can extend \iter.

Our macros don't use \iter's keyword-argument
syntax.  We could just use keywords with |defmacro|, but we would
still not be using \iter's clause indexing mechanism.  Unlike Lisp,
which uses just the first symbol of a form to determine what function
to call, \iter\ individuates clauses by the list of required keywords.
For instance, |for\dots in| and |for\dots in-vector| are different clauses
implemented by distinct Lisp functions.  

To buy into this indexing scheme, as well as the keyword-argument
syntax, use |defmacro-clause|:

\begin{clauses}

\defmacro{defmacro-clause}{arglist |\&body| body}
Defines a new \iter\ clause.  ~arglist~ is a list of symbols which are
alternating keywords and arguments.  \opt\ may be used, and the list
may be terminated by |\&sequence|.  ~body~ is an ordinary macro body,
as with |defmacro|.  If the first form of ~body~ is a string, it is
considered a documentation string and will be shown by
|display-iterate-clauses|. |defmacro-clause| will signal an error if 
defining the clause would result in an ambiguity.  E.g. you cannot
define the clause |for\dots from| because there would be no way to
distinguish it from a use of the |for| clause with optional keyword |from|.

\end{clauses}

\medskip

Here is |multiply| using |defmacro-clause|.  The keywords are capitalized
for readability.
\begin{program}
(defmacro-clause (MULTIPLY expr \opt\ INTO var)
  `(reducing ,expr by \#'* into ,var initial-value 1))
\end{program}
You don't have to worry about the case when |var| is not supplied; for
any clause with an |into| keyword, saying |into nil| is equivalent to
omitting the |into| entirely.

As another, more extended example, consider the fairly common
iteration pattern that 
involves finding the sequence element that maximizes (or minimizes) some
function.  \iter\ provides this as |finding\dots maximizing|, but it's
instructive to see how to write it.
Here, in pseudocode, is how you might write such a loop for
maximizing a function F:

\begin{quote}
\begin{tabbing}
set variable MAX-VAL to NIL; \\
set variable WINNER to NIL; \\
for\=\ each element EL in the sequence \\
\>  if\=\ MAX-VAL\=\ is NIL or F(EL) $>$ MAX-VAL then \\
\>\>    set MAX-VAL to F(EL); \\
\>\>    set WINNER to EL; \\
\>  end if; \\
end for; \\
return WINNER.
\end{tabbing}
\end{quote}

Here is the macro:
\begin{program}
(defmacro-clause (FINDING expr MAXIMIZING func \opt\ INTO var)
  (let ((max-val (gensym))
        (temp1 (gensym))
        (temp2 (gensym))
        (winner (or var iterate::*result-var*)))
    `(progn 
       (with ,max-val = nil)
       (with ,winner = nil)
       (cond 
        ((null ,max-val)
         (setq ,winner ,expr)
         (setq ,max-val (funcall ,func ,winner))
        (t
         (let* ((,temp1 ,expr)
                (,temp2 (funcall ,func ,temp1)))
           (when (> ,temp2 ,max-val)
             (setq ,max-val ,temp2)
             (setq ,winner ,temp1))))))
       (finally (leave ,winner)))))
\end{program}
Note that if no |into| variable is supplied, we use
|iterate::*result-var*|, which contains the internal variable into
which all clauses place their results.  If this variable is bound by
some clause, then \iter\ will return its value automatically;
otherwise, \nil\ will be returned.  

\subsection{Writing Drivers}

In principle, drivers can be implemented just as easily as other
\iter\ clauses.  In practice, they are a little harder to get right. 
As an example, consider writing a driver that
iterates over all the 
elements of a vector, ignoring its fill-pointer.  |for\dots in-vector|
won't work for this, because it observes the fill-pointer.  It's necessary to
use |array-dimension| instead of |length| to obtain the size of the
vector.  Here is one approach:

\begin{program}
(defmacro-clause (FOR var IN-WHOLE-VECTOR v)
  "All the elements of a vector (disregards fill-pointer)"
  (let ((vect (gensym))
        (index (gensym)))
    `(progn
       (with ,vect = ,v)
       (for ,index from 0 below (array-dimension ,vect 0))
       (for ,var = (aref ,vect ,index)))))
\end{program}
Note that we immediately put |v| in a variable, in case it is an
expression.  Again, this is just good Lisp macrology.  It also has a
subtle effect on the semantics of the driver: |v| is evaluated only
once, at the beginning of the loop, so changes to |v| in the loop have
no effect on the driver.  Similarly, the bounds for numerical iteration
e.g. the above |array-dimension| are also evaluated once only.  This is how
all of \iter's drivers work.


There is an important point concerning the |progn| in this code.  We
need the |progn|, of course, because we are returning several forms,
one of which is a driver.  But \iter\ drivers must occur at top-level.
Is this code in error?  No, because ~top-level~ is defined in \iter\
to include forms inside a |progn|.  This is just the definition of
top-level that Common Lisp uses, and for the same reason: to allow
macros to return multiple forms at top-level.

While our |for\dots in-whole-vector| clause will work, it is not
ideal.  In particular, it does not support generating.  Do do so, we
need to use |for\dots next| or |for\dots do-next|.  The job is
simplified by the |defmacro-driver| macro.

\begin{clauses}

\defmacro{defmacro-driver}{arglist |\&body| body}
Defines a driver clause in
both the |for| and |generate| forms, and provides a parameter
|generate| which ~body~ can examine to determine how it was invoked.
~arglist~ is as in |defmacro-clause|, and should begin with the symbol
|for|. 

\end{clauses}

With |defmacro-driver|, our driver looks like this:
\begin{program}
(defmacro-driver (FOR var IN-WHOLE-VECTOR v)
  "All the elements of a vector (disregards fill-pointer)"
   (let ((vect (gensym))
         (end (gensym))
         (index (gensym))
         (kwd (if generate 'generate 'for)))
     `(progn
        (with ,vect = ,v)
        (with ,end = (array-dimension ,vect 0))
        (with ,index = -1)
        (,kwd ,var next (progn (incf ,index)
                               (if (>= ,index ,end) (terminate))
                               (aref ,vect ,index))))))
\end{program}


We are still missing one thing: the |\&sequence| keywords.  
We can get them easily enough, by writing 
\begin{program}
(defmacro-driver (FOR var IN-WHOLE-VECTOR v \&sequence) 
  ...)
\end{program}
We can now refer to parameters |from|, |to|, |by|, etc. which contain
either the values for the corresponding keyword, or \nil\ if the
keyword was not supplied.  Implementing the right code for these
keywords is cumbersome but not difficult; it is left as an exercise.
But before you begin, see |defclause-sequence| below for an easier way.
       
\subsection{Extensibility Aids}
\label{aids}

This section documents assorted features that may be of use in
extending \iter.

\begin{clauses}

\defunexpvar{*result-var*}
Holds the variable that is used to return a value as a result of the
\iter\ form.  You may examine this and use it in a |with| clause, but
you should not change it.

\defmacro{defsynonym}{syn word}
 Makes ~syn~ a synonym for the existing \iter\ keyword ~word.~  Only
the first word in each clause can have synonyms. 

\defmacroxx{defclause-sequence}{element-name index-name}{|\&key|
access-fn size-fn sequence-type}{element-type
element-doc-string index-doc-string}  
Provides
a simple way to define sequence clauses.  Generates two
clauses, one for iterating over the sequence's elements, the other
for iterating over its indices.  The first symbol of both
clauses will have print-name |for|.
~element-name~  and ~index-name~ should be symbols.
~element-name~ is the second keyword of the element iterator (typically of
the form 
|in-~sequence-type~|), and ~index-name~ is the second keyword
of the index-iterator (typically of the form
|index-of-~sequence-type~|).  Either name may be 
\nil, in which case the corresponding clause is not defined.  If both
symbols are supplied, they should be in the same package.  The |for|
that begins the clauses will be in this package.

\cpar ~access-fn~ is the function to be used to
access elements of the sequence in the element iterator.  The function
should take two 
arguments, a sequence and an index, and return the appropriate element.
~size-fn~ should denote a function of one argument, a sequence, that
returns its size.  Both ~access-fn~ and ~size-fn~ are required for the
element iterator, but only ~size-fn~ is needed for the index iterator.

\cpar The ~sequence-type~ and ~element-type~ keywords are used to
suggest types for the variables 
used to hold the sequence and the
sequence elements, respectively.  The usual rules about \iter's
treatment of variable type declarations apply (see section \ref{types}).

\cpar ~element-doc-string~ and ~index-doc-string~ are
the documentation strings, for use with |display-iterate-clauses|. 

\cpar The generated element-iterator performs destructuring on the
element variable.

\cpar As an example, the above |for\dots in-whole-vector| example
could have been written:
\begin{program}
(defclause-sequence IN-WHOLE-VECTOR INDEX-OF-WHOLE-VECTOR
  :access-fn 'aref
  :size-fn \#'(lambda (v) (array-dimension v 0))
  :sequence-type 'vector
  :element-type t
  :element-doc-string 
     "Elements of a vector, disregarding fill-pointer"
  :index-doc-string 
     "Indices of vector, disregarding fill-pointer")
\end{program}

\end{clauses}

\subsection{Subtleties}

There are some subtleties to be aware of when writing \iter\ clauses.
First, the code returned by your macros may be |nconc|'ed into a list,
so you should always returned freshly consed lists, rather than
constants.  Second, \iter\ matches clauses by using |eq| on the first
symbol and |string=| on the subsequent ones, so the package of the
first symbol of a clause is relevant.  All of the clauses in this manual
have their first word in the \iter\ package.  
You can use the package system in the usual way to shadow
\iter\ clauses without replacing them.

%%% say more here, about the badness that only the first word of a
%%% clause is packagey.

\section{Non-portable Extensions to Iterate (Contribs)}
\label{contribs}

Currently, there is only one non-portable extension to iterate in the
distribution: iterate-pg. If you have made an extension that depends
on non-portable features, feel free to send them to |asf@boinkor.net|
for inclusion in the iterate distribution.

\subsection{An SQL query driver for iterate}

The pg package by Eric Marsden (see |http://cliki.net/pg|) provides an
interface to the PostgreSQL database. Using the \iterpg\ extension, it
is possible to handle the results of SQL queries using \iter.

This usage example should give you an idea of how to use it:

\begin{program}
(pg:with-pg-connection (c "somedb" "someuser")
  (iter (for (impl version date) in-relation "select * from version"
                                 on-connection *dbconn*)
        (collect version)))
\end{program}

To use the extension via |ASDF|, simply make your system depend on the
|iterate-pg| system instead of the |iterate| system. To load it
manually, use:

\begin{program}
  (asdf:oos 'asdf:load-op :iterate-pg)
\end{program}

%%%%%%%%%%%%%%%%%%%%%%%%%%%%%%%%%%%%%%%%%%%%%%%%%%%%%%%%%%%%%%%%
\section{Obtaining {\tt Iterate}}

\begin{sloppypar}
The information in this chapter is obsolete but included for
completeness's sake; Currently, the most up-to-date information on
\iter\ can be found at |http://boinkor.net/iterate.html|.
\end{sloppypar}

\begin{sloppypar}
\iter\ currently runs on Lisp Machines, and on
HP's, Sun3's and Sparcstations under Lucid.  
\iter\ source and binaries are available at the MIT AI Lab in the
subdirectories of |/src/local/lisplib/|.  The source file,
|iterate.lisp|, is also available for anonymous FTP in the directory
|/com/fpt/pub/| on the machine |TRIX.AI.MIT.EDU| (Internet number
128.52.32.6).  If you are unable to obtain |iterate| in one of these
ways, send mail to |jba@ai.mit.edu| and I will send you the source
file. 
\end{sloppypar}

\begin{sloppypar}
\iter\ resides in the |iterate| package (nickname |iter|).  Just say
\linebreak |(use-package :iterate)| to make all the necessary symbols
available. 
If a symbol is not exported, it appears in this manual with an
``|iterate::|'' prefix.
\end{sloppypar}

Send bug reports to |bug-iterate@ai.mit.edu|.  The |info-iterate|
mailing list will have notices of changes and problems; to have
yourself added, send mail to |info-iterate-request@ai.mit.edu|.


\medskip

\begin{flushleft}
 \bf Acknowledgements
\end{flushleft}
\smallskip

Richard Waters provided invaluable criticism which spurred me to improve
\iter\ greatly.  As early users, David Clemens, Oren Etzioni and Jeff
Siskind helped ferret out many bugs.

%\begin{theindex}
% The index files must be generated with the genindex program
% or from makeindex iterate-manual.idx which creates iterate-manual.ind.

%\baselineskip\normalbaselineskip
%\advance\baselineskip by -2pt

% Fix ``.'' spacing in lists
\documentstyle[12pt,aimemo]{article}

\setlength{\textwidth}{6.0in}
\setlength{\oddsidemargin}{0.25in}
\setlength{\evensidemargin}{0.25in}
\setlength{\topmargin}{0pt}
\setlength{\headsep}{24pt}
\setlength{\textheight}{8.5in}

\let\origpar=\par

% somehow, LaTex fucks with things so that the plain Tex \obeyspaces
% doesn't work.  This is the same, except we use an mbox containing
% a space instead of just a space.

\newcommand{\spc}{\mbox{ }}
\newcommand{\tAb}{\mbox{        }}
\def\obspaces{\catcode`\ =\active\catcode`\^^I=\active}
{\obspaces \global\let =\spc\global\let^^I=\tAb}

\newenvironment{program}{\medskip\samepage\tt\obeylines\obspaces}{\medskip}

% use | for \tt
\catcode`\|=13
\newif\ifvbar
\def|{\ifvbar
         \endgroup\vbarfalse
      \else 
         \begingroup\tt\vbartrue 
      \fi}



\newcommand{\lisp}{\tt}
\newcommand{\iter}{|iterate|}
\newcommand{\iterpg}{|iterate-pg|}
\newcommand{\setf}{|setf|}
\newcommand{\nil}{|nil|}
\newcommand{\nonnil}{non-\nil}
\newcommand{\opt}{{\lisp \&optional}}
\newcommand{\yields}{$\Rightarrow$}



\newbox\Dots
\setbox\Dots=\hbox{\small ...}
\renewcommand{\dots}{\copy\Dots}

% Here's a way to define an environment which will treat all
% paragraphs inside it in the same manner.  It's, as near as I can
% figure, the way that Latex does things.  Using \everypar won't work;
% apparently Latex resets that a lot.  What you have to do is: at the
% top of the file do \let\origpar=\par.  In the
% begin part of the environment, do the parshape (or whatever), then say
% \def\par{{\origpar}}.  The extra braces are necessary.   That's all
% you have to do; at the end of the environment things will return to
% normal automatically.  Why does this all work?  I have no idea.
% Another important thing: there should be a blank line (i.e. a \par)
% between the end of the last paragraph and the \end{environment}.

\newlength{\clindent}
\setlength{\clindent}{0.5in}
\newlength{\clparindent}
\setlength{\clparindent}{\parindent}

\newenvironment{clauses}{\advance\linewidth -\parindent
                        \hangindent\clindent\relax
                         \hangafter=0
                          \parindent=0pt
                          \def\par{{\origpar}}}{}

\newcommand{\cpar}{\hspace \clparindent}

\newcommand{\startitem}{\bigskip\pagebreak[3]}
\newcommand{\finishitem}{\smallskip}

\newcommand{\clause}[1]{\startitem\Clause{#1}\finishitem}
\newcommand{\Clause}[1]{\hbox{#1}}
\newcommand{\clindex}[1]{\index{{\lisp #1}}}
\newcommand{\clindexx}[2]{\index{{\lisp #1\dots #2}}}
\newcommand{\clindexxx}[3]{\index{{\lisp #1\dots #2\dots #3}}}
\newcommand{\clausex}[2]{\startitem\Clausex{#1}{#2}\finishitem}
\newcommand{\Clausex}[2]{\Clause{|#1| #2}\clindex{#1}}

\newcommand{\defvar}[1]{\startitem\deff{#1}{}{Variable}\finishitem}
\newcommand{\defunexpvar}[1]{\startitem\deffnoindex{iterate::#1}{}{Variable}
                                        \lispindex{#1}\finishitem}
\newcommand{\defun}[2]{\startitem\deff{#1}{#2}{Function}\finishitem}
\newcommand{\defmacro}[2]{\startitem\deff{#1}{#2}{Macro}\finishitem}
\newcommand{\defunx}[3]{\startitem\deffx{#1}{#2}{#3}{Function}\finishitem}
\newcommand{\defmacrox}[3]{\startitem\deffx{#1}{#2}{#3}{Macro}\finishitem}
\newcommand{\defunxx}[4]{\startitem\deffxx{#1}{#2}{#3}{#4}{Function}\finishitem}
\newcommand{\defmacroxx}[4]{\startitem\deffxx{#1}{#2}{#3}{#4}{Macro}\finishitem}

\newcommand{\deff}[3]{\deffnoindex{#1}{#2}{#3}\lispindex{#1}}
\newcommand{\deffnoindex}[3]{\hbox to \hsize{{\tt #1} {\it #2}\hfill [{\it #3}]}}
\newcommand{\deffx}[4]{\deff{#1}{#2}{#4}\moreargs{#1}{#3}}
\newcommand{\deffxx}[5]{\deffx{#1}{#2}{#3}{#5}\moreargs{#1}{#4}}

\newcommand{\moreargs}[2]{\hbox to \hsize{\phantom{\lisp #1} {\it #2}\hfill}}

\newcommand{\lispindex}[1]{\index{{\lisp #1}}}

 \newenvironment{note}[1]{\pagebreak[2]\bigskip
                          \hrule\smallskip\small
                         {\setlength{\parindent}{0pt}
                           \par{\bf #1:}}}{\smallskip\hrule\bigskip}



\makeindex

%%%%%%%%%%%%%%%%%%%%%%%%%%%%%%%%%%%%%%%%%%%%%%%%%%%%%%%%%%%%%%%%

%\thispagestyle{empty}

%\vspace*{2in}

%\title{The |Iterate| Manual}
%\author{Jonathan Amsterdam}

%\maketitle

%\thispagestyle{empty}

%\begin{center}
%\Large *** DRAFT ***
%\end{center}
%This manual is in beta-test. It will become an AI Memo in a couple of
%months.  I would appreciate any comments or
%corrections.  You can reach me at email address |jba@ai.ai.mit.edu|.

\begin{document}

\def\writtenby{Jonathan Amsterdam}
\def\contractno{This report describes research done at the Artificial
Intelligence Laboratory of the Massachusetts Institute of Technology.
Support for the laboratory's artificial intelligence research is
provided in part by the Advanced Research Projects Agency of the
Department of Defense under Office
of Naval Research contract N00014-85-K-0124.}

\AIMEMO{1236}{\Large The Iterate Manual \\[1in]}
{This is the manual for version 1.4 of \iter, a powerful iteration
macro for Common  
Lisp. \iter\ is similar to {\tt loop} but provides numerous additional
features, is well integrated with Lisp, and is extensible. 
\vfill }


% use ~ for \em
\catcode`\~=13
\newif\ifem
\def~{\ifem
         \/\endgroup\emfalse
      \else 
         \begingroup\em\emtrue 
      \fi}



%\pagebreak
%\setcounter{page}{1}

%\thispagestyle{empty}


\tableofcontents

\pagebreak
\pagestyle{headings}

\section{Introduction}

\begin{sloppypar}
This manual describes
\iter, a powerful iteration facility for Common Lisp.
\iter\ provides abstractions for many common iteration
patterns and allows for the definition of additional patterns.  \iter\ is
a macro that expands into ordinary Lisp at
compile-time, so it is more efficient than higher-order
functions like |map| and |reduce.|  
While it is similar to |loop|, \iter\ offers a more Lisp-like syntax
and enhanced extensibility.
(For a more complete
comparison of \iter\ with other iteration constructs, see MIT AI Lab Working
Paper No. 324, ~Don't Loop, Iterate.~)
\end{sloppypar}

An \iter\ form consists of the symbol |iter|\footnote{You can also use
|iterate|, but |iter| is preferred because it avoids potential conflicts with
possible future additions to Common Lisp, and because it saves
horizontal space when writing code.}
followed by one or more
forms, some of which may be \iter\ 
~clauses.~  Here is a simple example of \iter\ which collects the
numbers from 1 to 10 into a list, and returns the list.  The return
value is shown following the arrow.
\begin{program}
(iter (for i from 1 to 10)
      (collect i))          \yields (1 2 3 4 5 6 7 8 9 10)
\end{program}
This form contains two clauses: a |for| clause that steps the
variable |i| over the integers from 1 to 10, and a |collect|
clause that accumulates its argument into a list.  With a few
exceptions, all \iter\ clauses have the same format: alternating
symbols (called ~keywords~) and expressions (called
~arguments~).  The syntax and terminology are those of Common
Lisp's keyword lambda lists.  One difference is that \iter's keywords
do not have to begin with a colon---though they may, except for the
first symbol of a clause.  So you can also write  |(for i :from
1 :to 10)| if you prefer.

Any Lisp form can appear in the body of an \iter, where it will have
its usual meaning.  \iter\ walks the entire body, expanding macros,
and recognizing clauses at any level.  This example collects all the
odd numbers in a list:
\begin{program}
(iter (for el in list)
      (if (and (numberp el) (oddp el))
          (collect el)))
\end{program}

There are clauses for iterating over numbers, lists, arrays and other
objects, and for
collecting, summing, counting, maximizing and other useful operations.
\iter\ also supports the creation of new variable bindings, stepping
over multiple sequences at once, destructuring, and compiler
declarations of variable types.  The following example illustrates
some of these features: 
\begin{program}
(iter (for (key . item) in alist)
         (for i from 0)
         (declare (fixnum i))
         (collect (cons i key)))
\end{program}

This loop takes the keys of an alist and returns a new alist
associating the keys with their positions in the original list. The
compiler declaration for |i| will appear in the generated code
in the appropriate place.

%%%%%%%%%%%%%%%%%%%%%%%%%%%%%%%%%%%%%%%%%%%%%%%%%%%%%%%%%%%%%%%%
\section{Clauses}

Most of \iter's clauses will be familiar to |loop| programmers.  
(|loop| is an iteration macro that has been incorporated into Common
Lisp.  Seen Guy Steele's {\em Common Lisp, 2nd Edition}.)
In
nearly all cases they behave the same as their |loop| counterparts, so
a |loop| user can switch to \iter\ with little pain (and much gain).

All clauses with the standard keyword-argument syntax consist of two
parts: a ~required~ part, 
containing keywords that must be present and in the right order; and
an ~optional~ part, containing keywords that may be omitted and,
if present, may occur in any order.  In the descriptions below, the
parts are separated by the Lisp lambda-list keyword \opt.


\subsection{Drivers}

An iteration-driving clause
conceptually causes the iteration to go forward.  Driver clauses in
\iter\ allow iteration over numbers, lists, vectors, hashtables, packages,
files and streams.  Iteration-driving clauses must
appear at the top level of an \iter\ form; they cannot be nested
inside another clause.  The driver variable is updated at the point
where the driver clause occurs.  Before the clause is executed for the
first time, the value of the variable is undefined.

%Also, regardless of where the driver clause
%appears in the body, the driver variable is stepped at the top of the
%loop; hence it is stylistically preferable, though
%not required, to place driver clauses at the beginning of the \iter.

Multiple drivers may appear in a single \iter\ form, in which case all
of the driver variables are updated each time through the loop, in the
order in which the clauses appear.  The first driver to terminate will
terminate the entire loop.

In all cases, the value of the driver variable on exit from the loop,
including within the epilogue code (see the |finally| clause), is
undefined.

All the parameters of a driver clause are evaluated once, before the
loop begins.  Hence it is not possible to change the bounds or other
properties of an iteration by side-effect from within the loop.

With one exception, driver clauses begin with the word |for| (or
the synonym |as|) and mention an iteration variable, which is
given a binding within the \iter\ form.
The exception is |repeat|, which just executes a loop a
specified number of times:

\begin{clauses}

\clausex{repeat}{~n~}
Repeats the loop ~n~ times.  For example:
\begin{program}
(iter (repeat 100)
      (print "I will not talk in class."))
\end{program}
\cpar If $n \leq 0$, the
loop will never be executed.  If ~n~ is not an integer, the
actual number of executions will be $\lceil n \rceil$.

\end{clauses}

\pagebreak[3]

\subsubsection{Numerical Iteration}

\begin{clauses}
\clausex{for}{~var~ |\&sequence|} 
The general form for iterating over a sequence of numbers requires a
variable and, optionally, one or more keywords that provide the bounds
and step size of the iteration. The |\&sequence|
lambda-list keyword is a shorthand for these sequence keywords.  They are:
|from|, |upfrom|, |downfrom|, |to|, |downto|, |above|, |below| and |by|.
|from| provides the starting value for ~var~
and defaults to zero. |to| provides a final value and implies that the
successive values of ~var~ will be increasing; |downto| implies that
they will be decreasing.  The loop terminates when ~var~ passes
the final value (i.e. becomes smaller or larger than it, depending on
the direction of iteration); in other  words, the loop body
will never be 
executed for values of ~var~ past the final value.
|below| and |above| are similar to |to| and
|downto|, except that the loop terminates when ~var~ equals or passes
the final value.

\cpar If no final value is specified, the variable will be stepped
forever.  Using |from| or |upfrom| will result in increasing
values, while |downfrom| will give decreasing values.

\cpar On each iteration,
~var~ is incremented or decremented by the value of the sequence
keyword |by|, which defaults to 1.  It should always be a positive
number, even for downward iterations.

\cpar In the following examples, the sequence of numbers generated is shown
next to the clause.

\begin{program}
(for i upfrom 0) \yields\ 0 1 2 \ldots
(for i from 5)   \yields\ 5 6 7 \ldots    ; either from or upfrom is okay
(for i downfrom 0) \yields\ 0 -1 -2 \ldots
(for i from 1 to 3) \yields\ 1 2 3
(for i from 1 below 3) \yields\ 1 2
(for i from 1 to 3 by 2) \yields\ 1 3
(for i from 1 below 3 by 2) \yields\ 1
(for i from 5 downto 3) \yields\ 5 4 3
\end{program}

\end{clauses}

\subsubsection{Sequence Iteration}

There are a number of clauses for iterating over sequences.  In all of
them, the argument following |for| may be a list instead of a
symbol, in which case destructuring is performed.  See section
\ref{destructuring}. 

\begin{clauses}

\clause{|for| ~var~ |in| ~list~ \opt\ |by| ~step-function~}
\clindexx{for}{in}
~var~ is set to successive elements of list.  
~step-function,~ which 
defaults to |cdr|, is used to obtain the next sublist. 

\clause{|for| ~var~ |on| ~list~ \opt\ |by| ~step-function~}
\clindexx{for}{on}
~var~ is set to successive sublists of list. 
~step-function~ (default |cdr|) is used as in |for\dots in|.

\end{clauses}

\medskip

These two clauses use |atom| to test for the end
of a list.  Hence, given a list whose final |cdr| is not \nil,
they will silently ignore the last |cdr|.  Other choices are
|endp|,  which would signal an error, and |null|, which
would probably result in an error somewhere else.  If you wish to use
an end-test other than |atom|, set the variable
|iterate::*list-end-test*|\lispindex{*list-end-test*} to the name of
the desired function. 

\begin{clauses}
\clause{|for| ~var~ |in-vector| ~vector~ |\&sequence|}
\clindexx{for}{in-vector}
~var~ takes on successive elements from ~vector.~   The vector's
fill-pointer is observed.  Here and in subsequent clauses, the |\&sequence|
keywords include |with-index|, which takes a symbol as argument
and uses it for the index variable instead of an internally generated
symbol.  The other |\&sequence| keywords behave as in numerical
iteration, except that the default iteration bounds are the bounds of
the vector.  E.g. in |(for i in-vector v downto 3)|,
|i| will start off being bound to the last element in |v|,
and will  
be set to preceding elements down to and including the element with
index 3.

\clause{|for| ~var~ |in-sequence| ~seq~ |\&sequence|}
\clindexx{for}{in-sequence}
This uses Common Lisp's generalized sequence functions, |elt|
and |length|, to obtain elements and determine the length of
~seq.~   Hence it will work for any sequence, including lists, and will
observe the fill-pointers of vectors.

\clause{|for| ~var~ |in-string| ~string~ |\&sequence|}
\clindexx{for}{in-string}
~var~ is set to successive characters of ~string.~

\startitem
\Clause{|for| ~var~ |index-of-vector| ~vector~ |\&sequence|}
\clindexx{for}{index-of-vector}
\Clause{|for| ~var~ |index-of-sequence| ~sequence~ |\&sequence|}
\clindexx{for}{index-of-sequence}
\Clause{|for| ~var~ |index-of-string| ~string~ |\&sequence|}
\clindexx{for}{index-of-string}
\finishitem
~var~ is set to successive indices of the sequence.
These clauses avoid the overhead of accessing the sequence elements
for those applications where they do not need to be examined, or are
examined rarely.  They admit all the optional keywords of the other
sequence drivers except the (redundant) |with-index| keyword.

\clause{|for| ~(key value)~ |in-hashtable| ~table~}
\clindexx{for}{in-hashtable}

~key~ and ~value,~ which must appear as shown in a list
and may be destructuring templates, are set to the keys and values of
~table~.  If ~key~ is |nil|, then the hashtable's keys will be ignored;
similarly for ~value~.
The order in which 
elements of ~table~ will be retrieved is unpredictable.

\clause{|for| ~var~ |in-package| ~package~ \opt\ |external-only| ~ext~}
\clindexx{for}{in-package}
Iterates over all the symbols in ~package,~ or over only the
external symbols if ~ext~ is specified and non-|nil|.  ~ext~ is not
evaluated.  The same symbol may appear more than once.

\clause{|for| ~var~ |in-packages| ~packages~ \opt\ |having-access| ~symbol-types~}
\clindexx{for}{in-packages}
Iterates over all the symbols from the list of packages denoted by the
descriptor ~packages~ and having accessibility (or visibility) given by
~symbol-types~.  This defaults to the list |(:external :internal :inherited)|
and is not evaluated.  ~var~ must be a list of up to three variables: in each
iteration, these will be set to a symbol, its access-type and package (as per
|with-package-iterator| in ANSI-CL).  The same symbol may appear more than
once.

\end{clauses}

\begin{clauses}

\clause{|for| ~var~ |in-file| ~name~ \opt\ |using| ~reader~}
\clindexx{for}{in-file}
Opens the file ~name~ (which may be a string or pathname) for input,
and iterates 
over its contents. ~reader~ defaults to |read|, so by default ~var~
will be bound to the successive forms in the file.  The \iter\ body is
wrapped in an unwind-protect to ensure that the file is closed no
matter how the \iter\ is exited.

\clause{|for| ~var~ |in-stream| ~stream~ \opt\ |using| ~reader~}
\clindexx{for}{in-stream}
Like |for\dots in-file|, except that ~stream~ should be an existing
stream object that supports input operations.

\end{clauses}

\subsubsection{Generalized Drivers}

These are primarily useful for writing drivers that can also be used
as generators (see section \ref{generators}, below).
 
\begin{clauses}

\clause{|for| ~var~ |next| ~expr~}
\clindexx{for}{next}

~var~ is set to ~expr~ each time through the loop.  Destructuring is
performed.  When the clause is used as a generator, ~expr~ is the code
that is executed when |(next ~var~)| is 
encountered (see section \ref{generators}, below).
~expr~ should compute the first value for ~var~, as well
as all subsequent values, and is responsible for terminating the loop.
For compatibility with future versions of \iter, this termination
should be done with |terminate|\clindex{terminate}, which can be
considered a synonym for |finish| (see section \ref{control-flow}).

\cpar As an example, the following clauses are equivalent to |(for
i from 1 to 10)|:
\begin{program}
(initially (setq i 0))
(for i next (if (> i 10) (terminate) (incf i)))
\end{program}

\clause{|for| ~var~ |do-next| ~form~}
\clindexx{for}{do-next}
~form~ is evaluated each time through the loop.  Its value is ~not~
set to ~var~; that is ~form's~ job. ~var~ is only present so that
\iter\ knows it is a driver variable.  \linebreak |(for ~var~ next
~expr~)| is equivalent to |(for ~var~ do-next (dsetq ~var~ ~expr~))|.
(See section \ref{destructuring} for an explanation of |dsetq|.) 

\end{clauses}

\subsubsection{Generators}
\label{generators}

In all of the above clauses, the driver variable is updated on each
iteration.
Sometimes it is desirable to have greater control over updating.  
For instance, consider the problem of associating numbers, in
increasing order and with no gaps, with the
\nonnil\ elements of a list.  One obvious first pass at writing this is:

\begin{program}
(iter (for el in list)
      (for i upfrom 1)
      (if el (collect (cons el i))))
\end{program}
But on the list |(a b nil c)| this produces |((a . 1) (b . 2) (c .
4))| instead of the desired |((a . 1) (b . 2) (c . 3))|.  The problem
is that |i| is incremented each time through the loop, even when |el|
is |nil|.  

The problem could be solved elegantly if we could step |i| only when
we wished to.  This
can be accomplished for any \iter\ driver by writing |generate| (or
its synonym |generating|)
instead of |for|.  Doing so produces a ~generator~---a driver whose
values are yielded explicitly.  To obtain the next value of a
generator variable ~v~, write \linebreak |(next ~v~)|.  The value of a |next|
form is the next value of ~v~, as determined by its associated driver
clause.  |next| also has the side-effect of updating ~v~ to that
value.  If there is no next value, |next| will terminate the loop,
just as with a normal driver.


Using generators, we can now write our example like this:
\begin{program}
(iter (for el in list)
      (generate i upfrom 1)
      (if el (collect (cons el (next i)))))
\end{program}
Now |i| is updated only when |(next i)| is executed, and this occurs
only when |el| is \nonnil.

To better understand the relationship between ordinary drivers and
generators, observe that we can rewrite an ordinary driver using its
generator form immediately followed by |next|, as this example shows:
\begin{program}
(iter (generating i from 1 to 10)
      (next i)
      ...)
\end{program}
Provided that the loop body contains no |(next i)| forms, this will
behave just as if we had written |(for i from 1 to 10)|.

We can still refer to a driver variable ~v~ without using |next|; in
this case, its value is that given to it by the last evaluation of
|(next ~v~)|.  Before |(next ~v~)| has been called the first time, the
value of ~v~ is undefined.  

This semantics is more flexible than
one in which ~v~ begins the loop bound to its first value and calls
of |next| supply subsequent values, because it means the loop will not
terminate too soon if the generator's sequence is empty.  For
instance, consider the following code, which tags \nonnil\ elements of
a list using a list of tags, and also counts the null elements.
(We assume there are at least as many tags as \nonnil\ elements.)
\begin{program}
(let* ((counter 0)
       (tagged-list (iter (for el in list)
                          (generating tag in tag-list) 
                          (if (null el)
                              (incf counter)
                              (collect (cons el (next tag)))))))
  ...)
\end{program}
It may be that there are just as many tags as non-null elements of
|list|.  If all the elements of |list| are null, we still want the
counting to proceed, even though |tag-list| is \nil.  If |tag| had to be
assigned its first value before the loop begins, we would have had to
terminate the loop before the first iteration, since when |tag-list|
is \nil, |tag| 
has no first value.  With the existing semantics, however, |(next
tag)| will never execute, so the iteration will cover all the elements of
|list|. 
        
When the ``variable'' of a driver clause is actually a destructuring
template containing several variables, all the variables are eligible
for use with |next|.  As before, |(next ~v~)| evaluates to ~v's~ next
value; but the effect is to update all of the template's variables.
For instance, the following code will return the list |(a 2 c)|.
\begin{program}
(iter (generating (key . item) in '((a . 1) (b . 2) (c . 3)))
      (collect (next key))
      (collect (next item)))
\end{program}

Only driver clauses with variables can be made into generators.  This
includes all clauses mentioned so far except for |repeat|.  It does
~not~ include |for\dots previous|, |for\dots =|,
|for\dots initially\dots then| or |for\dots first\dots then| (see
below). 

\subsubsection{Previous Values of Driver Variables}

Often one would like to access the value of a variable on a previous
iteration.  \iter\ provides a special clause for accomplishing this.

\begin{clauses}

\clause{|for| ~pvar~ |previous| ~var~ \opt\ |initially| ~init~
|back| ~n~}
\clindexx{for}{previous}
Sets ~pvar~ to the previous value of ~var,~ which should be a driver
variable, a variable from another |for\dots previous| clause, or a
variable established by a |for\dots =|, 
|for\dots initially\dots then| or |for\dots first\dots then| clause
(see section \ref{setting}).
Initially, ~pvar~ is given the value ~init~ (which defaults to \nil).  
The ~init~ expression will be moved outside the loop body, so it
should not depend on anything computed within the loop.
~pvar~ retains the value of ~init~ until ~var~ is set to its second
value, at which point ~pvar~ is set to ~var's~ first value; and so on.  

\cpar The
argument ~n~ to |back| must be a constant, positive integer, and
defaults to 1.  It determines how many iterations back ~pvar~ should
track ~var~.  For example, when ~n~ is 2, then ~pvar~ will be assigned
~var's~ first value when ~var~ is set to its third value.

\cpar A |for\dots previous| clause may occur after or before its
associated driver clause. |for\dots previous| works with generators as
well as ordinary drivers.

\pagebreak[3]

\cpar Example:
\begin{program}
(iter (for el in '(1 2 3 4))
      (for p-el previous el)
      (for pp-el previous p-el initially 0)
      (collect pp-el))
\end{program}
This evaluates to |(0 0 1 2)|.  It could have been written more
economically as 
\begin{program}
(iter (for el in '(1 2 3 4))
      (for pp-el previous el back 2 initially 0)
      (collect pp-el))
\end{program}


\end{clauses}

\subsection{Variable Binding and Setting}
\label{setting}

Several clauses exist for establishing new variable bindings or for
setting variables in the loop.  They all support destructuring.

\begin{clauses}

\clausex{with}{~var~ \opt\ |=| ~value~}
Causes ~var~ to be bound to value before the loop body is entered.
If ~value~ is not supplied, ~var~ assumes a default
binding, which will be 
\nil\ in the absence of declarations.  Also, if ~value~ is not
supplied, no destructuring is performed; instead, ~var~ may be a list of
symbols, all of which are given default 
bindings.  If ~value~ is supplied, ~var~ is bound to it, with
destructuring.  

\cpar Because |with|
creates bindings whose scope includes the entire \iter\ form, it is
good style to put all |with| clauses at the beginning.

\cpar Successive occurrences of |with| result in sequential
bindings (as with 
|let*|).  There is no way to obtain parallel bindings; see
section \ref{bindings} for a rationale.  


\clause{|for| ~var~ |=| ~expr~}
\clindexx{for}{=}
On each iteration, ~expr~ is evaluated and ~var~ is set
to its value. 

\cpar This clause may appear to do the same thing as |for\dots next|.
In fact, they are quite different.  |for\dots =| provides only three
services: it sets up a binding for ~var~, sets it to ~expr~ on each
iteration, and makes it possible to use |for\dots previous| with
~var~.  |for\dots next| provides these services in addition to the
ability to turn the driver into a generator.
%Also, the code which sets ~var~ appears in the loop body in the same
%place as the |for\dots =| clause; the code for |for\dots next| appears
%at the top of the loop, as with other drivers (except when being used
%as a generator).

\clause{|for| ~var~ |initially| ~init-expr~ |then| ~then-expr~}
\clindexxx{for}{initially}{then}
Before the loop begins, ~var~ is set to ~init-expr;~ on all
iterations after the first it is set to ~then-expr.~  This clause
must occur at top-level.  ~init-expr~ will be moved outside the loop
body and ~then-expr~ will be moved to the end of the loop body, so
they are subject to code motion problems (see section
\ref{code-movement}). 

\cpar This clause may appear to be similar to |for\dots next|, but in
fact they differ significantly.  |for\dots initially\dots then| is
typically used to give ~var~ its first value before the loop begins,
and subsequent values on following iterations.  This is incompatible
with generators, whose first value and subsequent values must all be
computed by |(next ~var~)|.  Also, the update of ~var~ in 
|for\dots initially\dots then| does not occur at the location of the
clause. 
Use |for\dots initially\dots then| for
one-shot computations where its idiom is more convenient, but use
|for\dots next| for extending \iter\ with new drivers (see section
\ref{extend}).

\clause{|for| ~var~ |first| ~first-expr~ |then| ~then-expr~}
\clindexxx{for}{first}{then}
The first time through the loop, ~var~ is set to ~first-expr;~ on
subsequent iterations, it is set to ~then-expr.~  This differs from
|for\dots initially| in that ~var~ is set to ~first-expr~
inside the loop body, 
so ~first-expr~ may depend on the results of other clauses.  For
instance, 
\begin{program}
(iter (for num in list)
      (for i first num then (1+ i))
      ...)
\end{program}
will set |i| to the first element of |list| on the first
iteration, whereas 
\begin{program}
(iter (for num in list)
      (for i initially num then (1+ i))
      ...)
\end{program}
is probably erroneous; |i| will be bound to |num|'s
default binding (usually \nil) for the first iteration.

\end{clauses}

\begin{note}{Compatibility note}
|loop|'s |for\dots =| works like \iter's, but |loop| used the syntax
|for\dots =\dots then| to mean |for\dots initially\dots then|.  It was
felt that these two operations were sufficiently different to warrant
different keywords.

Also, the |for| in the above three clauses is misleading,
since none is true driver (e.g. none has a corresponding |generate|
form).  |setting| would have been a better choice, but |for| was used
to retain some compatibility with |loop|.  
\end{note}


%%%%%%%%%%%%%%%%%%%%%%%%%%%%%%%%
\subsection{Gathering Clauses}
Many of \iter's clauses accumulate values into a variable, or set a
variable under certain conditions.  At the end of the loop, this
variable contains the desired result.  All these clauses have an
optional |into| keyword, whose argument should be a symbol.  If the
|into| keyword is not supplied, the accumulation variable will be
internally generated and its value will be returned at the end of the
loop; if a variable is specified, that variable is used for the
accumulation, and is not returned as a result---it is up to the user
to return it explicitly, in the loop's epilogue code (see |finally|).
It is safe to examine the accumulation variable during the loop, but
it should not be modified.

These clauses all begin with a verb.  When the verb does
not conflict with an existing Common Lisp function, then it may be
used in either its infinitival or present-participle form (e.g. |sum|,
|summing|).  However, when there is a conflict with Common Lisp, only
the present-participle form may be used (e.g. |unioning|).  This is to
prevent \iter\ clauses from clashing with Common Lisp
functions. 

%although these clauses are described as ``producing a value,'' it is a
%mistake to think of the lisp list representing the clause as a
%value-producing form in the usual way.  clauses may legally be written
%where a value is expected, e.g. |(setq x (sum i))|, but the
%lisp value of a clause in such a context is undefined.

\subsubsection{Reductions}

~Reduction~ is an extremely common iteration pattern in which
the results of successive applications of a
binary operation are accumulated.
For example, a loop that computes the sum of the
elements of a list is performing a reduction with the addition
operation.  This could be written in Common Lisp as  |(reduce \#'+
list)| or with \iter\ as
\begin{program}
(iter (for el in list)
      (sum el))
\end{program}

\begin{clauses}

\clausex{sum}{~expr~ \opt\ |into| ~var~}
Each time through the loop, ~expr~ is evaluated and added to a
variable, which is bound initially to zero.  If ~expr~ has a type,
it is ~not~ used as the type of the sum variable, which is always
|number|.  To get the result variable to be of a more specific
type, use an explicit variable, as in
\begin{program}
(iter (for el in number-list)
      (sum el into x)
      (declare (fixnum x))
      (finally (return x)))
\end{program}

\clausex{multiply}{~expr~ \opt\ |into| ~var~}
Like |sum|, but the initial value of the result variable is $1$,
and the variable is updated by multiplying ~expr~ into it.

\clausex{counting}{~expr~ \opt\ |into| ~var~}
~expr~ is evaluated on each iteration.  If it is \nonnil, the
accumulation variable, initially zero, is incremented.

\startitem
\Clausex{maximize}{~expr~ \opt\ |into| ~var~}
\Clausex{minimize}{~expr~ \opt\ |into| ~var~}
\finishitem
~expr~ is evaluated on each iteration and its extremum (maximum or
minimum) is stored in the accumulation variable.  If ~expr~ is never
evaluated, then the result is \nil\ (if the accumulation variable is
untyped) or $0$ (if it has a numeric type).

\clausex{reducing}{~expr~ |by| ~func~ \opt\ |initial-value| ~init-val~ |into| ~var~}
This is a general way to perform reductions. ~func~ should be a
function of two arguments, the first of which will be the value
computed so far and
the second of which will be the value of ~expr~.  It should return the
new value.
|reducing| is roughly equivalent to the
Common Lisp |(reduce ~func~ ~sequence~ :key ~expr-function~)|, where
~expr-function~ is used to derive values from the successive elements
of ~sequence~.

\cpar If the |reducing| clause is never executed, the result is
undefined.

\cpar It is not necessary to provide an
initial value, but better code can be generated if one is supplied.
Regardless of its location in the \iter\ body, ~init-val~ will be
evaluated before the loop is entered, so it should not depend on any
value computed inside the \iter\ form.  

%\cpar if a ~var~ is not specified, you can get \iter\ to declare the
%type of the internal variable by putting a |the| expression around
%~func.~  see section \ref{types}.

\end{clauses}

%\begin{note}{implementation note}
%in principle, |maximize| and |minimize| can be thought of
%as reductions where the initial value is the smallest (or largest)
%value that the accumulation variable can assume.  because lisp's
%bignums can represent arbitrary integers, these clauses cannot be
%implemented as reductions in general.  if, however, the type of 
%~expr~ or ~var~ can be determined to be a fixnum
%or a float, \iter\ will implement the clause as a true reduction,
%using one of the constants |most-negative-fixnum|, |%most-positive-fixnum|, 
%|most-negative-short-float|, etc. as appropriate.
%\end{note}

\subsubsection{Accumulations}

All the predefined accumulation clauses add values to a sequence.  If
the sequence is a list, they all 
have the property that the partial list is kept in the correct order
and available for inspection at any point in the loop.

\begin{clauses}

\clausex{collect}{~expr~ \opt\ |into| ~var~ |at| ~place~ |result-type| ~type~}
Produces a sequence of the values of ~expr~ on each iteration. ~place~
indicates where the next value of ~expr~ is added to the list and may
be one of the symbols |start|, |beginning| (a synonym for |start|) or
|end|.  The symbol may be quoted, but need not be.  The default is
|end|.  For example,
\begin{program}
(iter (for i from 1 to 5)
      (collect i))
\end{program}
produces |(1 2 3 4 5)|, whereas
\begin{program}
(iter (for i from 1 to 5)
      (collect i at beginning))
\end{program}
produces |(5 4 3 2 1)| (and is likely to be faster in most Common Lisp
implementations).

\cpar If ~type~ is provided, it should be a subtype of |sequence|.
The default is |list|.  Specifying a type other than |list| will
result in |collect| returning a sequence of that type.  ~However,~ the
type of the sequence being constructed when inside the loop body is
undefined when a non-|list| type is specified.  (As with ~place~,
quoting ~type~ is optional.)

\clausex{adjoining}{~expr~ \opt\ |into| ~var~ |test| ~test~ |at| ~place~ |result-type| ~type~}
Like |collect|, but only adds the value of ~expr~ if it is not
already present.  ~test,~ which defaults to |\#'eql|, is
the test to be used with |member|.

\startitem
\Clausex{appending}{~expr~ \opt\ |into| ~var~ |at| ~place~}
\Clausex{nconcing}{\ ~expr~ \opt\ |into| ~var~ |at| ~place~}
\Clausex{unioning}{\ ~expr~ \opt\ |into| ~var~ |test| ~test~ |at| ~place~}
\Clausex{nunioning}{~expr~ \opt\ |into| ~var~ |test| ~test~ |at| ~place~}
\finishitem
These are like |collect|, but behave like the Common Lisp functions
|append|, |nconc|, |union| or |nunion|.  
As in Common Lisp, they work only on lists.  Also as in Common Lisp,
|unioning| and |nunioning| assume that the value of ~expr~ contains no
duplicates. 

\clausex{accumulate}{~expr~ |by| ~func~ \opt\ |initial-value| ~init-val~ |into| ~var~}
This is a general-purpose accumulation clause. ~func~ should be a
function of two arguments, the value of ~expr~ and the value
accumulated so far in the iteration, and it should return the updated
value.  If no initial value is supplied, \nil\ is used.  

%\cpar If a ~var~ is not specified, you can get \iter\ to declare the
%type of the internal variable by putting a |the| expression around
%~func.~  see section \ref{types}.

\cpar The differences between |accumulate| and |reducing| are slight.
One difference is that the functions take their arguments in a
different order.  Another is that in the absence of ~init-val~,
|accumulate| will use \nil, whereas |reducing| will generate different
code that avoids any dependence on the initial value.
The reason for having both clauses is that one usually
thinks of reductions (like |sum|) and accumulations (like |collect|)
as different beasts.

\end{clauses}


\subsubsection{Finders}

A ~finder~ is a clause whose value is an expression that meets some
condition.

\begin{clauses}

\clause{|finding| ~expr~ |such-that| ~test~ \opt\ |into| ~var~ |on-failure| ~failure-value~}
\clindexx{finding}{such-that}
If ~test~ (which is an expression) ever evaluates to \nonnil, the loop
is terminated, the 
epilogue code is run and the value of ~expr~ is returned.  Otherwise,
\nil\ (or ~failure-value,~ if provided) is returned.  If ~var~ is
provided, it will have either the \nonnil\ value of ~expr~ or
~failure-value~ when the epilogue code is run.

\cpar As a special case, if the ~test~ expression is a sharp-quoted
function, then it is applied to ~expr~ instead of being simply
evaluated.  E.g. |(finding x such-that \#'evenp)| is equivalent to
|(finding x such-that (evenp x))|.  

%\cpar although ~test~ need have nothing to do with ~%expr~ as in
%|(finding j such-that (> i 3))|, it usually 
%will: |(finding (length el) such-that (oddp (length el)))|.  to
%avoid performing the |length| computation twice, you could write
%|(finding (length el) such-that \#'oddp)| or |(finding (length
%el) such-that 'oddp)|; for these cases, \iter\ generates code that
%executes ~expr~ only once.  the code for |\#'oddp|
%is slightly different from that for {\lisp 'oddp}; see the discussion
%under {\lisp for\dots in} and {\lisp for\dots on}.

\cpar |On-failure| is a misnomer. Because it is always evaluated, it behaves
more like the default third argument to the |gethash| function. As a result,
|on-failure (error "Not found")| makes no sense. Instead, the clauses |leave|
or |thereis| can be used in conjunction with |finally| as follows:
\begin{program}
(iter (for x in '(1 2 3))
      (if (evenp x) (leave x))
      (finally (error "not found")))
\end{program}

\cpar This clause may appear multiple times when all defaults are
identical. It can also be used together with either |always|/|never| or
|thereis| if their defaults match. More specifically, |on-failure nil| is
compatible with |thereis|, while |on-failure t| is compatible with |always|
and |never| clauses.
\begin{program}
(iter (for i in '(7 -4 2 -3))
      (if (plusp i)
	  (finding    i  such-that (evenp i))
          (finding (- i) such-that (oddp i))))
\end{program}

\startitem
\Clause{|finding| ~expr~ |maximizing| ~m-expr~ \opt\ |into| ~var~}
\clindexx{finding}{maximizing}
\Clause{|finding| ~expr~ |minimizing| ~m-expr~ \opt\ |into| ~var~}
\clindexx{finding}{minimizing}
\finishitem
Computes the extremum (maximum or minimum) value of ~m-expr~ over all
iterations, and returns the value of ~expr~ corresponding to the
extremum.  ~expr~ is evaluated inside the loop at the time the new
extremum is established.  If ~m-expr~ is never evaluated (due to, for
example, being embedded in a conditional clause), then the returned
value depends on the type, if any, of ~expr~ (or ~var,~ if
one is supplied).  If there is no type, the returned
value will be \nil; if the type is numeric, the returned value will be
zero.

\cpar For these two clauses, ~var~ may be a list of two
symbols; in that case, the first is used to record ~expr~ and
the second, ~m-expr.~  

\cpar As with |finding\dots such-that|, if ~m-expr~ is a sharp-quoted
function, then it is called on ~expr~ instead of being evaluated.

\end{clauses}

\subsubsection{Boolean Tests}

\begin{clauses}

\clausex{first-iteration-p}{}
Returns |t| in the first cycle of the loop, otherwise \nil.

\clausex{first-time-p}{}
Returns |t| the first time the expression is evaluated, and then \nil\ forever.
This clause comes handy when printing (optional) elements separated
by a comma:

\begin{program}
(iter (for el in '(nil 1 2 nil 3))
      (when el
        (unless (first-time-p)
          (princ ", "))
        (princ el)))
\end{program}
produces |"1, 2, 3"|.

\end{clauses}

\subsubsection{Aggregated Boolean Tests}

\begin{clauses}

\clausex{always}{~expr~}
If ~expr~ ever evaluates to
\nil, then \nil\ is immediately returned; the epilogue code is not
executed.  If ~expr~ never evaluates to \nil, the epilogue code
is executed and the last value of ~expr~ (or |t| if ~expr~ was never
evaluated) is returned (whereas |loop| would constantly return |t|).

% mention last evaluated clause when multiple always clauses?

\clausex{never}{~expr~}
Like |(always (not ~expr~))|, except it does not influence the last
value returned by a possible other |always| clause. That is,
\begin{program}
(iter (repeat 2)
      (always 2)
      (never nil)) \yields 2 ; not t
\end{program}

\clausex{thereis}{~expr~}
If ~expr~ is ever \nonnil,
its value is immediately returned without running epilogue code.
Otherwise, the epilogue code is performed and \nil\ is returned.

This clause cannot be used together with |always| or |never|, because their
defaults are opposed (similarly, |(loop always 3 thereis nil)| refuses to
compile in some implementations of |loop|).

\end{clauses}


\subsection{Control Flow}
\label{control-flow}
Several clauses can be used to alter the usual flow of control in a loop.

Note: the clauses of this and subsequent sections don't adhere to \iter's
usual syntax, but instead use standard Common Lisp syntax.  Hence the
format for describing syntax subsequently is
like the standard format used in the Common Lisp manual, not like the
descriptions of clauses above.

\begin{clauses}

\clausex{finish}{}
Stops the loop and runs the epilogue code.

%for example:
%
%\begin{program}
%(iter (with answer = nil)
%         (initially (make-a-mess))
%         (for i from 1 to 10)
%         (when (correct? i)
%           (setq answer i)
%           (finish))
%         (finally (cleanup)))
%\end{program}
%
%this code will execute |cleanup| whether or not the test |(correct?
%i)| ever succeeds. 
%the (more elegant) formulation,
%\begin{program}
%(iter (initially (make-a-mess))
%         (for i from 1 to 10)
%         (finding i such-that (correct? i))
%         (finally (cleanup)))
%\end{program}
%would not execute |cleanup| if |(correct? i)| succeeded; it
%would do an immediate return.

\clausex{leave}{\opt\ ~value~}
Immediately returns ~value~ (default \nil) from the current \iter\
form, skipping the epilogue code.  Equivalent to using |return-from|.

\clausex{next-iteration}{}
Skips the remainder of the loop body and begins the next iteration of
the loop.

\clausex{while}{~expr~}
If ~expr~ ever evaluates to \nil, the loop is terminated and the
epilogue code executed.  Equivalent to |(if (not ~expr~) (finish))|.

\clausex{until}{~expr~}
Equivalent to |(if ~expr~ (finish))|.

\clausex{if-first-time}{~then~ \opt\ ~else~}
If this clause is being executed for the first time in this invocation
of the \iter\ form, then the ~then~ code is evaluated; otherwise the
~else~ code is evaluated.

\cpar |(for ~var~ first ~expr1~ then ~expr2~)| is almost equivalent to
\begin{program}
(if-first-time (dsetq ~var~ ~expr1~)
               (dsetq ~var~ ~expr2~))
\end{program}
The only difference is that the |for| version makes ~var~ available
for use with |for\dots previous|.

\end{clauses}

\subsection{Code Placement}
When fine control is desired over where code appears in a loop
generated by \iter, the following special clauses may be useful.
They are all subject to code-motion problems (see section
\ref{code-movement}).  

\begin{clauses}

\clausex{initially}{|\&rest| ~forms~}
The lisp ~forms~ are placed in the prologue section of the loop, where
they are executed once, before the loop body is entered.

\clausex{after-each}{|\&rest| ~forms~}
The ~forms~ are placed at the end of the loop body, where they
are executed after each iteration.  Unlike the other clauses in this
section, ~forms~ may contain \iter\ clauses.

\clausex{else}{|\&rest| ~forms~}
The lisp ~forms~ are placed in the epilogue section of the loop, where they
are executed if this |else| clause is never met during execution of the
loop and the loop terminates normally.

\clausex{finally}{|\&rest| ~forms~}
The lisp ~forms~ are placed in the epilogue section of the loop, where
they are executed after the loop has terminated normally.

\clausex{finally-protected}{|\&rest| ~forms~}
The lisp ~forms~ are placed in the second form of an unwind-protect
outside the loop.  They are always executed after the loop has
terminated, regardless of how the termination occurred.

\end{clauses}


%%%%%%%%%%%%%%%%%%%%%%%%%%%%%%%%%%%%%%%%%%%%%%%%%%%%%%%%%%%%%%%%
\section{Other Features}

\subsection{Multiple Accumulations}
\label{multiple}

\begin{sloppypar}
It is permitted to have more than one clause accumulate into the same
variable, as in the following:
\begin{program}
(iter (for i from 1 to 10)
      (collect i into nums)
      (collect (sqrt i) into nums)
      (finally (return nums)))
\end{program}
Clauses can only accumulate into the same variable if they are
compatible.  |collect|, |adjoining|, |appending|, |nconcing|,
|unioning| and |nunioning| are compatible with each other; |sum|,
|multiply| and |counting| are compatible; |always| and |never| are compatible;
|finding| \dots |such-that| is compatible with either |thereis| or |always|
and |never| when their defaults match; and |maximize| and |minimize| clauses
are compatible only with other |maximize| and |minimize| clauses,
respectively.

%note that the same variable ~cannot~ be both an accumulation
%variable and an ordinary variable; there can be only one variable with
%a given name within an \iter\ form. 
\end{sloppypar}

\subsection{Named Blocks}

Like Common Lisp |block|s, \iter\ forms can be given names.  The
name should be a single symbol, and it must be the first form in the
\iter.  The generated code behaves exactly like a named block; in
particular, |(return-from ~name~)| can be used to exit it:
\begin{program}
(iter fred
      (for i from 1 to 10)
      (iter barney
            (for j from i to 10)
            (if (> (* i j) 17)
                (return-from fred j))))
\end{program}
An \iter\ form that is not given a name is implicitly named \nil.

Sometimes one would like to write an expression in an inner \iter\ form,
but have it processed by an outer \iter\ form.  This is possible with
the |in| clause.  

\begin{clauses}

\clausex{in}{~name~ |\&rest| ~forms~}
Evaluates ~forms~ as if they were part of the \iter\ form named
~name~.  In other words, \iter\ clauses are processed by the \iter\
form named ~name,~ and not by any \iter\ forms that occur inside ~name.~

\cpar As an example, consider the problem of collecting a list of the
elements in a two-dimensional array.  The naive solution,
\begin{program}
(iter (for i below (array-dimension ar 0))
      (iter (for j below (array-dimension ar 1))
            (collect (aref ar i j))))
\end{program}
\noindent is wrong because the list created by the inner \iter\ is simply
ignored by the outer one.  But using |in| we can write:
\begin{program}
(iter outer (for i below (array-dimension ar 0))
      (iter (for j below (array-dimension ar 1))
            (in outer (collect (aref ar i j)))))
\end{program}
\noindent which has the desired result.

\end{clauses}

\subsection{Destructuring}
\label{destructuring}

In many places within \iter\ clauses where a variable is expected, a
list can be written instead.  In these cases, the value to be assigned
is ~destructured~ according to the pattern
described by the list.  As a simple example, the clause
\begin{program}
(for (key . item) in alist)
\end{program}
\noindent will result in |key| being set to the |car| of
each element in |alist|, and |item| being set to the |cdr|.  The
pattern list may be nested to arbitrary depth, and (as the example
shows) need not be terminated with \nil; the only requirement is that
each leaf be a bindable symbol (or \nil, in which case no binding is
generated for that piece of the structure).

Sometimes, you might like to do the equivalent of a
|multiple-value-setq| in a clause.  This 
``multiple-value destructuring'' can be expressed by writing 
\linebreak |(values $pat_1$ $pat_2 \ldots$)| for a destructuring
pattern, as in 
\begin{program}
(for (values (a . b) c d) = (three-valued-function ...))
\end{program}
\begin{sloppypar}
Note that the $pat_i$ can themselves be destructuring patterns (though
not multiple-value destructuring patterns).  You can't do multiple-value
destructuring in a |with| clause; instead wrap the whole \iter\
form in a |multiple-value-bind|.
\end{sloppypar}

\begin{note}{Rationale}
There are subtle interactions between variable declarations and
evaluation order that make the correct implementation of
multiple-value destructuring in a |with| somewhat tricky.
\end{note}

The destructuring feature of \iter\ is available as a separate
mechanism, using the |dsetq| macro:

\begin{clauses}
\defmacro{dsetq}{template expr}
Performs destructuring of ~expr~ using ~template~.   May be used
outside of an \iter\ form. Yields the primary value of ~expr~.

\end{clauses}

\subsection{On-line Help}

\begin{sloppypar}
There is a limited facility for on-line help, in the form of the 
|display-iterate-clauses| function.
\end{sloppypar}

\begin{clauses}

\defun{display-iterate-clauses}{\opt\ clause-spec}
Displays a list of \iter\ clauses.  If ~clause-spec~ is not
provided, all clauses are shown; if it is a symbol, all clauses
beginning with that symbol are shown; and if it is a list of symbols,
all clauses for which ~clause-spec~ is a prefix are shown.

\end{clauses}

\subsection{Parallel Binding and Stepping}
\label{bindings}

The parallel binding and stepping of variables is a feature that
\iter\ does ~not~ have.  This section attempts to provide a rationale.

We say that two variables are bound ~in parallel~ if neither
binding shadows the other.  This is the usual semantics of |let|
(as opposed to |let*|).  Similarly, we can say that iteration
variables are stepped in parallel if neither variable is updated
before the other, conceptually speaking; in other words, if the code
to update each variable can reference the old values of both variables.

|loop| allows parallel binding of variables and parallel stepping of
driver variables.  My view is that if you are depending on the 
serial/parallel distinction, you are doing something obscure.  If you
need to bind 
variables in parallel using |with|, then you must be using a
variable name that shadows a name in the existing lexical environment.
Don't do that.  The most common use for parallel stepping is to track
the values of variables on the previous iteration, but in fact this
does not require parallel stepping at all; the following will work:
\begin{program}
(iter (for current in list)
      (for prev previous current)
      ...)
\end{program}


%%%%%%%%%%%%%%%%%%%%%%%%%%%%%%%%%%%%%%%%%%%%%%%%%%%%%%%%%%%%%%%%
\section{Types and Declarations}
\label{types}

\subsection{Discussion}

Sometimes efficiency dictates that the types of variables be declared.
This type information needs to be communicated to \iter\ so it can
bind variables to appropriate values.  Furthermore, \iter\ must often
generate internal variables invisible to the user; there needs to be a
way for these to be declared.

As an example, consider this code, which will return the number of
odd elements in |number-list|:
\begin{program}
(iter (for el in number-list)
      (count (oddp el)))
\end{program}
In processing this form,
\iter\ will create an internal variable, let us call it |list17|, to
hold the successive |cdr|s of |number-list|, and  
will bind the variable to |number-list|.  It will also generate a
default binding for |el|; only inside the body of the loop will |el|
be set to the |car| of |list17|.  Finally, \iter\ will generate a
variable, call it |result|, to hold the result of the count, and will
bind it to zero.

When dealing with type declarations, \iter\ observes one simple rule:
~it will never generate a declaration unless requested to do so.~  The
reason is that such declarations might mask errors in compiled code by
avoiding error-checks; the resulting problems would be doubly hard to
track down because the declarations would be hidden from the
programmer.  Of course, a compiler might omit error-checks even in the
absence of declarations, though this behavior can usually be avoided,
e.g. by saying |(declaim (optimize (safety 3)))|.

So, the above \iter\ form will generate code with no declarations.
But say we wish to declare the types of |el| and the internal
variables |list17| and |result|.  How is this done?

Declaring the type of |el| is easy, since the programmer knows
the variable's name:
\begin{program}
(iter (for el in number-list)
      (declare (fixnum el))
      (counting (oddp el)))
\end{program}
\iter\ can read variable type declarations like this one.  Before
processing any clauses, it scans the entire top-level form for type
declarations 
and records the types, so that variable bindings can be performed
correctly.  In this case, |el| will be bound to zero
instead of \nil.  Also, \iter\ collects all the top-level declarations
and puts them at the begining of the generated code, so it is not
necessary to place all declarations at the beginning of an \iter\
form; instead, they can be written near the variables whose types they
declare.  

Since \iter\ is not part of the compiler, it will not know
about declarations that occur outside an \iter\ form; these
declarations must be repeated inside the form.

Here is another way we could have declared the type of |el|:
\begin{program}
(iter (for (the fixnum el) in number-list)
      (counting (oddp el)))
\end{program}
\iter\ extends the Common Lisp |the|\lispindex{the}
form to apply to variables as well as value-producing forms; anywhere
a variable is allowed---in a |with| clause, as the iteration
variable in a driver clause, as the |into| argument of an
accumulation clause, even inside a destructuring template---you can
write |(the ~type~ ~symbol~)| instead.

There is one crucial difference between using a |the| form and
actually declaring the variable: explicit declarations are always
placed in the generated code, but type information from a |the|
form is not turned into an actual declaration unless you tell \iter\
to do so using |iterate:declare-variables|.  See below.

\begin{sloppypar}
Declaring the types of internal variables is harder than declaring the
types of explicitly mentioned variables, since their names
are unknown.  You do it by declaring |iterate:declare-variables|
somewhere inside the top level of the \iter\ form.  (This will also
generate declarations for variables declared using |the|.)
\iter\ does not provide much selectivity here: it's all or none. 
And unfortunately, since \iter\ is not privy to compiler information
but instead reads declarations itself, it will not hear if you
|(declaim (iterate:declare-variables))|.  Instead, set the variable 
|iterate::*always-declare-variables*| to |t| at
compile-time, using |eval-when|.
\end{sloppypar}

To determine the appropriate types for internal variables, \iter\ uses
three sources of information:

\begin{itemize}

\item Often, the particular clause dictates a certain type for a
variable; \iter\ will use this information when available.  In the
current example, the variable |list17| will be given the type
|list|, since that is the only type that makes sense; and the
variable |result| will be given the type |fixnum|, on the
assumption that you will not be counting high enough to need bignums.
You can override this assumption only by using and explicitly declaring a
variable: 
\begin{verbatim}
(iter (declare (iterate:declare-variables))
      (for el in number-list)
      (count (oddp el) into my-result)
      (declare (integer my-result))
      (finally (return my-result)))
\end{verbatim}

\begin{sloppypar}
Other examples of the type assumptions that \iter\ makes are: type
|list| for |into| variables of collection clauses; type |list| for
expressions that are to be destructured; type |vector| for the
variable holding the vector in a |for\dots in-vector| clause, and
similarly for |string| and the |for\dots in-string| clause;
and the implementation-dependent type
|(type-of array-dimension-limit)| for the index and limit
variables generated by sequence iteration drivers like |for\dots
in-vector| and |for\dots in-string| (but not |for\dots in-sequence|,
because it may be used to iterate over a list). 
\end{sloppypar}

\item Sometimes, \iter\ will examine expressions and try to determine
their types in a simple-minded way.  If the expression is
self-evaluating (like a number, for instance), \iter\ knows that the
expression's type is the same as the type of the value it denotes, so
it can use that type.  If the expression is of the form |(the ~type~
~expr~)|, \iter\ is smart enough to extract ~type~ and use it.
However, the current version of \iter\ does not  
examine declarations of function result types or do any type
inference.  It will not determine, for 
example, that the type of |(+ 3 4)| is |fixnum|, or even
|number|.

\item In some cases, the type of an internal variable should match the
type of some other variable.  For instance, \iter\ generates an
internal variable for |(f x)| in the
clause |(for i from 1 to (f x))|, and in the absence of other
information will give it the same type as |i|.  If, however, the
expression had been written |(the fixnum (f x))|, then \iter\
would have given the internal variable the type |fixnum|
regardless of |i|'s type.  The type incompatibility errors that
could arise in this situation are not checked for.

\end{itemize}

Note that if you do declare |iterate:declare-variables|, then
\iter\ may declare user variables as well as internal ones if they do
not already have declarations, though only for variables that it
binds.  For instance, in this code:

\begin{program}
(iter (declare (iterate:declare-variables))
      (for i from 1 to 10)
      (collect i into var))
\end{program}
the variable |var| will be declared to be of type |list|.


\subsection{Summary}

\iter\ understands standard Common Lisp variable type declarations
that occur within an \iter\ form and
will pass them through to the generated code.  If the declaration
|(iterate:declare-variables)|\lispindex{declare-variables}
appears at the top level of an 
\iter\ form, or if 
|iterate::*always-declare-variables*|\lispindex{*always-declare-variables*} 
is \nonnil, then \iter\ will use the type information gleaned from user
declarations, self-evaluating expressions and |the| expressions,
combined with reasonable assumptions, to determine variable
types and declare them.


%%%%%%%%%%%%%%%%%%%%%%%%%%%%%%%%%%%%%%%%%%%%%%%%%%%%%%%%%%%%%%%%
\section{Problems with Code Movement}
\label{code-movement}

Some \iter\ clauses, or parts of clauses, result in code being
moved from the location of the clause to other parts of the loop.
Drivers behave this way, as do code-placement clauses like |initially|
and |finally|.  When using these clauses, there is a danger of writing
an expression that makes sense in its apparent location but will be
invalid or have a different meaning in another location.  For example:
\begin{program}
(iter (for i from 1 to 10)
      (let ((x 3))
        (initially (setq x 4))))
\end{program}
While it may appear that the |x| of |(initially (setq x 4))| is the
same as the |x| of |(let ((x 3)) \dots|, in fact they are not:
|initially| moves its code outside the loop body, so |x| would refer
to a global variable.  Here is another example of the same problem:
\begin{program}
(iter (for i from 1 to 10)
      (let ((x 3))
        (collect i into x)))
\end{program}
If this code were executed, |collect| would create a binding for its
|x| at the top level of the \iter\ form that the |let| will shadow.

Happily, \iter\ is smart enough to catch these errors; it
walks all problematical code to ensure that free variables are not
bound inside the loop body, and checks all variables it binds for the
same problem.

However, some errors cannot be caught:

\begin{program}
(iter (with x = 3)
      (for el in list)
      (setq x 1)
      (reducing el by \#'+ initial-value x))
\end{program}
|reducing| moves its |initial-value| argument to the initialization
part of the loop in order to produce more efficient code.  Since
\iter\ does not perform data-flow analysis, it cannot determine that
|x| is changed inside the loop; all it can establish is that |x| is
not bound internally.  Hence this code will not signal an
error and will use $3$ as the initial value of the reduction.

The following list summarizes all cases that are subject to these code
motion and variable-shadowing problems.
\begin{itemize}
\item Any variable for which \iter\ creates a binding, including those
used in |with| and the |into| keyword of many clauses.

\begin{sloppypar}
\item The special clauses which place code: |initially|, |after-each|, |else|,
|finally| and |finally-protected|. 
\end{sloppypar}

\item The variables of a |next| or |do-next| form.

\item The |initially| arguments of |for\dots initially\dots then| and
|for\dots previous|. 

\item The |then| argument of |for\dots initially\dots then|.

\item The |initial-value| arguments of |reducing| and |accumulate|.

\item The |on-failure| argument of |finding\dots such-that|.
\end{itemize}

%%%%%%%%%%%%%%%%%%%%%%%%%%%%%%%%%%%%%%%%%%%%%%%%%%%%%%%%%%%%%%%%
\section{Differences Between {\tt Iterate} and {\tt Loop}}
|loop| contains a great deal of complexity which \iter\ tries to
avoid.  Hence many esoteric features of |loop| don't exist in \iter.
Other features have been carried over, but in a cleaned-up form.  
And of course, many new features have been added; they are not
mentioned in this list.

\begin{itemize}

\item \iter's syntax is more Lisp-like than |loop|'s, having a higher
density of parens.

\item The current implementation of \iter, unlike the current version
of  |loop| (as documented in {\em Common Lisp, 2nd Ed.\/}), is
extensible (see section \ref{extend}).  

\item |loop| puts the updates of all driver variables at the top of
the loop; \iter\ leaves them where the driver clauses appear.

\item While for the most part \iter\ clauses that resemble |loop| clauses
behave similarly, there are some differences.  For instance, there is
no |for\dots =\dots then| in \iter; instead use 
|for\dots initially\dots then|.  

\item |loop| binds the variable |it| at certain times
to allow pseudo-English expressions like |when ~expr~ return it|. 
In \iter, you must bind ~expr~ to a variable yourself.  Note that
|when ~expr~ return it| is like |thereis ~expr~| except that the latter is an
accumulation clause and therefore competes with other accumulations
(remember section \ref{multiple} above).

% repeat different behaviour of |always| clause here?

\item |loop| has a special |return| clause, illustrated in the
previous item.  \iter\ doesn't need one,  since an ordinary Lisp
|return| has the same effect.

\item |loop| allows for parallel binding and stepping of iteration
variables.  \iter\ does not.  (See section \ref{bindings}.)

\item |loop| and \iter\ handle variable type declarations very
differently.  |loop| provides a special syntax for declaring variable
types, and does not examine declarations.  Moreover, the standard
implementation of |loop| will
generate declarations when none are requested.
\iter\ parses standard Common Lisp type declarations, and will never
declare a variable itself unless declarations are 
specifically requested.

\end{itemize}

%%%%%%%%%%%%%%%%%%%%%%%%%%%%%%%%%%%%%%%%%%%%%%%%%%%%%%%%%%%%%%%%
\section{Rolling Your Own}
\label{extend}

\subsection{Introduction}

\iter\ is extensible---you can write new clauses that embody new
iteration patterns.  You might want to write a new driver clause for a
data structure of your own, or you might want to write a clause that
collects or manipulates elements in a way not provided by \iter. 

This section describes how to write clauses for \iter.  Writing a
clause is like writing a macro.  In fact, writing a clause ~is~
writing a macro: since \iter\ code-walks its body and macroexpands,
you can add new abstractions to \iter\ with good old |defmacro|.

Actually, there are two extensions you can make to \iter\ that are
even easier than writing a macro.  They are adding a synonym for an
existing clause and defining a driver clause for an indexable
sequence.  These can be done with |defsynonym| and
|defclause-sequence|, respectively.  See section \ref{aids}, below.

The rest of this section explains how to write macros that expand into
\iter\ clauses.
Here's how you could add a simplified version of \iter's |multiply|
clause, if \iter\ didn't already have one:
\begin{program}
(defmacro multiply (expr)
  `(reducing ,expr by \#'* initial-value 1))
\end{program}

If you found yourself summing the square of an expression often, you
might want to write a macro for that.  A first cut might be
\begin{program}
(defmacro sum-of-squares (expr)
  `(sum (* ,expr ,expr)))
\end{program}
but if you are an experienced macro writer, you will realize that this
code will evaluate ~expr~ twice, which is probably a bad idea.  A
better version would use a temporary:
\begin{program}
(defmacro sum-of-squares (expr)
  (let ((temp (gensym)))
    `(let ((,temp ,expr))
       (sum (* ,temp ,temp)))))
\end{program}
Although this may seem complex, it is just the sort of thing you'd
have to go through to write any macro, which illustrates the point of
this section: if you can write macros, you can extend \iter.

Our macros don't use \iter's keyword-argument
syntax.  We could just use keywords with |defmacro|, but we would
still not be using \iter's clause indexing mechanism.  Unlike Lisp,
which uses just the first symbol of a form to determine what function
to call, \iter\ individuates clauses by the list of required keywords.
For instance, |for\dots in| and |for\dots in-vector| are different clauses
implemented by distinct Lisp functions.  

To buy into this indexing scheme, as well as the keyword-argument
syntax, use |defmacro-clause|:

\begin{clauses}

\defmacro{defmacro-clause}{arglist |\&body| body}
Defines a new \iter\ clause.  ~arglist~ is a list of symbols which are
alternating keywords and arguments.  \opt\ may be used, and the list
may be terminated by |\&sequence|.  ~body~ is an ordinary macro body,
as with |defmacro|.  If the first form of ~body~ is a string, it is
considered a documentation string and will be shown by
|display-iterate-clauses|. |defmacro-clause| will signal an error if 
defining the clause would result in an ambiguity.  E.g. you cannot
define the clause |for\dots from| because there would be no way to
distinguish it from a use of the |for| clause with optional keyword |from|.

\end{clauses}

\medskip

Here is |multiply| using |defmacro-clause|.  The keywords are capitalized
for readability.
\begin{program}
(defmacro-clause (MULTIPLY expr \opt\ INTO var)
  `(reducing ,expr by \#'* into ,var initial-value 1))
\end{program}
You don't have to worry about the case when |var| is not supplied; for
any clause with an |into| keyword, saying |into nil| is equivalent to
omitting the |into| entirely.

As another, more extended example, consider the fairly common
iteration pattern that 
involves finding the sequence element that maximizes (or minimizes) some
function.  \iter\ provides this as |finding\dots maximizing|, but it's
instructive to see how to write it.
Here, in pseudocode, is how you might write such a loop for
maximizing a function F:

\begin{quote}
\begin{tabbing}
set variable MAX-VAL to NIL; \\
set variable WINNER to NIL; \\
for\=\ each element EL in the sequence \\
\>  if\=\ MAX-VAL\=\ is NIL or F(EL) $>$ MAX-VAL then \\
\>\>    set MAX-VAL to F(EL); \\
\>\>    set WINNER to EL; \\
\>  end if; \\
end for; \\
return WINNER.
\end{tabbing}
\end{quote}

Here is the macro:
\begin{program}
(defmacro-clause (FINDING expr MAXIMIZING func \opt\ INTO var)
  (let ((max-val (gensym))
        (temp1 (gensym))
        (temp2 (gensym))
        (winner (or var iterate::*result-var*)))
    `(progn 
       (with ,max-val = nil)
       (with ,winner = nil)
       (cond 
        ((null ,max-val)
         (setq ,winner ,expr)
         (setq ,max-val (funcall ,func ,winner))
        (t
         (let* ((,temp1 ,expr)
                (,temp2 (funcall ,func ,temp1)))
           (when (> ,temp2 ,max-val)
             (setq ,max-val ,temp2)
             (setq ,winner ,temp1))))))
       (finally (leave ,winner)))))
\end{program}
Note that if no |into| variable is supplied, we use
|iterate::*result-var*|, which contains the internal variable into
which all clauses place their results.  If this variable is bound by
some clause, then \iter\ will return its value automatically;
otherwise, \nil\ will be returned.  

\subsection{Writing Drivers}

In principle, drivers can be implemented just as easily as other
\iter\ clauses.  In practice, they are a little harder to get right. 
As an example, consider writing a driver that
iterates over all the 
elements of a vector, ignoring its fill-pointer.  |for\dots in-vector|
won't work for this, because it observes the fill-pointer.  It's necessary to
use |array-dimension| instead of |length| to obtain the size of the
vector.  Here is one approach:

\begin{program}
(defmacro-clause (FOR var IN-WHOLE-VECTOR v)
  "All the elements of a vector (disregards fill-pointer)"
  (let ((vect (gensym))
        (index (gensym)))
    `(progn
       (with ,vect = ,v)
       (for ,index from 0 below (array-dimension ,vect 0))
       (for ,var = (aref ,vect ,index)))))
\end{program}
Note that we immediately put |v| in a variable, in case it is an
expression.  Again, this is just good Lisp macrology.  It also has a
subtle effect on the semantics of the driver: |v| is evaluated only
once, at the beginning of the loop, so changes to |v| in the loop have
no effect on the driver.  Similarly, the bounds for numerical iteration
e.g. the above |array-dimension| are also evaluated once only.  This is how
all of \iter's drivers work.


There is an important point concerning the |progn| in this code.  We
need the |progn|, of course, because we are returning several forms,
one of which is a driver.  But \iter\ drivers must occur at top-level.
Is this code in error?  No, because ~top-level~ is defined in \iter\
to include forms inside a |progn|.  This is just the definition of
top-level that Common Lisp uses, and for the same reason: to allow
macros to return multiple forms at top-level.

While our |for\dots in-whole-vector| clause will work, it is not
ideal.  In particular, it does not support generating.  Do do so, we
need to use |for\dots next| or |for\dots do-next|.  The job is
simplified by the |defmacro-driver| macro.

\begin{clauses}

\defmacro{defmacro-driver}{arglist |\&body| body}
Defines a driver clause in
both the |for| and |generate| forms, and provides a parameter
|generate| which ~body~ can examine to determine how it was invoked.
~arglist~ is as in |defmacro-clause|, and should begin with the symbol
|for|. 

\end{clauses}

With |defmacro-driver|, our driver looks like this:
\begin{program}
(defmacro-driver (FOR var IN-WHOLE-VECTOR v)
  "All the elements of a vector (disregards fill-pointer)"
   (let ((vect (gensym))
         (end (gensym))
         (index (gensym))
         (kwd (if generate 'generate 'for)))
     `(progn
        (with ,vect = ,v)
        (with ,end = (array-dimension ,vect 0))
        (with ,index = -1)
        (,kwd ,var next (progn (incf ,index)
                               (if (>= ,index ,end) (terminate))
                               (aref ,vect ,index))))))
\end{program}


We are still missing one thing: the |\&sequence| keywords.  
We can get them easily enough, by writing 
\begin{program}
(defmacro-driver (FOR var IN-WHOLE-VECTOR v \&sequence) 
  ...)
\end{program}
We can now refer to parameters |from|, |to|, |by|, etc. which contain
either the values for the corresponding keyword, or \nil\ if the
keyword was not supplied.  Implementing the right code for these
keywords is cumbersome but not difficult; it is left as an exercise.
But before you begin, see |defclause-sequence| below for an easier way.
       
\subsection{Extensibility Aids}
\label{aids}

This section documents assorted features that may be of use in
extending \iter.

\begin{clauses}

\defunexpvar{*result-var*}
Holds the variable that is used to return a value as a result of the
\iter\ form.  You may examine this and use it in a |with| clause, but
you should not change it.

\defmacro{defsynonym}{syn word}
 Makes ~syn~ a synonym for the existing \iter\ keyword ~word.~  Only
the first word in each clause can have synonyms. 

\defmacroxx{defclause-sequence}{element-name index-name}{|\&key|
access-fn size-fn sequence-type}{element-type
element-doc-string index-doc-string}  
Provides
a simple way to define sequence clauses.  Generates two
clauses, one for iterating over the sequence's elements, the other
for iterating over its indices.  The first symbol of both
clauses will have print-name |for|.
~element-name~  and ~index-name~ should be symbols.
~element-name~ is the second keyword of the element iterator (typically of
the form 
|in-~sequence-type~|), and ~index-name~ is the second keyword
of the index-iterator (typically of the form
|index-of-~sequence-type~|).  Either name may be 
\nil, in which case the corresponding clause is not defined.  If both
symbols are supplied, they should be in the same package.  The |for|
that begins the clauses will be in this package.

\cpar ~access-fn~ is the function to be used to
access elements of the sequence in the element iterator.  The function
should take two 
arguments, a sequence and an index, and return the appropriate element.
~size-fn~ should denote a function of one argument, a sequence, that
returns its size.  Both ~access-fn~ and ~size-fn~ are required for the
element iterator, but only ~size-fn~ is needed for the index iterator.

\cpar The ~sequence-type~ and ~element-type~ keywords are used to
suggest types for the variables 
used to hold the sequence and the
sequence elements, respectively.  The usual rules about \iter's
treatment of variable type declarations apply (see section \ref{types}).

\cpar ~element-doc-string~ and ~index-doc-string~ are
the documentation strings, for use with |display-iterate-clauses|. 

\cpar The generated element-iterator performs destructuring on the
element variable.

\cpar As an example, the above |for\dots in-whole-vector| example
could have been written:
\begin{program}
(defclause-sequence IN-WHOLE-VECTOR INDEX-OF-WHOLE-VECTOR
  :access-fn 'aref
  :size-fn \#'(lambda (v) (array-dimension v 0))
  :sequence-type 'vector
  :element-type t
  :element-doc-string 
     "Elements of a vector, disregarding fill-pointer"
  :index-doc-string 
     "Indices of vector, disregarding fill-pointer")
\end{program}

\end{clauses}

\subsection{Subtleties}

There are some subtleties to be aware of when writing \iter\ clauses.
First, the code returned by your macros may be |nconc|'ed into a list,
so you should always returned freshly consed lists, rather than
constants.  Second, \iter\ matches clauses by using |eq| on the first
symbol and |string=| on the subsequent ones, so the package of the
first symbol of a clause is relevant.  All of the clauses in this manual
have their first word in the \iter\ package.  
You can use the package system in the usual way to shadow
\iter\ clauses without replacing them.

%%% say more here, about the badness that only the first word of a
%%% clause is packagey.

\section{Non-portable Extensions to Iterate (Contribs)}
\label{contribs}

Currently, there is only one non-portable extension to iterate in the
distribution: iterate-pg. If you have made an extension that depends
on non-portable features, feel free to send them to |asf@boinkor.net|
for inclusion in the iterate distribution.

\subsection{An SQL query driver for iterate}

The pg package by Eric Marsden (see |http://cliki.net/pg|) provides an
interface to the PostgreSQL database. Using the \iterpg\ extension, it
is possible to handle the results of SQL queries using \iter.

This usage example should give you an idea of how to use it:

\begin{program}
(pg:with-pg-connection (c "somedb" "someuser")
  (iter (for (impl version date) in-relation "select * from version"
                                 on-connection *dbconn*)
        (collect version)))
\end{program}

To use the extension via |ASDF|, simply make your system depend on the
|iterate-pg| system instead of the |iterate| system. To load it
manually, use:

\begin{program}
  (asdf:oos 'asdf:load-op :iterate-pg)
\end{program}

%%%%%%%%%%%%%%%%%%%%%%%%%%%%%%%%%%%%%%%%%%%%%%%%%%%%%%%%%%%%%%%%
\section{Obtaining {\tt Iterate}}

\begin{sloppypar}
The information in this chapter is obsolete but included for
completeness's sake; Currently, the most up-to-date information on
\iter\ can be found at |http://boinkor.net/iterate.html|.
\end{sloppypar}

\begin{sloppypar}
\iter\ currently runs on Lisp Machines, and on
HP's, Sun3's and Sparcstations under Lucid.  
\iter\ source and binaries are available at the MIT AI Lab in the
subdirectories of |/src/local/lisplib/|.  The source file,
|iterate.lisp|, is also available for anonymous FTP in the directory
|/com/fpt/pub/| on the machine |TRIX.AI.MIT.EDU| (Internet number
128.52.32.6).  If you are unable to obtain |iterate| in one of these
ways, send mail to |jba@ai.mit.edu| and I will send you the source
file. 
\end{sloppypar}

\begin{sloppypar}
\iter\ resides in the |iterate| package (nickname |iter|).  Just say
\linebreak |(use-package :iterate)| to make all the necessary symbols
available. 
If a symbol is not exported, it appears in this manual with an
``|iterate::|'' prefix.
\end{sloppypar}

Send bug reports to |bug-iterate@ai.mit.edu|.  The |info-iterate|
mailing list will have notices of changes and problems; to have
yourself added, send mail to |info-iterate-request@ai.mit.edu|.


\medskip

\begin{flushleft}
 \bf Acknowledgements
\end{flushleft}
\smallskip

Richard Waters provided invaluable criticism which spurred me to improve
\iter\ greatly.  As early users, David Clemens, Oren Etzioni and Jeff
Siskind helped ferret out many bugs.

%\begin{theindex}
% The index files must be generated with the genindex program
% or from makeindex iterate-manual.idx which creates iterate-manual.ind.

%\baselineskip\normalbaselineskip
%\advance\baselineskip by -2pt

\input{iterate-manual.ind}

%\underline{Clauses}
%\input{iterate-manual.clindex}
%
%\indexspace
%\underline{Lisp}
%\input{iterate-manual.lispindex}

%\end{theindex}


\end{document}


% arch-tag: "c0f871a6-313c-11d8-abb9-000c76244c24"


%\underline{Clauses}
%% Fix ``.'' spacing in lists
\documentstyle[12pt,aimemo]{article}

\setlength{\textwidth}{6.0in}
\setlength{\oddsidemargin}{0.25in}
\setlength{\evensidemargin}{0.25in}
\setlength{\topmargin}{0pt}
\setlength{\headsep}{24pt}
\setlength{\textheight}{8.5in}

\let\origpar=\par

% somehow, LaTex fucks with things so that the plain Tex \obeyspaces
% doesn't work.  This is the same, except we use an mbox containing
% a space instead of just a space.

\newcommand{\spc}{\mbox{ }}
\newcommand{\tAb}{\mbox{        }}
\def\obspaces{\catcode`\ =\active\catcode`\^^I=\active}
{\obspaces \global\let =\spc\global\let^^I=\tAb}

\newenvironment{program}{\medskip\samepage\tt\obeylines\obspaces}{\medskip}

% use | for \tt
\catcode`\|=13
\newif\ifvbar
\def|{\ifvbar
         \endgroup\vbarfalse
      \else 
         \begingroup\tt\vbartrue 
      \fi}



\newcommand{\lisp}{\tt}
\newcommand{\iter}{|iterate|}
\newcommand{\iterpg}{|iterate-pg|}
\newcommand{\setf}{|setf|}
\newcommand{\nil}{|nil|}
\newcommand{\nonnil}{non-\nil}
\newcommand{\opt}{{\lisp \&optional}}
\newcommand{\yields}{$\Rightarrow$}



\newbox\Dots
\setbox\Dots=\hbox{\small ...}
\renewcommand{\dots}{\copy\Dots}

% Here's a way to define an environment which will treat all
% paragraphs inside it in the same manner.  It's, as near as I can
% figure, the way that Latex does things.  Using \everypar won't work;
% apparently Latex resets that a lot.  What you have to do is: at the
% top of the file do \let\origpar=\par.  In the
% begin part of the environment, do the parshape (or whatever), then say
% \def\par{{\origpar}}.  The extra braces are necessary.   That's all
% you have to do; at the end of the environment things will return to
% normal automatically.  Why does this all work?  I have no idea.
% Another important thing: there should be a blank line (i.e. a \par)
% between the end of the last paragraph and the \end{environment}.

\newlength{\clindent}
\setlength{\clindent}{0.5in}
\newlength{\clparindent}
\setlength{\clparindent}{\parindent}

\newenvironment{clauses}{\advance\linewidth -\parindent
                        \hangindent\clindent\relax
                         \hangafter=0
                          \parindent=0pt
                          \def\par{{\origpar}}}{}

\newcommand{\cpar}{\hspace \clparindent}

\newcommand{\startitem}{\bigskip\pagebreak[3]}
\newcommand{\finishitem}{\smallskip}

\newcommand{\clause}[1]{\startitem\Clause{#1}\finishitem}
\newcommand{\Clause}[1]{\hbox{#1}}
\newcommand{\clindex}[1]{\index{{\lisp #1}}}
\newcommand{\clindexx}[2]{\index{{\lisp #1\dots #2}}}
\newcommand{\clindexxx}[3]{\index{{\lisp #1\dots #2\dots #3}}}
\newcommand{\clausex}[2]{\startitem\Clausex{#1}{#2}\finishitem}
\newcommand{\Clausex}[2]{\Clause{|#1| #2}\clindex{#1}}

\newcommand{\defvar}[1]{\startitem\deff{#1}{}{Variable}\finishitem}
\newcommand{\defunexpvar}[1]{\startitem\deffnoindex{iterate::#1}{}{Variable}
                                        \lispindex{#1}\finishitem}
\newcommand{\defun}[2]{\startitem\deff{#1}{#2}{Function}\finishitem}
\newcommand{\defmacro}[2]{\startitem\deff{#1}{#2}{Macro}\finishitem}
\newcommand{\defunx}[3]{\startitem\deffx{#1}{#2}{#3}{Function}\finishitem}
\newcommand{\defmacrox}[3]{\startitem\deffx{#1}{#2}{#3}{Macro}\finishitem}
\newcommand{\defunxx}[4]{\startitem\deffxx{#1}{#2}{#3}{#4}{Function}\finishitem}
\newcommand{\defmacroxx}[4]{\startitem\deffxx{#1}{#2}{#3}{#4}{Macro}\finishitem}

\newcommand{\deff}[3]{\deffnoindex{#1}{#2}{#3}\lispindex{#1}}
\newcommand{\deffnoindex}[3]{\hbox to \hsize{{\tt #1} {\it #2}\hfill [{\it #3}]}}
\newcommand{\deffx}[4]{\deff{#1}{#2}{#4}\moreargs{#1}{#3}}
\newcommand{\deffxx}[5]{\deffx{#1}{#2}{#3}{#5}\moreargs{#1}{#4}}

\newcommand{\moreargs}[2]{\hbox to \hsize{\phantom{\lisp #1} {\it #2}\hfill}}

\newcommand{\lispindex}[1]{\index{{\lisp #1}}}

 \newenvironment{note}[1]{\pagebreak[2]\bigskip
                          \hrule\smallskip\small
                         {\setlength{\parindent}{0pt}
                           \par{\bf #1:}}}{\smallskip\hrule\bigskip}



\makeindex

%%%%%%%%%%%%%%%%%%%%%%%%%%%%%%%%%%%%%%%%%%%%%%%%%%%%%%%%%%%%%%%%

%\thispagestyle{empty}

%\vspace*{2in}

%\title{The |Iterate| Manual}
%\author{Jonathan Amsterdam}

%\maketitle

%\thispagestyle{empty}

%\begin{center}
%\Large *** DRAFT ***
%\end{center}
%This manual is in beta-test. It will become an AI Memo in a couple of
%months.  I would appreciate any comments or
%corrections.  You can reach me at email address |jba@ai.ai.mit.edu|.

\begin{document}

\def\writtenby{Jonathan Amsterdam}
\def\contractno{This report describes research done at the Artificial
Intelligence Laboratory of the Massachusetts Institute of Technology.
Support for the laboratory's artificial intelligence research is
provided in part by the Advanced Research Projects Agency of the
Department of Defense under Office
of Naval Research contract N00014-85-K-0124.}

\AIMEMO{1236}{\Large The Iterate Manual \\[1in]}
{This is the manual for version 1.4 of \iter, a powerful iteration
macro for Common  
Lisp. \iter\ is similar to {\tt loop} but provides numerous additional
features, is well integrated with Lisp, and is extensible. 
\vfill }


% use ~ for \em
\catcode`\~=13
\newif\ifem
\def~{\ifem
         \/\endgroup\emfalse
      \else 
         \begingroup\em\emtrue 
      \fi}



%\pagebreak
%\setcounter{page}{1}

%\thispagestyle{empty}


\tableofcontents

\pagebreak
\pagestyle{headings}

\section{Introduction}

\begin{sloppypar}
This manual describes
\iter, a powerful iteration facility for Common Lisp.
\iter\ provides abstractions for many common iteration
patterns and allows for the definition of additional patterns.  \iter\ is
a macro that expands into ordinary Lisp at
compile-time, so it is more efficient than higher-order
functions like |map| and |reduce.|  
While it is similar to |loop|, \iter\ offers a more Lisp-like syntax
and enhanced extensibility.
(For a more complete
comparison of \iter\ with other iteration constructs, see MIT AI Lab Working
Paper No. 324, ~Don't Loop, Iterate.~)
\end{sloppypar}

An \iter\ form consists of the symbol |iter|\footnote{You can also use
|iterate|, but |iter| is preferred because it avoids potential conflicts with
possible future additions to Common Lisp, and because it saves
horizontal space when writing code.}
followed by one or more
forms, some of which may be \iter\ 
~clauses.~  Here is a simple example of \iter\ which collects the
numbers from 1 to 10 into a list, and returns the list.  The return
value is shown following the arrow.
\begin{program}
(iter (for i from 1 to 10)
      (collect i))          \yields (1 2 3 4 5 6 7 8 9 10)
\end{program}
This form contains two clauses: a |for| clause that steps the
variable |i| over the integers from 1 to 10, and a |collect|
clause that accumulates its argument into a list.  With a few
exceptions, all \iter\ clauses have the same format: alternating
symbols (called ~keywords~) and expressions (called
~arguments~).  The syntax and terminology are those of Common
Lisp's keyword lambda lists.  One difference is that \iter's keywords
do not have to begin with a colon---though they may, except for the
first symbol of a clause.  So you can also write  |(for i :from
1 :to 10)| if you prefer.

Any Lisp form can appear in the body of an \iter, where it will have
its usual meaning.  \iter\ walks the entire body, expanding macros,
and recognizing clauses at any level.  This example collects all the
odd numbers in a list:
\begin{program}
(iter (for el in list)
      (if (and (numberp el) (oddp el))
          (collect el)))
\end{program}

There are clauses for iterating over numbers, lists, arrays and other
objects, and for
collecting, summing, counting, maximizing and other useful operations.
\iter\ also supports the creation of new variable bindings, stepping
over multiple sequences at once, destructuring, and compiler
declarations of variable types.  The following example illustrates
some of these features: 
\begin{program}
(iter (for (key . item) in alist)
         (for i from 0)
         (declare (fixnum i))
         (collect (cons i key)))
\end{program}

This loop takes the keys of an alist and returns a new alist
associating the keys with their positions in the original list. The
compiler declaration for |i| will appear in the generated code
in the appropriate place.

%%%%%%%%%%%%%%%%%%%%%%%%%%%%%%%%%%%%%%%%%%%%%%%%%%%%%%%%%%%%%%%%
\section{Clauses}

Most of \iter's clauses will be familiar to |loop| programmers.  
(|loop| is an iteration macro that has been incorporated into Common
Lisp.  Seen Guy Steele's {\em Common Lisp, 2nd Edition}.)
In
nearly all cases they behave the same as their |loop| counterparts, so
a |loop| user can switch to \iter\ with little pain (and much gain).

All clauses with the standard keyword-argument syntax consist of two
parts: a ~required~ part, 
containing keywords that must be present and in the right order; and
an ~optional~ part, containing keywords that may be omitted and,
if present, may occur in any order.  In the descriptions below, the
parts are separated by the Lisp lambda-list keyword \opt.


\subsection{Drivers}

An iteration-driving clause
conceptually causes the iteration to go forward.  Driver clauses in
\iter\ allow iteration over numbers, lists, vectors, hashtables, packages,
files and streams.  Iteration-driving clauses must
appear at the top level of an \iter\ form; they cannot be nested
inside another clause.  The driver variable is updated at the point
where the driver clause occurs.  Before the clause is executed for the
first time, the value of the variable is undefined.

%Also, regardless of where the driver clause
%appears in the body, the driver variable is stepped at the top of the
%loop; hence it is stylistically preferable, though
%not required, to place driver clauses at the beginning of the \iter.

Multiple drivers may appear in a single \iter\ form, in which case all
of the driver variables are updated each time through the loop, in the
order in which the clauses appear.  The first driver to terminate will
terminate the entire loop.

In all cases, the value of the driver variable on exit from the loop,
including within the epilogue code (see the |finally| clause), is
undefined.

All the parameters of a driver clause are evaluated once, before the
loop begins.  Hence it is not possible to change the bounds or other
properties of an iteration by side-effect from within the loop.

With one exception, driver clauses begin with the word |for| (or
the synonym |as|) and mention an iteration variable, which is
given a binding within the \iter\ form.
The exception is |repeat|, which just executes a loop a
specified number of times:

\begin{clauses}

\clausex{repeat}{~n~}
Repeats the loop ~n~ times.  For example:
\begin{program}
(iter (repeat 100)
      (print "I will not talk in class."))
\end{program}
\cpar If $n \leq 0$, the
loop will never be executed.  If ~n~ is not an integer, the
actual number of executions will be $\lceil n \rceil$.

\end{clauses}

\pagebreak[3]

\subsubsection{Numerical Iteration}

\begin{clauses}
\clausex{for}{~var~ |\&sequence|} 
The general form for iterating over a sequence of numbers requires a
variable and, optionally, one or more keywords that provide the bounds
and step size of the iteration. The |\&sequence|
lambda-list keyword is a shorthand for these sequence keywords.  They are:
|from|, |upfrom|, |downfrom|, |to|, |downto|, |above|, |below| and |by|.
|from| provides the starting value for ~var~
and defaults to zero. |to| provides a final value and implies that the
successive values of ~var~ will be increasing; |downto| implies that
they will be decreasing.  The loop terminates when ~var~ passes
the final value (i.e. becomes smaller or larger than it, depending on
the direction of iteration); in other  words, the loop body
will never be 
executed for values of ~var~ past the final value.
|below| and |above| are similar to |to| and
|downto|, except that the loop terminates when ~var~ equals or passes
the final value.

\cpar If no final value is specified, the variable will be stepped
forever.  Using |from| or |upfrom| will result in increasing
values, while |downfrom| will give decreasing values.

\cpar On each iteration,
~var~ is incremented or decremented by the value of the sequence
keyword |by|, which defaults to 1.  It should always be a positive
number, even for downward iterations.

\cpar In the following examples, the sequence of numbers generated is shown
next to the clause.

\begin{program}
(for i upfrom 0) \yields\ 0 1 2 \ldots
(for i from 5)   \yields\ 5 6 7 \ldots    ; either from or upfrom is okay
(for i downfrom 0) \yields\ 0 -1 -2 \ldots
(for i from 1 to 3) \yields\ 1 2 3
(for i from 1 below 3) \yields\ 1 2
(for i from 1 to 3 by 2) \yields\ 1 3
(for i from 1 below 3 by 2) \yields\ 1
(for i from 5 downto 3) \yields\ 5 4 3
\end{program}

\end{clauses}

\subsubsection{Sequence Iteration}

There are a number of clauses for iterating over sequences.  In all of
them, the argument following |for| may be a list instead of a
symbol, in which case destructuring is performed.  See section
\ref{destructuring}. 

\begin{clauses}

\clause{|for| ~var~ |in| ~list~ \opt\ |by| ~step-function~}
\clindexx{for}{in}
~var~ is set to successive elements of list.  
~step-function,~ which 
defaults to |cdr|, is used to obtain the next sublist. 

\clause{|for| ~var~ |on| ~list~ \opt\ |by| ~step-function~}
\clindexx{for}{on}
~var~ is set to successive sublists of list. 
~step-function~ (default |cdr|) is used as in |for\dots in|.

\end{clauses}

\medskip

These two clauses use |atom| to test for the end
of a list.  Hence, given a list whose final |cdr| is not \nil,
they will silently ignore the last |cdr|.  Other choices are
|endp|,  which would signal an error, and |null|, which
would probably result in an error somewhere else.  If you wish to use
an end-test other than |atom|, set the variable
|iterate::*list-end-test*|\lispindex{*list-end-test*} to the name of
the desired function. 

\begin{clauses}
\clause{|for| ~var~ |in-vector| ~vector~ |\&sequence|}
\clindexx{for}{in-vector}
~var~ takes on successive elements from ~vector.~   The vector's
fill-pointer is observed.  Here and in subsequent clauses, the |\&sequence|
keywords include |with-index|, which takes a symbol as argument
and uses it for the index variable instead of an internally generated
symbol.  The other |\&sequence| keywords behave as in numerical
iteration, except that the default iteration bounds are the bounds of
the vector.  E.g. in |(for i in-vector v downto 3)|,
|i| will start off being bound to the last element in |v|,
and will  
be set to preceding elements down to and including the element with
index 3.

\clause{|for| ~var~ |in-sequence| ~seq~ |\&sequence|}
\clindexx{for}{in-sequence}
This uses Common Lisp's generalized sequence functions, |elt|
and |length|, to obtain elements and determine the length of
~seq.~   Hence it will work for any sequence, including lists, and will
observe the fill-pointers of vectors.

\clause{|for| ~var~ |in-string| ~string~ |\&sequence|}
\clindexx{for}{in-string}
~var~ is set to successive characters of ~string.~

\startitem
\Clause{|for| ~var~ |index-of-vector| ~vector~ |\&sequence|}
\clindexx{for}{index-of-vector}
\Clause{|for| ~var~ |index-of-sequence| ~sequence~ |\&sequence|}
\clindexx{for}{index-of-sequence}
\Clause{|for| ~var~ |index-of-string| ~string~ |\&sequence|}
\clindexx{for}{index-of-string}
\finishitem
~var~ is set to successive indices of the sequence.
These clauses avoid the overhead of accessing the sequence elements
for those applications where they do not need to be examined, or are
examined rarely.  They admit all the optional keywords of the other
sequence drivers except the (redundant) |with-index| keyword.

\clause{|for| ~(key value)~ |in-hashtable| ~table~}
\clindexx{for}{in-hashtable}

~key~ and ~value,~ which must appear as shown in a list
and may be destructuring templates, are set to the keys and values of
~table~.  If ~key~ is |nil|, then the hashtable's keys will be ignored;
similarly for ~value~.
The order in which 
elements of ~table~ will be retrieved is unpredictable.

\clause{|for| ~var~ |in-package| ~package~ \opt\ |external-only| ~ext~}
\clindexx{for}{in-package}
Iterates over all the symbols in ~package,~ or over only the
external symbols if ~ext~ is specified and non-|nil|.  ~ext~ is not
evaluated.  The same symbol may appear more than once.

\clause{|for| ~var~ |in-packages| ~packages~ \opt\ |having-access| ~symbol-types~}
\clindexx{for}{in-packages}
Iterates over all the symbols from the list of packages denoted by the
descriptor ~packages~ and having accessibility (or visibility) given by
~symbol-types~.  This defaults to the list |(:external :internal :inherited)|
and is not evaluated.  ~var~ must be a list of up to three variables: in each
iteration, these will be set to a symbol, its access-type and package (as per
|with-package-iterator| in ANSI-CL).  The same symbol may appear more than
once.

\end{clauses}

\begin{clauses}

\clause{|for| ~var~ |in-file| ~name~ \opt\ |using| ~reader~}
\clindexx{for}{in-file}
Opens the file ~name~ (which may be a string or pathname) for input,
and iterates 
over its contents. ~reader~ defaults to |read|, so by default ~var~
will be bound to the successive forms in the file.  The \iter\ body is
wrapped in an unwind-protect to ensure that the file is closed no
matter how the \iter\ is exited.

\clause{|for| ~var~ |in-stream| ~stream~ \opt\ |using| ~reader~}
\clindexx{for}{in-stream}
Like |for\dots in-file|, except that ~stream~ should be an existing
stream object that supports input operations.

\end{clauses}

\subsubsection{Generalized Drivers}

These are primarily useful for writing drivers that can also be used
as generators (see section \ref{generators}, below).
 
\begin{clauses}

\clause{|for| ~var~ |next| ~expr~}
\clindexx{for}{next}

~var~ is set to ~expr~ each time through the loop.  Destructuring is
performed.  When the clause is used as a generator, ~expr~ is the code
that is executed when |(next ~var~)| is 
encountered (see section \ref{generators}, below).
~expr~ should compute the first value for ~var~, as well
as all subsequent values, and is responsible for terminating the loop.
For compatibility with future versions of \iter, this termination
should be done with |terminate|\clindex{terminate}, which can be
considered a synonym for |finish| (see section \ref{control-flow}).

\cpar As an example, the following clauses are equivalent to |(for
i from 1 to 10)|:
\begin{program}
(initially (setq i 0))
(for i next (if (> i 10) (terminate) (incf i)))
\end{program}

\clause{|for| ~var~ |do-next| ~form~}
\clindexx{for}{do-next}
~form~ is evaluated each time through the loop.  Its value is ~not~
set to ~var~; that is ~form's~ job. ~var~ is only present so that
\iter\ knows it is a driver variable.  \linebreak |(for ~var~ next
~expr~)| is equivalent to |(for ~var~ do-next (dsetq ~var~ ~expr~))|.
(See section \ref{destructuring} for an explanation of |dsetq|.) 

\end{clauses}

\subsubsection{Generators}
\label{generators}

In all of the above clauses, the driver variable is updated on each
iteration.
Sometimes it is desirable to have greater control over updating.  
For instance, consider the problem of associating numbers, in
increasing order and with no gaps, with the
\nonnil\ elements of a list.  One obvious first pass at writing this is:

\begin{program}
(iter (for el in list)
      (for i upfrom 1)
      (if el (collect (cons el i))))
\end{program}
But on the list |(a b nil c)| this produces |((a . 1) (b . 2) (c .
4))| instead of the desired |((a . 1) (b . 2) (c . 3))|.  The problem
is that |i| is incremented each time through the loop, even when |el|
is |nil|.  

The problem could be solved elegantly if we could step |i| only when
we wished to.  This
can be accomplished for any \iter\ driver by writing |generate| (or
its synonym |generating|)
instead of |for|.  Doing so produces a ~generator~---a driver whose
values are yielded explicitly.  To obtain the next value of a
generator variable ~v~, write \linebreak |(next ~v~)|.  The value of a |next|
form is the next value of ~v~, as determined by its associated driver
clause.  |next| also has the side-effect of updating ~v~ to that
value.  If there is no next value, |next| will terminate the loop,
just as with a normal driver.


Using generators, we can now write our example like this:
\begin{program}
(iter (for el in list)
      (generate i upfrom 1)
      (if el (collect (cons el (next i)))))
\end{program}
Now |i| is updated only when |(next i)| is executed, and this occurs
only when |el| is \nonnil.

To better understand the relationship between ordinary drivers and
generators, observe that we can rewrite an ordinary driver using its
generator form immediately followed by |next|, as this example shows:
\begin{program}
(iter (generating i from 1 to 10)
      (next i)
      ...)
\end{program}
Provided that the loop body contains no |(next i)| forms, this will
behave just as if we had written |(for i from 1 to 10)|.

We can still refer to a driver variable ~v~ without using |next|; in
this case, its value is that given to it by the last evaluation of
|(next ~v~)|.  Before |(next ~v~)| has been called the first time, the
value of ~v~ is undefined.  

This semantics is more flexible than
one in which ~v~ begins the loop bound to its first value and calls
of |next| supply subsequent values, because it means the loop will not
terminate too soon if the generator's sequence is empty.  For
instance, consider the following code, which tags \nonnil\ elements of
a list using a list of tags, and also counts the null elements.
(We assume there are at least as many tags as \nonnil\ elements.)
\begin{program}
(let* ((counter 0)
       (tagged-list (iter (for el in list)
                          (generating tag in tag-list) 
                          (if (null el)
                              (incf counter)
                              (collect (cons el (next tag)))))))
  ...)
\end{program}
It may be that there are just as many tags as non-null elements of
|list|.  If all the elements of |list| are null, we still want the
counting to proceed, even though |tag-list| is \nil.  If |tag| had to be
assigned its first value before the loop begins, we would have had to
terminate the loop before the first iteration, since when |tag-list|
is \nil, |tag| 
has no first value.  With the existing semantics, however, |(next
tag)| will never execute, so the iteration will cover all the elements of
|list|. 
        
When the ``variable'' of a driver clause is actually a destructuring
template containing several variables, all the variables are eligible
for use with |next|.  As before, |(next ~v~)| evaluates to ~v's~ next
value; but the effect is to update all of the template's variables.
For instance, the following code will return the list |(a 2 c)|.
\begin{program}
(iter (generating (key . item) in '((a . 1) (b . 2) (c . 3)))
      (collect (next key))
      (collect (next item)))
\end{program}

Only driver clauses with variables can be made into generators.  This
includes all clauses mentioned so far except for |repeat|.  It does
~not~ include |for\dots previous|, |for\dots =|,
|for\dots initially\dots then| or |for\dots first\dots then| (see
below). 

\subsubsection{Previous Values of Driver Variables}

Often one would like to access the value of a variable on a previous
iteration.  \iter\ provides a special clause for accomplishing this.

\begin{clauses}

\clause{|for| ~pvar~ |previous| ~var~ \opt\ |initially| ~init~
|back| ~n~}
\clindexx{for}{previous}
Sets ~pvar~ to the previous value of ~var,~ which should be a driver
variable, a variable from another |for\dots previous| clause, or a
variable established by a |for\dots =|, 
|for\dots initially\dots then| or |for\dots first\dots then| clause
(see section \ref{setting}).
Initially, ~pvar~ is given the value ~init~ (which defaults to \nil).  
The ~init~ expression will be moved outside the loop body, so it
should not depend on anything computed within the loop.
~pvar~ retains the value of ~init~ until ~var~ is set to its second
value, at which point ~pvar~ is set to ~var's~ first value; and so on.  

\cpar The
argument ~n~ to |back| must be a constant, positive integer, and
defaults to 1.  It determines how many iterations back ~pvar~ should
track ~var~.  For example, when ~n~ is 2, then ~pvar~ will be assigned
~var's~ first value when ~var~ is set to its third value.

\cpar A |for\dots previous| clause may occur after or before its
associated driver clause. |for\dots previous| works with generators as
well as ordinary drivers.

\pagebreak[3]

\cpar Example:
\begin{program}
(iter (for el in '(1 2 3 4))
      (for p-el previous el)
      (for pp-el previous p-el initially 0)
      (collect pp-el))
\end{program}
This evaluates to |(0 0 1 2)|.  It could have been written more
economically as 
\begin{program}
(iter (for el in '(1 2 3 4))
      (for pp-el previous el back 2 initially 0)
      (collect pp-el))
\end{program}


\end{clauses}

\subsection{Variable Binding and Setting}
\label{setting}

Several clauses exist for establishing new variable bindings or for
setting variables in the loop.  They all support destructuring.

\begin{clauses}

\clausex{with}{~var~ \opt\ |=| ~value~}
Causes ~var~ to be bound to value before the loop body is entered.
If ~value~ is not supplied, ~var~ assumes a default
binding, which will be 
\nil\ in the absence of declarations.  Also, if ~value~ is not
supplied, no destructuring is performed; instead, ~var~ may be a list of
symbols, all of which are given default 
bindings.  If ~value~ is supplied, ~var~ is bound to it, with
destructuring.  

\cpar Because |with|
creates bindings whose scope includes the entire \iter\ form, it is
good style to put all |with| clauses at the beginning.

\cpar Successive occurrences of |with| result in sequential
bindings (as with 
|let*|).  There is no way to obtain parallel bindings; see
section \ref{bindings} for a rationale.  


\clause{|for| ~var~ |=| ~expr~}
\clindexx{for}{=}
On each iteration, ~expr~ is evaluated and ~var~ is set
to its value. 

\cpar This clause may appear to do the same thing as |for\dots next|.
In fact, they are quite different.  |for\dots =| provides only three
services: it sets up a binding for ~var~, sets it to ~expr~ on each
iteration, and makes it possible to use |for\dots previous| with
~var~.  |for\dots next| provides these services in addition to the
ability to turn the driver into a generator.
%Also, the code which sets ~var~ appears in the loop body in the same
%place as the |for\dots =| clause; the code for |for\dots next| appears
%at the top of the loop, as with other drivers (except when being used
%as a generator).

\clause{|for| ~var~ |initially| ~init-expr~ |then| ~then-expr~}
\clindexxx{for}{initially}{then}
Before the loop begins, ~var~ is set to ~init-expr;~ on all
iterations after the first it is set to ~then-expr.~  This clause
must occur at top-level.  ~init-expr~ will be moved outside the loop
body and ~then-expr~ will be moved to the end of the loop body, so
they are subject to code motion problems (see section
\ref{code-movement}). 

\cpar This clause may appear to be similar to |for\dots next|, but in
fact they differ significantly.  |for\dots initially\dots then| is
typically used to give ~var~ its first value before the loop begins,
and subsequent values on following iterations.  This is incompatible
with generators, whose first value and subsequent values must all be
computed by |(next ~var~)|.  Also, the update of ~var~ in 
|for\dots initially\dots then| does not occur at the location of the
clause. 
Use |for\dots initially\dots then| for
one-shot computations where its idiom is more convenient, but use
|for\dots next| for extending \iter\ with new drivers (see section
\ref{extend}).

\clause{|for| ~var~ |first| ~first-expr~ |then| ~then-expr~}
\clindexxx{for}{first}{then}
The first time through the loop, ~var~ is set to ~first-expr;~ on
subsequent iterations, it is set to ~then-expr.~  This differs from
|for\dots initially| in that ~var~ is set to ~first-expr~
inside the loop body, 
so ~first-expr~ may depend on the results of other clauses.  For
instance, 
\begin{program}
(iter (for num in list)
      (for i first num then (1+ i))
      ...)
\end{program}
will set |i| to the first element of |list| on the first
iteration, whereas 
\begin{program}
(iter (for num in list)
      (for i initially num then (1+ i))
      ...)
\end{program}
is probably erroneous; |i| will be bound to |num|'s
default binding (usually \nil) for the first iteration.

\end{clauses}

\begin{note}{Compatibility note}
|loop|'s |for\dots =| works like \iter's, but |loop| used the syntax
|for\dots =\dots then| to mean |for\dots initially\dots then|.  It was
felt that these two operations were sufficiently different to warrant
different keywords.

Also, the |for| in the above three clauses is misleading,
since none is true driver (e.g. none has a corresponding |generate|
form).  |setting| would have been a better choice, but |for| was used
to retain some compatibility with |loop|.  
\end{note}


%%%%%%%%%%%%%%%%%%%%%%%%%%%%%%%%
\subsection{Gathering Clauses}
Many of \iter's clauses accumulate values into a variable, or set a
variable under certain conditions.  At the end of the loop, this
variable contains the desired result.  All these clauses have an
optional |into| keyword, whose argument should be a symbol.  If the
|into| keyword is not supplied, the accumulation variable will be
internally generated and its value will be returned at the end of the
loop; if a variable is specified, that variable is used for the
accumulation, and is not returned as a result---it is up to the user
to return it explicitly, in the loop's epilogue code (see |finally|).
It is safe to examine the accumulation variable during the loop, but
it should not be modified.

These clauses all begin with a verb.  When the verb does
not conflict with an existing Common Lisp function, then it may be
used in either its infinitival or present-participle form (e.g. |sum|,
|summing|).  However, when there is a conflict with Common Lisp, only
the present-participle form may be used (e.g. |unioning|).  This is to
prevent \iter\ clauses from clashing with Common Lisp
functions. 

%although these clauses are described as ``producing a value,'' it is a
%mistake to think of the lisp list representing the clause as a
%value-producing form in the usual way.  clauses may legally be written
%where a value is expected, e.g. |(setq x (sum i))|, but the
%lisp value of a clause in such a context is undefined.

\subsubsection{Reductions}

~Reduction~ is an extremely common iteration pattern in which
the results of successive applications of a
binary operation are accumulated.
For example, a loop that computes the sum of the
elements of a list is performing a reduction with the addition
operation.  This could be written in Common Lisp as  |(reduce \#'+
list)| or with \iter\ as
\begin{program}
(iter (for el in list)
      (sum el))
\end{program}

\begin{clauses}

\clausex{sum}{~expr~ \opt\ |into| ~var~}
Each time through the loop, ~expr~ is evaluated and added to a
variable, which is bound initially to zero.  If ~expr~ has a type,
it is ~not~ used as the type of the sum variable, which is always
|number|.  To get the result variable to be of a more specific
type, use an explicit variable, as in
\begin{program}
(iter (for el in number-list)
      (sum el into x)
      (declare (fixnum x))
      (finally (return x)))
\end{program}

\clausex{multiply}{~expr~ \opt\ |into| ~var~}
Like |sum|, but the initial value of the result variable is $1$,
and the variable is updated by multiplying ~expr~ into it.

\clausex{counting}{~expr~ \opt\ |into| ~var~}
~expr~ is evaluated on each iteration.  If it is \nonnil, the
accumulation variable, initially zero, is incremented.

\startitem
\Clausex{maximize}{~expr~ \opt\ |into| ~var~}
\Clausex{minimize}{~expr~ \opt\ |into| ~var~}
\finishitem
~expr~ is evaluated on each iteration and its extremum (maximum or
minimum) is stored in the accumulation variable.  If ~expr~ is never
evaluated, then the result is \nil\ (if the accumulation variable is
untyped) or $0$ (if it has a numeric type).

\clausex{reducing}{~expr~ |by| ~func~ \opt\ |initial-value| ~init-val~ |into| ~var~}
This is a general way to perform reductions. ~func~ should be a
function of two arguments, the first of which will be the value
computed so far and
the second of which will be the value of ~expr~.  It should return the
new value.
|reducing| is roughly equivalent to the
Common Lisp |(reduce ~func~ ~sequence~ :key ~expr-function~)|, where
~expr-function~ is used to derive values from the successive elements
of ~sequence~.

\cpar If the |reducing| clause is never executed, the result is
undefined.

\cpar It is not necessary to provide an
initial value, but better code can be generated if one is supplied.
Regardless of its location in the \iter\ body, ~init-val~ will be
evaluated before the loop is entered, so it should not depend on any
value computed inside the \iter\ form.  

%\cpar if a ~var~ is not specified, you can get \iter\ to declare the
%type of the internal variable by putting a |the| expression around
%~func.~  see section \ref{types}.

\end{clauses}

%\begin{note}{implementation note}
%in principle, |maximize| and |minimize| can be thought of
%as reductions where the initial value is the smallest (or largest)
%value that the accumulation variable can assume.  because lisp's
%bignums can represent arbitrary integers, these clauses cannot be
%implemented as reductions in general.  if, however, the type of 
%~expr~ or ~var~ can be determined to be a fixnum
%or a float, \iter\ will implement the clause as a true reduction,
%using one of the constants |most-negative-fixnum|, |%most-positive-fixnum|, 
%|most-negative-short-float|, etc. as appropriate.
%\end{note}

\subsubsection{Accumulations}

All the predefined accumulation clauses add values to a sequence.  If
the sequence is a list, they all 
have the property that the partial list is kept in the correct order
and available for inspection at any point in the loop.

\begin{clauses}

\clausex{collect}{~expr~ \opt\ |into| ~var~ |at| ~place~ |result-type| ~type~}
Produces a sequence of the values of ~expr~ on each iteration. ~place~
indicates where the next value of ~expr~ is added to the list and may
be one of the symbols |start|, |beginning| (a synonym for |start|) or
|end|.  The symbol may be quoted, but need not be.  The default is
|end|.  For example,
\begin{program}
(iter (for i from 1 to 5)
      (collect i))
\end{program}
produces |(1 2 3 4 5)|, whereas
\begin{program}
(iter (for i from 1 to 5)
      (collect i at beginning))
\end{program}
produces |(5 4 3 2 1)| (and is likely to be faster in most Common Lisp
implementations).

\cpar If ~type~ is provided, it should be a subtype of |sequence|.
The default is |list|.  Specifying a type other than |list| will
result in |collect| returning a sequence of that type.  ~However,~ the
type of the sequence being constructed when inside the loop body is
undefined when a non-|list| type is specified.  (As with ~place~,
quoting ~type~ is optional.)

\clausex{adjoining}{~expr~ \opt\ |into| ~var~ |test| ~test~ |at| ~place~ |result-type| ~type~}
Like |collect|, but only adds the value of ~expr~ if it is not
already present.  ~test,~ which defaults to |\#'eql|, is
the test to be used with |member|.

\startitem
\Clausex{appending}{~expr~ \opt\ |into| ~var~ |at| ~place~}
\Clausex{nconcing}{\ ~expr~ \opt\ |into| ~var~ |at| ~place~}
\Clausex{unioning}{\ ~expr~ \opt\ |into| ~var~ |test| ~test~ |at| ~place~}
\Clausex{nunioning}{~expr~ \opt\ |into| ~var~ |test| ~test~ |at| ~place~}
\finishitem
These are like |collect|, but behave like the Common Lisp functions
|append|, |nconc|, |union| or |nunion|.  
As in Common Lisp, they work only on lists.  Also as in Common Lisp,
|unioning| and |nunioning| assume that the value of ~expr~ contains no
duplicates. 

\clausex{accumulate}{~expr~ |by| ~func~ \opt\ |initial-value| ~init-val~ |into| ~var~}
This is a general-purpose accumulation clause. ~func~ should be a
function of two arguments, the value of ~expr~ and the value
accumulated so far in the iteration, and it should return the updated
value.  If no initial value is supplied, \nil\ is used.  

%\cpar If a ~var~ is not specified, you can get \iter\ to declare the
%type of the internal variable by putting a |the| expression around
%~func.~  see section \ref{types}.

\cpar The differences between |accumulate| and |reducing| are slight.
One difference is that the functions take their arguments in a
different order.  Another is that in the absence of ~init-val~,
|accumulate| will use \nil, whereas |reducing| will generate different
code that avoids any dependence on the initial value.
The reason for having both clauses is that one usually
thinks of reductions (like |sum|) and accumulations (like |collect|)
as different beasts.

\end{clauses}


\subsubsection{Finders}

A ~finder~ is a clause whose value is an expression that meets some
condition.

\begin{clauses}

\clause{|finding| ~expr~ |such-that| ~test~ \opt\ |into| ~var~ |on-failure| ~failure-value~}
\clindexx{finding}{such-that}
If ~test~ (which is an expression) ever evaluates to \nonnil, the loop
is terminated, the 
epilogue code is run and the value of ~expr~ is returned.  Otherwise,
\nil\ (or ~failure-value,~ if provided) is returned.  If ~var~ is
provided, it will have either the \nonnil\ value of ~expr~ or
~failure-value~ when the epilogue code is run.

\cpar As a special case, if the ~test~ expression is a sharp-quoted
function, then it is applied to ~expr~ instead of being simply
evaluated.  E.g. |(finding x such-that \#'evenp)| is equivalent to
|(finding x such-that (evenp x))|.  

%\cpar although ~test~ need have nothing to do with ~%expr~ as in
%|(finding j such-that (> i 3))|, it usually 
%will: |(finding (length el) such-that (oddp (length el)))|.  to
%avoid performing the |length| computation twice, you could write
%|(finding (length el) such-that \#'oddp)| or |(finding (length
%el) such-that 'oddp)|; for these cases, \iter\ generates code that
%executes ~expr~ only once.  the code for |\#'oddp|
%is slightly different from that for {\lisp 'oddp}; see the discussion
%under {\lisp for\dots in} and {\lisp for\dots on}.

\cpar |On-failure| is a misnomer. Because it is always evaluated, it behaves
more like the default third argument to the |gethash| function. As a result,
|on-failure (error "Not found")| makes no sense. Instead, the clauses |leave|
or |thereis| can be used in conjunction with |finally| as follows:
\begin{program}
(iter (for x in '(1 2 3))
      (if (evenp x) (leave x))
      (finally (error "not found")))
\end{program}

\cpar This clause may appear multiple times when all defaults are
identical. It can also be used together with either |always|/|never| or
|thereis| if their defaults match. More specifically, |on-failure nil| is
compatible with |thereis|, while |on-failure t| is compatible with |always|
and |never| clauses.
\begin{program}
(iter (for i in '(7 -4 2 -3))
      (if (plusp i)
	  (finding    i  such-that (evenp i))
          (finding (- i) such-that (oddp i))))
\end{program}

\startitem
\Clause{|finding| ~expr~ |maximizing| ~m-expr~ \opt\ |into| ~var~}
\clindexx{finding}{maximizing}
\Clause{|finding| ~expr~ |minimizing| ~m-expr~ \opt\ |into| ~var~}
\clindexx{finding}{minimizing}
\finishitem
Computes the extremum (maximum or minimum) value of ~m-expr~ over all
iterations, and returns the value of ~expr~ corresponding to the
extremum.  ~expr~ is evaluated inside the loop at the time the new
extremum is established.  If ~m-expr~ is never evaluated (due to, for
example, being embedded in a conditional clause), then the returned
value depends on the type, if any, of ~expr~ (or ~var,~ if
one is supplied).  If there is no type, the returned
value will be \nil; if the type is numeric, the returned value will be
zero.

\cpar For these two clauses, ~var~ may be a list of two
symbols; in that case, the first is used to record ~expr~ and
the second, ~m-expr.~  

\cpar As with |finding\dots such-that|, if ~m-expr~ is a sharp-quoted
function, then it is called on ~expr~ instead of being evaluated.

\end{clauses}

\subsubsection{Boolean Tests}

\begin{clauses}

\clausex{first-iteration-p}{}
Returns |t| in the first cycle of the loop, otherwise \nil.

\clausex{first-time-p}{}
Returns |t| the first time the expression is evaluated, and then \nil\ forever.
This clause comes handy when printing (optional) elements separated
by a comma:

\begin{program}
(iter (for el in '(nil 1 2 nil 3))
      (when el
        (unless (first-time-p)
          (princ ", "))
        (princ el)))
\end{program}
produces |"1, 2, 3"|.

\end{clauses}

\subsubsection{Aggregated Boolean Tests}

\begin{clauses}

\clausex{always}{~expr~}
If ~expr~ ever evaluates to
\nil, then \nil\ is immediately returned; the epilogue code is not
executed.  If ~expr~ never evaluates to \nil, the epilogue code
is executed and the last value of ~expr~ (or |t| if ~expr~ was never
evaluated) is returned (whereas |loop| would constantly return |t|).

% mention last evaluated clause when multiple always clauses?

\clausex{never}{~expr~}
Like |(always (not ~expr~))|, except it does not influence the last
value returned by a possible other |always| clause. That is,
\begin{program}
(iter (repeat 2)
      (always 2)
      (never nil)) \yields 2 ; not t
\end{program}

\clausex{thereis}{~expr~}
If ~expr~ is ever \nonnil,
its value is immediately returned without running epilogue code.
Otherwise, the epilogue code is performed and \nil\ is returned.

This clause cannot be used together with |always| or |never|, because their
defaults are opposed (similarly, |(loop always 3 thereis nil)| refuses to
compile in some implementations of |loop|).

\end{clauses}


\subsection{Control Flow}
\label{control-flow}
Several clauses can be used to alter the usual flow of control in a loop.

Note: the clauses of this and subsequent sections don't adhere to \iter's
usual syntax, but instead use standard Common Lisp syntax.  Hence the
format for describing syntax subsequently is
like the standard format used in the Common Lisp manual, not like the
descriptions of clauses above.

\begin{clauses}

\clausex{finish}{}
Stops the loop and runs the epilogue code.

%for example:
%
%\begin{program}
%(iter (with answer = nil)
%         (initially (make-a-mess))
%         (for i from 1 to 10)
%         (when (correct? i)
%           (setq answer i)
%           (finish))
%         (finally (cleanup)))
%\end{program}
%
%this code will execute |cleanup| whether or not the test |(correct?
%i)| ever succeeds. 
%the (more elegant) formulation,
%\begin{program}
%(iter (initially (make-a-mess))
%         (for i from 1 to 10)
%         (finding i such-that (correct? i))
%         (finally (cleanup)))
%\end{program}
%would not execute |cleanup| if |(correct? i)| succeeded; it
%would do an immediate return.

\clausex{leave}{\opt\ ~value~}
Immediately returns ~value~ (default \nil) from the current \iter\
form, skipping the epilogue code.  Equivalent to using |return-from|.

\clausex{next-iteration}{}
Skips the remainder of the loop body and begins the next iteration of
the loop.

\clausex{while}{~expr~}
If ~expr~ ever evaluates to \nil, the loop is terminated and the
epilogue code executed.  Equivalent to |(if (not ~expr~) (finish))|.

\clausex{until}{~expr~}
Equivalent to |(if ~expr~ (finish))|.

\clausex{if-first-time}{~then~ \opt\ ~else~}
If this clause is being executed for the first time in this invocation
of the \iter\ form, then the ~then~ code is evaluated; otherwise the
~else~ code is evaluated.

\cpar |(for ~var~ first ~expr1~ then ~expr2~)| is almost equivalent to
\begin{program}
(if-first-time (dsetq ~var~ ~expr1~)
               (dsetq ~var~ ~expr2~))
\end{program}
The only difference is that the |for| version makes ~var~ available
for use with |for\dots previous|.

\end{clauses}

\subsection{Code Placement}
When fine control is desired over where code appears in a loop
generated by \iter, the following special clauses may be useful.
They are all subject to code-motion problems (see section
\ref{code-movement}).  

\begin{clauses}

\clausex{initially}{|\&rest| ~forms~}
The lisp ~forms~ are placed in the prologue section of the loop, where
they are executed once, before the loop body is entered.

\clausex{after-each}{|\&rest| ~forms~}
The ~forms~ are placed at the end of the loop body, where they
are executed after each iteration.  Unlike the other clauses in this
section, ~forms~ may contain \iter\ clauses.

\clausex{else}{|\&rest| ~forms~}
The lisp ~forms~ are placed in the epilogue section of the loop, where they
are executed if this |else| clause is never met during execution of the
loop and the loop terminates normally.

\clausex{finally}{|\&rest| ~forms~}
The lisp ~forms~ are placed in the epilogue section of the loop, where
they are executed after the loop has terminated normally.

\clausex{finally-protected}{|\&rest| ~forms~}
The lisp ~forms~ are placed in the second form of an unwind-protect
outside the loop.  They are always executed after the loop has
terminated, regardless of how the termination occurred.

\end{clauses}


%%%%%%%%%%%%%%%%%%%%%%%%%%%%%%%%%%%%%%%%%%%%%%%%%%%%%%%%%%%%%%%%
\section{Other Features}

\subsection{Multiple Accumulations}
\label{multiple}

\begin{sloppypar}
It is permitted to have more than one clause accumulate into the same
variable, as in the following:
\begin{program}
(iter (for i from 1 to 10)
      (collect i into nums)
      (collect (sqrt i) into nums)
      (finally (return nums)))
\end{program}
Clauses can only accumulate into the same variable if they are
compatible.  |collect|, |adjoining|, |appending|, |nconcing|,
|unioning| and |nunioning| are compatible with each other; |sum|,
|multiply| and |counting| are compatible; |always| and |never| are compatible;
|finding| \dots |such-that| is compatible with either |thereis| or |always|
and |never| when their defaults match; and |maximize| and |minimize| clauses
are compatible only with other |maximize| and |minimize| clauses,
respectively.

%note that the same variable ~cannot~ be both an accumulation
%variable and an ordinary variable; there can be only one variable with
%a given name within an \iter\ form. 
\end{sloppypar}

\subsection{Named Blocks}

Like Common Lisp |block|s, \iter\ forms can be given names.  The
name should be a single symbol, and it must be the first form in the
\iter.  The generated code behaves exactly like a named block; in
particular, |(return-from ~name~)| can be used to exit it:
\begin{program}
(iter fred
      (for i from 1 to 10)
      (iter barney
            (for j from i to 10)
            (if (> (* i j) 17)
                (return-from fred j))))
\end{program}
An \iter\ form that is not given a name is implicitly named \nil.

Sometimes one would like to write an expression in an inner \iter\ form,
but have it processed by an outer \iter\ form.  This is possible with
the |in| clause.  

\begin{clauses}

\clausex{in}{~name~ |\&rest| ~forms~}
Evaluates ~forms~ as if they were part of the \iter\ form named
~name~.  In other words, \iter\ clauses are processed by the \iter\
form named ~name,~ and not by any \iter\ forms that occur inside ~name.~

\cpar As an example, consider the problem of collecting a list of the
elements in a two-dimensional array.  The naive solution,
\begin{program}
(iter (for i below (array-dimension ar 0))
      (iter (for j below (array-dimension ar 1))
            (collect (aref ar i j))))
\end{program}
\noindent is wrong because the list created by the inner \iter\ is simply
ignored by the outer one.  But using |in| we can write:
\begin{program}
(iter outer (for i below (array-dimension ar 0))
      (iter (for j below (array-dimension ar 1))
            (in outer (collect (aref ar i j)))))
\end{program}
\noindent which has the desired result.

\end{clauses}

\subsection{Destructuring}
\label{destructuring}

In many places within \iter\ clauses where a variable is expected, a
list can be written instead.  In these cases, the value to be assigned
is ~destructured~ according to the pattern
described by the list.  As a simple example, the clause
\begin{program}
(for (key . item) in alist)
\end{program}
\noindent will result in |key| being set to the |car| of
each element in |alist|, and |item| being set to the |cdr|.  The
pattern list may be nested to arbitrary depth, and (as the example
shows) need not be terminated with \nil; the only requirement is that
each leaf be a bindable symbol (or \nil, in which case no binding is
generated for that piece of the structure).

Sometimes, you might like to do the equivalent of a
|multiple-value-setq| in a clause.  This 
``multiple-value destructuring'' can be expressed by writing 
\linebreak |(values $pat_1$ $pat_2 \ldots$)| for a destructuring
pattern, as in 
\begin{program}
(for (values (a . b) c d) = (three-valued-function ...))
\end{program}
\begin{sloppypar}
Note that the $pat_i$ can themselves be destructuring patterns (though
not multiple-value destructuring patterns).  You can't do multiple-value
destructuring in a |with| clause; instead wrap the whole \iter\
form in a |multiple-value-bind|.
\end{sloppypar}

\begin{note}{Rationale}
There are subtle interactions between variable declarations and
evaluation order that make the correct implementation of
multiple-value destructuring in a |with| somewhat tricky.
\end{note}

The destructuring feature of \iter\ is available as a separate
mechanism, using the |dsetq| macro:

\begin{clauses}
\defmacro{dsetq}{template expr}
Performs destructuring of ~expr~ using ~template~.   May be used
outside of an \iter\ form. Yields the primary value of ~expr~.

\end{clauses}

\subsection{On-line Help}

\begin{sloppypar}
There is a limited facility for on-line help, in the form of the 
|display-iterate-clauses| function.
\end{sloppypar}

\begin{clauses}

\defun{display-iterate-clauses}{\opt\ clause-spec}
Displays a list of \iter\ clauses.  If ~clause-spec~ is not
provided, all clauses are shown; if it is a symbol, all clauses
beginning with that symbol are shown; and if it is a list of symbols,
all clauses for which ~clause-spec~ is a prefix are shown.

\end{clauses}

\subsection{Parallel Binding and Stepping}
\label{bindings}

The parallel binding and stepping of variables is a feature that
\iter\ does ~not~ have.  This section attempts to provide a rationale.

We say that two variables are bound ~in parallel~ if neither
binding shadows the other.  This is the usual semantics of |let|
(as opposed to |let*|).  Similarly, we can say that iteration
variables are stepped in parallel if neither variable is updated
before the other, conceptually speaking; in other words, if the code
to update each variable can reference the old values of both variables.

|loop| allows parallel binding of variables and parallel stepping of
driver variables.  My view is that if you are depending on the 
serial/parallel distinction, you are doing something obscure.  If you
need to bind 
variables in parallel using |with|, then you must be using a
variable name that shadows a name in the existing lexical environment.
Don't do that.  The most common use for parallel stepping is to track
the values of variables on the previous iteration, but in fact this
does not require parallel stepping at all; the following will work:
\begin{program}
(iter (for current in list)
      (for prev previous current)
      ...)
\end{program}


%%%%%%%%%%%%%%%%%%%%%%%%%%%%%%%%%%%%%%%%%%%%%%%%%%%%%%%%%%%%%%%%
\section{Types and Declarations}
\label{types}

\subsection{Discussion}

Sometimes efficiency dictates that the types of variables be declared.
This type information needs to be communicated to \iter\ so it can
bind variables to appropriate values.  Furthermore, \iter\ must often
generate internal variables invisible to the user; there needs to be a
way for these to be declared.

As an example, consider this code, which will return the number of
odd elements in |number-list|:
\begin{program}
(iter (for el in number-list)
      (count (oddp el)))
\end{program}
In processing this form,
\iter\ will create an internal variable, let us call it |list17|, to
hold the successive |cdr|s of |number-list|, and  
will bind the variable to |number-list|.  It will also generate a
default binding for |el|; only inside the body of the loop will |el|
be set to the |car| of |list17|.  Finally, \iter\ will generate a
variable, call it |result|, to hold the result of the count, and will
bind it to zero.

When dealing with type declarations, \iter\ observes one simple rule:
~it will never generate a declaration unless requested to do so.~  The
reason is that such declarations might mask errors in compiled code by
avoiding error-checks; the resulting problems would be doubly hard to
track down because the declarations would be hidden from the
programmer.  Of course, a compiler might omit error-checks even in the
absence of declarations, though this behavior can usually be avoided,
e.g. by saying |(declaim (optimize (safety 3)))|.

So, the above \iter\ form will generate code with no declarations.
But say we wish to declare the types of |el| and the internal
variables |list17| and |result|.  How is this done?

Declaring the type of |el| is easy, since the programmer knows
the variable's name:
\begin{program}
(iter (for el in number-list)
      (declare (fixnum el))
      (counting (oddp el)))
\end{program}
\iter\ can read variable type declarations like this one.  Before
processing any clauses, it scans the entire top-level form for type
declarations 
and records the types, so that variable bindings can be performed
correctly.  In this case, |el| will be bound to zero
instead of \nil.  Also, \iter\ collects all the top-level declarations
and puts them at the begining of the generated code, so it is not
necessary to place all declarations at the beginning of an \iter\
form; instead, they can be written near the variables whose types they
declare.  

Since \iter\ is not part of the compiler, it will not know
about declarations that occur outside an \iter\ form; these
declarations must be repeated inside the form.

Here is another way we could have declared the type of |el|:
\begin{program}
(iter (for (the fixnum el) in number-list)
      (counting (oddp el)))
\end{program}
\iter\ extends the Common Lisp |the|\lispindex{the}
form to apply to variables as well as value-producing forms; anywhere
a variable is allowed---in a |with| clause, as the iteration
variable in a driver clause, as the |into| argument of an
accumulation clause, even inside a destructuring template---you can
write |(the ~type~ ~symbol~)| instead.

There is one crucial difference between using a |the| form and
actually declaring the variable: explicit declarations are always
placed in the generated code, but type information from a |the|
form is not turned into an actual declaration unless you tell \iter\
to do so using |iterate:declare-variables|.  See below.

\begin{sloppypar}
Declaring the types of internal variables is harder than declaring the
types of explicitly mentioned variables, since their names
are unknown.  You do it by declaring |iterate:declare-variables|
somewhere inside the top level of the \iter\ form.  (This will also
generate declarations for variables declared using |the|.)
\iter\ does not provide much selectivity here: it's all or none. 
And unfortunately, since \iter\ is not privy to compiler information
but instead reads declarations itself, it will not hear if you
|(declaim (iterate:declare-variables))|.  Instead, set the variable 
|iterate::*always-declare-variables*| to |t| at
compile-time, using |eval-when|.
\end{sloppypar}

To determine the appropriate types for internal variables, \iter\ uses
three sources of information:

\begin{itemize}

\item Often, the particular clause dictates a certain type for a
variable; \iter\ will use this information when available.  In the
current example, the variable |list17| will be given the type
|list|, since that is the only type that makes sense; and the
variable |result| will be given the type |fixnum|, on the
assumption that you will not be counting high enough to need bignums.
You can override this assumption only by using and explicitly declaring a
variable: 
\begin{verbatim}
(iter (declare (iterate:declare-variables))
      (for el in number-list)
      (count (oddp el) into my-result)
      (declare (integer my-result))
      (finally (return my-result)))
\end{verbatim}

\begin{sloppypar}
Other examples of the type assumptions that \iter\ makes are: type
|list| for |into| variables of collection clauses; type |list| for
expressions that are to be destructured; type |vector| for the
variable holding the vector in a |for\dots in-vector| clause, and
similarly for |string| and the |for\dots in-string| clause;
and the implementation-dependent type
|(type-of array-dimension-limit)| for the index and limit
variables generated by sequence iteration drivers like |for\dots
in-vector| and |for\dots in-string| (but not |for\dots in-sequence|,
because it may be used to iterate over a list). 
\end{sloppypar}

\item Sometimes, \iter\ will examine expressions and try to determine
their types in a simple-minded way.  If the expression is
self-evaluating (like a number, for instance), \iter\ knows that the
expression's type is the same as the type of the value it denotes, so
it can use that type.  If the expression is of the form |(the ~type~
~expr~)|, \iter\ is smart enough to extract ~type~ and use it.
However, the current version of \iter\ does not  
examine declarations of function result types or do any type
inference.  It will not determine, for 
example, that the type of |(+ 3 4)| is |fixnum|, or even
|number|.

\item In some cases, the type of an internal variable should match the
type of some other variable.  For instance, \iter\ generates an
internal variable for |(f x)| in the
clause |(for i from 1 to (f x))|, and in the absence of other
information will give it the same type as |i|.  If, however, the
expression had been written |(the fixnum (f x))|, then \iter\
would have given the internal variable the type |fixnum|
regardless of |i|'s type.  The type incompatibility errors that
could arise in this situation are not checked for.

\end{itemize}

Note that if you do declare |iterate:declare-variables|, then
\iter\ may declare user variables as well as internal ones if they do
not already have declarations, though only for variables that it
binds.  For instance, in this code:

\begin{program}
(iter (declare (iterate:declare-variables))
      (for i from 1 to 10)
      (collect i into var))
\end{program}
the variable |var| will be declared to be of type |list|.


\subsection{Summary}

\iter\ understands standard Common Lisp variable type declarations
that occur within an \iter\ form and
will pass them through to the generated code.  If the declaration
|(iterate:declare-variables)|\lispindex{declare-variables}
appears at the top level of an 
\iter\ form, or if 
|iterate::*always-declare-variables*|\lispindex{*always-declare-variables*} 
is \nonnil, then \iter\ will use the type information gleaned from user
declarations, self-evaluating expressions and |the| expressions,
combined with reasonable assumptions, to determine variable
types and declare them.


%%%%%%%%%%%%%%%%%%%%%%%%%%%%%%%%%%%%%%%%%%%%%%%%%%%%%%%%%%%%%%%%
\section{Problems with Code Movement}
\label{code-movement}

Some \iter\ clauses, or parts of clauses, result in code being
moved from the location of the clause to other parts of the loop.
Drivers behave this way, as do code-placement clauses like |initially|
and |finally|.  When using these clauses, there is a danger of writing
an expression that makes sense in its apparent location but will be
invalid or have a different meaning in another location.  For example:
\begin{program}
(iter (for i from 1 to 10)
      (let ((x 3))
        (initially (setq x 4))))
\end{program}
While it may appear that the |x| of |(initially (setq x 4))| is the
same as the |x| of |(let ((x 3)) \dots|, in fact they are not:
|initially| moves its code outside the loop body, so |x| would refer
to a global variable.  Here is another example of the same problem:
\begin{program}
(iter (for i from 1 to 10)
      (let ((x 3))
        (collect i into x)))
\end{program}
If this code were executed, |collect| would create a binding for its
|x| at the top level of the \iter\ form that the |let| will shadow.

Happily, \iter\ is smart enough to catch these errors; it
walks all problematical code to ensure that free variables are not
bound inside the loop body, and checks all variables it binds for the
same problem.

However, some errors cannot be caught:

\begin{program}
(iter (with x = 3)
      (for el in list)
      (setq x 1)
      (reducing el by \#'+ initial-value x))
\end{program}
|reducing| moves its |initial-value| argument to the initialization
part of the loop in order to produce more efficient code.  Since
\iter\ does not perform data-flow analysis, it cannot determine that
|x| is changed inside the loop; all it can establish is that |x| is
not bound internally.  Hence this code will not signal an
error and will use $3$ as the initial value of the reduction.

The following list summarizes all cases that are subject to these code
motion and variable-shadowing problems.
\begin{itemize}
\item Any variable for which \iter\ creates a binding, including those
used in |with| and the |into| keyword of many clauses.

\begin{sloppypar}
\item The special clauses which place code: |initially|, |after-each|, |else|,
|finally| and |finally-protected|. 
\end{sloppypar}

\item The variables of a |next| or |do-next| form.

\item The |initially| arguments of |for\dots initially\dots then| and
|for\dots previous|. 

\item The |then| argument of |for\dots initially\dots then|.

\item The |initial-value| arguments of |reducing| and |accumulate|.

\item The |on-failure| argument of |finding\dots such-that|.
\end{itemize}

%%%%%%%%%%%%%%%%%%%%%%%%%%%%%%%%%%%%%%%%%%%%%%%%%%%%%%%%%%%%%%%%
\section{Differences Between {\tt Iterate} and {\tt Loop}}
|loop| contains a great deal of complexity which \iter\ tries to
avoid.  Hence many esoteric features of |loop| don't exist in \iter.
Other features have been carried over, but in a cleaned-up form.  
And of course, many new features have been added; they are not
mentioned in this list.

\begin{itemize}

\item \iter's syntax is more Lisp-like than |loop|'s, having a higher
density of parens.

\item The current implementation of \iter, unlike the current version
of  |loop| (as documented in {\em Common Lisp, 2nd Ed.\/}), is
extensible (see section \ref{extend}).  

\item |loop| puts the updates of all driver variables at the top of
the loop; \iter\ leaves them where the driver clauses appear.

\item While for the most part \iter\ clauses that resemble |loop| clauses
behave similarly, there are some differences.  For instance, there is
no |for\dots =\dots then| in \iter; instead use 
|for\dots initially\dots then|.  

\item |loop| binds the variable |it| at certain times
to allow pseudo-English expressions like |when ~expr~ return it|. 
In \iter, you must bind ~expr~ to a variable yourself.  Note that
|when ~expr~ return it| is like |thereis ~expr~| except that the latter is an
accumulation clause and therefore competes with other accumulations
(remember section \ref{multiple} above).

% repeat different behaviour of |always| clause here?

\item |loop| has a special |return| clause, illustrated in the
previous item.  \iter\ doesn't need one,  since an ordinary Lisp
|return| has the same effect.

\item |loop| allows for parallel binding and stepping of iteration
variables.  \iter\ does not.  (See section \ref{bindings}.)

\item |loop| and \iter\ handle variable type declarations very
differently.  |loop| provides a special syntax for declaring variable
types, and does not examine declarations.  Moreover, the standard
implementation of |loop| will
generate declarations when none are requested.
\iter\ parses standard Common Lisp type declarations, and will never
declare a variable itself unless declarations are 
specifically requested.

\end{itemize}

%%%%%%%%%%%%%%%%%%%%%%%%%%%%%%%%%%%%%%%%%%%%%%%%%%%%%%%%%%%%%%%%
\section{Rolling Your Own}
\label{extend}

\subsection{Introduction}

\iter\ is extensible---you can write new clauses that embody new
iteration patterns.  You might want to write a new driver clause for a
data structure of your own, or you might want to write a clause that
collects or manipulates elements in a way not provided by \iter. 

This section describes how to write clauses for \iter.  Writing a
clause is like writing a macro.  In fact, writing a clause ~is~
writing a macro: since \iter\ code-walks its body and macroexpands,
you can add new abstractions to \iter\ with good old |defmacro|.

Actually, there are two extensions you can make to \iter\ that are
even easier than writing a macro.  They are adding a synonym for an
existing clause and defining a driver clause for an indexable
sequence.  These can be done with |defsynonym| and
|defclause-sequence|, respectively.  See section \ref{aids}, below.

The rest of this section explains how to write macros that expand into
\iter\ clauses.
Here's how you could add a simplified version of \iter's |multiply|
clause, if \iter\ didn't already have one:
\begin{program}
(defmacro multiply (expr)
  `(reducing ,expr by \#'* initial-value 1))
\end{program}

If you found yourself summing the square of an expression often, you
might want to write a macro for that.  A first cut might be
\begin{program}
(defmacro sum-of-squares (expr)
  `(sum (* ,expr ,expr)))
\end{program}
but if you are an experienced macro writer, you will realize that this
code will evaluate ~expr~ twice, which is probably a bad idea.  A
better version would use a temporary:
\begin{program}
(defmacro sum-of-squares (expr)
  (let ((temp (gensym)))
    `(let ((,temp ,expr))
       (sum (* ,temp ,temp)))))
\end{program}
Although this may seem complex, it is just the sort of thing you'd
have to go through to write any macro, which illustrates the point of
this section: if you can write macros, you can extend \iter.

Our macros don't use \iter's keyword-argument
syntax.  We could just use keywords with |defmacro|, but we would
still not be using \iter's clause indexing mechanism.  Unlike Lisp,
which uses just the first symbol of a form to determine what function
to call, \iter\ individuates clauses by the list of required keywords.
For instance, |for\dots in| and |for\dots in-vector| are different clauses
implemented by distinct Lisp functions.  

To buy into this indexing scheme, as well as the keyword-argument
syntax, use |defmacro-clause|:

\begin{clauses}

\defmacro{defmacro-clause}{arglist |\&body| body}
Defines a new \iter\ clause.  ~arglist~ is a list of symbols which are
alternating keywords and arguments.  \opt\ may be used, and the list
may be terminated by |\&sequence|.  ~body~ is an ordinary macro body,
as with |defmacro|.  If the first form of ~body~ is a string, it is
considered a documentation string and will be shown by
|display-iterate-clauses|. |defmacro-clause| will signal an error if 
defining the clause would result in an ambiguity.  E.g. you cannot
define the clause |for\dots from| because there would be no way to
distinguish it from a use of the |for| clause with optional keyword |from|.

\end{clauses}

\medskip

Here is |multiply| using |defmacro-clause|.  The keywords are capitalized
for readability.
\begin{program}
(defmacro-clause (MULTIPLY expr \opt\ INTO var)
  `(reducing ,expr by \#'* into ,var initial-value 1))
\end{program}
You don't have to worry about the case when |var| is not supplied; for
any clause with an |into| keyword, saying |into nil| is equivalent to
omitting the |into| entirely.

As another, more extended example, consider the fairly common
iteration pattern that 
involves finding the sequence element that maximizes (or minimizes) some
function.  \iter\ provides this as |finding\dots maximizing|, but it's
instructive to see how to write it.
Here, in pseudocode, is how you might write such a loop for
maximizing a function F:

\begin{quote}
\begin{tabbing}
set variable MAX-VAL to NIL; \\
set variable WINNER to NIL; \\
for\=\ each element EL in the sequence \\
\>  if\=\ MAX-VAL\=\ is NIL or F(EL) $>$ MAX-VAL then \\
\>\>    set MAX-VAL to F(EL); \\
\>\>    set WINNER to EL; \\
\>  end if; \\
end for; \\
return WINNER.
\end{tabbing}
\end{quote}

Here is the macro:
\begin{program}
(defmacro-clause (FINDING expr MAXIMIZING func \opt\ INTO var)
  (let ((max-val (gensym))
        (temp1 (gensym))
        (temp2 (gensym))
        (winner (or var iterate::*result-var*)))
    `(progn 
       (with ,max-val = nil)
       (with ,winner = nil)
       (cond 
        ((null ,max-val)
         (setq ,winner ,expr)
         (setq ,max-val (funcall ,func ,winner))
        (t
         (let* ((,temp1 ,expr)
                (,temp2 (funcall ,func ,temp1)))
           (when (> ,temp2 ,max-val)
             (setq ,max-val ,temp2)
             (setq ,winner ,temp1))))))
       (finally (leave ,winner)))))
\end{program}
Note that if no |into| variable is supplied, we use
|iterate::*result-var*|, which contains the internal variable into
which all clauses place their results.  If this variable is bound by
some clause, then \iter\ will return its value automatically;
otherwise, \nil\ will be returned.  

\subsection{Writing Drivers}

In principle, drivers can be implemented just as easily as other
\iter\ clauses.  In practice, they are a little harder to get right. 
As an example, consider writing a driver that
iterates over all the 
elements of a vector, ignoring its fill-pointer.  |for\dots in-vector|
won't work for this, because it observes the fill-pointer.  It's necessary to
use |array-dimension| instead of |length| to obtain the size of the
vector.  Here is one approach:

\begin{program}
(defmacro-clause (FOR var IN-WHOLE-VECTOR v)
  "All the elements of a vector (disregards fill-pointer)"
  (let ((vect (gensym))
        (index (gensym)))
    `(progn
       (with ,vect = ,v)
       (for ,index from 0 below (array-dimension ,vect 0))
       (for ,var = (aref ,vect ,index)))))
\end{program}
Note that we immediately put |v| in a variable, in case it is an
expression.  Again, this is just good Lisp macrology.  It also has a
subtle effect on the semantics of the driver: |v| is evaluated only
once, at the beginning of the loop, so changes to |v| in the loop have
no effect on the driver.  Similarly, the bounds for numerical iteration
e.g. the above |array-dimension| are also evaluated once only.  This is how
all of \iter's drivers work.


There is an important point concerning the |progn| in this code.  We
need the |progn|, of course, because we are returning several forms,
one of which is a driver.  But \iter\ drivers must occur at top-level.
Is this code in error?  No, because ~top-level~ is defined in \iter\
to include forms inside a |progn|.  This is just the definition of
top-level that Common Lisp uses, and for the same reason: to allow
macros to return multiple forms at top-level.

While our |for\dots in-whole-vector| clause will work, it is not
ideal.  In particular, it does not support generating.  Do do so, we
need to use |for\dots next| or |for\dots do-next|.  The job is
simplified by the |defmacro-driver| macro.

\begin{clauses}

\defmacro{defmacro-driver}{arglist |\&body| body}
Defines a driver clause in
both the |for| and |generate| forms, and provides a parameter
|generate| which ~body~ can examine to determine how it was invoked.
~arglist~ is as in |defmacro-clause|, and should begin with the symbol
|for|. 

\end{clauses}

With |defmacro-driver|, our driver looks like this:
\begin{program}
(defmacro-driver (FOR var IN-WHOLE-VECTOR v)
  "All the elements of a vector (disregards fill-pointer)"
   (let ((vect (gensym))
         (end (gensym))
         (index (gensym))
         (kwd (if generate 'generate 'for)))
     `(progn
        (with ,vect = ,v)
        (with ,end = (array-dimension ,vect 0))
        (with ,index = -1)
        (,kwd ,var next (progn (incf ,index)
                               (if (>= ,index ,end) (terminate))
                               (aref ,vect ,index))))))
\end{program}


We are still missing one thing: the |\&sequence| keywords.  
We can get them easily enough, by writing 
\begin{program}
(defmacro-driver (FOR var IN-WHOLE-VECTOR v \&sequence) 
  ...)
\end{program}
We can now refer to parameters |from|, |to|, |by|, etc. which contain
either the values for the corresponding keyword, or \nil\ if the
keyword was not supplied.  Implementing the right code for these
keywords is cumbersome but not difficult; it is left as an exercise.
But before you begin, see |defclause-sequence| below for an easier way.
       
\subsection{Extensibility Aids}
\label{aids}

This section documents assorted features that may be of use in
extending \iter.

\begin{clauses}

\defunexpvar{*result-var*}
Holds the variable that is used to return a value as a result of the
\iter\ form.  You may examine this and use it in a |with| clause, but
you should not change it.

\defmacro{defsynonym}{syn word}
 Makes ~syn~ a synonym for the existing \iter\ keyword ~word.~  Only
the first word in each clause can have synonyms. 

\defmacroxx{defclause-sequence}{element-name index-name}{|\&key|
access-fn size-fn sequence-type}{element-type
element-doc-string index-doc-string}  
Provides
a simple way to define sequence clauses.  Generates two
clauses, one for iterating over the sequence's elements, the other
for iterating over its indices.  The first symbol of both
clauses will have print-name |for|.
~element-name~  and ~index-name~ should be symbols.
~element-name~ is the second keyword of the element iterator (typically of
the form 
|in-~sequence-type~|), and ~index-name~ is the second keyword
of the index-iterator (typically of the form
|index-of-~sequence-type~|).  Either name may be 
\nil, in which case the corresponding clause is not defined.  If both
symbols are supplied, they should be in the same package.  The |for|
that begins the clauses will be in this package.

\cpar ~access-fn~ is the function to be used to
access elements of the sequence in the element iterator.  The function
should take two 
arguments, a sequence and an index, and return the appropriate element.
~size-fn~ should denote a function of one argument, a sequence, that
returns its size.  Both ~access-fn~ and ~size-fn~ are required for the
element iterator, but only ~size-fn~ is needed for the index iterator.

\cpar The ~sequence-type~ and ~element-type~ keywords are used to
suggest types for the variables 
used to hold the sequence and the
sequence elements, respectively.  The usual rules about \iter's
treatment of variable type declarations apply (see section \ref{types}).

\cpar ~element-doc-string~ and ~index-doc-string~ are
the documentation strings, for use with |display-iterate-clauses|. 

\cpar The generated element-iterator performs destructuring on the
element variable.

\cpar As an example, the above |for\dots in-whole-vector| example
could have been written:
\begin{program}
(defclause-sequence IN-WHOLE-VECTOR INDEX-OF-WHOLE-VECTOR
  :access-fn 'aref
  :size-fn \#'(lambda (v) (array-dimension v 0))
  :sequence-type 'vector
  :element-type t
  :element-doc-string 
     "Elements of a vector, disregarding fill-pointer"
  :index-doc-string 
     "Indices of vector, disregarding fill-pointer")
\end{program}

\end{clauses}

\subsection{Subtleties}

There are some subtleties to be aware of when writing \iter\ clauses.
First, the code returned by your macros may be |nconc|'ed into a list,
so you should always returned freshly consed lists, rather than
constants.  Second, \iter\ matches clauses by using |eq| on the first
symbol and |string=| on the subsequent ones, so the package of the
first symbol of a clause is relevant.  All of the clauses in this manual
have their first word in the \iter\ package.  
You can use the package system in the usual way to shadow
\iter\ clauses without replacing them.

%%% say more here, about the badness that only the first word of a
%%% clause is packagey.

\section{Non-portable Extensions to Iterate (Contribs)}
\label{contribs}

Currently, there is only one non-portable extension to iterate in the
distribution: iterate-pg. If you have made an extension that depends
on non-portable features, feel free to send them to |asf@boinkor.net|
for inclusion in the iterate distribution.

\subsection{An SQL query driver for iterate}

The pg package by Eric Marsden (see |http://cliki.net/pg|) provides an
interface to the PostgreSQL database. Using the \iterpg\ extension, it
is possible to handle the results of SQL queries using \iter.

This usage example should give you an idea of how to use it:

\begin{program}
(pg:with-pg-connection (c "somedb" "someuser")
  (iter (for (impl version date) in-relation "select * from version"
                                 on-connection *dbconn*)
        (collect version)))
\end{program}

To use the extension via |ASDF|, simply make your system depend on the
|iterate-pg| system instead of the |iterate| system. To load it
manually, use:

\begin{program}
  (asdf:oos 'asdf:load-op :iterate-pg)
\end{program}

%%%%%%%%%%%%%%%%%%%%%%%%%%%%%%%%%%%%%%%%%%%%%%%%%%%%%%%%%%%%%%%%
\section{Obtaining {\tt Iterate}}

\begin{sloppypar}
The information in this chapter is obsolete but included for
completeness's sake; Currently, the most up-to-date information on
\iter\ can be found at |http://boinkor.net/iterate.html|.
\end{sloppypar}

\begin{sloppypar}
\iter\ currently runs on Lisp Machines, and on
HP's, Sun3's and Sparcstations under Lucid.  
\iter\ source and binaries are available at the MIT AI Lab in the
subdirectories of |/src/local/lisplib/|.  The source file,
|iterate.lisp|, is also available for anonymous FTP in the directory
|/com/fpt/pub/| on the machine |TRIX.AI.MIT.EDU| (Internet number
128.52.32.6).  If you are unable to obtain |iterate| in one of these
ways, send mail to |jba@ai.mit.edu| and I will send you the source
file. 
\end{sloppypar}

\begin{sloppypar}
\iter\ resides in the |iterate| package (nickname |iter|).  Just say
\linebreak |(use-package :iterate)| to make all the necessary symbols
available. 
If a symbol is not exported, it appears in this manual with an
``|iterate::|'' prefix.
\end{sloppypar}

Send bug reports to |bug-iterate@ai.mit.edu|.  The |info-iterate|
mailing list will have notices of changes and problems; to have
yourself added, send mail to |info-iterate-request@ai.mit.edu|.


\medskip

\begin{flushleft}
 \bf Acknowledgements
\end{flushleft}
\smallskip

Richard Waters provided invaluable criticism which spurred me to improve
\iter\ greatly.  As early users, David Clemens, Oren Etzioni and Jeff
Siskind helped ferret out many bugs.

%\begin{theindex}
% The index files must be generated with the genindex program
% or from makeindex iterate-manual.idx which creates iterate-manual.ind.

%\baselineskip\normalbaselineskip
%\advance\baselineskip by -2pt

\input{iterate-manual.ind}

%\underline{Clauses}
%\input{iterate-manual.clindex}
%
%\indexspace
%\underline{Lisp}
%\input{iterate-manual.lispindex}

%\end{theindex}


\end{document}


% arch-tag: "c0f871a6-313c-11d8-abb9-000c76244c24"

%
%\indexspace
%\underline{Lisp}
%% Fix ``.'' spacing in lists
\documentstyle[12pt,aimemo]{article}

\setlength{\textwidth}{6.0in}
\setlength{\oddsidemargin}{0.25in}
\setlength{\evensidemargin}{0.25in}
\setlength{\topmargin}{0pt}
\setlength{\headsep}{24pt}
\setlength{\textheight}{8.5in}

\let\origpar=\par

% somehow, LaTex fucks with things so that the plain Tex \obeyspaces
% doesn't work.  This is the same, except we use an mbox containing
% a space instead of just a space.

\newcommand{\spc}{\mbox{ }}
\newcommand{\tAb}{\mbox{        }}
\def\obspaces{\catcode`\ =\active\catcode`\^^I=\active}
{\obspaces \global\let =\spc\global\let^^I=\tAb}

\newenvironment{program}{\medskip\samepage\tt\obeylines\obspaces}{\medskip}

% use | for \tt
\catcode`\|=13
\newif\ifvbar
\def|{\ifvbar
         \endgroup\vbarfalse
      \else 
         \begingroup\tt\vbartrue 
      \fi}



\newcommand{\lisp}{\tt}
\newcommand{\iter}{|iterate|}
\newcommand{\iterpg}{|iterate-pg|}
\newcommand{\setf}{|setf|}
\newcommand{\nil}{|nil|}
\newcommand{\nonnil}{non-\nil}
\newcommand{\opt}{{\lisp \&optional}}
\newcommand{\yields}{$\Rightarrow$}



\newbox\Dots
\setbox\Dots=\hbox{\small ...}
\renewcommand{\dots}{\copy\Dots}

% Here's a way to define an environment which will treat all
% paragraphs inside it in the same manner.  It's, as near as I can
% figure, the way that Latex does things.  Using \everypar won't work;
% apparently Latex resets that a lot.  What you have to do is: at the
% top of the file do \let\origpar=\par.  In the
% begin part of the environment, do the parshape (or whatever), then say
% \def\par{{\origpar}}.  The extra braces are necessary.   That's all
% you have to do; at the end of the environment things will return to
% normal automatically.  Why does this all work?  I have no idea.
% Another important thing: there should be a blank line (i.e. a \par)
% between the end of the last paragraph and the \end{environment}.

\newlength{\clindent}
\setlength{\clindent}{0.5in}
\newlength{\clparindent}
\setlength{\clparindent}{\parindent}

\newenvironment{clauses}{\advance\linewidth -\parindent
                        \hangindent\clindent\relax
                         \hangafter=0
                          \parindent=0pt
                          \def\par{{\origpar}}}{}

\newcommand{\cpar}{\hspace \clparindent}

\newcommand{\startitem}{\bigskip\pagebreak[3]}
\newcommand{\finishitem}{\smallskip}

\newcommand{\clause}[1]{\startitem\Clause{#1}\finishitem}
\newcommand{\Clause}[1]{\hbox{#1}}
\newcommand{\clindex}[1]{\index{{\lisp #1}}}
\newcommand{\clindexx}[2]{\index{{\lisp #1\dots #2}}}
\newcommand{\clindexxx}[3]{\index{{\lisp #1\dots #2\dots #3}}}
\newcommand{\clausex}[2]{\startitem\Clausex{#1}{#2}\finishitem}
\newcommand{\Clausex}[2]{\Clause{|#1| #2}\clindex{#1}}

\newcommand{\defvar}[1]{\startitem\deff{#1}{}{Variable}\finishitem}
\newcommand{\defunexpvar}[1]{\startitem\deffnoindex{iterate::#1}{}{Variable}
                                        \lispindex{#1}\finishitem}
\newcommand{\defun}[2]{\startitem\deff{#1}{#2}{Function}\finishitem}
\newcommand{\defmacro}[2]{\startitem\deff{#1}{#2}{Macro}\finishitem}
\newcommand{\defunx}[3]{\startitem\deffx{#1}{#2}{#3}{Function}\finishitem}
\newcommand{\defmacrox}[3]{\startitem\deffx{#1}{#2}{#3}{Macro}\finishitem}
\newcommand{\defunxx}[4]{\startitem\deffxx{#1}{#2}{#3}{#4}{Function}\finishitem}
\newcommand{\defmacroxx}[4]{\startitem\deffxx{#1}{#2}{#3}{#4}{Macro}\finishitem}

\newcommand{\deff}[3]{\deffnoindex{#1}{#2}{#3}\lispindex{#1}}
\newcommand{\deffnoindex}[3]{\hbox to \hsize{{\tt #1} {\it #2}\hfill [{\it #3}]}}
\newcommand{\deffx}[4]{\deff{#1}{#2}{#4}\moreargs{#1}{#3}}
\newcommand{\deffxx}[5]{\deffx{#1}{#2}{#3}{#5}\moreargs{#1}{#4}}

\newcommand{\moreargs}[2]{\hbox to \hsize{\phantom{\lisp #1} {\it #2}\hfill}}

\newcommand{\lispindex}[1]{\index{{\lisp #1}}}

 \newenvironment{note}[1]{\pagebreak[2]\bigskip
                          \hrule\smallskip\small
                         {\setlength{\parindent}{0pt}
                           \par{\bf #1:}}}{\smallskip\hrule\bigskip}



\makeindex

%%%%%%%%%%%%%%%%%%%%%%%%%%%%%%%%%%%%%%%%%%%%%%%%%%%%%%%%%%%%%%%%

%\thispagestyle{empty}

%\vspace*{2in}

%\title{The |Iterate| Manual}
%\author{Jonathan Amsterdam}

%\maketitle

%\thispagestyle{empty}

%\begin{center}
%\Large *** DRAFT ***
%\end{center}
%This manual is in beta-test. It will become an AI Memo in a couple of
%months.  I would appreciate any comments or
%corrections.  You can reach me at email address |jba@ai.ai.mit.edu|.

\begin{document}

\def\writtenby{Jonathan Amsterdam}
\def\contractno{This report describes research done at the Artificial
Intelligence Laboratory of the Massachusetts Institute of Technology.
Support for the laboratory's artificial intelligence research is
provided in part by the Advanced Research Projects Agency of the
Department of Defense under Office
of Naval Research contract N00014-85-K-0124.}

\AIMEMO{1236}{\Large The Iterate Manual \\[1in]}
{This is the manual for version 1.4 of \iter, a powerful iteration
macro for Common  
Lisp. \iter\ is similar to {\tt loop} but provides numerous additional
features, is well integrated with Lisp, and is extensible. 
\vfill }


% use ~ for \em
\catcode`\~=13
\newif\ifem
\def~{\ifem
         \/\endgroup\emfalse
      \else 
         \begingroup\em\emtrue 
      \fi}



%\pagebreak
%\setcounter{page}{1}

%\thispagestyle{empty}


\tableofcontents

\pagebreak
\pagestyle{headings}

\section{Introduction}

\begin{sloppypar}
This manual describes
\iter, a powerful iteration facility for Common Lisp.
\iter\ provides abstractions for many common iteration
patterns and allows for the definition of additional patterns.  \iter\ is
a macro that expands into ordinary Lisp at
compile-time, so it is more efficient than higher-order
functions like |map| and |reduce.|  
While it is similar to |loop|, \iter\ offers a more Lisp-like syntax
and enhanced extensibility.
(For a more complete
comparison of \iter\ with other iteration constructs, see MIT AI Lab Working
Paper No. 324, ~Don't Loop, Iterate.~)
\end{sloppypar}

An \iter\ form consists of the symbol |iter|\footnote{You can also use
|iterate|, but |iter| is preferred because it avoids potential conflicts with
possible future additions to Common Lisp, and because it saves
horizontal space when writing code.}
followed by one or more
forms, some of which may be \iter\ 
~clauses.~  Here is a simple example of \iter\ which collects the
numbers from 1 to 10 into a list, and returns the list.  The return
value is shown following the arrow.
\begin{program}
(iter (for i from 1 to 10)
      (collect i))          \yields (1 2 3 4 5 6 7 8 9 10)
\end{program}
This form contains two clauses: a |for| clause that steps the
variable |i| over the integers from 1 to 10, and a |collect|
clause that accumulates its argument into a list.  With a few
exceptions, all \iter\ clauses have the same format: alternating
symbols (called ~keywords~) and expressions (called
~arguments~).  The syntax and terminology are those of Common
Lisp's keyword lambda lists.  One difference is that \iter's keywords
do not have to begin with a colon---though they may, except for the
first symbol of a clause.  So you can also write  |(for i :from
1 :to 10)| if you prefer.

Any Lisp form can appear in the body of an \iter, where it will have
its usual meaning.  \iter\ walks the entire body, expanding macros,
and recognizing clauses at any level.  This example collects all the
odd numbers in a list:
\begin{program}
(iter (for el in list)
      (if (and (numberp el) (oddp el))
          (collect el)))
\end{program}

There are clauses for iterating over numbers, lists, arrays and other
objects, and for
collecting, summing, counting, maximizing and other useful operations.
\iter\ also supports the creation of new variable bindings, stepping
over multiple sequences at once, destructuring, and compiler
declarations of variable types.  The following example illustrates
some of these features: 
\begin{program}
(iter (for (key . item) in alist)
         (for i from 0)
         (declare (fixnum i))
         (collect (cons i key)))
\end{program}

This loop takes the keys of an alist and returns a new alist
associating the keys with their positions in the original list. The
compiler declaration for |i| will appear in the generated code
in the appropriate place.

%%%%%%%%%%%%%%%%%%%%%%%%%%%%%%%%%%%%%%%%%%%%%%%%%%%%%%%%%%%%%%%%
\section{Clauses}

Most of \iter's clauses will be familiar to |loop| programmers.  
(|loop| is an iteration macro that has been incorporated into Common
Lisp.  Seen Guy Steele's {\em Common Lisp, 2nd Edition}.)
In
nearly all cases they behave the same as their |loop| counterparts, so
a |loop| user can switch to \iter\ with little pain (and much gain).

All clauses with the standard keyword-argument syntax consist of two
parts: a ~required~ part, 
containing keywords that must be present and in the right order; and
an ~optional~ part, containing keywords that may be omitted and,
if present, may occur in any order.  In the descriptions below, the
parts are separated by the Lisp lambda-list keyword \opt.


\subsection{Drivers}

An iteration-driving clause
conceptually causes the iteration to go forward.  Driver clauses in
\iter\ allow iteration over numbers, lists, vectors, hashtables, packages,
files and streams.  Iteration-driving clauses must
appear at the top level of an \iter\ form; they cannot be nested
inside another clause.  The driver variable is updated at the point
where the driver clause occurs.  Before the clause is executed for the
first time, the value of the variable is undefined.

%Also, regardless of where the driver clause
%appears in the body, the driver variable is stepped at the top of the
%loop; hence it is stylistically preferable, though
%not required, to place driver clauses at the beginning of the \iter.

Multiple drivers may appear in a single \iter\ form, in which case all
of the driver variables are updated each time through the loop, in the
order in which the clauses appear.  The first driver to terminate will
terminate the entire loop.

In all cases, the value of the driver variable on exit from the loop,
including within the epilogue code (see the |finally| clause), is
undefined.

All the parameters of a driver clause are evaluated once, before the
loop begins.  Hence it is not possible to change the bounds or other
properties of an iteration by side-effect from within the loop.

With one exception, driver clauses begin with the word |for| (or
the synonym |as|) and mention an iteration variable, which is
given a binding within the \iter\ form.
The exception is |repeat|, which just executes a loop a
specified number of times:

\begin{clauses}

\clausex{repeat}{~n~}
Repeats the loop ~n~ times.  For example:
\begin{program}
(iter (repeat 100)
      (print "I will not talk in class."))
\end{program}
\cpar If $n \leq 0$, the
loop will never be executed.  If ~n~ is not an integer, the
actual number of executions will be $\lceil n \rceil$.

\end{clauses}

\pagebreak[3]

\subsubsection{Numerical Iteration}

\begin{clauses}
\clausex{for}{~var~ |\&sequence|} 
The general form for iterating over a sequence of numbers requires a
variable and, optionally, one or more keywords that provide the bounds
and step size of the iteration. The |\&sequence|
lambda-list keyword is a shorthand for these sequence keywords.  They are:
|from|, |upfrom|, |downfrom|, |to|, |downto|, |above|, |below| and |by|.
|from| provides the starting value for ~var~
and defaults to zero. |to| provides a final value and implies that the
successive values of ~var~ will be increasing; |downto| implies that
they will be decreasing.  The loop terminates when ~var~ passes
the final value (i.e. becomes smaller or larger than it, depending on
the direction of iteration); in other  words, the loop body
will never be 
executed for values of ~var~ past the final value.
|below| and |above| are similar to |to| and
|downto|, except that the loop terminates when ~var~ equals or passes
the final value.

\cpar If no final value is specified, the variable will be stepped
forever.  Using |from| or |upfrom| will result in increasing
values, while |downfrom| will give decreasing values.

\cpar On each iteration,
~var~ is incremented or decremented by the value of the sequence
keyword |by|, which defaults to 1.  It should always be a positive
number, even for downward iterations.

\cpar In the following examples, the sequence of numbers generated is shown
next to the clause.

\begin{program}
(for i upfrom 0) \yields\ 0 1 2 \ldots
(for i from 5)   \yields\ 5 6 7 \ldots    ; either from or upfrom is okay
(for i downfrom 0) \yields\ 0 -1 -2 \ldots
(for i from 1 to 3) \yields\ 1 2 3
(for i from 1 below 3) \yields\ 1 2
(for i from 1 to 3 by 2) \yields\ 1 3
(for i from 1 below 3 by 2) \yields\ 1
(for i from 5 downto 3) \yields\ 5 4 3
\end{program}

\end{clauses}

\subsubsection{Sequence Iteration}

There are a number of clauses for iterating over sequences.  In all of
them, the argument following |for| may be a list instead of a
symbol, in which case destructuring is performed.  See section
\ref{destructuring}. 

\begin{clauses}

\clause{|for| ~var~ |in| ~list~ \opt\ |by| ~step-function~}
\clindexx{for}{in}
~var~ is set to successive elements of list.  
~step-function,~ which 
defaults to |cdr|, is used to obtain the next sublist. 

\clause{|for| ~var~ |on| ~list~ \opt\ |by| ~step-function~}
\clindexx{for}{on}
~var~ is set to successive sublists of list. 
~step-function~ (default |cdr|) is used as in |for\dots in|.

\end{clauses}

\medskip

These two clauses use |atom| to test for the end
of a list.  Hence, given a list whose final |cdr| is not \nil,
they will silently ignore the last |cdr|.  Other choices are
|endp|,  which would signal an error, and |null|, which
would probably result in an error somewhere else.  If you wish to use
an end-test other than |atom|, set the variable
|iterate::*list-end-test*|\lispindex{*list-end-test*} to the name of
the desired function. 

\begin{clauses}
\clause{|for| ~var~ |in-vector| ~vector~ |\&sequence|}
\clindexx{for}{in-vector}
~var~ takes on successive elements from ~vector.~   The vector's
fill-pointer is observed.  Here and in subsequent clauses, the |\&sequence|
keywords include |with-index|, which takes a symbol as argument
and uses it for the index variable instead of an internally generated
symbol.  The other |\&sequence| keywords behave as in numerical
iteration, except that the default iteration bounds are the bounds of
the vector.  E.g. in |(for i in-vector v downto 3)|,
|i| will start off being bound to the last element in |v|,
and will  
be set to preceding elements down to and including the element with
index 3.

\clause{|for| ~var~ |in-sequence| ~seq~ |\&sequence|}
\clindexx{for}{in-sequence}
This uses Common Lisp's generalized sequence functions, |elt|
and |length|, to obtain elements and determine the length of
~seq.~   Hence it will work for any sequence, including lists, and will
observe the fill-pointers of vectors.

\clause{|for| ~var~ |in-string| ~string~ |\&sequence|}
\clindexx{for}{in-string}
~var~ is set to successive characters of ~string.~

\startitem
\Clause{|for| ~var~ |index-of-vector| ~vector~ |\&sequence|}
\clindexx{for}{index-of-vector}
\Clause{|for| ~var~ |index-of-sequence| ~sequence~ |\&sequence|}
\clindexx{for}{index-of-sequence}
\Clause{|for| ~var~ |index-of-string| ~string~ |\&sequence|}
\clindexx{for}{index-of-string}
\finishitem
~var~ is set to successive indices of the sequence.
These clauses avoid the overhead of accessing the sequence elements
for those applications where they do not need to be examined, or are
examined rarely.  They admit all the optional keywords of the other
sequence drivers except the (redundant) |with-index| keyword.

\clause{|for| ~(key value)~ |in-hashtable| ~table~}
\clindexx{for}{in-hashtable}

~key~ and ~value,~ which must appear as shown in a list
and may be destructuring templates, are set to the keys and values of
~table~.  If ~key~ is |nil|, then the hashtable's keys will be ignored;
similarly for ~value~.
The order in which 
elements of ~table~ will be retrieved is unpredictable.

\clause{|for| ~var~ |in-package| ~package~ \opt\ |external-only| ~ext~}
\clindexx{for}{in-package}
Iterates over all the symbols in ~package,~ or over only the
external symbols if ~ext~ is specified and non-|nil|.  ~ext~ is not
evaluated.  The same symbol may appear more than once.

\clause{|for| ~var~ |in-packages| ~packages~ \opt\ |having-access| ~symbol-types~}
\clindexx{for}{in-packages}
Iterates over all the symbols from the list of packages denoted by the
descriptor ~packages~ and having accessibility (or visibility) given by
~symbol-types~.  This defaults to the list |(:external :internal :inherited)|
and is not evaluated.  ~var~ must be a list of up to three variables: in each
iteration, these will be set to a symbol, its access-type and package (as per
|with-package-iterator| in ANSI-CL).  The same symbol may appear more than
once.

\end{clauses}

\begin{clauses}

\clause{|for| ~var~ |in-file| ~name~ \opt\ |using| ~reader~}
\clindexx{for}{in-file}
Opens the file ~name~ (which may be a string or pathname) for input,
and iterates 
over its contents. ~reader~ defaults to |read|, so by default ~var~
will be bound to the successive forms in the file.  The \iter\ body is
wrapped in an unwind-protect to ensure that the file is closed no
matter how the \iter\ is exited.

\clause{|for| ~var~ |in-stream| ~stream~ \opt\ |using| ~reader~}
\clindexx{for}{in-stream}
Like |for\dots in-file|, except that ~stream~ should be an existing
stream object that supports input operations.

\end{clauses}

\subsubsection{Generalized Drivers}

These are primarily useful for writing drivers that can also be used
as generators (see section \ref{generators}, below).
 
\begin{clauses}

\clause{|for| ~var~ |next| ~expr~}
\clindexx{for}{next}

~var~ is set to ~expr~ each time through the loop.  Destructuring is
performed.  When the clause is used as a generator, ~expr~ is the code
that is executed when |(next ~var~)| is 
encountered (see section \ref{generators}, below).
~expr~ should compute the first value for ~var~, as well
as all subsequent values, and is responsible for terminating the loop.
For compatibility with future versions of \iter, this termination
should be done with |terminate|\clindex{terminate}, which can be
considered a synonym for |finish| (see section \ref{control-flow}).

\cpar As an example, the following clauses are equivalent to |(for
i from 1 to 10)|:
\begin{program}
(initially (setq i 0))
(for i next (if (> i 10) (terminate) (incf i)))
\end{program}

\clause{|for| ~var~ |do-next| ~form~}
\clindexx{for}{do-next}
~form~ is evaluated each time through the loop.  Its value is ~not~
set to ~var~; that is ~form's~ job. ~var~ is only present so that
\iter\ knows it is a driver variable.  \linebreak |(for ~var~ next
~expr~)| is equivalent to |(for ~var~ do-next (dsetq ~var~ ~expr~))|.
(See section \ref{destructuring} for an explanation of |dsetq|.) 

\end{clauses}

\subsubsection{Generators}
\label{generators}

In all of the above clauses, the driver variable is updated on each
iteration.
Sometimes it is desirable to have greater control over updating.  
For instance, consider the problem of associating numbers, in
increasing order and with no gaps, with the
\nonnil\ elements of a list.  One obvious first pass at writing this is:

\begin{program}
(iter (for el in list)
      (for i upfrom 1)
      (if el (collect (cons el i))))
\end{program}
But on the list |(a b nil c)| this produces |((a . 1) (b . 2) (c .
4))| instead of the desired |((a . 1) (b . 2) (c . 3))|.  The problem
is that |i| is incremented each time through the loop, even when |el|
is |nil|.  

The problem could be solved elegantly if we could step |i| only when
we wished to.  This
can be accomplished for any \iter\ driver by writing |generate| (or
its synonym |generating|)
instead of |for|.  Doing so produces a ~generator~---a driver whose
values are yielded explicitly.  To obtain the next value of a
generator variable ~v~, write \linebreak |(next ~v~)|.  The value of a |next|
form is the next value of ~v~, as determined by its associated driver
clause.  |next| also has the side-effect of updating ~v~ to that
value.  If there is no next value, |next| will terminate the loop,
just as with a normal driver.


Using generators, we can now write our example like this:
\begin{program}
(iter (for el in list)
      (generate i upfrom 1)
      (if el (collect (cons el (next i)))))
\end{program}
Now |i| is updated only when |(next i)| is executed, and this occurs
only when |el| is \nonnil.

To better understand the relationship between ordinary drivers and
generators, observe that we can rewrite an ordinary driver using its
generator form immediately followed by |next|, as this example shows:
\begin{program}
(iter (generating i from 1 to 10)
      (next i)
      ...)
\end{program}
Provided that the loop body contains no |(next i)| forms, this will
behave just as if we had written |(for i from 1 to 10)|.

We can still refer to a driver variable ~v~ without using |next|; in
this case, its value is that given to it by the last evaluation of
|(next ~v~)|.  Before |(next ~v~)| has been called the first time, the
value of ~v~ is undefined.  

This semantics is more flexible than
one in which ~v~ begins the loop bound to its first value and calls
of |next| supply subsequent values, because it means the loop will not
terminate too soon if the generator's sequence is empty.  For
instance, consider the following code, which tags \nonnil\ elements of
a list using a list of tags, and also counts the null elements.
(We assume there are at least as many tags as \nonnil\ elements.)
\begin{program}
(let* ((counter 0)
       (tagged-list (iter (for el in list)
                          (generating tag in tag-list) 
                          (if (null el)
                              (incf counter)
                              (collect (cons el (next tag)))))))
  ...)
\end{program}
It may be that there are just as many tags as non-null elements of
|list|.  If all the elements of |list| are null, we still want the
counting to proceed, even though |tag-list| is \nil.  If |tag| had to be
assigned its first value before the loop begins, we would have had to
terminate the loop before the first iteration, since when |tag-list|
is \nil, |tag| 
has no first value.  With the existing semantics, however, |(next
tag)| will never execute, so the iteration will cover all the elements of
|list|. 
        
When the ``variable'' of a driver clause is actually a destructuring
template containing several variables, all the variables are eligible
for use with |next|.  As before, |(next ~v~)| evaluates to ~v's~ next
value; but the effect is to update all of the template's variables.
For instance, the following code will return the list |(a 2 c)|.
\begin{program}
(iter (generating (key . item) in '((a . 1) (b . 2) (c . 3)))
      (collect (next key))
      (collect (next item)))
\end{program}

Only driver clauses with variables can be made into generators.  This
includes all clauses mentioned so far except for |repeat|.  It does
~not~ include |for\dots previous|, |for\dots =|,
|for\dots initially\dots then| or |for\dots first\dots then| (see
below). 

\subsubsection{Previous Values of Driver Variables}

Often one would like to access the value of a variable on a previous
iteration.  \iter\ provides a special clause for accomplishing this.

\begin{clauses}

\clause{|for| ~pvar~ |previous| ~var~ \opt\ |initially| ~init~
|back| ~n~}
\clindexx{for}{previous}
Sets ~pvar~ to the previous value of ~var,~ which should be a driver
variable, a variable from another |for\dots previous| clause, or a
variable established by a |for\dots =|, 
|for\dots initially\dots then| or |for\dots first\dots then| clause
(see section \ref{setting}).
Initially, ~pvar~ is given the value ~init~ (which defaults to \nil).  
The ~init~ expression will be moved outside the loop body, so it
should not depend on anything computed within the loop.
~pvar~ retains the value of ~init~ until ~var~ is set to its second
value, at which point ~pvar~ is set to ~var's~ first value; and so on.  

\cpar The
argument ~n~ to |back| must be a constant, positive integer, and
defaults to 1.  It determines how many iterations back ~pvar~ should
track ~var~.  For example, when ~n~ is 2, then ~pvar~ will be assigned
~var's~ first value when ~var~ is set to its third value.

\cpar A |for\dots previous| clause may occur after or before its
associated driver clause. |for\dots previous| works with generators as
well as ordinary drivers.

\pagebreak[3]

\cpar Example:
\begin{program}
(iter (for el in '(1 2 3 4))
      (for p-el previous el)
      (for pp-el previous p-el initially 0)
      (collect pp-el))
\end{program}
This evaluates to |(0 0 1 2)|.  It could have been written more
economically as 
\begin{program}
(iter (for el in '(1 2 3 4))
      (for pp-el previous el back 2 initially 0)
      (collect pp-el))
\end{program}


\end{clauses}

\subsection{Variable Binding and Setting}
\label{setting}

Several clauses exist for establishing new variable bindings or for
setting variables in the loop.  They all support destructuring.

\begin{clauses}

\clausex{with}{~var~ \opt\ |=| ~value~}
Causes ~var~ to be bound to value before the loop body is entered.
If ~value~ is not supplied, ~var~ assumes a default
binding, which will be 
\nil\ in the absence of declarations.  Also, if ~value~ is not
supplied, no destructuring is performed; instead, ~var~ may be a list of
symbols, all of which are given default 
bindings.  If ~value~ is supplied, ~var~ is bound to it, with
destructuring.  

\cpar Because |with|
creates bindings whose scope includes the entire \iter\ form, it is
good style to put all |with| clauses at the beginning.

\cpar Successive occurrences of |with| result in sequential
bindings (as with 
|let*|).  There is no way to obtain parallel bindings; see
section \ref{bindings} for a rationale.  


\clause{|for| ~var~ |=| ~expr~}
\clindexx{for}{=}
On each iteration, ~expr~ is evaluated and ~var~ is set
to its value. 

\cpar This clause may appear to do the same thing as |for\dots next|.
In fact, they are quite different.  |for\dots =| provides only three
services: it sets up a binding for ~var~, sets it to ~expr~ on each
iteration, and makes it possible to use |for\dots previous| with
~var~.  |for\dots next| provides these services in addition to the
ability to turn the driver into a generator.
%Also, the code which sets ~var~ appears in the loop body in the same
%place as the |for\dots =| clause; the code for |for\dots next| appears
%at the top of the loop, as with other drivers (except when being used
%as a generator).

\clause{|for| ~var~ |initially| ~init-expr~ |then| ~then-expr~}
\clindexxx{for}{initially}{then}
Before the loop begins, ~var~ is set to ~init-expr;~ on all
iterations after the first it is set to ~then-expr.~  This clause
must occur at top-level.  ~init-expr~ will be moved outside the loop
body and ~then-expr~ will be moved to the end of the loop body, so
they are subject to code motion problems (see section
\ref{code-movement}). 

\cpar This clause may appear to be similar to |for\dots next|, but in
fact they differ significantly.  |for\dots initially\dots then| is
typically used to give ~var~ its first value before the loop begins,
and subsequent values on following iterations.  This is incompatible
with generators, whose first value and subsequent values must all be
computed by |(next ~var~)|.  Also, the update of ~var~ in 
|for\dots initially\dots then| does not occur at the location of the
clause. 
Use |for\dots initially\dots then| for
one-shot computations where its idiom is more convenient, but use
|for\dots next| for extending \iter\ with new drivers (see section
\ref{extend}).

\clause{|for| ~var~ |first| ~first-expr~ |then| ~then-expr~}
\clindexxx{for}{first}{then}
The first time through the loop, ~var~ is set to ~first-expr;~ on
subsequent iterations, it is set to ~then-expr.~  This differs from
|for\dots initially| in that ~var~ is set to ~first-expr~
inside the loop body, 
so ~first-expr~ may depend on the results of other clauses.  For
instance, 
\begin{program}
(iter (for num in list)
      (for i first num then (1+ i))
      ...)
\end{program}
will set |i| to the first element of |list| on the first
iteration, whereas 
\begin{program}
(iter (for num in list)
      (for i initially num then (1+ i))
      ...)
\end{program}
is probably erroneous; |i| will be bound to |num|'s
default binding (usually \nil) for the first iteration.

\end{clauses}

\begin{note}{Compatibility note}
|loop|'s |for\dots =| works like \iter's, but |loop| used the syntax
|for\dots =\dots then| to mean |for\dots initially\dots then|.  It was
felt that these two operations were sufficiently different to warrant
different keywords.

Also, the |for| in the above three clauses is misleading,
since none is true driver (e.g. none has a corresponding |generate|
form).  |setting| would have been a better choice, but |for| was used
to retain some compatibility with |loop|.  
\end{note}


%%%%%%%%%%%%%%%%%%%%%%%%%%%%%%%%
\subsection{Gathering Clauses}
Many of \iter's clauses accumulate values into a variable, or set a
variable under certain conditions.  At the end of the loop, this
variable contains the desired result.  All these clauses have an
optional |into| keyword, whose argument should be a symbol.  If the
|into| keyword is not supplied, the accumulation variable will be
internally generated and its value will be returned at the end of the
loop; if a variable is specified, that variable is used for the
accumulation, and is not returned as a result---it is up to the user
to return it explicitly, in the loop's epilogue code (see |finally|).
It is safe to examine the accumulation variable during the loop, but
it should not be modified.

These clauses all begin with a verb.  When the verb does
not conflict with an existing Common Lisp function, then it may be
used in either its infinitival or present-participle form (e.g. |sum|,
|summing|).  However, when there is a conflict with Common Lisp, only
the present-participle form may be used (e.g. |unioning|).  This is to
prevent \iter\ clauses from clashing with Common Lisp
functions. 

%although these clauses are described as ``producing a value,'' it is a
%mistake to think of the lisp list representing the clause as a
%value-producing form in the usual way.  clauses may legally be written
%where a value is expected, e.g. |(setq x (sum i))|, but the
%lisp value of a clause in such a context is undefined.

\subsubsection{Reductions}

~Reduction~ is an extremely common iteration pattern in which
the results of successive applications of a
binary operation are accumulated.
For example, a loop that computes the sum of the
elements of a list is performing a reduction with the addition
operation.  This could be written in Common Lisp as  |(reduce \#'+
list)| or with \iter\ as
\begin{program}
(iter (for el in list)
      (sum el))
\end{program}

\begin{clauses}

\clausex{sum}{~expr~ \opt\ |into| ~var~}
Each time through the loop, ~expr~ is evaluated and added to a
variable, which is bound initially to zero.  If ~expr~ has a type,
it is ~not~ used as the type of the sum variable, which is always
|number|.  To get the result variable to be of a more specific
type, use an explicit variable, as in
\begin{program}
(iter (for el in number-list)
      (sum el into x)
      (declare (fixnum x))
      (finally (return x)))
\end{program}

\clausex{multiply}{~expr~ \opt\ |into| ~var~}
Like |sum|, but the initial value of the result variable is $1$,
and the variable is updated by multiplying ~expr~ into it.

\clausex{counting}{~expr~ \opt\ |into| ~var~}
~expr~ is evaluated on each iteration.  If it is \nonnil, the
accumulation variable, initially zero, is incremented.

\startitem
\Clausex{maximize}{~expr~ \opt\ |into| ~var~}
\Clausex{minimize}{~expr~ \opt\ |into| ~var~}
\finishitem
~expr~ is evaluated on each iteration and its extremum (maximum or
minimum) is stored in the accumulation variable.  If ~expr~ is never
evaluated, then the result is \nil\ (if the accumulation variable is
untyped) or $0$ (if it has a numeric type).

\clausex{reducing}{~expr~ |by| ~func~ \opt\ |initial-value| ~init-val~ |into| ~var~}
This is a general way to perform reductions. ~func~ should be a
function of two arguments, the first of which will be the value
computed so far and
the second of which will be the value of ~expr~.  It should return the
new value.
|reducing| is roughly equivalent to the
Common Lisp |(reduce ~func~ ~sequence~ :key ~expr-function~)|, where
~expr-function~ is used to derive values from the successive elements
of ~sequence~.

\cpar If the |reducing| clause is never executed, the result is
undefined.

\cpar It is not necessary to provide an
initial value, but better code can be generated if one is supplied.
Regardless of its location in the \iter\ body, ~init-val~ will be
evaluated before the loop is entered, so it should not depend on any
value computed inside the \iter\ form.  

%\cpar if a ~var~ is not specified, you can get \iter\ to declare the
%type of the internal variable by putting a |the| expression around
%~func.~  see section \ref{types}.

\end{clauses}

%\begin{note}{implementation note}
%in principle, |maximize| and |minimize| can be thought of
%as reductions where the initial value is the smallest (or largest)
%value that the accumulation variable can assume.  because lisp's
%bignums can represent arbitrary integers, these clauses cannot be
%implemented as reductions in general.  if, however, the type of 
%~expr~ or ~var~ can be determined to be a fixnum
%or a float, \iter\ will implement the clause as a true reduction,
%using one of the constants |most-negative-fixnum|, |%most-positive-fixnum|, 
%|most-negative-short-float|, etc. as appropriate.
%\end{note}

\subsubsection{Accumulations}

All the predefined accumulation clauses add values to a sequence.  If
the sequence is a list, they all 
have the property that the partial list is kept in the correct order
and available for inspection at any point in the loop.

\begin{clauses}

\clausex{collect}{~expr~ \opt\ |into| ~var~ |at| ~place~ |result-type| ~type~}
Produces a sequence of the values of ~expr~ on each iteration. ~place~
indicates where the next value of ~expr~ is added to the list and may
be one of the symbols |start|, |beginning| (a synonym for |start|) or
|end|.  The symbol may be quoted, but need not be.  The default is
|end|.  For example,
\begin{program}
(iter (for i from 1 to 5)
      (collect i))
\end{program}
produces |(1 2 3 4 5)|, whereas
\begin{program}
(iter (for i from 1 to 5)
      (collect i at beginning))
\end{program}
produces |(5 4 3 2 1)| (and is likely to be faster in most Common Lisp
implementations).

\cpar If ~type~ is provided, it should be a subtype of |sequence|.
The default is |list|.  Specifying a type other than |list| will
result in |collect| returning a sequence of that type.  ~However,~ the
type of the sequence being constructed when inside the loop body is
undefined when a non-|list| type is specified.  (As with ~place~,
quoting ~type~ is optional.)

\clausex{adjoining}{~expr~ \opt\ |into| ~var~ |test| ~test~ |at| ~place~ |result-type| ~type~}
Like |collect|, but only adds the value of ~expr~ if it is not
already present.  ~test,~ which defaults to |\#'eql|, is
the test to be used with |member|.

\startitem
\Clausex{appending}{~expr~ \opt\ |into| ~var~ |at| ~place~}
\Clausex{nconcing}{\ ~expr~ \opt\ |into| ~var~ |at| ~place~}
\Clausex{unioning}{\ ~expr~ \opt\ |into| ~var~ |test| ~test~ |at| ~place~}
\Clausex{nunioning}{~expr~ \opt\ |into| ~var~ |test| ~test~ |at| ~place~}
\finishitem
These are like |collect|, but behave like the Common Lisp functions
|append|, |nconc|, |union| or |nunion|.  
As in Common Lisp, they work only on lists.  Also as in Common Lisp,
|unioning| and |nunioning| assume that the value of ~expr~ contains no
duplicates. 

\clausex{accumulate}{~expr~ |by| ~func~ \opt\ |initial-value| ~init-val~ |into| ~var~}
This is a general-purpose accumulation clause. ~func~ should be a
function of two arguments, the value of ~expr~ and the value
accumulated so far in the iteration, and it should return the updated
value.  If no initial value is supplied, \nil\ is used.  

%\cpar If a ~var~ is not specified, you can get \iter\ to declare the
%type of the internal variable by putting a |the| expression around
%~func.~  see section \ref{types}.

\cpar The differences between |accumulate| and |reducing| are slight.
One difference is that the functions take their arguments in a
different order.  Another is that in the absence of ~init-val~,
|accumulate| will use \nil, whereas |reducing| will generate different
code that avoids any dependence on the initial value.
The reason for having both clauses is that one usually
thinks of reductions (like |sum|) and accumulations (like |collect|)
as different beasts.

\end{clauses}


\subsubsection{Finders}

A ~finder~ is a clause whose value is an expression that meets some
condition.

\begin{clauses}

\clause{|finding| ~expr~ |such-that| ~test~ \opt\ |into| ~var~ |on-failure| ~failure-value~}
\clindexx{finding}{such-that}
If ~test~ (which is an expression) ever evaluates to \nonnil, the loop
is terminated, the 
epilogue code is run and the value of ~expr~ is returned.  Otherwise,
\nil\ (or ~failure-value,~ if provided) is returned.  If ~var~ is
provided, it will have either the \nonnil\ value of ~expr~ or
~failure-value~ when the epilogue code is run.

\cpar As a special case, if the ~test~ expression is a sharp-quoted
function, then it is applied to ~expr~ instead of being simply
evaluated.  E.g. |(finding x such-that \#'evenp)| is equivalent to
|(finding x such-that (evenp x))|.  

%\cpar although ~test~ need have nothing to do with ~%expr~ as in
%|(finding j such-that (> i 3))|, it usually 
%will: |(finding (length el) such-that (oddp (length el)))|.  to
%avoid performing the |length| computation twice, you could write
%|(finding (length el) such-that \#'oddp)| or |(finding (length
%el) such-that 'oddp)|; for these cases, \iter\ generates code that
%executes ~expr~ only once.  the code for |\#'oddp|
%is slightly different from that for {\lisp 'oddp}; see the discussion
%under {\lisp for\dots in} and {\lisp for\dots on}.

\cpar |On-failure| is a misnomer. Because it is always evaluated, it behaves
more like the default third argument to the |gethash| function. As a result,
|on-failure (error "Not found")| makes no sense. Instead, the clauses |leave|
or |thereis| can be used in conjunction with |finally| as follows:
\begin{program}
(iter (for x in '(1 2 3))
      (if (evenp x) (leave x))
      (finally (error "not found")))
\end{program}

\cpar This clause may appear multiple times when all defaults are
identical. It can also be used together with either |always|/|never| or
|thereis| if their defaults match. More specifically, |on-failure nil| is
compatible with |thereis|, while |on-failure t| is compatible with |always|
and |never| clauses.
\begin{program}
(iter (for i in '(7 -4 2 -3))
      (if (plusp i)
	  (finding    i  such-that (evenp i))
          (finding (- i) such-that (oddp i))))
\end{program}

\startitem
\Clause{|finding| ~expr~ |maximizing| ~m-expr~ \opt\ |into| ~var~}
\clindexx{finding}{maximizing}
\Clause{|finding| ~expr~ |minimizing| ~m-expr~ \opt\ |into| ~var~}
\clindexx{finding}{minimizing}
\finishitem
Computes the extremum (maximum or minimum) value of ~m-expr~ over all
iterations, and returns the value of ~expr~ corresponding to the
extremum.  ~expr~ is evaluated inside the loop at the time the new
extremum is established.  If ~m-expr~ is never evaluated (due to, for
example, being embedded in a conditional clause), then the returned
value depends on the type, if any, of ~expr~ (or ~var,~ if
one is supplied).  If there is no type, the returned
value will be \nil; if the type is numeric, the returned value will be
zero.

\cpar For these two clauses, ~var~ may be a list of two
symbols; in that case, the first is used to record ~expr~ and
the second, ~m-expr.~  

\cpar As with |finding\dots such-that|, if ~m-expr~ is a sharp-quoted
function, then it is called on ~expr~ instead of being evaluated.

\end{clauses}

\subsubsection{Boolean Tests}

\begin{clauses}

\clausex{first-iteration-p}{}
Returns |t| in the first cycle of the loop, otherwise \nil.

\clausex{first-time-p}{}
Returns |t| the first time the expression is evaluated, and then \nil\ forever.
This clause comes handy when printing (optional) elements separated
by a comma:

\begin{program}
(iter (for el in '(nil 1 2 nil 3))
      (when el
        (unless (first-time-p)
          (princ ", "))
        (princ el)))
\end{program}
produces |"1, 2, 3"|.

\end{clauses}

\subsubsection{Aggregated Boolean Tests}

\begin{clauses}

\clausex{always}{~expr~}
If ~expr~ ever evaluates to
\nil, then \nil\ is immediately returned; the epilogue code is not
executed.  If ~expr~ never evaluates to \nil, the epilogue code
is executed and the last value of ~expr~ (or |t| if ~expr~ was never
evaluated) is returned (whereas |loop| would constantly return |t|).

% mention last evaluated clause when multiple always clauses?

\clausex{never}{~expr~}
Like |(always (not ~expr~))|, except it does not influence the last
value returned by a possible other |always| clause. That is,
\begin{program}
(iter (repeat 2)
      (always 2)
      (never nil)) \yields 2 ; not t
\end{program}

\clausex{thereis}{~expr~}
If ~expr~ is ever \nonnil,
its value is immediately returned without running epilogue code.
Otherwise, the epilogue code is performed and \nil\ is returned.

This clause cannot be used together with |always| or |never|, because their
defaults are opposed (similarly, |(loop always 3 thereis nil)| refuses to
compile in some implementations of |loop|).

\end{clauses}


\subsection{Control Flow}
\label{control-flow}
Several clauses can be used to alter the usual flow of control in a loop.

Note: the clauses of this and subsequent sections don't adhere to \iter's
usual syntax, but instead use standard Common Lisp syntax.  Hence the
format for describing syntax subsequently is
like the standard format used in the Common Lisp manual, not like the
descriptions of clauses above.

\begin{clauses}

\clausex{finish}{}
Stops the loop and runs the epilogue code.

%for example:
%
%\begin{program}
%(iter (with answer = nil)
%         (initially (make-a-mess))
%         (for i from 1 to 10)
%         (when (correct? i)
%           (setq answer i)
%           (finish))
%         (finally (cleanup)))
%\end{program}
%
%this code will execute |cleanup| whether or not the test |(correct?
%i)| ever succeeds. 
%the (more elegant) formulation,
%\begin{program}
%(iter (initially (make-a-mess))
%         (for i from 1 to 10)
%         (finding i such-that (correct? i))
%         (finally (cleanup)))
%\end{program}
%would not execute |cleanup| if |(correct? i)| succeeded; it
%would do an immediate return.

\clausex{leave}{\opt\ ~value~}
Immediately returns ~value~ (default \nil) from the current \iter\
form, skipping the epilogue code.  Equivalent to using |return-from|.

\clausex{next-iteration}{}
Skips the remainder of the loop body and begins the next iteration of
the loop.

\clausex{while}{~expr~}
If ~expr~ ever evaluates to \nil, the loop is terminated and the
epilogue code executed.  Equivalent to |(if (not ~expr~) (finish))|.

\clausex{until}{~expr~}
Equivalent to |(if ~expr~ (finish))|.

\clausex{if-first-time}{~then~ \opt\ ~else~}
If this clause is being executed for the first time in this invocation
of the \iter\ form, then the ~then~ code is evaluated; otherwise the
~else~ code is evaluated.

\cpar |(for ~var~ first ~expr1~ then ~expr2~)| is almost equivalent to
\begin{program}
(if-first-time (dsetq ~var~ ~expr1~)
               (dsetq ~var~ ~expr2~))
\end{program}
The only difference is that the |for| version makes ~var~ available
for use with |for\dots previous|.

\end{clauses}

\subsection{Code Placement}
When fine control is desired over where code appears in a loop
generated by \iter, the following special clauses may be useful.
They are all subject to code-motion problems (see section
\ref{code-movement}).  

\begin{clauses}

\clausex{initially}{|\&rest| ~forms~}
The lisp ~forms~ are placed in the prologue section of the loop, where
they are executed once, before the loop body is entered.

\clausex{after-each}{|\&rest| ~forms~}
The ~forms~ are placed at the end of the loop body, where they
are executed after each iteration.  Unlike the other clauses in this
section, ~forms~ may contain \iter\ clauses.

\clausex{else}{|\&rest| ~forms~}
The lisp ~forms~ are placed in the epilogue section of the loop, where they
are executed if this |else| clause is never met during execution of the
loop and the loop terminates normally.

\clausex{finally}{|\&rest| ~forms~}
The lisp ~forms~ are placed in the epilogue section of the loop, where
they are executed after the loop has terminated normally.

\clausex{finally-protected}{|\&rest| ~forms~}
The lisp ~forms~ are placed in the second form of an unwind-protect
outside the loop.  They are always executed after the loop has
terminated, regardless of how the termination occurred.

\end{clauses}


%%%%%%%%%%%%%%%%%%%%%%%%%%%%%%%%%%%%%%%%%%%%%%%%%%%%%%%%%%%%%%%%
\section{Other Features}

\subsection{Multiple Accumulations}
\label{multiple}

\begin{sloppypar}
It is permitted to have more than one clause accumulate into the same
variable, as in the following:
\begin{program}
(iter (for i from 1 to 10)
      (collect i into nums)
      (collect (sqrt i) into nums)
      (finally (return nums)))
\end{program}
Clauses can only accumulate into the same variable if they are
compatible.  |collect|, |adjoining|, |appending|, |nconcing|,
|unioning| and |nunioning| are compatible with each other; |sum|,
|multiply| and |counting| are compatible; |always| and |never| are compatible;
|finding| \dots |such-that| is compatible with either |thereis| or |always|
and |never| when their defaults match; and |maximize| and |minimize| clauses
are compatible only with other |maximize| and |minimize| clauses,
respectively.

%note that the same variable ~cannot~ be both an accumulation
%variable and an ordinary variable; there can be only one variable with
%a given name within an \iter\ form. 
\end{sloppypar}

\subsection{Named Blocks}

Like Common Lisp |block|s, \iter\ forms can be given names.  The
name should be a single symbol, and it must be the first form in the
\iter.  The generated code behaves exactly like a named block; in
particular, |(return-from ~name~)| can be used to exit it:
\begin{program}
(iter fred
      (for i from 1 to 10)
      (iter barney
            (for j from i to 10)
            (if (> (* i j) 17)
                (return-from fred j))))
\end{program}
An \iter\ form that is not given a name is implicitly named \nil.

Sometimes one would like to write an expression in an inner \iter\ form,
but have it processed by an outer \iter\ form.  This is possible with
the |in| clause.  

\begin{clauses}

\clausex{in}{~name~ |\&rest| ~forms~}
Evaluates ~forms~ as if they were part of the \iter\ form named
~name~.  In other words, \iter\ clauses are processed by the \iter\
form named ~name,~ and not by any \iter\ forms that occur inside ~name.~

\cpar As an example, consider the problem of collecting a list of the
elements in a two-dimensional array.  The naive solution,
\begin{program}
(iter (for i below (array-dimension ar 0))
      (iter (for j below (array-dimension ar 1))
            (collect (aref ar i j))))
\end{program}
\noindent is wrong because the list created by the inner \iter\ is simply
ignored by the outer one.  But using |in| we can write:
\begin{program}
(iter outer (for i below (array-dimension ar 0))
      (iter (for j below (array-dimension ar 1))
            (in outer (collect (aref ar i j)))))
\end{program}
\noindent which has the desired result.

\end{clauses}

\subsection{Destructuring}
\label{destructuring}

In many places within \iter\ clauses where a variable is expected, a
list can be written instead.  In these cases, the value to be assigned
is ~destructured~ according to the pattern
described by the list.  As a simple example, the clause
\begin{program}
(for (key . item) in alist)
\end{program}
\noindent will result in |key| being set to the |car| of
each element in |alist|, and |item| being set to the |cdr|.  The
pattern list may be nested to arbitrary depth, and (as the example
shows) need not be terminated with \nil; the only requirement is that
each leaf be a bindable symbol (or \nil, in which case no binding is
generated for that piece of the structure).

Sometimes, you might like to do the equivalent of a
|multiple-value-setq| in a clause.  This 
``multiple-value destructuring'' can be expressed by writing 
\linebreak |(values $pat_1$ $pat_2 \ldots$)| for a destructuring
pattern, as in 
\begin{program}
(for (values (a . b) c d) = (three-valued-function ...))
\end{program}
\begin{sloppypar}
Note that the $pat_i$ can themselves be destructuring patterns (though
not multiple-value destructuring patterns).  You can't do multiple-value
destructuring in a |with| clause; instead wrap the whole \iter\
form in a |multiple-value-bind|.
\end{sloppypar}

\begin{note}{Rationale}
There are subtle interactions between variable declarations and
evaluation order that make the correct implementation of
multiple-value destructuring in a |with| somewhat tricky.
\end{note}

The destructuring feature of \iter\ is available as a separate
mechanism, using the |dsetq| macro:

\begin{clauses}
\defmacro{dsetq}{template expr}
Performs destructuring of ~expr~ using ~template~.   May be used
outside of an \iter\ form. Yields the primary value of ~expr~.

\end{clauses}

\subsection{On-line Help}

\begin{sloppypar}
There is a limited facility for on-line help, in the form of the 
|display-iterate-clauses| function.
\end{sloppypar}

\begin{clauses}

\defun{display-iterate-clauses}{\opt\ clause-spec}
Displays a list of \iter\ clauses.  If ~clause-spec~ is not
provided, all clauses are shown; if it is a symbol, all clauses
beginning with that symbol are shown; and if it is a list of symbols,
all clauses for which ~clause-spec~ is a prefix are shown.

\end{clauses}

\subsection{Parallel Binding and Stepping}
\label{bindings}

The parallel binding and stepping of variables is a feature that
\iter\ does ~not~ have.  This section attempts to provide a rationale.

We say that two variables are bound ~in parallel~ if neither
binding shadows the other.  This is the usual semantics of |let|
(as opposed to |let*|).  Similarly, we can say that iteration
variables are stepped in parallel if neither variable is updated
before the other, conceptually speaking; in other words, if the code
to update each variable can reference the old values of both variables.

|loop| allows parallel binding of variables and parallel stepping of
driver variables.  My view is that if you are depending on the 
serial/parallel distinction, you are doing something obscure.  If you
need to bind 
variables in parallel using |with|, then you must be using a
variable name that shadows a name in the existing lexical environment.
Don't do that.  The most common use for parallel stepping is to track
the values of variables on the previous iteration, but in fact this
does not require parallel stepping at all; the following will work:
\begin{program}
(iter (for current in list)
      (for prev previous current)
      ...)
\end{program}


%%%%%%%%%%%%%%%%%%%%%%%%%%%%%%%%%%%%%%%%%%%%%%%%%%%%%%%%%%%%%%%%
\section{Types and Declarations}
\label{types}

\subsection{Discussion}

Sometimes efficiency dictates that the types of variables be declared.
This type information needs to be communicated to \iter\ so it can
bind variables to appropriate values.  Furthermore, \iter\ must often
generate internal variables invisible to the user; there needs to be a
way for these to be declared.

As an example, consider this code, which will return the number of
odd elements in |number-list|:
\begin{program}
(iter (for el in number-list)
      (count (oddp el)))
\end{program}
In processing this form,
\iter\ will create an internal variable, let us call it |list17|, to
hold the successive |cdr|s of |number-list|, and  
will bind the variable to |number-list|.  It will also generate a
default binding for |el|; only inside the body of the loop will |el|
be set to the |car| of |list17|.  Finally, \iter\ will generate a
variable, call it |result|, to hold the result of the count, and will
bind it to zero.

When dealing with type declarations, \iter\ observes one simple rule:
~it will never generate a declaration unless requested to do so.~  The
reason is that such declarations might mask errors in compiled code by
avoiding error-checks; the resulting problems would be doubly hard to
track down because the declarations would be hidden from the
programmer.  Of course, a compiler might omit error-checks even in the
absence of declarations, though this behavior can usually be avoided,
e.g. by saying |(declaim (optimize (safety 3)))|.

So, the above \iter\ form will generate code with no declarations.
But say we wish to declare the types of |el| and the internal
variables |list17| and |result|.  How is this done?

Declaring the type of |el| is easy, since the programmer knows
the variable's name:
\begin{program}
(iter (for el in number-list)
      (declare (fixnum el))
      (counting (oddp el)))
\end{program}
\iter\ can read variable type declarations like this one.  Before
processing any clauses, it scans the entire top-level form for type
declarations 
and records the types, so that variable bindings can be performed
correctly.  In this case, |el| will be bound to zero
instead of \nil.  Also, \iter\ collects all the top-level declarations
and puts them at the begining of the generated code, so it is not
necessary to place all declarations at the beginning of an \iter\
form; instead, they can be written near the variables whose types they
declare.  

Since \iter\ is not part of the compiler, it will not know
about declarations that occur outside an \iter\ form; these
declarations must be repeated inside the form.

Here is another way we could have declared the type of |el|:
\begin{program}
(iter (for (the fixnum el) in number-list)
      (counting (oddp el)))
\end{program}
\iter\ extends the Common Lisp |the|\lispindex{the}
form to apply to variables as well as value-producing forms; anywhere
a variable is allowed---in a |with| clause, as the iteration
variable in a driver clause, as the |into| argument of an
accumulation clause, even inside a destructuring template---you can
write |(the ~type~ ~symbol~)| instead.

There is one crucial difference between using a |the| form and
actually declaring the variable: explicit declarations are always
placed in the generated code, but type information from a |the|
form is not turned into an actual declaration unless you tell \iter\
to do so using |iterate:declare-variables|.  See below.

\begin{sloppypar}
Declaring the types of internal variables is harder than declaring the
types of explicitly mentioned variables, since their names
are unknown.  You do it by declaring |iterate:declare-variables|
somewhere inside the top level of the \iter\ form.  (This will also
generate declarations for variables declared using |the|.)
\iter\ does not provide much selectivity here: it's all or none. 
And unfortunately, since \iter\ is not privy to compiler information
but instead reads declarations itself, it will not hear if you
|(declaim (iterate:declare-variables))|.  Instead, set the variable 
|iterate::*always-declare-variables*| to |t| at
compile-time, using |eval-when|.
\end{sloppypar}

To determine the appropriate types for internal variables, \iter\ uses
three sources of information:

\begin{itemize}

\item Often, the particular clause dictates a certain type for a
variable; \iter\ will use this information when available.  In the
current example, the variable |list17| will be given the type
|list|, since that is the only type that makes sense; and the
variable |result| will be given the type |fixnum|, on the
assumption that you will not be counting high enough to need bignums.
You can override this assumption only by using and explicitly declaring a
variable: 
\begin{verbatim}
(iter (declare (iterate:declare-variables))
      (for el in number-list)
      (count (oddp el) into my-result)
      (declare (integer my-result))
      (finally (return my-result)))
\end{verbatim}

\begin{sloppypar}
Other examples of the type assumptions that \iter\ makes are: type
|list| for |into| variables of collection clauses; type |list| for
expressions that are to be destructured; type |vector| for the
variable holding the vector in a |for\dots in-vector| clause, and
similarly for |string| and the |for\dots in-string| clause;
and the implementation-dependent type
|(type-of array-dimension-limit)| for the index and limit
variables generated by sequence iteration drivers like |for\dots
in-vector| and |for\dots in-string| (but not |for\dots in-sequence|,
because it may be used to iterate over a list). 
\end{sloppypar}

\item Sometimes, \iter\ will examine expressions and try to determine
their types in a simple-minded way.  If the expression is
self-evaluating (like a number, for instance), \iter\ knows that the
expression's type is the same as the type of the value it denotes, so
it can use that type.  If the expression is of the form |(the ~type~
~expr~)|, \iter\ is smart enough to extract ~type~ and use it.
However, the current version of \iter\ does not  
examine declarations of function result types or do any type
inference.  It will not determine, for 
example, that the type of |(+ 3 4)| is |fixnum|, or even
|number|.

\item In some cases, the type of an internal variable should match the
type of some other variable.  For instance, \iter\ generates an
internal variable for |(f x)| in the
clause |(for i from 1 to (f x))|, and in the absence of other
information will give it the same type as |i|.  If, however, the
expression had been written |(the fixnum (f x))|, then \iter\
would have given the internal variable the type |fixnum|
regardless of |i|'s type.  The type incompatibility errors that
could arise in this situation are not checked for.

\end{itemize}

Note that if you do declare |iterate:declare-variables|, then
\iter\ may declare user variables as well as internal ones if they do
not already have declarations, though only for variables that it
binds.  For instance, in this code:

\begin{program}
(iter (declare (iterate:declare-variables))
      (for i from 1 to 10)
      (collect i into var))
\end{program}
the variable |var| will be declared to be of type |list|.


\subsection{Summary}

\iter\ understands standard Common Lisp variable type declarations
that occur within an \iter\ form and
will pass them through to the generated code.  If the declaration
|(iterate:declare-variables)|\lispindex{declare-variables}
appears at the top level of an 
\iter\ form, or if 
|iterate::*always-declare-variables*|\lispindex{*always-declare-variables*} 
is \nonnil, then \iter\ will use the type information gleaned from user
declarations, self-evaluating expressions and |the| expressions,
combined with reasonable assumptions, to determine variable
types and declare them.


%%%%%%%%%%%%%%%%%%%%%%%%%%%%%%%%%%%%%%%%%%%%%%%%%%%%%%%%%%%%%%%%
\section{Problems with Code Movement}
\label{code-movement}

Some \iter\ clauses, or parts of clauses, result in code being
moved from the location of the clause to other parts of the loop.
Drivers behave this way, as do code-placement clauses like |initially|
and |finally|.  When using these clauses, there is a danger of writing
an expression that makes sense in its apparent location but will be
invalid or have a different meaning in another location.  For example:
\begin{program}
(iter (for i from 1 to 10)
      (let ((x 3))
        (initially (setq x 4))))
\end{program}
While it may appear that the |x| of |(initially (setq x 4))| is the
same as the |x| of |(let ((x 3)) \dots|, in fact they are not:
|initially| moves its code outside the loop body, so |x| would refer
to a global variable.  Here is another example of the same problem:
\begin{program}
(iter (for i from 1 to 10)
      (let ((x 3))
        (collect i into x)))
\end{program}
If this code were executed, |collect| would create a binding for its
|x| at the top level of the \iter\ form that the |let| will shadow.

Happily, \iter\ is smart enough to catch these errors; it
walks all problematical code to ensure that free variables are not
bound inside the loop body, and checks all variables it binds for the
same problem.

However, some errors cannot be caught:

\begin{program}
(iter (with x = 3)
      (for el in list)
      (setq x 1)
      (reducing el by \#'+ initial-value x))
\end{program}
|reducing| moves its |initial-value| argument to the initialization
part of the loop in order to produce more efficient code.  Since
\iter\ does not perform data-flow analysis, it cannot determine that
|x| is changed inside the loop; all it can establish is that |x| is
not bound internally.  Hence this code will not signal an
error and will use $3$ as the initial value of the reduction.

The following list summarizes all cases that are subject to these code
motion and variable-shadowing problems.
\begin{itemize}
\item Any variable for which \iter\ creates a binding, including those
used in |with| and the |into| keyword of many clauses.

\begin{sloppypar}
\item The special clauses which place code: |initially|, |after-each|, |else|,
|finally| and |finally-protected|. 
\end{sloppypar}

\item The variables of a |next| or |do-next| form.

\item The |initially| arguments of |for\dots initially\dots then| and
|for\dots previous|. 

\item The |then| argument of |for\dots initially\dots then|.

\item The |initial-value| arguments of |reducing| and |accumulate|.

\item The |on-failure| argument of |finding\dots such-that|.
\end{itemize}

%%%%%%%%%%%%%%%%%%%%%%%%%%%%%%%%%%%%%%%%%%%%%%%%%%%%%%%%%%%%%%%%
\section{Differences Between {\tt Iterate} and {\tt Loop}}
|loop| contains a great deal of complexity which \iter\ tries to
avoid.  Hence many esoteric features of |loop| don't exist in \iter.
Other features have been carried over, but in a cleaned-up form.  
And of course, many new features have been added; they are not
mentioned in this list.

\begin{itemize}

\item \iter's syntax is more Lisp-like than |loop|'s, having a higher
density of parens.

\item The current implementation of \iter, unlike the current version
of  |loop| (as documented in {\em Common Lisp, 2nd Ed.\/}), is
extensible (see section \ref{extend}).  

\item |loop| puts the updates of all driver variables at the top of
the loop; \iter\ leaves them where the driver clauses appear.

\item While for the most part \iter\ clauses that resemble |loop| clauses
behave similarly, there are some differences.  For instance, there is
no |for\dots =\dots then| in \iter; instead use 
|for\dots initially\dots then|.  

\item |loop| binds the variable |it| at certain times
to allow pseudo-English expressions like |when ~expr~ return it|. 
In \iter, you must bind ~expr~ to a variable yourself.  Note that
|when ~expr~ return it| is like |thereis ~expr~| except that the latter is an
accumulation clause and therefore competes with other accumulations
(remember section \ref{multiple} above).

% repeat different behaviour of |always| clause here?

\item |loop| has a special |return| clause, illustrated in the
previous item.  \iter\ doesn't need one,  since an ordinary Lisp
|return| has the same effect.

\item |loop| allows for parallel binding and stepping of iteration
variables.  \iter\ does not.  (See section \ref{bindings}.)

\item |loop| and \iter\ handle variable type declarations very
differently.  |loop| provides a special syntax for declaring variable
types, and does not examine declarations.  Moreover, the standard
implementation of |loop| will
generate declarations when none are requested.
\iter\ parses standard Common Lisp type declarations, and will never
declare a variable itself unless declarations are 
specifically requested.

\end{itemize}

%%%%%%%%%%%%%%%%%%%%%%%%%%%%%%%%%%%%%%%%%%%%%%%%%%%%%%%%%%%%%%%%
\section{Rolling Your Own}
\label{extend}

\subsection{Introduction}

\iter\ is extensible---you can write new clauses that embody new
iteration patterns.  You might want to write a new driver clause for a
data structure of your own, or you might want to write a clause that
collects or manipulates elements in a way not provided by \iter. 

This section describes how to write clauses for \iter.  Writing a
clause is like writing a macro.  In fact, writing a clause ~is~
writing a macro: since \iter\ code-walks its body and macroexpands,
you can add new abstractions to \iter\ with good old |defmacro|.

Actually, there are two extensions you can make to \iter\ that are
even easier than writing a macro.  They are adding a synonym for an
existing clause and defining a driver clause for an indexable
sequence.  These can be done with |defsynonym| and
|defclause-sequence|, respectively.  See section \ref{aids}, below.

The rest of this section explains how to write macros that expand into
\iter\ clauses.
Here's how you could add a simplified version of \iter's |multiply|
clause, if \iter\ didn't already have one:
\begin{program}
(defmacro multiply (expr)
  `(reducing ,expr by \#'* initial-value 1))
\end{program}

If you found yourself summing the square of an expression often, you
might want to write a macro for that.  A first cut might be
\begin{program}
(defmacro sum-of-squares (expr)
  `(sum (* ,expr ,expr)))
\end{program}
but if you are an experienced macro writer, you will realize that this
code will evaluate ~expr~ twice, which is probably a bad idea.  A
better version would use a temporary:
\begin{program}
(defmacro sum-of-squares (expr)
  (let ((temp (gensym)))
    `(let ((,temp ,expr))
       (sum (* ,temp ,temp)))))
\end{program}
Although this may seem complex, it is just the sort of thing you'd
have to go through to write any macro, which illustrates the point of
this section: if you can write macros, you can extend \iter.

Our macros don't use \iter's keyword-argument
syntax.  We could just use keywords with |defmacro|, but we would
still not be using \iter's clause indexing mechanism.  Unlike Lisp,
which uses just the first symbol of a form to determine what function
to call, \iter\ individuates clauses by the list of required keywords.
For instance, |for\dots in| and |for\dots in-vector| are different clauses
implemented by distinct Lisp functions.  

To buy into this indexing scheme, as well as the keyword-argument
syntax, use |defmacro-clause|:

\begin{clauses}

\defmacro{defmacro-clause}{arglist |\&body| body}
Defines a new \iter\ clause.  ~arglist~ is a list of symbols which are
alternating keywords and arguments.  \opt\ may be used, and the list
may be terminated by |\&sequence|.  ~body~ is an ordinary macro body,
as with |defmacro|.  If the first form of ~body~ is a string, it is
considered a documentation string and will be shown by
|display-iterate-clauses|. |defmacro-clause| will signal an error if 
defining the clause would result in an ambiguity.  E.g. you cannot
define the clause |for\dots from| because there would be no way to
distinguish it from a use of the |for| clause with optional keyword |from|.

\end{clauses}

\medskip

Here is |multiply| using |defmacro-clause|.  The keywords are capitalized
for readability.
\begin{program}
(defmacro-clause (MULTIPLY expr \opt\ INTO var)
  `(reducing ,expr by \#'* into ,var initial-value 1))
\end{program}
You don't have to worry about the case when |var| is not supplied; for
any clause with an |into| keyword, saying |into nil| is equivalent to
omitting the |into| entirely.

As another, more extended example, consider the fairly common
iteration pattern that 
involves finding the sequence element that maximizes (or minimizes) some
function.  \iter\ provides this as |finding\dots maximizing|, but it's
instructive to see how to write it.
Here, in pseudocode, is how you might write such a loop for
maximizing a function F:

\begin{quote}
\begin{tabbing}
set variable MAX-VAL to NIL; \\
set variable WINNER to NIL; \\
for\=\ each element EL in the sequence \\
\>  if\=\ MAX-VAL\=\ is NIL or F(EL) $>$ MAX-VAL then \\
\>\>    set MAX-VAL to F(EL); \\
\>\>    set WINNER to EL; \\
\>  end if; \\
end for; \\
return WINNER.
\end{tabbing}
\end{quote}

Here is the macro:
\begin{program}
(defmacro-clause (FINDING expr MAXIMIZING func \opt\ INTO var)
  (let ((max-val (gensym))
        (temp1 (gensym))
        (temp2 (gensym))
        (winner (or var iterate::*result-var*)))
    `(progn 
       (with ,max-val = nil)
       (with ,winner = nil)
       (cond 
        ((null ,max-val)
         (setq ,winner ,expr)
         (setq ,max-val (funcall ,func ,winner))
        (t
         (let* ((,temp1 ,expr)
                (,temp2 (funcall ,func ,temp1)))
           (when (> ,temp2 ,max-val)
             (setq ,max-val ,temp2)
             (setq ,winner ,temp1))))))
       (finally (leave ,winner)))))
\end{program}
Note that if no |into| variable is supplied, we use
|iterate::*result-var*|, which contains the internal variable into
which all clauses place their results.  If this variable is bound by
some clause, then \iter\ will return its value automatically;
otherwise, \nil\ will be returned.  

\subsection{Writing Drivers}

In principle, drivers can be implemented just as easily as other
\iter\ clauses.  In practice, they are a little harder to get right. 
As an example, consider writing a driver that
iterates over all the 
elements of a vector, ignoring its fill-pointer.  |for\dots in-vector|
won't work for this, because it observes the fill-pointer.  It's necessary to
use |array-dimension| instead of |length| to obtain the size of the
vector.  Here is one approach:

\begin{program}
(defmacro-clause (FOR var IN-WHOLE-VECTOR v)
  "All the elements of a vector (disregards fill-pointer)"
  (let ((vect (gensym))
        (index (gensym)))
    `(progn
       (with ,vect = ,v)
       (for ,index from 0 below (array-dimension ,vect 0))
       (for ,var = (aref ,vect ,index)))))
\end{program}
Note that we immediately put |v| in a variable, in case it is an
expression.  Again, this is just good Lisp macrology.  It also has a
subtle effect on the semantics of the driver: |v| is evaluated only
once, at the beginning of the loop, so changes to |v| in the loop have
no effect on the driver.  Similarly, the bounds for numerical iteration
e.g. the above |array-dimension| are also evaluated once only.  This is how
all of \iter's drivers work.


There is an important point concerning the |progn| in this code.  We
need the |progn|, of course, because we are returning several forms,
one of which is a driver.  But \iter\ drivers must occur at top-level.
Is this code in error?  No, because ~top-level~ is defined in \iter\
to include forms inside a |progn|.  This is just the definition of
top-level that Common Lisp uses, and for the same reason: to allow
macros to return multiple forms at top-level.

While our |for\dots in-whole-vector| clause will work, it is not
ideal.  In particular, it does not support generating.  Do do so, we
need to use |for\dots next| or |for\dots do-next|.  The job is
simplified by the |defmacro-driver| macro.

\begin{clauses}

\defmacro{defmacro-driver}{arglist |\&body| body}
Defines a driver clause in
both the |for| and |generate| forms, and provides a parameter
|generate| which ~body~ can examine to determine how it was invoked.
~arglist~ is as in |defmacro-clause|, and should begin with the symbol
|for|. 

\end{clauses}

With |defmacro-driver|, our driver looks like this:
\begin{program}
(defmacro-driver (FOR var IN-WHOLE-VECTOR v)
  "All the elements of a vector (disregards fill-pointer)"
   (let ((vect (gensym))
         (end (gensym))
         (index (gensym))
         (kwd (if generate 'generate 'for)))
     `(progn
        (with ,vect = ,v)
        (with ,end = (array-dimension ,vect 0))
        (with ,index = -1)
        (,kwd ,var next (progn (incf ,index)
                               (if (>= ,index ,end) (terminate))
                               (aref ,vect ,index))))))
\end{program}


We are still missing one thing: the |\&sequence| keywords.  
We can get them easily enough, by writing 
\begin{program}
(defmacro-driver (FOR var IN-WHOLE-VECTOR v \&sequence) 
  ...)
\end{program}
We can now refer to parameters |from|, |to|, |by|, etc. which contain
either the values for the corresponding keyword, or \nil\ if the
keyword was not supplied.  Implementing the right code for these
keywords is cumbersome but not difficult; it is left as an exercise.
But before you begin, see |defclause-sequence| below for an easier way.
       
\subsection{Extensibility Aids}
\label{aids}

This section documents assorted features that may be of use in
extending \iter.

\begin{clauses}

\defunexpvar{*result-var*}
Holds the variable that is used to return a value as a result of the
\iter\ form.  You may examine this and use it in a |with| clause, but
you should not change it.

\defmacro{defsynonym}{syn word}
 Makes ~syn~ a synonym for the existing \iter\ keyword ~word.~  Only
the first word in each clause can have synonyms. 

\defmacroxx{defclause-sequence}{element-name index-name}{|\&key|
access-fn size-fn sequence-type}{element-type
element-doc-string index-doc-string}  
Provides
a simple way to define sequence clauses.  Generates two
clauses, one for iterating over the sequence's elements, the other
for iterating over its indices.  The first symbol of both
clauses will have print-name |for|.
~element-name~  and ~index-name~ should be symbols.
~element-name~ is the second keyword of the element iterator (typically of
the form 
|in-~sequence-type~|), and ~index-name~ is the second keyword
of the index-iterator (typically of the form
|index-of-~sequence-type~|).  Either name may be 
\nil, in which case the corresponding clause is not defined.  If both
symbols are supplied, they should be in the same package.  The |for|
that begins the clauses will be in this package.

\cpar ~access-fn~ is the function to be used to
access elements of the sequence in the element iterator.  The function
should take two 
arguments, a sequence and an index, and return the appropriate element.
~size-fn~ should denote a function of one argument, a sequence, that
returns its size.  Both ~access-fn~ and ~size-fn~ are required for the
element iterator, but only ~size-fn~ is needed for the index iterator.

\cpar The ~sequence-type~ and ~element-type~ keywords are used to
suggest types for the variables 
used to hold the sequence and the
sequence elements, respectively.  The usual rules about \iter's
treatment of variable type declarations apply (see section \ref{types}).

\cpar ~element-doc-string~ and ~index-doc-string~ are
the documentation strings, for use with |display-iterate-clauses|. 

\cpar The generated element-iterator performs destructuring on the
element variable.

\cpar As an example, the above |for\dots in-whole-vector| example
could have been written:
\begin{program}
(defclause-sequence IN-WHOLE-VECTOR INDEX-OF-WHOLE-VECTOR
  :access-fn 'aref
  :size-fn \#'(lambda (v) (array-dimension v 0))
  :sequence-type 'vector
  :element-type t
  :element-doc-string 
     "Elements of a vector, disregarding fill-pointer"
  :index-doc-string 
     "Indices of vector, disregarding fill-pointer")
\end{program}

\end{clauses}

\subsection{Subtleties}

There are some subtleties to be aware of when writing \iter\ clauses.
First, the code returned by your macros may be |nconc|'ed into a list,
so you should always returned freshly consed lists, rather than
constants.  Second, \iter\ matches clauses by using |eq| on the first
symbol and |string=| on the subsequent ones, so the package of the
first symbol of a clause is relevant.  All of the clauses in this manual
have their first word in the \iter\ package.  
You can use the package system in the usual way to shadow
\iter\ clauses without replacing them.

%%% say more here, about the badness that only the first word of a
%%% clause is packagey.

\section{Non-portable Extensions to Iterate (Contribs)}
\label{contribs}

Currently, there is only one non-portable extension to iterate in the
distribution: iterate-pg. If you have made an extension that depends
on non-portable features, feel free to send them to |asf@boinkor.net|
for inclusion in the iterate distribution.

\subsection{An SQL query driver for iterate}

The pg package by Eric Marsden (see |http://cliki.net/pg|) provides an
interface to the PostgreSQL database. Using the \iterpg\ extension, it
is possible to handle the results of SQL queries using \iter.

This usage example should give you an idea of how to use it:

\begin{program}
(pg:with-pg-connection (c "somedb" "someuser")
  (iter (for (impl version date) in-relation "select * from version"
                                 on-connection *dbconn*)
        (collect version)))
\end{program}

To use the extension via |ASDF|, simply make your system depend on the
|iterate-pg| system instead of the |iterate| system. To load it
manually, use:

\begin{program}
  (asdf:oos 'asdf:load-op :iterate-pg)
\end{program}

%%%%%%%%%%%%%%%%%%%%%%%%%%%%%%%%%%%%%%%%%%%%%%%%%%%%%%%%%%%%%%%%
\section{Obtaining {\tt Iterate}}

\begin{sloppypar}
The information in this chapter is obsolete but included for
completeness's sake; Currently, the most up-to-date information on
\iter\ can be found at |http://boinkor.net/iterate.html|.
\end{sloppypar}

\begin{sloppypar}
\iter\ currently runs on Lisp Machines, and on
HP's, Sun3's and Sparcstations under Lucid.  
\iter\ source and binaries are available at the MIT AI Lab in the
subdirectories of |/src/local/lisplib/|.  The source file,
|iterate.lisp|, is also available for anonymous FTP in the directory
|/com/fpt/pub/| on the machine |TRIX.AI.MIT.EDU| (Internet number
128.52.32.6).  If you are unable to obtain |iterate| in one of these
ways, send mail to |jba@ai.mit.edu| and I will send you the source
file. 
\end{sloppypar}

\begin{sloppypar}
\iter\ resides in the |iterate| package (nickname |iter|).  Just say
\linebreak |(use-package :iterate)| to make all the necessary symbols
available. 
If a symbol is not exported, it appears in this manual with an
``|iterate::|'' prefix.
\end{sloppypar}

Send bug reports to |bug-iterate@ai.mit.edu|.  The |info-iterate|
mailing list will have notices of changes and problems; to have
yourself added, send mail to |info-iterate-request@ai.mit.edu|.


\medskip

\begin{flushleft}
 \bf Acknowledgements
\end{flushleft}
\smallskip

Richard Waters provided invaluable criticism which spurred me to improve
\iter\ greatly.  As early users, David Clemens, Oren Etzioni and Jeff
Siskind helped ferret out many bugs.

%\begin{theindex}
% The index files must be generated with the genindex program
% or from makeindex iterate-manual.idx which creates iterate-manual.ind.

%\baselineskip\normalbaselineskip
%\advance\baselineskip by -2pt

\input{iterate-manual.ind}

%\underline{Clauses}
%\input{iterate-manual.clindex}
%
%\indexspace
%\underline{Lisp}
%\input{iterate-manual.lispindex}

%\end{theindex}


\end{document}


% arch-tag: "c0f871a6-313c-11d8-abb9-000c76244c24"


%\end{theindex}


\end{document}


% arch-tag: "c0f871a6-313c-11d8-abb9-000c76244c24"

%
%\indexspace
%\underline{Lisp}
%% Fix ``.'' spacing in lists
\documentstyle[12pt,aimemo]{article}

\setlength{\textwidth}{6.0in}
\setlength{\oddsidemargin}{0.25in}
\setlength{\evensidemargin}{0.25in}
\setlength{\topmargin}{0pt}
\setlength{\headsep}{24pt}
\setlength{\textheight}{8.5in}

\let\origpar=\par

% somehow, LaTex fucks with things so that the plain Tex \obeyspaces
% doesn't work.  This is the same, except we use an mbox containing
% a space instead of just a space.

\newcommand{\spc}{\mbox{ }}
\newcommand{\tAb}{\mbox{        }}
\def\obspaces{\catcode`\ =\active\catcode`\^^I=\active}
{\obspaces \global\let =\spc\global\let^^I=\tAb}

\newenvironment{program}{\medskip\samepage\tt\obeylines\obspaces}{\medskip}

% use | for \tt
\catcode`\|=13
\newif\ifvbar
\def|{\ifvbar
         \endgroup\vbarfalse
      \else 
         \begingroup\tt\vbartrue 
      \fi}



\newcommand{\lisp}{\tt}
\newcommand{\iter}{|iterate|}
\newcommand{\iterpg}{|iterate-pg|}
\newcommand{\setf}{|setf|}
\newcommand{\nil}{|nil|}
\newcommand{\nonnil}{non-\nil}
\newcommand{\opt}{{\lisp \&optional}}
\newcommand{\yields}{$\Rightarrow$}



\newbox\Dots
\setbox\Dots=\hbox{\small ...}
\renewcommand{\dots}{\copy\Dots}

% Here's a way to define an environment which will treat all
% paragraphs inside it in the same manner.  It's, as near as I can
% figure, the way that Latex does things.  Using \everypar won't work;
% apparently Latex resets that a lot.  What you have to do is: at the
% top of the file do \let\origpar=\par.  In the
% begin part of the environment, do the parshape (or whatever), then say
% \def\par{{\origpar}}.  The extra braces are necessary.   That's all
% you have to do; at the end of the environment things will return to
% normal automatically.  Why does this all work?  I have no idea.
% Another important thing: there should be a blank line (i.e. a \par)
% between the end of the last paragraph and the \end{environment}.

\newlength{\clindent}
\setlength{\clindent}{0.5in}
\newlength{\clparindent}
\setlength{\clparindent}{\parindent}

\newenvironment{clauses}{\advance\linewidth -\parindent
                        \hangindent\clindent\relax
                         \hangafter=0
                          \parindent=0pt
                          \def\par{{\origpar}}}{}

\newcommand{\cpar}{\hspace \clparindent}

\newcommand{\startitem}{\bigskip\pagebreak[3]}
\newcommand{\finishitem}{\smallskip}

\newcommand{\clause}[1]{\startitem\Clause{#1}\finishitem}
\newcommand{\Clause}[1]{\hbox{#1}}
\newcommand{\clindex}[1]{\index{{\lisp #1}}}
\newcommand{\clindexx}[2]{\index{{\lisp #1\dots #2}}}
\newcommand{\clindexxx}[3]{\index{{\lisp #1\dots #2\dots #3}}}
\newcommand{\clausex}[2]{\startitem\Clausex{#1}{#2}\finishitem}
\newcommand{\Clausex}[2]{\Clause{|#1| #2}\clindex{#1}}

\newcommand{\defvar}[1]{\startitem\deff{#1}{}{Variable}\finishitem}
\newcommand{\defunexpvar}[1]{\startitem\deffnoindex{iterate::#1}{}{Variable}
                                        \lispindex{#1}\finishitem}
\newcommand{\defun}[2]{\startitem\deff{#1}{#2}{Function}\finishitem}
\newcommand{\defmacro}[2]{\startitem\deff{#1}{#2}{Macro}\finishitem}
\newcommand{\defunx}[3]{\startitem\deffx{#1}{#2}{#3}{Function}\finishitem}
\newcommand{\defmacrox}[3]{\startitem\deffx{#1}{#2}{#3}{Macro}\finishitem}
\newcommand{\defunxx}[4]{\startitem\deffxx{#1}{#2}{#3}{#4}{Function}\finishitem}
\newcommand{\defmacroxx}[4]{\startitem\deffxx{#1}{#2}{#3}{#4}{Macro}\finishitem}

\newcommand{\deff}[3]{\deffnoindex{#1}{#2}{#3}\lispindex{#1}}
\newcommand{\deffnoindex}[3]{\hbox to \hsize{{\tt #1} {\it #2}\hfill [{\it #3}]}}
\newcommand{\deffx}[4]{\deff{#1}{#2}{#4}\moreargs{#1}{#3}}
\newcommand{\deffxx}[5]{\deffx{#1}{#2}{#3}{#5}\moreargs{#1}{#4}}

\newcommand{\moreargs}[2]{\hbox to \hsize{\phantom{\lisp #1} {\it #2}\hfill}}

\newcommand{\lispindex}[1]{\index{{\lisp #1}}}

 \newenvironment{note}[1]{\pagebreak[2]\bigskip
                          \hrule\smallskip\small
                         {\setlength{\parindent}{0pt}
                           \par{\bf #1:}}}{\smallskip\hrule\bigskip}



\makeindex

%%%%%%%%%%%%%%%%%%%%%%%%%%%%%%%%%%%%%%%%%%%%%%%%%%%%%%%%%%%%%%%%

%\thispagestyle{empty}

%\vspace*{2in}

%\title{The |Iterate| Manual}
%\author{Jonathan Amsterdam}

%\maketitle

%\thispagestyle{empty}

%\begin{center}
%\Large *** DRAFT ***
%\end{center}
%This manual is in beta-test. It will become an AI Memo in a couple of
%months.  I would appreciate any comments or
%corrections.  You can reach me at email address |jba@ai.ai.mit.edu|.

\begin{document}

\def\writtenby{Jonathan Amsterdam}
\def\contractno{This report describes research done at the Artificial
Intelligence Laboratory of the Massachusetts Institute of Technology.
Support for the laboratory's artificial intelligence research is
provided in part by the Advanced Research Projects Agency of the
Department of Defense under Office
of Naval Research contract N00014-85-K-0124.}

\AIMEMO{1236}{\Large The Iterate Manual \\[1in]}
{This is the manual for version 1.4 of \iter, a powerful iteration
macro for Common  
Lisp. \iter\ is similar to {\tt loop} but provides numerous additional
features, is well integrated with Lisp, and is extensible. 
\vfill }


% use ~ for \em
\catcode`\~=13
\newif\ifem
\def~{\ifem
         \/\endgroup\emfalse
      \else 
         \begingroup\em\emtrue 
      \fi}



%\pagebreak
%\setcounter{page}{1}

%\thispagestyle{empty}


\tableofcontents

\pagebreak
\pagestyle{headings}

\section{Introduction}

\begin{sloppypar}
This manual describes
\iter, a powerful iteration facility for Common Lisp.
\iter\ provides abstractions for many common iteration
patterns and allows for the definition of additional patterns.  \iter\ is
a macro that expands into ordinary Lisp at
compile-time, so it is more efficient than higher-order
functions like |map| and |reduce.|  
While it is similar to |loop|, \iter\ offers a more Lisp-like syntax
and enhanced extensibility.
(For a more complete
comparison of \iter\ with other iteration constructs, see MIT AI Lab Working
Paper No. 324, ~Don't Loop, Iterate.~)
\end{sloppypar}

An \iter\ form consists of the symbol |iter|\footnote{You can also use
|iterate|, but |iter| is preferred because it avoids potential conflicts with
possible future additions to Common Lisp, and because it saves
horizontal space when writing code.}
followed by one or more
forms, some of which may be \iter\ 
~clauses.~  Here is a simple example of \iter\ which collects the
numbers from 1 to 10 into a list, and returns the list.  The return
value is shown following the arrow.
\begin{program}
(iter (for i from 1 to 10)
      (collect i))          \yields (1 2 3 4 5 6 7 8 9 10)
\end{program}
This form contains two clauses: a |for| clause that steps the
variable |i| over the integers from 1 to 10, and a |collect|
clause that accumulates its argument into a list.  With a few
exceptions, all \iter\ clauses have the same format: alternating
symbols (called ~keywords~) and expressions (called
~arguments~).  The syntax and terminology are those of Common
Lisp's keyword lambda lists.  One difference is that \iter's keywords
do not have to begin with a colon---though they may, except for the
first symbol of a clause.  So you can also write  |(for i :from
1 :to 10)| if you prefer.

Any Lisp form can appear in the body of an \iter, where it will have
its usual meaning.  \iter\ walks the entire body, expanding macros,
and recognizing clauses at any level.  This example collects all the
odd numbers in a list:
\begin{program}
(iter (for el in list)
      (if (and (numberp el) (oddp el))
          (collect el)))
\end{program}

There are clauses for iterating over numbers, lists, arrays and other
objects, and for
collecting, summing, counting, maximizing and other useful operations.
\iter\ also supports the creation of new variable bindings, stepping
over multiple sequences at once, destructuring, and compiler
declarations of variable types.  The following example illustrates
some of these features: 
\begin{program}
(iter (for (key . item) in alist)
         (for i from 0)
         (declare (fixnum i))
         (collect (cons i key)))
\end{program}

This loop takes the keys of an alist and returns a new alist
associating the keys with their positions in the original list. The
compiler declaration for |i| will appear in the generated code
in the appropriate place.

%%%%%%%%%%%%%%%%%%%%%%%%%%%%%%%%%%%%%%%%%%%%%%%%%%%%%%%%%%%%%%%%
\section{Clauses}

Most of \iter's clauses will be familiar to |loop| programmers.  
(|loop| is an iteration macro that has been incorporated into Common
Lisp.  Seen Guy Steele's {\em Common Lisp, 2nd Edition}.)
In
nearly all cases they behave the same as their |loop| counterparts, so
a |loop| user can switch to \iter\ with little pain (and much gain).

All clauses with the standard keyword-argument syntax consist of two
parts: a ~required~ part, 
containing keywords that must be present and in the right order; and
an ~optional~ part, containing keywords that may be omitted and,
if present, may occur in any order.  In the descriptions below, the
parts are separated by the Lisp lambda-list keyword \opt.


\subsection{Drivers}

An iteration-driving clause
conceptually causes the iteration to go forward.  Driver clauses in
\iter\ allow iteration over numbers, lists, vectors, hashtables, packages,
files and streams.  Iteration-driving clauses must
appear at the top level of an \iter\ form; they cannot be nested
inside another clause.  The driver variable is updated at the point
where the driver clause occurs.  Before the clause is executed for the
first time, the value of the variable is undefined.

%Also, regardless of where the driver clause
%appears in the body, the driver variable is stepped at the top of the
%loop; hence it is stylistically preferable, though
%not required, to place driver clauses at the beginning of the \iter.

Multiple drivers may appear in a single \iter\ form, in which case all
of the driver variables are updated each time through the loop, in the
order in which the clauses appear.  The first driver to terminate will
terminate the entire loop.

In all cases, the value of the driver variable on exit from the loop,
including within the epilogue code (see the |finally| clause), is
undefined.

All the parameters of a driver clause are evaluated once, before the
loop begins.  Hence it is not possible to change the bounds or other
properties of an iteration by side-effect from within the loop.

With one exception, driver clauses begin with the word |for| (or
the synonym |as|) and mention an iteration variable, which is
given a binding within the \iter\ form.
The exception is |repeat|, which just executes a loop a
specified number of times:

\begin{clauses}

\clausex{repeat}{~n~}
Repeats the loop ~n~ times.  For example:
\begin{program}
(iter (repeat 100)
      (print "I will not talk in class."))
\end{program}
\cpar If $n \leq 0$, the
loop will never be executed.  If ~n~ is not an integer, the
actual number of executions will be $\lceil n \rceil$.

\end{clauses}

\pagebreak[3]

\subsubsection{Numerical Iteration}

\begin{clauses}
\clausex{for}{~var~ |\&sequence|} 
The general form for iterating over a sequence of numbers requires a
variable and, optionally, one or more keywords that provide the bounds
and step size of the iteration. The |\&sequence|
lambda-list keyword is a shorthand for these sequence keywords.  They are:
|from|, |upfrom|, |downfrom|, |to|, |downto|, |above|, |below| and |by|.
|from| provides the starting value for ~var~
and defaults to zero. |to| provides a final value and implies that the
successive values of ~var~ will be increasing; |downto| implies that
they will be decreasing.  The loop terminates when ~var~ passes
the final value (i.e. becomes smaller or larger than it, depending on
the direction of iteration); in other  words, the loop body
will never be 
executed for values of ~var~ past the final value.
|below| and |above| are similar to |to| and
|downto|, except that the loop terminates when ~var~ equals or passes
the final value.

\cpar If no final value is specified, the variable will be stepped
forever.  Using |from| or |upfrom| will result in increasing
values, while |downfrom| will give decreasing values.

\cpar On each iteration,
~var~ is incremented or decremented by the value of the sequence
keyword |by|, which defaults to 1.  It should always be a positive
number, even for downward iterations.

\cpar In the following examples, the sequence of numbers generated is shown
next to the clause.

\begin{program}
(for i upfrom 0) \yields\ 0 1 2 \ldots
(for i from 5)   \yields\ 5 6 7 \ldots    ; either from or upfrom is okay
(for i downfrom 0) \yields\ 0 -1 -2 \ldots
(for i from 1 to 3) \yields\ 1 2 3
(for i from 1 below 3) \yields\ 1 2
(for i from 1 to 3 by 2) \yields\ 1 3
(for i from 1 below 3 by 2) \yields\ 1
(for i from 5 downto 3) \yields\ 5 4 3
\end{program}

\end{clauses}

\subsubsection{Sequence Iteration}

There are a number of clauses for iterating over sequences.  In all of
them, the argument following |for| may be a list instead of a
symbol, in which case destructuring is performed.  See section
\ref{destructuring}. 

\begin{clauses}

\clause{|for| ~var~ |in| ~list~ \opt\ |by| ~step-function~}
\clindexx{for}{in}
~var~ is set to successive elements of list.  
~step-function,~ which 
defaults to |cdr|, is used to obtain the next sublist. 

\clause{|for| ~var~ |on| ~list~ \opt\ |by| ~step-function~}
\clindexx{for}{on}
~var~ is set to successive sublists of list. 
~step-function~ (default |cdr|) is used as in |for\dots in|.

\end{clauses}

\medskip

These two clauses use |atom| to test for the end
of a list.  Hence, given a list whose final |cdr| is not \nil,
they will silently ignore the last |cdr|.  Other choices are
|endp|,  which would signal an error, and |null|, which
would probably result in an error somewhere else.  If you wish to use
an end-test other than |atom|, set the variable
|iterate::*list-end-test*|\lispindex{*list-end-test*} to the name of
the desired function. 

\begin{clauses}
\clause{|for| ~var~ |in-vector| ~vector~ |\&sequence|}
\clindexx{for}{in-vector}
~var~ takes on successive elements from ~vector.~   The vector's
fill-pointer is observed.  Here and in subsequent clauses, the |\&sequence|
keywords include |with-index|, which takes a symbol as argument
and uses it for the index variable instead of an internally generated
symbol.  The other |\&sequence| keywords behave as in numerical
iteration, except that the default iteration bounds are the bounds of
the vector.  E.g. in |(for i in-vector v downto 3)|,
|i| will start off being bound to the last element in |v|,
and will  
be set to preceding elements down to and including the element with
index 3.

\clause{|for| ~var~ |in-sequence| ~seq~ |\&sequence|}
\clindexx{for}{in-sequence}
This uses Common Lisp's generalized sequence functions, |elt|
and |length|, to obtain elements and determine the length of
~seq.~   Hence it will work for any sequence, including lists, and will
observe the fill-pointers of vectors.

\clause{|for| ~var~ |in-string| ~string~ |\&sequence|}
\clindexx{for}{in-string}
~var~ is set to successive characters of ~string.~

\startitem
\Clause{|for| ~var~ |index-of-vector| ~vector~ |\&sequence|}
\clindexx{for}{index-of-vector}
\Clause{|for| ~var~ |index-of-sequence| ~sequence~ |\&sequence|}
\clindexx{for}{index-of-sequence}
\Clause{|for| ~var~ |index-of-string| ~string~ |\&sequence|}
\clindexx{for}{index-of-string}
\finishitem
~var~ is set to successive indices of the sequence.
These clauses avoid the overhead of accessing the sequence elements
for those applications where they do not need to be examined, or are
examined rarely.  They admit all the optional keywords of the other
sequence drivers except the (redundant) |with-index| keyword.

\clause{|for| ~(key value)~ |in-hashtable| ~table~}
\clindexx{for}{in-hashtable}

~key~ and ~value,~ which must appear as shown in a list
and may be destructuring templates, are set to the keys and values of
~table~.  If ~key~ is |nil|, then the hashtable's keys will be ignored;
similarly for ~value~.
The order in which 
elements of ~table~ will be retrieved is unpredictable.

\clause{|for| ~var~ |in-package| ~package~ \opt\ |external-only| ~ext~}
\clindexx{for}{in-package}
Iterates over all the symbols in ~package,~ or over only the
external symbols if ~ext~ is specified and non-|nil|.  ~ext~ is not
evaluated.  The same symbol may appear more than once.

\clause{|for| ~var~ |in-packages| ~packages~ \opt\ |having-access| ~symbol-types~}
\clindexx{for}{in-packages}
Iterates over all the symbols from the list of packages denoted by the
descriptor ~packages~ and having accessibility (or visibility) given by
~symbol-types~.  This defaults to the list |(:external :internal :inherited)|
and is not evaluated.  ~var~ must be a list of up to three variables: in each
iteration, these will be set to a symbol, its access-type and package (as per
|with-package-iterator| in ANSI-CL).  The same symbol may appear more than
once.

\end{clauses}

\begin{clauses}

\clause{|for| ~var~ |in-file| ~name~ \opt\ |using| ~reader~}
\clindexx{for}{in-file}
Opens the file ~name~ (which may be a string or pathname) for input,
and iterates 
over its contents. ~reader~ defaults to |read|, so by default ~var~
will be bound to the successive forms in the file.  The \iter\ body is
wrapped in an unwind-protect to ensure that the file is closed no
matter how the \iter\ is exited.

\clause{|for| ~var~ |in-stream| ~stream~ \opt\ |using| ~reader~}
\clindexx{for}{in-stream}
Like |for\dots in-file|, except that ~stream~ should be an existing
stream object that supports input operations.

\end{clauses}

\subsubsection{Generalized Drivers}

These are primarily useful for writing drivers that can also be used
as generators (see section \ref{generators}, below).
 
\begin{clauses}

\clause{|for| ~var~ |next| ~expr~}
\clindexx{for}{next}

~var~ is set to ~expr~ each time through the loop.  Destructuring is
performed.  When the clause is used as a generator, ~expr~ is the code
that is executed when |(next ~var~)| is 
encountered (see section \ref{generators}, below).
~expr~ should compute the first value for ~var~, as well
as all subsequent values, and is responsible for terminating the loop.
For compatibility with future versions of \iter, this termination
should be done with |terminate|\clindex{terminate}, which can be
considered a synonym for |finish| (see section \ref{control-flow}).

\cpar As an example, the following clauses are equivalent to |(for
i from 1 to 10)|:
\begin{program}
(initially (setq i 0))
(for i next (if (> i 10) (terminate) (incf i)))
\end{program}

\clause{|for| ~var~ |do-next| ~form~}
\clindexx{for}{do-next}
~form~ is evaluated each time through the loop.  Its value is ~not~
set to ~var~; that is ~form's~ job. ~var~ is only present so that
\iter\ knows it is a driver variable.  \linebreak |(for ~var~ next
~expr~)| is equivalent to |(for ~var~ do-next (dsetq ~var~ ~expr~))|.
(See section \ref{destructuring} for an explanation of |dsetq|.) 

\end{clauses}

\subsubsection{Generators}
\label{generators}

In all of the above clauses, the driver variable is updated on each
iteration.
Sometimes it is desirable to have greater control over updating.  
For instance, consider the problem of associating numbers, in
increasing order and with no gaps, with the
\nonnil\ elements of a list.  One obvious first pass at writing this is:

\begin{program}
(iter (for el in list)
      (for i upfrom 1)
      (if el (collect (cons el i))))
\end{program}
But on the list |(a b nil c)| this produces |((a . 1) (b . 2) (c .
4))| instead of the desired |((a . 1) (b . 2) (c . 3))|.  The problem
is that |i| is incremented each time through the loop, even when |el|
is |nil|.  

The problem could be solved elegantly if we could step |i| only when
we wished to.  This
can be accomplished for any \iter\ driver by writing |generate| (or
its synonym |generating|)
instead of |for|.  Doing so produces a ~generator~---a driver whose
values are yielded explicitly.  To obtain the next value of a
generator variable ~v~, write \linebreak |(next ~v~)|.  The value of a |next|
form is the next value of ~v~, as determined by its associated driver
clause.  |next| also has the side-effect of updating ~v~ to that
value.  If there is no next value, |next| will terminate the loop,
just as with a normal driver.


Using generators, we can now write our example like this:
\begin{program}
(iter (for el in list)
      (generate i upfrom 1)
      (if el (collect (cons el (next i)))))
\end{program}
Now |i| is updated only when |(next i)| is executed, and this occurs
only when |el| is \nonnil.

To better understand the relationship between ordinary drivers and
generators, observe that we can rewrite an ordinary driver using its
generator form immediately followed by |next|, as this example shows:
\begin{program}
(iter (generating i from 1 to 10)
      (next i)
      ...)
\end{program}
Provided that the loop body contains no |(next i)| forms, this will
behave just as if we had written |(for i from 1 to 10)|.

We can still refer to a driver variable ~v~ without using |next|; in
this case, its value is that given to it by the last evaluation of
|(next ~v~)|.  Before |(next ~v~)| has been called the first time, the
value of ~v~ is undefined.  

This semantics is more flexible than
one in which ~v~ begins the loop bound to its first value and calls
of |next| supply subsequent values, because it means the loop will not
terminate too soon if the generator's sequence is empty.  For
instance, consider the following code, which tags \nonnil\ elements of
a list using a list of tags, and also counts the null elements.
(We assume there are at least as many tags as \nonnil\ elements.)
\begin{program}
(let* ((counter 0)
       (tagged-list (iter (for el in list)
                          (generating tag in tag-list) 
                          (if (null el)
                              (incf counter)
                              (collect (cons el (next tag)))))))
  ...)
\end{program}
It may be that there are just as many tags as non-null elements of
|list|.  If all the elements of |list| are null, we still want the
counting to proceed, even though |tag-list| is \nil.  If |tag| had to be
assigned its first value before the loop begins, we would have had to
terminate the loop before the first iteration, since when |tag-list|
is \nil, |tag| 
has no first value.  With the existing semantics, however, |(next
tag)| will never execute, so the iteration will cover all the elements of
|list|. 
        
When the ``variable'' of a driver clause is actually a destructuring
template containing several variables, all the variables are eligible
for use with |next|.  As before, |(next ~v~)| evaluates to ~v's~ next
value; but the effect is to update all of the template's variables.
For instance, the following code will return the list |(a 2 c)|.
\begin{program}
(iter (generating (key . item) in '((a . 1) (b . 2) (c . 3)))
      (collect (next key))
      (collect (next item)))
\end{program}

Only driver clauses with variables can be made into generators.  This
includes all clauses mentioned so far except for |repeat|.  It does
~not~ include |for\dots previous|, |for\dots =|,
|for\dots initially\dots then| or |for\dots first\dots then| (see
below). 

\subsubsection{Previous Values of Driver Variables}

Often one would like to access the value of a variable on a previous
iteration.  \iter\ provides a special clause for accomplishing this.

\begin{clauses}

\clause{|for| ~pvar~ |previous| ~var~ \opt\ |initially| ~init~
|back| ~n~}
\clindexx{for}{previous}
Sets ~pvar~ to the previous value of ~var,~ which should be a driver
variable, a variable from another |for\dots previous| clause, or a
variable established by a |for\dots =|, 
|for\dots initially\dots then| or |for\dots first\dots then| clause
(see section \ref{setting}).
Initially, ~pvar~ is given the value ~init~ (which defaults to \nil).  
The ~init~ expression will be moved outside the loop body, so it
should not depend on anything computed within the loop.
~pvar~ retains the value of ~init~ until ~var~ is set to its second
value, at which point ~pvar~ is set to ~var's~ first value; and so on.  

\cpar The
argument ~n~ to |back| must be a constant, positive integer, and
defaults to 1.  It determines how many iterations back ~pvar~ should
track ~var~.  For example, when ~n~ is 2, then ~pvar~ will be assigned
~var's~ first value when ~var~ is set to its third value.

\cpar A |for\dots previous| clause may occur after or before its
associated driver clause. |for\dots previous| works with generators as
well as ordinary drivers.

\pagebreak[3]

\cpar Example:
\begin{program}
(iter (for el in '(1 2 3 4))
      (for p-el previous el)
      (for pp-el previous p-el initially 0)
      (collect pp-el))
\end{program}
This evaluates to |(0 0 1 2)|.  It could have been written more
economically as 
\begin{program}
(iter (for el in '(1 2 3 4))
      (for pp-el previous el back 2 initially 0)
      (collect pp-el))
\end{program}


\end{clauses}

\subsection{Variable Binding and Setting}
\label{setting}

Several clauses exist for establishing new variable bindings or for
setting variables in the loop.  They all support destructuring.

\begin{clauses}

\clausex{with}{~var~ \opt\ |=| ~value~}
Causes ~var~ to be bound to value before the loop body is entered.
If ~value~ is not supplied, ~var~ assumes a default
binding, which will be 
\nil\ in the absence of declarations.  Also, if ~value~ is not
supplied, no destructuring is performed; instead, ~var~ may be a list of
symbols, all of which are given default 
bindings.  If ~value~ is supplied, ~var~ is bound to it, with
destructuring.  

\cpar Because |with|
creates bindings whose scope includes the entire \iter\ form, it is
good style to put all |with| clauses at the beginning.

\cpar Successive occurrences of |with| result in sequential
bindings (as with 
|let*|).  There is no way to obtain parallel bindings; see
section \ref{bindings} for a rationale.  


\clause{|for| ~var~ |=| ~expr~}
\clindexx{for}{=}
On each iteration, ~expr~ is evaluated and ~var~ is set
to its value. 

\cpar This clause may appear to do the same thing as |for\dots next|.
In fact, they are quite different.  |for\dots =| provides only three
services: it sets up a binding for ~var~, sets it to ~expr~ on each
iteration, and makes it possible to use |for\dots previous| with
~var~.  |for\dots next| provides these services in addition to the
ability to turn the driver into a generator.
%Also, the code which sets ~var~ appears in the loop body in the same
%place as the |for\dots =| clause; the code for |for\dots next| appears
%at the top of the loop, as with other drivers (except when being used
%as a generator).

\clause{|for| ~var~ |initially| ~init-expr~ |then| ~then-expr~}
\clindexxx{for}{initially}{then}
Before the loop begins, ~var~ is set to ~init-expr;~ on all
iterations after the first it is set to ~then-expr.~  This clause
must occur at top-level.  ~init-expr~ will be moved outside the loop
body and ~then-expr~ will be moved to the end of the loop body, so
they are subject to code motion problems (see section
\ref{code-movement}). 

\cpar This clause may appear to be similar to |for\dots next|, but in
fact they differ significantly.  |for\dots initially\dots then| is
typically used to give ~var~ its first value before the loop begins,
and subsequent values on following iterations.  This is incompatible
with generators, whose first value and subsequent values must all be
computed by |(next ~var~)|.  Also, the update of ~var~ in 
|for\dots initially\dots then| does not occur at the location of the
clause. 
Use |for\dots initially\dots then| for
one-shot computations where its idiom is more convenient, but use
|for\dots next| for extending \iter\ with new drivers (see section
\ref{extend}).

\clause{|for| ~var~ |first| ~first-expr~ |then| ~then-expr~}
\clindexxx{for}{first}{then}
The first time through the loop, ~var~ is set to ~first-expr;~ on
subsequent iterations, it is set to ~then-expr.~  This differs from
|for\dots initially| in that ~var~ is set to ~first-expr~
inside the loop body, 
so ~first-expr~ may depend on the results of other clauses.  For
instance, 
\begin{program}
(iter (for num in list)
      (for i first num then (1+ i))
      ...)
\end{program}
will set |i| to the first element of |list| on the first
iteration, whereas 
\begin{program}
(iter (for num in list)
      (for i initially num then (1+ i))
      ...)
\end{program}
is probably erroneous; |i| will be bound to |num|'s
default binding (usually \nil) for the first iteration.

\end{clauses}

\begin{note}{Compatibility note}
|loop|'s |for\dots =| works like \iter's, but |loop| used the syntax
|for\dots =\dots then| to mean |for\dots initially\dots then|.  It was
felt that these two operations were sufficiently different to warrant
different keywords.

Also, the |for| in the above three clauses is misleading,
since none is true driver (e.g. none has a corresponding |generate|
form).  |setting| would have been a better choice, but |for| was used
to retain some compatibility with |loop|.  
\end{note}


%%%%%%%%%%%%%%%%%%%%%%%%%%%%%%%%
\subsection{Gathering Clauses}
Many of \iter's clauses accumulate values into a variable, or set a
variable under certain conditions.  At the end of the loop, this
variable contains the desired result.  All these clauses have an
optional |into| keyword, whose argument should be a symbol.  If the
|into| keyword is not supplied, the accumulation variable will be
internally generated and its value will be returned at the end of the
loop; if a variable is specified, that variable is used for the
accumulation, and is not returned as a result---it is up to the user
to return it explicitly, in the loop's epilogue code (see |finally|).
It is safe to examine the accumulation variable during the loop, but
it should not be modified.

These clauses all begin with a verb.  When the verb does
not conflict with an existing Common Lisp function, then it may be
used in either its infinitival or present-participle form (e.g. |sum|,
|summing|).  However, when there is a conflict with Common Lisp, only
the present-participle form may be used (e.g. |unioning|).  This is to
prevent \iter\ clauses from clashing with Common Lisp
functions. 

%although these clauses are described as ``producing a value,'' it is a
%mistake to think of the lisp list representing the clause as a
%value-producing form in the usual way.  clauses may legally be written
%where a value is expected, e.g. |(setq x (sum i))|, but the
%lisp value of a clause in such a context is undefined.

\subsubsection{Reductions}

~Reduction~ is an extremely common iteration pattern in which
the results of successive applications of a
binary operation are accumulated.
For example, a loop that computes the sum of the
elements of a list is performing a reduction with the addition
operation.  This could be written in Common Lisp as  |(reduce \#'+
list)| or with \iter\ as
\begin{program}
(iter (for el in list)
      (sum el))
\end{program}

\begin{clauses}

\clausex{sum}{~expr~ \opt\ |into| ~var~}
Each time through the loop, ~expr~ is evaluated and added to a
variable, which is bound initially to zero.  If ~expr~ has a type,
it is ~not~ used as the type of the sum variable, which is always
|number|.  To get the result variable to be of a more specific
type, use an explicit variable, as in
\begin{program}
(iter (for el in number-list)
      (sum el into x)
      (declare (fixnum x))
      (finally (return x)))
\end{program}

\clausex{multiply}{~expr~ \opt\ |into| ~var~}
Like |sum|, but the initial value of the result variable is $1$,
and the variable is updated by multiplying ~expr~ into it.

\clausex{counting}{~expr~ \opt\ |into| ~var~}
~expr~ is evaluated on each iteration.  If it is \nonnil, the
accumulation variable, initially zero, is incremented.

\startitem
\Clausex{maximize}{~expr~ \opt\ |into| ~var~}
\Clausex{minimize}{~expr~ \opt\ |into| ~var~}
\finishitem
~expr~ is evaluated on each iteration and its extremum (maximum or
minimum) is stored in the accumulation variable.  If ~expr~ is never
evaluated, then the result is \nil\ (if the accumulation variable is
untyped) or $0$ (if it has a numeric type).

\clausex{reducing}{~expr~ |by| ~func~ \opt\ |initial-value| ~init-val~ |into| ~var~}
This is a general way to perform reductions. ~func~ should be a
function of two arguments, the first of which will be the value
computed so far and
the second of which will be the value of ~expr~.  It should return the
new value.
|reducing| is roughly equivalent to the
Common Lisp |(reduce ~func~ ~sequence~ :key ~expr-function~)|, where
~expr-function~ is used to derive values from the successive elements
of ~sequence~.

\cpar If the |reducing| clause is never executed, the result is
undefined.

\cpar It is not necessary to provide an
initial value, but better code can be generated if one is supplied.
Regardless of its location in the \iter\ body, ~init-val~ will be
evaluated before the loop is entered, so it should not depend on any
value computed inside the \iter\ form.  

%\cpar if a ~var~ is not specified, you can get \iter\ to declare the
%type of the internal variable by putting a |the| expression around
%~func.~  see section \ref{types}.

\end{clauses}

%\begin{note}{implementation note}
%in principle, |maximize| and |minimize| can be thought of
%as reductions where the initial value is the smallest (or largest)
%value that the accumulation variable can assume.  because lisp's
%bignums can represent arbitrary integers, these clauses cannot be
%implemented as reductions in general.  if, however, the type of 
%~expr~ or ~var~ can be determined to be a fixnum
%or a float, \iter\ will implement the clause as a true reduction,
%using one of the constants |most-negative-fixnum|, |%most-positive-fixnum|, 
%|most-negative-short-float|, etc. as appropriate.
%\end{note}

\subsubsection{Accumulations}

All the predefined accumulation clauses add values to a sequence.  If
the sequence is a list, they all 
have the property that the partial list is kept in the correct order
and available for inspection at any point in the loop.

\begin{clauses}

\clausex{collect}{~expr~ \opt\ |into| ~var~ |at| ~place~ |result-type| ~type~}
Produces a sequence of the values of ~expr~ on each iteration. ~place~
indicates where the next value of ~expr~ is added to the list and may
be one of the symbols |start|, |beginning| (a synonym for |start|) or
|end|.  The symbol may be quoted, but need not be.  The default is
|end|.  For example,
\begin{program}
(iter (for i from 1 to 5)
      (collect i))
\end{program}
produces |(1 2 3 4 5)|, whereas
\begin{program}
(iter (for i from 1 to 5)
      (collect i at beginning))
\end{program}
produces |(5 4 3 2 1)| (and is likely to be faster in most Common Lisp
implementations).

\cpar If ~type~ is provided, it should be a subtype of |sequence|.
The default is |list|.  Specifying a type other than |list| will
result in |collect| returning a sequence of that type.  ~However,~ the
type of the sequence being constructed when inside the loop body is
undefined when a non-|list| type is specified.  (As with ~place~,
quoting ~type~ is optional.)

\clausex{adjoining}{~expr~ \opt\ |into| ~var~ |test| ~test~ |at| ~place~ |result-type| ~type~}
Like |collect|, but only adds the value of ~expr~ if it is not
already present.  ~test,~ which defaults to |\#'eql|, is
the test to be used with |member|.

\startitem
\Clausex{appending}{~expr~ \opt\ |into| ~var~ |at| ~place~}
\Clausex{nconcing}{\ ~expr~ \opt\ |into| ~var~ |at| ~place~}
\Clausex{unioning}{\ ~expr~ \opt\ |into| ~var~ |test| ~test~ |at| ~place~}
\Clausex{nunioning}{~expr~ \opt\ |into| ~var~ |test| ~test~ |at| ~place~}
\finishitem
These are like |collect|, but behave like the Common Lisp functions
|append|, |nconc|, |union| or |nunion|.  
As in Common Lisp, they work only on lists.  Also as in Common Lisp,
|unioning| and |nunioning| assume that the value of ~expr~ contains no
duplicates. 

\clausex{accumulate}{~expr~ |by| ~func~ \opt\ |initial-value| ~init-val~ |into| ~var~}
This is a general-purpose accumulation clause. ~func~ should be a
function of two arguments, the value of ~expr~ and the value
accumulated so far in the iteration, and it should return the updated
value.  If no initial value is supplied, \nil\ is used.  

%\cpar If a ~var~ is not specified, you can get \iter\ to declare the
%type of the internal variable by putting a |the| expression around
%~func.~  see section \ref{types}.

\cpar The differences between |accumulate| and |reducing| are slight.
One difference is that the functions take their arguments in a
different order.  Another is that in the absence of ~init-val~,
|accumulate| will use \nil, whereas |reducing| will generate different
code that avoids any dependence on the initial value.
The reason for having both clauses is that one usually
thinks of reductions (like |sum|) and accumulations (like |collect|)
as different beasts.

\end{clauses}


\subsubsection{Finders}

A ~finder~ is a clause whose value is an expression that meets some
condition.

\begin{clauses}

\clause{|finding| ~expr~ |such-that| ~test~ \opt\ |into| ~var~ |on-failure| ~failure-value~}
\clindexx{finding}{such-that}
If ~test~ (which is an expression) ever evaluates to \nonnil, the loop
is terminated, the 
epilogue code is run and the value of ~expr~ is returned.  Otherwise,
\nil\ (or ~failure-value,~ if provided) is returned.  If ~var~ is
provided, it will have either the \nonnil\ value of ~expr~ or
~failure-value~ when the epilogue code is run.

\cpar As a special case, if the ~test~ expression is a sharp-quoted
function, then it is applied to ~expr~ instead of being simply
evaluated.  E.g. |(finding x such-that \#'evenp)| is equivalent to
|(finding x such-that (evenp x))|.  

%\cpar although ~test~ need have nothing to do with ~%expr~ as in
%|(finding j such-that (> i 3))|, it usually 
%will: |(finding (length el) such-that (oddp (length el)))|.  to
%avoid performing the |length| computation twice, you could write
%|(finding (length el) such-that \#'oddp)| or |(finding (length
%el) such-that 'oddp)|; for these cases, \iter\ generates code that
%executes ~expr~ only once.  the code for |\#'oddp|
%is slightly different from that for {\lisp 'oddp}; see the discussion
%under {\lisp for\dots in} and {\lisp for\dots on}.

\cpar |On-failure| is a misnomer. Because it is always evaluated, it behaves
more like the default third argument to the |gethash| function. As a result,
|on-failure (error "Not found")| makes no sense. Instead, the clauses |leave|
or |thereis| can be used in conjunction with |finally| as follows:
\begin{program}
(iter (for x in '(1 2 3))
      (if (evenp x) (leave x))
      (finally (error "not found")))
\end{program}

\cpar This clause may appear multiple times when all defaults are
identical. It can also be used together with either |always|/|never| or
|thereis| if their defaults match. More specifically, |on-failure nil| is
compatible with |thereis|, while |on-failure t| is compatible with |always|
and |never| clauses.
\begin{program}
(iter (for i in '(7 -4 2 -3))
      (if (plusp i)
	  (finding    i  such-that (evenp i))
          (finding (- i) such-that (oddp i))))
\end{program}

\startitem
\Clause{|finding| ~expr~ |maximizing| ~m-expr~ \opt\ |into| ~var~}
\clindexx{finding}{maximizing}
\Clause{|finding| ~expr~ |minimizing| ~m-expr~ \opt\ |into| ~var~}
\clindexx{finding}{minimizing}
\finishitem
Computes the extremum (maximum or minimum) value of ~m-expr~ over all
iterations, and returns the value of ~expr~ corresponding to the
extremum.  ~expr~ is evaluated inside the loop at the time the new
extremum is established.  If ~m-expr~ is never evaluated (due to, for
example, being embedded in a conditional clause), then the returned
value depends on the type, if any, of ~expr~ (or ~var,~ if
one is supplied).  If there is no type, the returned
value will be \nil; if the type is numeric, the returned value will be
zero.

\cpar For these two clauses, ~var~ may be a list of two
symbols; in that case, the first is used to record ~expr~ and
the second, ~m-expr.~  

\cpar As with |finding\dots such-that|, if ~m-expr~ is a sharp-quoted
function, then it is called on ~expr~ instead of being evaluated.

\end{clauses}

\subsubsection{Boolean Tests}

\begin{clauses}

\clausex{first-iteration-p}{}
Returns |t| in the first cycle of the loop, otherwise \nil.

\clausex{first-time-p}{}
Returns |t| the first time the expression is evaluated, and then \nil\ forever.
This clause comes handy when printing (optional) elements separated
by a comma:

\begin{program}
(iter (for el in '(nil 1 2 nil 3))
      (when el
        (unless (first-time-p)
          (princ ", "))
        (princ el)))
\end{program}
produces |"1, 2, 3"|.

\end{clauses}

\subsubsection{Aggregated Boolean Tests}

\begin{clauses}

\clausex{always}{~expr~}
If ~expr~ ever evaluates to
\nil, then \nil\ is immediately returned; the epilogue code is not
executed.  If ~expr~ never evaluates to \nil, the epilogue code
is executed and the last value of ~expr~ (or |t| if ~expr~ was never
evaluated) is returned (whereas |loop| would constantly return |t|).

% mention last evaluated clause when multiple always clauses?

\clausex{never}{~expr~}
Like |(always (not ~expr~))|, except it does not influence the last
value returned by a possible other |always| clause. That is,
\begin{program}
(iter (repeat 2)
      (always 2)
      (never nil)) \yields 2 ; not t
\end{program}

\clausex{thereis}{~expr~}
If ~expr~ is ever \nonnil,
its value is immediately returned without running epilogue code.
Otherwise, the epilogue code is performed and \nil\ is returned.

This clause cannot be used together with |always| or |never|, because their
defaults are opposed (similarly, |(loop always 3 thereis nil)| refuses to
compile in some implementations of |loop|).

\end{clauses}


\subsection{Control Flow}
\label{control-flow}
Several clauses can be used to alter the usual flow of control in a loop.

Note: the clauses of this and subsequent sections don't adhere to \iter's
usual syntax, but instead use standard Common Lisp syntax.  Hence the
format for describing syntax subsequently is
like the standard format used in the Common Lisp manual, not like the
descriptions of clauses above.

\begin{clauses}

\clausex{finish}{}
Stops the loop and runs the epilogue code.

%for example:
%
%\begin{program}
%(iter (with answer = nil)
%         (initially (make-a-mess))
%         (for i from 1 to 10)
%         (when (correct? i)
%           (setq answer i)
%           (finish))
%         (finally (cleanup)))
%\end{program}
%
%this code will execute |cleanup| whether or not the test |(correct?
%i)| ever succeeds. 
%the (more elegant) formulation,
%\begin{program}
%(iter (initially (make-a-mess))
%         (for i from 1 to 10)
%         (finding i such-that (correct? i))
%         (finally (cleanup)))
%\end{program}
%would not execute |cleanup| if |(correct? i)| succeeded; it
%would do an immediate return.

\clausex{leave}{\opt\ ~value~}
Immediately returns ~value~ (default \nil) from the current \iter\
form, skipping the epilogue code.  Equivalent to using |return-from|.

\clausex{next-iteration}{}
Skips the remainder of the loop body and begins the next iteration of
the loop.

\clausex{while}{~expr~}
If ~expr~ ever evaluates to \nil, the loop is terminated and the
epilogue code executed.  Equivalent to |(if (not ~expr~) (finish))|.

\clausex{until}{~expr~}
Equivalent to |(if ~expr~ (finish))|.

\clausex{if-first-time}{~then~ \opt\ ~else~}
If this clause is being executed for the first time in this invocation
of the \iter\ form, then the ~then~ code is evaluated; otherwise the
~else~ code is evaluated.

\cpar |(for ~var~ first ~expr1~ then ~expr2~)| is almost equivalent to
\begin{program}
(if-first-time (dsetq ~var~ ~expr1~)
               (dsetq ~var~ ~expr2~))
\end{program}
The only difference is that the |for| version makes ~var~ available
for use with |for\dots previous|.

\end{clauses}

\subsection{Code Placement}
When fine control is desired over where code appears in a loop
generated by \iter, the following special clauses may be useful.
They are all subject to code-motion problems (see section
\ref{code-movement}).  

\begin{clauses}

\clausex{initially}{|\&rest| ~forms~}
The lisp ~forms~ are placed in the prologue section of the loop, where
they are executed once, before the loop body is entered.

\clausex{after-each}{|\&rest| ~forms~}
The ~forms~ are placed at the end of the loop body, where they
are executed after each iteration.  Unlike the other clauses in this
section, ~forms~ may contain \iter\ clauses.

\clausex{else}{|\&rest| ~forms~}
The lisp ~forms~ are placed in the epilogue section of the loop, where they
are executed if this |else| clause is never met during execution of the
loop and the loop terminates normally.

\clausex{finally}{|\&rest| ~forms~}
The lisp ~forms~ are placed in the epilogue section of the loop, where
they are executed after the loop has terminated normally.

\clausex{finally-protected}{|\&rest| ~forms~}
The lisp ~forms~ are placed in the second form of an unwind-protect
outside the loop.  They are always executed after the loop has
terminated, regardless of how the termination occurred.

\end{clauses}


%%%%%%%%%%%%%%%%%%%%%%%%%%%%%%%%%%%%%%%%%%%%%%%%%%%%%%%%%%%%%%%%
\section{Other Features}

\subsection{Multiple Accumulations}
\label{multiple}

\begin{sloppypar}
It is permitted to have more than one clause accumulate into the same
variable, as in the following:
\begin{program}
(iter (for i from 1 to 10)
      (collect i into nums)
      (collect (sqrt i) into nums)
      (finally (return nums)))
\end{program}
Clauses can only accumulate into the same variable if they are
compatible.  |collect|, |adjoining|, |appending|, |nconcing|,
|unioning| and |nunioning| are compatible with each other; |sum|,
|multiply| and |counting| are compatible; |always| and |never| are compatible;
|finding| \dots |such-that| is compatible with either |thereis| or |always|
and |never| when their defaults match; and |maximize| and |minimize| clauses
are compatible only with other |maximize| and |minimize| clauses,
respectively.

%note that the same variable ~cannot~ be both an accumulation
%variable and an ordinary variable; there can be only one variable with
%a given name within an \iter\ form. 
\end{sloppypar}

\subsection{Named Blocks}

Like Common Lisp |block|s, \iter\ forms can be given names.  The
name should be a single symbol, and it must be the first form in the
\iter.  The generated code behaves exactly like a named block; in
particular, |(return-from ~name~)| can be used to exit it:
\begin{program}
(iter fred
      (for i from 1 to 10)
      (iter barney
            (for j from i to 10)
            (if (> (* i j) 17)
                (return-from fred j))))
\end{program}
An \iter\ form that is not given a name is implicitly named \nil.

Sometimes one would like to write an expression in an inner \iter\ form,
but have it processed by an outer \iter\ form.  This is possible with
the |in| clause.  

\begin{clauses}

\clausex{in}{~name~ |\&rest| ~forms~}
Evaluates ~forms~ as if they were part of the \iter\ form named
~name~.  In other words, \iter\ clauses are processed by the \iter\
form named ~name,~ and not by any \iter\ forms that occur inside ~name.~

\cpar As an example, consider the problem of collecting a list of the
elements in a two-dimensional array.  The naive solution,
\begin{program}
(iter (for i below (array-dimension ar 0))
      (iter (for j below (array-dimension ar 1))
            (collect (aref ar i j))))
\end{program}
\noindent is wrong because the list created by the inner \iter\ is simply
ignored by the outer one.  But using |in| we can write:
\begin{program}
(iter outer (for i below (array-dimension ar 0))
      (iter (for j below (array-dimension ar 1))
            (in outer (collect (aref ar i j)))))
\end{program}
\noindent which has the desired result.

\end{clauses}

\subsection{Destructuring}
\label{destructuring}

In many places within \iter\ clauses where a variable is expected, a
list can be written instead.  In these cases, the value to be assigned
is ~destructured~ according to the pattern
described by the list.  As a simple example, the clause
\begin{program}
(for (key . item) in alist)
\end{program}
\noindent will result in |key| being set to the |car| of
each element in |alist|, and |item| being set to the |cdr|.  The
pattern list may be nested to arbitrary depth, and (as the example
shows) need not be terminated with \nil; the only requirement is that
each leaf be a bindable symbol (or \nil, in which case no binding is
generated for that piece of the structure).

Sometimes, you might like to do the equivalent of a
|multiple-value-setq| in a clause.  This 
``multiple-value destructuring'' can be expressed by writing 
\linebreak |(values $pat_1$ $pat_2 \ldots$)| for a destructuring
pattern, as in 
\begin{program}
(for (values (a . b) c d) = (three-valued-function ...))
\end{program}
\begin{sloppypar}
Note that the $pat_i$ can themselves be destructuring patterns (though
not multiple-value destructuring patterns).  You can't do multiple-value
destructuring in a |with| clause; instead wrap the whole \iter\
form in a |multiple-value-bind|.
\end{sloppypar}

\begin{note}{Rationale}
There are subtle interactions between variable declarations and
evaluation order that make the correct implementation of
multiple-value destructuring in a |with| somewhat tricky.
\end{note}

The destructuring feature of \iter\ is available as a separate
mechanism, using the |dsetq| macro:

\begin{clauses}
\defmacro{dsetq}{template expr}
Performs destructuring of ~expr~ using ~template~.   May be used
outside of an \iter\ form. Yields the primary value of ~expr~.

\end{clauses}

\subsection{On-line Help}

\begin{sloppypar}
There is a limited facility for on-line help, in the form of the 
|display-iterate-clauses| function.
\end{sloppypar}

\begin{clauses}

\defun{display-iterate-clauses}{\opt\ clause-spec}
Displays a list of \iter\ clauses.  If ~clause-spec~ is not
provided, all clauses are shown; if it is a symbol, all clauses
beginning with that symbol are shown; and if it is a list of symbols,
all clauses for which ~clause-spec~ is a prefix are shown.

\end{clauses}

\subsection{Parallel Binding and Stepping}
\label{bindings}

The parallel binding and stepping of variables is a feature that
\iter\ does ~not~ have.  This section attempts to provide a rationale.

We say that two variables are bound ~in parallel~ if neither
binding shadows the other.  This is the usual semantics of |let|
(as opposed to |let*|).  Similarly, we can say that iteration
variables are stepped in parallel if neither variable is updated
before the other, conceptually speaking; in other words, if the code
to update each variable can reference the old values of both variables.

|loop| allows parallel binding of variables and parallel stepping of
driver variables.  My view is that if you are depending on the 
serial/parallel distinction, you are doing something obscure.  If you
need to bind 
variables in parallel using |with|, then you must be using a
variable name that shadows a name in the existing lexical environment.
Don't do that.  The most common use for parallel stepping is to track
the values of variables on the previous iteration, but in fact this
does not require parallel stepping at all; the following will work:
\begin{program}
(iter (for current in list)
      (for prev previous current)
      ...)
\end{program}


%%%%%%%%%%%%%%%%%%%%%%%%%%%%%%%%%%%%%%%%%%%%%%%%%%%%%%%%%%%%%%%%
\section{Types and Declarations}
\label{types}

\subsection{Discussion}

Sometimes efficiency dictates that the types of variables be declared.
This type information needs to be communicated to \iter\ so it can
bind variables to appropriate values.  Furthermore, \iter\ must often
generate internal variables invisible to the user; there needs to be a
way for these to be declared.

As an example, consider this code, which will return the number of
odd elements in |number-list|:
\begin{program}
(iter (for el in number-list)
      (count (oddp el)))
\end{program}
In processing this form,
\iter\ will create an internal variable, let us call it |list17|, to
hold the successive |cdr|s of |number-list|, and  
will bind the variable to |number-list|.  It will also generate a
default binding for |el|; only inside the body of the loop will |el|
be set to the |car| of |list17|.  Finally, \iter\ will generate a
variable, call it |result|, to hold the result of the count, and will
bind it to zero.

When dealing with type declarations, \iter\ observes one simple rule:
~it will never generate a declaration unless requested to do so.~  The
reason is that such declarations might mask errors in compiled code by
avoiding error-checks; the resulting problems would be doubly hard to
track down because the declarations would be hidden from the
programmer.  Of course, a compiler might omit error-checks even in the
absence of declarations, though this behavior can usually be avoided,
e.g. by saying |(declaim (optimize (safety 3)))|.

So, the above \iter\ form will generate code with no declarations.
But say we wish to declare the types of |el| and the internal
variables |list17| and |result|.  How is this done?

Declaring the type of |el| is easy, since the programmer knows
the variable's name:
\begin{program}
(iter (for el in number-list)
      (declare (fixnum el))
      (counting (oddp el)))
\end{program}
\iter\ can read variable type declarations like this one.  Before
processing any clauses, it scans the entire top-level form for type
declarations 
and records the types, so that variable bindings can be performed
correctly.  In this case, |el| will be bound to zero
instead of \nil.  Also, \iter\ collects all the top-level declarations
and puts them at the begining of the generated code, so it is not
necessary to place all declarations at the beginning of an \iter\
form; instead, they can be written near the variables whose types they
declare.  

Since \iter\ is not part of the compiler, it will not know
about declarations that occur outside an \iter\ form; these
declarations must be repeated inside the form.

Here is another way we could have declared the type of |el|:
\begin{program}
(iter (for (the fixnum el) in number-list)
      (counting (oddp el)))
\end{program}
\iter\ extends the Common Lisp |the|\lispindex{the}
form to apply to variables as well as value-producing forms; anywhere
a variable is allowed---in a |with| clause, as the iteration
variable in a driver clause, as the |into| argument of an
accumulation clause, even inside a destructuring template---you can
write |(the ~type~ ~symbol~)| instead.

There is one crucial difference between using a |the| form and
actually declaring the variable: explicit declarations are always
placed in the generated code, but type information from a |the|
form is not turned into an actual declaration unless you tell \iter\
to do so using |iterate:declare-variables|.  See below.

\begin{sloppypar}
Declaring the types of internal variables is harder than declaring the
types of explicitly mentioned variables, since their names
are unknown.  You do it by declaring |iterate:declare-variables|
somewhere inside the top level of the \iter\ form.  (This will also
generate declarations for variables declared using |the|.)
\iter\ does not provide much selectivity here: it's all or none. 
And unfortunately, since \iter\ is not privy to compiler information
but instead reads declarations itself, it will not hear if you
|(declaim (iterate:declare-variables))|.  Instead, set the variable 
|iterate::*always-declare-variables*| to |t| at
compile-time, using |eval-when|.
\end{sloppypar}

To determine the appropriate types for internal variables, \iter\ uses
three sources of information:

\begin{itemize}

\item Often, the particular clause dictates a certain type for a
variable; \iter\ will use this information when available.  In the
current example, the variable |list17| will be given the type
|list|, since that is the only type that makes sense; and the
variable |result| will be given the type |fixnum|, on the
assumption that you will not be counting high enough to need bignums.
You can override this assumption only by using and explicitly declaring a
variable: 
\begin{verbatim}
(iter (declare (iterate:declare-variables))
      (for el in number-list)
      (count (oddp el) into my-result)
      (declare (integer my-result))
      (finally (return my-result)))
\end{verbatim}

\begin{sloppypar}
Other examples of the type assumptions that \iter\ makes are: type
|list| for |into| variables of collection clauses; type |list| for
expressions that are to be destructured; type |vector| for the
variable holding the vector in a |for\dots in-vector| clause, and
similarly for |string| and the |for\dots in-string| clause;
and the implementation-dependent type
|(type-of array-dimension-limit)| for the index and limit
variables generated by sequence iteration drivers like |for\dots
in-vector| and |for\dots in-string| (but not |for\dots in-sequence|,
because it may be used to iterate over a list). 
\end{sloppypar}

\item Sometimes, \iter\ will examine expressions and try to determine
their types in a simple-minded way.  If the expression is
self-evaluating (like a number, for instance), \iter\ knows that the
expression's type is the same as the type of the value it denotes, so
it can use that type.  If the expression is of the form |(the ~type~
~expr~)|, \iter\ is smart enough to extract ~type~ and use it.
However, the current version of \iter\ does not  
examine declarations of function result types or do any type
inference.  It will not determine, for 
example, that the type of |(+ 3 4)| is |fixnum|, or even
|number|.

\item In some cases, the type of an internal variable should match the
type of some other variable.  For instance, \iter\ generates an
internal variable for |(f x)| in the
clause |(for i from 1 to (f x))|, and in the absence of other
information will give it the same type as |i|.  If, however, the
expression had been written |(the fixnum (f x))|, then \iter\
would have given the internal variable the type |fixnum|
regardless of |i|'s type.  The type incompatibility errors that
could arise in this situation are not checked for.

\end{itemize}

Note that if you do declare |iterate:declare-variables|, then
\iter\ may declare user variables as well as internal ones if they do
not already have declarations, though only for variables that it
binds.  For instance, in this code:

\begin{program}
(iter (declare (iterate:declare-variables))
      (for i from 1 to 10)
      (collect i into var))
\end{program}
the variable |var| will be declared to be of type |list|.


\subsection{Summary}

\iter\ understands standard Common Lisp variable type declarations
that occur within an \iter\ form and
will pass them through to the generated code.  If the declaration
|(iterate:declare-variables)|\lispindex{declare-variables}
appears at the top level of an 
\iter\ form, or if 
|iterate::*always-declare-variables*|\lispindex{*always-declare-variables*} 
is \nonnil, then \iter\ will use the type information gleaned from user
declarations, self-evaluating expressions and |the| expressions,
combined with reasonable assumptions, to determine variable
types and declare them.


%%%%%%%%%%%%%%%%%%%%%%%%%%%%%%%%%%%%%%%%%%%%%%%%%%%%%%%%%%%%%%%%
\section{Problems with Code Movement}
\label{code-movement}

Some \iter\ clauses, or parts of clauses, result in code being
moved from the location of the clause to other parts of the loop.
Drivers behave this way, as do code-placement clauses like |initially|
and |finally|.  When using these clauses, there is a danger of writing
an expression that makes sense in its apparent location but will be
invalid or have a different meaning in another location.  For example:
\begin{program}
(iter (for i from 1 to 10)
      (let ((x 3))
        (initially (setq x 4))))
\end{program}
While it may appear that the |x| of |(initially (setq x 4))| is the
same as the |x| of |(let ((x 3)) \dots|, in fact they are not:
|initially| moves its code outside the loop body, so |x| would refer
to a global variable.  Here is another example of the same problem:
\begin{program}
(iter (for i from 1 to 10)
      (let ((x 3))
        (collect i into x)))
\end{program}
If this code were executed, |collect| would create a binding for its
|x| at the top level of the \iter\ form that the |let| will shadow.

Happily, \iter\ is smart enough to catch these errors; it
walks all problematical code to ensure that free variables are not
bound inside the loop body, and checks all variables it binds for the
same problem.

However, some errors cannot be caught:

\begin{program}
(iter (with x = 3)
      (for el in list)
      (setq x 1)
      (reducing el by \#'+ initial-value x))
\end{program}
|reducing| moves its |initial-value| argument to the initialization
part of the loop in order to produce more efficient code.  Since
\iter\ does not perform data-flow analysis, it cannot determine that
|x| is changed inside the loop; all it can establish is that |x| is
not bound internally.  Hence this code will not signal an
error and will use $3$ as the initial value of the reduction.

The following list summarizes all cases that are subject to these code
motion and variable-shadowing problems.
\begin{itemize}
\item Any variable for which \iter\ creates a binding, including those
used in |with| and the |into| keyword of many clauses.

\begin{sloppypar}
\item The special clauses which place code: |initially|, |after-each|, |else|,
|finally| and |finally-protected|. 
\end{sloppypar}

\item The variables of a |next| or |do-next| form.

\item The |initially| arguments of |for\dots initially\dots then| and
|for\dots previous|. 

\item The |then| argument of |for\dots initially\dots then|.

\item The |initial-value| arguments of |reducing| and |accumulate|.

\item The |on-failure| argument of |finding\dots such-that|.
\end{itemize}

%%%%%%%%%%%%%%%%%%%%%%%%%%%%%%%%%%%%%%%%%%%%%%%%%%%%%%%%%%%%%%%%
\section{Differences Between {\tt Iterate} and {\tt Loop}}
|loop| contains a great deal of complexity which \iter\ tries to
avoid.  Hence many esoteric features of |loop| don't exist in \iter.
Other features have been carried over, but in a cleaned-up form.  
And of course, many new features have been added; they are not
mentioned in this list.

\begin{itemize}

\item \iter's syntax is more Lisp-like than |loop|'s, having a higher
density of parens.

\item The current implementation of \iter, unlike the current version
of  |loop| (as documented in {\em Common Lisp, 2nd Ed.\/}), is
extensible (see section \ref{extend}).  

\item |loop| puts the updates of all driver variables at the top of
the loop; \iter\ leaves them where the driver clauses appear.

\item While for the most part \iter\ clauses that resemble |loop| clauses
behave similarly, there are some differences.  For instance, there is
no |for\dots =\dots then| in \iter; instead use 
|for\dots initially\dots then|.  

\item |loop| binds the variable |it| at certain times
to allow pseudo-English expressions like |when ~expr~ return it|. 
In \iter, you must bind ~expr~ to a variable yourself.  Note that
|when ~expr~ return it| is like |thereis ~expr~| except that the latter is an
accumulation clause and therefore competes with other accumulations
(remember section \ref{multiple} above).

% repeat different behaviour of |always| clause here?

\item |loop| has a special |return| clause, illustrated in the
previous item.  \iter\ doesn't need one,  since an ordinary Lisp
|return| has the same effect.

\item |loop| allows for parallel binding and stepping of iteration
variables.  \iter\ does not.  (See section \ref{bindings}.)

\item |loop| and \iter\ handle variable type declarations very
differently.  |loop| provides a special syntax for declaring variable
types, and does not examine declarations.  Moreover, the standard
implementation of |loop| will
generate declarations when none are requested.
\iter\ parses standard Common Lisp type declarations, and will never
declare a variable itself unless declarations are 
specifically requested.

\end{itemize}

%%%%%%%%%%%%%%%%%%%%%%%%%%%%%%%%%%%%%%%%%%%%%%%%%%%%%%%%%%%%%%%%
\section{Rolling Your Own}
\label{extend}

\subsection{Introduction}

\iter\ is extensible---you can write new clauses that embody new
iteration patterns.  You might want to write a new driver clause for a
data structure of your own, or you might want to write a clause that
collects or manipulates elements in a way not provided by \iter. 

This section describes how to write clauses for \iter.  Writing a
clause is like writing a macro.  In fact, writing a clause ~is~
writing a macro: since \iter\ code-walks its body and macroexpands,
you can add new abstractions to \iter\ with good old |defmacro|.

Actually, there are two extensions you can make to \iter\ that are
even easier than writing a macro.  They are adding a synonym for an
existing clause and defining a driver clause for an indexable
sequence.  These can be done with |defsynonym| and
|defclause-sequence|, respectively.  See section \ref{aids}, below.

The rest of this section explains how to write macros that expand into
\iter\ clauses.
Here's how you could add a simplified version of \iter's |multiply|
clause, if \iter\ didn't already have one:
\begin{program}
(defmacro multiply (expr)
  `(reducing ,expr by \#'* initial-value 1))
\end{program}

If you found yourself summing the square of an expression often, you
might want to write a macro for that.  A first cut might be
\begin{program}
(defmacro sum-of-squares (expr)
  `(sum (* ,expr ,expr)))
\end{program}
but if you are an experienced macro writer, you will realize that this
code will evaluate ~expr~ twice, which is probably a bad idea.  A
better version would use a temporary:
\begin{program}
(defmacro sum-of-squares (expr)
  (let ((temp (gensym)))
    `(let ((,temp ,expr))
       (sum (* ,temp ,temp)))))
\end{program}
Although this may seem complex, it is just the sort of thing you'd
have to go through to write any macro, which illustrates the point of
this section: if you can write macros, you can extend \iter.

Our macros don't use \iter's keyword-argument
syntax.  We could just use keywords with |defmacro|, but we would
still not be using \iter's clause indexing mechanism.  Unlike Lisp,
which uses just the first symbol of a form to determine what function
to call, \iter\ individuates clauses by the list of required keywords.
For instance, |for\dots in| and |for\dots in-vector| are different clauses
implemented by distinct Lisp functions.  

To buy into this indexing scheme, as well as the keyword-argument
syntax, use |defmacro-clause|:

\begin{clauses}

\defmacro{defmacro-clause}{arglist |\&body| body}
Defines a new \iter\ clause.  ~arglist~ is a list of symbols which are
alternating keywords and arguments.  \opt\ may be used, and the list
may be terminated by |\&sequence|.  ~body~ is an ordinary macro body,
as with |defmacro|.  If the first form of ~body~ is a string, it is
considered a documentation string and will be shown by
|display-iterate-clauses|. |defmacro-clause| will signal an error if 
defining the clause would result in an ambiguity.  E.g. you cannot
define the clause |for\dots from| because there would be no way to
distinguish it from a use of the |for| clause with optional keyword |from|.

\end{clauses}

\medskip

Here is |multiply| using |defmacro-clause|.  The keywords are capitalized
for readability.
\begin{program}
(defmacro-clause (MULTIPLY expr \opt\ INTO var)
  `(reducing ,expr by \#'* into ,var initial-value 1))
\end{program}
You don't have to worry about the case when |var| is not supplied; for
any clause with an |into| keyword, saying |into nil| is equivalent to
omitting the |into| entirely.

As another, more extended example, consider the fairly common
iteration pattern that 
involves finding the sequence element that maximizes (or minimizes) some
function.  \iter\ provides this as |finding\dots maximizing|, but it's
instructive to see how to write it.
Here, in pseudocode, is how you might write such a loop for
maximizing a function F:

\begin{quote}
\begin{tabbing}
set variable MAX-VAL to NIL; \\
set variable WINNER to NIL; \\
for\=\ each element EL in the sequence \\
\>  if\=\ MAX-VAL\=\ is NIL or F(EL) $>$ MAX-VAL then \\
\>\>    set MAX-VAL to F(EL); \\
\>\>    set WINNER to EL; \\
\>  end if; \\
end for; \\
return WINNER.
\end{tabbing}
\end{quote}

Here is the macro:
\begin{program}
(defmacro-clause (FINDING expr MAXIMIZING func \opt\ INTO var)
  (let ((max-val (gensym))
        (temp1 (gensym))
        (temp2 (gensym))
        (winner (or var iterate::*result-var*)))
    `(progn 
       (with ,max-val = nil)
       (with ,winner = nil)
       (cond 
        ((null ,max-val)
         (setq ,winner ,expr)
         (setq ,max-val (funcall ,func ,winner))
        (t
         (let* ((,temp1 ,expr)
                (,temp2 (funcall ,func ,temp1)))
           (when (> ,temp2 ,max-val)
             (setq ,max-val ,temp2)
             (setq ,winner ,temp1))))))
       (finally (leave ,winner)))))
\end{program}
Note that if no |into| variable is supplied, we use
|iterate::*result-var*|, which contains the internal variable into
which all clauses place their results.  If this variable is bound by
some clause, then \iter\ will return its value automatically;
otherwise, \nil\ will be returned.  

\subsection{Writing Drivers}

In principle, drivers can be implemented just as easily as other
\iter\ clauses.  In practice, they are a little harder to get right. 
As an example, consider writing a driver that
iterates over all the 
elements of a vector, ignoring its fill-pointer.  |for\dots in-vector|
won't work for this, because it observes the fill-pointer.  It's necessary to
use |array-dimension| instead of |length| to obtain the size of the
vector.  Here is one approach:

\begin{program}
(defmacro-clause (FOR var IN-WHOLE-VECTOR v)
  "All the elements of a vector (disregards fill-pointer)"
  (let ((vect (gensym))
        (index (gensym)))
    `(progn
       (with ,vect = ,v)
       (for ,index from 0 below (array-dimension ,vect 0))
       (for ,var = (aref ,vect ,index)))))
\end{program}
Note that we immediately put |v| in a variable, in case it is an
expression.  Again, this is just good Lisp macrology.  It also has a
subtle effect on the semantics of the driver: |v| is evaluated only
once, at the beginning of the loop, so changes to |v| in the loop have
no effect on the driver.  Similarly, the bounds for numerical iteration
e.g. the above |array-dimension| are also evaluated once only.  This is how
all of \iter's drivers work.


There is an important point concerning the |progn| in this code.  We
need the |progn|, of course, because we are returning several forms,
one of which is a driver.  But \iter\ drivers must occur at top-level.
Is this code in error?  No, because ~top-level~ is defined in \iter\
to include forms inside a |progn|.  This is just the definition of
top-level that Common Lisp uses, and for the same reason: to allow
macros to return multiple forms at top-level.

While our |for\dots in-whole-vector| clause will work, it is not
ideal.  In particular, it does not support generating.  Do do so, we
need to use |for\dots next| or |for\dots do-next|.  The job is
simplified by the |defmacro-driver| macro.

\begin{clauses}

\defmacro{defmacro-driver}{arglist |\&body| body}
Defines a driver clause in
both the |for| and |generate| forms, and provides a parameter
|generate| which ~body~ can examine to determine how it was invoked.
~arglist~ is as in |defmacro-clause|, and should begin with the symbol
|for|. 

\end{clauses}

With |defmacro-driver|, our driver looks like this:
\begin{program}
(defmacro-driver (FOR var IN-WHOLE-VECTOR v)
  "All the elements of a vector (disregards fill-pointer)"
   (let ((vect (gensym))
         (end (gensym))
         (index (gensym))
         (kwd (if generate 'generate 'for)))
     `(progn
        (with ,vect = ,v)
        (with ,end = (array-dimension ,vect 0))
        (with ,index = -1)
        (,kwd ,var next (progn (incf ,index)
                               (if (>= ,index ,end) (terminate))
                               (aref ,vect ,index))))))
\end{program}


We are still missing one thing: the |\&sequence| keywords.  
We can get them easily enough, by writing 
\begin{program}
(defmacro-driver (FOR var IN-WHOLE-VECTOR v \&sequence) 
  ...)
\end{program}
We can now refer to parameters |from|, |to|, |by|, etc. which contain
either the values for the corresponding keyword, or \nil\ if the
keyword was not supplied.  Implementing the right code for these
keywords is cumbersome but not difficult; it is left as an exercise.
But before you begin, see |defclause-sequence| below for an easier way.
       
\subsection{Extensibility Aids}
\label{aids}

This section documents assorted features that may be of use in
extending \iter.

\begin{clauses}

\defunexpvar{*result-var*}
Holds the variable that is used to return a value as a result of the
\iter\ form.  You may examine this and use it in a |with| clause, but
you should not change it.

\defmacro{defsynonym}{syn word}
 Makes ~syn~ a synonym for the existing \iter\ keyword ~word.~  Only
the first word in each clause can have synonyms. 

\defmacroxx{defclause-sequence}{element-name index-name}{|\&key|
access-fn size-fn sequence-type}{element-type
element-doc-string index-doc-string}  
Provides
a simple way to define sequence clauses.  Generates two
clauses, one for iterating over the sequence's elements, the other
for iterating over its indices.  The first symbol of both
clauses will have print-name |for|.
~element-name~  and ~index-name~ should be symbols.
~element-name~ is the second keyword of the element iterator (typically of
the form 
|in-~sequence-type~|), and ~index-name~ is the second keyword
of the index-iterator (typically of the form
|index-of-~sequence-type~|).  Either name may be 
\nil, in which case the corresponding clause is not defined.  If both
symbols are supplied, they should be in the same package.  The |for|
that begins the clauses will be in this package.

\cpar ~access-fn~ is the function to be used to
access elements of the sequence in the element iterator.  The function
should take two 
arguments, a sequence and an index, and return the appropriate element.
~size-fn~ should denote a function of one argument, a sequence, that
returns its size.  Both ~access-fn~ and ~size-fn~ are required for the
element iterator, but only ~size-fn~ is needed for the index iterator.

\cpar The ~sequence-type~ and ~element-type~ keywords are used to
suggest types for the variables 
used to hold the sequence and the
sequence elements, respectively.  The usual rules about \iter's
treatment of variable type declarations apply (see section \ref{types}).

\cpar ~element-doc-string~ and ~index-doc-string~ are
the documentation strings, for use with |display-iterate-clauses|. 

\cpar The generated element-iterator performs destructuring on the
element variable.

\cpar As an example, the above |for\dots in-whole-vector| example
could have been written:
\begin{program}
(defclause-sequence IN-WHOLE-VECTOR INDEX-OF-WHOLE-VECTOR
  :access-fn 'aref
  :size-fn \#'(lambda (v) (array-dimension v 0))
  :sequence-type 'vector
  :element-type t
  :element-doc-string 
     "Elements of a vector, disregarding fill-pointer"
  :index-doc-string 
     "Indices of vector, disregarding fill-pointer")
\end{program}

\end{clauses}

\subsection{Subtleties}

There are some subtleties to be aware of when writing \iter\ clauses.
First, the code returned by your macros may be |nconc|'ed into a list,
so you should always returned freshly consed lists, rather than
constants.  Second, \iter\ matches clauses by using |eq| on the first
symbol and |string=| on the subsequent ones, so the package of the
first symbol of a clause is relevant.  All of the clauses in this manual
have their first word in the \iter\ package.  
You can use the package system in the usual way to shadow
\iter\ clauses without replacing them.

%%% say more here, about the badness that only the first word of a
%%% clause is packagey.

\section{Non-portable Extensions to Iterate (Contribs)}
\label{contribs}

Currently, there is only one non-portable extension to iterate in the
distribution: iterate-pg. If you have made an extension that depends
on non-portable features, feel free to send them to |asf@boinkor.net|
for inclusion in the iterate distribution.

\subsection{An SQL query driver for iterate}

The pg package by Eric Marsden (see |http://cliki.net/pg|) provides an
interface to the PostgreSQL database. Using the \iterpg\ extension, it
is possible to handle the results of SQL queries using \iter.

This usage example should give you an idea of how to use it:

\begin{program}
(pg:with-pg-connection (c "somedb" "someuser")
  (iter (for (impl version date) in-relation "select * from version"
                                 on-connection *dbconn*)
        (collect version)))
\end{program}

To use the extension via |ASDF|, simply make your system depend on the
|iterate-pg| system instead of the |iterate| system. To load it
manually, use:

\begin{program}
  (asdf:oos 'asdf:load-op :iterate-pg)
\end{program}

%%%%%%%%%%%%%%%%%%%%%%%%%%%%%%%%%%%%%%%%%%%%%%%%%%%%%%%%%%%%%%%%
\section{Obtaining {\tt Iterate}}

\begin{sloppypar}
The information in this chapter is obsolete but included for
completeness's sake; Currently, the most up-to-date information on
\iter\ can be found at |http://boinkor.net/iterate.html|.
\end{sloppypar}

\begin{sloppypar}
\iter\ currently runs on Lisp Machines, and on
HP's, Sun3's and Sparcstations under Lucid.  
\iter\ source and binaries are available at the MIT AI Lab in the
subdirectories of |/src/local/lisplib/|.  The source file,
|iterate.lisp|, is also available for anonymous FTP in the directory
|/com/fpt/pub/| on the machine |TRIX.AI.MIT.EDU| (Internet number
128.52.32.6).  If you are unable to obtain |iterate| in one of these
ways, send mail to |jba@ai.mit.edu| and I will send you the source
file. 
\end{sloppypar}

\begin{sloppypar}
\iter\ resides in the |iterate| package (nickname |iter|).  Just say
\linebreak |(use-package :iterate)| to make all the necessary symbols
available. 
If a symbol is not exported, it appears in this manual with an
``|iterate::|'' prefix.
\end{sloppypar}

Send bug reports to |bug-iterate@ai.mit.edu|.  The |info-iterate|
mailing list will have notices of changes and problems; to have
yourself added, send mail to |info-iterate-request@ai.mit.edu|.


\medskip

\begin{flushleft}
 \bf Acknowledgements
\end{flushleft}
\smallskip

Richard Waters provided invaluable criticism which spurred me to improve
\iter\ greatly.  As early users, David Clemens, Oren Etzioni and Jeff
Siskind helped ferret out many bugs.

%\begin{theindex}
% The index files must be generated with the genindex program
% or from makeindex iterate-manual.idx which creates iterate-manual.ind.

%\baselineskip\normalbaselineskip
%\advance\baselineskip by -2pt

% Fix ``.'' spacing in lists
\documentstyle[12pt,aimemo]{article}

\setlength{\textwidth}{6.0in}
\setlength{\oddsidemargin}{0.25in}
\setlength{\evensidemargin}{0.25in}
\setlength{\topmargin}{0pt}
\setlength{\headsep}{24pt}
\setlength{\textheight}{8.5in}

\let\origpar=\par

% somehow, LaTex fucks with things so that the plain Tex \obeyspaces
% doesn't work.  This is the same, except we use an mbox containing
% a space instead of just a space.

\newcommand{\spc}{\mbox{ }}
\newcommand{\tAb}{\mbox{        }}
\def\obspaces{\catcode`\ =\active\catcode`\^^I=\active}
{\obspaces \global\let =\spc\global\let^^I=\tAb}

\newenvironment{program}{\medskip\samepage\tt\obeylines\obspaces}{\medskip}

% use | for \tt
\catcode`\|=13
\newif\ifvbar
\def|{\ifvbar
         \endgroup\vbarfalse
      \else 
         \begingroup\tt\vbartrue 
      \fi}



\newcommand{\lisp}{\tt}
\newcommand{\iter}{|iterate|}
\newcommand{\iterpg}{|iterate-pg|}
\newcommand{\setf}{|setf|}
\newcommand{\nil}{|nil|}
\newcommand{\nonnil}{non-\nil}
\newcommand{\opt}{{\lisp \&optional}}
\newcommand{\yields}{$\Rightarrow$}



\newbox\Dots
\setbox\Dots=\hbox{\small ...}
\renewcommand{\dots}{\copy\Dots}

% Here's a way to define an environment which will treat all
% paragraphs inside it in the same manner.  It's, as near as I can
% figure, the way that Latex does things.  Using \everypar won't work;
% apparently Latex resets that a lot.  What you have to do is: at the
% top of the file do \let\origpar=\par.  In the
% begin part of the environment, do the parshape (or whatever), then say
% \def\par{{\origpar}}.  The extra braces are necessary.   That's all
% you have to do; at the end of the environment things will return to
% normal automatically.  Why does this all work?  I have no idea.
% Another important thing: there should be a blank line (i.e. a \par)
% between the end of the last paragraph and the \end{environment}.

\newlength{\clindent}
\setlength{\clindent}{0.5in}
\newlength{\clparindent}
\setlength{\clparindent}{\parindent}

\newenvironment{clauses}{\advance\linewidth -\parindent
                        \hangindent\clindent\relax
                         \hangafter=0
                          \parindent=0pt
                          \def\par{{\origpar}}}{}

\newcommand{\cpar}{\hspace \clparindent}

\newcommand{\startitem}{\bigskip\pagebreak[3]}
\newcommand{\finishitem}{\smallskip}

\newcommand{\clause}[1]{\startitem\Clause{#1}\finishitem}
\newcommand{\Clause}[1]{\hbox{#1}}
\newcommand{\clindex}[1]{\index{{\lisp #1}}}
\newcommand{\clindexx}[2]{\index{{\lisp #1\dots #2}}}
\newcommand{\clindexxx}[3]{\index{{\lisp #1\dots #2\dots #3}}}
\newcommand{\clausex}[2]{\startitem\Clausex{#1}{#2}\finishitem}
\newcommand{\Clausex}[2]{\Clause{|#1| #2}\clindex{#1}}

\newcommand{\defvar}[1]{\startitem\deff{#1}{}{Variable}\finishitem}
\newcommand{\defunexpvar}[1]{\startitem\deffnoindex{iterate::#1}{}{Variable}
                                        \lispindex{#1}\finishitem}
\newcommand{\defun}[2]{\startitem\deff{#1}{#2}{Function}\finishitem}
\newcommand{\defmacro}[2]{\startitem\deff{#1}{#2}{Macro}\finishitem}
\newcommand{\defunx}[3]{\startitem\deffx{#1}{#2}{#3}{Function}\finishitem}
\newcommand{\defmacrox}[3]{\startitem\deffx{#1}{#2}{#3}{Macro}\finishitem}
\newcommand{\defunxx}[4]{\startitem\deffxx{#1}{#2}{#3}{#4}{Function}\finishitem}
\newcommand{\defmacroxx}[4]{\startitem\deffxx{#1}{#2}{#3}{#4}{Macro}\finishitem}

\newcommand{\deff}[3]{\deffnoindex{#1}{#2}{#3}\lispindex{#1}}
\newcommand{\deffnoindex}[3]{\hbox to \hsize{{\tt #1} {\it #2}\hfill [{\it #3}]}}
\newcommand{\deffx}[4]{\deff{#1}{#2}{#4}\moreargs{#1}{#3}}
\newcommand{\deffxx}[5]{\deffx{#1}{#2}{#3}{#5}\moreargs{#1}{#4}}

\newcommand{\moreargs}[2]{\hbox to \hsize{\phantom{\lisp #1} {\it #2}\hfill}}

\newcommand{\lispindex}[1]{\index{{\lisp #1}}}

 \newenvironment{note}[1]{\pagebreak[2]\bigskip
                          \hrule\smallskip\small
                         {\setlength{\parindent}{0pt}
                           \par{\bf #1:}}}{\smallskip\hrule\bigskip}



\makeindex

%%%%%%%%%%%%%%%%%%%%%%%%%%%%%%%%%%%%%%%%%%%%%%%%%%%%%%%%%%%%%%%%

%\thispagestyle{empty}

%\vspace*{2in}

%\title{The |Iterate| Manual}
%\author{Jonathan Amsterdam}

%\maketitle

%\thispagestyle{empty}

%\begin{center}
%\Large *** DRAFT ***
%\end{center}
%This manual is in beta-test. It will become an AI Memo in a couple of
%months.  I would appreciate any comments or
%corrections.  You can reach me at email address |jba@ai.ai.mit.edu|.

\begin{document}

\def\writtenby{Jonathan Amsterdam}
\def\contractno{This report describes research done at the Artificial
Intelligence Laboratory of the Massachusetts Institute of Technology.
Support for the laboratory's artificial intelligence research is
provided in part by the Advanced Research Projects Agency of the
Department of Defense under Office
of Naval Research contract N00014-85-K-0124.}

\AIMEMO{1236}{\Large The Iterate Manual \\[1in]}
{This is the manual for version 1.4 of \iter, a powerful iteration
macro for Common  
Lisp. \iter\ is similar to {\tt loop} but provides numerous additional
features, is well integrated with Lisp, and is extensible. 
\vfill }


% use ~ for \em
\catcode`\~=13
\newif\ifem
\def~{\ifem
         \/\endgroup\emfalse
      \else 
         \begingroup\em\emtrue 
      \fi}



%\pagebreak
%\setcounter{page}{1}

%\thispagestyle{empty}


\tableofcontents

\pagebreak
\pagestyle{headings}

\section{Introduction}

\begin{sloppypar}
This manual describes
\iter, a powerful iteration facility for Common Lisp.
\iter\ provides abstractions for many common iteration
patterns and allows for the definition of additional patterns.  \iter\ is
a macro that expands into ordinary Lisp at
compile-time, so it is more efficient than higher-order
functions like |map| and |reduce.|  
While it is similar to |loop|, \iter\ offers a more Lisp-like syntax
and enhanced extensibility.
(For a more complete
comparison of \iter\ with other iteration constructs, see MIT AI Lab Working
Paper No. 324, ~Don't Loop, Iterate.~)
\end{sloppypar}

An \iter\ form consists of the symbol |iter|\footnote{You can also use
|iterate|, but |iter| is preferred because it avoids potential conflicts with
possible future additions to Common Lisp, and because it saves
horizontal space when writing code.}
followed by one or more
forms, some of which may be \iter\ 
~clauses.~  Here is a simple example of \iter\ which collects the
numbers from 1 to 10 into a list, and returns the list.  The return
value is shown following the arrow.
\begin{program}
(iter (for i from 1 to 10)
      (collect i))          \yields (1 2 3 4 5 6 7 8 9 10)
\end{program}
This form contains two clauses: a |for| clause that steps the
variable |i| over the integers from 1 to 10, and a |collect|
clause that accumulates its argument into a list.  With a few
exceptions, all \iter\ clauses have the same format: alternating
symbols (called ~keywords~) and expressions (called
~arguments~).  The syntax and terminology are those of Common
Lisp's keyword lambda lists.  One difference is that \iter's keywords
do not have to begin with a colon---though they may, except for the
first symbol of a clause.  So you can also write  |(for i :from
1 :to 10)| if you prefer.

Any Lisp form can appear in the body of an \iter, where it will have
its usual meaning.  \iter\ walks the entire body, expanding macros,
and recognizing clauses at any level.  This example collects all the
odd numbers in a list:
\begin{program}
(iter (for el in list)
      (if (and (numberp el) (oddp el))
          (collect el)))
\end{program}

There are clauses for iterating over numbers, lists, arrays and other
objects, and for
collecting, summing, counting, maximizing and other useful operations.
\iter\ also supports the creation of new variable bindings, stepping
over multiple sequences at once, destructuring, and compiler
declarations of variable types.  The following example illustrates
some of these features: 
\begin{program}
(iter (for (key . item) in alist)
         (for i from 0)
         (declare (fixnum i))
         (collect (cons i key)))
\end{program}

This loop takes the keys of an alist and returns a new alist
associating the keys with their positions in the original list. The
compiler declaration for |i| will appear in the generated code
in the appropriate place.

%%%%%%%%%%%%%%%%%%%%%%%%%%%%%%%%%%%%%%%%%%%%%%%%%%%%%%%%%%%%%%%%
\section{Clauses}

Most of \iter's clauses will be familiar to |loop| programmers.  
(|loop| is an iteration macro that has been incorporated into Common
Lisp.  Seen Guy Steele's {\em Common Lisp, 2nd Edition}.)
In
nearly all cases they behave the same as their |loop| counterparts, so
a |loop| user can switch to \iter\ with little pain (and much gain).

All clauses with the standard keyword-argument syntax consist of two
parts: a ~required~ part, 
containing keywords that must be present and in the right order; and
an ~optional~ part, containing keywords that may be omitted and,
if present, may occur in any order.  In the descriptions below, the
parts are separated by the Lisp lambda-list keyword \opt.


\subsection{Drivers}

An iteration-driving clause
conceptually causes the iteration to go forward.  Driver clauses in
\iter\ allow iteration over numbers, lists, vectors, hashtables, packages,
files and streams.  Iteration-driving clauses must
appear at the top level of an \iter\ form; they cannot be nested
inside another clause.  The driver variable is updated at the point
where the driver clause occurs.  Before the clause is executed for the
first time, the value of the variable is undefined.

%Also, regardless of where the driver clause
%appears in the body, the driver variable is stepped at the top of the
%loop; hence it is stylistically preferable, though
%not required, to place driver clauses at the beginning of the \iter.

Multiple drivers may appear in a single \iter\ form, in which case all
of the driver variables are updated each time through the loop, in the
order in which the clauses appear.  The first driver to terminate will
terminate the entire loop.

In all cases, the value of the driver variable on exit from the loop,
including within the epilogue code (see the |finally| clause), is
undefined.

All the parameters of a driver clause are evaluated once, before the
loop begins.  Hence it is not possible to change the bounds or other
properties of an iteration by side-effect from within the loop.

With one exception, driver clauses begin with the word |for| (or
the synonym |as|) and mention an iteration variable, which is
given a binding within the \iter\ form.
The exception is |repeat|, which just executes a loop a
specified number of times:

\begin{clauses}

\clausex{repeat}{~n~}
Repeats the loop ~n~ times.  For example:
\begin{program}
(iter (repeat 100)
      (print "I will not talk in class."))
\end{program}
\cpar If $n \leq 0$, the
loop will never be executed.  If ~n~ is not an integer, the
actual number of executions will be $\lceil n \rceil$.

\end{clauses}

\pagebreak[3]

\subsubsection{Numerical Iteration}

\begin{clauses}
\clausex{for}{~var~ |\&sequence|} 
The general form for iterating over a sequence of numbers requires a
variable and, optionally, one or more keywords that provide the bounds
and step size of the iteration. The |\&sequence|
lambda-list keyword is a shorthand for these sequence keywords.  They are:
|from|, |upfrom|, |downfrom|, |to|, |downto|, |above|, |below| and |by|.
|from| provides the starting value for ~var~
and defaults to zero. |to| provides a final value and implies that the
successive values of ~var~ will be increasing; |downto| implies that
they will be decreasing.  The loop terminates when ~var~ passes
the final value (i.e. becomes smaller or larger than it, depending on
the direction of iteration); in other  words, the loop body
will never be 
executed for values of ~var~ past the final value.
|below| and |above| are similar to |to| and
|downto|, except that the loop terminates when ~var~ equals or passes
the final value.

\cpar If no final value is specified, the variable will be stepped
forever.  Using |from| or |upfrom| will result in increasing
values, while |downfrom| will give decreasing values.

\cpar On each iteration,
~var~ is incremented or decremented by the value of the sequence
keyword |by|, which defaults to 1.  It should always be a positive
number, even for downward iterations.

\cpar In the following examples, the sequence of numbers generated is shown
next to the clause.

\begin{program}
(for i upfrom 0) \yields\ 0 1 2 \ldots
(for i from 5)   \yields\ 5 6 7 \ldots    ; either from or upfrom is okay
(for i downfrom 0) \yields\ 0 -1 -2 \ldots
(for i from 1 to 3) \yields\ 1 2 3
(for i from 1 below 3) \yields\ 1 2
(for i from 1 to 3 by 2) \yields\ 1 3
(for i from 1 below 3 by 2) \yields\ 1
(for i from 5 downto 3) \yields\ 5 4 3
\end{program}

\end{clauses}

\subsubsection{Sequence Iteration}

There are a number of clauses for iterating over sequences.  In all of
them, the argument following |for| may be a list instead of a
symbol, in which case destructuring is performed.  See section
\ref{destructuring}. 

\begin{clauses}

\clause{|for| ~var~ |in| ~list~ \opt\ |by| ~step-function~}
\clindexx{for}{in}
~var~ is set to successive elements of list.  
~step-function,~ which 
defaults to |cdr|, is used to obtain the next sublist. 

\clause{|for| ~var~ |on| ~list~ \opt\ |by| ~step-function~}
\clindexx{for}{on}
~var~ is set to successive sublists of list. 
~step-function~ (default |cdr|) is used as in |for\dots in|.

\end{clauses}

\medskip

These two clauses use |atom| to test for the end
of a list.  Hence, given a list whose final |cdr| is not \nil,
they will silently ignore the last |cdr|.  Other choices are
|endp|,  which would signal an error, and |null|, which
would probably result in an error somewhere else.  If you wish to use
an end-test other than |atom|, set the variable
|iterate::*list-end-test*|\lispindex{*list-end-test*} to the name of
the desired function. 

\begin{clauses}
\clause{|for| ~var~ |in-vector| ~vector~ |\&sequence|}
\clindexx{for}{in-vector}
~var~ takes on successive elements from ~vector.~   The vector's
fill-pointer is observed.  Here and in subsequent clauses, the |\&sequence|
keywords include |with-index|, which takes a symbol as argument
and uses it for the index variable instead of an internally generated
symbol.  The other |\&sequence| keywords behave as in numerical
iteration, except that the default iteration bounds are the bounds of
the vector.  E.g. in |(for i in-vector v downto 3)|,
|i| will start off being bound to the last element in |v|,
and will  
be set to preceding elements down to and including the element with
index 3.

\clause{|for| ~var~ |in-sequence| ~seq~ |\&sequence|}
\clindexx{for}{in-sequence}
This uses Common Lisp's generalized sequence functions, |elt|
and |length|, to obtain elements and determine the length of
~seq.~   Hence it will work for any sequence, including lists, and will
observe the fill-pointers of vectors.

\clause{|for| ~var~ |in-string| ~string~ |\&sequence|}
\clindexx{for}{in-string}
~var~ is set to successive characters of ~string.~

\startitem
\Clause{|for| ~var~ |index-of-vector| ~vector~ |\&sequence|}
\clindexx{for}{index-of-vector}
\Clause{|for| ~var~ |index-of-sequence| ~sequence~ |\&sequence|}
\clindexx{for}{index-of-sequence}
\Clause{|for| ~var~ |index-of-string| ~string~ |\&sequence|}
\clindexx{for}{index-of-string}
\finishitem
~var~ is set to successive indices of the sequence.
These clauses avoid the overhead of accessing the sequence elements
for those applications where they do not need to be examined, or are
examined rarely.  They admit all the optional keywords of the other
sequence drivers except the (redundant) |with-index| keyword.

\clause{|for| ~(key value)~ |in-hashtable| ~table~}
\clindexx{for}{in-hashtable}

~key~ and ~value,~ which must appear as shown in a list
and may be destructuring templates, are set to the keys and values of
~table~.  If ~key~ is |nil|, then the hashtable's keys will be ignored;
similarly for ~value~.
The order in which 
elements of ~table~ will be retrieved is unpredictable.

\clause{|for| ~var~ |in-package| ~package~ \opt\ |external-only| ~ext~}
\clindexx{for}{in-package}
Iterates over all the symbols in ~package,~ or over only the
external symbols if ~ext~ is specified and non-|nil|.  ~ext~ is not
evaluated.  The same symbol may appear more than once.

\clause{|for| ~var~ |in-packages| ~packages~ \opt\ |having-access| ~symbol-types~}
\clindexx{for}{in-packages}
Iterates over all the symbols from the list of packages denoted by the
descriptor ~packages~ and having accessibility (or visibility) given by
~symbol-types~.  This defaults to the list |(:external :internal :inherited)|
and is not evaluated.  ~var~ must be a list of up to three variables: in each
iteration, these will be set to a symbol, its access-type and package (as per
|with-package-iterator| in ANSI-CL).  The same symbol may appear more than
once.

\end{clauses}

\begin{clauses}

\clause{|for| ~var~ |in-file| ~name~ \opt\ |using| ~reader~}
\clindexx{for}{in-file}
Opens the file ~name~ (which may be a string or pathname) for input,
and iterates 
over its contents. ~reader~ defaults to |read|, so by default ~var~
will be bound to the successive forms in the file.  The \iter\ body is
wrapped in an unwind-protect to ensure that the file is closed no
matter how the \iter\ is exited.

\clause{|for| ~var~ |in-stream| ~stream~ \opt\ |using| ~reader~}
\clindexx{for}{in-stream}
Like |for\dots in-file|, except that ~stream~ should be an existing
stream object that supports input operations.

\end{clauses}

\subsubsection{Generalized Drivers}

These are primarily useful for writing drivers that can also be used
as generators (see section \ref{generators}, below).
 
\begin{clauses}

\clause{|for| ~var~ |next| ~expr~}
\clindexx{for}{next}

~var~ is set to ~expr~ each time through the loop.  Destructuring is
performed.  When the clause is used as a generator, ~expr~ is the code
that is executed when |(next ~var~)| is 
encountered (see section \ref{generators}, below).
~expr~ should compute the first value for ~var~, as well
as all subsequent values, and is responsible for terminating the loop.
For compatibility with future versions of \iter, this termination
should be done with |terminate|\clindex{terminate}, which can be
considered a synonym for |finish| (see section \ref{control-flow}).

\cpar As an example, the following clauses are equivalent to |(for
i from 1 to 10)|:
\begin{program}
(initially (setq i 0))
(for i next (if (> i 10) (terminate) (incf i)))
\end{program}

\clause{|for| ~var~ |do-next| ~form~}
\clindexx{for}{do-next}
~form~ is evaluated each time through the loop.  Its value is ~not~
set to ~var~; that is ~form's~ job. ~var~ is only present so that
\iter\ knows it is a driver variable.  \linebreak |(for ~var~ next
~expr~)| is equivalent to |(for ~var~ do-next (dsetq ~var~ ~expr~))|.
(See section \ref{destructuring} for an explanation of |dsetq|.) 

\end{clauses}

\subsubsection{Generators}
\label{generators}

In all of the above clauses, the driver variable is updated on each
iteration.
Sometimes it is desirable to have greater control over updating.  
For instance, consider the problem of associating numbers, in
increasing order and with no gaps, with the
\nonnil\ elements of a list.  One obvious first pass at writing this is:

\begin{program}
(iter (for el in list)
      (for i upfrom 1)
      (if el (collect (cons el i))))
\end{program}
But on the list |(a b nil c)| this produces |((a . 1) (b . 2) (c .
4))| instead of the desired |((a . 1) (b . 2) (c . 3))|.  The problem
is that |i| is incremented each time through the loop, even when |el|
is |nil|.  

The problem could be solved elegantly if we could step |i| only when
we wished to.  This
can be accomplished for any \iter\ driver by writing |generate| (or
its synonym |generating|)
instead of |for|.  Doing so produces a ~generator~---a driver whose
values are yielded explicitly.  To obtain the next value of a
generator variable ~v~, write \linebreak |(next ~v~)|.  The value of a |next|
form is the next value of ~v~, as determined by its associated driver
clause.  |next| also has the side-effect of updating ~v~ to that
value.  If there is no next value, |next| will terminate the loop,
just as with a normal driver.


Using generators, we can now write our example like this:
\begin{program}
(iter (for el in list)
      (generate i upfrom 1)
      (if el (collect (cons el (next i)))))
\end{program}
Now |i| is updated only when |(next i)| is executed, and this occurs
only when |el| is \nonnil.

To better understand the relationship between ordinary drivers and
generators, observe that we can rewrite an ordinary driver using its
generator form immediately followed by |next|, as this example shows:
\begin{program}
(iter (generating i from 1 to 10)
      (next i)
      ...)
\end{program}
Provided that the loop body contains no |(next i)| forms, this will
behave just as if we had written |(for i from 1 to 10)|.

We can still refer to a driver variable ~v~ without using |next|; in
this case, its value is that given to it by the last evaluation of
|(next ~v~)|.  Before |(next ~v~)| has been called the first time, the
value of ~v~ is undefined.  

This semantics is more flexible than
one in which ~v~ begins the loop bound to its first value and calls
of |next| supply subsequent values, because it means the loop will not
terminate too soon if the generator's sequence is empty.  For
instance, consider the following code, which tags \nonnil\ elements of
a list using a list of tags, and also counts the null elements.
(We assume there are at least as many tags as \nonnil\ elements.)
\begin{program}
(let* ((counter 0)
       (tagged-list (iter (for el in list)
                          (generating tag in tag-list) 
                          (if (null el)
                              (incf counter)
                              (collect (cons el (next tag)))))))
  ...)
\end{program}
It may be that there are just as many tags as non-null elements of
|list|.  If all the elements of |list| are null, we still want the
counting to proceed, even though |tag-list| is \nil.  If |tag| had to be
assigned its first value before the loop begins, we would have had to
terminate the loop before the first iteration, since when |tag-list|
is \nil, |tag| 
has no first value.  With the existing semantics, however, |(next
tag)| will never execute, so the iteration will cover all the elements of
|list|. 
        
When the ``variable'' of a driver clause is actually a destructuring
template containing several variables, all the variables are eligible
for use with |next|.  As before, |(next ~v~)| evaluates to ~v's~ next
value; but the effect is to update all of the template's variables.
For instance, the following code will return the list |(a 2 c)|.
\begin{program}
(iter (generating (key . item) in '((a . 1) (b . 2) (c . 3)))
      (collect (next key))
      (collect (next item)))
\end{program}

Only driver clauses with variables can be made into generators.  This
includes all clauses mentioned so far except for |repeat|.  It does
~not~ include |for\dots previous|, |for\dots =|,
|for\dots initially\dots then| or |for\dots first\dots then| (see
below). 

\subsubsection{Previous Values of Driver Variables}

Often one would like to access the value of a variable on a previous
iteration.  \iter\ provides a special clause for accomplishing this.

\begin{clauses}

\clause{|for| ~pvar~ |previous| ~var~ \opt\ |initially| ~init~
|back| ~n~}
\clindexx{for}{previous}
Sets ~pvar~ to the previous value of ~var,~ which should be a driver
variable, a variable from another |for\dots previous| clause, or a
variable established by a |for\dots =|, 
|for\dots initially\dots then| or |for\dots first\dots then| clause
(see section \ref{setting}).
Initially, ~pvar~ is given the value ~init~ (which defaults to \nil).  
The ~init~ expression will be moved outside the loop body, so it
should not depend on anything computed within the loop.
~pvar~ retains the value of ~init~ until ~var~ is set to its second
value, at which point ~pvar~ is set to ~var's~ first value; and so on.  

\cpar The
argument ~n~ to |back| must be a constant, positive integer, and
defaults to 1.  It determines how many iterations back ~pvar~ should
track ~var~.  For example, when ~n~ is 2, then ~pvar~ will be assigned
~var's~ first value when ~var~ is set to its third value.

\cpar A |for\dots previous| clause may occur after or before its
associated driver clause. |for\dots previous| works with generators as
well as ordinary drivers.

\pagebreak[3]

\cpar Example:
\begin{program}
(iter (for el in '(1 2 3 4))
      (for p-el previous el)
      (for pp-el previous p-el initially 0)
      (collect pp-el))
\end{program}
This evaluates to |(0 0 1 2)|.  It could have been written more
economically as 
\begin{program}
(iter (for el in '(1 2 3 4))
      (for pp-el previous el back 2 initially 0)
      (collect pp-el))
\end{program}


\end{clauses}

\subsection{Variable Binding and Setting}
\label{setting}

Several clauses exist for establishing new variable bindings or for
setting variables in the loop.  They all support destructuring.

\begin{clauses}

\clausex{with}{~var~ \opt\ |=| ~value~}
Causes ~var~ to be bound to value before the loop body is entered.
If ~value~ is not supplied, ~var~ assumes a default
binding, which will be 
\nil\ in the absence of declarations.  Also, if ~value~ is not
supplied, no destructuring is performed; instead, ~var~ may be a list of
symbols, all of which are given default 
bindings.  If ~value~ is supplied, ~var~ is bound to it, with
destructuring.  

\cpar Because |with|
creates bindings whose scope includes the entire \iter\ form, it is
good style to put all |with| clauses at the beginning.

\cpar Successive occurrences of |with| result in sequential
bindings (as with 
|let*|).  There is no way to obtain parallel bindings; see
section \ref{bindings} for a rationale.  


\clause{|for| ~var~ |=| ~expr~}
\clindexx{for}{=}
On each iteration, ~expr~ is evaluated and ~var~ is set
to its value. 

\cpar This clause may appear to do the same thing as |for\dots next|.
In fact, they are quite different.  |for\dots =| provides only three
services: it sets up a binding for ~var~, sets it to ~expr~ on each
iteration, and makes it possible to use |for\dots previous| with
~var~.  |for\dots next| provides these services in addition to the
ability to turn the driver into a generator.
%Also, the code which sets ~var~ appears in the loop body in the same
%place as the |for\dots =| clause; the code for |for\dots next| appears
%at the top of the loop, as with other drivers (except when being used
%as a generator).

\clause{|for| ~var~ |initially| ~init-expr~ |then| ~then-expr~}
\clindexxx{for}{initially}{then}
Before the loop begins, ~var~ is set to ~init-expr;~ on all
iterations after the first it is set to ~then-expr.~  This clause
must occur at top-level.  ~init-expr~ will be moved outside the loop
body and ~then-expr~ will be moved to the end of the loop body, so
they are subject to code motion problems (see section
\ref{code-movement}). 

\cpar This clause may appear to be similar to |for\dots next|, but in
fact they differ significantly.  |for\dots initially\dots then| is
typically used to give ~var~ its first value before the loop begins,
and subsequent values on following iterations.  This is incompatible
with generators, whose first value and subsequent values must all be
computed by |(next ~var~)|.  Also, the update of ~var~ in 
|for\dots initially\dots then| does not occur at the location of the
clause. 
Use |for\dots initially\dots then| for
one-shot computations where its idiom is more convenient, but use
|for\dots next| for extending \iter\ with new drivers (see section
\ref{extend}).

\clause{|for| ~var~ |first| ~first-expr~ |then| ~then-expr~}
\clindexxx{for}{first}{then}
The first time through the loop, ~var~ is set to ~first-expr;~ on
subsequent iterations, it is set to ~then-expr.~  This differs from
|for\dots initially| in that ~var~ is set to ~first-expr~
inside the loop body, 
so ~first-expr~ may depend on the results of other clauses.  For
instance, 
\begin{program}
(iter (for num in list)
      (for i first num then (1+ i))
      ...)
\end{program}
will set |i| to the first element of |list| on the first
iteration, whereas 
\begin{program}
(iter (for num in list)
      (for i initially num then (1+ i))
      ...)
\end{program}
is probably erroneous; |i| will be bound to |num|'s
default binding (usually \nil) for the first iteration.

\end{clauses}

\begin{note}{Compatibility note}
|loop|'s |for\dots =| works like \iter's, but |loop| used the syntax
|for\dots =\dots then| to mean |for\dots initially\dots then|.  It was
felt that these two operations were sufficiently different to warrant
different keywords.

Also, the |for| in the above three clauses is misleading,
since none is true driver (e.g. none has a corresponding |generate|
form).  |setting| would have been a better choice, but |for| was used
to retain some compatibility with |loop|.  
\end{note}


%%%%%%%%%%%%%%%%%%%%%%%%%%%%%%%%
\subsection{Gathering Clauses}
Many of \iter's clauses accumulate values into a variable, or set a
variable under certain conditions.  At the end of the loop, this
variable contains the desired result.  All these clauses have an
optional |into| keyword, whose argument should be a symbol.  If the
|into| keyword is not supplied, the accumulation variable will be
internally generated and its value will be returned at the end of the
loop; if a variable is specified, that variable is used for the
accumulation, and is not returned as a result---it is up to the user
to return it explicitly, in the loop's epilogue code (see |finally|).
It is safe to examine the accumulation variable during the loop, but
it should not be modified.

These clauses all begin with a verb.  When the verb does
not conflict with an existing Common Lisp function, then it may be
used in either its infinitival or present-participle form (e.g. |sum|,
|summing|).  However, when there is a conflict with Common Lisp, only
the present-participle form may be used (e.g. |unioning|).  This is to
prevent \iter\ clauses from clashing with Common Lisp
functions. 

%although these clauses are described as ``producing a value,'' it is a
%mistake to think of the lisp list representing the clause as a
%value-producing form in the usual way.  clauses may legally be written
%where a value is expected, e.g. |(setq x (sum i))|, but the
%lisp value of a clause in such a context is undefined.

\subsubsection{Reductions}

~Reduction~ is an extremely common iteration pattern in which
the results of successive applications of a
binary operation are accumulated.
For example, a loop that computes the sum of the
elements of a list is performing a reduction with the addition
operation.  This could be written in Common Lisp as  |(reduce \#'+
list)| or with \iter\ as
\begin{program}
(iter (for el in list)
      (sum el))
\end{program}

\begin{clauses}

\clausex{sum}{~expr~ \opt\ |into| ~var~}
Each time through the loop, ~expr~ is evaluated and added to a
variable, which is bound initially to zero.  If ~expr~ has a type,
it is ~not~ used as the type of the sum variable, which is always
|number|.  To get the result variable to be of a more specific
type, use an explicit variable, as in
\begin{program}
(iter (for el in number-list)
      (sum el into x)
      (declare (fixnum x))
      (finally (return x)))
\end{program}

\clausex{multiply}{~expr~ \opt\ |into| ~var~}
Like |sum|, but the initial value of the result variable is $1$,
and the variable is updated by multiplying ~expr~ into it.

\clausex{counting}{~expr~ \opt\ |into| ~var~}
~expr~ is evaluated on each iteration.  If it is \nonnil, the
accumulation variable, initially zero, is incremented.

\startitem
\Clausex{maximize}{~expr~ \opt\ |into| ~var~}
\Clausex{minimize}{~expr~ \opt\ |into| ~var~}
\finishitem
~expr~ is evaluated on each iteration and its extremum (maximum or
minimum) is stored in the accumulation variable.  If ~expr~ is never
evaluated, then the result is \nil\ (if the accumulation variable is
untyped) or $0$ (if it has a numeric type).

\clausex{reducing}{~expr~ |by| ~func~ \opt\ |initial-value| ~init-val~ |into| ~var~}
This is a general way to perform reductions. ~func~ should be a
function of two arguments, the first of which will be the value
computed so far and
the second of which will be the value of ~expr~.  It should return the
new value.
|reducing| is roughly equivalent to the
Common Lisp |(reduce ~func~ ~sequence~ :key ~expr-function~)|, where
~expr-function~ is used to derive values from the successive elements
of ~sequence~.

\cpar If the |reducing| clause is never executed, the result is
undefined.

\cpar It is not necessary to provide an
initial value, but better code can be generated if one is supplied.
Regardless of its location in the \iter\ body, ~init-val~ will be
evaluated before the loop is entered, so it should not depend on any
value computed inside the \iter\ form.  

%\cpar if a ~var~ is not specified, you can get \iter\ to declare the
%type of the internal variable by putting a |the| expression around
%~func.~  see section \ref{types}.

\end{clauses}

%\begin{note}{implementation note}
%in principle, |maximize| and |minimize| can be thought of
%as reductions where the initial value is the smallest (or largest)
%value that the accumulation variable can assume.  because lisp's
%bignums can represent arbitrary integers, these clauses cannot be
%implemented as reductions in general.  if, however, the type of 
%~expr~ or ~var~ can be determined to be a fixnum
%or a float, \iter\ will implement the clause as a true reduction,
%using one of the constants |most-negative-fixnum|, |%most-positive-fixnum|, 
%|most-negative-short-float|, etc. as appropriate.
%\end{note}

\subsubsection{Accumulations}

All the predefined accumulation clauses add values to a sequence.  If
the sequence is a list, they all 
have the property that the partial list is kept in the correct order
and available for inspection at any point in the loop.

\begin{clauses}

\clausex{collect}{~expr~ \opt\ |into| ~var~ |at| ~place~ |result-type| ~type~}
Produces a sequence of the values of ~expr~ on each iteration. ~place~
indicates where the next value of ~expr~ is added to the list and may
be one of the symbols |start|, |beginning| (a synonym for |start|) or
|end|.  The symbol may be quoted, but need not be.  The default is
|end|.  For example,
\begin{program}
(iter (for i from 1 to 5)
      (collect i))
\end{program}
produces |(1 2 3 4 5)|, whereas
\begin{program}
(iter (for i from 1 to 5)
      (collect i at beginning))
\end{program}
produces |(5 4 3 2 1)| (and is likely to be faster in most Common Lisp
implementations).

\cpar If ~type~ is provided, it should be a subtype of |sequence|.
The default is |list|.  Specifying a type other than |list| will
result in |collect| returning a sequence of that type.  ~However,~ the
type of the sequence being constructed when inside the loop body is
undefined when a non-|list| type is specified.  (As with ~place~,
quoting ~type~ is optional.)

\clausex{adjoining}{~expr~ \opt\ |into| ~var~ |test| ~test~ |at| ~place~ |result-type| ~type~}
Like |collect|, but only adds the value of ~expr~ if it is not
already present.  ~test,~ which defaults to |\#'eql|, is
the test to be used with |member|.

\startitem
\Clausex{appending}{~expr~ \opt\ |into| ~var~ |at| ~place~}
\Clausex{nconcing}{\ ~expr~ \opt\ |into| ~var~ |at| ~place~}
\Clausex{unioning}{\ ~expr~ \opt\ |into| ~var~ |test| ~test~ |at| ~place~}
\Clausex{nunioning}{~expr~ \opt\ |into| ~var~ |test| ~test~ |at| ~place~}
\finishitem
These are like |collect|, but behave like the Common Lisp functions
|append|, |nconc|, |union| or |nunion|.  
As in Common Lisp, they work only on lists.  Also as in Common Lisp,
|unioning| and |nunioning| assume that the value of ~expr~ contains no
duplicates. 

\clausex{accumulate}{~expr~ |by| ~func~ \opt\ |initial-value| ~init-val~ |into| ~var~}
This is a general-purpose accumulation clause. ~func~ should be a
function of two arguments, the value of ~expr~ and the value
accumulated so far in the iteration, and it should return the updated
value.  If no initial value is supplied, \nil\ is used.  

%\cpar If a ~var~ is not specified, you can get \iter\ to declare the
%type of the internal variable by putting a |the| expression around
%~func.~  see section \ref{types}.

\cpar The differences between |accumulate| and |reducing| are slight.
One difference is that the functions take their arguments in a
different order.  Another is that in the absence of ~init-val~,
|accumulate| will use \nil, whereas |reducing| will generate different
code that avoids any dependence on the initial value.
The reason for having both clauses is that one usually
thinks of reductions (like |sum|) and accumulations (like |collect|)
as different beasts.

\end{clauses}


\subsubsection{Finders}

A ~finder~ is a clause whose value is an expression that meets some
condition.

\begin{clauses}

\clause{|finding| ~expr~ |such-that| ~test~ \opt\ |into| ~var~ |on-failure| ~failure-value~}
\clindexx{finding}{such-that}
If ~test~ (which is an expression) ever evaluates to \nonnil, the loop
is terminated, the 
epilogue code is run and the value of ~expr~ is returned.  Otherwise,
\nil\ (or ~failure-value,~ if provided) is returned.  If ~var~ is
provided, it will have either the \nonnil\ value of ~expr~ or
~failure-value~ when the epilogue code is run.

\cpar As a special case, if the ~test~ expression is a sharp-quoted
function, then it is applied to ~expr~ instead of being simply
evaluated.  E.g. |(finding x such-that \#'evenp)| is equivalent to
|(finding x such-that (evenp x))|.  

%\cpar although ~test~ need have nothing to do with ~%expr~ as in
%|(finding j such-that (> i 3))|, it usually 
%will: |(finding (length el) such-that (oddp (length el)))|.  to
%avoid performing the |length| computation twice, you could write
%|(finding (length el) such-that \#'oddp)| or |(finding (length
%el) such-that 'oddp)|; for these cases, \iter\ generates code that
%executes ~expr~ only once.  the code for |\#'oddp|
%is slightly different from that for {\lisp 'oddp}; see the discussion
%under {\lisp for\dots in} and {\lisp for\dots on}.

\cpar |On-failure| is a misnomer. Because it is always evaluated, it behaves
more like the default third argument to the |gethash| function. As a result,
|on-failure (error "Not found")| makes no sense. Instead, the clauses |leave|
or |thereis| can be used in conjunction with |finally| as follows:
\begin{program}
(iter (for x in '(1 2 3))
      (if (evenp x) (leave x))
      (finally (error "not found")))
\end{program}

\cpar This clause may appear multiple times when all defaults are
identical. It can also be used together with either |always|/|never| or
|thereis| if their defaults match. More specifically, |on-failure nil| is
compatible with |thereis|, while |on-failure t| is compatible with |always|
and |never| clauses.
\begin{program}
(iter (for i in '(7 -4 2 -3))
      (if (plusp i)
	  (finding    i  such-that (evenp i))
          (finding (- i) such-that (oddp i))))
\end{program}

\startitem
\Clause{|finding| ~expr~ |maximizing| ~m-expr~ \opt\ |into| ~var~}
\clindexx{finding}{maximizing}
\Clause{|finding| ~expr~ |minimizing| ~m-expr~ \opt\ |into| ~var~}
\clindexx{finding}{minimizing}
\finishitem
Computes the extremum (maximum or minimum) value of ~m-expr~ over all
iterations, and returns the value of ~expr~ corresponding to the
extremum.  ~expr~ is evaluated inside the loop at the time the new
extremum is established.  If ~m-expr~ is never evaluated (due to, for
example, being embedded in a conditional clause), then the returned
value depends on the type, if any, of ~expr~ (or ~var,~ if
one is supplied).  If there is no type, the returned
value will be \nil; if the type is numeric, the returned value will be
zero.

\cpar For these two clauses, ~var~ may be a list of two
symbols; in that case, the first is used to record ~expr~ and
the second, ~m-expr.~  

\cpar As with |finding\dots such-that|, if ~m-expr~ is a sharp-quoted
function, then it is called on ~expr~ instead of being evaluated.

\end{clauses}

\subsubsection{Boolean Tests}

\begin{clauses}

\clausex{first-iteration-p}{}
Returns |t| in the first cycle of the loop, otherwise \nil.

\clausex{first-time-p}{}
Returns |t| the first time the expression is evaluated, and then \nil\ forever.
This clause comes handy when printing (optional) elements separated
by a comma:

\begin{program}
(iter (for el in '(nil 1 2 nil 3))
      (when el
        (unless (first-time-p)
          (princ ", "))
        (princ el)))
\end{program}
produces |"1, 2, 3"|.

\end{clauses}

\subsubsection{Aggregated Boolean Tests}

\begin{clauses}

\clausex{always}{~expr~}
If ~expr~ ever evaluates to
\nil, then \nil\ is immediately returned; the epilogue code is not
executed.  If ~expr~ never evaluates to \nil, the epilogue code
is executed and the last value of ~expr~ (or |t| if ~expr~ was never
evaluated) is returned (whereas |loop| would constantly return |t|).

% mention last evaluated clause when multiple always clauses?

\clausex{never}{~expr~}
Like |(always (not ~expr~))|, except it does not influence the last
value returned by a possible other |always| clause. That is,
\begin{program}
(iter (repeat 2)
      (always 2)
      (never nil)) \yields 2 ; not t
\end{program}

\clausex{thereis}{~expr~}
If ~expr~ is ever \nonnil,
its value is immediately returned without running epilogue code.
Otherwise, the epilogue code is performed and \nil\ is returned.

This clause cannot be used together with |always| or |never|, because their
defaults are opposed (similarly, |(loop always 3 thereis nil)| refuses to
compile in some implementations of |loop|).

\end{clauses}


\subsection{Control Flow}
\label{control-flow}
Several clauses can be used to alter the usual flow of control in a loop.

Note: the clauses of this and subsequent sections don't adhere to \iter's
usual syntax, but instead use standard Common Lisp syntax.  Hence the
format for describing syntax subsequently is
like the standard format used in the Common Lisp manual, not like the
descriptions of clauses above.

\begin{clauses}

\clausex{finish}{}
Stops the loop and runs the epilogue code.

%for example:
%
%\begin{program}
%(iter (with answer = nil)
%         (initially (make-a-mess))
%         (for i from 1 to 10)
%         (when (correct? i)
%           (setq answer i)
%           (finish))
%         (finally (cleanup)))
%\end{program}
%
%this code will execute |cleanup| whether or not the test |(correct?
%i)| ever succeeds. 
%the (more elegant) formulation,
%\begin{program}
%(iter (initially (make-a-mess))
%         (for i from 1 to 10)
%         (finding i such-that (correct? i))
%         (finally (cleanup)))
%\end{program}
%would not execute |cleanup| if |(correct? i)| succeeded; it
%would do an immediate return.

\clausex{leave}{\opt\ ~value~}
Immediately returns ~value~ (default \nil) from the current \iter\
form, skipping the epilogue code.  Equivalent to using |return-from|.

\clausex{next-iteration}{}
Skips the remainder of the loop body and begins the next iteration of
the loop.

\clausex{while}{~expr~}
If ~expr~ ever evaluates to \nil, the loop is terminated and the
epilogue code executed.  Equivalent to |(if (not ~expr~) (finish))|.

\clausex{until}{~expr~}
Equivalent to |(if ~expr~ (finish))|.

\clausex{if-first-time}{~then~ \opt\ ~else~}
If this clause is being executed for the first time in this invocation
of the \iter\ form, then the ~then~ code is evaluated; otherwise the
~else~ code is evaluated.

\cpar |(for ~var~ first ~expr1~ then ~expr2~)| is almost equivalent to
\begin{program}
(if-first-time (dsetq ~var~ ~expr1~)
               (dsetq ~var~ ~expr2~))
\end{program}
The only difference is that the |for| version makes ~var~ available
for use with |for\dots previous|.

\end{clauses}

\subsection{Code Placement}
When fine control is desired over where code appears in a loop
generated by \iter, the following special clauses may be useful.
They are all subject to code-motion problems (see section
\ref{code-movement}).  

\begin{clauses}

\clausex{initially}{|\&rest| ~forms~}
The lisp ~forms~ are placed in the prologue section of the loop, where
they are executed once, before the loop body is entered.

\clausex{after-each}{|\&rest| ~forms~}
The ~forms~ are placed at the end of the loop body, where they
are executed after each iteration.  Unlike the other clauses in this
section, ~forms~ may contain \iter\ clauses.

\clausex{else}{|\&rest| ~forms~}
The lisp ~forms~ are placed in the epilogue section of the loop, where they
are executed if this |else| clause is never met during execution of the
loop and the loop terminates normally.

\clausex{finally}{|\&rest| ~forms~}
The lisp ~forms~ are placed in the epilogue section of the loop, where
they are executed after the loop has terminated normally.

\clausex{finally-protected}{|\&rest| ~forms~}
The lisp ~forms~ are placed in the second form of an unwind-protect
outside the loop.  They are always executed after the loop has
terminated, regardless of how the termination occurred.

\end{clauses}


%%%%%%%%%%%%%%%%%%%%%%%%%%%%%%%%%%%%%%%%%%%%%%%%%%%%%%%%%%%%%%%%
\section{Other Features}

\subsection{Multiple Accumulations}
\label{multiple}

\begin{sloppypar}
It is permitted to have more than one clause accumulate into the same
variable, as in the following:
\begin{program}
(iter (for i from 1 to 10)
      (collect i into nums)
      (collect (sqrt i) into nums)
      (finally (return nums)))
\end{program}
Clauses can only accumulate into the same variable if they are
compatible.  |collect|, |adjoining|, |appending|, |nconcing|,
|unioning| and |nunioning| are compatible with each other; |sum|,
|multiply| and |counting| are compatible; |always| and |never| are compatible;
|finding| \dots |such-that| is compatible with either |thereis| or |always|
and |never| when their defaults match; and |maximize| and |minimize| clauses
are compatible only with other |maximize| and |minimize| clauses,
respectively.

%note that the same variable ~cannot~ be both an accumulation
%variable and an ordinary variable; there can be only one variable with
%a given name within an \iter\ form. 
\end{sloppypar}

\subsection{Named Blocks}

Like Common Lisp |block|s, \iter\ forms can be given names.  The
name should be a single symbol, and it must be the first form in the
\iter.  The generated code behaves exactly like a named block; in
particular, |(return-from ~name~)| can be used to exit it:
\begin{program}
(iter fred
      (for i from 1 to 10)
      (iter barney
            (for j from i to 10)
            (if (> (* i j) 17)
                (return-from fred j))))
\end{program}
An \iter\ form that is not given a name is implicitly named \nil.

Sometimes one would like to write an expression in an inner \iter\ form,
but have it processed by an outer \iter\ form.  This is possible with
the |in| clause.  

\begin{clauses}

\clausex{in}{~name~ |\&rest| ~forms~}
Evaluates ~forms~ as if they were part of the \iter\ form named
~name~.  In other words, \iter\ clauses are processed by the \iter\
form named ~name,~ and not by any \iter\ forms that occur inside ~name.~

\cpar As an example, consider the problem of collecting a list of the
elements in a two-dimensional array.  The naive solution,
\begin{program}
(iter (for i below (array-dimension ar 0))
      (iter (for j below (array-dimension ar 1))
            (collect (aref ar i j))))
\end{program}
\noindent is wrong because the list created by the inner \iter\ is simply
ignored by the outer one.  But using |in| we can write:
\begin{program}
(iter outer (for i below (array-dimension ar 0))
      (iter (for j below (array-dimension ar 1))
            (in outer (collect (aref ar i j)))))
\end{program}
\noindent which has the desired result.

\end{clauses}

\subsection{Destructuring}
\label{destructuring}

In many places within \iter\ clauses where a variable is expected, a
list can be written instead.  In these cases, the value to be assigned
is ~destructured~ according to the pattern
described by the list.  As a simple example, the clause
\begin{program}
(for (key . item) in alist)
\end{program}
\noindent will result in |key| being set to the |car| of
each element in |alist|, and |item| being set to the |cdr|.  The
pattern list may be nested to arbitrary depth, and (as the example
shows) need not be terminated with \nil; the only requirement is that
each leaf be a bindable symbol (or \nil, in which case no binding is
generated for that piece of the structure).

Sometimes, you might like to do the equivalent of a
|multiple-value-setq| in a clause.  This 
``multiple-value destructuring'' can be expressed by writing 
\linebreak |(values $pat_1$ $pat_2 \ldots$)| for a destructuring
pattern, as in 
\begin{program}
(for (values (a . b) c d) = (three-valued-function ...))
\end{program}
\begin{sloppypar}
Note that the $pat_i$ can themselves be destructuring patterns (though
not multiple-value destructuring patterns).  You can't do multiple-value
destructuring in a |with| clause; instead wrap the whole \iter\
form in a |multiple-value-bind|.
\end{sloppypar}

\begin{note}{Rationale}
There are subtle interactions between variable declarations and
evaluation order that make the correct implementation of
multiple-value destructuring in a |with| somewhat tricky.
\end{note}

The destructuring feature of \iter\ is available as a separate
mechanism, using the |dsetq| macro:

\begin{clauses}
\defmacro{dsetq}{template expr}
Performs destructuring of ~expr~ using ~template~.   May be used
outside of an \iter\ form. Yields the primary value of ~expr~.

\end{clauses}

\subsection{On-line Help}

\begin{sloppypar}
There is a limited facility for on-line help, in the form of the 
|display-iterate-clauses| function.
\end{sloppypar}

\begin{clauses}

\defun{display-iterate-clauses}{\opt\ clause-spec}
Displays a list of \iter\ clauses.  If ~clause-spec~ is not
provided, all clauses are shown; if it is a symbol, all clauses
beginning with that symbol are shown; and if it is a list of symbols,
all clauses for which ~clause-spec~ is a prefix are shown.

\end{clauses}

\subsection{Parallel Binding and Stepping}
\label{bindings}

The parallel binding and stepping of variables is a feature that
\iter\ does ~not~ have.  This section attempts to provide a rationale.

We say that two variables are bound ~in parallel~ if neither
binding shadows the other.  This is the usual semantics of |let|
(as opposed to |let*|).  Similarly, we can say that iteration
variables are stepped in parallel if neither variable is updated
before the other, conceptually speaking; in other words, if the code
to update each variable can reference the old values of both variables.

|loop| allows parallel binding of variables and parallel stepping of
driver variables.  My view is that if you are depending on the 
serial/parallel distinction, you are doing something obscure.  If you
need to bind 
variables in parallel using |with|, then you must be using a
variable name that shadows a name in the existing lexical environment.
Don't do that.  The most common use for parallel stepping is to track
the values of variables on the previous iteration, but in fact this
does not require parallel stepping at all; the following will work:
\begin{program}
(iter (for current in list)
      (for prev previous current)
      ...)
\end{program}


%%%%%%%%%%%%%%%%%%%%%%%%%%%%%%%%%%%%%%%%%%%%%%%%%%%%%%%%%%%%%%%%
\section{Types and Declarations}
\label{types}

\subsection{Discussion}

Sometimes efficiency dictates that the types of variables be declared.
This type information needs to be communicated to \iter\ so it can
bind variables to appropriate values.  Furthermore, \iter\ must often
generate internal variables invisible to the user; there needs to be a
way for these to be declared.

As an example, consider this code, which will return the number of
odd elements in |number-list|:
\begin{program}
(iter (for el in number-list)
      (count (oddp el)))
\end{program}
In processing this form,
\iter\ will create an internal variable, let us call it |list17|, to
hold the successive |cdr|s of |number-list|, and  
will bind the variable to |number-list|.  It will also generate a
default binding for |el|; only inside the body of the loop will |el|
be set to the |car| of |list17|.  Finally, \iter\ will generate a
variable, call it |result|, to hold the result of the count, and will
bind it to zero.

When dealing with type declarations, \iter\ observes one simple rule:
~it will never generate a declaration unless requested to do so.~  The
reason is that such declarations might mask errors in compiled code by
avoiding error-checks; the resulting problems would be doubly hard to
track down because the declarations would be hidden from the
programmer.  Of course, a compiler might omit error-checks even in the
absence of declarations, though this behavior can usually be avoided,
e.g. by saying |(declaim (optimize (safety 3)))|.

So, the above \iter\ form will generate code with no declarations.
But say we wish to declare the types of |el| and the internal
variables |list17| and |result|.  How is this done?

Declaring the type of |el| is easy, since the programmer knows
the variable's name:
\begin{program}
(iter (for el in number-list)
      (declare (fixnum el))
      (counting (oddp el)))
\end{program}
\iter\ can read variable type declarations like this one.  Before
processing any clauses, it scans the entire top-level form for type
declarations 
and records the types, so that variable bindings can be performed
correctly.  In this case, |el| will be bound to zero
instead of \nil.  Also, \iter\ collects all the top-level declarations
and puts them at the begining of the generated code, so it is not
necessary to place all declarations at the beginning of an \iter\
form; instead, they can be written near the variables whose types they
declare.  

Since \iter\ is not part of the compiler, it will not know
about declarations that occur outside an \iter\ form; these
declarations must be repeated inside the form.

Here is another way we could have declared the type of |el|:
\begin{program}
(iter (for (the fixnum el) in number-list)
      (counting (oddp el)))
\end{program}
\iter\ extends the Common Lisp |the|\lispindex{the}
form to apply to variables as well as value-producing forms; anywhere
a variable is allowed---in a |with| clause, as the iteration
variable in a driver clause, as the |into| argument of an
accumulation clause, even inside a destructuring template---you can
write |(the ~type~ ~symbol~)| instead.

There is one crucial difference between using a |the| form and
actually declaring the variable: explicit declarations are always
placed in the generated code, but type information from a |the|
form is not turned into an actual declaration unless you tell \iter\
to do so using |iterate:declare-variables|.  See below.

\begin{sloppypar}
Declaring the types of internal variables is harder than declaring the
types of explicitly mentioned variables, since their names
are unknown.  You do it by declaring |iterate:declare-variables|
somewhere inside the top level of the \iter\ form.  (This will also
generate declarations for variables declared using |the|.)
\iter\ does not provide much selectivity here: it's all or none. 
And unfortunately, since \iter\ is not privy to compiler information
but instead reads declarations itself, it will not hear if you
|(declaim (iterate:declare-variables))|.  Instead, set the variable 
|iterate::*always-declare-variables*| to |t| at
compile-time, using |eval-when|.
\end{sloppypar}

To determine the appropriate types for internal variables, \iter\ uses
three sources of information:

\begin{itemize}

\item Often, the particular clause dictates a certain type for a
variable; \iter\ will use this information when available.  In the
current example, the variable |list17| will be given the type
|list|, since that is the only type that makes sense; and the
variable |result| will be given the type |fixnum|, on the
assumption that you will not be counting high enough to need bignums.
You can override this assumption only by using and explicitly declaring a
variable: 
\begin{verbatim}
(iter (declare (iterate:declare-variables))
      (for el in number-list)
      (count (oddp el) into my-result)
      (declare (integer my-result))
      (finally (return my-result)))
\end{verbatim}

\begin{sloppypar}
Other examples of the type assumptions that \iter\ makes are: type
|list| for |into| variables of collection clauses; type |list| for
expressions that are to be destructured; type |vector| for the
variable holding the vector in a |for\dots in-vector| clause, and
similarly for |string| and the |for\dots in-string| clause;
and the implementation-dependent type
|(type-of array-dimension-limit)| for the index and limit
variables generated by sequence iteration drivers like |for\dots
in-vector| and |for\dots in-string| (but not |for\dots in-sequence|,
because it may be used to iterate over a list). 
\end{sloppypar}

\item Sometimes, \iter\ will examine expressions and try to determine
their types in a simple-minded way.  If the expression is
self-evaluating (like a number, for instance), \iter\ knows that the
expression's type is the same as the type of the value it denotes, so
it can use that type.  If the expression is of the form |(the ~type~
~expr~)|, \iter\ is smart enough to extract ~type~ and use it.
However, the current version of \iter\ does not  
examine declarations of function result types or do any type
inference.  It will not determine, for 
example, that the type of |(+ 3 4)| is |fixnum|, or even
|number|.

\item In some cases, the type of an internal variable should match the
type of some other variable.  For instance, \iter\ generates an
internal variable for |(f x)| in the
clause |(for i from 1 to (f x))|, and in the absence of other
information will give it the same type as |i|.  If, however, the
expression had been written |(the fixnum (f x))|, then \iter\
would have given the internal variable the type |fixnum|
regardless of |i|'s type.  The type incompatibility errors that
could arise in this situation are not checked for.

\end{itemize}

Note that if you do declare |iterate:declare-variables|, then
\iter\ may declare user variables as well as internal ones if they do
not already have declarations, though only for variables that it
binds.  For instance, in this code:

\begin{program}
(iter (declare (iterate:declare-variables))
      (for i from 1 to 10)
      (collect i into var))
\end{program}
the variable |var| will be declared to be of type |list|.


\subsection{Summary}

\iter\ understands standard Common Lisp variable type declarations
that occur within an \iter\ form and
will pass them through to the generated code.  If the declaration
|(iterate:declare-variables)|\lispindex{declare-variables}
appears at the top level of an 
\iter\ form, or if 
|iterate::*always-declare-variables*|\lispindex{*always-declare-variables*} 
is \nonnil, then \iter\ will use the type information gleaned from user
declarations, self-evaluating expressions and |the| expressions,
combined with reasonable assumptions, to determine variable
types and declare them.


%%%%%%%%%%%%%%%%%%%%%%%%%%%%%%%%%%%%%%%%%%%%%%%%%%%%%%%%%%%%%%%%
\section{Problems with Code Movement}
\label{code-movement}

Some \iter\ clauses, or parts of clauses, result in code being
moved from the location of the clause to other parts of the loop.
Drivers behave this way, as do code-placement clauses like |initially|
and |finally|.  When using these clauses, there is a danger of writing
an expression that makes sense in its apparent location but will be
invalid or have a different meaning in another location.  For example:
\begin{program}
(iter (for i from 1 to 10)
      (let ((x 3))
        (initially (setq x 4))))
\end{program}
While it may appear that the |x| of |(initially (setq x 4))| is the
same as the |x| of |(let ((x 3)) \dots|, in fact they are not:
|initially| moves its code outside the loop body, so |x| would refer
to a global variable.  Here is another example of the same problem:
\begin{program}
(iter (for i from 1 to 10)
      (let ((x 3))
        (collect i into x)))
\end{program}
If this code were executed, |collect| would create a binding for its
|x| at the top level of the \iter\ form that the |let| will shadow.

Happily, \iter\ is smart enough to catch these errors; it
walks all problematical code to ensure that free variables are not
bound inside the loop body, and checks all variables it binds for the
same problem.

However, some errors cannot be caught:

\begin{program}
(iter (with x = 3)
      (for el in list)
      (setq x 1)
      (reducing el by \#'+ initial-value x))
\end{program}
|reducing| moves its |initial-value| argument to the initialization
part of the loop in order to produce more efficient code.  Since
\iter\ does not perform data-flow analysis, it cannot determine that
|x| is changed inside the loop; all it can establish is that |x| is
not bound internally.  Hence this code will not signal an
error and will use $3$ as the initial value of the reduction.

The following list summarizes all cases that are subject to these code
motion and variable-shadowing problems.
\begin{itemize}
\item Any variable for which \iter\ creates a binding, including those
used in |with| and the |into| keyword of many clauses.

\begin{sloppypar}
\item The special clauses which place code: |initially|, |after-each|, |else|,
|finally| and |finally-protected|. 
\end{sloppypar}

\item The variables of a |next| or |do-next| form.

\item The |initially| arguments of |for\dots initially\dots then| and
|for\dots previous|. 

\item The |then| argument of |for\dots initially\dots then|.

\item The |initial-value| arguments of |reducing| and |accumulate|.

\item The |on-failure| argument of |finding\dots such-that|.
\end{itemize}

%%%%%%%%%%%%%%%%%%%%%%%%%%%%%%%%%%%%%%%%%%%%%%%%%%%%%%%%%%%%%%%%
\section{Differences Between {\tt Iterate} and {\tt Loop}}
|loop| contains a great deal of complexity which \iter\ tries to
avoid.  Hence many esoteric features of |loop| don't exist in \iter.
Other features have been carried over, but in a cleaned-up form.  
And of course, many new features have been added; they are not
mentioned in this list.

\begin{itemize}

\item \iter's syntax is more Lisp-like than |loop|'s, having a higher
density of parens.

\item The current implementation of \iter, unlike the current version
of  |loop| (as documented in {\em Common Lisp, 2nd Ed.\/}), is
extensible (see section \ref{extend}).  

\item |loop| puts the updates of all driver variables at the top of
the loop; \iter\ leaves them where the driver clauses appear.

\item While for the most part \iter\ clauses that resemble |loop| clauses
behave similarly, there are some differences.  For instance, there is
no |for\dots =\dots then| in \iter; instead use 
|for\dots initially\dots then|.  

\item |loop| binds the variable |it| at certain times
to allow pseudo-English expressions like |when ~expr~ return it|. 
In \iter, you must bind ~expr~ to a variable yourself.  Note that
|when ~expr~ return it| is like |thereis ~expr~| except that the latter is an
accumulation clause and therefore competes with other accumulations
(remember section \ref{multiple} above).

% repeat different behaviour of |always| clause here?

\item |loop| has a special |return| clause, illustrated in the
previous item.  \iter\ doesn't need one,  since an ordinary Lisp
|return| has the same effect.

\item |loop| allows for parallel binding and stepping of iteration
variables.  \iter\ does not.  (See section \ref{bindings}.)

\item |loop| and \iter\ handle variable type declarations very
differently.  |loop| provides a special syntax for declaring variable
types, and does not examine declarations.  Moreover, the standard
implementation of |loop| will
generate declarations when none are requested.
\iter\ parses standard Common Lisp type declarations, and will never
declare a variable itself unless declarations are 
specifically requested.

\end{itemize}

%%%%%%%%%%%%%%%%%%%%%%%%%%%%%%%%%%%%%%%%%%%%%%%%%%%%%%%%%%%%%%%%
\section{Rolling Your Own}
\label{extend}

\subsection{Introduction}

\iter\ is extensible---you can write new clauses that embody new
iteration patterns.  You might want to write a new driver clause for a
data structure of your own, or you might want to write a clause that
collects or manipulates elements in a way not provided by \iter. 

This section describes how to write clauses for \iter.  Writing a
clause is like writing a macro.  In fact, writing a clause ~is~
writing a macro: since \iter\ code-walks its body and macroexpands,
you can add new abstractions to \iter\ with good old |defmacro|.

Actually, there are two extensions you can make to \iter\ that are
even easier than writing a macro.  They are adding a synonym for an
existing clause and defining a driver clause for an indexable
sequence.  These can be done with |defsynonym| and
|defclause-sequence|, respectively.  See section \ref{aids}, below.

The rest of this section explains how to write macros that expand into
\iter\ clauses.
Here's how you could add a simplified version of \iter's |multiply|
clause, if \iter\ didn't already have one:
\begin{program}
(defmacro multiply (expr)
  `(reducing ,expr by \#'* initial-value 1))
\end{program}

If you found yourself summing the square of an expression often, you
might want to write a macro for that.  A first cut might be
\begin{program}
(defmacro sum-of-squares (expr)
  `(sum (* ,expr ,expr)))
\end{program}
but if you are an experienced macro writer, you will realize that this
code will evaluate ~expr~ twice, which is probably a bad idea.  A
better version would use a temporary:
\begin{program}
(defmacro sum-of-squares (expr)
  (let ((temp (gensym)))
    `(let ((,temp ,expr))
       (sum (* ,temp ,temp)))))
\end{program}
Although this may seem complex, it is just the sort of thing you'd
have to go through to write any macro, which illustrates the point of
this section: if you can write macros, you can extend \iter.

Our macros don't use \iter's keyword-argument
syntax.  We could just use keywords with |defmacro|, but we would
still not be using \iter's clause indexing mechanism.  Unlike Lisp,
which uses just the first symbol of a form to determine what function
to call, \iter\ individuates clauses by the list of required keywords.
For instance, |for\dots in| and |for\dots in-vector| are different clauses
implemented by distinct Lisp functions.  

To buy into this indexing scheme, as well as the keyword-argument
syntax, use |defmacro-clause|:

\begin{clauses}

\defmacro{defmacro-clause}{arglist |\&body| body}
Defines a new \iter\ clause.  ~arglist~ is a list of symbols which are
alternating keywords and arguments.  \opt\ may be used, and the list
may be terminated by |\&sequence|.  ~body~ is an ordinary macro body,
as with |defmacro|.  If the first form of ~body~ is a string, it is
considered a documentation string and will be shown by
|display-iterate-clauses|. |defmacro-clause| will signal an error if 
defining the clause would result in an ambiguity.  E.g. you cannot
define the clause |for\dots from| because there would be no way to
distinguish it from a use of the |for| clause with optional keyword |from|.

\end{clauses}

\medskip

Here is |multiply| using |defmacro-clause|.  The keywords are capitalized
for readability.
\begin{program}
(defmacro-clause (MULTIPLY expr \opt\ INTO var)
  `(reducing ,expr by \#'* into ,var initial-value 1))
\end{program}
You don't have to worry about the case when |var| is not supplied; for
any clause with an |into| keyword, saying |into nil| is equivalent to
omitting the |into| entirely.

As another, more extended example, consider the fairly common
iteration pattern that 
involves finding the sequence element that maximizes (or minimizes) some
function.  \iter\ provides this as |finding\dots maximizing|, but it's
instructive to see how to write it.
Here, in pseudocode, is how you might write such a loop for
maximizing a function F:

\begin{quote}
\begin{tabbing}
set variable MAX-VAL to NIL; \\
set variable WINNER to NIL; \\
for\=\ each element EL in the sequence \\
\>  if\=\ MAX-VAL\=\ is NIL or F(EL) $>$ MAX-VAL then \\
\>\>    set MAX-VAL to F(EL); \\
\>\>    set WINNER to EL; \\
\>  end if; \\
end for; \\
return WINNER.
\end{tabbing}
\end{quote}

Here is the macro:
\begin{program}
(defmacro-clause (FINDING expr MAXIMIZING func \opt\ INTO var)
  (let ((max-val (gensym))
        (temp1 (gensym))
        (temp2 (gensym))
        (winner (or var iterate::*result-var*)))
    `(progn 
       (with ,max-val = nil)
       (with ,winner = nil)
       (cond 
        ((null ,max-val)
         (setq ,winner ,expr)
         (setq ,max-val (funcall ,func ,winner))
        (t
         (let* ((,temp1 ,expr)
                (,temp2 (funcall ,func ,temp1)))
           (when (> ,temp2 ,max-val)
             (setq ,max-val ,temp2)
             (setq ,winner ,temp1))))))
       (finally (leave ,winner)))))
\end{program}
Note that if no |into| variable is supplied, we use
|iterate::*result-var*|, which contains the internal variable into
which all clauses place their results.  If this variable is bound by
some clause, then \iter\ will return its value automatically;
otherwise, \nil\ will be returned.  

\subsection{Writing Drivers}

In principle, drivers can be implemented just as easily as other
\iter\ clauses.  In practice, they are a little harder to get right. 
As an example, consider writing a driver that
iterates over all the 
elements of a vector, ignoring its fill-pointer.  |for\dots in-vector|
won't work for this, because it observes the fill-pointer.  It's necessary to
use |array-dimension| instead of |length| to obtain the size of the
vector.  Here is one approach:

\begin{program}
(defmacro-clause (FOR var IN-WHOLE-VECTOR v)
  "All the elements of a vector (disregards fill-pointer)"
  (let ((vect (gensym))
        (index (gensym)))
    `(progn
       (with ,vect = ,v)
       (for ,index from 0 below (array-dimension ,vect 0))
       (for ,var = (aref ,vect ,index)))))
\end{program}
Note that we immediately put |v| in a variable, in case it is an
expression.  Again, this is just good Lisp macrology.  It also has a
subtle effect on the semantics of the driver: |v| is evaluated only
once, at the beginning of the loop, so changes to |v| in the loop have
no effect on the driver.  Similarly, the bounds for numerical iteration
e.g. the above |array-dimension| are also evaluated once only.  This is how
all of \iter's drivers work.


There is an important point concerning the |progn| in this code.  We
need the |progn|, of course, because we are returning several forms,
one of which is a driver.  But \iter\ drivers must occur at top-level.
Is this code in error?  No, because ~top-level~ is defined in \iter\
to include forms inside a |progn|.  This is just the definition of
top-level that Common Lisp uses, and for the same reason: to allow
macros to return multiple forms at top-level.

While our |for\dots in-whole-vector| clause will work, it is not
ideal.  In particular, it does not support generating.  Do do so, we
need to use |for\dots next| or |for\dots do-next|.  The job is
simplified by the |defmacro-driver| macro.

\begin{clauses}

\defmacro{defmacro-driver}{arglist |\&body| body}
Defines a driver clause in
both the |for| and |generate| forms, and provides a parameter
|generate| which ~body~ can examine to determine how it was invoked.
~arglist~ is as in |defmacro-clause|, and should begin with the symbol
|for|. 

\end{clauses}

With |defmacro-driver|, our driver looks like this:
\begin{program}
(defmacro-driver (FOR var IN-WHOLE-VECTOR v)
  "All the elements of a vector (disregards fill-pointer)"
   (let ((vect (gensym))
         (end (gensym))
         (index (gensym))
         (kwd (if generate 'generate 'for)))
     `(progn
        (with ,vect = ,v)
        (with ,end = (array-dimension ,vect 0))
        (with ,index = -1)
        (,kwd ,var next (progn (incf ,index)
                               (if (>= ,index ,end) (terminate))
                               (aref ,vect ,index))))))
\end{program}


We are still missing one thing: the |\&sequence| keywords.  
We can get them easily enough, by writing 
\begin{program}
(defmacro-driver (FOR var IN-WHOLE-VECTOR v \&sequence) 
  ...)
\end{program}
We can now refer to parameters |from|, |to|, |by|, etc. which contain
either the values for the corresponding keyword, or \nil\ if the
keyword was not supplied.  Implementing the right code for these
keywords is cumbersome but not difficult; it is left as an exercise.
But before you begin, see |defclause-sequence| below for an easier way.
       
\subsection{Extensibility Aids}
\label{aids}

This section documents assorted features that may be of use in
extending \iter.

\begin{clauses}

\defunexpvar{*result-var*}
Holds the variable that is used to return a value as a result of the
\iter\ form.  You may examine this and use it in a |with| clause, but
you should not change it.

\defmacro{defsynonym}{syn word}
 Makes ~syn~ a synonym for the existing \iter\ keyword ~word.~  Only
the first word in each clause can have synonyms. 

\defmacroxx{defclause-sequence}{element-name index-name}{|\&key|
access-fn size-fn sequence-type}{element-type
element-doc-string index-doc-string}  
Provides
a simple way to define sequence clauses.  Generates two
clauses, one for iterating over the sequence's elements, the other
for iterating over its indices.  The first symbol of both
clauses will have print-name |for|.
~element-name~  and ~index-name~ should be symbols.
~element-name~ is the second keyword of the element iterator (typically of
the form 
|in-~sequence-type~|), and ~index-name~ is the second keyword
of the index-iterator (typically of the form
|index-of-~sequence-type~|).  Either name may be 
\nil, in which case the corresponding clause is not defined.  If both
symbols are supplied, they should be in the same package.  The |for|
that begins the clauses will be in this package.

\cpar ~access-fn~ is the function to be used to
access elements of the sequence in the element iterator.  The function
should take two 
arguments, a sequence and an index, and return the appropriate element.
~size-fn~ should denote a function of one argument, a sequence, that
returns its size.  Both ~access-fn~ and ~size-fn~ are required for the
element iterator, but only ~size-fn~ is needed for the index iterator.

\cpar The ~sequence-type~ and ~element-type~ keywords are used to
suggest types for the variables 
used to hold the sequence and the
sequence elements, respectively.  The usual rules about \iter's
treatment of variable type declarations apply (see section \ref{types}).

\cpar ~element-doc-string~ and ~index-doc-string~ are
the documentation strings, for use with |display-iterate-clauses|. 

\cpar The generated element-iterator performs destructuring on the
element variable.

\cpar As an example, the above |for\dots in-whole-vector| example
could have been written:
\begin{program}
(defclause-sequence IN-WHOLE-VECTOR INDEX-OF-WHOLE-VECTOR
  :access-fn 'aref
  :size-fn \#'(lambda (v) (array-dimension v 0))
  :sequence-type 'vector
  :element-type t
  :element-doc-string 
     "Elements of a vector, disregarding fill-pointer"
  :index-doc-string 
     "Indices of vector, disregarding fill-pointer")
\end{program}

\end{clauses}

\subsection{Subtleties}

There are some subtleties to be aware of when writing \iter\ clauses.
First, the code returned by your macros may be |nconc|'ed into a list,
so you should always returned freshly consed lists, rather than
constants.  Second, \iter\ matches clauses by using |eq| on the first
symbol and |string=| on the subsequent ones, so the package of the
first symbol of a clause is relevant.  All of the clauses in this manual
have their first word in the \iter\ package.  
You can use the package system in the usual way to shadow
\iter\ clauses without replacing them.

%%% say more here, about the badness that only the first word of a
%%% clause is packagey.

\section{Non-portable Extensions to Iterate (Contribs)}
\label{contribs}

Currently, there is only one non-portable extension to iterate in the
distribution: iterate-pg. If you have made an extension that depends
on non-portable features, feel free to send them to |asf@boinkor.net|
for inclusion in the iterate distribution.

\subsection{An SQL query driver for iterate}

The pg package by Eric Marsden (see |http://cliki.net/pg|) provides an
interface to the PostgreSQL database. Using the \iterpg\ extension, it
is possible to handle the results of SQL queries using \iter.

This usage example should give you an idea of how to use it:

\begin{program}
(pg:with-pg-connection (c "somedb" "someuser")
  (iter (for (impl version date) in-relation "select * from version"
                                 on-connection *dbconn*)
        (collect version)))
\end{program}

To use the extension via |ASDF|, simply make your system depend on the
|iterate-pg| system instead of the |iterate| system. To load it
manually, use:

\begin{program}
  (asdf:oos 'asdf:load-op :iterate-pg)
\end{program}

%%%%%%%%%%%%%%%%%%%%%%%%%%%%%%%%%%%%%%%%%%%%%%%%%%%%%%%%%%%%%%%%
\section{Obtaining {\tt Iterate}}

\begin{sloppypar}
The information in this chapter is obsolete but included for
completeness's sake; Currently, the most up-to-date information on
\iter\ can be found at |http://boinkor.net/iterate.html|.
\end{sloppypar}

\begin{sloppypar}
\iter\ currently runs on Lisp Machines, and on
HP's, Sun3's and Sparcstations under Lucid.  
\iter\ source and binaries are available at the MIT AI Lab in the
subdirectories of |/src/local/lisplib/|.  The source file,
|iterate.lisp|, is also available for anonymous FTP in the directory
|/com/fpt/pub/| on the machine |TRIX.AI.MIT.EDU| (Internet number
128.52.32.6).  If you are unable to obtain |iterate| in one of these
ways, send mail to |jba@ai.mit.edu| and I will send you the source
file. 
\end{sloppypar}

\begin{sloppypar}
\iter\ resides in the |iterate| package (nickname |iter|).  Just say
\linebreak |(use-package :iterate)| to make all the necessary symbols
available. 
If a symbol is not exported, it appears in this manual with an
``|iterate::|'' prefix.
\end{sloppypar}

Send bug reports to |bug-iterate@ai.mit.edu|.  The |info-iterate|
mailing list will have notices of changes and problems; to have
yourself added, send mail to |info-iterate-request@ai.mit.edu|.


\medskip

\begin{flushleft}
 \bf Acknowledgements
\end{flushleft}
\smallskip

Richard Waters provided invaluable criticism which spurred me to improve
\iter\ greatly.  As early users, David Clemens, Oren Etzioni and Jeff
Siskind helped ferret out many bugs.

%\begin{theindex}
% The index files must be generated with the genindex program
% or from makeindex iterate-manual.idx which creates iterate-manual.ind.

%\baselineskip\normalbaselineskip
%\advance\baselineskip by -2pt

\input{iterate-manual.ind}

%\underline{Clauses}
%\input{iterate-manual.clindex}
%
%\indexspace
%\underline{Lisp}
%\input{iterate-manual.lispindex}

%\end{theindex}


\end{document}


% arch-tag: "c0f871a6-313c-11d8-abb9-000c76244c24"


%\underline{Clauses}
%% Fix ``.'' spacing in lists
\documentstyle[12pt,aimemo]{article}

\setlength{\textwidth}{6.0in}
\setlength{\oddsidemargin}{0.25in}
\setlength{\evensidemargin}{0.25in}
\setlength{\topmargin}{0pt}
\setlength{\headsep}{24pt}
\setlength{\textheight}{8.5in}

\let\origpar=\par

% somehow, LaTex fucks with things so that the plain Tex \obeyspaces
% doesn't work.  This is the same, except we use an mbox containing
% a space instead of just a space.

\newcommand{\spc}{\mbox{ }}
\newcommand{\tAb}{\mbox{        }}
\def\obspaces{\catcode`\ =\active\catcode`\^^I=\active}
{\obspaces \global\let =\spc\global\let^^I=\tAb}

\newenvironment{program}{\medskip\samepage\tt\obeylines\obspaces}{\medskip}

% use | for \tt
\catcode`\|=13
\newif\ifvbar
\def|{\ifvbar
         \endgroup\vbarfalse
      \else 
         \begingroup\tt\vbartrue 
      \fi}



\newcommand{\lisp}{\tt}
\newcommand{\iter}{|iterate|}
\newcommand{\iterpg}{|iterate-pg|}
\newcommand{\setf}{|setf|}
\newcommand{\nil}{|nil|}
\newcommand{\nonnil}{non-\nil}
\newcommand{\opt}{{\lisp \&optional}}
\newcommand{\yields}{$\Rightarrow$}



\newbox\Dots
\setbox\Dots=\hbox{\small ...}
\renewcommand{\dots}{\copy\Dots}

% Here's a way to define an environment which will treat all
% paragraphs inside it in the same manner.  It's, as near as I can
% figure, the way that Latex does things.  Using \everypar won't work;
% apparently Latex resets that a lot.  What you have to do is: at the
% top of the file do \let\origpar=\par.  In the
% begin part of the environment, do the parshape (or whatever), then say
% \def\par{{\origpar}}.  The extra braces are necessary.   That's all
% you have to do; at the end of the environment things will return to
% normal automatically.  Why does this all work?  I have no idea.
% Another important thing: there should be a blank line (i.e. a \par)
% between the end of the last paragraph and the \end{environment}.

\newlength{\clindent}
\setlength{\clindent}{0.5in}
\newlength{\clparindent}
\setlength{\clparindent}{\parindent}

\newenvironment{clauses}{\advance\linewidth -\parindent
                        \hangindent\clindent\relax
                         \hangafter=0
                          \parindent=0pt
                          \def\par{{\origpar}}}{}

\newcommand{\cpar}{\hspace \clparindent}

\newcommand{\startitem}{\bigskip\pagebreak[3]}
\newcommand{\finishitem}{\smallskip}

\newcommand{\clause}[1]{\startitem\Clause{#1}\finishitem}
\newcommand{\Clause}[1]{\hbox{#1}}
\newcommand{\clindex}[1]{\index{{\lisp #1}}}
\newcommand{\clindexx}[2]{\index{{\lisp #1\dots #2}}}
\newcommand{\clindexxx}[3]{\index{{\lisp #1\dots #2\dots #3}}}
\newcommand{\clausex}[2]{\startitem\Clausex{#1}{#2}\finishitem}
\newcommand{\Clausex}[2]{\Clause{|#1| #2}\clindex{#1}}

\newcommand{\defvar}[1]{\startitem\deff{#1}{}{Variable}\finishitem}
\newcommand{\defunexpvar}[1]{\startitem\deffnoindex{iterate::#1}{}{Variable}
                                        \lispindex{#1}\finishitem}
\newcommand{\defun}[2]{\startitem\deff{#1}{#2}{Function}\finishitem}
\newcommand{\defmacro}[2]{\startitem\deff{#1}{#2}{Macro}\finishitem}
\newcommand{\defunx}[3]{\startitem\deffx{#1}{#2}{#3}{Function}\finishitem}
\newcommand{\defmacrox}[3]{\startitem\deffx{#1}{#2}{#3}{Macro}\finishitem}
\newcommand{\defunxx}[4]{\startitem\deffxx{#1}{#2}{#3}{#4}{Function}\finishitem}
\newcommand{\defmacroxx}[4]{\startitem\deffxx{#1}{#2}{#3}{#4}{Macro}\finishitem}

\newcommand{\deff}[3]{\deffnoindex{#1}{#2}{#3}\lispindex{#1}}
\newcommand{\deffnoindex}[3]{\hbox to \hsize{{\tt #1} {\it #2}\hfill [{\it #3}]}}
\newcommand{\deffx}[4]{\deff{#1}{#2}{#4}\moreargs{#1}{#3}}
\newcommand{\deffxx}[5]{\deffx{#1}{#2}{#3}{#5}\moreargs{#1}{#4}}

\newcommand{\moreargs}[2]{\hbox to \hsize{\phantom{\lisp #1} {\it #2}\hfill}}

\newcommand{\lispindex}[1]{\index{{\lisp #1}}}

 \newenvironment{note}[1]{\pagebreak[2]\bigskip
                          \hrule\smallskip\small
                         {\setlength{\parindent}{0pt}
                           \par{\bf #1:}}}{\smallskip\hrule\bigskip}



\makeindex

%%%%%%%%%%%%%%%%%%%%%%%%%%%%%%%%%%%%%%%%%%%%%%%%%%%%%%%%%%%%%%%%

%\thispagestyle{empty}

%\vspace*{2in}

%\title{The |Iterate| Manual}
%\author{Jonathan Amsterdam}

%\maketitle

%\thispagestyle{empty}

%\begin{center}
%\Large *** DRAFT ***
%\end{center}
%This manual is in beta-test. It will become an AI Memo in a couple of
%months.  I would appreciate any comments or
%corrections.  You can reach me at email address |jba@ai.ai.mit.edu|.

\begin{document}

\def\writtenby{Jonathan Amsterdam}
\def\contractno{This report describes research done at the Artificial
Intelligence Laboratory of the Massachusetts Institute of Technology.
Support for the laboratory's artificial intelligence research is
provided in part by the Advanced Research Projects Agency of the
Department of Defense under Office
of Naval Research contract N00014-85-K-0124.}

\AIMEMO{1236}{\Large The Iterate Manual \\[1in]}
{This is the manual for version 1.4 of \iter, a powerful iteration
macro for Common  
Lisp. \iter\ is similar to {\tt loop} but provides numerous additional
features, is well integrated with Lisp, and is extensible. 
\vfill }


% use ~ for \em
\catcode`\~=13
\newif\ifem
\def~{\ifem
         \/\endgroup\emfalse
      \else 
         \begingroup\em\emtrue 
      \fi}



%\pagebreak
%\setcounter{page}{1}

%\thispagestyle{empty}


\tableofcontents

\pagebreak
\pagestyle{headings}

\section{Introduction}

\begin{sloppypar}
This manual describes
\iter, a powerful iteration facility for Common Lisp.
\iter\ provides abstractions for many common iteration
patterns and allows for the definition of additional patterns.  \iter\ is
a macro that expands into ordinary Lisp at
compile-time, so it is more efficient than higher-order
functions like |map| and |reduce.|  
While it is similar to |loop|, \iter\ offers a more Lisp-like syntax
and enhanced extensibility.
(For a more complete
comparison of \iter\ with other iteration constructs, see MIT AI Lab Working
Paper No. 324, ~Don't Loop, Iterate.~)
\end{sloppypar}

An \iter\ form consists of the symbol |iter|\footnote{You can also use
|iterate|, but |iter| is preferred because it avoids potential conflicts with
possible future additions to Common Lisp, and because it saves
horizontal space when writing code.}
followed by one or more
forms, some of which may be \iter\ 
~clauses.~  Here is a simple example of \iter\ which collects the
numbers from 1 to 10 into a list, and returns the list.  The return
value is shown following the arrow.
\begin{program}
(iter (for i from 1 to 10)
      (collect i))          \yields (1 2 3 4 5 6 7 8 9 10)
\end{program}
This form contains two clauses: a |for| clause that steps the
variable |i| over the integers from 1 to 10, and a |collect|
clause that accumulates its argument into a list.  With a few
exceptions, all \iter\ clauses have the same format: alternating
symbols (called ~keywords~) and expressions (called
~arguments~).  The syntax and terminology are those of Common
Lisp's keyword lambda lists.  One difference is that \iter's keywords
do not have to begin with a colon---though they may, except for the
first symbol of a clause.  So you can also write  |(for i :from
1 :to 10)| if you prefer.

Any Lisp form can appear in the body of an \iter, where it will have
its usual meaning.  \iter\ walks the entire body, expanding macros,
and recognizing clauses at any level.  This example collects all the
odd numbers in a list:
\begin{program}
(iter (for el in list)
      (if (and (numberp el) (oddp el))
          (collect el)))
\end{program}

There are clauses for iterating over numbers, lists, arrays and other
objects, and for
collecting, summing, counting, maximizing and other useful operations.
\iter\ also supports the creation of new variable bindings, stepping
over multiple sequences at once, destructuring, and compiler
declarations of variable types.  The following example illustrates
some of these features: 
\begin{program}
(iter (for (key . item) in alist)
         (for i from 0)
         (declare (fixnum i))
         (collect (cons i key)))
\end{program}

This loop takes the keys of an alist and returns a new alist
associating the keys with their positions in the original list. The
compiler declaration for |i| will appear in the generated code
in the appropriate place.

%%%%%%%%%%%%%%%%%%%%%%%%%%%%%%%%%%%%%%%%%%%%%%%%%%%%%%%%%%%%%%%%
\section{Clauses}

Most of \iter's clauses will be familiar to |loop| programmers.  
(|loop| is an iteration macro that has been incorporated into Common
Lisp.  Seen Guy Steele's {\em Common Lisp, 2nd Edition}.)
In
nearly all cases they behave the same as their |loop| counterparts, so
a |loop| user can switch to \iter\ with little pain (and much gain).

All clauses with the standard keyword-argument syntax consist of two
parts: a ~required~ part, 
containing keywords that must be present and in the right order; and
an ~optional~ part, containing keywords that may be omitted and,
if present, may occur in any order.  In the descriptions below, the
parts are separated by the Lisp lambda-list keyword \opt.


\subsection{Drivers}

An iteration-driving clause
conceptually causes the iteration to go forward.  Driver clauses in
\iter\ allow iteration over numbers, lists, vectors, hashtables, packages,
files and streams.  Iteration-driving clauses must
appear at the top level of an \iter\ form; they cannot be nested
inside another clause.  The driver variable is updated at the point
where the driver clause occurs.  Before the clause is executed for the
first time, the value of the variable is undefined.

%Also, regardless of where the driver clause
%appears in the body, the driver variable is stepped at the top of the
%loop; hence it is stylistically preferable, though
%not required, to place driver clauses at the beginning of the \iter.

Multiple drivers may appear in a single \iter\ form, in which case all
of the driver variables are updated each time through the loop, in the
order in which the clauses appear.  The first driver to terminate will
terminate the entire loop.

In all cases, the value of the driver variable on exit from the loop,
including within the epilogue code (see the |finally| clause), is
undefined.

All the parameters of a driver clause are evaluated once, before the
loop begins.  Hence it is not possible to change the bounds or other
properties of an iteration by side-effect from within the loop.

With one exception, driver clauses begin with the word |for| (or
the synonym |as|) and mention an iteration variable, which is
given a binding within the \iter\ form.
The exception is |repeat|, which just executes a loop a
specified number of times:

\begin{clauses}

\clausex{repeat}{~n~}
Repeats the loop ~n~ times.  For example:
\begin{program}
(iter (repeat 100)
      (print "I will not talk in class."))
\end{program}
\cpar If $n \leq 0$, the
loop will never be executed.  If ~n~ is not an integer, the
actual number of executions will be $\lceil n \rceil$.

\end{clauses}

\pagebreak[3]

\subsubsection{Numerical Iteration}

\begin{clauses}
\clausex{for}{~var~ |\&sequence|} 
The general form for iterating over a sequence of numbers requires a
variable and, optionally, one or more keywords that provide the bounds
and step size of the iteration. The |\&sequence|
lambda-list keyword is a shorthand for these sequence keywords.  They are:
|from|, |upfrom|, |downfrom|, |to|, |downto|, |above|, |below| and |by|.
|from| provides the starting value for ~var~
and defaults to zero. |to| provides a final value and implies that the
successive values of ~var~ will be increasing; |downto| implies that
they will be decreasing.  The loop terminates when ~var~ passes
the final value (i.e. becomes smaller or larger than it, depending on
the direction of iteration); in other  words, the loop body
will never be 
executed for values of ~var~ past the final value.
|below| and |above| are similar to |to| and
|downto|, except that the loop terminates when ~var~ equals or passes
the final value.

\cpar If no final value is specified, the variable will be stepped
forever.  Using |from| or |upfrom| will result in increasing
values, while |downfrom| will give decreasing values.

\cpar On each iteration,
~var~ is incremented or decremented by the value of the sequence
keyword |by|, which defaults to 1.  It should always be a positive
number, even for downward iterations.

\cpar In the following examples, the sequence of numbers generated is shown
next to the clause.

\begin{program}
(for i upfrom 0) \yields\ 0 1 2 \ldots
(for i from 5)   \yields\ 5 6 7 \ldots    ; either from or upfrom is okay
(for i downfrom 0) \yields\ 0 -1 -2 \ldots
(for i from 1 to 3) \yields\ 1 2 3
(for i from 1 below 3) \yields\ 1 2
(for i from 1 to 3 by 2) \yields\ 1 3
(for i from 1 below 3 by 2) \yields\ 1
(for i from 5 downto 3) \yields\ 5 4 3
\end{program}

\end{clauses}

\subsubsection{Sequence Iteration}

There are a number of clauses for iterating over sequences.  In all of
them, the argument following |for| may be a list instead of a
symbol, in which case destructuring is performed.  See section
\ref{destructuring}. 

\begin{clauses}

\clause{|for| ~var~ |in| ~list~ \opt\ |by| ~step-function~}
\clindexx{for}{in}
~var~ is set to successive elements of list.  
~step-function,~ which 
defaults to |cdr|, is used to obtain the next sublist. 

\clause{|for| ~var~ |on| ~list~ \opt\ |by| ~step-function~}
\clindexx{for}{on}
~var~ is set to successive sublists of list. 
~step-function~ (default |cdr|) is used as in |for\dots in|.

\end{clauses}

\medskip

These two clauses use |atom| to test for the end
of a list.  Hence, given a list whose final |cdr| is not \nil,
they will silently ignore the last |cdr|.  Other choices are
|endp|,  which would signal an error, and |null|, which
would probably result in an error somewhere else.  If you wish to use
an end-test other than |atom|, set the variable
|iterate::*list-end-test*|\lispindex{*list-end-test*} to the name of
the desired function. 

\begin{clauses}
\clause{|for| ~var~ |in-vector| ~vector~ |\&sequence|}
\clindexx{for}{in-vector}
~var~ takes on successive elements from ~vector.~   The vector's
fill-pointer is observed.  Here and in subsequent clauses, the |\&sequence|
keywords include |with-index|, which takes a symbol as argument
and uses it for the index variable instead of an internally generated
symbol.  The other |\&sequence| keywords behave as in numerical
iteration, except that the default iteration bounds are the bounds of
the vector.  E.g. in |(for i in-vector v downto 3)|,
|i| will start off being bound to the last element in |v|,
and will  
be set to preceding elements down to and including the element with
index 3.

\clause{|for| ~var~ |in-sequence| ~seq~ |\&sequence|}
\clindexx{for}{in-sequence}
This uses Common Lisp's generalized sequence functions, |elt|
and |length|, to obtain elements and determine the length of
~seq.~   Hence it will work for any sequence, including lists, and will
observe the fill-pointers of vectors.

\clause{|for| ~var~ |in-string| ~string~ |\&sequence|}
\clindexx{for}{in-string}
~var~ is set to successive characters of ~string.~

\startitem
\Clause{|for| ~var~ |index-of-vector| ~vector~ |\&sequence|}
\clindexx{for}{index-of-vector}
\Clause{|for| ~var~ |index-of-sequence| ~sequence~ |\&sequence|}
\clindexx{for}{index-of-sequence}
\Clause{|for| ~var~ |index-of-string| ~string~ |\&sequence|}
\clindexx{for}{index-of-string}
\finishitem
~var~ is set to successive indices of the sequence.
These clauses avoid the overhead of accessing the sequence elements
for those applications where they do not need to be examined, or are
examined rarely.  They admit all the optional keywords of the other
sequence drivers except the (redundant) |with-index| keyword.

\clause{|for| ~(key value)~ |in-hashtable| ~table~}
\clindexx{for}{in-hashtable}

~key~ and ~value,~ which must appear as shown in a list
and may be destructuring templates, are set to the keys and values of
~table~.  If ~key~ is |nil|, then the hashtable's keys will be ignored;
similarly for ~value~.
The order in which 
elements of ~table~ will be retrieved is unpredictable.

\clause{|for| ~var~ |in-package| ~package~ \opt\ |external-only| ~ext~}
\clindexx{for}{in-package}
Iterates over all the symbols in ~package,~ or over only the
external symbols if ~ext~ is specified and non-|nil|.  ~ext~ is not
evaluated.  The same symbol may appear more than once.

\clause{|for| ~var~ |in-packages| ~packages~ \opt\ |having-access| ~symbol-types~}
\clindexx{for}{in-packages}
Iterates over all the symbols from the list of packages denoted by the
descriptor ~packages~ and having accessibility (or visibility) given by
~symbol-types~.  This defaults to the list |(:external :internal :inherited)|
and is not evaluated.  ~var~ must be a list of up to three variables: in each
iteration, these will be set to a symbol, its access-type and package (as per
|with-package-iterator| in ANSI-CL).  The same symbol may appear more than
once.

\end{clauses}

\begin{clauses}

\clause{|for| ~var~ |in-file| ~name~ \opt\ |using| ~reader~}
\clindexx{for}{in-file}
Opens the file ~name~ (which may be a string or pathname) for input,
and iterates 
over its contents. ~reader~ defaults to |read|, so by default ~var~
will be bound to the successive forms in the file.  The \iter\ body is
wrapped in an unwind-protect to ensure that the file is closed no
matter how the \iter\ is exited.

\clause{|for| ~var~ |in-stream| ~stream~ \opt\ |using| ~reader~}
\clindexx{for}{in-stream}
Like |for\dots in-file|, except that ~stream~ should be an existing
stream object that supports input operations.

\end{clauses}

\subsubsection{Generalized Drivers}

These are primarily useful for writing drivers that can also be used
as generators (see section \ref{generators}, below).
 
\begin{clauses}

\clause{|for| ~var~ |next| ~expr~}
\clindexx{for}{next}

~var~ is set to ~expr~ each time through the loop.  Destructuring is
performed.  When the clause is used as a generator, ~expr~ is the code
that is executed when |(next ~var~)| is 
encountered (see section \ref{generators}, below).
~expr~ should compute the first value for ~var~, as well
as all subsequent values, and is responsible for terminating the loop.
For compatibility with future versions of \iter, this termination
should be done with |terminate|\clindex{terminate}, which can be
considered a synonym for |finish| (see section \ref{control-flow}).

\cpar As an example, the following clauses are equivalent to |(for
i from 1 to 10)|:
\begin{program}
(initially (setq i 0))
(for i next (if (> i 10) (terminate) (incf i)))
\end{program}

\clause{|for| ~var~ |do-next| ~form~}
\clindexx{for}{do-next}
~form~ is evaluated each time through the loop.  Its value is ~not~
set to ~var~; that is ~form's~ job. ~var~ is only present so that
\iter\ knows it is a driver variable.  \linebreak |(for ~var~ next
~expr~)| is equivalent to |(for ~var~ do-next (dsetq ~var~ ~expr~))|.
(See section \ref{destructuring} for an explanation of |dsetq|.) 

\end{clauses}

\subsubsection{Generators}
\label{generators}

In all of the above clauses, the driver variable is updated on each
iteration.
Sometimes it is desirable to have greater control over updating.  
For instance, consider the problem of associating numbers, in
increasing order and with no gaps, with the
\nonnil\ elements of a list.  One obvious first pass at writing this is:

\begin{program}
(iter (for el in list)
      (for i upfrom 1)
      (if el (collect (cons el i))))
\end{program}
But on the list |(a b nil c)| this produces |((a . 1) (b . 2) (c .
4))| instead of the desired |((a . 1) (b . 2) (c . 3))|.  The problem
is that |i| is incremented each time through the loop, even when |el|
is |nil|.  

The problem could be solved elegantly if we could step |i| only when
we wished to.  This
can be accomplished for any \iter\ driver by writing |generate| (or
its synonym |generating|)
instead of |for|.  Doing so produces a ~generator~---a driver whose
values are yielded explicitly.  To obtain the next value of a
generator variable ~v~, write \linebreak |(next ~v~)|.  The value of a |next|
form is the next value of ~v~, as determined by its associated driver
clause.  |next| also has the side-effect of updating ~v~ to that
value.  If there is no next value, |next| will terminate the loop,
just as with a normal driver.


Using generators, we can now write our example like this:
\begin{program}
(iter (for el in list)
      (generate i upfrom 1)
      (if el (collect (cons el (next i)))))
\end{program}
Now |i| is updated only when |(next i)| is executed, and this occurs
only when |el| is \nonnil.

To better understand the relationship between ordinary drivers and
generators, observe that we can rewrite an ordinary driver using its
generator form immediately followed by |next|, as this example shows:
\begin{program}
(iter (generating i from 1 to 10)
      (next i)
      ...)
\end{program}
Provided that the loop body contains no |(next i)| forms, this will
behave just as if we had written |(for i from 1 to 10)|.

We can still refer to a driver variable ~v~ without using |next|; in
this case, its value is that given to it by the last evaluation of
|(next ~v~)|.  Before |(next ~v~)| has been called the first time, the
value of ~v~ is undefined.  

This semantics is more flexible than
one in which ~v~ begins the loop bound to its first value and calls
of |next| supply subsequent values, because it means the loop will not
terminate too soon if the generator's sequence is empty.  For
instance, consider the following code, which tags \nonnil\ elements of
a list using a list of tags, and also counts the null elements.
(We assume there are at least as many tags as \nonnil\ elements.)
\begin{program}
(let* ((counter 0)
       (tagged-list (iter (for el in list)
                          (generating tag in tag-list) 
                          (if (null el)
                              (incf counter)
                              (collect (cons el (next tag)))))))
  ...)
\end{program}
It may be that there are just as many tags as non-null elements of
|list|.  If all the elements of |list| are null, we still want the
counting to proceed, even though |tag-list| is \nil.  If |tag| had to be
assigned its first value before the loop begins, we would have had to
terminate the loop before the first iteration, since when |tag-list|
is \nil, |tag| 
has no first value.  With the existing semantics, however, |(next
tag)| will never execute, so the iteration will cover all the elements of
|list|. 
        
When the ``variable'' of a driver clause is actually a destructuring
template containing several variables, all the variables are eligible
for use with |next|.  As before, |(next ~v~)| evaluates to ~v's~ next
value; but the effect is to update all of the template's variables.
For instance, the following code will return the list |(a 2 c)|.
\begin{program}
(iter (generating (key . item) in '((a . 1) (b . 2) (c . 3)))
      (collect (next key))
      (collect (next item)))
\end{program}

Only driver clauses with variables can be made into generators.  This
includes all clauses mentioned so far except for |repeat|.  It does
~not~ include |for\dots previous|, |for\dots =|,
|for\dots initially\dots then| or |for\dots first\dots then| (see
below). 

\subsubsection{Previous Values of Driver Variables}

Often one would like to access the value of a variable on a previous
iteration.  \iter\ provides a special clause for accomplishing this.

\begin{clauses}

\clause{|for| ~pvar~ |previous| ~var~ \opt\ |initially| ~init~
|back| ~n~}
\clindexx{for}{previous}
Sets ~pvar~ to the previous value of ~var,~ which should be a driver
variable, a variable from another |for\dots previous| clause, or a
variable established by a |for\dots =|, 
|for\dots initially\dots then| or |for\dots first\dots then| clause
(see section \ref{setting}).
Initially, ~pvar~ is given the value ~init~ (which defaults to \nil).  
The ~init~ expression will be moved outside the loop body, so it
should not depend on anything computed within the loop.
~pvar~ retains the value of ~init~ until ~var~ is set to its second
value, at which point ~pvar~ is set to ~var's~ first value; and so on.  

\cpar The
argument ~n~ to |back| must be a constant, positive integer, and
defaults to 1.  It determines how many iterations back ~pvar~ should
track ~var~.  For example, when ~n~ is 2, then ~pvar~ will be assigned
~var's~ first value when ~var~ is set to its third value.

\cpar A |for\dots previous| clause may occur after or before its
associated driver clause. |for\dots previous| works with generators as
well as ordinary drivers.

\pagebreak[3]

\cpar Example:
\begin{program}
(iter (for el in '(1 2 3 4))
      (for p-el previous el)
      (for pp-el previous p-el initially 0)
      (collect pp-el))
\end{program}
This evaluates to |(0 0 1 2)|.  It could have been written more
economically as 
\begin{program}
(iter (for el in '(1 2 3 4))
      (for pp-el previous el back 2 initially 0)
      (collect pp-el))
\end{program}


\end{clauses}

\subsection{Variable Binding and Setting}
\label{setting}

Several clauses exist for establishing new variable bindings or for
setting variables in the loop.  They all support destructuring.

\begin{clauses}

\clausex{with}{~var~ \opt\ |=| ~value~}
Causes ~var~ to be bound to value before the loop body is entered.
If ~value~ is not supplied, ~var~ assumes a default
binding, which will be 
\nil\ in the absence of declarations.  Also, if ~value~ is not
supplied, no destructuring is performed; instead, ~var~ may be a list of
symbols, all of which are given default 
bindings.  If ~value~ is supplied, ~var~ is bound to it, with
destructuring.  

\cpar Because |with|
creates bindings whose scope includes the entire \iter\ form, it is
good style to put all |with| clauses at the beginning.

\cpar Successive occurrences of |with| result in sequential
bindings (as with 
|let*|).  There is no way to obtain parallel bindings; see
section \ref{bindings} for a rationale.  


\clause{|for| ~var~ |=| ~expr~}
\clindexx{for}{=}
On each iteration, ~expr~ is evaluated and ~var~ is set
to its value. 

\cpar This clause may appear to do the same thing as |for\dots next|.
In fact, they are quite different.  |for\dots =| provides only three
services: it sets up a binding for ~var~, sets it to ~expr~ on each
iteration, and makes it possible to use |for\dots previous| with
~var~.  |for\dots next| provides these services in addition to the
ability to turn the driver into a generator.
%Also, the code which sets ~var~ appears in the loop body in the same
%place as the |for\dots =| clause; the code for |for\dots next| appears
%at the top of the loop, as with other drivers (except when being used
%as a generator).

\clause{|for| ~var~ |initially| ~init-expr~ |then| ~then-expr~}
\clindexxx{for}{initially}{then}
Before the loop begins, ~var~ is set to ~init-expr;~ on all
iterations after the first it is set to ~then-expr.~  This clause
must occur at top-level.  ~init-expr~ will be moved outside the loop
body and ~then-expr~ will be moved to the end of the loop body, so
they are subject to code motion problems (see section
\ref{code-movement}). 

\cpar This clause may appear to be similar to |for\dots next|, but in
fact they differ significantly.  |for\dots initially\dots then| is
typically used to give ~var~ its first value before the loop begins,
and subsequent values on following iterations.  This is incompatible
with generators, whose first value and subsequent values must all be
computed by |(next ~var~)|.  Also, the update of ~var~ in 
|for\dots initially\dots then| does not occur at the location of the
clause. 
Use |for\dots initially\dots then| for
one-shot computations where its idiom is more convenient, but use
|for\dots next| for extending \iter\ with new drivers (see section
\ref{extend}).

\clause{|for| ~var~ |first| ~first-expr~ |then| ~then-expr~}
\clindexxx{for}{first}{then}
The first time through the loop, ~var~ is set to ~first-expr;~ on
subsequent iterations, it is set to ~then-expr.~  This differs from
|for\dots initially| in that ~var~ is set to ~first-expr~
inside the loop body, 
so ~first-expr~ may depend on the results of other clauses.  For
instance, 
\begin{program}
(iter (for num in list)
      (for i first num then (1+ i))
      ...)
\end{program}
will set |i| to the first element of |list| on the first
iteration, whereas 
\begin{program}
(iter (for num in list)
      (for i initially num then (1+ i))
      ...)
\end{program}
is probably erroneous; |i| will be bound to |num|'s
default binding (usually \nil) for the first iteration.

\end{clauses}

\begin{note}{Compatibility note}
|loop|'s |for\dots =| works like \iter's, but |loop| used the syntax
|for\dots =\dots then| to mean |for\dots initially\dots then|.  It was
felt that these two operations were sufficiently different to warrant
different keywords.

Also, the |for| in the above three clauses is misleading,
since none is true driver (e.g. none has a corresponding |generate|
form).  |setting| would have been a better choice, but |for| was used
to retain some compatibility with |loop|.  
\end{note}


%%%%%%%%%%%%%%%%%%%%%%%%%%%%%%%%
\subsection{Gathering Clauses}
Many of \iter's clauses accumulate values into a variable, or set a
variable under certain conditions.  At the end of the loop, this
variable contains the desired result.  All these clauses have an
optional |into| keyword, whose argument should be a symbol.  If the
|into| keyword is not supplied, the accumulation variable will be
internally generated and its value will be returned at the end of the
loop; if a variable is specified, that variable is used for the
accumulation, and is not returned as a result---it is up to the user
to return it explicitly, in the loop's epilogue code (see |finally|).
It is safe to examine the accumulation variable during the loop, but
it should not be modified.

These clauses all begin with a verb.  When the verb does
not conflict with an existing Common Lisp function, then it may be
used in either its infinitival or present-participle form (e.g. |sum|,
|summing|).  However, when there is a conflict with Common Lisp, only
the present-participle form may be used (e.g. |unioning|).  This is to
prevent \iter\ clauses from clashing with Common Lisp
functions. 

%although these clauses are described as ``producing a value,'' it is a
%mistake to think of the lisp list representing the clause as a
%value-producing form in the usual way.  clauses may legally be written
%where a value is expected, e.g. |(setq x (sum i))|, but the
%lisp value of a clause in such a context is undefined.

\subsubsection{Reductions}

~Reduction~ is an extremely common iteration pattern in which
the results of successive applications of a
binary operation are accumulated.
For example, a loop that computes the sum of the
elements of a list is performing a reduction with the addition
operation.  This could be written in Common Lisp as  |(reduce \#'+
list)| or with \iter\ as
\begin{program}
(iter (for el in list)
      (sum el))
\end{program}

\begin{clauses}

\clausex{sum}{~expr~ \opt\ |into| ~var~}
Each time through the loop, ~expr~ is evaluated and added to a
variable, which is bound initially to zero.  If ~expr~ has a type,
it is ~not~ used as the type of the sum variable, which is always
|number|.  To get the result variable to be of a more specific
type, use an explicit variable, as in
\begin{program}
(iter (for el in number-list)
      (sum el into x)
      (declare (fixnum x))
      (finally (return x)))
\end{program}

\clausex{multiply}{~expr~ \opt\ |into| ~var~}
Like |sum|, but the initial value of the result variable is $1$,
and the variable is updated by multiplying ~expr~ into it.

\clausex{counting}{~expr~ \opt\ |into| ~var~}
~expr~ is evaluated on each iteration.  If it is \nonnil, the
accumulation variable, initially zero, is incremented.

\startitem
\Clausex{maximize}{~expr~ \opt\ |into| ~var~}
\Clausex{minimize}{~expr~ \opt\ |into| ~var~}
\finishitem
~expr~ is evaluated on each iteration and its extremum (maximum or
minimum) is stored in the accumulation variable.  If ~expr~ is never
evaluated, then the result is \nil\ (if the accumulation variable is
untyped) or $0$ (if it has a numeric type).

\clausex{reducing}{~expr~ |by| ~func~ \opt\ |initial-value| ~init-val~ |into| ~var~}
This is a general way to perform reductions. ~func~ should be a
function of two arguments, the first of which will be the value
computed so far and
the second of which will be the value of ~expr~.  It should return the
new value.
|reducing| is roughly equivalent to the
Common Lisp |(reduce ~func~ ~sequence~ :key ~expr-function~)|, where
~expr-function~ is used to derive values from the successive elements
of ~sequence~.

\cpar If the |reducing| clause is never executed, the result is
undefined.

\cpar It is not necessary to provide an
initial value, but better code can be generated if one is supplied.
Regardless of its location in the \iter\ body, ~init-val~ will be
evaluated before the loop is entered, so it should not depend on any
value computed inside the \iter\ form.  

%\cpar if a ~var~ is not specified, you can get \iter\ to declare the
%type of the internal variable by putting a |the| expression around
%~func.~  see section \ref{types}.

\end{clauses}

%\begin{note}{implementation note}
%in principle, |maximize| and |minimize| can be thought of
%as reductions where the initial value is the smallest (or largest)
%value that the accumulation variable can assume.  because lisp's
%bignums can represent arbitrary integers, these clauses cannot be
%implemented as reductions in general.  if, however, the type of 
%~expr~ or ~var~ can be determined to be a fixnum
%or a float, \iter\ will implement the clause as a true reduction,
%using one of the constants |most-negative-fixnum|, |%most-positive-fixnum|, 
%|most-negative-short-float|, etc. as appropriate.
%\end{note}

\subsubsection{Accumulations}

All the predefined accumulation clauses add values to a sequence.  If
the sequence is a list, they all 
have the property that the partial list is kept in the correct order
and available for inspection at any point in the loop.

\begin{clauses}

\clausex{collect}{~expr~ \opt\ |into| ~var~ |at| ~place~ |result-type| ~type~}
Produces a sequence of the values of ~expr~ on each iteration. ~place~
indicates where the next value of ~expr~ is added to the list and may
be one of the symbols |start|, |beginning| (a synonym for |start|) or
|end|.  The symbol may be quoted, but need not be.  The default is
|end|.  For example,
\begin{program}
(iter (for i from 1 to 5)
      (collect i))
\end{program}
produces |(1 2 3 4 5)|, whereas
\begin{program}
(iter (for i from 1 to 5)
      (collect i at beginning))
\end{program}
produces |(5 4 3 2 1)| (and is likely to be faster in most Common Lisp
implementations).

\cpar If ~type~ is provided, it should be a subtype of |sequence|.
The default is |list|.  Specifying a type other than |list| will
result in |collect| returning a sequence of that type.  ~However,~ the
type of the sequence being constructed when inside the loop body is
undefined when a non-|list| type is specified.  (As with ~place~,
quoting ~type~ is optional.)

\clausex{adjoining}{~expr~ \opt\ |into| ~var~ |test| ~test~ |at| ~place~ |result-type| ~type~}
Like |collect|, but only adds the value of ~expr~ if it is not
already present.  ~test,~ which defaults to |\#'eql|, is
the test to be used with |member|.

\startitem
\Clausex{appending}{~expr~ \opt\ |into| ~var~ |at| ~place~}
\Clausex{nconcing}{\ ~expr~ \opt\ |into| ~var~ |at| ~place~}
\Clausex{unioning}{\ ~expr~ \opt\ |into| ~var~ |test| ~test~ |at| ~place~}
\Clausex{nunioning}{~expr~ \opt\ |into| ~var~ |test| ~test~ |at| ~place~}
\finishitem
These are like |collect|, but behave like the Common Lisp functions
|append|, |nconc|, |union| or |nunion|.  
As in Common Lisp, they work only on lists.  Also as in Common Lisp,
|unioning| and |nunioning| assume that the value of ~expr~ contains no
duplicates. 

\clausex{accumulate}{~expr~ |by| ~func~ \opt\ |initial-value| ~init-val~ |into| ~var~}
This is a general-purpose accumulation clause. ~func~ should be a
function of two arguments, the value of ~expr~ and the value
accumulated so far in the iteration, and it should return the updated
value.  If no initial value is supplied, \nil\ is used.  

%\cpar If a ~var~ is not specified, you can get \iter\ to declare the
%type of the internal variable by putting a |the| expression around
%~func.~  see section \ref{types}.

\cpar The differences between |accumulate| and |reducing| are slight.
One difference is that the functions take their arguments in a
different order.  Another is that in the absence of ~init-val~,
|accumulate| will use \nil, whereas |reducing| will generate different
code that avoids any dependence on the initial value.
The reason for having both clauses is that one usually
thinks of reductions (like |sum|) and accumulations (like |collect|)
as different beasts.

\end{clauses}


\subsubsection{Finders}

A ~finder~ is a clause whose value is an expression that meets some
condition.

\begin{clauses}

\clause{|finding| ~expr~ |such-that| ~test~ \opt\ |into| ~var~ |on-failure| ~failure-value~}
\clindexx{finding}{such-that}
If ~test~ (which is an expression) ever evaluates to \nonnil, the loop
is terminated, the 
epilogue code is run and the value of ~expr~ is returned.  Otherwise,
\nil\ (or ~failure-value,~ if provided) is returned.  If ~var~ is
provided, it will have either the \nonnil\ value of ~expr~ or
~failure-value~ when the epilogue code is run.

\cpar As a special case, if the ~test~ expression is a sharp-quoted
function, then it is applied to ~expr~ instead of being simply
evaluated.  E.g. |(finding x such-that \#'evenp)| is equivalent to
|(finding x such-that (evenp x))|.  

%\cpar although ~test~ need have nothing to do with ~%expr~ as in
%|(finding j such-that (> i 3))|, it usually 
%will: |(finding (length el) such-that (oddp (length el)))|.  to
%avoid performing the |length| computation twice, you could write
%|(finding (length el) such-that \#'oddp)| or |(finding (length
%el) such-that 'oddp)|; for these cases, \iter\ generates code that
%executes ~expr~ only once.  the code for |\#'oddp|
%is slightly different from that for {\lisp 'oddp}; see the discussion
%under {\lisp for\dots in} and {\lisp for\dots on}.

\cpar |On-failure| is a misnomer. Because it is always evaluated, it behaves
more like the default third argument to the |gethash| function. As a result,
|on-failure (error "Not found")| makes no sense. Instead, the clauses |leave|
or |thereis| can be used in conjunction with |finally| as follows:
\begin{program}
(iter (for x in '(1 2 3))
      (if (evenp x) (leave x))
      (finally (error "not found")))
\end{program}

\cpar This clause may appear multiple times when all defaults are
identical. It can also be used together with either |always|/|never| or
|thereis| if their defaults match. More specifically, |on-failure nil| is
compatible with |thereis|, while |on-failure t| is compatible with |always|
and |never| clauses.
\begin{program}
(iter (for i in '(7 -4 2 -3))
      (if (plusp i)
	  (finding    i  such-that (evenp i))
          (finding (- i) such-that (oddp i))))
\end{program}

\startitem
\Clause{|finding| ~expr~ |maximizing| ~m-expr~ \opt\ |into| ~var~}
\clindexx{finding}{maximizing}
\Clause{|finding| ~expr~ |minimizing| ~m-expr~ \opt\ |into| ~var~}
\clindexx{finding}{minimizing}
\finishitem
Computes the extremum (maximum or minimum) value of ~m-expr~ over all
iterations, and returns the value of ~expr~ corresponding to the
extremum.  ~expr~ is evaluated inside the loop at the time the new
extremum is established.  If ~m-expr~ is never evaluated (due to, for
example, being embedded in a conditional clause), then the returned
value depends on the type, if any, of ~expr~ (or ~var,~ if
one is supplied).  If there is no type, the returned
value will be \nil; if the type is numeric, the returned value will be
zero.

\cpar For these two clauses, ~var~ may be a list of two
symbols; in that case, the first is used to record ~expr~ and
the second, ~m-expr.~  

\cpar As with |finding\dots such-that|, if ~m-expr~ is a sharp-quoted
function, then it is called on ~expr~ instead of being evaluated.

\end{clauses}

\subsubsection{Boolean Tests}

\begin{clauses}

\clausex{first-iteration-p}{}
Returns |t| in the first cycle of the loop, otherwise \nil.

\clausex{first-time-p}{}
Returns |t| the first time the expression is evaluated, and then \nil\ forever.
This clause comes handy when printing (optional) elements separated
by a comma:

\begin{program}
(iter (for el in '(nil 1 2 nil 3))
      (when el
        (unless (first-time-p)
          (princ ", "))
        (princ el)))
\end{program}
produces |"1, 2, 3"|.

\end{clauses}

\subsubsection{Aggregated Boolean Tests}

\begin{clauses}

\clausex{always}{~expr~}
If ~expr~ ever evaluates to
\nil, then \nil\ is immediately returned; the epilogue code is not
executed.  If ~expr~ never evaluates to \nil, the epilogue code
is executed and the last value of ~expr~ (or |t| if ~expr~ was never
evaluated) is returned (whereas |loop| would constantly return |t|).

% mention last evaluated clause when multiple always clauses?

\clausex{never}{~expr~}
Like |(always (not ~expr~))|, except it does not influence the last
value returned by a possible other |always| clause. That is,
\begin{program}
(iter (repeat 2)
      (always 2)
      (never nil)) \yields 2 ; not t
\end{program}

\clausex{thereis}{~expr~}
If ~expr~ is ever \nonnil,
its value is immediately returned without running epilogue code.
Otherwise, the epilogue code is performed and \nil\ is returned.

This clause cannot be used together with |always| or |never|, because their
defaults are opposed (similarly, |(loop always 3 thereis nil)| refuses to
compile in some implementations of |loop|).

\end{clauses}


\subsection{Control Flow}
\label{control-flow}
Several clauses can be used to alter the usual flow of control in a loop.

Note: the clauses of this and subsequent sections don't adhere to \iter's
usual syntax, but instead use standard Common Lisp syntax.  Hence the
format for describing syntax subsequently is
like the standard format used in the Common Lisp manual, not like the
descriptions of clauses above.

\begin{clauses}

\clausex{finish}{}
Stops the loop and runs the epilogue code.

%for example:
%
%\begin{program}
%(iter (with answer = nil)
%         (initially (make-a-mess))
%         (for i from 1 to 10)
%         (when (correct? i)
%           (setq answer i)
%           (finish))
%         (finally (cleanup)))
%\end{program}
%
%this code will execute |cleanup| whether or not the test |(correct?
%i)| ever succeeds. 
%the (more elegant) formulation,
%\begin{program}
%(iter (initially (make-a-mess))
%         (for i from 1 to 10)
%         (finding i such-that (correct? i))
%         (finally (cleanup)))
%\end{program}
%would not execute |cleanup| if |(correct? i)| succeeded; it
%would do an immediate return.

\clausex{leave}{\opt\ ~value~}
Immediately returns ~value~ (default \nil) from the current \iter\
form, skipping the epilogue code.  Equivalent to using |return-from|.

\clausex{next-iteration}{}
Skips the remainder of the loop body and begins the next iteration of
the loop.

\clausex{while}{~expr~}
If ~expr~ ever evaluates to \nil, the loop is terminated and the
epilogue code executed.  Equivalent to |(if (not ~expr~) (finish))|.

\clausex{until}{~expr~}
Equivalent to |(if ~expr~ (finish))|.

\clausex{if-first-time}{~then~ \opt\ ~else~}
If this clause is being executed for the first time in this invocation
of the \iter\ form, then the ~then~ code is evaluated; otherwise the
~else~ code is evaluated.

\cpar |(for ~var~ first ~expr1~ then ~expr2~)| is almost equivalent to
\begin{program}
(if-first-time (dsetq ~var~ ~expr1~)
               (dsetq ~var~ ~expr2~))
\end{program}
The only difference is that the |for| version makes ~var~ available
for use with |for\dots previous|.

\end{clauses}

\subsection{Code Placement}
When fine control is desired over where code appears in a loop
generated by \iter, the following special clauses may be useful.
They are all subject to code-motion problems (see section
\ref{code-movement}).  

\begin{clauses}

\clausex{initially}{|\&rest| ~forms~}
The lisp ~forms~ are placed in the prologue section of the loop, where
they are executed once, before the loop body is entered.

\clausex{after-each}{|\&rest| ~forms~}
The ~forms~ are placed at the end of the loop body, where they
are executed after each iteration.  Unlike the other clauses in this
section, ~forms~ may contain \iter\ clauses.

\clausex{else}{|\&rest| ~forms~}
The lisp ~forms~ are placed in the epilogue section of the loop, where they
are executed if this |else| clause is never met during execution of the
loop and the loop terminates normally.

\clausex{finally}{|\&rest| ~forms~}
The lisp ~forms~ are placed in the epilogue section of the loop, where
they are executed after the loop has terminated normally.

\clausex{finally-protected}{|\&rest| ~forms~}
The lisp ~forms~ are placed in the second form of an unwind-protect
outside the loop.  They are always executed after the loop has
terminated, regardless of how the termination occurred.

\end{clauses}


%%%%%%%%%%%%%%%%%%%%%%%%%%%%%%%%%%%%%%%%%%%%%%%%%%%%%%%%%%%%%%%%
\section{Other Features}

\subsection{Multiple Accumulations}
\label{multiple}

\begin{sloppypar}
It is permitted to have more than one clause accumulate into the same
variable, as in the following:
\begin{program}
(iter (for i from 1 to 10)
      (collect i into nums)
      (collect (sqrt i) into nums)
      (finally (return nums)))
\end{program}
Clauses can only accumulate into the same variable if they are
compatible.  |collect|, |adjoining|, |appending|, |nconcing|,
|unioning| and |nunioning| are compatible with each other; |sum|,
|multiply| and |counting| are compatible; |always| and |never| are compatible;
|finding| \dots |such-that| is compatible with either |thereis| or |always|
and |never| when their defaults match; and |maximize| and |minimize| clauses
are compatible only with other |maximize| and |minimize| clauses,
respectively.

%note that the same variable ~cannot~ be both an accumulation
%variable and an ordinary variable; there can be only one variable with
%a given name within an \iter\ form. 
\end{sloppypar}

\subsection{Named Blocks}

Like Common Lisp |block|s, \iter\ forms can be given names.  The
name should be a single symbol, and it must be the first form in the
\iter.  The generated code behaves exactly like a named block; in
particular, |(return-from ~name~)| can be used to exit it:
\begin{program}
(iter fred
      (for i from 1 to 10)
      (iter barney
            (for j from i to 10)
            (if (> (* i j) 17)
                (return-from fred j))))
\end{program}
An \iter\ form that is not given a name is implicitly named \nil.

Sometimes one would like to write an expression in an inner \iter\ form,
but have it processed by an outer \iter\ form.  This is possible with
the |in| clause.  

\begin{clauses}

\clausex{in}{~name~ |\&rest| ~forms~}
Evaluates ~forms~ as if they were part of the \iter\ form named
~name~.  In other words, \iter\ clauses are processed by the \iter\
form named ~name,~ and not by any \iter\ forms that occur inside ~name.~

\cpar As an example, consider the problem of collecting a list of the
elements in a two-dimensional array.  The naive solution,
\begin{program}
(iter (for i below (array-dimension ar 0))
      (iter (for j below (array-dimension ar 1))
            (collect (aref ar i j))))
\end{program}
\noindent is wrong because the list created by the inner \iter\ is simply
ignored by the outer one.  But using |in| we can write:
\begin{program}
(iter outer (for i below (array-dimension ar 0))
      (iter (for j below (array-dimension ar 1))
            (in outer (collect (aref ar i j)))))
\end{program}
\noindent which has the desired result.

\end{clauses}

\subsection{Destructuring}
\label{destructuring}

In many places within \iter\ clauses where a variable is expected, a
list can be written instead.  In these cases, the value to be assigned
is ~destructured~ according to the pattern
described by the list.  As a simple example, the clause
\begin{program}
(for (key . item) in alist)
\end{program}
\noindent will result in |key| being set to the |car| of
each element in |alist|, and |item| being set to the |cdr|.  The
pattern list may be nested to arbitrary depth, and (as the example
shows) need not be terminated with \nil; the only requirement is that
each leaf be a bindable symbol (or \nil, in which case no binding is
generated for that piece of the structure).

Sometimes, you might like to do the equivalent of a
|multiple-value-setq| in a clause.  This 
``multiple-value destructuring'' can be expressed by writing 
\linebreak |(values $pat_1$ $pat_2 \ldots$)| for a destructuring
pattern, as in 
\begin{program}
(for (values (a . b) c d) = (three-valued-function ...))
\end{program}
\begin{sloppypar}
Note that the $pat_i$ can themselves be destructuring patterns (though
not multiple-value destructuring patterns).  You can't do multiple-value
destructuring in a |with| clause; instead wrap the whole \iter\
form in a |multiple-value-bind|.
\end{sloppypar}

\begin{note}{Rationale}
There are subtle interactions between variable declarations and
evaluation order that make the correct implementation of
multiple-value destructuring in a |with| somewhat tricky.
\end{note}

The destructuring feature of \iter\ is available as a separate
mechanism, using the |dsetq| macro:

\begin{clauses}
\defmacro{dsetq}{template expr}
Performs destructuring of ~expr~ using ~template~.   May be used
outside of an \iter\ form. Yields the primary value of ~expr~.

\end{clauses}

\subsection{On-line Help}

\begin{sloppypar}
There is a limited facility for on-line help, in the form of the 
|display-iterate-clauses| function.
\end{sloppypar}

\begin{clauses}

\defun{display-iterate-clauses}{\opt\ clause-spec}
Displays a list of \iter\ clauses.  If ~clause-spec~ is not
provided, all clauses are shown; if it is a symbol, all clauses
beginning with that symbol are shown; and if it is a list of symbols,
all clauses for which ~clause-spec~ is a prefix are shown.

\end{clauses}

\subsection{Parallel Binding and Stepping}
\label{bindings}

The parallel binding and stepping of variables is a feature that
\iter\ does ~not~ have.  This section attempts to provide a rationale.

We say that two variables are bound ~in parallel~ if neither
binding shadows the other.  This is the usual semantics of |let|
(as opposed to |let*|).  Similarly, we can say that iteration
variables are stepped in parallel if neither variable is updated
before the other, conceptually speaking; in other words, if the code
to update each variable can reference the old values of both variables.

|loop| allows parallel binding of variables and parallel stepping of
driver variables.  My view is that if you are depending on the 
serial/parallel distinction, you are doing something obscure.  If you
need to bind 
variables in parallel using |with|, then you must be using a
variable name that shadows a name in the existing lexical environment.
Don't do that.  The most common use for parallel stepping is to track
the values of variables on the previous iteration, but in fact this
does not require parallel stepping at all; the following will work:
\begin{program}
(iter (for current in list)
      (for prev previous current)
      ...)
\end{program}


%%%%%%%%%%%%%%%%%%%%%%%%%%%%%%%%%%%%%%%%%%%%%%%%%%%%%%%%%%%%%%%%
\section{Types and Declarations}
\label{types}

\subsection{Discussion}

Sometimes efficiency dictates that the types of variables be declared.
This type information needs to be communicated to \iter\ so it can
bind variables to appropriate values.  Furthermore, \iter\ must often
generate internal variables invisible to the user; there needs to be a
way for these to be declared.

As an example, consider this code, which will return the number of
odd elements in |number-list|:
\begin{program}
(iter (for el in number-list)
      (count (oddp el)))
\end{program}
In processing this form,
\iter\ will create an internal variable, let us call it |list17|, to
hold the successive |cdr|s of |number-list|, and  
will bind the variable to |number-list|.  It will also generate a
default binding for |el|; only inside the body of the loop will |el|
be set to the |car| of |list17|.  Finally, \iter\ will generate a
variable, call it |result|, to hold the result of the count, and will
bind it to zero.

When dealing with type declarations, \iter\ observes one simple rule:
~it will never generate a declaration unless requested to do so.~  The
reason is that such declarations might mask errors in compiled code by
avoiding error-checks; the resulting problems would be doubly hard to
track down because the declarations would be hidden from the
programmer.  Of course, a compiler might omit error-checks even in the
absence of declarations, though this behavior can usually be avoided,
e.g. by saying |(declaim (optimize (safety 3)))|.

So, the above \iter\ form will generate code with no declarations.
But say we wish to declare the types of |el| and the internal
variables |list17| and |result|.  How is this done?

Declaring the type of |el| is easy, since the programmer knows
the variable's name:
\begin{program}
(iter (for el in number-list)
      (declare (fixnum el))
      (counting (oddp el)))
\end{program}
\iter\ can read variable type declarations like this one.  Before
processing any clauses, it scans the entire top-level form for type
declarations 
and records the types, so that variable bindings can be performed
correctly.  In this case, |el| will be bound to zero
instead of \nil.  Also, \iter\ collects all the top-level declarations
and puts them at the begining of the generated code, so it is not
necessary to place all declarations at the beginning of an \iter\
form; instead, they can be written near the variables whose types they
declare.  

Since \iter\ is not part of the compiler, it will not know
about declarations that occur outside an \iter\ form; these
declarations must be repeated inside the form.

Here is another way we could have declared the type of |el|:
\begin{program}
(iter (for (the fixnum el) in number-list)
      (counting (oddp el)))
\end{program}
\iter\ extends the Common Lisp |the|\lispindex{the}
form to apply to variables as well as value-producing forms; anywhere
a variable is allowed---in a |with| clause, as the iteration
variable in a driver clause, as the |into| argument of an
accumulation clause, even inside a destructuring template---you can
write |(the ~type~ ~symbol~)| instead.

There is one crucial difference between using a |the| form and
actually declaring the variable: explicit declarations are always
placed in the generated code, but type information from a |the|
form is not turned into an actual declaration unless you tell \iter\
to do so using |iterate:declare-variables|.  See below.

\begin{sloppypar}
Declaring the types of internal variables is harder than declaring the
types of explicitly mentioned variables, since their names
are unknown.  You do it by declaring |iterate:declare-variables|
somewhere inside the top level of the \iter\ form.  (This will also
generate declarations for variables declared using |the|.)
\iter\ does not provide much selectivity here: it's all or none. 
And unfortunately, since \iter\ is not privy to compiler information
but instead reads declarations itself, it will not hear if you
|(declaim (iterate:declare-variables))|.  Instead, set the variable 
|iterate::*always-declare-variables*| to |t| at
compile-time, using |eval-when|.
\end{sloppypar}

To determine the appropriate types for internal variables, \iter\ uses
three sources of information:

\begin{itemize}

\item Often, the particular clause dictates a certain type for a
variable; \iter\ will use this information when available.  In the
current example, the variable |list17| will be given the type
|list|, since that is the only type that makes sense; and the
variable |result| will be given the type |fixnum|, on the
assumption that you will not be counting high enough to need bignums.
You can override this assumption only by using and explicitly declaring a
variable: 
\begin{verbatim}
(iter (declare (iterate:declare-variables))
      (for el in number-list)
      (count (oddp el) into my-result)
      (declare (integer my-result))
      (finally (return my-result)))
\end{verbatim}

\begin{sloppypar}
Other examples of the type assumptions that \iter\ makes are: type
|list| for |into| variables of collection clauses; type |list| for
expressions that are to be destructured; type |vector| for the
variable holding the vector in a |for\dots in-vector| clause, and
similarly for |string| and the |for\dots in-string| clause;
and the implementation-dependent type
|(type-of array-dimension-limit)| for the index and limit
variables generated by sequence iteration drivers like |for\dots
in-vector| and |for\dots in-string| (but not |for\dots in-sequence|,
because it may be used to iterate over a list). 
\end{sloppypar}

\item Sometimes, \iter\ will examine expressions and try to determine
their types in a simple-minded way.  If the expression is
self-evaluating (like a number, for instance), \iter\ knows that the
expression's type is the same as the type of the value it denotes, so
it can use that type.  If the expression is of the form |(the ~type~
~expr~)|, \iter\ is smart enough to extract ~type~ and use it.
However, the current version of \iter\ does not  
examine declarations of function result types or do any type
inference.  It will not determine, for 
example, that the type of |(+ 3 4)| is |fixnum|, or even
|number|.

\item In some cases, the type of an internal variable should match the
type of some other variable.  For instance, \iter\ generates an
internal variable for |(f x)| in the
clause |(for i from 1 to (f x))|, and in the absence of other
information will give it the same type as |i|.  If, however, the
expression had been written |(the fixnum (f x))|, then \iter\
would have given the internal variable the type |fixnum|
regardless of |i|'s type.  The type incompatibility errors that
could arise in this situation are not checked for.

\end{itemize}

Note that if you do declare |iterate:declare-variables|, then
\iter\ may declare user variables as well as internal ones if they do
not already have declarations, though only for variables that it
binds.  For instance, in this code:

\begin{program}
(iter (declare (iterate:declare-variables))
      (for i from 1 to 10)
      (collect i into var))
\end{program}
the variable |var| will be declared to be of type |list|.


\subsection{Summary}

\iter\ understands standard Common Lisp variable type declarations
that occur within an \iter\ form and
will pass them through to the generated code.  If the declaration
|(iterate:declare-variables)|\lispindex{declare-variables}
appears at the top level of an 
\iter\ form, or if 
|iterate::*always-declare-variables*|\lispindex{*always-declare-variables*} 
is \nonnil, then \iter\ will use the type information gleaned from user
declarations, self-evaluating expressions and |the| expressions,
combined with reasonable assumptions, to determine variable
types and declare them.


%%%%%%%%%%%%%%%%%%%%%%%%%%%%%%%%%%%%%%%%%%%%%%%%%%%%%%%%%%%%%%%%
\section{Problems with Code Movement}
\label{code-movement}

Some \iter\ clauses, or parts of clauses, result in code being
moved from the location of the clause to other parts of the loop.
Drivers behave this way, as do code-placement clauses like |initially|
and |finally|.  When using these clauses, there is a danger of writing
an expression that makes sense in its apparent location but will be
invalid or have a different meaning in another location.  For example:
\begin{program}
(iter (for i from 1 to 10)
      (let ((x 3))
        (initially (setq x 4))))
\end{program}
While it may appear that the |x| of |(initially (setq x 4))| is the
same as the |x| of |(let ((x 3)) \dots|, in fact they are not:
|initially| moves its code outside the loop body, so |x| would refer
to a global variable.  Here is another example of the same problem:
\begin{program}
(iter (for i from 1 to 10)
      (let ((x 3))
        (collect i into x)))
\end{program}
If this code were executed, |collect| would create a binding for its
|x| at the top level of the \iter\ form that the |let| will shadow.

Happily, \iter\ is smart enough to catch these errors; it
walks all problematical code to ensure that free variables are not
bound inside the loop body, and checks all variables it binds for the
same problem.

However, some errors cannot be caught:

\begin{program}
(iter (with x = 3)
      (for el in list)
      (setq x 1)
      (reducing el by \#'+ initial-value x))
\end{program}
|reducing| moves its |initial-value| argument to the initialization
part of the loop in order to produce more efficient code.  Since
\iter\ does not perform data-flow analysis, it cannot determine that
|x| is changed inside the loop; all it can establish is that |x| is
not bound internally.  Hence this code will not signal an
error and will use $3$ as the initial value of the reduction.

The following list summarizes all cases that are subject to these code
motion and variable-shadowing problems.
\begin{itemize}
\item Any variable for which \iter\ creates a binding, including those
used in |with| and the |into| keyword of many clauses.

\begin{sloppypar}
\item The special clauses which place code: |initially|, |after-each|, |else|,
|finally| and |finally-protected|. 
\end{sloppypar}

\item The variables of a |next| or |do-next| form.

\item The |initially| arguments of |for\dots initially\dots then| and
|for\dots previous|. 

\item The |then| argument of |for\dots initially\dots then|.

\item The |initial-value| arguments of |reducing| and |accumulate|.

\item The |on-failure| argument of |finding\dots such-that|.
\end{itemize}

%%%%%%%%%%%%%%%%%%%%%%%%%%%%%%%%%%%%%%%%%%%%%%%%%%%%%%%%%%%%%%%%
\section{Differences Between {\tt Iterate} and {\tt Loop}}
|loop| contains a great deal of complexity which \iter\ tries to
avoid.  Hence many esoteric features of |loop| don't exist in \iter.
Other features have been carried over, but in a cleaned-up form.  
And of course, many new features have been added; they are not
mentioned in this list.

\begin{itemize}

\item \iter's syntax is more Lisp-like than |loop|'s, having a higher
density of parens.

\item The current implementation of \iter, unlike the current version
of  |loop| (as documented in {\em Common Lisp, 2nd Ed.\/}), is
extensible (see section \ref{extend}).  

\item |loop| puts the updates of all driver variables at the top of
the loop; \iter\ leaves them where the driver clauses appear.

\item While for the most part \iter\ clauses that resemble |loop| clauses
behave similarly, there are some differences.  For instance, there is
no |for\dots =\dots then| in \iter; instead use 
|for\dots initially\dots then|.  

\item |loop| binds the variable |it| at certain times
to allow pseudo-English expressions like |when ~expr~ return it|. 
In \iter, you must bind ~expr~ to a variable yourself.  Note that
|when ~expr~ return it| is like |thereis ~expr~| except that the latter is an
accumulation clause and therefore competes with other accumulations
(remember section \ref{multiple} above).

% repeat different behaviour of |always| clause here?

\item |loop| has a special |return| clause, illustrated in the
previous item.  \iter\ doesn't need one,  since an ordinary Lisp
|return| has the same effect.

\item |loop| allows for parallel binding and stepping of iteration
variables.  \iter\ does not.  (See section \ref{bindings}.)

\item |loop| and \iter\ handle variable type declarations very
differently.  |loop| provides a special syntax for declaring variable
types, and does not examine declarations.  Moreover, the standard
implementation of |loop| will
generate declarations when none are requested.
\iter\ parses standard Common Lisp type declarations, and will never
declare a variable itself unless declarations are 
specifically requested.

\end{itemize}

%%%%%%%%%%%%%%%%%%%%%%%%%%%%%%%%%%%%%%%%%%%%%%%%%%%%%%%%%%%%%%%%
\section{Rolling Your Own}
\label{extend}

\subsection{Introduction}

\iter\ is extensible---you can write new clauses that embody new
iteration patterns.  You might want to write a new driver clause for a
data structure of your own, or you might want to write a clause that
collects or manipulates elements in a way not provided by \iter. 

This section describes how to write clauses for \iter.  Writing a
clause is like writing a macro.  In fact, writing a clause ~is~
writing a macro: since \iter\ code-walks its body and macroexpands,
you can add new abstractions to \iter\ with good old |defmacro|.

Actually, there are two extensions you can make to \iter\ that are
even easier than writing a macro.  They are adding a synonym for an
existing clause and defining a driver clause for an indexable
sequence.  These can be done with |defsynonym| and
|defclause-sequence|, respectively.  See section \ref{aids}, below.

The rest of this section explains how to write macros that expand into
\iter\ clauses.
Here's how you could add a simplified version of \iter's |multiply|
clause, if \iter\ didn't already have one:
\begin{program}
(defmacro multiply (expr)
  `(reducing ,expr by \#'* initial-value 1))
\end{program}

If you found yourself summing the square of an expression often, you
might want to write a macro for that.  A first cut might be
\begin{program}
(defmacro sum-of-squares (expr)
  `(sum (* ,expr ,expr)))
\end{program}
but if you are an experienced macro writer, you will realize that this
code will evaluate ~expr~ twice, which is probably a bad idea.  A
better version would use a temporary:
\begin{program}
(defmacro sum-of-squares (expr)
  (let ((temp (gensym)))
    `(let ((,temp ,expr))
       (sum (* ,temp ,temp)))))
\end{program}
Although this may seem complex, it is just the sort of thing you'd
have to go through to write any macro, which illustrates the point of
this section: if you can write macros, you can extend \iter.

Our macros don't use \iter's keyword-argument
syntax.  We could just use keywords with |defmacro|, but we would
still not be using \iter's clause indexing mechanism.  Unlike Lisp,
which uses just the first symbol of a form to determine what function
to call, \iter\ individuates clauses by the list of required keywords.
For instance, |for\dots in| and |for\dots in-vector| are different clauses
implemented by distinct Lisp functions.  

To buy into this indexing scheme, as well as the keyword-argument
syntax, use |defmacro-clause|:

\begin{clauses}

\defmacro{defmacro-clause}{arglist |\&body| body}
Defines a new \iter\ clause.  ~arglist~ is a list of symbols which are
alternating keywords and arguments.  \opt\ may be used, and the list
may be terminated by |\&sequence|.  ~body~ is an ordinary macro body,
as with |defmacro|.  If the first form of ~body~ is a string, it is
considered a documentation string and will be shown by
|display-iterate-clauses|. |defmacro-clause| will signal an error if 
defining the clause would result in an ambiguity.  E.g. you cannot
define the clause |for\dots from| because there would be no way to
distinguish it from a use of the |for| clause with optional keyword |from|.

\end{clauses}

\medskip

Here is |multiply| using |defmacro-clause|.  The keywords are capitalized
for readability.
\begin{program}
(defmacro-clause (MULTIPLY expr \opt\ INTO var)
  `(reducing ,expr by \#'* into ,var initial-value 1))
\end{program}
You don't have to worry about the case when |var| is not supplied; for
any clause with an |into| keyword, saying |into nil| is equivalent to
omitting the |into| entirely.

As another, more extended example, consider the fairly common
iteration pattern that 
involves finding the sequence element that maximizes (or minimizes) some
function.  \iter\ provides this as |finding\dots maximizing|, but it's
instructive to see how to write it.
Here, in pseudocode, is how you might write such a loop for
maximizing a function F:

\begin{quote}
\begin{tabbing}
set variable MAX-VAL to NIL; \\
set variable WINNER to NIL; \\
for\=\ each element EL in the sequence \\
\>  if\=\ MAX-VAL\=\ is NIL or F(EL) $>$ MAX-VAL then \\
\>\>    set MAX-VAL to F(EL); \\
\>\>    set WINNER to EL; \\
\>  end if; \\
end for; \\
return WINNER.
\end{tabbing}
\end{quote}

Here is the macro:
\begin{program}
(defmacro-clause (FINDING expr MAXIMIZING func \opt\ INTO var)
  (let ((max-val (gensym))
        (temp1 (gensym))
        (temp2 (gensym))
        (winner (or var iterate::*result-var*)))
    `(progn 
       (with ,max-val = nil)
       (with ,winner = nil)
       (cond 
        ((null ,max-val)
         (setq ,winner ,expr)
         (setq ,max-val (funcall ,func ,winner))
        (t
         (let* ((,temp1 ,expr)
                (,temp2 (funcall ,func ,temp1)))
           (when (> ,temp2 ,max-val)
             (setq ,max-val ,temp2)
             (setq ,winner ,temp1))))))
       (finally (leave ,winner)))))
\end{program}
Note that if no |into| variable is supplied, we use
|iterate::*result-var*|, which contains the internal variable into
which all clauses place their results.  If this variable is bound by
some clause, then \iter\ will return its value automatically;
otherwise, \nil\ will be returned.  

\subsection{Writing Drivers}

In principle, drivers can be implemented just as easily as other
\iter\ clauses.  In practice, they are a little harder to get right. 
As an example, consider writing a driver that
iterates over all the 
elements of a vector, ignoring its fill-pointer.  |for\dots in-vector|
won't work for this, because it observes the fill-pointer.  It's necessary to
use |array-dimension| instead of |length| to obtain the size of the
vector.  Here is one approach:

\begin{program}
(defmacro-clause (FOR var IN-WHOLE-VECTOR v)
  "All the elements of a vector (disregards fill-pointer)"
  (let ((vect (gensym))
        (index (gensym)))
    `(progn
       (with ,vect = ,v)
       (for ,index from 0 below (array-dimension ,vect 0))
       (for ,var = (aref ,vect ,index)))))
\end{program}
Note that we immediately put |v| in a variable, in case it is an
expression.  Again, this is just good Lisp macrology.  It also has a
subtle effect on the semantics of the driver: |v| is evaluated only
once, at the beginning of the loop, so changes to |v| in the loop have
no effect on the driver.  Similarly, the bounds for numerical iteration
e.g. the above |array-dimension| are also evaluated once only.  This is how
all of \iter's drivers work.


There is an important point concerning the |progn| in this code.  We
need the |progn|, of course, because we are returning several forms,
one of which is a driver.  But \iter\ drivers must occur at top-level.
Is this code in error?  No, because ~top-level~ is defined in \iter\
to include forms inside a |progn|.  This is just the definition of
top-level that Common Lisp uses, and for the same reason: to allow
macros to return multiple forms at top-level.

While our |for\dots in-whole-vector| clause will work, it is not
ideal.  In particular, it does not support generating.  Do do so, we
need to use |for\dots next| or |for\dots do-next|.  The job is
simplified by the |defmacro-driver| macro.

\begin{clauses}

\defmacro{defmacro-driver}{arglist |\&body| body}
Defines a driver clause in
both the |for| and |generate| forms, and provides a parameter
|generate| which ~body~ can examine to determine how it was invoked.
~arglist~ is as in |defmacro-clause|, and should begin with the symbol
|for|. 

\end{clauses}

With |defmacro-driver|, our driver looks like this:
\begin{program}
(defmacro-driver (FOR var IN-WHOLE-VECTOR v)
  "All the elements of a vector (disregards fill-pointer)"
   (let ((vect (gensym))
         (end (gensym))
         (index (gensym))
         (kwd (if generate 'generate 'for)))
     `(progn
        (with ,vect = ,v)
        (with ,end = (array-dimension ,vect 0))
        (with ,index = -1)
        (,kwd ,var next (progn (incf ,index)
                               (if (>= ,index ,end) (terminate))
                               (aref ,vect ,index))))))
\end{program}


We are still missing one thing: the |\&sequence| keywords.  
We can get them easily enough, by writing 
\begin{program}
(defmacro-driver (FOR var IN-WHOLE-VECTOR v \&sequence) 
  ...)
\end{program}
We can now refer to parameters |from|, |to|, |by|, etc. which contain
either the values for the corresponding keyword, or \nil\ if the
keyword was not supplied.  Implementing the right code for these
keywords is cumbersome but not difficult; it is left as an exercise.
But before you begin, see |defclause-sequence| below for an easier way.
       
\subsection{Extensibility Aids}
\label{aids}

This section documents assorted features that may be of use in
extending \iter.

\begin{clauses}

\defunexpvar{*result-var*}
Holds the variable that is used to return a value as a result of the
\iter\ form.  You may examine this and use it in a |with| clause, but
you should not change it.

\defmacro{defsynonym}{syn word}
 Makes ~syn~ a synonym for the existing \iter\ keyword ~word.~  Only
the first word in each clause can have synonyms. 

\defmacroxx{defclause-sequence}{element-name index-name}{|\&key|
access-fn size-fn sequence-type}{element-type
element-doc-string index-doc-string}  
Provides
a simple way to define sequence clauses.  Generates two
clauses, one for iterating over the sequence's elements, the other
for iterating over its indices.  The first symbol of both
clauses will have print-name |for|.
~element-name~  and ~index-name~ should be symbols.
~element-name~ is the second keyword of the element iterator (typically of
the form 
|in-~sequence-type~|), and ~index-name~ is the second keyword
of the index-iterator (typically of the form
|index-of-~sequence-type~|).  Either name may be 
\nil, in which case the corresponding clause is not defined.  If both
symbols are supplied, they should be in the same package.  The |for|
that begins the clauses will be in this package.

\cpar ~access-fn~ is the function to be used to
access elements of the sequence in the element iterator.  The function
should take two 
arguments, a sequence and an index, and return the appropriate element.
~size-fn~ should denote a function of one argument, a sequence, that
returns its size.  Both ~access-fn~ and ~size-fn~ are required for the
element iterator, but only ~size-fn~ is needed for the index iterator.

\cpar The ~sequence-type~ and ~element-type~ keywords are used to
suggest types for the variables 
used to hold the sequence and the
sequence elements, respectively.  The usual rules about \iter's
treatment of variable type declarations apply (see section \ref{types}).

\cpar ~element-doc-string~ and ~index-doc-string~ are
the documentation strings, for use with |display-iterate-clauses|. 

\cpar The generated element-iterator performs destructuring on the
element variable.

\cpar As an example, the above |for\dots in-whole-vector| example
could have been written:
\begin{program}
(defclause-sequence IN-WHOLE-VECTOR INDEX-OF-WHOLE-VECTOR
  :access-fn 'aref
  :size-fn \#'(lambda (v) (array-dimension v 0))
  :sequence-type 'vector
  :element-type t
  :element-doc-string 
     "Elements of a vector, disregarding fill-pointer"
  :index-doc-string 
     "Indices of vector, disregarding fill-pointer")
\end{program}

\end{clauses}

\subsection{Subtleties}

There are some subtleties to be aware of when writing \iter\ clauses.
First, the code returned by your macros may be |nconc|'ed into a list,
so you should always returned freshly consed lists, rather than
constants.  Second, \iter\ matches clauses by using |eq| on the first
symbol and |string=| on the subsequent ones, so the package of the
first symbol of a clause is relevant.  All of the clauses in this manual
have their first word in the \iter\ package.  
You can use the package system in the usual way to shadow
\iter\ clauses without replacing them.

%%% say more here, about the badness that only the first word of a
%%% clause is packagey.

\section{Non-portable Extensions to Iterate (Contribs)}
\label{contribs}

Currently, there is only one non-portable extension to iterate in the
distribution: iterate-pg. If you have made an extension that depends
on non-portable features, feel free to send them to |asf@boinkor.net|
for inclusion in the iterate distribution.

\subsection{An SQL query driver for iterate}

The pg package by Eric Marsden (see |http://cliki.net/pg|) provides an
interface to the PostgreSQL database. Using the \iterpg\ extension, it
is possible to handle the results of SQL queries using \iter.

This usage example should give you an idea of how to use it:

\begin{program}
(pg:with-pg-connection (c "somedb" "someuser")
  (iter (for (impl version date) in-relation "select * from version"
                                 on-connection *dbconn*)
        (collect version)))
\end{program}

To use the extension via |ASDF|, simply make your system depend on the
|iterate-pg| system instead of the |iterate| system. To load it
manually, use:

\begin{program}
  (asdf:oos 'asdf:load-op :iterate-pg)
\end{program}

%%%%%%%%%%%%%%%%%%%%%%%%%%%%%%%%%%%%%%%%%%%%%%%%%%%%%%%%%%%%%%%%
\section{Obtaining {\tt Iterate}}

\begin{sloppypar}
The information in this chapter is obsolete but included for
completeness's sake; Currently, the most up-to-date information on
\iter\ can be found at |http://boinkor.net/iterate.html|.
\end{sloppypar}

\begin{sloppypar}
\iter\ currently runs on Lisp Machines, and on
HP's, Sun3's and Sparcstations under Lucid.  
\iter\ source and binaries are available at the MIT AI Lab in the
subdirectories of |/src/local/lisplib/|.  The source file,
|iterate.lisp|, is also available for anonymous FTP in the directory
|/com/fpt/pub/| on the machine |TRIX.AI.MIT.EDU| (Internet number
128.52.32.6).  If you are unable to obtain |iterate| in one of these
ways, send mail to |jba@ai.mit.edu| and I will send you the source
file. 
\end{sloppypar}

\begin{sloppypar}
\iter\ resides in the |iterate| package (nickname |iter|).  Just say
\linebreak |(use-package :iterate)| to make all the necessary symbols
available. 
If a symbol is not exported, it appears in this manual with an
``|iterate::|'' prefix.
\end{sloppypar}

Send bug reports to |bug-iterate@ai.mit.edu|.  The |info-iterate|
mailing list will have notices of changes and problems; to have
yourself added, send mail to |info-iterate-request@ai.mit.edu|.


\medskip

\begin{flushleft}
 \bf Acknowledgements
\end{flushleft}
\smallskip

Richard Waters provided invaluable criticism which spurred me to improve
\iter\ greatly.  As early users, David Clemens, Oren Etzioni and Jeff
Siskind helped ferret out many bugs.

%\begin{theindex}
% The index files must be generated with the genindex program
% or from makeindex iterate-manual.idx which creates iterate-manual.ind.

%\baselineskip\normalbaselineskip
%\advance\baselineskip by -2pt

\input{iterate-manual.ind}

%\underline{Clauses}
%\input{iterate-manual.clindex}
%
%\indexspace
%\underline{Lisp}
%\input{iterate-manual.lispindex}

%\end{theindex}


\end{document}


% arch-tag: "c0f871a6-313c-11d8-abb9-000c76244c24"

%
%\indexspace
%\underline{Lisp}
%% Fix ``.'' spacing in lists
\documentstyle[12pt,aimemo]{article}

\setlength{\textwidth}{6.0in}
\setlength{\oddsidemargin}{0.25in}
\setlength{\evensidemargin}{0.25in}
\setlength{\topmargin}{0pt}
\setlength{\headsep}{24pt}
\setlength{\textheight}{8.5in}

\let\origpar=\par

% somehow, LaTex fucks with things so that the plain Tex \obeyspaces
% doesn't work.  This is the same, except we use an mbox containing
% a space instead of just a space.

\newcommand{\spc}{\mbox{ }}
\newcommand{\tAb}{\mbox{        }}
\def\obspaces{\catcode`\ =\active\catcode`\^^I=\active}
{\obspaces \global\let =\spc\global\let^^I=\tAb}

\newenvironment{program}{\medskip\samepage\tt\obeylines\obspaces}{\medskip}

% use | for \tt
\catcode`\|=13
\newif\ifvbar
\def|{\ifvbar
         \endgroup\vbarfalse
      \else 
         \begingroup\tt\vbartrue 
      \fi}



\newcommand{\lisp}{\tt}
\newcommand{\iter}{|iterate|}
\newcommand{\iterpg}{|iterate-pg|}
\newcommand{\setf}{|setf|}
\newcommand{\nil}{|nil|}
\newcommand{\nonnil}{non-\nil}
\newcommand{\opt}{{\lisp \&optional}}
\newcommand{\yields}{$\Rightarrow$}



\newbox\Dots
\setbox\Dots=\hbox{\small ...}
\renewcommand{\dots}{\copy\Dots}

% Here's a way to define an environment which will treat all
% paragraphs inside it in the same manner.  It's, as near as I can
% figure, the way that Latex does things.  Using \everypar won't work;
% apparently Latex resets that a lot.  What you have to do is: at the
% top of the file do \let\origpar=\par.  In the
% begin part of the environment, do the parshape (or whatever), then say
% \def\par{{\origpar}}.  The extra braces are necessary.   That's all
% you have to do; at the end of the environment things will return to
% normal automatically.  Why does this all work?  I have no idea.
% Another important thing: there should be a blank line (i.e. a \par)
% between the end of the last paragraph and the \end{environment}.

\newlength{\clindent}
\setlength{\clindent}{0.5in}
\newlength{\clparindent}
\setlength{\clparindent}{\parindent}

\newenvironment{clauses}{\advance\linewidth -\parindent
                        \hangindent\clindent\relax
                         \hangafter=0
                          \parindent=0pt
                          \def\par{{\origpar}}}{}

\newcommand{\cpar}{\hspace \clparindent}

\newcommand{\startitem}{\bigskip\pagebreak[3]}
\newcommand{\finishitem}{\smallskip}

\newcommand{\clause}[1]{\startitem\Clause{#1}\finishitem}
\newcommand{\Clause}[1]{\hbox{#1}}
\newcommand{\clindex}[1]{\index{{\lisp #1}}}
\newcommand{\clindexx}[2]{\index{{\lisp #1\dots #2}}}
\newcommand{\clindexxx}[3]{\index{{\lisp #1\dots #2\dots #3}}}
\newcommand{\clausex}[2]{\startitem\Clausex{#1}{#2}\finishitem}
\newcommand{\Clausex}[2]{\Clause{|#1| #2}\clindex{#1}}

\newcommand{\defvar}[1]{\startitem\deff{#1}{}{Variable}\finishitem}
\newcommand{\defunexpvar}[1]{\startitem\deffnoindex{iterate::#1}{}{Variable}
                                        \lispindex{#1}\finishitem}
\newcommand{\defun}[2]{\startitem\deff{#1}{#2}{Function}\finishitem}
\newcommand{\defmacro}[2]{\startitem\deff{#1}{#2}{Macro}\finishitem}
\newcommand{\defunx}[3]{\startitem\deffx{#1}{#2}{#3}{Function}\finishitem}
\newcommand{\defmacrox}[3]{\startitem\deffx{#1}{#2}{#3}{Macro}\finishitem}
\newcommand{\defunxx}[4]{\startitem\deffxx{#1}{#2}{#3}{#4}{Function}\finishitem}
\newcommand{\defmacroxx}[4]{\startitem\deffxx{#1}{#2}{#3}{#4}{Macro}\finishitem}

\newcommand{\deff}[3]{\deffnoindex{#1}{#2}{#3}\lispindex{#1}}
\newcommand{\deffnoindex}[3]{\hbox to \hsize{{\tt #1} {\it #2}\hfill [{\it #3}]}}
\newcommand{\deffx}[4]{\deff{#1}{#2}{#4}\moreargs{#1}{#3}}
\newcommand{\deffxx}[5]{\deffx{#1}{#2}{#3}{#5}\moreargs{#1}{#4}}

\newcommand{\moreargs}[2]{\hbox to \hsize{\phantom{\lisp #1} {\it #2}\hfill}}

\newcommand{\lispindex}[1]{\index{{\lisp #1}}}

 \newenvironment{note}[1]{\pagebreak[2]\bigskip
                          \hrule\smallskip\small
                         {\setlength{\parindent}{0pt}
                           \par{\bf #1:}}}{\smallskip\hrule\bigskip}



\makeindex

%%%%%%%%%%%%%%%%%%%%%%%%%%%%%%%%%%%%%%%%%%%%%%%%%%%%%%%%%%%%%%%%

%\thispagestyle{empty}

%\vspace*{2in}

%\title{The |Iterate| Manual}
%\author{Jonathan Amsterdam}

%\maketitle

%\thispagestyle{empty}

%\begin{center}
%\Large *** DRAFT ***
%\end{center}
%This manual is in beta-test. It will become an AI Memo in a couple of
%months.  I would appreciate any comments or
%corrections.  You can reach me at email address |jba@ai.ai.mit.edu|.

\begin{document}

\def\writtenby{Jonathan Amsterdam}
\def\contractno{This report describes research done at the Artificial
Intelligence Laboratory of the Massachusetts Institute of Technology.
Support for the laboratory's artificial intelligence research is
provided in part by the Advanced Research Projects Agency of the
Department of Defense under Office
of Naval Research contract N00014-85-K-0124.}

\AIMEMO{1236}{\Large The Iterate Manual \\[1in]}
{This is the manual for version 1.4 of \iter, a powerful iteration
macro for Common  
Lisp. \iter\ is similar to {\tt loop} but provides numerous additional
features, is well integrated with Lisp, and is extensible. 
\vfill }


% use ~ for \em
\catcode`\~=13
\newif\ifem
\def~{\ifem
         \/\endgroup\emfalse
      \else 
         \begingroup\em\emtrue 
      \fi}



%\pagebreak
%\setcounter{page}{1}

%\thispagestyle{empty}


\tableofcontents

\pagebreak
\pagestyle{headings}

\section{Introduction}

\begin{sloppypar}
This manual describes
\iter, a powerful iteration facility for Common Lisp.
\iter\ provides abstractions for many common iteration
patterns and allows for the definition of additional patterns.  \iter\ is
a macro that expands into ordinary Lisp at
compile-time, so it is more efficient than higher-order
functions like |map| and |reduce.|  
While it is similar to |loop|, \iter\ offers a more Lisp-like syntax
and enhanced extensibility.
(For a more complete
comparison of \iter\ with other iteration constructs, see MIT AI Lab Working
Paper No. 324, ~Don't Loop, Iterate.~)
\end{sloppypar}

An \iter\ form consists of the symbol |iter|\footnote{You can also use
|iterate|, but |iter| is preferred because it avoids potential conflicts with
possible future additions to Common Lisp, and because it saves
horizontal space when writing code.}
followed by one or more
forms, some of which may be \iter\ 
~clauses.~  Here is a simple example of \iter\ which collects the
numbers from 1 to 10 into a list, and returns the list.  The return
value is shown following the arrow.
\begin{program}
(iter (for i from 1 to 10)
      (collect i))          \yields (1 2 3 4 5 6 7 8 9 10)
\end{program}
This form contains two clauses: a |for| clause that steps the
variable |i| over the integers from 1 to 10, and a |collect|
clause that accumulates its argument into a list.  With a few
exceptions, all \iter\ clauses have the same format: alternating
symbols (called ~keywords~) and expressions (called
~arguments~).  The syntax and terminology are those of Common
Lisp's keyword lambda lists.  One difference is that \iter's keywords
do not have to begin with a colon---though they may, except for the
first symbol of a clause.  So you can also write  |(for i :from
1 :to 10)| if you prefer.

Any Lisp form can appear in the body of an \iter, where it will have
its usual meaning.  \iter\ walks the entire body, expanding macros,
and recognizing clauses at any level.  This example collects all the
odd numbers in a list:
\begin{program}
(iter (for el in list)
      (if (and (numberp el) (oddp el))
          (collect el)))
\end{program}

There are clauses for iterating over numbers, lists, arrays and other
objects, and for
collecting, summing, counting, maximizing and other useful operations.
\iter\ also supports the creation of new variable bindings, stepping
over multiple sequences at once, destructuring, and compiler
declarations of variable types.  The following example illustrates
some of these features: 
\begin{program}
(iter (for (key . item) in alist)
         (for i from 0)
         (declare (fixnum i))
         (collect (cons i key)))
\end{program}

This loop takes the keys of an alist and returns a new alist
associating the keys with their positions in the original list. The
compiler declaration for |i| will appear in the generated code
in the appropriate place.

%%%%%%%%%%%%%%%%%%%%%%%%%%%%%%%%%%%%%%%%%%%%%%%%%%%%%%%%%%%%%%%%
\section{Clauses}

Most of \iter's clauses will be familiar to |loop| programmers.  
(|loop| is an iteration macro that has been incorporated into Common
Lisp.  Seen Guy Steele's {\em Common Lisp, 2nd Edition}.)
In
nearly all cases they behave the same as their |loop| counterparts, so
a |loop| user can switch to \iter\ with little pain (and much gain).

All clauses with the standard keyword-argument syntax consist of two
parts: a ~required~ part, 
containing keywords that must be present and in the right order; and
an ~optional~ part, containing keywords that may be omitted and,
if present, may occur in any order.  In the descriptions below, the
parts are separated by the Lisp lambda-list keyword \opt.


\subsection{Drivers}

An iteration-driving clause
conceptually causes the iteration to go forward.  Driver clauses in
\iter\ allow iteration over numbers, lists, vectors, hashtables, packages,
files and streams.  Iteration-driving clauses must
appear at the top level of an \iter\ form; they cannot be nested
inside another clause.  The driver variable is updated at the point
where the driver clause occurs.  Before the clause is executed for the
first time, the value of the variable is undefined.

%Also, regardless of where the driver clause
%appears in the body, the driver variable is stepped at the top of the
%loop; hence it is stylistically preferable, though
%not required, to place driver clauses at the beginning of the \iter.

Multiple drivers may appear in a single \iter\ form, in which case all
of the driver variables are updated each time through the loop, in the
order in which the clauses appear.  The first driver to terminate will
terminate the entire loop.

In all cases, the value of the driver variable on exit from the loop,
including within the epilogue code (see the |finally| clause), is
undefined.

All the parameters of a driver clause are evaluated once, before the
loop begins.  Hence it is not possible to change the bounds or other
properties of an iteration by side-effect from within the loop.

With one exception, driver clauses begin with the word |for| (or
the synonym |as|) and mention an iteration variable, which is
given a binding within the \iter\ form.
The exception is |repeat|, which just executes a loop a
specified number of times:

\begin{clauses}

\clausex{repeat}{~n~}
Repeats the loop ~n~ times.  For example:
\begin{program}
(iter (repeat 100)
      (print "I will not talk in class."))
\end{program}
\cpar If $n \leq 0$, the
loop will never be executed.  If ~n~ is not an integer, the
actual number of executions will be $\lceil n \rceil$.

\end{clauses}

\pagebreak[3]

\subsubsection{Numerical Iteration}

\begin{clauses}
\clausex{for}{~var~ |\&sequence|} 
The general form for iterating over a sequence of numbers requires a
variable and, optionally, one or more keywords that provide the bounds
and step size of the iteration. The |\&sequence|
lambda-list keyword is a shorthand for these sequence keywords.  They are:
|from|, |upfrom|, |downfrom|, |to|, |downto|, |above|, |below| and |by|.
|from| provides the starting value for ~var~
and defaults to zero. |to| provides a final value and implies that the
successive values of ~var~ will be increasing; |downto| implies that
they will be decreasing.  The loop terminates when ~var~ passes
the final value (i.e. becomes smaller or larger than it, depending on
the direction of iteration); in other  words, the loop body
will never be 
executed for values of ~var~ past the final value.
|below| and |above| are similar to |to| and
|downto|, except that the loop terminates when ~var~ equals or passes
the final value.

\cpar If no final value is specified, the variable will be stepped
forever.  Using |from| or |upfrom| will result in increasing
values, while |downfrom| will give decreasing values.

\cpar On each iteration,
~var~ is incremented or decremented by the value of the sequence
keyword |by|, which defaults to 1.  It should always be a positive
number, even for downward iterations.

\cpar In the following examples, the sequence of numbers generated is shown
next to the clause.

\begin{program}
(for i upfrom 0) \yields\ 0 1 2 \ldots
(for i from 5)   \yields\ 5 6 7 \ldots    ; either from or upfrom is okay
(for i downfrom 0) \yields\ 0 -1 -2 \ldots
(for i from 1 to 3) \yields\ 1 2 3
(for i from 1 below 3) \yields\ 1 2
(for i from 1 to 3 by 2) \yields\ 1 3
(for i from 1 below 3 by 2) \yields\ 1
(for i from 5 downto 3) \yields\ 5 4 3
\end{program}

\end{clauses}

\subsubsection{Sequence Iteration}

There are a number of clauses for iterating over sequences.  In all of
them, the argument following |for| may be a list instead of a
symbol, in which case destructuring is performed.  See section
\ref{destructuring}. 

\begin{clauses}

\clause{|for| ~var~ |in| ~list~ \opt\ |by| ~step-function~}
\clindexx{for}{in}
~var~ is set to successive elements of list.  
~step-function,~ which 
defaults to |cdr|, is used to obtain the next sublist. 

\clause{|for| ~var~ |on| ~list~ \opt\ |by| ~step-function~}
\clindexx{for}{on}
~var~ is set to successive sublists of list. 
~step-function~ (default |cdr|) is used as in |for\dots in|.

\end{clauses}

\medskip

These two clauses use |atom| to test for the end
of a list.  Hence, given a list whose final |cdr| is not \nil,
they will silently ignore the last |cdr|.  Other choices are
|endp|,  which would signal an error, and |null|, which
would probably result in an error somewhere else.  If you wish to use
an end-test other than |atom|, set the variable
|iterate::*list-end-test*|\lispindex{*list-end-test*} to the name of
the desired function. 

\begin{clauses}
\clause{|for| ~var~ |in-vector| ~vector~ |\&sequence|}
\clindexx{for}{in-vector}
~var~ takes on successive elements from ~vector.~   The vector's
fill-pointer is observed.  Here and in subsequent clauses, the |\&sequence|
keywords include |with-index|, which takes a symbol as argument
and uses it for the index variable instead of an internally generated
symbol.  The other |\&sequence| keywords behave as in numerical
iteration, except that the default iteration bounds are the bounds of
the vector.  E.g. in |(for i in-vector v downto 3)|,
|i| will start off being bound to the last element in |v|,
and will  
be set to preceding elements down to and including the element with
index 3.

\clause{|for| ~var~ |in-sequence| ~seq~ |\&sequence|}
\clindexx{for}{in-sequence}
This uses Common Lisp's generalized sequence functions, |elt|
and |length|, to obtain elements and determine the length of
~seq.~   Hence it will work for any sequence, including lists, and will
observe the fill-pointers of vectors.

\clause{|for| ~var~ |in-string| ~string~ |\&sequence|}
\clindexx{for}{in-string}
~var~ is set to successive characters of ~string.~

\startitem
\Clause{|for| ~var~ |index-of-vector| ~vector~ |\&sequence|}
\clindexx{for}{index-of-vector}
\Clause{|for| ~var~ |index-of-sequence| ~sequence~ |\&sequence|}
\clindexx{for}{index-of-sequence}
\Clause{|for| ~var~ |index-of-string| ~string~ |\&sequence|}
\clindexx{for}{index-of-string}
\finishitem
~var~ is set to successive indices of the sequence.
These clauses avoid the overhead of accessing the sequence elements
for those applications where they do not need to be examined, or are
examined rarely.  They admit all the optional keywords of the other
sequence drivers except the (redundant) |with-index| keyword.

\clause{|for| ~(key value)~ |in-hashtable| ~table~}
\clindexx{for}{in-hashtable}

~key~ and ~value,~ which must appear as shown in a list
and may be destructuring templates, are set to the keys and values of
~table~.  If ~key~ is |nil|, then the hashtable's keys will be ignored;
similarly for ~value~.
The order in which 
elements of ~table~ will be retrieved is unpredictable.

\clause{|for| ~var~ |in-package| ~package~ \opt\ |external-only| ~ext~}
\clindexx{for}{in-package}
Iterates over all the symbols in ~package,~ or over only the
external symbols if ~ext~ is specified and non-|nil|.  ~ext~ is not
evaluated.  The same symbol may appear more than once.

\clause{|for| ~var~ |in-packages| ~packages~ \opt\ |having-access| ~symbol-types~}
\clindexx{for}{in-packages}
Iterates over all the symbols from the list of packages denoted by the
descriptor ~packages~ and having accessibility (or visibility) given by
~symbol-types~.  This defaults to the list |(:external :internal :inherited)|
and is not evaluated.  ~var~ must be a list of up to three variables: in each
iteration, these will be set to a symbol, its access-type and package (as per
|with-package-iterator| in ANSI-CL).  The same symbol may appear more than
once.

\end{clauses}

\begin{clauses}

\clause{|for| ~var~ |in-file| ~name~ \opt\ |using| ~reader~}
\clindexx{for}{in-file}
Opens the file ~name~ (which may be a string or pathname) for input,
and iterates 
over its contents. ~reader~ defaults to |read|, so by default ~var~
will be bound to the successive forms in the file.  The \iter\ body is
wrapped in an unwind-protect to ensure that the file is closed no
matter how the \iter\ is exited.

\clause{|for| ~var~ |in-stream| ~stream~ \opt\ |using| ~reader~}
\clindexx{for}{in-stream}
Like |for\dots in-file|, except that ~stream~ should be an existing
stream object that supports input operations.

\end{clauses}

\subsubsection{Generalized Drivers}

These are primarily useful for writing drivers that can also be used
as generators (see section \ref{generators}, below).
 
\begin{clauses}

\clause{|for| ~var~ |next| ~expr~}
\clindexx{for}{next}

~var~ is set to ~expr~ each time through the loop.  Destructuring is
performed.  When the clause is used as a generator, ~expr~ is the code
that is executed when |(next ~var~)| is 
encountered (see section \ref{generators}, below).
~expr~ should compute the first value for ~var~, as well
as all subsequent values, and is responsible for terminating the loop.
For compatibility with future versions of \iter, this termination
should be done with |terminate|\clindex{terminate}, which can be
considered a synonym for |finish| (see section \ref{control-flow}).

\cpar As an example, the following clauses are equivalent to |(for
i from 1 to 10)|:
\begin{program}
(initially (setq i 0))
(for i next (if (> i 10) (terminate) (incf i)))
\end{program}

\clause{|for| ~var~ |do-next| ~form~}
\clindexx{for}{do-next}
~form~ is evaluated each time through the loop.  Its value is ~not~
set to ~var~; that is ~form's~ job. ~var~ is only present so that
\iter\ knows it is a driver variable.  \linebreak |(for ~var~ next
~expr~)| is equivalent to |(for ~var~ do-next (dsetq ~var~ ~expr~))|.
(See section \ref{destructuring} for an explanation of |dsetq|.) 

\end{clauses}

\subsubsection{Generators}
\label{generators}

In all of the above clauses, the driver variable is updated on each
iteration.
Sometimes it is desirable to have greater control over updating.  
For instance, consider the problem of associating numbers, in
increasing order and with no gaps, with the
\nonnil\ elements of a list.  One obvious first pass at writing this is:

\begin{program}
(iter (for el in list)
      (for i upfrom 1)
      (if el (collect (cons el i))))
\end{program}
But on the list |(a b nil c)| this produces |((a . 1) (b . 2) (c .
4))| instead of the desired |((a . 1) (b . 2) (c . 3))|.  The problem
is that |i| is incremented each time through the loop, even when |el|
is |nil|.  

The problem could be solved elegantly if we could step |i| only when
we wished to.  This
can be accomplished for any \iter\ driver by writing |generate| (or
its synonym |generating|)
instead of |for|.  Doing so produces a ~generator~---a driver whose
values are yielded explicitly.  To obtain the next value of a
generator variable ~v~, write \linebreak |(next ~v~)|.  The value of a |next|
form is the next value of ~v~, as determined by its associated driver
clause.  |next| also has the side-effect of updating ~v~ to that
value.  If there is no next value, |next| will terminate the loop,
just as with a normal driver.


Using generators, we can now write our example like this:
\begin{program}
(iter (for el in list)
      (generate i upfrom 1)
      (if el (collect (cons el (next i)))))
\end{program}
Now |i| is updated only when |(next i)| is executed, and this occurs
only when |el| is \nonnil.

To better understand the relationship between ordinary drivers and
generators, observe that we can rewrite an ordinary driver using its
generator form immediately followed by |next|, as this example shows:
\begin{program}
(iter (generating i from 1 to 10)
      (next i)
      ...)
\end{program}
Provided that the loop body contains no |(next i)| forms, this will
behave just as if we had written |(for i from 1 to 10)|.

We can still refer to a driver variable ~v~ without using |next|; in
this case, its value is that given to it by the last evaluation of
|(next ~v~)|.  Before |(next ~v~)| has been called the first time, the
value of ~v~ is undefined.  

This semantics is more flexible than
one in which ~v~ begins the loop bound to its first value and calls
of |next| supply subsequent values, because it means the loop will not
terminate too soon if the generator's sequence is empty.  For
instance, consider the following code, which tags \nonnil\ elements of
a list using a list of tags, and also counts the null elements.
(We assume there are at least as many tags as \nonnil\ elements.)
\begin{program}
(let* ((counter 0)
       (tagged-list (iter (for el in list)
                          (generating tag in tag-list) 
                          (if (null el)
                              (incf counter)
                              (collect (cons el (next tag)))))))
  ...)
\end{program}
It may be that there are just as many tags as non-null elements of
|list|.  If all the elements of |list| are null, we still want the
counting to proceed, even though |tag-list| is \nil.  If |tag| had to be
assigned its first value before the loop begins, we would have had to
terminate the loop before the first iteration, since when |tag-list|
is \nil, |tag| 
has no first value.  With the existing semantics, however, |(next
tag)| will never execute, so the iteration will cover all the elements of
|list|. 
        
When the ``variable'' of a driver clause is actually a destructuring
template containing several variables, all the variables are eligible
for use with |next|.  As before, |(next ~v~)| evaluates to ~v's~ next
value; but the effect is to update all of the template's variables.
For instance, the following code will return the list |(a 2 c)|.
\begin{program}
(iter (generating (key . item) in '((a . 1) (b . 2) (c . 3)))
      (collect (next key))
      (collect (next item)))
\end{program}

Only driver clauses with variables can be made into generators.  This
includes all clauses mentioned so far except for |repeat|.  It does
~not~ include |for\dots previous|, |for\dots =|,
|for\dots initially\dots then| or |for\dots first\dots then| (see
below). 

\subsubsection{Previous Values of Driver Variables}

Often one would like to access the value of a variable on a previous
iteration.  \iter\ provides a special clause for accomplishing this.

\begin{clauses}

\clause{|for| ~pvar~ |previous| ~var~ \opt\ |initially| ~init~
|back| ~n~}
\clindexx{for}{previous}
Sets ~pvar~ to the previous value of ~var,~ which should be a driver
variable, a variable from another |for\dots previous| clause, or a
variable established by a |for\dots =|, 
|for\dots initially\dots then| or |for\dots first\dots then| clause
(see section \ref{setting}).
Initially, ~pvar~ is given the value ~init~ (which defaults to \nil).  
The ~init~ expression will be moved outside the loop body, so it
should not depend on anything computed within the loop.
~pvar~ retains the value of ~init~ until ~var~ is set to its second
value, at which point ~pvar~ is set to ~var's~ first value; and so on.  

\cpar The
argument ~n~ to |back| must be a constant, positive integer, and
defaults to 1.  It determines how many iterations back ~pvar~ should
track ~var~.  For example, when ~n~ is 2, then ~pvar~ will be assigned
~var's~ first value when ~var~ is set to its third value.

\cpar A |for\dots previous| clause may occur after or before its
associated driver clause. |for\dots previous| works with generators as
well as ordinary drivers.

\pagebreak[3]

\cpar Example:
\begin{program}
(iter (for el in '(1 2 3 4))
      (for p-el previous el)
      (for pp-el previous p-el initially 0)
      (collect pp-el))
\end{program}
This evaluates to |(0 0 1 2)|.  It could have been written more
economically as 
\begin{program}
(iter (for el in '(1 2 3 4))
      (for pp-el previous el back 2 initially 0)
      (collect pp-el))
\end{program}


\end{clauses}

\subsection{Variable Binding and Setting}
\label{setting}

Several clauses exist for establishing new variable bindings or for
setting variables in the loop.  They all support destructuring.

\begin{clauses}

\clausex{with}{~var~ \opt\ |=| ~value~}
Causes ~var~ to be bound to value before the loop body is entered.
If ~value~ is not supplied, ~var~ assumes a default
binding, which will be 
\nil\ in the absence of declarations.  Also, if ~value~ is not
supplied, no destructuring is performed; instead, ~var~ may be a list of
symbols, all of which are given default 
bindings.  If ~value~ is supplied, ~var~ is bound to it, with
destructuring.  

\cpar Because |with|
creates bindings whose scope includes the entire \iter\ form, it is
good style to put all |with| clauses at the beginning.

\cpar Successive occurrences of |with| result in sequential
bindings (as with 
|let*|).  There is no way to obtain parallel bindings; see
section \ref{bindings} for a rationale.  


\clause{|for| ~var~ |=| ~expr~}
\clindexx{for}{=}
On each iteration, ~expr~ is evaluated and ~var~ is set
to its value. 

\cpar This clause may appear to do the same thing as |for\dots next|.
In fact, they are quite different.  |for\dots =| provides only three
services: it sets up a binding for ~var~, sets it to ~expr~ on each
iteration, and makes it possible to use |for\dots previous| with
~var~.  |for\dots next| provides these services in addition to the
ability to turn the driver into a generator.
%Also, the code which sets ~var~ appears in the loop body in the same
%place as the |for\dots =| clause; the code for |for\dots next| appears
%at the top of the loop, as with other drivers (except when being used
%as a generator).

\clause{|for| ~var~ |initially| ~init-expr~ |then| ~then-expr~}
\clindexxx{for}{initially}{then}
Before the loop begins, ~var~ is set to ~init-expr;~ on all
iterations after the first it is set to ~then-expr.~  This clause
must occur at top-level.  ~init-expr~ will be moved outside the loop
body and ~then-expr~ will be moved to the end of the loop body, so
they are subject to code motion problems (see section
\ref{code-movement}). 

\cpar This clause may appear to be similar to |for\dots next|, but in
fact they differ significantly.  |for\dots initially\dots then| is
typically used to give ~var~ its first value before the loop begins,
and subsequent values on following iterations.  This is incompatible
with generators, whose first value and subsequent values must all be
computed by |(next ~var~)|.  Also, the update of ~var~ in 
|for\dots initially\dots then| does not occur at the location of the
clause. 
Use |for\dots initially\dots then| for
one-shot computations where its idiom is more convenient, but use
|for\dots next| for extending \iter\ with new drivers (see section
\ref{extend}).

\clause{|for| ~var~ |first| ~first-expr~ |then| ~then-expr~}
\clindexxx{for}{first}{then}
The first time through the loop, ~var~ is set to ~first-expr;~ on
subsequent iterations, it is set to ~then-expr.~  This differs from
|for\dots initially| in that ~var~ is set to ~first-expr~
inside the loop body, 
so ~first-expr~ may depend on the results of other clauses.  For
instance, 
\begin{program}
(iter (for num in list)
      (for i first num then (1+ i))
      ...)
\end{program}
will set |i| to the first element of |list| on the first
iteration, whereas 
\begin{program}
(iter (for num in list)
      (for i initially num then (1+ i))
      ...)
\end{program}
is probably erroneous; |i| will be bound to |num|'s
default binding (usually \nil) for the first iteration.

\end{clauses}

\begin{note}{Compatibility note}
|loop|'s |for\dots =| works like \iter's, but |loop| used the syntax
|for\dots =\dots then| to mean |for\dots initially\dots then|.  It was
felt that these two operations were sufficiently different to warrant
different keywords.

Also, the |for| in the above three clauses is misleading,
since none is true driver (e.g. none has a corresponding |generate|
form).  |setting| would have been a better choice, but |for| was used
to retain some compatibility with |loop|.  
\end{note}


%%%%%%%%%%%%%%%%%%%%%%%%%%%%%%%%
\subsection{Gathering Clauses}
Many of \iter's clauses accumulate values into a variable, or set a
variable under certain conditions.  At the end of the loop, this
variable contains the desired result.  All these clauses have an
optional |into| keyword, whose argument should be a symbol.  If the
|into| keyword is not supplied, the accumulation variable will be
internally generated and its value will be returned at the end of the
loop; if a variable is specified, that variable is used for the
accumulation, and is not returned as a result---it is up to the user
to return it explicitly, in the loop's epilogue code (see |finally|).
It is safe to examine the accumulation variable during the loop, but
it should not be modified.

These clauses all begin with a verb.  When the verb does
not conflict with an existing Common Lisp function, then it may be
used in either its infinitival or present-participle form (e.g. |sum|,
|summing|).  However, when there is a conflict with Common Lisp, only
the present-participle form may be used (e.g. |unioning|).  This is to
prevent \iter\ clauses from clashing with Common Lisp
functions. 

%although these clauses are described as ``producing a value,'' it is a
%mistake to think of the lisp list representing the clause as a
%value-producing form in the usual way.  clauses may legally be written
%where a value is expected, e.g. |(setq x (sum i))|, but the
%lisp value of a clause in such a context is undefined.

\subsubsection{Reductions}

~Reduction~ is an extremely common iteration pattern in which
the results of successive applications of a
binary operation are accumulated.
For example, a loop that computes the sum of the
elements of a list is performing a reduction with the addition
operation.  This could be written in Common Lisp as  |(reduce \#'+
list)| or with \iter\ as
\begin{program}
(iter (for el in list)
      (sum el))
\end{program}

\begin{clauses}

\clausex{sum}{~expr~ \opt\ |into| ~var~}
Each time through the loop, ~expr~ is evaluated and added to a
variable, which is bound initially to zero.  If ~expr~ has a type,
it is ~not~ used as the type of the sum variable, which is always
|number|.  To get the result variable to be of a more specific
type, use an explicit variable, as in
\begin{program}
(iter (for el in number-list)
      (sum el into x)
      (declare (fixnum x))
      (finally (return x)))
\end{program}

\clausex{multiply}{~expr~ \opt\ |into| ~var~}
Like |sum|, but the initial value of the result variable is $1$,
and the variable is updated by multiplying ~expr~ into it.

\clausex{counting}{~expr~ \opt\ |into| ~var~}
~expr~ is evaluated on each iteration.  If it is \nonnil, the
accumulation variable, initially zero, is incremented.

\startitem
\Clausex{maximize}{~expr~ \opt\ |into| ~var~}
\Clausex{minimize}{~expr~ \opt\ |into| ~var~}
\finishitem
~expr~ is evaluated on each iteration and its extremum (maximum or
minimum) is stored in the accumulation variable.  If ~expr~ is never
evaluated, then the result is \nil\ (if the accumulation variable is
untyped) or $0$ (if it has a numeric type).

\clausex{reducing}{~expr~ |by| ~func~ \opt\ |initial-value| ~init-val~ |into| ~var~}
This is a general way to perform reductions. ~func~ should be a
function of two arguments, the first of which will be the value
computed so far and
the second of which will be the value of ~expr~.  It should return the
new value.
|reducing| is roughly equivalent to the
Common Lisp |(reduce ~func~ ~sequence~ :key ~expr-function~)|, where
~expr-function~ is used to derive values from the successive elements
of ~sequence~.

\cpar If the |reducing| clause is never executed, the result is
undefined.

\cpar It is not necessary to provide an
initial value, but better code can be generated if one is supplied.
Regardless of its location in the \iter\ body, ~init-val~ will be
evaluated before the loop is entered, so it should not depend on any
value computed inside the \iter\ form.  

%\cpar if a ~var~ is not specified, you can get \iter\ to declare the
%type of the internal variable by putting a |the| expression around
%~func.~  see section \ref{types}.

\end{clauses}

%\begin{note}{implementation note}
%in principle, |maximize| and |minimize| can be thought of
%as reductions where the initial value is the smallest (or largest)
%value that the accumulation variable can assume.  because lisp's
%bignums can represent arbitrary integers, these clauses cannot be
%implemented as reductions in general.  if, however, the type of 
%~expr~ or ~var~ can be determined to be a fixnum
%or a float, \iter\ will implement the clause as a true reduction,
%using one of the constants |most-negative-fixnum|, |%most-positive-fixnum|, 
%|most-negative-short-float|, etc. as appropriate.
%\end{note}

\subsubsection{Accumulations}

All the predefined accumulation clauses add values to a sequence.  If
the sequence is a list, they all 
have the property that the partial list is kept in the correct order
and available for inspection at any point in the loop.

\begin{clauses}

\clausex{collect}{~expr~ \opt\ |into| ~var~ |at| ~place~ |result-type| ~type~}
Produces a sequence of the values of ~expr~ on each iteration. ~place~
indicates where the next value of ~expr~ is added to the list and may
be one of the symbols |start|, |beginning| (a synonym for |start|) or
|end|.  The symbol may be quoted, but need not be.  The default is
|end|.  For example,
\begin{program}
(iter (for i from 1 to 5)
      (collect i))
\end{program}
produces |(1 2 3 4 5)|, whereas
\begin{program}
(iter (for i from 1 to 5)
      (collect i at beginning))
\end{program}
produces |(5 4 3 2 1)| (and is likely to be faster in most Common Lisp
implementations).

\cpar If ~type~ is provided, it should be a subtype of |sequence|.
The default is |list|.  Specifying a type other than |list| will
result in |collect| returning a sequence of that type.  ~However,~ the
type of the sequence being constructed when inside the loop body is
undefined when a non-|list| type is specified.  (As with ~place~,
quoting ~type~ is optional.)

\clausex{adjoining}{~expr~ \opt\ |into| ~var~ |test| ~test~ |at| ~place~ |result-type| ~type~}
Like |collect|, but only adds the value of ~expr~ if it is not
already present.  ~test,~ which defaults to |\#'eql|, is
the test to be used with |member|.

\startitem
\Clausex{appending}{~expr~ \opt\ |into| ~var~ |at| ~place~}
\Clausex{nconcing}{\ ~expr~ \opt\ |into| ~var~ |at| ~place~}
\Clausex{unioning}{\ ~expr~ \opt\ |into| ~var~ |test| ~test~ |at| ~place~}
\Clausex{nunioning}{~expr~ \opt\ |into| ~var~ |test| ~test~ |at| ~place~}
\finishitem
These are like |collect|, but behave like the Common Lisp functions
|append|, |nconc|, |union| or |nunion|.  
As in Common Lisp, they work only on lists.  Also as in Common Lisp,
|unioning| and |nunioning| assume that the value of ~expr~ contains no
duplicates. 

\clausex{accumulate}{~expr~ |by| ~func~ \opt\ |initial-value| ~init-val~ |into| ~var~}
This is a general-purpose accumulation clause. ~func~ should be a
function of two arguments, the value of ~expr~ and the value
accumulated so far in the iteration, and it should return the updated
value.  If no initial value is supplied, \nil\ is used.  

%\cpar If a ~var~ is not specified, you can get \iter\ to declare the
%type of the internal variable by putting a |the| expression around
%~func.~  see section \ref{types}.

\cpar The differences between |accumulate| and |reducing| are slight.
One difference is that the functions take their arguments in a
different order.  Another is that in the absence of ~init-val~,
|accumulate| will use \nil, whereas |reducing| will generate different
code that avoids any dependence on the initial value.
The reason for having both clauses is that one usually
thinks of reductions (like |sum|) and accumulations (like |collect|)
as different beasts.

\end{clauses}


\subsubsection{Finders}

A ~finder~ is a clause whose value is an expression that meets some
condition.

\begin{clauses}

\clause{|finding| ~expr~ |such-that| ~test~ \opt\ |into| ~var~ |on-failure| ~failure-value~}
\clindexx{finding}{such-that}
If ~test~ (which is an expression) ever evaluates to \nonnil, the loop
is terminated, the 
epilogue code is run and the value of ~expr~ is returned.  Otherwise,
\nil\ (or ~failure-value,~ if provided) is returned.  If ~var~ is
provided, it will have either the \nonnil\ value of ~expr~ or
~failure-value~ when the epilogue code is run.

\cpar As a special case, if the ~test~ expression is a sharp-quoted
function, then it is applied to ~expr~ instead of being simply
evaluated.  E.g. |(finding x such-that \#'evenp)| is equivalent to
|(finding x such-that (evenp x))|.  

%\cpar although ~test~ need have nothing to do with ~%expr~ as in
%|(finding j such-that (> i 3))|, it usually 
%will: |(finding (length el) such-that (oddp (length el)))|.  to
%avoid performing the |length| computation twice, you could write
%|(finding (length el) such-that \#'oddp)| or |(finding (length
%el) such-that 'oddp)|; for these cases, \iter\ generates code that
%executes ~expr~ only once.  the code for |\#'oddp|
%is slightly different from that for {\lisp 'oddp}; see the discussion
%under {\lisp for\dots in} and {\lisp for\dots on}.

\cpar |On-failure| is a misnomer. Because it is always evaluated, it behaves
more like the default third argument to the |gethash| function. As a result,
|on-failure (error "Not found")| makes no sense. Instead, the clauses |leave|
or |thereis| can be used in conjunction with |finally| as follows:
\begin{program}
(iter (for x in '(1 2 3))
      (if (evenp x) (leave x))
      (finally (error "not found")))
\end{program}

\cpar This clause may appear multiple times when all defaults are
identical. It can also be used together with either |always|/|never| or
|thereis| if their defaults match. More specifically, |on-failure nil| is
compatible with |thereis|, while |on-failure t| is compatible with |always|
and |never| clauses.
\begin{program}
(iter (for i in '(7 -4 2 -3))
      (if (plusp i)
	  (finding    i  such-that (evenp i))
          (finding (- i) such-that (oddp i))))
\end{program}

\startitem
\Clause{|finding| ~expr~ |maximizing| ~m-expr~ \opt\ |into| ~var~}
\clindexx{finding}{maximizing}
\Clause{|finding| ~expr~ |minimizing| ~m-expr~ \opt\ |into| ~var~}
\clindexx{finding}{minimizing}
\finishitem
Computes the extremum (maximum or minimum) value of ~m-expr~ over all
iterations, and returns the value of ~expr~ corresponding to the
extremum.  ~expr~ is evaluated inside the loop at the time the new
extremum is established.  If ~m-expr~ is never evaluated (due to, for
example, being embedded in a conditional clause), then the returned
value depends on the type, if any, of ~expr~ (or ~var,~ if
one is supplied).  If there is no type, the returned
value will be \nil; if the type is numeric, the returned value will be
zero.

\cpar For these two clauses, ~var~ may be a list of two
symbols; in that case, the first is used to record ~expr~ and
the second, ~m-expr.~  

\cpar As with |finding\dots such-that|, if ~m-expr~ is a sharp-quoted
function, then it is called on ~expr~ instead of being evaluated.

\end{clauses}

\subsubsection{Boolean Tests}

\begin{clauses}

\clausex{first-iteration-p}{}
Returns |t| in the first cycle of the loop, otherwise \nil.

\clausex{first-time-p}{}
Returns |t| the first time the expression is evaluated, and then \nil\ forever.
This clause comes handy when printing (optional) elements separated
by a comma:

\begin{program}
(iter (for el in '(nil 1 2 nil 3))
      (when el
        (unless (first-time-p)
          (princ ", "))
        (princ el)))
\end{program}
produces |"1, 2, 3"|.

\end{clauses}

\subsubsection{Aggregated Boolean Tests}

\begin{clauses}

\clausex{always}{~expr~}
If ~expr~ ever evaluates to
\nil, then \nil\ is immediately returned; the epilogue code is not
executed.  If ~expr~ never evaluates to \nil, the epilogue code
is executed and the last value of ~expr~ (or |t| if ~expr~ was never
evaluated) is returned (whereas |loop| would constantly return |t|).

% mention last evaluated clause when multiple always clauses?

\clausex{never}{~expr~}
Like |(always (not ~expr~))|, except it does not influence the last
value returned by a possible other |always| clause. That is,
\begin{program}
(iter (repeat 2)
      (always 2)
      (never nil)) \yields 2 ; not t
\end{program}

\clausex{thereis}{~expr~}
If ~expr~ is ever \nonnil,
its value is immediately returned without running epilogue code.
Otherwise, the epilogue code is performed and \nil\ is returned.

This clause cannot be used together with |always| or |never|, because their
defaults are opposed (similarly, |(loop always 3 thereis nil)| refuses to
compile in some implementations of |loop|).

\end{clauses}


\subsection{Control Flow}
\label{control-flow}
Several clauses can be used to alter the usual flow of control in a loop.

Note: the clauses of this and subsequent sections don't adhere to \iter's
usual syntax, but instead use standard Common Lisp syntax.  Hence the
format for describing syntax subsequently is
like the standard format used in the Common Lisp manual, not like the
descriptions of clauses above.

\begin{clauses}

\clausex{finish}{}
Stops the loop and runs the epilogue code.

%for example:
%
%\begin{program}
%(iter (with answer = nil)
%         (initially (make-a-mess))
%         (for i from 1 to 10)
%         (when (correct? i)
%           (setq answer i)
%           (finish))
%         (finally (cleanup)))
%\end{program}
%
%this code will execute |cleanup| whether or not the test |(correct?
%i)| ever succeeds. 
%the (more elegant) formulation,
%\begin{program}
%(iter (initially (make-a-mess))
%         (for i from 1 to 10)
%         (finding i such-that (correct? i))
%         (finally (cleanup)))
%\end{program}
%would not execute |cleanup| if |(correct? i)| succeeded; it
%would do an immediate return.

\clausex{leave}{\opt\ ~value~}
Immediately returns ~value~ (default \nil) from the current \iter\
form, skipping the epilogue code.  Equivalent to using |return-from|.

\clausex{next-iteration}{}
Skips the remainder of the loop body and begins the next iteration of
the loop.

\clausex{while}{~expr~}
If ~expr~ ever evaluates to \nil, the loop is terminated and the
epilogue code executed.  Equivalent to |(if (not ~expr~) (finish))|.

\clausex{until}{~expr~}
Equivalent to |(if ~expr~ (finish))|.

\clausex{if-first-time}{~then~ \opt\ ~else~}
If this clause is being executed for the first time in this invocation
of the \iter\ form, then the ~then~ code is evaluated; otherwise the
~else~ code is evaluated.

\cpar |(for ~var~ first ~expr1~ then ~expr2~)| is almost equivalent to
\begin{program}
(if-first-time (dsetq ~var~ ~expr1~)
               (dsetq ~var~ ~expr2~))
\end{program}
The only difference is that the |for| version makes ~var~ available
for use with |for\dots previous|.

\end{clauses}

\subsection{Code Placement}
When fine control is desired over where code appears in a loop
generated by \iter, the following special clauses may be useful.
They are all subject to code-motion problems (see section
\ref{code-movement}).  

\begin{clauses}

\clausex{initially}{|\&rest| ~forms~}
The lisp ~forms~ are placed in the prologue section of the loop, where
they are executed once, before the loop body is entered.

\clausex{after-each}{|\&rest| ~forms~}
The ~forms~ are placed at the end of the loop body, where they
are executed after each iteration.  Unlike the other clauses in this
section, ~forms~ may contain \iter\ clauses.

\clausex{else}{|\&rest| ~forms~}
The lisp ~forms~ are placed in the epilogue section of the loop, where they
are executed if this |else| clause is never met during execution of the
loop and the loop terminates normally.

\clausex{finally}{|\&rest| ~forms~}
The lisp ~forms~ are placed in the epilogue section of the loop, where
they are executed after the loop has terminated normally.

\clausex{finally-protected}{|\&rest| ~forms~}
The lisp ~forms~ are placed in the second form of an unwind-protect
outside the loop.  They are always executed after the loop has
terminated, regardless of how the termination occurred.

\end{clauses}


%%%%%%%%%%%%%%%%%%%%%%%%%%%%%%%%%%%%%%%%%%%%%%%%%%%%%%%%%%%%%%%%
\section{Other Features}

\subsection{Multiple Accumulations}
\label{multiple}

\begin{sloppypar}
It is permitted to have more than one clause accumulate into the same
variable, as in the following:
\begin{program}
(iter (for i from 1 to 10)
      (collect i into nums)
      (collect (sqrt i) into nums)
      (finally (return nums)))
\end{program}
Clauses can only accumulate into the same variable if they are
compatible.  |collect|, |adjoining|, |appending|, |nconcing|,
|unioning| and |nunioning| are compatible with each other; |sum|,
|multiply| and |counting| are compatible; |always| and |never| are compatible;
|finding| \dots |such-that| is compatible with either |thereis| or |always|
and |never| when their defaults match; and |maximize| and |minimize| clauses
are compatible only with other |maximize| and |minimize| clauses,
respectively.

%note that the same variable ~cannot~ be both an accumulation
%variable and an ordinary variable; there can be only one variable with
%a given name within an \iter\ form. 
\end{sloppypar}

\subsection{Named Blocks}

Like Common Lisp |block|s, \iter\ forms can be given names.  The
name should be a single symbol, and it must be the first form in the
\iter.  The generated code behaves exactly like a named block; in
particular, |(return-from ~name~)| can be used to exit it:
\begin{program}
(iter fred
      (for i from 1 to 10)
      (iter barney
            (for j from i to 10)
            (if (> (* i j) 17)
                (return-from fred j))))
\end{program}
An \iter\ form that is not given a name is implicitly named \nil.

Sometimes one would like to write an expression in an inner \iter\ form,
but have it processed by an outer \iter\ form.  This is possible with
the |in| clause.  

\begin{clauses}

\clausex{in}{~name~ |\&rest| ~forms~}
Evaluates ~forms~ as if they were part of the \iter\ form named
~name~.  In other words, \iter\ clauses are processed by the \iter\
form named ~name,~ and not by any \iter\ forms that occur inside ~name.~

\cpar As an example, consider the problem of collecting a list of the
elements in a two-dimensional array.  The naive solution,
\begin{program}
(iter (for i below (array-dimension ar 0))
      (iter (for j below (array-dimension ar 1))
            (collect (aref ar i j))))
\end{program}
\noindent is wrong because the list created by the inner \iter\ is simply
ignored by the outer one.  But using |in| we can write:
\begin{program}
(iter outer (for i below (array-dimension ar 0))
      (iter (for j below (array-dimension ar 1))
            (in outer (collect (aref ar i j)))))
\end{program}
\noindent which has the desired result.

\end{clauses}

\subsection{Destructuring}
\label{destructuring}

In many places within \iter\ clauses where a variable is expected, a
list can be written instead.  In these cases, the value to be assigned
is ~destructured~ according to the pattern
described by the list.  As a simple example, the clause
\begin{program}
(for (key . item) in alist)
\end{program}
\noindent will result in |key| being set to the |car| of
each element in |alist|, and |item| being set to the |cdr|.  The
pattern list may be nested to arbitrary depth, and (as the example
shows) need not be terminated with \nil; the only requirement is that
each leaf be a bindable symbol (or \nil, in which case no binding is
generated for that piece of the structure).

Sometimes, you might like to do the equivalent of a
|multiple-value-setq| in a clause.  This 
``multiple-value destructuring'' can be expressed by writing 
\linebreak |(values $pat_1$ $pat_2 \ldots$)| for a destructuring
pattern, as in 
\begin{program}
(for (values (a . b) c d) = (three-valued-function ...))
\end{program}
\begin{sloppypar}
Note that the $pat_i$ can themselves be destructuring patterns (though
not multiple-value destructuring patterns).  You can't do multiple-value
destructuring in a |with| clause; instead wrap the whole \iter\
form in a |multiple-value-bind|.
\end{sloppypar}

\begin{note}{Rationale}
There are subtle interactions between variable declarations and
evaluation order that make the correct implementation of
multiple-value destructuring in a |with| somewhat tricky.
\end{note}

The destructuring feature of \iter\ is available as a separate
mechanism, using the |dsetq| macro:

\begin{clauses}
\defmacro{dsetq}{template expr}
Performs destructuring of ~expr~ using ~template~.   May be used
outside of an \iter\ form. Yields the primary value of ~expr~.

\end{clauses}

\subsection{On-line Help}

\begin{sloppypar}
There is a limited facility for on-line help, in the form of the 
|display-iterate-clauses| function.
\end{sloppypar}

\begin{clauses}

\defun{display-iterate-clauses}{\opt\ clause-spec}
Displays a list of \iter\ clauses.  If ~clause-spec~ is not
provided, all clauses are shown; if it is a symbol, all clauses
beginning with that symbol are shown; and if it is a list of symbols,
all clauses for which ~clause-spec~ is a prefix are shown.

\end{clauses}

\subsection{Parallel Binding and Stepping}
\label{bindings}

The parallel binding and stepping of variables is a feature that
\iter\ does ~not~ have.  This section attempts to provide a rationale.

We say that two variables are bound ~in parallel~ if neither
binding shadows the other.  This is the usual semantics of |let|
(as opposed to |let*|).  Similarly, we can say that iteration
variables are stepped in parallel if neither variable is updated
before the other, conceptually speaking; in other words, if the code
to update each variable can reference the old values of both variables.

|loop| allows parallel binding of variables and parallel stepping of
driver variables.  My view is that if you are depending on the 
serial/parallel distinction, you are doing something obscure.  If you
need to bind 
variables in parallel using |with|, then you must be using a
variable name that shadows a name in the existing lexical environment.
Don't do that.  The most common use for parallel stepping is to track
the values of variables on the previous iteration, but in fact this
does not require parallel stepping at all; the following will work:
\begin{program}
(iter (for current in list)
      (for prev previous current)
      ...)
\end{program}


%%%%%%%%%%%%%%%%%%%%%%%%%%%%%%%%%%%%%%%%%%%%%%%%%%%%%%%%%%%%%%%%
\section{Types and Declarations}
\label{types}

\subsection{Discussion}

Sometimes efficiency dictates that the types of variables be declared.
This type information needs to be communicated to \iter\ so it can
bind variables to appropriate values.  Furthermore, \iter\ must often
generate internal variables invisible to the user; there needs to be a
way for these to be declared.

As an example, consider this code, which will return the number of
odd elements in |number-list|:
\begin{program}
(iter (for el in number-list)
      (count (oddp el)))
\end{program}
In processing this form,
\iter\ will create an internal variable, let us call it |list17|, to
hold the successive |cdr|s of |number-list|, and  
will bind the variable to |number-list|.  It will also generate a
default binding for |el|; only inside the body of the loop will |el|
be set to the |car| of |list17|.  Finally, \iter\ will generate a
variable, call it |result|, to hold the result of the count, and will
bind it to zero.

When dealing with type declarations, \iter\ observes one simple rule:
~it will never generate a declaration unless requested to do so.~  The
reason is that such declarations might mask errors in compiled code by
avoiding error-checks; the resulting problems would be doubly hard to
track down because the declarations would be hidden from the
programmer.  Of course, a compiler might omit error-checks even in the
absence of declarations, though this behavior can usually be avoided,
e.g. by saying |(declaim (optimize (safety 3)))|.

So, the above \iter\ form will generate code with no declarations.
But say we wish to declare the types of |el| and the internal
variables |list17| and |result|.  How is this done?

Declaring the type of |el| is easy, since the programmer knows
the variable's name:
\begin{program}
(iter (for el in number-list)
      (declare (fixnum el))
      (counting (oddp el)))
\end{program}
\iter\ can read variable type declarations like this one.  Before
processing any clauses, it scans the entire top-level form for type
declarations 
and records the types, so that variable bindings can be performed
correctly.  In this case, |el| will be bound to zero
instead of \nil.  Also, \iter\ collects all the top-level declarations
and puts them at the begining of the generated code, so it is not
necessary to place all declarations at the beginning of an \iter\
form; instead, they can be written near the variables whose types they
declare.  

Since \iter\ is not part of the compiler, it will not know
about declarations that occur outside an \iter\ form; these
declarations must be repeated inside the form.

Here is another way we could have declared the type of |el|:
\begin{program}
(iter (for (the fixnum el) in number-list)
      (counting (oddp el)))
\end{program}
\iter\ extends the Common Lisp |the|\lispindex{the}
form to apply to variables as well as value-producing forms; anywhere
a variable is allowed---in a |with| clause, as the iteration
variable in a driver clause, as the |into| argument of an
accumulation clause, even inside a destructuring template---you can
write |(the ~type~ ~symbol~)| instead.

There is one crucial difference between using a |the| form and
actually declaring the variable: explicit declarations are always
placed in the generated code, but type information from a |the|
form is not turned into an actual declaration unless you tell \iter\
to do so using |iterate:declare-variables|.  See below.

\begin{sloppypar}
Declaring the types of internal variables is harder than declaring the
types of explicitly mentioned variables, since their names
are unknown.  You do it by declaring |iterate:declare-variables|
somewhere inside the top level of the \iter\ form.  (This will also
generate declarations for variables declared using |the|.)
\iter\ does not provide much selectivity here: it's all or none. 
And unfortunately, since \iter\ is not privy to compiler information
but instead reads declarations itself, it will not hear if you
|(declaim (iterate:declare-variables))|.  Instead, set the variable 
|iterate::*always-declare-variables*| to |t| at
compile-time, using |eval-when|.
\end{sloppypar}

To determine the appropriate types for internal variables, \iter\ uses
three sources of information:

\begin{itemize}

\item Often, the particular clause dictates a certain type for a
variable; \iter\ will use this information when available.  In the
current example, the variable |list17| will be given the type
|list|, since that is the only type that makes sense; and the
variable |result| will be given the type |fixnum|, on the
assumption that you will not be counting high enough to need bignums.
You can override this assumption only by using and explicitly declaring a
variable: 
\begin{verbatim}
(iter (declare (iterate:declare-variables))
      (for el in number-list)
      (count (oddp el) into my-result)
      (declare (integer my-result))
      (finally (return my-result)))
\end{verbatim}

\begin{sloppypar}
Other examples of the type assumptions that \iter\ makes are: type
|list| for |into| variables of collection clauses; type |list| for
expressions that are to be destructured; type |vector| for the
variable holding the vector in a |for\dots in-vector| clause, and
similarly for |string| and the |for\dots in-string| clause;
and the implementation-dependent type
|(type-of array-dimension-limit)| for the index and limit
variables generated by sequence iteration drivers like |for\dots
in-vector| and |for\dots in-string| (but not |for\dots in-sequence|,
because it may be used to iterate over a list). 
\end{sloppypar}

\item Sometimes, \iter\ will examine expressions and try to determine
their types in a simple-minded way.  If the expression is
self-evaluating (like a number, for instance), \iter\ knows that the
expression's type is the same as the type of the value it denotes, so
it can use that type.  If the expression is of the form |(the ~type~
~expr~)|, \iter\ is smart enough to extract ~type~ and use it.
However, the current version of \iter\ does not  
examine declarations of function result types or do any type
inference.  It will not determine, for 
example, that the type of |(+ 3 4)| is |fixnum|, or even
|number|.

\item In some cases, the type of an internal variable should match the
type of some other variable.  For instance, \iter\ generates an
internal variable for |(f x)| in the
clause |(for i from 1 to (f x))|, and in the absence of other
information will give it the same type as |i|.  If, however, the
expression had been written |(the fixnum (f x))|, then \iter\
would have given the internal variable the type |fixnum|
regardless of |i|'s type.  The type incompatibility errors that
could arise in this situation are not checked for.

\end{itemize}

Note that if you do declare |iterate:declare-variables|, then
\iter\ may declare user variables as well as internal ones if they do
not already have declarations, though only for variables that it
binds.  For instance, in this code:

\begin{program}
(iter (declare (iterate:declare-variables))
      (for i from 1 to 10)
      (collect i into var))
\end{program}
the variable |var| will be declared to be of type |list|.


\subsection{Summary}

\iter\ understands standard Common Lisp variable type declarations
that occur within an \iter\ form and
will pass them through to the generated code.  If the declaration
|(iterate:declare-variables)|\lispindex{declare-variables}
appears at the top level of an 
\iter\ form, or if 
|iterate::*always-declare-variables*|\lispindex{*always-declare-variables*} 
is \nonnil, then \iter\ will use the type information gleaned from user
declarations, self-evaluating expressions and |the| expressions,
combined with reasonable assumptions, to determine variable
types and declare them.


%%%%%%%%%%%%%%%%%%%%%%%%%%%%%%%%%%%%%%%%%%%%%%%%%%%%%%%%%%%%%%%%
\section{Problems with Code Movement}
\label{code-movement}

Some \iter\ clauses, or parts of clauses, result in code being
moved from the location of the clause to other parts of the loop.
Drivers behave this way, as do code-placement clauses like |initially|
and |finally|.  When using these clauses, there is a danger of writing
an expression that makes sense in its apparent location but will be
invalid or have a different meaning in another location.  For example:
\begin{program}
(iter (for i from 1 to 10)
      (let ((x 3))
        (initially (setq x 4))))
\end{program}
While it may appear that the |x| of |(initially (setq x 4))| is the
same as the |x| of |(let ((x 3)) \dots|, in fact they are not:
|initially| moves its code outside the loop body, so |x| would refer
to a global variable.  Here is another example of the same problem:
\begin{program}
(iter (for i from 1 to 10)
      (let ((x 3))
        (collect i into x)))
\end{program}
If this code were executed, |collect| would create a binding for its
|x| at the top level of the \iter\ form that the |let| will shadow.

Happily, \iter\ is smart enough to catch these errors; it
walks all problematical code to ensure that free variables are not
bound inside the loop body, and checks all variables it binds for the
same problem.

However, some errors cannot be caught:

\begin{program}
(iter (with x = 3)
      (for el in list)
      (setq x 1)
      (reducing el by \#'+ initial-value x))
\end{program}
|reducing| moves its |initial-value| argument to the initialization
part of the loop in order to produce more efficient code.  Since
\iter\ does not perform data-flow analysis, it cannot determine that
|x| is changed inside the loop; all it can establish is that |x| is
not bound internally.  Hence this code will not signal an
error and will use $3$ as the initial value of the reduction.

The following list summarizes all cases that are subject to these code
motion and variable-shadowing problems.
\begin{itemize}
\item Any variable for which \iter\ creates a binding, including those
used in |with| and the |into| keyword of many clauses.

\begin{sloppypar}
\item The special clauses which place code: |initially|, |after-each|, |else|,
|finally| and |finally-protected|. 
\end{sloppypar}

\item The variables of a |next| or |do-next| form.

\item The |initially| arguments of |for\dots initially\dots then| and
|for\dots previous|. 

\item The |then| argument of |for\dots initially\dots then|.

\item The |initial-value| arguments of |reducing| and |accumulate|.

\item The |on-failure| argument of |finding\dots such-that|.
\end{itemize}

%%%%%%%%%%%%%%%%%%%%%%%%%%%%%%%%%%%%%%%%%%%%%%%%%%%%%%%%%%%%%%%%
\section{Differences Between {\tt Iterate} and {\tt Loop}}
|loop| contains a great deal of complexity which \iter\ tries to
avoid.  Hence many esoteric features of |loop| don't exist in \iter.
Other features have been carried over, but in a cleaned-up form.  
And of course, many new features have been added; they are not
mentioned in this list.

\begin{itemize}

\item \iter's syntax is more Lisp-like than |loop|'s, having a higher
density of parens.

\item The current implementation of \iter, unlike the current version
of  |loop| (as documented in {\em Common Lisp, 2nd Ed.\/}), is
extensible (see section \ref{extend}).  

\item |loop| puts the updates of all driver variables at the top of
the loop; \iter\ leaves them where the driver clauses appear.

\item While for the most part \iter\ clauses that resemble |loop| clauses
behave similarly, there are some differences.  For instance, there is
no |for\dots =\dots then| in \iter; instead use 
|for\dots initially\dots then|.  

\item |loop| binds the variable |it| at certain times
to allow pseudo-English expressions like |when ~expr~ return it|. 
In \iter, you must bind ~expr~ to a variable yourself.  Note that
|when ~expr~ return it| is like |thereis ~expr~| except that the latter is an
accumulation clause and therefore competes with other accumulations
(remember section \ref{multiple} above).

% repeat different behaviour of |always| clause here?

\item |loop| has a special |return| clause, illustrated in the
previous item.  \iter\ doesn't need one,  since an ordinary Lisp
|return| has the same effect.

\item |loop| allows for parallel binding and stepping of iteration
variables.  \iter\ does not.  (See section \ref{bindings}.)

\item |loop| and \iter\ handle variable type declarations very
differently.  |loop| provides a special syntax for declaring variable
types, and does not examine declarations.  Moreover, the standard
implementation of |loop| will
generate declarations when none are requested.
\iter\ parses standard Common Lisp type declarations, and will never
declare a variable itself unless declarations are 
specifically requested.

\end{itemize}

%%%%%%%%%%%%%%%%%%%%%%%%%%%%%%%%%%%%%%%%%%%%%%%%%%%%%%%%%%%%%%%%
\section{Rolling Your Own}
\label{extend}

\subsection{Introduction}

\iter\ is extensible---you can write new clauses that embody new
iteration patterns.  You might want to write a new driver clause for a
data structure of your own, or you might want to write a clause that
collects or manipulates elements in a way not provided by \iter. 

This section describes how to write clauses for \iter.  Writing a
clause is like writing a macro.  In fact, writing a clause ~is~
writing a macro: since \iter\ code-walks its body and macroexpands,
you can add new abstractions to \iter\ with good old |defmacro|.

Actually, there are two extensions you can make to \iter\ that are
even easier than writing a macro.  They are adding a synonym for an
existing clause and defining a driver clause for an indexable
sequence.  These can be done with |defsynonym| and
|defclause-sequence|, respectively.  See section \ref{aids}, below.

The rest of this section explains how to write macros that expand into
\iter\ clauses.
Here's how you could add a simplified version of \iter's |multiply|
clause, if \iter\ didn't already have one:
\begin{program}
(defmacro multiply (expr)
  `(reducing ,expr by \#'* initial-value 1))
\end{program}

If you found yourself summing the square of an expression often, you
might want to write a macro for that.  A first cut might be
\begin{program}
(defmacro sum-of-squares (expr)
  `(sum (* ,expr ,expr)))
\end{program}
but if you are an experienced macro writer, you will realize that this
code will evaluate ~expr~ twice, which is probably a bad idea.  A
better version would use a temporary:
\begin{program}
(defmacro sum-of-squares (expr)
  (let ((temp (gensym)))
    `(let ((,temp ,expr))
       (sum (* ,temp ,temp)))))
\end{program}
Although this may seem complex, it is just the sort of thing you'd
have to go through to write any macro, which illustrates the point of
this section: if you can write macros, you can extend \iter.

Our macros don't use \iter's keyword-argument
syntax.  We could just use keywords with |defmacro|, but we would
still not be using \iter's clause indexing mechanism.  Unlike Lisp,
which uses just the first symbol of a form to determine what function
to call, \iter\ individuates clauses by the list of required keywords.
For instance, |for\dots in| and |for\dots in-vector| are different clauses
implemented by distinct Lisp functions.  

To buy into this indexing scheme, as well as the keyword-argument
syntax, use |defmacro-clause|:

\begin{clauses}

\defmacro{defmacro-clause}{arglist |\&body| body}
Defines a new \iter\ clause.  ~arglist~ is a list of symbols which are
alternating keywords and arguments.  \opt\ may be used, and the list
may be terminated by |\&sequence|.  ~body~ is an ordinary macro body,
as with |defmacro|.  If the first form of ~body~ is a string, it is
considered a documentation string and will be shown by
|display-iterate-clauses|. |defmacro-clause| will signal an error if 
defining the clause would result in an ambiguity.  E.g. you cannot
define the clause |for\dots from| because there would be no way to
distinguish it from a use of the |for| clause with optional keyword |from|.

\end{clauses}

\medskip

Here is |multiply| using |defmacro-clause|.  The keywords are capitalized
for readability.
\begin{program}
(defmacro-clause (MULTIPLY expr \opt\ INTO var)
  `(reducing ,expr by \#'* into ,var initial-value 1))
\end{program}
You don't have to worry about the case when |var| is not supplied; for
any clause with an |into| keyword, saying |into nil| is equivalent to
omitting the |into| entirely.

As another, more extended example, consider the fairly common
iteration pattern that 
involves finding the sequence element that maximizes (or minimizes) some
function.  \iter\ provides this as |finding\dots maximizing|, but it's
instructive to see how to write it.
Here, in pseudocode, is how you might write such a loop for
maximizing a function F:

\begin{quote}
\begin{tabbing}
set variable MAX-VAL to NIL; \\
set variable WINNER to NIL; \\
for\=\ each element EL in the sequence \\
\>  if\=\ MAX-VAL\=\ is NIL or F(EL) $>$ MAX-VAL then \\
\>\>    set MAX-VAL to F(EL); \\
\>\>    set WINNER to EL; \\
\>  end if; \\
end for; \\
return WINNER.
\end{tabbing}
\end{quote}

Here is the macro:
\begin{program}
(defmacro-clause (FINDING expr MAXIMIZING func \opt\ INTO var)
  (let ((max-val (gensym))
        (temp1 (gensym))
        (temp2 (gensym))
        (winner (or var iterate::*result-var*)))
    `(progn 
       (with ,max-val = nil)
       (with ,winner = nil)
       (cond 
        ((null ,max-val)
         (setq ,winner ,expr)
         (setq ,max-val (funcall ,func ,winner))
        (t
         (let* ((,temp1 ,expr)
                (,temp2 (funcall ,func ,temp1)))
           (when (> ,temp2 ,max-val)
             (setq ,max-val ,temp2)
             (setq ,winner ,temp1))))))
       (finally (leave ,winner)))))
\end{program}
Note that if no |into| variable is supplied, we use
|iterate::*result-var*|, which contains the internal variable into
which all clauses place their results.  If this variable is bound by
some clause, then \iter\ will return its value automatically;
otherwise, \nil\ will be returned.  

\subsection{Writing Drivers}

In principle, drivers can be implemented just as easily as other
\iter\ clauses.  In practice, they are a little harder to get right. 
As an example, consider writing a driver that
iterates over all the 
elements of a vector, ignoring its fill-pointer.  |for\dots in-vector|
won't work for this, because it observes the fill-pointer.  It's necessary to
use |array-dimension| instead of |length| to obtain the size of the
vector.  Here is one approach:

\begin{program}
(defmacro-clause (FOR var IN-WHOLE-VECTOR v)
  "All the elements of a vector (disregards fill-pointer)"
  (let ((vect (gensym))
        (index (gensym)))
    `(progn
       (with ,vect = ,v)
       (for ,index from 0 below (array-dimension ,vect 0))
       (for ,var = (aref ,vect ,index)))))
\end{program}
Note that we immediately put |v| in a variable, in case it is an
expression.  Again, this is just good Lisp macrology.  It also has a
subtle effect on the semantics of the driver: |v| is evaluated only
once, at the beginning of the loop, so changes to |v| in the loop have
no effect on the driver.  Similarly, the bounds for numerical iteration
e.g. the above |array-dimension| are also evaluated once only.  This is how
all of \iter's drivers work.


There is an important point concerning the |progn| in this code.  We
need the |progn|, of course, because we are returning several forms,
one of which is a driver.  But \iter\ drivers must occur at top-level.
Is this code in error?  No, because ~top-level~ is defined in \iter\
to include forms inside a |progn|.  This is just the definition of
top-level that Common Lisp uses, and for the same reason: to allow
macros to return multiple forms at top-level.

While our |for\dots in-whole-vector| clause will work, it is not
ideal.  In particular, it does not support generating.  Do do so, we
need to use |for\dots next| or |for\dots do-next|.  The job is
simplified by the |defmacro-driver| macro.

\begin{clauses}

\defmacro{defmacro-driver}{arglist |\&body| body}
Defines a driver clause in
both the |for| and |generate| forms, and provides a parameter
|generate| which ~body~ can examine to determine how it was invoked.
~arglist~ is as in |defmacro-clause|, and should begin with the symbol
|for|. 

\end{clauses}

With |defmacro-driver|, our driver looks like this:
\begin{program}
(defmacro-driver (FOR var IN-WHOLE-VECTOR v)
  "All the elements of a vector (disregards fill-pointer)"
   (let ((vect (gensym))
         (end (gensym))
         (index (gensym))
         (kwd (if generate 'generate 'for)))
     `(progn
        (with ,vect = ,v)
        (with ,end = (array-dimension ,vect 0))
        (with ,index = -1)
        (,kwd ,var next (progn (incf ,index)
                               (if (>= ,index ,end) (terminate))
                               (aref ,vect ,index))))))
\end{program}


We are still missing one thing: the |\&sequence| keywords.  
We can get them easily enough, by writing 
\begin{program}
(defmacro-driver (FOR var IN-WHOLE-VECTOR v \&sequence) 
  ...)
\end{program}
We can now refer to parameters |from|, |to|, |by|, etc. which contain
either the values for the corresponding keyword, or \nil\ if the
keyword was not supplied.  Implementing the right code for these
keywords is cumbersome but not difficult; it is left as an exercise.
But before you begin, see |defclause-sequence| below for an easier way.
       
\subsection{Extensibility Aids}
\label{aids}

This section documents assorted features that may be of use in
extending \iter.

\begin{clauses}

\defunexpvar{*result-var*}
Holds the variable that is used to return a value as a result of the
\iter\ form.  You may examine this and use it in a |with| clause, but
you should not change it.

\defmacro{defsynonym}{syn word}
 Makes ~syn~ a synonym for the existing \iter\ keyword ~word.~  Only
the first word in each clause can have synonyms. 

\defmacroxx{defclause-sequence}{element-name index-name}{|\&key|
access-fn size-fn sequence-type}{element-type
element-doc-string index-doc-string}  
Provides
a simple way to define sequence clauses.  Generates two
clauses, one for iterating over the sequence's elements, the other
for iterating over its indices.  The first symbol of both
clauses will have print-name |for|.
~element-name~  and ~index-name~ should be symbols.
~element-name~ is the second keyword of the element iterator (typically of
the form 
|in-~sequence-type~|), and ~index-name~ is the second keyword
of the index-iterator (typically of the form
|index-of-~sequence-type~|).  Either name may be 
\nil, in which case the corresponding clause is not defined.  If both
symbols are supplied, they should be in the same package.  The |for|
that begins the clauses will be in this package.

\cpar ~access-fn~ is the function to be used to
access elements of the sequence in the element iterator.  The function
should take two 
arguments, a sequence and an index, and return the appropriate element.
~size-fn~ should denote a function of one argument, a sequence, that
returns its size.  Both ~access-fn~ and ~size-fn~ are required for the
element iterator, but only ~size-fn~ is needed for the index iterator.

\cpar The ~sequence-type~ and ~element-type~ keywords are used to
suggest types for the variables 
used to hold the sequence and the
sequence elements, respectively.  The usual rules about \iter's
treatment of variable type declarations apply (see section \ref{types}).

\cpar ~element-doc-string~ and ~index-doc-string~ are
the documentation strings, for use with |display-iterate-clauses|. 

\cpar The generated element-iterator performs destructuring on the
element variable.

\cpar As an example, the above |for\dots in-whole-vector| example
could have been written:
\begin{program}
(defclause-sequence IN-WHOLE-VECTOR INDEX-OF-WHOLE-VECTOR
  :access-fn 'aref
  :size-fn \#'(lambda (v) (array-dimension v 0))
  :sequence-type 'vector
  :element-type t
  :element-doc-string 
     "Elements of a vector, disregarding fill-pointer"
  :index-doc-string 
     "Indices of vector, disregarding fill-pointer")
\end{program}

\end{clauses}

\subsection{Subtleties}

There are some subtleties to be aware of when writing \iter\ clauses.
First, the code returned by your macros may be |nconc|'ed into a list,
so you should always returned freshly consed lists, rather than
constants.  Second, \iter\ matches clauses by using |eq| on the first
symbol and |string=| on the subsequent ones, so the package of the
first symbol of a clause is relevant.  All of the clauses in this manual
have their first word in the \iter\ package.  
You can use the package system in the usual way to shadow
\iter\ clauses without replacing them.

%%% say more here, about the badness that only the first word of a
%%% clause is packagey.

\section{Non-portable Extensions to Iterate (Contribs)}
\label{contribs}

Currently, there is only one non-portable extension to iterate in the
distribution: iterate-pg. If you have made an extension that depends
on non-portable features, feel free to send them to |asf@boinkor.net|
for inclusion in the iterate distribution.

\subsection{An SQL query driver for iterate}

The pg package by Eric Marsden (see |http://cliki.net/pg|) provides an
interface to the PostgreSQL database. Using the \iterpg\ extension, it
is possible to handle the results of SQL queries using \iter.

This usage example should give you an idea of how to use it:

\begin{program}
(pg:with-pg-connection (c "somedb" "someuser")
  (iter (for (impl version date) in-relation "select * from version"
                                 on-connection *dbconn*)
        (collect version)))
\end{program}

To use the extension via |ASDF|, simply make your system depend on the
|iterate-pg| system instead of the |iterate| system. To load it
manually, use:

\begin{program}
  (asdf:oos 'asdf:load-op :iterate-pg)
\end{program}

%%%%%%%%%%%%%%%%%%%%%%%%%%%%%%%%%%%%%%%%%%%%%%%%%%%%%%%%%%%%%%%%
\section{Obtaining {\tt Iterate}}

\begin{sloppypar}
The information in this chapter is obsolete but included for
completeness's sake; Currently, the most up-to-date information on
\iter\ can be found at |http://boinkor.net/iterate.html|.
\end{sloppypar}

\begin{sloppypar}
\iter\ currently runs on Lisp Machines, and on
HP's, Sun3's and Sparcstations under Lucid.  
\iter\ source and binaries are available at the MIT AI Lab in the
subdirectories of |/src/local/lisplib/|.  The source file,
|iterate.lisp|, is also available for anonymous FTP in the directory
|/com/fpt/pub/| on the machine |TRIX.AI.MIT.EDU| (Internet number
128.52.32.6).  If you are unable to obtain |iterate| in one of these
ways, send mail to |jba@ai.mit.edu| and I will send you the source
file. 
\end{sloppypar}

\begin{sloppypar}
\iter\ resides in the |iterate| package (nickname |iter|).  Just say
\linebreak |(use-package :iterate)| to make all the necessary symbols
available. 
If a symbol is not exported, it appears in this manual with an
``|iterate::|'' prefix.
\end{sloppypar}

Send bug reports to |bug-iterate@ai.mit.edu|.  The |info-iterate|
mailing list will have notices of changes and problems; to have
yourself added, send mail to |info-iterate-request@ai.mit.edu|.


\medskip

\begin{flushleft}
 \bf Acknowledgements
\end{flushleft}
\smallskip

Richard Waters provided invaluable criticism which spurred me to improve
\iter\ greatly.  As early users, David Clemens, Oren Etzioni and Jeff
Siskind helped ferret out many bugs.

%\begin{theindex}
% The index files must be generated with the genindex program
% or from makeindex iterate-manual.idx which creates iterate-manual.ind.

%\baselineskip\normalbaselineskip
%\advance\baselineskip by -2pt

\input{iterate-manual.ind}

%\underline{Clauses}
%\input{iterate-manual.clindex}
%
%\indexspace
%\underline{Lisp}
%\input{iterate-manual.lispindex}

%\end{theindex}


\end{document}


% arch-tag: "c0f871a6-313c-11d8-abb9-000c76244c24"


%\end{theindex}


\end{document}


% arch-tag: "c0f871a6-313c-11d8-abb9-000c76244c24"


%\end{theindex}


\end{document}


% arch-tag: "c0f871a6-313c-11d8-abb9-000c76244c24"

%
%\indexspace
%\underline{Lisp}
%% Fix ``.'' spacing in lists
\documentstyle[12pt,aimemo]{article}

\setlength{\textwidth}{6.0in}
\setlength{\oddsidemargin}{0.25in}
\setlength{\evensidemargin}{0.25in}
\setlength{\topmargin}{0pt}
\setlength{\headsep}{24pt}
\setlength{\textheight}{8.5in}

\let\origpar=\par

% somehow, LaTex fucks with things so that the plain Tex \obeyspaces
% doesn't work.  This is the same, except we use an mbox containing
% a space instead of just a space.

\newcommand{\spc}{\mbox{ }}
\newcommand{\tAb}{\mbox{        }}
\def\obspaces{\catcode`\ =\active\catcode`\^^I=\active}
{\obspaces \global\let =\spc\global\let^^I=\tAb}

\newenvironment{program}{\medskip\samepage\tt\obeylines\obspaces}{\medskip}

% use | for \tt
\catcode`\|=13
\newif\ifvbar
\def|{\ifvbar
         \endgroup\vbarfalse
      \else 
         \begingroup\tt\vbartrue 
      \fi}



\newcommand{\lisp}{\tt}
\newcommand{\iter}{|iterate|}
\newcommand{\iterpg}{|iterate-pg|}
\newcommand{\setf}{|setf|}
\newcommand{\nil}{|nil|}
\newcommand{\nonnil}{non-\nil}
\newcommand{\opt}{{\lisp \&optional}}
\newcommand{\yields}{$\Rightarrow$}



\newbox\Dots
\setbox\Dots=\hbox{\small ...}
\renewcommand{\dots}{\copy\Dots}

% Here's a way to define an environment which will treat all
% paragraphs inside it in the same manner.  It's, as near as I can
% figure, the way that Latex does things.  Using \everypar won't work;
% apparently Latex resets that a lot.  What you have to do is: at the
% top of the file do \let\origpar=\par.  In the
% begin part of the environment, do the parshape (or whatever), then say
% \def\par{{\origpar}}.  The extra braces are necessary.   That's all
% you have to do; at the end of the environment things will return to
% normal automatically.  Why does this all work?  I have no idea.
% Another important thing: there should be a blank line (i.e. a \par)
% between the end of the last paragraph and the \end{environment}.

\newlength{\clindent}
\setlength{\clindent}{0.5in}
\newlength{\clparindent}
\setlength{\clparindent}{\parindent}

\newenvironment{clauses}{\advance\linewidth -\parindent
                        \hangindent\clindent\relax
                         \hangafter=0
                          \parindent=0pt
                          \def\par{{\origpar}}}{}

\newcommand{\cpar}{\hspace \clparindent}

\newcommand{\startitem}{\bigskip\pagebreak[3]}
\newcommand{\finishitem}{\smallskip}

\newcommand{\clause}[1]{\startitem\Clause{#1}\finishitem}
\newcommand{\Clause}[1]{\hbox{#1}}
\newcommand{\clindex}[1]{\index{{\lisp #1}}}
\newcommand{\clindexx}[2]{\index{{\lisp #1\dots #2}}}
\newcommand{\clindexxx}[3]{\index{{\lisp #1\dots #2\dots #3}}}
\newcommand{\clausex}[2]{\startitem\Clausex{#1}{#2}\finishitem}
\newcommand{\Clausex}[2]{\Clause{|#1| #2}\clindex{#1}}

\newcommand{\defvar}[1]{\startitem\deff{#1}{}{Variable}\finishitem}
\newcommand{\defunexpvar}[1]{\startitem\deffnoindex{iterate::#1}{}{Variable}
                                        \lispindex{#1}\finishitem}
\newcommand{\defun}[2]{\startitem\deff{#1}{#2}{Function}\finishitem}
\newcommand{\defmacro}[2]{\startitem\deff{#1}{#2}{Macro}\finishitem}
\newcommand{\defunx}[3]{\startitem\deffx{#1}{#2}{#3}{Function}\finishitem}
\newcommand{\defmacrox}[3]{\startitem\deffx{#1}{#2}{#3}{Macro}\finishitem}
\newcommand{\defunxx}[4]{\startitem\deffxx{#1}{#2}{#3}{#4}{Function}\finishitem}
\newcommand{\defmacroxx}[4]{\startitem\deffxx{#1}{#2}{#3}{#4}{Macro}\finishitem}

\newcommand{\deff}[3]{\deffnoindex{#1}{#2}{#3}\lispindex{#1}}
\newcommand{\deffnoindex}[3]{\hbox to \hsize{{\tt #1} {\it #2}\hfill [{\it #3}]}}
\newcommand{\deffx}[4]{\deff{#1}{#2}{#4}\moreargs{#1}{#3}}
\newcommand{\deffxx}[5]{\deffx{#1}{#2}{#3}{#5}\moreargs{#1}{#4}}

\newcommand{\moreargs}[2]{\hbox to \hsize{\phantom{\lisp #1} {\it #2}\hfill}}

\newcommand{\lispindex}[1]{\index{{\lisp #1}}}

 \newenvironment{note}[1]{\pagebreak[2]\bigskip
                          \hrule\smallskip\small
                         {\setlength{\parindent}{0pt}
                           \par{\bf #1:}}}{\smallskip\hrule\bigskip}



\makeindex

%%%%%%%%%%%%%%%%%%%%%%%%%%%%%%%%%%%%%%%%%%%%%%%%%%%%%%%%%%%%%%%%

%\thispagestyle{empty}

%\vspace*{2in}

%\title{The |Iterate| Manual}
%\author{Jonathan Amsterdam}

%\maketitle

%\thispagestyle{empty}

%\begin{center}
%\Large *** DRAFT ***
%\end{center}
%This manual is in beta-test. It will become an AI Memo in a couple of
%months.  I would appreciate any comments or
%corrections.  You can reach me at email address |jba@ai.ai.mit.edu|.

\begin{document}

\def\writtenby{Jonathan Amsterdam}
\def\contractno{This report describes research done at the Artificial
Intelligence Laboratory of the Massachusetts Institute of Technology.
Support for the laboratory's artificial intelligence research is
provided in part by the Advanced Research Projects Agency of the
Department of Defense under Office
of Naval Research contract N00014-85-K-0124.}

\AIMEMO{1236}{\Large The Iterate Manual \\[1in]}
{This is the manual for version 1.4 of \iter, a powerful iteration
macro for Common  
Lisp. \iter\ is similar to {\tt loop} but provides numerous additional
features, is well integrated with Lisp, and is extensible. 
\vfill }


% use ~ for \em
\catcode`\~=13
\newif\ifem
\def~{\ifem
         \/\endgroup\emfalse
      \else 
         \begingroup\em\emtrue 
      \fi}



%\pagebreak
%\setcounter{page}{1}

%\thispagestyle{empty}


\tableofcontents

\pagebreak
\pagestyle{headings}

\section{Introduction}

\begin{sloppypar}
This manual describes
\iter, a powerful iteration facility for Common Lisp.
\iter\ provides abstractions for many common iteration
patterns and allows for the definition of additional patterns.  \iter\ is
a macro that expands into ordinary Lisp at
compile-time, so it is more efficient than higher-order
functions like |map| and |reduce.|  
While it is similar to |loop|, \iter\ offers a more Lisp-like syntax
and enhanced extensibility.
(For a more complete
comparison of \iter\ with other iteration constructs, see MIT AI Lab Working
Paper No. 324, ~Don't Loop, Iterate.~)
\end{sloppypar}

An \iter\ form consists of the symbol |iter|\footnote{You can also use
|iterate|, but |iter| is preferred because it avoids potential conflicts with
possible future additions to Common Lisp, and because it saves
horizontal space when writing code.}
followed by one or more
forms, some of which may be \iter\ 
~clauses.~  Here is a simple example of \iter\ which collects the
numbers from 1 to 10 into a list, and returns the list.  The return
value is shown following the arrow.
\begin{program}
(iter (for i from 1 to 10)
      (collect i))          \yields (1 2 3 4 5 6 7 8 9 10)
\end{program}
This form contains two clauses: a |for| clause that steps the
variable |i| over the integers from 1 to 10, and a |collect|
clause that accumulates its argument into a list.  With a few
exceptions, all \iter\ clauses have the same format: alternating
symbols (called ~keywords~) and expressions (called
~arguments~).  The syntax and terminology are those of Common
Lisp's keyword lambda lists.  One difference is that \iter's keywords
do not have to begin with a colon---though they may, except for the
first symbol of a clause.  So you can also write  |(for i :from
1 :to 10)| if you prefer.

Any Lisp form can appear in the body of an \iter, where it will have
its usual meaning.  \iter\ walks the entire body, expanding macros,
and recognizing clauses at any level.  This example collects all the
odd numbers in a list:
\begin{program}
(iter (for el in list)
      (if (and (numberp el) (oddp el))
          (collect el)))
\end{program}

There are clauses for iterating over numbers, lists, arrays and other
objects, and for
collecting, summing, counting, maximizing and other useful operations.
\iter\ also supports the creation of new variable bindings, stepping
over multiple sequences at once, destructuring, and compiler
declarations of variable types.  The following example illustrates
some of these features: 
\begin{program}
(iter (for (key . item) in alist)
         (for i from 0)
         (declare (fixnum i))
         (collect (cons i key)))
\end{program}

This loop takes the keys of an alist and returns a new alist
associating the keys with their positions in the original list. The
compiler declaration for |i| will appear in the generated code
in the appropriate place.

%%%%%%%%%%%%%%%%%%%%%%%%%%%%%%%%%%%%%%%%%%%%%%%%%%%%%%%%%%%%%%%%
\section{Clauses}

Most of \iter's clauses will be familiar to |loop| programmers.  
(|loop| is an iteration macro that has been incorporated into Common
Lisp.  Seen Guy Steele's {\em Common Lisp, 2nd Edition}.)
In
nearly all cases they behave the same as their |loop| counterparts, so
a |loop| user can switch to \iter\ with little pain (and much gain).

All clauses with the standard keyword-argument syntax consist of two
parts: a ~required~ part, 
containing keywords that must be present and in the right order; and
an ~optional~ part, containing keywords that may be omitted and,
if present, may occur in any order.  In the descriptions below, the
parts are separated by the Lisp lambda-list keyword \opt.


\subsection{Drivers}

An iteration-driving clause
conceptually causes the iteration to go forward.  Driver clauses in
\iter\ allow iteration over numbers, lists, vectors, hashtables, packages,
files and streams.  Iteration-driving clauses must
appear at the top level of an \iter\ form; they cannot be nested
inside another clause.  The driver variable is updated at the point
where the driver clause occurs.  Before the clause is executed for the
first time, the value of the variable is undefined.

%Also, regardless of where the driver clause
%appears in the body, the driver variable is stepped at the top of the
%loop; hence it is stylistically preferable, though
%not required, to place driver clauses at the beginning of the \iter.

Multiple drivers may appear in a single \iter\ form, in which case all
of the driver variables are updated each time through the loop, in the
order in which the clauses appear.  The first driver to terminate will
terminate the entire loop.

In all cases, the value of the driver variable on exit from the loop,
including within the epilogue code (see the |finally| clause), is
undefined.

All the parameters of a driver clause are evaluated once, before the
loop begins.  Hence it is not possible to change the bounds or other
properties of an iteration by side-effect from within the loop.

With one exception, driver clauses begin with the word |for| (or
the synonym |as|) and mention an iteration variable, which is
given a binding within the \iter\ form.
The exception is |repeat|, which just executes a loop a
specified number of times:

\begin{clauses}

\clausex{repeat}{~n~}
Repeats the loop ~n~ times.  For example:
\begin{program}
(iter (repeat 100)
      (print "I will not talk in class."))
\end{program}
\cpar If $n \leq 0$, the
loop will never be executed.  If ~n~ is not an integer, the
actual number of executions will be $\lceil n \rceil$.

\end{clauses}

\pagebreak[3]

\subsubsection{Numerical Iteration}

\begin{clauses}
\clausex{for}{~var~ |\&sequence|} 
The general form for iterating over a sequence of numbers requires a
variable and, optionally, one or more keywords that provide the bounds
and step size of the iteration. The |\&sequence|
lambda-list keyword is a shorthand for these sequence keywords.  They are:
|from|, |upfrom|, |downfrom|, |to|, |downto|, |above|, |below| and |by|.
|from| provides the starting value for ~var~
and defaults to zero. |to| provides a final value and implies that the
successive values of ~var~ will be increasing; |downto| implies that
they will be decreasing.  The loop terminates when ~var~ passes
the final value (i.e. becomes smaller or larger than it, depending on
the direction of iteration); in other  words, the loop body
will never be 
executed for values of ~var~ past the final value.
|below| and |above| are similar to |to| and
|downto|, except that the loop terminates when ~var~ equals or passes
the final value.

\cpar If no final value is specified, the variable will be stepped
forever.  Using |from| or |upfrom| will result in increasing
values, while |downfrom| will give decreasing values.

\cpar On each iteration,
~var~ is incremented or decremented by the value of the sequence
keyword |by|, which defaults to 1.  It should always be a positive
number, even for downward iterations.

\cpar In the following examples, the sequence of numbers generated is shown
next to the clause.

\begin{program}
(for i upfrom 0) \yields\ 0 1 2 \ldots
(for i from 5)   \yields\ 5 6 7 \ldots    ; either from or upfrom is okay
(for i downfrom 0) \yields\ 0 -1 -2 \ldots
(for i from 1 to 3) \yields\ 1 2 3
(for i from 1 below 3) \yields\ 1 2
(for i from 1 to 3 by 2) \yields\ 1 3
(for i from 1 below 3 by 2) \yields\ 1
(for i from 5 downto 3) \yields\ 5 4 3
\end{program}

\end{clauses}

\subsubsection{Sequence Iteration}

There are a number of clauses for iterating over sequences.  In all of
them, the argument following |for| may be a list instead of a
symbol, in which case destructuring is performed.  See section
\ref{destructuring}. 

\begin{clauses}

\clause{|for| ~var~ |in| ~list~ \opt\ |by| ~step-function~}
\clindexx{for}{in}
~var~ is set to successive elements of list.  
~step-function,~ which 
defaults to |cdr|, is used to obtain the next sublist. 

\clause{|for| ~var~ |on| ~list~ \opt\ |by| ~step-function~}
\clindexx{for}{on}
~var~ is set to successive sublists of list. 
~step-function~ (default |cdr|) is used as in |for\dots in|.

\end{clauses}

\medskip

These two clauses use |atom| to test for the end
of a list.  Hence, given a list whose final |cdr| is not \nil,
they will silently ignore the last |cdr|.  Other choices are
|endp|,  which would signal an error, and |null|, which
would probably result in an error somewhere else.  If you wish to use
an end-test other than |atom|, set the variable
|iterate::*list-end-test*|\lispindex{*list-end-test*} to the name of
the desired function. 

\begin{clauses}
\clause{|for| ~var~ |in-vector| ~vector~ |\&sequence|}
\clindexx{for}{in-vector}
~var~ takes on successive elements from ~vector.~   The vector's
fill-pointer is observed.  Here and in subsequent clauses, the |\&sequence|
keywords include |with-index|, which takes a symbol as argument
and uses it for the index variable instead of an internally generated
symbol.  The other |\&sequence| keywords behave as in numerical
iteration, except that the default iteration bounds are the bounds of
the vector.  E.g. in |(for i in-vector v downto 3)|,
|i| will start off being bound to the last element in |v|,
and will  
be set to preceding elements down to and including the element with
index 3.

\clause{|for| ~var~ |in-sequence| ~seq~ |\&sequence|}
\clindexx{for}{in-sequence}
This uses Common Lisp's generalized sequence functions, |elt|
and |length|, to obtain elements and determine the length of
~seq.~   Hence it will work for any sequence, including lists, and will
observe the fill-pointers of vectors.

\clause{|for| ~var~ |in-string| ~string~ |\&sequence|}
\clindexx{for}{in-string}
~var~ is set to successive characters of ~string.~

\startitem
\Clause{|for| ~var~ |index-of-vector| ~vector~ |\&sequence|}
\clindexx{for}{index-of-vector}
\Clause{|for| ~var~ |index-of-sequence| ~sequence~ |\&sequence|}
\clindexx{for}{index-of-sequence}
\Clause{|for| ~var~ |index-of-string| ~string~ |\&sequence|}
\clindexx{for}{index-of-string}
\finishitem
~var~ is set to successive indices of the sequence.
These clauses avoid the overhead of accessing the sequence elements
for those applications where they do not need to be examined, or are
examined rarely.  They admit all the optional keywords of the other
sequence drivers except the (redundant) |with-index| keyword.

\clause{|for| ~(key value)~ |in-hashtable| ~table~}
\clindexx{for}{in-hashtable}

~key~ and ~value,~ which must appear as shown in a list
and may be destructuring templates, are set to the keys and values of
~table~.  If ~key~ is |nil|, then the hashtable's keys will be ignored;
similarly for ~value~.
The order in which 
elements of ~table~ will be retrieved is unpredictable.

\clause{|for| ~var~ |in-package| ~package~ \opt\ |external-only| ~ext~}
\clindexx{for}{in-package}
Iterates over all the symbols in ~package,~ or over only the
external symbols if ~ext~ is specified and non-|nil|.  ~ext~ is not
evaluated.  The same symbol may appear more than once.

\clause{|for| ~var~ |in-packages| ~packages~ \opt\ |having-access| ~symbol-types~}
\clindexx{for}{in-packages}
Iterates over all the symbols from the list of packages denoted by the
descriptor ~packages~ and having accessibility (or visibility) given by
~symbol-types~.  This defaults to the list |(:external :internal :inherited)|
and is not evaluated.  ~var~ must be a list of up to three variables: in each
iteration, these will be set to a symbol, its access-type and package (as per
|with-package-iterator| in ANSI-CL).  The same symbol may appear more than
once.

\end{clauses}

\begin{clauses}

\clause{|for| ~var~ |in-file| ~name~ \opt\ |using| ~reader~}
\clindexx{for}{in-file}
Opens the file ~name~ (which may be a string or pathname) for input,
and iterates 
over its contents. ~reader~ defaults to |read|, so by default ~var~
will be bound to the successive forms in the file.  The \iter\ body is
wrapped in an unwind-protect to ensure that the file is closed no
matter how the \iter\ is exited.

\clause{|for| ~var~ |in-stream| ~stream~ \opt\ |using| ~reader~}
\clindexx{for}{in-stream}
Like |for\dots in-file|, except that ~stream~ should be an existing
stream object that supports input operations.

\end{clauses}

\subsubsection{Generalized Drivers}

These are primarily useful for writing drivers that can also be used
as generators (see section \ref{generators}, below).
 
\begin{clauses}

\clause{|for| ~var~ |next| ~expr~}
\clindexx{for}{next}

~var~ is set to ~expr~ each time through the loop.  Destructuring is
performed.  When the clause is used as a generator, ~expr~ is the code
that is executed when |(next ~var~)| is 
encountered (see section \ref{generators}, below).
~expr~ should compute the first value for ~var~, as well
as all subsequent values, and is responsible for terminating the loop.
For compatibility with future versions of \iter, this termination
should be done with |terminate|\clindex{terminate}, which can be
considered a synonym for |finish| (see section \ref{control-flow}).

\cpar As an example, the following clauses are equivalent to |(for
i from 1 to 10)|:
\begin{program}
(initially (setq i 0))
(for i next (if (> i 10) (terminate) (incf i)))
\end{program}

\clause{|for| ~var~ |do-next| ~form~}
\clindexx{for}{do-next}
~form~ is evaluated each time through the loop.  Its value is ~not~
set to ~var~; that is ~form's~ job. ~var~ is only present so that
\iter\ knows it is a driver variable.  \linebreak |(for ~var~ next
~expr~)| is equivalent to |(for ~var~ do-next (dsetq ~var~ ~expr~))|.
(See section \ref{destructuring} for an explanation of |dsetq|.) 

\end{clauses}

\subsubsection{Generators}
\label{generators}

In all of the above clauses, the driver variable is updated on each
iteration.
Sometimes it is desirable to have greater control over updating.  
For instance, consider the problem of associating numbers, in
increasing order and with no gaps, with the
\nonnil\ elements of a list.  One obvious first pass at writing this is:

\begin{program}
(iter (for el in list)
      (for i upfrom 1)
      (if el (collect (cons el i))))
\end{program}
But on the list |(a b nil c)| this produces |((a . 1) (b . 2) (c .
4))| instead of the desired |((a . 1) (b . 2) (c . 3))|.  The problem
is that |i| is incremented each time through the loop, even when |el|
is |nil|.  

The problem could be solved elegantly if we could step |i| only when
we wished to.  This
can be accomplished for any \iter\ driver by writing |generate| (or
its synonym |generating|)
instead of |for|.  Doing so produces a ~generator~---a driver whose
values are yielded explicitly.  To obtain the next value of a
generator variable ~v~, write \linebreak |(next ~v~)|.  The value of a |next|
form is the next value of ~v~, as determined by its associated driver
clause.  |next| also has the side-effect of updating ~v~ to that
value.  If there is no next value, |next| will terminate the loop,
just as with a normal driver.


Using generators, we can now write our example like this:
\begin{program}
(iter (for el in list)
      (generate i upfrom 1)
      (if el (collect (cons el (next i)))))
\end{program}
Now |i| is updated only when |(next i)| is executed, and this occurs
only when |el| is \nonnil.

To better understand the relationship between ordinary drivers and
generators, observe that we can rewrite an ordinary driver using its
generator form immediately followed by |next|, as this example shows:
\begin{program}
(iter (generating i from 1 to 10)
      (next i)
      ...)
\end{program}
Provided that the loop body contains no |(next i)| forms, this will
behave just as if we had written |(for i from 1 to 10)|.

We can still refer to a driver variable ~v~ without using |next|; in
this case, its value is that given to it by the last evaluation of
|(next ~v~)|.  Before |(next ~v~)| has been called the first time, the
value of ~v~ is undefined.  

This semantics is more flexible than
one in which ~v~ begins the loop bound to its first value and calls
of |next| supply subsequent values, because it means the loop will not
terminate too soon if the generator's sequence is empty.  For
instance, consider the following code, which tags \nonnil\ elements of
a list using a list of tags, and also counts the null elements.
(We assume there are at least as many tags as \nonnil\ elements.)
\begin{program}
(let* ((counter 0)
       (tagged-list (iter (for el in list)
                          (generating tag in tag-list) 
                          (if (null el)
                              (incf counter)
                              (collect (cons el (next tag)))))))
  ...)
\end{program}
It may be that there are just as many tags as non-null elements of
|list|.  If all the elements of |list| are null, we still want the
counting to proceed, even though |tag-list| is \nil.  If |tag| had to be
assigned its first value before the loop begins, we would have had to
terminate the loop before the first iteration, since when |tag-list|
is \nil, |tag| 
has no first value.  With the existing semantics, however, |(next
tag)| will never execute, so the iteration will cover all the elements of
|list|. 
        
When the ``variable'' of a driver clause is actually a destructuring
template containing several variables, all the variables are eligible
for use with |next|.  As before, |(next ~v~)| evaluates to ~v's~ next
value; but the effect is to update all of the template's variables.
For instance, the following code will return the list |(a 2 c)|.
\begin{program}
(iter (generating (key . item) in '((a . 1) (b . 2) (c . 3)))
      (collect (next key))
      (collect (next item)))
\end{program}

Only driver clauses with variables can be made into generators.  This
includes all clauses mentioned so far except for |repeat|.  It does
~not~ include |for\dots previous|, |for\dots =|,
|for\dots initially\dots then| or |for\dots first\dots then| (see
below). 

\subsubsection{Previous Values of Driver Variables}

Often one would like to access the value of a variable on a previous
iteration.  \iter\ provides a special clause for accomplishing this.

\begin{clauses}

\clause{|for| ~pvar~ |previous| ~var~ \opt\ |initially| ~init~
|back| ~n~}
\clindexx{for}{previous}
Sets ~pvar~ to the previous value of ~var,~ which should be a driver
variable, a variable from another |for\dots previous| clause, or a
variable established by a |for\dots =|, 
|for\dots initially\dots then| or |for\dots first\dots then| clause
(see section \ref{setting}).
Initially, ~pvar~ is given the value ~init~ (which defaults to \nil).  
The ~init~ expression will be moved outside the loop body, so it
should not depend on anything computed within the loop.
~pvar~ retains the value of ~init~ until ~var~ is set to its second
value, at which point ~pvar~ is set to ~var's~ first value; and so on.  

\cpar The
argument ~n~ to |back| must be a constant, positive integer, and
defaults to 1.  It determines how many iterations back ~pvar~ should
track ~var~.  For example, when ~n~ is 2, then ~pvar~ will be assigned
~var's~ first value when ~var~ is set to its third value.

\cpar A |for\dots previous| clause may occur after or before its
associated driver clause. |for\dots previous| works with generators as
well as ordinary drivers.

\pagebreak[3]

\cpar Example:
\begin{program}
(iter (for el in '(1 2 3 4))
      (for p-el previous el)
      (for pp-el previous p-el initially 0)
      (collect pp-el))
\end{program}
This evaluates to |(0 0 1 2)|.  It could have been written more
economically as 
\begin{program}
(iter (for el in '(1 2 3 4))
      (for pp-el previous el back 2 initially 0)
      (collect pp-el))
\end{program}


\end{clauses}

\subsection{Variable Binding and Setting}
\label{setting}

Several clauses exist for establishing new variable bindings or for
setting variables in the loop.  They all support destructuring.

\begin{clauses}

\clausex{with}{~var~ \opt\ |=| ~value~}
Causes ~var~ to be bound to value before the loop body is entered.
If ~value~ is not supplied, ~var~ assumes a default
binding, which will be 
\nil\ in the absence of declarations.  Also, if ~value~ is not
supplied, no destructuring is performed; instead, ~var~ may be a list of
symbols, all of which are given default 
bindings.  If ~value~ is supplied, ~var~ is bound to it, with
destructuring.  

\cpar Because |with|
creates bindings whose scope includes the entire \iter\ form, it is
good style to put all |with| clauses at the beginning.

\cpar Successive occurrences of |with| result in sequential
bindings (as with 
|let*|).  There is no way to obtain parallel bindings; see
section \ref{bindings} for a rationale.  


\clause{|for| ~var~ |=| ~expr~}
\clindexx{for}{=}
On each iteration, ~expr~ is evaluated and ~var~ is set
to its value. 

\cpar This clause may appear to do the same thing as |for\dots next|.
In fact, they are quite different.  |for\dots =| provides only three
services: it sets up a binding for ~var~, sets it to ~expr~ on each
iteration, and makes it possible to use |for\dots previous| with
~var~.  |for\dots next| provides these services in addition to the
ability to turn the driver into a generator.
%Also, the code which sets ~var~ appears in the loop body in the same
%place as the |for\dots =| clause; the code for |for\dots next| appears
%at the top of the loop, as with other drivers (except when being used
%as a generator).

\clause{|for| ~var~ |initially| ~init-expr~ |then| ~then-expr~}
\clindexxx{for}{initially}{then}
Before the loop begins, ~var~ is set to ~init-expr;~ on all
iterations after the first it is set to ~then-expr.~  This clause
must occur at top-level.  ~init-expr~ will be moved outside the loop
body and ~then-expr~ will be moved to the end of the loop body, so
they are subject to code motion problems (see section
\ref{code-movement}). 

\cpar This clause may appear to be similar to |for\dots next|, but in
fact they differ significantly.  |for\dots initially\dots then| is
typically used to give ~var~ its first value before the loop begins,
and subsequent values on following iterations.  This is incompatible
with generators, whose first value and subsequent values must all be
computed by |(next ~var~)|.  Also, the update of ~var~ in 
|for\dots initially\dots then| does not occur at the location of the
clause. 
Use |for\dots initially\dots then| for
one-shot computations where its idiom is more convenient, but use
|for\dots next| for extending \iter\ with new drivers (see section
\ref{extend}).

\clause{|for| ~var~ |first| ~first-expr~ |then| ~then-expr~}
\clindexxx{for}{first}{then}
The first time through the loop, ~var~ is set to ~first-expr;~ on
subsequent iterations, it is set to ~then-expr.~  This differs from
|for\dots initially| in that ~var~ is set to ~first-expr~
inside the loop body, 
so ~first-expr~ may depend on the results of other clauses.  For
instance, 
\begin{program}
(iter (for num in list)
      (for i first num then (1+ i))
      ...)
\end{program}
will set |i| to the first element of |list| on the first
iteration, whereas 
\begin{program}
(iter (for num in list)
      (for i initially num then (1+ i))
      ...)
\end{program}
is probably erroneous; |i| will be bound to |num|'s
default binding (usually \nil) for the first iteration.

\end{clauses}

\begin{note}{Compatibility note}
|loop|'s |for\dots =| works like \iter's, but |loop| used the syntax
|for\dots =\dots then| to mean |for\dots initially\dots then|.  It was
felt that these two operations were sufficiently different to warrant
different keywords.

Also, the |for| in the above three clauses is misleading,
since none is true driver (e.g. none has a corresponding |generate|
form).  |setting| would have been a better choice, but |for| was used
to retain some compatibility with |loop|.  
\end{note}


%%%%%%%%%%%%%%%%%%%%%%%%%%%%%%%%
\subsection{Gathering Clauses}
Many of \iter's clauses accumulate values into a variable, or set a
variable under certain conditions.  At the end of the loop, this
variable contains the desired result.  All these clauses have an
optional |into| keyword, whose argument should be a symbol.  If the
|into| keyword is not supplied, the accumulation variable will be
internally generated and its value will be returned at the end of the
loop; if a variable is specified, that variable is used for the
accumulation, and is not returned as a result---it is up to the user
to return it explicitly, in the loop's epilogue code (see |finally|).
It is safe to examine the accumulation variable during the loop, but
it should not be modified.

These clauses all begin with a verb.  When the verb does
not conflict with an existing Common Lisp function, then it may be
used in either its infinitival or present-participle form (e.g. |sum|,
|summing|).  However, when there is a conflict with Common Lisp, only
the present-participle form may be used (e.g. |unioning|).  This is to
prevent \iter\ clauses from clashing with Common Lisp
functions. 

%although these clauses are described as ``producing a value,'' it is a
%mistake to think of the lisp list representing the clause as a
%value-producing form in the usual way.  clauses may legally be written
%where a value is expected, e.g. |(setq x (sum i))|, but the
%lisp value of a clause in such a context is undefined.

\subsubsection{Reductions}

~Reduction~ is an extremely common iteration pattern in which
the results of successive applications of a
binary operation are accumulated.
For example, a loop that computes the sum of the
elements of a list is performing a reduction with the addition
operation.  This could be written in Common Lisp as  |(reduce \#'+
list)| or with \iter\ as
\begin{program}
(iter (for el in list)
      (sum el))
\end{program}

\begin{clauses}

\clausex{sum}{~expr~ \opt\ |into| ~var~}
Each time through the loop, ~expr~ is evaluated and added to a
variable, which is bound initially to zero.  If ~expr~ has a type,
it is ~not~ used as the type of the sum variable, which is always
|number|.  To get the result variable to be of a more specific
type, use an explicit variable, as in
\begin{program}
(iter (for el in number-list)
      (sum el into x)
      (declare (fixnum x))
      (finally (return x)))
\end{program}

\clausex{multiply}{~expr~ \opt\ |into| ~var~}
Like |sum|, but the initial value of the result variable is $1$,
and the variable is updated by multiplying ~expr~ into it.

\clausex{counting}{~expr~ \opt\ |into| ~var~}
~expr~ is evaluated on each iteration.  If it is \nonnil, the
accumulation variable, initially zero, is incremented.

\startitem
\Clausex{maximize}{~expr~ \opt\ |into| ~var~}
\Clausex{minimize}{~expr~ \opt\ |into| ~var~}
\finishitem
~expr~ is evaluated on each iteration and its extremum (maximum or
minimum) is stored in the accumulation variable.  If ~expr~ is never
evaluated, then the result is \nil\ (if the accumulation variable is
untyped) or $0$ (if it has a numeric type).

\clausex{reducing}{~expr~ |by| ~func~ \opt\ |initial-value| ~init-val~ |into| ~var~}
This is a general way to perform reductions. ~func~ should be a
function of two arguments, the first of which will be the value
computed so far and
the second of which will be the value of ~expr~.  It should return the
new value.
|reducing| is roughly equivalent to the
Common Lisp |(reduce ~func~ ~sequence~ :key ~expr-function~)|, where
~expr-function~ is used to derive values from the successive elements
of ~sequence~.

\cpar If the |reducing| clause is never executed, the result is
undefined.

\cpar It is not necessary to provide an
initial value, but better code can be generated if one is supplied.
Regardless of its location in the \iter\ body, ~init-val~ will be
evaluated before the loop is entered, so it should not depend on any
value computed inside the \iter\ form.  

%\cpar if a ~var~ is not specified, you can get \iter\ to declare the
%type of the internal variable by putting a |the| expression around
%~func.~  see section \ref{types}.

\end{clauses}

%\begin{note}{implementation note}
%in principle, |maximize| and |minimize| can be thought of
%as reductions where the initial value is the smallest (or largest)
%value that the accumulation variable can assume.  because lisp's
%bignums can represent arbitrary integers, these clauses cannot be
%implemented as reductions in general.  if, however, the type of 
%~expr~ or ~var~ can be determined to be a fixnum
%or a float, \iter\ will implement the clause as a true reduction,
%using one of the constants |most-negative-fixnum|, |%most-positive-fixnum|, 
%|most-negative-short-float|, etc. as appropriate.
%\end{note}

\subsubsection{Accumulations}

All the predefined accumulation clauses add values to a sequence.  If
the sequence is a list, they all 
have the property that the partial list is kept in the correct order
and available for inspection at any point in the loop.

\begin{clauses}

\clausex{collect}{~expr~ \opt\ |into| ~var~ |at| ~place~ |result-type| ~type~}
Produces a sequence of the values of ~expr~ on each iteration. ~place~
indicates where the next value of ~expr~ is added to the list and may
be one of the symbols |start|, |beginning| (a synonym for |start|) or
|end|.  The symbol may be quoted, but need not be.  The default is
|end|.  For example,
\begin{program}
(iter (for i from 1 to 5)
      (collect i))
\end{program}
produces |(1 2 3 4 5)|, whereas
\begin{program}
(iter (for i from 1 to 5)
      (collect i at beginning))
\end{program}
produces |(5 4 3 2 1)| (and is likely to be faster in most Common Lisp
implementations).

\cpar If ~type~ is provided, it should be a subtype of |sequence|.
The default is |list|.  Specifying a type other than |list| will
result in |collect| returning a sequence of that type.  ~However,~ the
type of the sequence being constructed when inside the loop body is
undefined when a non-|list| type is specified.  (As with ~place~,
quoting ~type~ is optional.)

\clausex{adjoining}{~expr~ \opt\ |into| ~var~ |test| ~test~ |at| ~place~ |result-type| ~type~}
Like |collect|, but only adds the value of ~expr~ if it is not
already present.  ~test,~ which defaults to |\#'eql|, is
the test to be used with |member|.

\startitem
\Clausex{appending}{~expr~ \opt\ |into| ~var~ |at| ~place~}
\Clausex{nconcing}{\ ~expr~ \opt\ |into| ~var~ |at| ~place~}
\Clausex{unioning}{\ ~expr~ \opt\ |into| ~var~ |test| ~test~ |at| ~place~}
\Clausex{nunioning}{~expr~ \opt\ |into| ~var~ |test| ~test~ |at| ~place~}
\finishitem
These are like |collect|, but behave like the Common Lisp functions
|append|, |nconc|, |union| or |nunion|.  
As in Common Lisp, they work only on lists.  Also as in Common Lisp,
|unioning| and |nunioning| assume that the value of ~expr~ contains no
duplicates. 

\clausex{accumulate}{~expr~ |by| ~func~ \opt\ |initial-value| ~init-val~ |into| ~var~}
This is a general-purpose accumulation clause. ~func~ should be a
function of two arguments, the value of ~expr~ and the value
accumulated so far in the iteration, and it should return the updated
value.  If no initial value is supplied, \nil\ is used.  

%\cpar If a ~var~ is not specified, you can get \iter\ to declare the
%type of the internal variable by putting a |the| expression around
%~func.~  see section \ref{types}.

\cpar The differences between |accumulate| and |reducing| are slight.
One difference is that the functions take their arguments in a
different order.  Another is that in the absence of ~init-val~,
|accumulate| will use \nil, whereas |reducing| will generate different
code that avoids any dependence on the initial value.
The reason for having both clauses is that one usually
thinks of reductions (like |sum|) and accumulations (like |collect|)
as different beasts.

\end{clauses}


\subsubsection{Finders}

A ~finder~ is a clause whose value is an expression that meets some
condition.

\begin{clauses}

\clause{|finding| ~expr~ |such-that| ~test~ \opt\ |into| ~var~ |on-failure| ~failure-value~}
\clindexx{finding}{such-that}
If ~test~ (which is an expression) ever evaluates to \nonnil, the loop
is terminated, the 
epilogue code is run and the value of ~expr~ is returned.  Otherwise,
\nil\ (or ~failure-value,~ if provided) is returned.  If ~var~ is
provided, it will have either the \nonnil\ value of ~expr~ or
~failure-value~ when the epilogue code is run.

\cpar As a special case, if the ~test~ expression is a sharp-quoted
function, then it is applied to ~expr~ instead of being simply
evaluated.  E.g. |(finding x such-that \#'evenp)| is equivalent to
|(finding x such-that (evenp x))|.  

%\cpar although ~test~ need have nothing to do with ~%expr~ as in
%|(finding j such-that (> i 3))|, it usually 
%will: |(finding (length el) such-that (oddp (length el)))|.  to
%avoid performing the |length| computation twice, you could write
%|(finding (length el) such-that \#'oddp)| or |(finding (length
%el) such-that 'oddp)|; for these cases, \iter\ generates code that
%executes ~expr~ only once.  the code for |\#'oddp|
%is slightly different from that for {\lisp 'oddp}; see the discussion
%under {\lisp for\dots in} and {\lisp for\dots on}.

\cpar |On-failure| is a misnomer. Because it is always evaluated, it behaves
more like the default third argument to the |gethash| function. As a result,
|on-failure (error "Not found")| makes no sense. Instead, the clauses |leave|
or |thereis| can be used in conjunction with |finally| as follows:
\begin{program}
(iter (for x in '(1 2 3))
      (if (evenp x) (leave x))
      (finally (error "not found")))
\end{program}

\cpar This clause may appear multiple times when all defaults are
identical. It can also be used together with either |always|/|never| or
|thereis| if their defaults match. More specifically, |on-failure nil| is
compatible with |thereis|, while |on-failure t| is compatible with |always|
and |never| clauses.
\begin{program}
(iter (for i in '(7 -4 2 -3))
      (if (plusp i)
	  (finding    i  such-that (evenp i))
          (finding (- i) such-that (oddp i))))
\end{program}

\startitem
\Clause{|finding| ~expr~ |maximizing| ~m-expr~ \opt\ |into| ~var~}
\clindexx{finding}{maximizing}
\Clause{|finding| ~expr~ |minimizing| ~m-expr~ \opt\ |into| ~var~}
\clindexx{finding}{minimizing}
\finishitem
Computes the extremum (maximum or minimum) value of ~m-expr~ over all
iterations, and returns the value of ~expr~ corresponding to the
extremum.  ~expr~ is evaluated inside the loop at the time the new
extremum is established.  If ~m-expr~ is never evaluated (due to, for
example, being embedded in a conditional clause), then the returned
value depends on the type, if any, of ~expr~ (or ~var,~ if
one is supplied).  If there is no type, the returned
value will be \nil; if the type is numeric, the returned value will be
zero.

\cpar For these two clauses, ~var~ may be a list of two
symbols; in that case, the first is used to record ~expr~ and
the second, ~m-expr.~  

\cpar As with |finding\dots such-that|, if ~m-expr~ is a sharp-quoted
function, then it is called on ~expr~ instead of being evaluated.

\end{clauses}

\subsubsection{Boolean Tests}

\begin{clauses}

\clausex{first-iteration-p}{}
Returns |t| in the first cycle of the loop, otherwise \nil.

\clausex{first-time-p}{}
Returns |t| the first time the expression is evaluated, and then \nil\ forever.
This clause comes handy when printing (optional) elements separated
by a comma:

\begin{program}
(iter (for el in '(nil 1 2 nil 3))
      (when el
        (unless (first-time-p)
          (princ ", "))
        (princ el)))
\end{program}
produces |"1, 2, 3"|.

\end{clauses}

\subsubsection{Aggregated Boolean Tests}

\begin{clauses}

\clausex{always}{~expr~}
If ~expr~ ever evaluates to
\nil, then \nil\ is immediately returned; the epilogue code is not
executed.  If ~expr~ never evaluates to \nil, the epilogue code
is executed and the last value of ~expr~ (or |t| if ~expr~ was never
evaluated) is returned (whereas |loop| would constantly return |t|).

% mention last evaluated clause when multiple always clauses?

\clausex{never}{~expr~}
Like |(always (not ~expr~))|, except it does not influence the last
value returned by a possible other |always| clause. That is,
\begin{program}
(iter (repeat 2)
      (always 2)
      (never nil)) \yields 2 ; not t
\end{program}

\clausex{thereis}{~expr~}
If ~expr~ is ever \nonnil,
its value is immediately returned without running epilogue code.
Otherwise, the epilogue code is performed and \nil\ is returned.

This clause cannot be used together with |always| or |never|, because their
defaults are opposed (similarly, |(loop always 3 thereis nil)| refuses to
compile in some implementations of |loop|).

\end{clauses}


\subsection{Control Flow}
\label{control-flow}
Several clauses can be used to alter the usual flow of control in a loop.

Note: the clauses of this and subsequent sections don't adhere to \iter's
usual syntax, but instead use standard Common Lisp syntax.  Hence the
format for describing syntax subsequently is
like the standard format used in the Common Lisp manual, not like the
descriptions of clauses above.

\begin{clauses}

\clausex{finish}{}
Stops the loop and runs the epilogue code.

%for example:
%
%\begin{program}
%(iter (with answer = nil)
%         (initially (make-a-mess))
%         (for i from 1 to 10)
%         (when (correct? i)
%           (setq answer i)
%           (finish))
%         (finally (cleanup)))
%\end{program}
%
%this code will execute |cleanup| whether or not the test |(correct?
%i)| ever succeeds. 
%the (more elegant) formulation,
%\begin{program}
%(iter (initially (make-a-mess))
%         (for i from 1 to 10)
%         (finding i such-that (correct? i))
%         (finally (cleanup)))
%\end{program}
%would not execute |cleanup| if |(correct? i)| succeeded; it
%would do an immediate return.

\clausex{leave}{\opt\ ~value~}
Immediately returns ~value~ (default \nil) from the current \iter\
form, skipping the epilogue code.  Equivalent to using |return-from|.

\clausex{next-iteration}{}
Skips the remainder of the loop body and begins the next iteration of
the loop.

\clausex{while}{~expr~}
If ~expr~ ever evaluates to \nil, the loop is terminated and the
epilogue code executed.  Equivalent to |(if (not ~expr~) (finish))|.

\clausex{until}{~expr~}
Equivalent to |(if ~expr~ (finish))|.

\clausex{if-first-time}{~then~ \opt\ ~else~}
If this clause is being executed for the first time in this invocation
of the \iter\ form, then the ~then~ code is evaluated; otherwise the
~else~ code is evaluated.

\cpar |(for ~var~ first ~expr1~ then ~expr2~)| is almost equivalent to
\begin{program}
(if-first-time (dsetq ~var~ ~expr1~)
               (dsetq ~var~ ~expr2~))
\end{program}
The only difference is that the |for| version makes ~var~ available
for use with |for\dots previous|.

\end{clauses}

\subsection{Code Placement}
When fine control is desired over where code appears in a loop
generated by \iter, the following special clauses may be useful.
They are all subject to code-motion problems (see section
\ref{code-movement}).  

\begin{clauses}

\clausex{initially}{|\&rest| ~forms~}
The lisp ~forms~ are placed in the prologue section of the loop, where
they are executed once, before the loop body is entered.

\clausex{after-each}{|\&rest| ~forms~}
The ~forms~ are placed at the end of the loop body, where they
are executed after each iteration.  Unlike the other clauses in this
section, ~forms~ may contain \iter\ clauses.

\clausex{else}{|\&rest| ~forms~}
The lisp ~forms~ are placed in the epilogue section of the loop, where they
are executed if this |else| clause is never met during execution of the
loop and the loop terminates normally.

\clausex{finally}{|\&rest| ~forms~}
The lisp ~forms~ are placed in the epilogue section of the loop, where
they are executed after the loop has terminated normally.

\clausex{finally-protected}{|\&rest| ~forms~}
The lisp ~forms~ are placed in the second form of an unwind-protect
outside the loop.  They are always executed after the loop has
terminated, regardless of how the termination occurred.

\end{clauses}


%%%%%%%%%%%%%%%%%%%%%%%%%%%%%%%%%%%%%%%%%%%%%%%%%%%%%%%%%%%%%%%%
\section{Other Features}

\subsection{Multiple Accumulations}
\label{multiple}

\begin{sloppypar}
It is permitted to have more than one clause accumulate into the same
variable, as in the following:
\begin{program}
(iter (for i from 1 to 10)
      (collect i into nums)
      (collect (sqrt i) into nums)
      (finally (return nums)))
\end{program}
Clauses can only accumulate into the same variable if they are
compatible.  |collect|, |adjoining|, |appending|, |nconcing|,
|unioning| and |nunioning| are compatible with each other; |sum|,
|multiply| and |counting| are compatible; |always| and |never| are compatible;
|finding| \dots |such-that| is compatible with either |thereis| or |always|
and |never| when their defaults match; and |maximize| and |minimize| clauses
are compatible only with other |maximize| and |minimize| clauses,
respectively.

%note that the same variable ~cannot~ be both an accumulation
%variable and an ordinary variable; there can be only one variable with
%a given name within an \iter\ form. 
\end{sloppypar}

\subsection{Named Blocks}

Like Common Lisp |block|s, \iter\ forms can be given names.  The
name should be a single symbol, and it must be the first form in the
\iter.  The generated code behaves exactly like a named block; in
particular, |(return-from ~name~)| can be used to exit it:
\begin{program}
(iter fred
      (for i from 1 to 10)
      (iter barney
            (for j from i to 10)
            (if (> (* i j) 17)
                (return-from fred j))))
\end{program}
An \iter\ form that is not given a name is implicitly named \nil.

Sometimes one would like to write an expression in an inner \iter\ form,
but have it processed by an outer \iter\ form.  This is possible with
the |in| clause.  

\begin{clauses}

\clausex{in}{~name~ |\&rest| ~forms~}
Evaluates ~forms~ as if they were part of the \iter\ form named
~name~.  In other words, \iter\ clauses are processed by the \iter\
form named ~name,~ and not by any \iter\ forms that occur inside ~name.~

\cpar As an example, consider the problem of collecting a list of the
elements in a two-dimensional array.  The naive solution,
\begin{program}
(iter (for i below (array-dimension ar 0))
      (iter (for j below (array-dimension ar 1))
            (collect (aref ar i j))))
\end{program}
\noindent is wrong because the list created by the inner \iter\ is simply
ignored by the outer one.  But using |in| we can write:
\begin{program}
(iter outer (for i below (array-dimension ar 0))
      (iter (for j below (array-dimension ar 1))
            (in outer (collect (aref ar i j)))))
\end{program}
\noindent which has the desired result.

\end{clauses}

\subsection{Destructuring}
\label{destructuring}

In many places within \iter\ clauses where a variable is expected, a
list can be written instead.  In these cases, the value to be assigned
is ~destructured~ according to the pattern
described by the list.  As a simple example, the clause
\begin{program}
(for (key . item) in alist)
\end{program}
\noindent will result in |key| being set to the |car| of
each element in |alist|, and |item| being set to the |cdr|.  The
pattern list may be nested to arbitrary depth, and (as the example
shows) need not be terminated with \nil; the only requirement is that
each leaf be a bindable symbol (or \nil, in which case no binding is
generated for that piece of the structure).

Sometimes, you might like to do the equivalent of a
|multiple-value-setq| in a clause.  This 
``multiple-value destructuring'' can be expressed by writing 
\linebreak |(values $pat_1$ $pat_2 \ldots$)| for a destructuring
pattern, as in 
\begin{program}
(for (values (a . b) c d) = (three-valued-function ...))
\end{program}
\begin{sloppypar}
Note that the $pat_i$ can themselves be destructuring patterns (though
not multiple-value destructuring patterns).  You can't do multiple-value
destructuring in a |with| clause; instead wrap the whole \iter\
form in a |multiple-value-bind|.
\end{sloppypar}

\begin{note}{Rationale}
There are subtle interactions between variable declarations and
evaluation order that make the correct implementation of
multiple-value destructuring in a |with| somewhat tricky.
\end{note}

The destructuring feature of \iter\ is available as a separate
mechanism, using the |dsetq| macro:

\begin{clauses}
\defmacro{dsetq}{template expr}
Performs destructuring of ~expr~ using ~template~.   May be used
outside of an \iter\ form. Yields the primary value of ~expr~.

\end{clauses}

\subsection{On-line Help}

\begin{sloppypar}
There is a limited facility for on-line help, in the form of the 
|display-iterate-clauses| function.
\end{sloppypar}

\begin{clauses}

\defun{display-iterate-clauses}{\opt\ clause-spec}
Displays a list of \iter\ clauses.  If ~clause-spec~ is not
provided, all clauses are shown; if it is a symbol, all clauses
beginning with that symbol are shown; and if it is a list of symbols,
all clauses for which ~clause-spec~ is a prefix are shown.

\end{clauses}

\subsection{Parallel Binding and Stepping}
\label{bindings}

The parallel binding and stepping of variables is a feature that
\iter\ does ~not~ have.  This section attempts to provide a rationale.

We say that two variables are bound ~in parallel~ if neither
binding shadows the other.  This is the usual semantics of |let|
(as opposed to |let*|).  Similarly, we can say that iteration
variables are stepped in parallel if neither variable is updated
before the other, conceptually speaking; in other words, if the code
to update each variable can reference the old values of both variables.

|loop| allows parallel binding of variables and parallel stepping of
driver variables.  My view is that if you are depending on the 
serial/parallel distinction, you are doing something obscure.  If you
need to bind 
variables in parallel using |with|, then you must be using a
variable name that shadows a name in the existing lexical environment.
Don't do that.  The most common use for parallel stepping is to track
the values of variables on the previous iteration, but in fact this
does not require parallel stepping at all; the following will work:
\begin{program}
(iter (for current in list)
      (for prev previous current)
      ...)
\end{program}


%%%%%%%%%%%%%%%%%%%%%%%%%%%%%%%%%%%%%%%%%%%%%%%%%%%%%%%%%%%%%%%%
\section{Types and Declarations}
\label{types}

\subsection{Discussion}

Sometimes efficiency dictates that the types of variables be declared.
This type information needs to be communicated to \iter\ so it can
bind variables to appropriate values.  Furthermore, \iter\ must often
generate internal variables invisible to the user; there needs to be a
way for these to be declared.

As an example, consider this code, which will return the number of
odd elements in |number-list|:
\begin{program}
(iter (for el in number-list)
      (count (oddp el)))
\end{program}
In processing this form,
\iter\ will create an internal variable, let us call it |list17|, to
hold the successive |cdr|s of |number-list|, and  
will bind the variable to |number-list|.  It will also generate a
default binding for |el|; only inside the body of the loop will |el|
be set to the |car| of |list17|.  Finally, \iter\ will generate a
variable, call it |result|, to hold the result of the count, and will
bind it to zero.

When dealing with type declarations, \iter\ observes one simple rule:
~it will never generate a declaration unless requested to do so.~  The
reason is that such declarations might mask errors in compiled code by
avoiding error-checks; the resulting problems would be doubly hard to
track down because the declarations would be hidden from the
programmer.  Of course, a compiler might omit error-checks even in the
absence of declarations, though this behavior can usually be avoided,
e.g. by saying |(declaim (optimize (safety 3)))|.

So, the above \iter\ form will generate code with no declarations.
But say we wish to declare the types of |el| and the internal
variables |list17| and |result|.  How is this done?

Declaring the type of |el| is easy, since the programmer knows
the variable's name:
\begin{program}
(iter (for el in number-list)
      (declare (fixnum el))
      (counting (oddp el)))
\end{program}
\iter\ can read variable type declarations like this one.  Before
processing any clauses, it scans the entire top-level form for type
declarations 
and records the types, so that variable bindings can be performed
correctly.  In this case, |el| will be bound to zero
instead of \nil.  Also, \iter\ collects all the top-level declarations
and puts them at the begining of the generated code, so it is not
necessary to place all declarations at the beginning of an \iter\
form; instead, they can be written near the variables whose types they
declare.  

Since \iter\ is not part of the compiler, it will not know
about declarations that occur outside an \iter\ form; these
declarations must be repeated inside the form.

Here is another way we could have declared the type of |el|:
\begin{program}
(iter (for (the fixnum el) in number-list)
      (counting (oddp el)))
\end{program}
\iter\ extends the Common Lisp |the|\lispindex{the}
form to apply to variables as well as value-producing forms; anywhere
a variable is allowed---in a |with| clause, as the iteration
variable in a driver clause, as the |into| argument of an
accumulation clause, even inside a destructuring template---you can
write |(the ~type~ ~symbol~)| instead.

There is one crucial difference between using a |the| form and
actually declaring the variable: explicit declarations are always
placed in the generated code, but type information from a |the|
form is not turned into an actual declaration unless you tell \iter\
to do so using |iterate:declare-variables|.  See below.

\begin{sloppypar}
Declaring the types of internal variables is harder than declaring the
types of explicitly mentioned variables, since their names
are unknown.  You do it by declaring |iterate:declare-variables|
somewhere inside the top level of the \iter\ form.  (This will also
generate declarations for variables declared using |the|.)
\iter\ does not provide much selectivity here: it's all or none. 
And unfortunately, since \iter\ is not privy to compiler information
but instead reads declarations itself, it will not hear if you
|(declaim (iterate:declare-variables))|.  Instead, set the variable 
|iterate::*always-declare-variables*| to |t| at
compile-time, using |eval-when|.
\end{sloppypar}

To determine the appropriate types for internal variables, \iter\ uses
three sources of information:

\begin{itemize}

\item Often, the particular clause dictates a certain type for a
variable; \iter\ will use this information when available.  In the
current example, the variable |list17| will be given the type
|list|, since that is the only type that makes sense; and the
variable |result| will be given the type |fixnum|, on the
assumption that you will not be counting high enough to need bignums.
You can override this assumption only by using and explicitly declaring a
variable: 
\begin{verbatim}
(iter (declare (iterate:declare-variables))
      (for el in number-list)
      (count (oddp el) into my-result)
      (declare (integer my-result))
      (finally (return my-result)))
\end{verbatim}

\begin{sloppypar}
Other examples of the type assumptions that \iter\ makes are: type
|list| for |into| variables of collection clauses; type |list| for
expressions that are to be destructured; type |vector| for the
variable holding the vector in a |for\dots in-vector| clause, and
similarly for |string| and the |for\dots in-string| clause;
and the implementation-dependent type
|(type-of array-dimension-limit)| for the index and limit
variables generated by sequence iteration drivers like |for\dots
in-vector| and |for\dots in-string| (but not |for\dots in-sequence|,
because it may be used to iterate over a list). 
\end{sloppypar}

\item Sometimes, \iter\ will examine expressions and try to determine
their types in a simple-minded way.  If the expression is
self-evaluating (like a number, for instance), \iter\ knows that the
expression's type is the same as the type of the value it denotes, so
it can use that type.  If the expression is of the form |(the ~type~
~expr~)|, \iter\ is smart enough to extract ~type~ and use it.
However, the current version of \iter\ does not  
examine declarations of function result types or do any type
inference.  It will not determine, for 
example, that the type of |(+ 3 4)| is |fixnum|, or even
|number|.

\item In some cases, the type of an internal variable should match the
type of some other variable.  For instance, \iter\ generates an
internal variable for |(f x)| in the
clause |(for i from 1 to (f x))|, and in the absence of other
information will give it the same type as |i|.  If, however, the
expression had been written |(the fixnum (f x))|, then \iter\
would have given the internal variable the type |fixnum|
regardless of |i|'s type.  The type incompatibility errors that
could arise in this situation are not checked for.

\end{itemize}

Note that if you do declare |iterate:declare-variables|, then
\iter\ may declare user variables as well as internal ones if they do
not already have declarations, though only for variables that it
binds.  For instance, in this code:

\begin{program}
(iter (declare (iterate:declare-variables))
      (for i from 1 to 10)
      (collect i into var))
\end{program}
the variable |var| will be declared to be of type |list|.


\subsection{Summary}

\iter\ understands standard Common Lisp variable type declarations
that occur within an \iter\ form and
will pass them through to the generated code.  If the declaration
|(iterate:declare-variables)|\lispindex{declare-variables}
appears at the top level of an 
\iter\ form, or if 
|iterate::*always-declare-variables*|\lispindex{*always-declare-variables*} 
is \nonnil, then \iter\ will use the type information gleaned from user
declarations, self-evaluating expressions and |the| expressions,
combined with reasonable assumptions, to determine variable
types and declare them.


%%%%%%%%%%%%%%%%%%%%%%%%%%%%%%%%%%%%%%%%%%%%%%%%%%%%%%%%%%%%%%%%
\section{Problems with Code Movement}
\label{code-movement}

Some \iter\ clauses, or parts of clauses, result in code being
moved from the location of the clause to other parts of the loop.
Drivers behave this way, as do code-placement clauses like |initially|
and |finally|.  When using these clauses, there is a danger of writing
an expression that makes sense in its apparent location but will be
invalid or have a different meaning in another location.  For example:
\begin{program}
(iter (for i from 1 to 10)
      (let ((x 3))
        (initially (setq x 4))))
\end{program}
While it may appear that the |x| of |(initially (setq x 4))| is the
same as the |x| of |(let ((x 3)) \dots|, in fact they are not:
|initially| moves its code outside the loop body, so |x| would refer
to a global variable.  Here is another example of the same problem:
\begin{program}
(iter (for i from 1 to 10)
      (let ((x 3))
        (collect i into x)))
\end{program}
If this code were executed, |collect| would create a binding for its
|x| at the top level of the \iter\ form that the |let| will shadow.

Happily, \iter\ is smart enough to catch these errors; it
walks all problematical code to ensure that free variables are not
bound inside the loop body, and checks all variables it binds for the
same problem.

However, some errors cannot be caught:

\begin{program}
(iter (with x = 3)
      (for el in list)
      (setq x 1)
      (reducing el by \#'+ initial-value x))
\end{program}
|reducing| moves its |initial-value| argument to the initialization
part of the loop in order to produce more efficient code.  Since
\iter\ does not perform data-flow analysis, it cannot determine that
|x| is changed inside the loop; all it can establish is that |x| is
not bound internally.  Hence this code will not signal an
error and will use $3$ as the initial value of the reduction.

The following list summarizes all cases that are subject to these code
motion and variable-shadowing problems.
\begin{itemize}
\item Any variable for which \iter\ creates a binding, including those
used in |with| and the |into| keyword of many clauses.

\begin{sloppypar}
\item The special clauses which place code: |initially|, |after-each|, |else|,
|finally| and |finally-protected|. 
\end{sloppypar}

\item The variables of a |next| or |do-next| form.

\item The |initially| arguments of |for\dots initially\dots then| and
|for\dots previous|. 

\item The |then| argument of |for\dots initially\dots then|.

\item The |initial-value| arguments of |reducing| and |accumulate|.

\item The |on-failure| argument of |finding\dots such-that|.
\end{itemize}

%%%%%%%%%%%%%%%%%%%%%%%%%%%%%%%%%%%%%%%%%%%%%%%%%%%%%%%%%%%%%%%%
\section{Differences Between {\tt Iterate} and {\tt Loop}}
|loop| contains a great deal of complexity which \iter\ tries to
avoid.  Hence many esoteric features of |loop| don't exist in \iter.
Other features have been carried over, but in a cleaned-up form.  
And of course, many new features have been added; they are not
mentioned in this list.

\begin{itemize}

\item \iter's syntax is more Lisp-like than |loop|'s, having a higher
density of parens.

\item The current implementation of \iter, unlike the current version
of  |loop| (as documented in {\em Common Lisp, 2nd Ed.\/}), is
extensible (see section \ref{extend}).  

\item |loop| puts the updates of all driver variables at the top of
the loop; \iter\ leaves them where the driver clauses appear.

\item While for the most part \iter\ clauses that resemble |loop| clauses
behave similarly, there are some differences.  For instance, there is
no |for\dots =\dots then| in \iter; instead use 
|for\dots initially\dots then|.  

\item |loop| binds the variable |it| at certain times
to allow pseudo-English expressions like |when ~expr~ return it|. 
In \iter, you must bind ~expr~ to a variable yourself.  Note that
|when ~expr~ return it| is like |thereis ~expr~| except that the latter is an
accumulation clause and therefore competes with other accumulations
(remember section \ref{multiple} above).

% repeat different behaviour of |always| clause here?

\item |loop| has a special |return| clause, illustrated in the
previous item.  \iter\ doesn't need one,  since an ordinary Lisp
|return| has the same effect.

\item |loop| allows for parallel binding and stepping of iteration
variables.  \iter\ does not.  (See section \ref{bindings}.)

\item |loop| and \iter\ handle variable type declarations very
differently.  |loop| provides a special syntax for declaring variable
types, and does not examine declarations.  Moreover, the standard
implementation of |loop| will
generate declarations when none are requested.
\iter\ parses standard Common Lisp type declarations, and will never
declare a variable itself unless declarations are 
specifically requested.

\end{itemize}

%%%%%%%%%%%%%%%%%%%%%%%%%%%%%%%%%%%%%%%%%%%%%%%%%%%%%%%%%%%%%%%%
\section{Rolling Your Own}
\label{extend}

\subsection{Introduction}

\iter\ is extensible---you can write new clauses that embody new
iteration patterns.  You might want to write a new driver clause for a
data structure of your own, or you might want to write a clause that
collects or manipulates elements in a way not provided by \iter. 

This section describes how to write clauses for \iter.  Writing a
clause is like writing a macro.  In fact, writing a clause ~is~
writing a macro: since \iter\ code-walks its body and macroexpands,
you can add new abstractions to \iter\ with good old |defmacro|.

Actually, there are two extensions you can make to \iter\ that are
even easier than writing a macro.  They are adding a synonym for an
existing clause and defining a driver clause for an indexable
sequence.  These can be done with |defsynonym| and
|defclause-sequence|, respectively.  See section \ref{aids}, below.

The rest of this section explains how to write macros that expand into
\iter\ clauses.
Here's how you could add a simplified version of \iter's |multiply|
clause, if \iter\ didn't already have one:
\begin{program}
(defmacro multiply (expr)
  `(reducing ,expr by \#'* initial-value 1))
\end{program}

If you found yourself summing the square of an expression often, you
might want to write a macro for that.  A first cut might be
\begin{program}
(defmacro sum-of-squares (expr)
  `(sum (* ,expr ,expr)))
\end{program}
but if you are an experienced macro writer, you will realize that this
code will evaluate ~expr~ twice, which is probably a bad idea.  A
better version would use a temporary:
\begin{program}
(defmacro sum-of-squares (expr)
  (let ((temp (gensym)))
    `(let ((,temp ,expr))
       (sum (* ,temp ,temp)))))
\end{program}
Although this may seem complex, it is just the sort of thing you'd
have to go through to write any macro, which illustrates the point of
this section: if you can write macros, you can extend \iter.

Our macros don't use \iter's keyword-argument
syntax.  We could just use keywords with |defmacro|, but we would
still not be using \iter's clause indexing mechanism.  Unlike Lisp,
which uses just the first symbol of a form to determine what function
to call, \iter\ individuates clauses by the list of required keywords.
For instance, |for\dots in| and |for\dots in-vector| are different clauses
implemented by distinct Lisp functions.  

To buy into this indexing scheme, as well as the keyword-argument
syntax, use |defmacro-clause|:

\begin{clauses}

\defmacro{defmacro-clause}{arglist |\&body| body}
Defines a new \iter\ clause.  ~arglist~ is a list of symbols which are
alternating keywords and arguments.  \opt\ may be used, and the list
may be terminated by |\&sequence|.  ~body~ is an ordinary macro body,
as with |defmacro|.  If the first form of ~body~ is a string, it is
considered a documentation string and will be shown by
|display-iterate-clauses|. |defmacro-clause| will signal an error if 
defining the clause would result in an ambiguity.  E.g. you cannot
define the clause |for\dots from| because there would be no way to
distinguish it from a use of the |for| clause with optional keyword |from|.

\end{clauses}

\medskip

Here is |multiply| using |defmacro-clause|.  The keywords are capitalized
for readability.
\begin{program}
(defmacro-clause (MULTIPLY expr \opt\ INTO var)
  `(reducing ,expr by \#'* into ,var initial-value 1))
\end{program}
You don't have to worry about the case when |var| is not supplied; for
any clause with an |into| keyword, saying |into nil| is equivalent to
omitting the |into| entirely.

As another, more extended example, consider the fairly common
iteration pattern that 
involves finding the sequence element that maximizes (or minimizes) some
function.  \iter\ provides this as |finding\dots maximizing|, but it's
instructive to see how to write it.
Here, in pseudocode, is how you might write such a loop for
maximizing a function F:

\begin{quote}
\begin{tabbing}
set variable MAX-VAL to NIL; \\
set variable WINNER to NIL; \\
for\=\ each element EL in the sequence \\
\>  if\=\ MAX-VAL\=\ is NIL or F(EL) $>$ MAX-VAL then \\
\>\>    set MAX-VAL to F(EL); \\
\>\>    set WINNER to EL; \\
\>  end if; \\
end for; \\
return WINNER.
\end{tabbing}
\end{quote}

Here is the macro:
\begin{program}
(defmacro-clause (FINDING expr MAXIMIZING func \opt\ INTO var)
  (let ((max-val (gensym))
        (temp1 (gensym))
        (temp2 (gensym))
        (winner (or var iterate::*result-var*)))
    `(progn 
       (with ,max-val = nil)
       (with ,winner = nil)
       (cond 
        ((null ,max-val)
         (setq ,winner ,expr)
         (setq ,max-val (funcall ,func ,winner))
        (t
         (let* ((,temp1 ,expr)
                (,temp2 (funcall ,func ,temp1)))
           (when (> ,temp2 ,max-val)
             (setq ,max-val ,temp2)
             (setq ,winner ,temp1))))))
       (finally (leave ,winner)))))
\end{program}
Note that if no |into| variable is supplied, we use
|iterate::*result-var*|, which contains the internal variable into
which all clauses place their results.  If this variable is bound by
some clause, then \iter\ will return its value automatically;
otherwise, \nil\ will be returned.  

\subsection{Writing Drivers}

In principle, drivers can be implemented just as easily as other
\iter\ clauses.  In practice, they are a little harder to get right. 
As an example, consider writing a driver that
iterates over all the 
elements of a vector, ignoring its fill-pointer.  |for\dots in-vector|
won't work for this, because it observes the fill-pointer.  It's necessary to
use |array-dimension| instead of |length| to obtain the size of the
vector.  Here is one approach:

\begin{program}
(defmacro-clause (FOR var IN-WHOLE-VECTOR v)
  "All the elements of a vector (disregards fill-pointer)"
  (let ((vect (gensym))
        (index (gensym)))
    `(progn
       (with ,vect = ,v)
       (for ,index from 0 below (array-dimension ,vect 0))
       (for ,var = (aref ,vect ,index)))))
\end{program}
Note that we immediately put |v| in a variable, in case it is an
expression.  Again, this is just good Lisp macrology.  It also has a
subtle effect on the semantics of the driver: |v| is evaluated only
once, at the beginning of the loop, so changes to |v| in the loop have
no effect on the driver.  Similarly, the bounds for numerical iteration
e.g. the above |array-dimension| are also evaluated once only.  This is how
all of \iter's drivers work.


There is an important point concerning the |progn| in this code.  We
need the |progn|, of course, because we are returning several forms,
one of which is a driver.  But \iter\ drivers must occur at top-level.
Is this code in error?  No, because ~top-level~ is defined in \iter\
to include forms inside a |progn|.  This is just the definition of
top-level that Common Lisp uses, and for the same reason: to allow
macros to return multiple forms at top-level.

While our |for\dots in-whole-vector| clause will work, it is not
ideal.  In particular, it does not support generating.  Do do so, we
need to use |for\dots next| or |for\dots do-next|.  The job is
simplified by the |defmacro-driver| macro.

\begin{clauses}

\defmacro{defmacro-driver}{arglist |\&body| body}
Defines a driver clause in
both the |for| and |generate| forms, and provides a parameter
|generate| which ~body~ can examine to determine how it was invoked.
~arglist~ is as in |defmacro-clause|, and should begin with the symbol
|for|. 

\end{clauses}

With |defmacro-driver|, our driver looks like this:
\begin{program}
(defmacro-driver (FOR var IN-WHOLE-VECTOR v)
  "All the elements of a vector (disregards fill-pointer)"
   (let ((vect (gensym))
         (end (gensym))
         (index (gensym))
         (kwd (if generate 'generate 'for)))
     `(progn
        (with ,vect = ,v)
        (with ,end = (array-dimension ,vect 0))
        (with ,index = -1)
        (,kwd ,var next (progn (incf ,index)
                               (if (>= ,index ,end) (terminate))
                               (aref ,vect ,index))))))
\end{program}


We are still missing one thing: the |\&sequence| keywords.  
We can get them easily enough, by writing 
\begin{program}
(defmacro-driver (FOR var IN-WHOLE-VECTOR v \&sequence) 
  ...)
\end{program}
We can now refer to parameters |from|, |to|, |by|, etc. which contain
either the values for the corresponding keyword, or \nil\ if the
keyword was not supplied.  Implementing the right code for these
keywords is cumbersome but not difficult; it is left as an exercise.
But before you begin, see |defclause-sequence| below for an easier way.
       
\subsection{Extensibility Aids}
\label{aids}

This section documents assorted features that may be of use in
extending \iter.

\begin{clauses}

\defunexpvar{*result-var*}
Holds the variable that is used to return a value as a result of the
\iter\ form.  You may examine this and use it in a |with| clause, but
you should not change it.

\defmacro{defsynonym}{syn word}
 Makes ~syn~ a synonym for the existing \iter\ keyword ~word.~  Only
the first word in each clause can have synonyms. 

\defmacroxx{defclause-sequence}{element-name index-name}{|\&key|
access-fn size-fn sequence-type}{element-type
element-doc-string index-doc-string}  
Provides
a simple way to define sequence clauses.  Generates two
clauses, one for iterating over the sequence's elements, the other
for iterating over its indices.  The first symbol of both
clauses will have print-name |for|.
~element-name~  and ~index-name~ should be symbols.
~element-name~ is the second keyword of the element iterator (typically of
the form 
|in-~sequence-type~|), and ~index-name~ is the second keyword
of the index-iterator (typically of the form
|index-of-~sequence-type~|).  Either name may be 
\nil, in which case the corresponding clause is not defined.  If both
symbols are supplied, they should be in the same package.  The |for|
that begins the clauses will be in this package.

\cpar ~access-fn~ is the function to be used to
access elements of the sequence in the element iterator.  The function
should take two 
arguments, a sequence and an index, and return the appropriate element.
~size-fn~ should denote a function of one argument, a sequence, that
returns its size.  Both ~access-fn~ and ~size-fn~ are required for the
element iterator, but only ~size-fn~ is needed for the index iterator.

\cpar The ~sequence-type~ and ~element-type~ keywords are used to
suggest types for the variables 
used to hold the sequence and the
sequence elements, respectively.  The usual rules about \iter's
treatment of variable type declarations apply (see section \ref{types}).

\cpar ~element-doc-string~ and ~index-doc-string~ are
the documentation strings, for use with |display-iterate-clauses|. 

\cpar The generated element-iterator performs destructuring on the
element variable.

\cpar As an example, the above |for\dots in-whole-vector| example
could have been written:
\begin{program}
(defclause-sequence IN-WHOLE-VECTOR INDEX-OF-WHOLE-VECTOR
  :access-fn 'aref
  :size-fn \#'(lambda (v) (array-dimension v 0))
  :sequence-type 'vector
  :element-type t
  :element-doc-string 
     "Elements of a vector, disregarding fill-pointer"
  :index-doc-string 
     "Indices of vector, disregarding fill-pointer")
\end{program}

\end{clauses}

\subsection{Subtleties}

There are some subtleties to be aware of when writing \iter\ clauses.
First, the code returned by your macros may be |nconc|'ed into a list,
so you should always returned freshly consed lists, rather than
constants.  Second, \iter\ matches clauses by using |eq| on the first
symbol and |string=| on the subsequent ones, so the package of the
first symbol of a clause is relevant.  All of the clauses in this manual
have their first word in the \iter\ package.  
You can use the package system in the usual way to shadow
\iter\ clauses without replacing them.

%%% say more here, about the badness that only the first word of a
%%% clause is packagey.

\section{Non-portable Extensions to Iterate (Contribs)}
\label{contribs}

Currently, there is only one non-portable extension to iterate in the
distribution: iterate-pg. If you have made an extension that depends
on non-portable features, feel free to send them to |asf@boinkor.net|
for inclusion in the iterate distribution.

\subsection{An SQL query driver for iterate}

The pg package by Eric Marsden (see |http://cliki.net/pg|) provides an
interface to the PostgreSQL database. Using the \iterpg\ extension, it
is possible to handle the results of SQL queries using \iter.

This usage example should give you an idea of how to use it:

\begin{program}
(pg:with-pg-connection (c "somedb" "someuser")
  (iter (for (impl version date) in-relation "select * from version"
                                 on-connection *dbconn*)
        (collect version)))
\end{program}

To use the extension via |ASDF|, simply make your system depend on the
|iterate-pg| system instead of the |iterate| system. To load it
manually, use:

\begin{program}
  (asdf:oos 'asdf:load-op :iterate-pg)
\end{program}

%%%%%%%%%%%%%%%%%%%%%%%%%%%%%%%%%%%%%%%%%%%%%%%%%%%%%%%%%%%%%%%%
\section{Obtaining {\tt Iterate}}

\begin{sloppypar}
The information in this chapter is obsolete but included for
completeness's sake; Currently, the most up-to-date information on
\iter\ can be found at |http://boinkor.net/iterate.html|.
\end{sloppypar}

\begin{sloppypar}
\iter\ currently runs on Lisp Machines, and on
HP's, Sun3's and Sparcstations under Lucid.  
\iter\ source and binaries are available at the MIT AI Lab in the
subdirectories of |/src/local/lisplib/|.  The source file,
|iterate.lisp|, is also available for anonymous FTP in the directory
|/com/fpt/pub/| on the machine |TRIX.AI.MIT.EDU| (Internet number
128.52.32.6).  If you are unable to obtain |iterate| in one of these
ways, send mail to |jba@ai.mit.edu| and I will send you the source
file. 
\end{sloppypar}

\begin{sloppypar}
\iter\ resides in the |iterate| package (nickname |iter|).  Just say
\linebreak |(use-package :iterate)| to make all the necessary symbols
available. 
If a symbol is not exported, it appears in this manual with an
``|iterate::|'' prefix.
\end{sloppypar}

Send bug reports to |bug-iterate@ai.mit.edu|.  The |info-iterate|
mailing list will have notices of changes and problems; to have
yourself added, send mail to |info-iterate-request@ai.mit.edu|.


\medskip

\begin{flushleft}
 \bf Acknowledgements
\end{flushleft}
\smallskip

Richard Waters provided invaluable criticism which spurred me to improve
\iter\ greatly.  As early users, David Clemens, Oren Etzioni and Jeff
Siskind helped ferret out many bugs.

%\begin{theindex}
% The index files must be generated with the genindex program
% or from makeindex iterate-manual.idx which creates iterate-manual.ind.

%\baselineskip\normalbaselineskip
%\advance\baselineskip by -2pt

% Fix ``.'' spacing in lists
\documentstyle[12pt,aimemo]{article}

\setlength{\textwidth}{6.0in}
\setlength{\oddsidemargin}{0.25in}
\setlength{\evensidemargin}{0.25in}
\setlength{\topmargin}{0pt}
\setlength{\headsep}{24pt}
\setlength{\textheight}{8.5in}

\let\origpar=\par

% somehow, LaTex fucks with things so that the plain Tex \obeyspaces
% doesn't work.  This is the same, except we use an mbox containing
% a space instead of just a space.

\newcommand{\spc}{\mbox{ }}
\newcommand{\tAb}{\mbox{        }}
\def\obspaces{\catcode`\ =\active\catcode`\^^I=\active}
{\obspaces \global\let =\spc\global\let^^I=\tAb}

\newenvironment{program}{\medskip\samepage\tt\obeylines\obspaces}{\medskip}

% use | for \tt
\catcode`\|=13
\newif\ifvbar
\def|{\ifvbar
         \endgroup\vbarfalse
      \else 
         \begingroup\tt\vbartrue 
      \fi}



\newcommand{\lisp}{\tt}
\newcommand{\iter}{|iterate|}
\newcommand{\iterpg}{|iterate-pg|}
\newcommand{\setf}{|setf|}
\newcommand{\nil}{|nil|}
\newcommand{\nonnil}{non-\nil}
\newcommand{\opt}{{\lisp \&optional}}
\newcommand{\yields}{$\Rightarrow$}



\newbox\Dots
\setbox\Dots=\hbox{\small ...}
\renewcommand{\dots}{\copy\Dots}

% Here's a way to define an environment which will treat all
% paragraphs inside it in the same manner.  It's, as near as I can
% figure, the way that Latex does things.  Using \everypar won't work;
% apparently Latex resets that a lot.  What you have to do is: at the
% top of the file do \let\origpar=\par.  In the
% begin part of the environment, do the parshape (or whatever), then say
% \def\par{{\origpar}}.  The extra braces are necessary.   That's all
% you have to do; at the end of the environment things will return to
% normal automatically.  Why does this all work?  I have no idea.
% Another important thing: there should be a blank line (i.e. a \par)
% between the end of the last paragraph and the \end{environment}.

\newlength{\clindent}
\setlength{\clindent}{0.5in}
\newlength{\clparindent}
\setlength{\clparindent}{\parindent}

\newenvironment{clauses}{\advance\linewidth -\parindent
                        \hangindent\clindent\relax
                         \hangafter=0
                          \parindent=0pt
                          \def\par{{\origpar}}}{}

\newcommand{\cpar}{\hspace \clparindent}

\newcommand{\startitem}{\bigskip\pagebreak[3]}
\newcommand{\finishitem}{\smallskip}

\newcommand{\clause}[1]{\startitem\Clause{#1}\finishitem}
\newcommand{\Clause}[1]{\hbox{#1}}
\newcommand{\clindex}[1]{\index{{\lisp #1}}}
\newcommand{\clindexx}[2]{\index{{\lisp #1\dots #2}}}
\newcommand{\clindexxx}[3]{\index{{\lisp #1\dots #2\dots #3}}}
\newcommand{\clausex}[2]{\startitem\Clausex{#1}{#2}\finishitem}
\newcommand{\Clausex}[2]{\Clause{|#1| #2}\clindex{#1}}

\newcommand{\defvar}[1]{\startitem\deff{#1}{}{Variable}\finishitem}
\newcommand{\defunexpvar}[1]{\startitem\deffnoindex{iterate::#1}{}{Variable}
                                        \lispindex{#1}\finishitem}
\newcommand{\defun}[2]{\startitem\deff{#1}{#2}{Function}\finishitem}
\newcommand{\defmacro}[2]{\startitem\deff{#1}{#2}{Macro}\finishitem}
\newcommand{\defunx}[3]{\startitem\deffx{#1}{#2}{#3}{Function}\finishitem}
\newcommand{\defmacrox}[3]{\startitem\deffx{#1}{#2}{#3}{Macro}\finishitem}
\newcommand{\defunxx}[4]{\startitem\deffxx{#1}{#2}{#3}{#4}{Function}\finishitem}
\newcommand{\defmacroxx}[4]{\startitem\deffxx{#1}{#2}{#3}{#4}{Macro}\finishitem}

\newcommand{\deff}[3]{\deffnoindex{#1}{#2}{#3}\lispindex{#1}}
\newcommand{\deffnoindex}[3]{\hbox to \hsize{{\tt #1} {\it #2}\hfill [{\it #3}]}}
\newcommand{\deffx}[4]{\deff{#1}{#2}{#4}\moreargs{#1}{#3}}
\newcommand{\deffxx}[5]{\deffx{#1}{#2}{#3}{#5}\moreargs{#1}{#4}}

\newcommand{\moreargs}[2]{\hbox to \hsize{\phantom{\lisp #1} {\it #2}\hfill}}

\newcommand{\lispindex}[1]{\index{{\lisp #1}}}

 \newenvironment{note}[1]{\pagebreak[2]\bigskip
                          \hrule\smallskip\small
                         {\setlength{\parindent}{0pt}
                           \par{\bf #1:}}}{\smallskip\hrule\bigskip}



\makeindex

%%%%%%%%%%%%%%%%%%%%%%%%%%%%%%%%%%%%%%%%%%%%%%%%%%%%%%%%%%%%%%%%

%\thispagestyle{empty}

%\vspace*{2in}

%\title{The |Iterate| Manual}
%\author{Jonathan Amsterdam}

%\maketitle

%\thispagestyle{empty}

%\begin{center}
%\Large *** DRAFT ***
%\end{center}
%This manual is in beta-test. It will become an AI Memo in a couple of
%months.  I would appreciate any comments or
%corrections.  You can reach me at email address |jba@ai.ai.mit.edu|.

\begin{document}

\def\writtenby{Jonathan Amsterdam}
\def\contractno{This report describes research done at the Artificial
Intelligence Laboratory of the Massachusetts Institute of Technology.
Support for the laboratory's artificial intelligence research is
provided in part by the Advanced Research Projects Agency of the
Department of Defense under Office
of Naval Research contract N00014-85-K-0124.}

\AIMEMO{1236}{\Large The Iterate Manual \\[1in]}
{This is the manual for version 1.4 of \iter, a powerful iteration
macro for Common  
Lisp. \iter\ is similar to {\tt loop} but provides numerous additional
features, is well integrated with Lisp, and is extensible. 
\vfill }


% use ~ for \em
\catcode`\~=13
\newif\ifem
\def~{\ifem
         \/\endgroup\emfalse
      \else 
         \begingroup\em\emtrue 
      \fi}



%\pagebreak
%\setcounter{page}{1}

%\thispagestyle{empty}


\tableofcontents

\pagebreak
\pagestyle{headings}

\section{Introduction}

\begin{sloppypar}
This manual describes
\iter, a powerful iteration facility for Common Lisp.
\iter\ provides abstractions for many common iteration
patterns and allows for the definition of additional patterns.  \iter\ is
a macro that expands into ordinary Lisp at
compile-time, so it is more efficient than higher-order
functions like |map| and |reduce.|  
While it is similar to |loop|, \iter\ offers a more Lisp-like syntax
and enhanced extensibility.
(For a more complete
comparison of \iter\ with other iteration constructs, see MIT AI Lab Working
Paper No. 324, ~Don't Loop, Iterate.~)
\end{sloppypar}

An \iter\ form consists of the symbol |iter|\footnote{You can also use
|iterate|, but |iter| is preferred because it avoids potential conflicts with
possible future additions to Common Lisp, and because it saves
horizontal space when writing code.}
followed by one or more
forms, some of which may be \iter\ 
~clauses.~  Here is a simple example of \iter\ which collects the
numbers from 1 to 10 into a list, and returns the list.  The return
value is shown following the arrow.
\begin{program}
(iter (for i from 1 to 10)
      (collect i))          \yields (1 2 3 4 5 6 7 8 9 10)
\end{program}
This form contains two clauses: a |for| clause that steps the
variable |i| over the integers from 1 to 10, and a |collect|
clause that accumulates its argument into a list.  With a few
exceptions, all \iter\ clauses have the same format: alternating
symbols (called ~keywords~) and expressions (called
~arguments~).  The syntax and terminology are those of Common
Lisp's keyword lambda lists.  One difference is that \iter's keywords
do not have to begin with a colon---though they may, except for the
first symbol of a clause.  So you can also write  |(for i :from
1 :to 10)| if you prefer.

Any Lisp form can appear in the body of an \iter, where it will have
its usual meaning.  \iter\ walks the entire body, expanding macros,
and recognizing clauses at any level.  This example collects all the
odd numbers in a list:
\begin{program}
(iter (for el in list)
      (if (and (numberp el) (oddp el))
          (collect el)))
\end{program}

There are clauses for iterating over numbers, lists, arrays and other
objects, and for
collecting, summing, counting, maximizing and other useful operations.
\iter\ also supports the creation of new variable bindings, stepping
over multiple sequences at once, destructuring, and compiler
declarations of variable types.  The following example illustrates
some of these features: 
\begin{program}
(iter (for (key . item) in alist)
         (for i from 0)
         (declare (fixnum i))
         (collect (cons i key)))
\end{program}

This loop takes the keys of an alist and returns a new alist
associating the keys with their positions in the original list. The
compiler declaration for |i| will appear in the generated code
in the appropriate place.

%%%%%%%%%%%%%%%%%%%%%%%%%%%%%%%%%%%%%%%%%%%%%%%%%%%%%%%%%%%%%%%%
\section{Clauses}

Most of \iter's clauses will be familiar to |loop| programmers.  
(|loop| is an iteration macro that has been incorporated into Common
Lisp.  Seen Guy Steele's {\em Common Lisp, 2nd Edition}.)
In
nearly all cases they behave the same as their |loop| counterparts, so
a |loop| user can switch to \iter\ with little pain (and much gain).

All clauses with the standard keyword-argument syntax consist of two
parts: a ~required~ part, 
containing keywords that must be present and in the right order; and
an ~optional~ part, containing keywords that may be omitted and,
if present, may occur in any order.  In the descriptions below, the
parts are separated by the Lisp lambda-list keyword \opt.


\subsection{Drivers}

An iteration-driving clause
conceptually causes the iteration to go forward.  Driver clauses in
\iter\ allow iteration over numbers, lists, vectors, hashtables, packages,
files and streams.  Iteration-driving clauses must
appear at the top level of an \iter\ form; they cannot be nested
inside another clause.  The driver variable is updated at the point
where the driver clause occurs.  Before the clause is executed for the
first time, the value of the variable is undefined.

%Also, regardless of where the driver clause
%appears in the body, the driver variable is stepped at the top of the
%loop; hence it is stylistically preferable, though
%not required, to place driver clauses at the beginning of the \iter.

Multiple drivers may appear in a single \iter\ form, in which case all
of the driver variables are updated each time through the loop, in the
order in which the clauses appear.  The first driver to terminate will
terminate the entire loop.

In all cases, the value of the driver variable on exit from the loop,
including within the epilogue code (see the |finally| clause), is
undefined.

All the parameters of a driver clause are evaluated once, before the
loop begins.  Hence it is not possible to change the bounds or other
properties of an iteration by side-effect from within the loop.

With one exception, driver clauses begin with the word |for| (or
the synonym |as|) and mention an iteration variable, which is
given a binding within the \iter\ form.
The exception is |repeat|, which just executes a loop a
specified number of times:

\begin{clauses}

\clausex{repeat}{~n~}
Repeats the loop ~n~ times.  For example:
\begin{program}
(iter (repeat 100)
      (print "I will not talk in class."))
\end{program}
\cpar If $n \leq 0$, the
loop will never be executed.  If ~n~ is not an integer, the
actual number of executions will be $\lceil n \rceil$.

\end{clauses}

\pagebreak[3]

\subsubsection{Numerical Iteration}

\begin{clauses}
\clausex{for}{~var~ |\&sequence|} 
The general form for iterating over a sequence of numbers requires a
variable and, optionally, one or more keywords that provide the bounds
and step size of the iteration. The |\&sequence|
lambda-list keyword is a shorthand for these sequence keywords.  They are:
|from|, |upfrom|, |downfrom|, |to|, |downto|, |above|, |below| and |by|.
|from| provides the starting value for ~var~
and defaults to zero. |to| provides a final value and implies that the
successive values of ~var~ will be increasing; |downto| implies that
they will be decreasing.  The loop terminates when ~var~ passes
the final value (i.e. becomes smaller or larger than it, depending on
the direction of iteration); in other  words, the loop body
will never be 
executed for values of ~var~ past the final value.
|below| and |above| are similar to |to| and
|downto|, except that the loop terminates when ~var~ equals or passes
the final value.

\cpar If no final value is specified, the variable will be stepped
forever.  Using |from| or |upfrom| will result in increasing
values, while |downfrom| will give decreasing values.

\cpar On each iteration,
~var~ is incremented or decremented by the value of the sequence
keyword |by|, which defaults to 1.  It should always be a positive
number, even for downward iterations.

\cpar In the following examples, the sequence of numbers generated is shown
next to the clause.

\begin{program}
(for i upfrom 0) \yields\ 0 1 2 \ldots
(for i from 5)   \yields\ 5 6 7 \ldots    ; either from or upfrom is okay
(for i downfrom 0) \yields\ 0 -1 -2 \ldots
(for i from 1 to 3) \yields\ 1 2 3
(for i from 1 below 3) \yields\ 1 2
(for i from 1 to 3 by 2) \yields\ 1 3
(for i from 1 below 3 by 2) \yields\ 1
(for i from 5 downto 3) \yields\ 5 4 3
\end{program}

\end{clauses}

\subsubsection{Sequence Iteration}

There are a number of clauses for iterating over sequences.  In all of
them, the argument following |for| may be a list instead of a
symbol, in which case destructuring is performed.  See section
\ref{destructuring}. 

\begin{clauses}

\clause{|for| ~var~ |in| ~list~ \opt\ |by| ~step-function~}
\clindexx{for}{in}
~var~ is set to successive elements of list.  
~step-function,~ which 
defaults to |cdr|, is used to obtain the next sublist. 

\clause{|for| ~var~ |on| ~list~ \opt\ |by| ~step-function~}
\clindexx{for}{on}
~var~ is set to successive sublists of list. 
~step-function~ (default |cdr|) is used as in |for\dots in|.

\end{clauses}

\medskip

These two clauses use |atom| to test for the end
of a list.  Hence, given a list whose final |cdr| is not \nil,
they will silently ignore the last |cdr|.  Other choices are
|endp|,  which would signal an error, and |null|, which
would probably result in an error somewhere else.  If you wish to use
an end-test other than |atom|, set the variable
|iterate::*list-end-test*|\lispindex{*list-end-test*} to the name of
the desired function. 

\begin{clauses}
\clause{|for| ~var~ |in-vector| ~vector~ |\&sequence|}
\clindexx{for}{in-vector}
~var~ takes on successive elements from ~vector.~   The vector's
fill-pointer is observed.  Here and in subsequent clauses, the |\&sequence|
keywords include |with-index|, which takes a symbol as argument
and uses it for the index variable instead of an internally generated
symbol.  The other |\&sequence| keywords behave as in numerical
iteration, except that the default iteration bounds are the bounds of
the vector.  E.g. in |(for i in-vector v downto 3)|,
|i| will start off being bound to the last element in |v|,
and will  
be set to preceding elements down to and including the element with
index 3.

\clause{|for| ~var~ |in-sequence| ~seq~ |\&sequence|}
\clindexx{for}{in-sequence}
This uses Common Lisp's generalized sequence functions, |elt|
and |length|, to obtain elements and determine the length of
~seq.~   Hence it will work for any sequence, including lists, and will
observe the fill-pointers of vectors.

\clause{|for| ~var~ |in-string| ~string~ |\&sequence|}
\clindexx{for}{in-string}
~var~ is set to successive characters of ~string.~

\startitem
\Clause{|for| ~var~ |index-of-vector| ~vector~ |\&sequence|}
\clindexx{for}{index-of-vector}
\Clause{|for| ~var~ |index-of-sequence| ~sequence~ |\&sequence|}
\clindexx{for}{index-of-sequence}
\Clause{|for| ~var~ |index-of-string| ~string~ |\&sequence|}
\clindexx{for}{index-of-string}
\finishitem
~var~ is set to successive indices of the sequence.
These clauses avoid the overhead of accessing the sequence elements
for those applications where they do not need to be examined, or are
examined rarely.  They admit all the optional keywords of the other
sequence drivers except the (redundant) |with-index| keyword.

\clause{|for| ~(key value)~ |in-hashtable| ~table~}
\clindexx{for}{in-hashtable}

~key~ and ~value,~ which must appear as shown in a list
and may be destructuring templates, are set to the keys and values of
~table~.  If ~key~ is |nil|, then the hashtable's keys will be ignored;
similarly for ~value~.
The order in which 
elements of ~table~ will be retrieved is unpredictable.

\clause{|for| ~var~ |in-package| ~package~ \opt\ |external-only| ~ext~}
\clindexx{for}{in-package}
Iterates over all the symbols in ~package,~ or over only the
external symbols if ~ext~ is specified and non-|nil|.  ~ext~ is not
evaluated.  The same symbol may appear more than once.

\clause{|for| ~var~ |in-packages| ~packages~ \opt\ |having-access| ~symbol-types~}
\clindexx{for}{in-packages}
Iterates over all the symbols from the list of packages denoted by the
descriptor ~packages~ and having accessibility (or visibility) given by
~symbol-types~.  This defaults to the list |(:external :internal :inherited)|
and is not evaluated.  ~var~ must be a list of up to three variables: in each
iteration, these will be set to a symbol, its access-type and package (as per
|with-package-iterator| in ANSI-CL).  The same symbol may appear more than
once.

\end{clauses}

\begin{clauses}

\clause{|for| ~var~ |in-file| ~name~ \opt\ |using| ~reader~}
\clindexx{for}{in-file}
Opens the file ~name~ (which may be a string or pathname) for input,
and iterates 
over its contents. ~reader~ defaults to |read|, so by default ~var~
will be bound to the successive forms in the file.  The \iter\ body is
wrapped in an unwind-protect to ensure that the file is closed no
matter how the \iter\ is exited.

\clause{|for| ~var~ |in-stream| ~stream~ \opt\ |using| ~reader~}
\clindexx{for}{in-stream}
Like |for\dots in-file|, except that ~stream~ should be an existing
stream object that supports input operations.

\end{clauses}

\subsubsection{Generalized Drivers}

These are primarily useful for writing drivers that can also be used
as generators (see section \ref{generators}, below).
 
\begin{clauses}

\clause{|for| ~var~ |next| ~expr~}
\clindexx{for}{next}

~var~ is set to ~expr~ each time through the loop.  Destructuring is
performed.  When the clause is used as a generator, ~expr~ is the code
that is executed when |(next ~var~)| is 
encountered (see section \ref{generators}, below).
~expr~ should compute the first value for ~var~, as well
as all subsequent values, and is responsible for terminating the loop.
For compatibility with future versions of \iter, this termination
should be done with |terminate|\clindex{terminate}, which can be
considered a synonym for |finish| (see section \ref{control-flow}).

\cpar As an example, the following clauses are equivalent to |(for
i from 1 to 10)|:
\begin{program}
(initially (setq i 0))
(for i next (if (> i 10) (terminate) (incf i)))
\end{program}

\clause{|for| ~var~ |do-next| ~form~}
\clindexx{for}{do-next}
~form~ is evaluated each time through the loop.  Its value is ~not~
set to ~var~; that is ~form's~ job. ~var~ is only present so that
\iter\ knows it is a driver variable.  \linebreak |(for ~var~ next
~expr~)| is equivalent to |(for ~var~ do-next (dsetq ~var~ ~expr~))|.
(See section \ref{destructuring} for an explanation of |dsetq|.) 

\end{clauses}

\subsubsection{Generators}
\label{generators}

In all of the above clauses, the driver variable is updated on each
iteration.
Sometimes it is desirable to have greater control over updating.  
For instance, consider the problem of associating numbers, in
increasing order and with no gaps, with the
\nonnil\ elements of a list.  One obvious first pass at writing this is:

\begin{program}
(iter (for el in list)
      (for i upfrom 1)
      (if el (collect (cons el i))))
\end{program}
But on the list |(a b nil c)| this produces |((a . 1) (b . 2) (c .
4))| instead of the desired |((a . 1) (b . 2) (c . 3))|.  The problem
is that |i| is incremented each time through the loop, even when |el|
is |nil|.  

The problem could be solved elegantly if we could step |i| only when
we wished to.  This
can be accomplished for any \iter\ driver by writing |generate| (or
its synonym |generating|)
instead of |for|.  Doing so produces a ~generator~---a driver whose
values are yielded explicitly.  To obtain the next value of a
generator variable ~v~, write \linebreak |(next ~v~)|.  The value of a |next|
form is the next value of ~v~, as determined by its associated driver
clause.  |next| also has the side-effect of updating ~v~ to that
value.  If there is no next value, |next| will terminate the loop,
just as with a normal driver.


Using generators, we can now write our example like this:
\begin{program}
(iter (for el in list)
      (generate i upfrom 1)
      (if el (collect (cons el (next i)))))
\end{program}
Now |i| is updated only when |(next i)| is executed, and this occurs
only when |el| is \nonnil.

To better understand the relationship between ordinary drivers and
generators, observe that we can rewrite an ordinary driver using its
generator form immediately followed by |next|, as this example shows:
\begin{program}
(iter (generating i from 1 to 10)
      (next i)
      ...)
\end{program}
Provided that the loop body contains no |(next i)| forms, this will
behave just as if we had written |(for i from 1 to 10)|.

We can still refer to a driver variable ~v~ without using |next|; in
this case, its value is that given to it by the last evaluation of
|(next ~v~)|.  Before |(next ~v~)| has been called the first time, the
value of ~v~ is undefined.  

This semantics is more flexible than
one in which ~v~ begins the loop bound to its first value and calls
of |next| supply subsequent values, because it means the loop will not
terminate too soon if the generator's sequence is empty.  For
instance, consider the following code, which tags \nonnil\ elements of
a list using a list of tags, and also counts the null elements.
(We assume there are at least as many tags as \nonnil\ elements.)
\begin{program}
(let* ((counter 0)
       (tagged-list (iter (for el in list)
                          (generating tag in tag-list) 
                          (if (null el)
                              (incf counter)
                              (collect (cons el (next tag)))))))
  ...)
\end{program}
It may be that there are just as many tags as non-null elements of
|list|.  If all the elements of |list| are null, we still want the
counting to proceed, even though |tag-list| is \nil.  If |tag| had to be
assigned its first value before the loop begins, we would have had to
terminate the loop before the first iteration, since when |tag-list|
is \nil, |tag| 
has no first value.  With the existing semantics, however, |(next
tag)| will never execute, so the iteration will cover all the elements of
|list|. 
        
When the ``variable'' of a driver clause is actually a destructuring
template containing several variables, all the variables are eligible
for use with |next|.  As before, |(next ~v~)| evaluates to ~v's~ next
value; but the effect is to update all of the template's variables.
For instance, the following code will return the list |(a 2 c)|.
\begin{program}
(iter (generating (key . item) in '((a . 1) (b . 2) (c . 3)))
      (collect (next key))
      (collect (next item)))
\end{program}

Only driver clauses with variables can be made into generators.  This
includes all clauses mentioned so far except for |repeat|.  It does
~not~ include |for\dots previous|, |for\dots =|,
|for\dots initially\dots then| or |for\dots first\dots then| (see
below). 

\subsubsection{Previous Values of Driver Variables}

Often one would like to access the value of a variable on a previous
iteration.  \iter\ provides a special clause for accomplishing this.

\begin{clauses}

\clause{|for| ~pvar~ |previous| ~var~ \opt\ |initially| ~init~
|back| ~n~}
\clindexx{for}{previous}
Sets ~pvar~ to the previous value of ~var,~ which should be a driver
variable, a variable from another |for\dots previous| clause, or a
variable established by a |for\dots =|, 
|for\dots initially\dots then| or |for\dots first\dots then| clause
(see section \ref{setting}).
Initially, ~pvar~ is given the value ~init~ (which defaults to \nil).  
The ~init~ expression will be moved outside the loop body, so it
should not depend on anything computed within the loop.
~pvar~ retains the value of ~init~ until ~var~ is set to its second
value, at which point ~pvar~ is set to ~var's~ first value; and so on.  

\cpar The
argument ~n~ to |back| must be a constant, positive integer, and
defaults to 1.  It determines how many iterations back ~pvar~ should
track ~var~.  For example, when ~n~ is 2, then ~pvar~ will be assigned
~var's~ first value when ~var~ is set to its third value.

\cpar A |for\dots previous| clause may occur after or before its
associated driver clause. |for\dots previous| works with generators as
well as ordinary drivers.

\pagebreak[3]

\cpar Example:
\begin{program}
(iter (for el in '(1 2 3 4))
      (for p-el previous el)
      (for pp-el previous p-el initially 0)
      (collect pp-el))
\end{program}
This evaluates to |(0 0 1 2)|.  It could have been written more
economically as 
\begin{program}
(iter (for el in '(1 2 3 4))
      (for pp-el previous el back 2 initially 0)
      (collect pp-el))
\end{program}


\end{clauses}

\subsection{Variable Binding and Setting}
\label{setting}

Several clauses exist for establishing new variable bindings or for
setting variables in the loop.  They all support destructuring.

\begin{clauses}

\clausex{with}{~var~ \opt\ |=| ~value~}
Causes ~var~ to be bound to value before the loop body is entered.
If ~value~ is not supplied, ~var~ assumes a default
binding, which will be 
\nil\ in the absence of declarations.  Also, if ~value~ is not
supplied, no destructuring is performed; instead, ~var~ may be a list of
symbols, all of which are given default 
bindings.  If ~value~ is supplied, ~var~ is bound to it, with
destructuring.  

\cpar Because |with|
creates bindings whose scope includes the entire \iter\ form, it is
good style to put all |with| clauses at the beginning.

\cpar Successive occurrences of |with| result in sequential
bindings (as with 
|let*|).  There is no way to obtain parallel bindings; see
section \ref{bindings} for a rationale.  


\clause{|for| ~var~ |=| ~expr~}
\clindexx{for}{=}
On each iteration, ~expr~ is evaluated and ~var~ is set
to its value. 

\cpar This clause may appear to do the same thing as |for\dots next|.
In fact, they are quite different.  |for\dots =| provides only three
services: it sets up a binding for ~var~, sets it to ~expr~ on each
iteration, and makes it possible to use |for\dots previous| with
~var~.  |for\dots next| provides these services in addition to the
ability to turn the driver into a generator.
%Also, the code which sets ~var~ appears in the loop body in the same
%place as the |for\dots =| clause; the code for |for\dots next| appears
%at the top of the loop, as with other drivers (except when being used
%as a generator).

\clause{|for| ~var~ |initially| ~init-expr~ |then| ~then-expr~}
\clindexxx{for}{initially}{then}
Before the loop begins, ~var~ is set to ~init-expr;~ on all
iterations after the first it is set to ~then-expr.~  This clause
must occur at top-level.  ~init-expr~ will be moved outside the loop
body and ~then-expr~ will be moved to the end of the loop body, so
they are subject to code motion problems (see section
\ref{code-movement}). 

\cpar This clause may appear to be similar to |for\dots next|, but in
fact they differ significantly.  |for\dots initially\dots then| is
typically used to give ~var~ its first value before the loop begins,
and subsequent values on following iterations.  This is incompatible
with generators, whose first value and subsequent values must all be
computed by |(next ~var~)|.  Also, the update of ~var~ in 
|for\dots initially\dots then| does not occur at the location of the
clause. 
Use |for\dots initially\dots then| for
one-shot computations where its idiom is more convenient, but use
|for\dots next| for extending \iter\ with new drivers (see section
\ref{extend}).

\clause{|for| ~var~ |first| ~first-expr~ |then| ~then-expr~}
\clindexxx{for}{first}{then}
The first time through the loop, ~var~ is set to ~first-expr;~ on
subsequent iterations, it is set to ~then-expr.~  This differs from
|for\dots initially| in that ~var~ is set to ~first-expr~
inside the loop body, 
so ~first-expr~ may depend on the results of other clauses.  For
instance, 
\begin{program}
(iter (for num in list)
      (for i first num then (1+ i))
      ...)
\end{program}
will set |i| to the first element of |list| on the first
iteration, whereas 
\begin{program}
(iter (for num in list)
      (for i initially num then (1+ i))
      ...)
\end{program}
is probably erroneous; |i| will be bound to |num|'s
default binding (usually \nil) for the first iteration.

\end{clauses}

\begin{note}{Compatibility note}
|loop|'s |for\dots =| works like \iter's, but |loop| used the syntax
|for\dots =\dots then| to mean |for\dots initially\dots then|.  It was
felt that these two operations were sufficiently different to warrant
different keywords.

Also, the |for| in the above three clauses is misleading,
since none is true driver (e.g. none has a corresponding |generate|
form).  |setting| would have been a better choice, but |for| was used
to retain some compatibility with |loop|.  
\end{note}


%%%%%%%%%%%%%%%%%%%%%%%%%%%%%%%%
\subsection{Gathering Clauses}
Many of \iter's clauses accumulate values into a variable, or set a
variable under certain conditions.  At the end of the loop, this
variable contains the desired result.  All these clauses have an
optional |into| keyword, whose argument should be a symbol.  If the
|into| keyword is not supplied, the accumulation variable will be
internally generated and its value will be returned at the end of the
loop; if a variable is specified, that variable is used for the
accumulation, and is not returned as a result---it is up to the user
to return it explicitly, in the loop's epilogue code (see |finally|).
It is safe to examine the accumulation variable during the loop, but
it should not be modified.

These clauses all begin with a verb.  When the verb does
not conflict with an existing Common Lisp function, then it may be
used in either its infinitival or present-participle form (e.g. |sum|,
|summing|).  However, when there is a conflict with Common Lisp, only
the present-participle form may be used (e.g. |unioning|).  This is to
prevent \iter\ clauses from clashing with Common Lisp
functions. 

%although these clauses are described as ``producing a value,'' it is a
%mistake to think of the lisp list representing the clause as a
%value-producing form in the usual way.  clauses may legally be written
%where a value is expected, e.g. |(setq x (sum i))|, but the
%lisp value of a clause in such a context is undefined.

\subsubsection{Reductions}

~Reduction~ is an extremely common iteration pattern in which
the results of successive applications of a
binary operation are accumulated.
For example, a loop that computes the sum of the
elements of a list is performing a reduction with the addition
operation.  This could be written in Common Lisp as  |(reduce \#'+
list)| or with \iter\ as
\begin{program}
(iter (for el in list)
      (sum el))
\end{program}

\begin{clauses}

\clausex{sum}{~expr~ \opt\ |into| ~var~}
Each time through the loop, ~expr~ is evaluated and added to a
variable, which is bound initially to zero.  If ~expr~ has a type,
it is ~not~ used as the type of the sum variable, which is always
|number|.  To get the result variable to be of a more specific
type, use an explicit variable, as in
\begin{program}
(iter (for el in number-list)
      (sum el into x)
      (declare (fixnum x))
      (finally (return x)))
\end{program}

\clausex{multiply}{~expr~ \opt\ |into| ~var~}
Like |sum|, but the initial value of the result variable is $1$,
and the variable is updated by multiplying ~expr~ into it.

\clausex{counting}{~expr~ \opt\ |into| ~var~}
~expr~ is evaluated on each iteration.  If it is \nonnil, the
accumulation variable, initially zero, is incremented.

\startitem
\Clausex{maximize}{~expr~ \opt\ |into| ~var~}
\Clausex{minimize}{~expr~ \opt\ |into| ~var~}
\finishitem
~expr~ is evaluated on each iteration and its extremum (maximum or
minimum) is stored in the accumulation variable.  If ~expr~ is never
evaluated, then the result is \nil\ (if the accumulation variable is
untyped) or $0$ (if it has a numeric type).

\clausex{reducing}{~expr~ |by| ~func~ \opt\ |initial-value| ~init-val~ |into| ~var~}
This is a general way to perform reductions. ~func~ should be a
function of two arguments, the first of which will be the value
computed so far and
the second of which will be the value of ~expr~.  It should return the
new value.
|reducing| is roughly equivalent to the
Common Lisp |(reduce ~func~ ~sequence~ :key ~expr-function~)|, where
~expr-function~ is used to derive values from the successive elements
of ~sequence~.

\cpar If the |reducing| clause is never executed, the result is
undefined.

\cpar It is not necessary to provide an
initial value, but better code can be generated if one is supplied.
Regardless of its location in the \iter\ body, ~init-val~ will be
evaluated before the loop is entered, so it should not depend on any
value computed inside the \iter\ form.  

%\cpar if a ~var~ is not specified, you can get \iter\ to declare the
%type of the internal variable by putting a |the| expression around
%~func.~  see section \ref{types}.

\end{clauses}

%\begin{note}{implementation note}
%in principle, |maximize| and |minimize| can be thought of
%as reductions where the initial value is the smallest (or largest)
%value that the accumulation variable can assume.  because lisp's
%bignums can represent arbitrary integers, these clauses cannot be
%implemented as reductions in general.  if, however, the type of 
%~expr~ or ~var~ can be determined to be a fixnum
%or a float, \iter\ will implement the clause as a true reduction,
%using one of the constants |most-negative-fixnum|, |%most-positive-fixnum|, 
%|most-negative-short-float|, etc. as appropriate.
%\end{note}

\subsubsection{Accumulations}

All the predefined accumulation clauses add values to a sequence.  If
the sequence is a list, they all 
have the property that the partial list is kept in the correct order
and available for inspection at any point in the loop.

\begin{clauses}

\clausex{collect}{~expr~ \opt\ |into| ~var~ |at| ~place~ |result-type| ~type~}
Produces a sequence of the values of ~expr~ on each iteration. ~place~
indicates where the next value of ~expr~ is added to the list and may
be one of the symbols |start|, |beginning| (a synonym for |start|) or
|end|.  The symbol may be quoted, but need not be.  The default is
|end|.  For example,
\begin{program}
(iter (for i from 1 to 5)
      (collect i))
\end{program}
produces |(1 2 3 4 5)|, whereas
\begin{program}
(iter (for i from 1 to 5)
      (collect i at beginning))
\end{program}
produces |(5 4 3 2 1)| (and is likely to be faster in most Common Lisp
implementations).

\cpar If ~type~ is provided, it should be a subtype of |sequence|.
The default is |list|.  Specifying a type other than |list| will
result in |collect| returning a sequence of that type.  ~However,~ the
type of the sequence being constructed when inside the loop body is
undefined when a non-|list| type is specified.  (As with ~place~,
quoting ~type~ is optional.)

\clausex{adjoining}{~expr~ \opt\ |into| ~var~ |test| ~test~ |at| ~place~ |result-type| ~type~}
Like |collect|, but only adds the value of ~expr~ if it is not
already present.  ~test,~ which defaults to |\#'eql|, is
the test to be used with |member|.

\startitem
\Clausex{appending}{~expr~ \opt\ |into| ~var~ |at| ~place~}
\Clausex{nconcing}{\ ~expr~ \opt\ |into| ~var~ |at| ~place~}
\Clausex{unioning}{\ ~expr~ \opt\ |into| ~var~ |test| ~test~ |at| ~place~}
\Clausex{nunioning}{~expr~ \opt\ |into| ~var~ |test| ~test~ |at| ~place~}
\finishitem
These are like |collect|, but behave like the Common Lisp functions
|append|, |nconc|, |union| or |nunion|.  
As in Common Lisp, they work only on lists.  Also as in Common Lisp,
|unioning| and |nunioning| assume that the value of ~expr~ contains no
duplicates. 

\clausex{accumulate}{~expr~ |by| ~func~ \opt\ |initial-value| ~init-val~ |into| ~var~}
This is a general-purpose accumulation clause. ~func~ should be a
function of two arguments, the value of ~expr~ and the value
accumulated so far in the iteration, and it should return the updated
value.  If no initial value is supplied, \nil\ is used.  

%\cpar If a ~var~ is not specified, you can get \iter\ to declare the
%type of the internal variable by putting a |the| expression around
%~func.~  see section \ref{types}.

\cpar The differences between |accumulate| and |reducing| are slight.
One difference is that the functions take their arguments in a
different order.  Another is that in the absence of ~init-val~,
|accumulate| will use \nil, whereas |reducing| will generate different
code that avoids any dependence on the initial value.
The reason for having both clauses is that one usually
thinks of reductions (like |sum|) and accumulations (like |collect|)
as different beasts.

\end{clauses}


\subsubsection{Finders}

A ~finder~ is a clause whose value is an expression that meets some
condition.

\begin{clauses}

\clause{|finding| ~expr~ |such-that| ~test~ \opt\ |into| ~var~ |on-failure| ~failure-value~}
\clindexx{finding}{such-that}
If ~test~ (which is an expression) ever evaluates to \nonnil, the loop
is terminated, the 
epilogue code is run and the value of ~expr~ is returned.  Otherwise,
\nil\ (or ~failure-value,~ if provided) is returned.  If ~var~ is
provided, it will have either the \nonnil\ value of ~expr~ or
~failure-value~ when the epilogue code is run.

\cpar As a special case, if the ~test~ expression is a sharp-quoted
function, then it is applied to ~expr~ instead of being simply
evaluated.  E.g. |(finding x such-that \#'evenp)| is equivalent to
|(finding x such-that (evenp x))|.  

%\cpar although ~test~ need have nothing to do with ~%expr~ as in
%|(finding j such-that (> i 3))|, it usually 
%will: |(finding (length el) such-that (oddp (length el)))|.  to
%avoid performing the |length| computation twice, you could write
%|(finding (length el) such-that \#'oddp)| or |(finding (length
%el) such-that 'oddp)|; for these cases, \iter\ generates code that
%executes ~expr~ only once.  the code for |\#'oddp|
%is slightly different from that for {\lisp 'oddp}; see the discussion
%under {\lisp for\dots in} and {\lisp for\dots on}.

\cpar |On-failure| is a misnomer. Because it is always evaluated, it behaves
more like the default third argument to the |gethash| function. As a result,
|on-failure (error "Not found")| makes no sense. Instead, the clauses |leave|
or |thereis| can be used in conjunction with |finally| as follows:
\begin{program}
(iter (for x in '(1 2 3))
      (if (evenp x) (leave x))
      (finally (error "not found")))
\end{program}

\cpar This clause may appear multiple times when all defaults are
identical. It can also be used together with either |always|/|never| or
|thereis| if their defaults match. More specifically, |on-failure nil| is
compatible with |thereis|, while |on-failure t| is compatible with |always|
and |never| clauses.
\begin{program}
(iter (for i in '(7 -4 2 -3))
      (if (plusp i)
	  (finding    i  such-that (evenp i))
          (finding (- i) such-that (oddp i))))
\end{program}

\startitem
\Clause{|finding| ~expr~ |maximizing| ~m-expr~ \opt\ |into| ~var~}
\clindexx{finding}{maximizing}
\Clause{|finding| ~expr~ |minimizing| ~m-expr~ \opt\ |into| ~var~}
\clindexx{finding}{minimizing}
\finishitem
Computes the extremum (maximum or minimum) value of ~m-expr~ over all
iterations, and returns the value of ~expr~ corresponding to the
extremum.  ~expr~ is evaluated inside the loop at the time the new
extremum is established.  If ~m-expr~ is never evaluated (due to, for
example, being embedded in a conditional clause), then the returned
value depends on the type, if any, of ~expr~ (or ~var,~ if
one is supplied).  If there is no type, the returned
value will be \nil; if the type is numeric, the returned value will be
zero.

\cpar For these two clauses, ~var~ may be a list of two
symbols; in that case, the first is used to record ~expr~ and
the second, ~m-expr.~  

\cpar As with |finding\dots such-that|, if ~m-expr~ is a sharp-quoted
function, then it is called on ~expr~ instead of being evaluated.

\end{clauses}

\subsubsection{Boolean Tests}

\begin{clauses}

\clausex{first-iteration-p}{}
Returns |t| in the first cycle of the loop, otherwise \nil.

\clausex{first-time-p}{}
Returns |t| the first time the expression is evaluated, and then \nil\ forever.
This clause comes handy when printing (optional) elements separated
by a comma:

\begin{program}
(iter (for el in '(nil 1 2 nil 3))
      (when el
        (unless (first-time-p)
          (princ ", "))
        (princ el)))
\end{program}
produces |"1, 2, 3"|.

\end{clauses}

\subsubsection{Aggregated Boolean Tests}

\begin{clauses}

\clausex{always}{~expr~}
If ~expr~ ever evaluates to
\nil, then \nil\ is immediately returned; the epilogue code is not
executed.  If ~expr~ never evaluates to \nil, the epilogue code
is executed and the last value of ~expr~ (or |t| if ~expr~ was never
evaluated) is returned (whereas |loop| would constantly return |t|).

% mention last evaluated clause when multiple always clauses?

\clausex{never}{~expr~}
Like |(always (not ~expr~))|, except it does not influence the last
value returned by a possible other |always| clause. That is,
\begin{program}
(iter (repeat 2)
      (always 2)
      (never nil)) \yields 2 ; not t
\end{program}

\clausex{thereis}{~expr~}
If ~expr~ is ever \nonnil,
its value is immediately returned without running epilogue code.
Otherwise, the epilogue code is performed and \nil\ is returned.

This clause cannot be used together with |always| or |never|, because their
defaults are opposed (similarly, |(loop always 3 thereis nil)| refuses to
compile in some implementations of |loop|).

\end{clauses}


\subsection{Control Flow}
\label{control-flow}
Several clauses can be used to alter the usual flow of control in a loop.

Note: the clauses of this and subsequent sections don't adhere to \iter's
usual syntax, but instead use standard Common Lisp syntax.  Hence the
format for describing syntax subsequently is
like the standard format used in the Common Lisp manual, not like the
descriptions of clauses above.

\begin{clauses}

\clausex{finish}{}
Stops the loop and runs the epilogue code.

%for example:
%
%\begin{program}
%(iter (with answer = nil)
%         (initially (make-a-mess))
%         (for i from 1 to 10)
%         (when (correct? i)
%           (setq answer i)
%           (finish))
%         (finally (cleanup)))
%\end{program}
%
%this code will execute |cleanup| whether or not the test |(correct?
%i)| ever succeeds. 
%the (more elegant) formulation,
%\begin{program}
%(iter (initially (make-a-mess))
%         (for i from 1 to 10)
%         (finding i such-that (correct? i))
%         (finally (cleanup)))
%\end{program}
%would not execute |cleanup| if |(correct? i)| succeeded; it
%would do an immediate return.

\clausex{leave}{\opt\ ~value~}
Immediately returns ~value~ (default \nil) from the current \iter\
form, skipping the epilogue code.  Equivalent to using |return-from|.

\clausex{next-iteration}{}
Skips the remainder of the loop body and begins the next iteration of
the loop.

\clausex{while}{~expr~}
If ~expr~ ever evaluates to \nil, the loop is terminated and the
epilogue code executed.  Equivalent to |(if (not ~expr~) (finish))|.

\clausex{until}{~expr~}
Equivalent to |(if ~expr~ (finish))|.

\clausex{if-first-time}{~then~ \opt\ ~else~}
If this clause is being executed for the first time in this invocation
of the \iter\ form, then the ~then~ code is evaluated; otherwise the
~else~ code is evaluated.

\cpar |(for ~var~ first ~expr1~ then ~expr2~)| is almost equivalent to
\begin{program}
(if-first-time (dsetq ~var~ ~expr1~)
               (dsetq ~var~ ~expr2~))
\end{program}
The only difference is that the |for| version makes ~var~ available
for use with |for\dots previous|.

\end{clauses}

\subsection{Code Placement}
When fine control is desired over where code appears in a loop
generated by \iter, the following special clauses may be useful.
They are all subject to code-motion problems (see section
\ref{code-movement}).  

\begin{clauses}

\clausex{initially}{|\&rest| ~forms~}
The lisp ~forms~ are placed in the prologue section of the loop, where
they are executed once, before the loop body is entered.

\clausex{after-each}{|\&rest| ~forms~}
The ~forms~ are placed at the end of the loop body, where they
are executed after each iteration.  Unlike the other clauses in this
section, ~forms~ may contain \iter\ clauses.

\clausex{else}{|\&rest| ~forms~}
The lisp ~forms~ are placed in the epilogue section of the loop, where they
are executed if this |else| clause is never met during execution of the
loop and the loop terminates normally.

\clausex{finally}{|\&rest| ~forms~}
The lisp ~forms~ are placed in the epilogue section of the loop, where
they are executed after the loop has terminated normally.

\clausex{finally-protected}{|\&rest| ~forms~}
The lisp ~forms~ are placed in the second form of an unwind-protect
outside the loop.  They are always executed after the loop has
terminated, regardless of how the termination occurred.

\end{clauses}


%%%%%%%%%%%%%%%%%%%%%%%%%%%%%%%%%%%%%%%%%%%%%%%%%%%%%%%%%%%%%%%%
\section{Other Features}

\subsection{Multiple Accumulations}
\label{multiple}

\begin{sloppypar}
It is permitted to have more than one clause accumulate into the same
variable, as in the following:
\begin{program}
(iter (for i from 1 to 10)
      (collect i into nums)
      (collect (sqrt i) into nums)
      (finally (return nums)))
\end{program}
Clauses can only accumulate into the same variable if they are
compatible.  |collect|, |adjoining|, |appending|, |nconcing|,
|unioning| and |nunioning| are compatible with each other; |sum|,
|multiply| and |counting| are compatible; |always| and |never| are compatible;
|finding| \dots |such-that| is compatible with either |thereis| or |always|
and |never| when their defaults match; and |maximize| and |minimize| clauses
are compatible only with other |maximize| and |minimize| clauses,
respectively.

%note that the same variable ~cannot~ be both an accumulation
%variable and an ordinary variable; there can be only one variable with
%a given name within an \iter\ form. 
\end{sloppypar}

\subsection{Named Blocks}

Like Common Lisp |block|s, \iter\ forms can be given names.  The
name should be a single symbol, and it must be the first form in the
\iter.  The generated code behaves exactly like a named block; in
particular, |(return-from ~name~)| can be used to exit it:
\begin{program}
(iter fred
      (for i from 1 to 10)
      (iter barney
            (for j from i to 10)
            (if (> (* i j) 17)
                (return-from fred j))))
\end{program}
An \iter\ form that is not given a name is implicitly named \nil.

Sometimes one would like to write an expression in an inner \iter\ form,
but have it processed by an outer \iter\ form.  This is possible with
the |in| clause.  

\begin{clauses}

\clausex{in}{~name~ |\&rest| ~forms~}
Evaluates ~forms~ as if they were part of the \iter\ form named
~name~.  In other words, \iter\ clauses are processed by the \iter\
form named ~name,~ and not by any \iter\ forms that occur inside ~name.~

\cpar As an example, consider the problem of collecting a list of the
elements in a two-dimensional array.  The naive solution,
\begin{program}
(iter (for i below (array-dimension ar 0))
      (iter (for j below (array-dimension ar 1))
            (collect (aref ar i j))))
\end{program}
\noindent is wrong because the list created by the inner \iter\ is simply
ignored by the outer one.  But using |in| we can write:
\begin{program}
(iter outer (for i below (array-dimension ar 0))
      (iter (for j below (array-dimension ar 1))
            (in outer (collect (aref ar i j)))))
\end{program}
\noindent which has the desired result.

\end{clauses}

\subsection{Destructuring}
\label{destructuring}

In many places within \iter\ clauses where a variable is expected, a
list can be written instead.  In these cases, the value to be assigned
is ~destructured~ according to the pattern
described by the list.  As a simple example, the clause
\begin{program}
(for (key . item) in alist)
\end{program}
\noindent will result in |key| being set to the |car| of
each element in |alist|, and |item| being set to the |cdr|.  The
pattern list may be nested to arbitrary depth, and (as the example
shows) need not be terminated with \nil; the only requirement is that
each leaf be a bindable symbol (or \nil, in which case no binding is
generated for that piece of the structure).

Sometimes, you might like to do the equivalent of a
|multiple-value-setq| in a clause.  This 
``multiple-value destructuring'' can be expressed by writing 
\linebreak |(values $pat_1$ $pat_2 \ldots$)| for a destructuring
pattern, as in 
\begin{program}
(for (values (a . b) c d) = (three-valued-function ...))
\end{program}
\begin{sloppypar}
Note that the $pat_i$ can themselves be destructuring patterns (though
not multiple-value destructuring patterns).  You can't do multiple-value
destructuring in a |with| clause; instead wrap the whole \iter\
form in a |multiple-value-bind|.
\end{sloppypar}

\begin{note}{Rationale}
There are subtle interactions between variable declarations and
evaluation order that make the correct implementation of
multiple-value destructuring in a |with| somewhat tricky.
\end{note}

The destructuring feature of \iter\ is available as a separate
mechanism, using the |dsetq| macro:

\begin{clauses}
\defmacro{dsetq}{template expr}
Performs destructuring of ~expr~ using ~template~.   May be used
outside of an \iter\ form. Yields the primary value of ~expr~.

\end{clauses}

\subsection{On-line Help}

\begin{sloppypar}
There is a limited facility for on-line help, in the form of the 
|display-iterate-clauses| function.
\end{sloppypar}

\begin{clauses}

\defun{display-iterate-clauses}{\opt\ clause-spec}
Displays a list of \iter\ clauses.  If ~clause-spec~ is not
provided, all clauses are shown; if it is a symbol, all clauses
beginning with that symbol are shown; and if it is a list of symbols,
all clauses for which ~clause-spec~ is a prefix are shown.

\end{clauses}

\subsection{Parallel Binding and Stepping}
\label{bindings}

The parallel binding and stepping of variables is a feature that
\iter\ does ~not~ have.  This section attempts to provide a rationale.

We say that two variables are bound ~in parallel~ if neither
binding shadows the other.  This is the usual semantics of |let|
(as opposed to |let*|).  Similarly, we can say that iteration
variables are stepped in parallel if neither variable is updated
before the other, conceptually speaking; in other words, if the code
to update each variable can reference the old values of both variables.

|loop| allows parallel binding of variables and parallel stepping of
driver variables.  My view is that if you are depending on the 
serial/parallel distinction, you are doing something obscure.  If you
need to bind 
variables in parallel using |with|, then you must be using a
variable name that shadows a name in the existing lexical environment.
Don't do that.  The most common use for parallel stepping is to track
the values of variables on the previous iteration, but in fact this
does not require parallel stepping at all; the following will work:
\begin{program}
(iter (for current in list)
      (for prev previous current)
      ...)
\end{program}


%%%%%%%%%%%%%%%%%%%%%%%%%%%%%%%%%%%%%%%%%%%%%%%%%%%%%%%%%%%%%%%%
\section{Types and Declarations}
\label{types}

\subsection{Discussion}

Sometimes efficiency dictates that the types of variables be declared.
This type information needs to be communicated to \iter\ so it can
bind variables to appropriate values.  Furthermore, \iter\ must often
generate internal variables invisible to the user; there needs to be a
way for these to be declared.

As an example, consider this code, which will return the number of
odd elements in |number-list|:
\begin{program}
(iter (for el in number-list)
      (count (oddp el)))
\end{program}
In processing this form,
\iter\ will create an internal variable, let us call it |list17|, to
hold the successive |cdr|s of |number-list|, and  
will bind the variable to |number-list|.  It will also generate a
default binding for |el|; only inside the body of the loop will |el|
be set to the |car| of |list17|.  Finally, \iter\ will generate a
variable, call it |result|, to hold the result of the count, and will
bind it to zero.

When dealing with type declarations, \iter\ observes one simple rule:
~it will never generate a declaration unless requested to do so.~  The
reason is that such declarations might mask errors in compiled code by
avoiding error-checks; the resulting problems would be doubly hard to
track down because the declarations would be hidden from the
programmer.  Of course, a compiler might omit error-checks even in the
absence of declarations, though this behavior can usually be avoided,
e.g. by saying |(declaim (optimize (safety 3)))|.

So, the above \iter\ form will generate code with no declarations.
But say we wish to declare the types of |el| and the internal
variables |list17| and |result|.  How is this done?

Declaring the type of |el| is easy, since the programmer knows
the variable's name:
\begin{program}
(iter (for el in number-list)
      (declare (fixnum el))
      (counting (oddp el)))
\end{program}
\iter\ can read variable type declarations like this one.  Before
processing any clauses, it scans the entire top-level form for type
declarations 
and records the types, so that variable bindings can be performed
correctly.  In this case, |el| will be bound to zero
instead of \nil.  Also, \iter\ collects all the top-level declarations
and puts them at the begining of the generated code, so it is not
necessary to place all declarations at the beginning of an \iter\
form; instead, they can be written near the variables whose types they
declare.  

Since \iter\ is not part of the compiler, it will not know
about declarations that occur outside an \iter\ form; these
declarations must be repeated inside the form.

Here is another way we could have declared the type of |el|:
\begin{program}
(iter (for (the fixnum el) in number-list)
      (counting (oddp el)))
\end{program}
\iter\ extends the Common Lisp |the|\lispindex{the}
form to apply to variables as well as value-producing forms; anywhere
a variable is allowed---in a |with| clause, as the iteration
variable in a driver clause, as the |into| argument of an
accumulation clause, even inside a destructuring template---you can
write |(the ~type~ ~symbol~)| instead.

There is one crucial difference between using a |the| form and
actually declaring the variable: explicit declarations are always
placed in the generated code, but type information from a |the|
form is not turned into an actual declaration unless you tell \iter\
to do so using |iterate:declare-variables|.  See below.

\begin{sloppypar}
Declaring the types of internal variables is harder than declaring the
types of explicitly mentioned variables, since their names
are unknown.  You do it by declaring |iterate:declare-variables|
somewhere inside the top level of the \iter\ form.  (This will also
generate declarations for variables declared using |the|.)
\iter\ does not provide much selectivity here: it's all or none. 
And unfortunately, since \iter\ is not privy to compiler information
but instead reads declarations itself, it will not hear if you
|(declaim (iterate:declare-variables))|.  Instead, set the variable 
|iterate::*always-declare-variables*| to |t| at
compile-time, using |eval-when|.
\end{sloppypar}

To determine the appropriate types for internal variables, \iter\ uses
three sources of information:

\begin{itemize}

\item Often, the particular clause dictates a certain type for a
variable; \iter\ will use this information when available.  In the
current example, the variable |list17| will be given the type
|list|, since that is the only type that makes sense; and the
variable |result| will be given the type |fixnum|, on the
assumption that you will not be counting high enough to need bignums.
You can override this assumption only by using and explicitly declaring a
variable: 
\begin{verbatim}
(iter (declare (iterate:declare-variables))
      (for el in number-list)
      (count (oddp el) into my-result)
      (declare (integer my-result))
      (finally (return my-result)))
\end{verbatim}

\begin{sloppypar}
Other examples of the type assumptions that \iter\ makes are: type
|list| for |into| variables of collection clauses; type |list| for
expressions that are to be destructured; type |vector| for the
variable holding the vector in a |for\dots in-vector| clause, and
similarly for |string| and the |for\dots in-string| clause;
and the implementation-dependent type
|(type-of array-dimension-limit)| for the index and limit
variables generated by sequence iteration drivers like |for\dots
in-vector| and |for\dots in-string| (but not |for\dots in-sequence|,
because it may be used to iterate over a list). 
\end{sloppypar}

\item Sometimes, \iter\ will examine expressions and try to determine
their types in a simple-minded way.  If the expression is
self-evaluating (like a number, for instance), \iter\ knows that the
expression's type is the same as the type of the value it denotes, so
it can use that type.  If the expression is of the form |(the ~type~
~expr~)|, \iter\ is smart enough to extract ~type~ and use it.
However, the current version of \iter\ does not  
examine declarations of function result types or do any type
inference.  It will not determine, for 
example, that the type of |(+ 3 4)| is |fixnum|, or even
|number|.

\item In some cases, the type of an internal variable should match the
type of some other variable.  For instance, \iter\ generates an
internal variable for |(f x)| in the
clause |(for i from 1 to (f x))|, and in the absence of other
information will give it the same type as |i|.  If, however, the
expression had been written |(the fixnum (f x))|, then \iter\
would have given the internal variable the type |fixnum|
regardless of |i|'s type.  The type incompatibility errors that
could arise in this situation are not checked for.

\end{itemize}

Note that if you do declare |iterate:declare-variables|, then
\iter\ may declare user variables as well as internal ones if they do
not already have declarations, though only for variables that it
binds.  For instance, in this code:

\begin{program}
(iter (declare (iterate:declare-variables))
      (for i from 1 to 10)
      (collect i into var))
\end{program}
the variable |var| will be declared to be of type |list|.


\subsection{Summary}

\iter\ understands standard Common Lisp variable type declarations
that occur within an \iter\ form and
will pass them through to the generated code.  If the declaration
|(iterate:declare-variables)|\lispindex{declare-variables}
appears at the top level of an 
\iter\ form, or if 
|iterate::*always-declare-variables*|\lispindex{*always-declare-variables*} 
is \nonnil, then \iter\ will use the type information gleaned from user
declarations, self-evaluating expressions and |the| expressions,
combined with reasonable assumptions, to determine variable
types and declare them.


%%%%%%%%%%%%%%%%%%%%%%%%%%%%%%%%%%%%%%%%%%%%%%%%%%%%%%%%%%%%%%%%
\section{Problems with Code Movement}
\label{code-movement}

Some \iter\ clauses, or parts of clauses, result in code being
moved from the location of the clause to other parts of the loop.
Drivers behave this way, as do code-placement clauses like |initially|
and |finally|.  When using these clauses, there is a danger of writing
an expression that makes sense in its apparent location but will be
invalid or have a different meaning in another location.  For example:
\begin{program}
(iter (for i from 1 to 10)
      (let ((x 3))
        (initially (setq x 4))))
\end{program}
While it may appear that the |x| of |(initially (setq x 4))| is the
same as the |x| of |(let ((x 3)) \dots|, in fact they are not:
|initially| moves its code outside the loop body, so |x| would refer
to a global variable.  Here is another example of the same problem:
\begin{program}
(iter (for i from 1 to 10)
      (let ((x 3))
        (collect i into x)))
\end{program}
If this code were executed, |collect| would create a binding for its
|x| at the top level of the \iter\ form that the |let| will shadow.

Happily, \iter\ is smart enough to catch these errors; it
walks all problematical code to ensure that free variables are not
bound inside the loop body, and checks all variables it binds for the
same problem.

However, some errors cannot be caught:

\begin{program}
(iter (with x = 3)
      (for el in list)
      (setq x 1)
      (reducing el by \#'+ initial-value x))
\end{program}
|reducing| moves its |initial-value| argument to the initialization
part of the loop in order to produce more efficient code.  Since
\iter\ does not perform data-flow analysis, it cannot determine that
|x| is changed inside the loop; all it can establish is that |x| is
not bound internally.  Hence this code will not signal an
error and will use $3$ as the initial value of the reduction.

The following list summarizes all cases that are subject to these code
motion and variable-shadowing problems.
\begin{itemize}
\item Any variable for which \iter\ creates a binding, including those
used in |with| and the |into| keyword of many clauses.

\begin{sloppypar}
\item The special clauses which place code: |initially|, |after-each|, |else|,
|finally| and |finally-protected|. 
\end{sloppypar}

\item The variables of a |next| or |do-next| form.

\item The |initially| arguments of |for\dots initially\dots then| and
|for\dots previous|. 

\item The |then| argument of |for\dots initially\dots then|.

\item The |initial-value| arguments of |reducing| and |accumulate|.

\item The |on-failure| argument of |finding\dots such-that|.
\end{itemize}

%%%%%%%%%%%%%%%%%%%%%%%%%%%%%%%%%%%%%%%%%%%%%%%%%%%%%%%%%%%%%%%%
\section{Differences Between {\tt Iterate} and {\tt Loop}}
|loop| contains a great deal of complexity which \iter\ tries to
avoid.  Hence many esoteric features of |loop| don't exist in \iter.
Other features have been carried over, but in a cleaned-up form.  
And of course, many new features have been added; they are not
mentioned in this list.

\begin{itemize}

\item \iter's syntax is more Lisp-like than |loop|'s, having a higher
density of parens.

\item The current implementation of \iter, unlike the current version
of  |loop| (as documented in {\em Common Lisp, 2nd Ed.\/}), is
extensible (see section \ref{extend}).  

\item |loop| puts the updates of all driver variables at the top of
the loop; \iter\ leaves them where the driver clauses appear.

\item While for the most part \iter\ clauses that resemble |loop| clauses
behave similarly, there are some differences.  For instance, there is
no |for\dots =\dots then| in \iter; instead use 
|for\dots initially\dots then|.  

\item |loop| binds the variable |it| at certain times
to allow pseudo-English expressions like |when ~expr~ return it|. 
In \iter, you must bind ~expr~ to a variable yourself.  Note that
|when ~expr~ return it| is like |thereis ~expr~| except that the latter is an
accumulation clause and therefore competes with other accumulations
(remember section \ref{multiple} above).

% repeat different behaviour of |always| clause here?

\item |loop| has a special |return| clause, illustrated in the
previous item.  \iter\ doesn't need one,  since an ordinary Lisp
|return| has the same effect.

\item |loop| allows for parallel binding and stepping of iteration
variables.  \iter\ does not.  (See section \ref{bindings}.)

\item |loop| and \iter\ handle variable type declarations very
differently.  |loop| provides a special syntax for declaring variable
types, and does not examine declarations.  Moreover, the standard
implementation of |loop| will
generate declarations when none are requested.
\iter\ parses standard Common Lisp type declarations, and will never
declare a variable itself unless declarations are 
specifically requested.

\end{itemize}

%%%%%%%%%%%%%%%%%%%%%%%%%%%%%%%%%%%%%%%%%%%%%%%%%%%%%%%%%%%%%%%%
\section{Rolling Your Own}
\label{extend}

\subsection{Introduction}

\iter\ is extensible---you can write new clauses that embody new
iteration patterns.  You might want to write a new driver clause for a
data structure of your own, or you might want to write a clause that
collects or manipulates elements in a way not provided by \iter. 

This section describes how to write clauses for \iter.  Writing a
clause is like writing a macro.  In fact, writing a clause ~is~
writing a macro: since \iter\ code-walks its body and macroexpands,
you can add new abstractions to \iter\ with good old |defmacro|.

Actually, there are two extensions you can make to \iter\ that are
even easier than writing a macro.  They are adding a synonym for an
existing clause and defining a driver clause for an indexable
sequence.  These can be done with |defsynonym| and
|defclause-sequence|, respectively.  See section \ref{aids}, below.

The rest of this section explains how to write macros that expand into
\iter\ clauses.
Here's how you could add a simplified version of \iter's |multiply|
clause, if \iter\ didn't already have one:
\begin{program}
(defmacro multiply (expr)
  `(reducing ,expr by \#'* initial-value 1))
\end{program}

If you found yourself summing the square of an expression often, you
might want to write a macro for that.  A first cut might be
\begin{program}
(defmacro sum-of-squares (expr)
  `(sum (* ,expr ,expr)))
\end{program}
but if you are an experienced macro writer, you will realize that this
code will evaluate ~expr~ twice, which is probably a bad idea.  A
better version would use a temporary:
\begin{program}
(defmacro sum-of-squares (expr)
  (let ((temp (gensym)))
    `(let ((,temp ,expr))
       (sum (* ,temp ,temp)))))
\end{program}
Although this may seem complex, it is just the sort of thing you'd
have to go through to write any macro, which illustrates the point of
this section: if you can write macros, you can extend \iter.

Our macros don't use \iter's keyword-argument
syntax.  We could just use keywords with |defmacro|, but we would
still not be using \iter's clause indexing mechanism.  Unlike Lisp,
which uses just the first symbol of a form to determine what function
to call, \iter\ individuates clauses by the list of required keywords.
For instance, |for\dots in| and |for\dots in-vector| are different clauses
implemented by distinct Lisp functions.  

To buy into this indexing scheme, as well as the keyword-argument
syntax, use |defmacro-clause|:

\begin{clauses}

\defmacro{defmacro-clause}{arglist |\&body| body}
Defines a new \iter\ clause.  ~arglist~ is a list of symbols which are
alternating keywords and arguments.  \opt\ may be used, and the list
may be terminated by |\&sequence|.  ~body~ is an ordinary macro body,
as with |defmacro|.  If the first form of ~body~ is a string, it is
considered a documentation string and will be shown by
|display-iterate-clauses|. |defmacro-clause| will signal an error if 
defining the clause would result in an ambiguity.  E.g. you cannot
define the clause |for\dots from| because there would be no way to
distinguish it from a use of the |for| clause with optional keyword |from|.

\end{clauses}

\medskip

Here is |multiply| using |defmacro-clause|.  The keywords are capitalized
for readability.
\begin{program}
(defmacro-clause (MULTIPLY expr \opt\ INTO var)
  `(reducing ,expr by \#'* into ,var initial-value 1))
\end{program}
You don't have to worry about the case when |var| is not supplied; for
any clause with an |into| keyword, saying |into nil| is equivalent to
omitting the |into| entirely.

As another, more extended example, consider the fairly common
iteration pattern that 
involves finding the sequence element that maximizes (or minimizes) some
function.  \iter\ provides this as |finding\dots maximizing|, but it's
instructive to see how to write it.
Here, in pseudocode, is how you might write such a loop for
maximizing a function F:

\begin{quote}
\begin{tabbing}
set variable MAX-VAL to NIL; \\
set variable WINNER to NIL; \\
for\=\ each element EL in the sequence \\
\>  if\=\ MAX-VAL\=\ is NIL or F(EL) $>$ MAX-VAL then \\
\>\>    set MAX-VAL to F(EL); \\
\>\>    set WINNER to EL; \\
\>  end if; \\
end for; \\
return WINNER.
\end{tabbing}
\end{quote}

Here is the macro:
\begin{program}
(defmacro-clause (FINDING expr MAXIMIZING func \opt\ INTO var)
  (let ((max-val (gensym))
        (temp1 (gensym))
        (temp2 (gensym))
        (winner (or var iterate::*result-var*)))
    `(progn 
       (with ,max-val = nil)
       (with ,winner = nil)
       (cond 
        ((null ,max-val)
         (setq ,winner ,expr)
         (setq ,max-val (funcall ,func ,winner))
        (t
         (let* ((,temp1 ,expr)
                (,temp2 (funcall ,func ,temp1)))
           (when (> ,temp2 ,max-val)
             (setq ,max-val ,temp2)
             (setq ,winner ,temp1))))))
       (finally (leave ,winner)))))
\end{program}
Note that if no |into| variable is supplied, we use
|iterate::*result-var*|, which contains the internal variable into
which all clauses place their results.  If this variable is bound by
some clause, then \iter\ will return its value automatically;
otherwise, \nil\ will be returned.  

\subsection{Writing Drivers}

In principle, drivers can be implemented just as easily as other
\iter\ clauses.  In practice, they are a little harder to get right. 
As an example, consider writing a driver that
iterates over all the 
elements of a vector, ignoring its fill-pointer.  |for\dots in-vector|
won't work for this, because it observes the fill-pointer.  It's necessary to
use |array-dimension| instead of |length| to obtain the size of the
vector.  Here is one approach:

\begin{program}
(defmacro-clause (FOR var IN-WHOLE-VECTOR v)
  "All the elements of a vector (disregards fill-pointer)"
  (let ((vect (gensym))
        (index (gensym)))
    `(progn
       (with ,vect = ,v)
       (for ,index from 0 below (array-dimension ,vect 0))
       (for ,var = (aref ,vect ,index)))))
\end{program}
Note that we immediately put |v| in a variable, in case it is an
expression.  Again, this is just good Lisp macrology.  It also has a
subtle effect on the semantics of the driver: |v| is evaluated only
once, at the beginning of the loop, so changes to |v| in the loop have
no effect on the driver.  Similarly, the bounds for numerical iteration
e.g. the above |array-dimension| are also evaluated once only.  This is how
all of \iter's drivers work.


There is an important point concerning the |progn| in this code.  We
need the |progn|, of course, because we are returning several forms,
one of which is a driver.  But \iter\ drivers must occur at top-level.
Is this code in error?  No, because ~top-level~ is defined in \iter\
to include forms inside a |progn|.  This is just the definition of
top-level that Common Lisp uses, and for the same reason: to allow
macros to return multiple forms at top-level.

While our |for\dots in-whole-vector| clause will work, it is not
ideal.  In particular, it does not support generating.  Do do so, we
need to use |for\dots next| or |for\dots do-next|.  The job is
simplified by the |defmacro-driver| macro.

\begin{clauses}

\defmacro{defmacro-driver}{arglist |\&body| body}
Defines a driver clause in
both the |for| and |generate| forms, and provides a parameter
|generate| which ~body~ can examine to determine how it was invoked.
~arglist~ is as in |defmacro-clause|, and should begin with the symbol
|for|. 

\end{clauses}

With |defmacro-driver|, our driver looks like this:
\begin{program}
(defmacro-driver (FOR var IN-WHOLE-VECTOR v)
  "All the elements of a vector (disregards fill-pointer)"
   (let ((vect (gensym))
         (end (gensym))
         (index (gensym))
         (kwd (if generate 'generate 'for)))
     `(progn
        (with ,vect = ,v)
        (with ,end = (array-dimension ,vect 0))
        (with ,index = -1)
        (,kwd ,var next (progn (incf ,index)
                               (if (>= ,index ,end) (terminate))
                               (aref ,vect ,index))))))
\end{program}


We are still missing one thing: the |\&sequence| keywords.  
We can get them easily enough, by writing 
\begin{program}
(defmacro-driver (FOR var IN-WHOLE-VECTOR v \&sequence) 
  ...)
\end{program}
We can now refer to parameters |from|, |to|, |by|, etc. which contain
either the values for the corresponding keyword, or \nil\ if the
keyword was not supplied.  Implementing the right code for these
keywords is cumbersome but not difficult; it is left as an exercise.
But before you begin, see |defclause-sequence| below for an easier way.
       
\subsection{Extensibility Aids}
\label{aids}

This section documents assorted features that may be of use in
extending \iter.

\begin{clauses}

\defunexpvar{*result-var*}
Holds the variable that is used to return a value as a result of the
\iter\ form.  You may examine this and use it in a |with| clause, but
you should not change it.

\defmacro{defsynonym}{syn word}
 Makes ~syn~ a synonym for the existing \iter\ keyword ~word.~  Only
the first word in each clause can have synonyms. 

\defmacroxx{defclause-sequence}{element-name index-name}{|\&key|
access-fn size-fn sequence-type}{element-type
element-doc-string index-doc-string}  
Provides
a simple way to define sequence clauses.  Generates two
clauses, one for iterating over the sequence's elements, the other
for iterating over its indices.  The first symbol of both
clauses will have print-name |for|.
~element-name~  and ~index-name~ should be symbols.
~element-name~ is the second keyword of the element iterator (typically of
the form 
|in-~sequence-type~|), and ~index-name~ is the second keyword
of the index-iterator (typically of the form
|index-of-~sequence-type~|).  Either name may be 
\nil, in which case the corresponding clause is not defined.  If both
symbols are supplied, they should be in the same package.  The |for|
that begins the clauses will be in this package.

\cpar ~access-fn~ is the function to be used to
access elements of the sequence in the element iterator.  The function
should take two 
arguments, a sequence and an index, and return the appropriate element.
~size-fn~ should denote a function of one argument, a sequence, that
returns its size.  Both ~access-fn~ and ~size-fn~ are required for the
element iterator, but only ~size-fn~ is needed for the index iterator.

\cpar The ~sequence-type~ and ~element-type~ keywords are used to
suggest types for the variables 
used to hold the sequence and the
sequence elements, respectively.  The usual rules about \iter's
treatment of variable type declarations apply (see section \ref{types}).

\cpar ~element-doc-string~ and ~index-doc-string~ are
the documentation strings, for use with |display-iterate-clauses|. 

\cpar The generated element-iterator performs destructuring on the
element variable.

\cpar As an example, the above |for\dots in-whole-vector| example
could have been written:
\begin{program}
(defclause-sequence IN-WHOLE-VECTOR INDEX-OF-WHOLE-VECTOR
  :access-fn 'aref
  :size-fn \#'(lambda (v) (array-dimension v 0))
  :sequence-type 'vector
  :element-type t
  :element-doc-string 
     "Elements of a vector, disregarding fill-pointer"
  :index-doc-string 
     "Indices of vector, disregarding fill-pointer")
\end{program}

\end{clauses}

\subsection{Subtleties}

There are some subtleties to be aware of when writing \iter\ clauses.
First, the code returned by your macros may be |nconc|'ed into a list,
so you should always returned freshly consed lists, rather than
constants.  Second, \iter\ matches clauses by using |eq| on the first
symbol and |string=| on the subsequent ones, so the package of the
first symbol of a clause is relevant.  All of the clauses in this manual
have their first word in the \iter\ package.  
You can use the package system in the usual way to shadow
\iter\ clauses without replacing them.

%%% say more here, about the badness that only the first word of a
%%% clause is packagey.

\section{Non-portable Extensions to Iterate (Contribs)}
\label{contribs}

Currently, there is only one non-portable extension to iterate in the
distribution: iterate-pg. If you have made an extension that depends
on non-portable features, feel free to send them to |asf@boinkor.net|
for inclusion in the iterate distribution.

\subsection{An SQL query driver for iterate}

The pg package by Eric Marsden (see |http://cliki.net/pg|) provides an
interface to the PostgreSQL database. Using the \iterpg\ extension, it
is possible to handle the results of SQL queries using \iter.

This usage example should give you an idea of how to use it:

\begin{program}
(pg:with-pg-connection (c "somedb" "someuser")
  (iter (for (impl version date) in-relation "select * from version"
                                 on-connection *dbconn*)
        (collect version)))
\end{program}

To use the extension via |ASDF|, simply make your system depend on the
|iterate-pg| system instead of the |iterate| system. To load it
manually, use:

\begin{program}
  (asdf:oos 'asdf:load-op :iterate-pg)
\end{program}

%%%%%%%%%%%%%%%%%%%%%%%%%%%%%%%%%%%%%%%%%%%%%%%%%%%%%%%%%%%%%%%%
\section{Obtaining {\tt Iterate}}

\begin{sloppypar}
The information in this chapter is obsolete but included for
completeness's sake; Currently, the most up-to-date information on
\iter\ can be found at |http://boinkor.net/iterate.html|.
\end{sloppypar}

\begin{sloppypar}
\iter\ currently runs on Lisp Machines, and on
HP's, Sun3's and Sparcstations under Lucid.  
\iter\ source and binaries are available at the MIT AI Lab in the
subdirectories of |/src/local/lisplib/|.  The source file,
|iterate.lisp|, is also available for anonymous FTP in the directory
|/com/fpt/pub/| on the machine |TRIX.AI.MIT.EDU| (Internet number
128.52.32.6).  If you are unable to obtain |iterate| in one of these
ways, send mail to |jba@ai.mit.edu| and I will send you the source
file. 
\end{sloppypar}

\begin{sloppypar}
\iter\ resides in the |iterate| package (nickname |iter|).  Just say
\linebreak |(use-package :iterate)| to make all the necessary symbols
available. 
If a symbol is not exported, it appears in this manual with an
``|iterate::|'' prefix.
\end{sloppypar}

Send bug reports to |bug-iterate@ai.mit.edu|.  The |info-iterate|
mailing list will have notices of changes and problems; to have
yourself added, send mail to |info-iterate-request@ai.mit.edu|.


\medskip

\begin{flushleft}
 \bf Acknowledgements
\end{flushleft}
\smallskip

Richard Waters provided invaluable criticism which spurred me to improve
\iter\ greatly.  As early users, David Clemens, Oren Etzioni and Jeff
Siskind helped ferret out many bugs.

%\begin{theindex}
% The index files must be generated with the genindex program
% or from makeindex iterate-manual.idx which creates iterate-manual.ind.

%\baselineskip\normalbaselineskip
%\advance\baselineskip by -2pt

% Fix ``.'' spacing in lists
\documentstyle[12pt,aimemo]{article}

\setlength{\textwidth}{6.0in}
\setlength{\oddsidemargin}{0.25in}
\setlength{\evensidemargin}{0.25in}
\setlength{\topmargin}{0pt}
\setlength{\headsep}{24pt}
\setlength{\textheight}{8.5in}

\let\origpar=\par

% somehow, LaTex fucks with things so that the plain Tex \obeyspaces
% doesn't work.  This is the same, except we use an mbox containing
% a space instead of just a space.

\newcommand{\spc}{\mbox{ }}
\newcommand{\tAb}{\mbox{        }}
\def\obspaces{\catcode`\ =\active\catcode`\^^I=\active}
{\obspaces \global\let =\spc\global\let^^I=\tAb}

\newenvironment{program}{\medskip\samepage\tt\obeylines\obspaces}{\medskip}

% use | for \tt
\catcode`\|=13
\newif\ifvbar
\def|{\ifvbar
         \endgroup\vbarfalse
      \else 
         \begingroup\tt\vbartrue 
      \fi}



\newcommand{\lisp}{\tt}
\newcommand{\iter}{|iterate|}
\newcommand{\iterpg}{|iterate-pg|}
\newcommand{\setf}{|setf|}
\newcommand{\nil}{|nil|}
\newcommand{\nonnil}{non-\nil}
\newcommand{\opt}{{\lisp \&optional}}
\newcommand{\yields}{$\Rightarrow$}



\newbox\Dots
\setbox\Dots=\hbox{\small ...}
\renewcommand{\dots}{\copy\Dots}

% Here's a way to define an environment which will treat all
% paragraphs inside it in the same manner.  It's, as near as I can
% figure, the way that Latex does things.  Using \everypar won't work;
% apparently Latex resets that a lot.  What you have to do is: at the
% top of the file do \let\origpar=\par.  In the
% begin part of the environment, do the parshape (or whatever), then say
% \def\par{{\origpar}}.  The extra braces are necessary.   That's all
% you have to do; at the end of the environment things will return to
% normal automatically.  Why does this all work?  I have no idea.
% Another important thing: there should be a blank line (i.e. a \par)
% between the end of the last paragraph and the \end{environment}.

\newlength{\clindent}
\setlength{\clindent}{0.5in}
\newlength{\clparindent}
\setlength{\clparindent}{\parindent}

\newenvironment{clauses}{\advance\linewidth -\parindent
                        \hangindent\clindent\relax
                         \hangafter=0
                          \parindent=0pt
                          \def\par{{\origpar}}}{}

\newcommand{\cpar}{\hspace \clparindent}

\newcommand{\startitem}{\bigskip\pagebreak[3]}
\newcommand{\finishitem}{\smallskip}

\newcommand{\clause}[1]{\startitem\Clause{#1}\finishitem}
\newcommand{\Clause}[1]{\hbox{#1}}
\newcommand{\clindex}[1]{\index{{\lisp #1}}}
\newcommand{\clindexx}[2]{\index{{\lisp #1\dots #2}}}
\newcommand{\clindexxx}[3]{\index{{\lisp #1\dots #2\dots #3}}}
\newcommand{\clausex}[2]{\startitem\Clausex{#1}{#2}\finishitem}
\newcommand{\Clausex}[2]{\Clause{|#1| #2}\clindex{#1}}

\newcommand{\defvar}[1]{\startitem\deff{#1}{}{Variable}\finishitem}
\newcommand{\defunexpvar}[1]{\startitem\deffnoindex{iterate::#1}{}{Variable}
                                        \lispindex{#1}\finishitem}
\newcommand{\defun}[2]{\startitem\deff{#1}{#2}{Function}\finishitem}
\newcommand{\defmacro}[2]{\startitem\deff{#1}{#2}{Macro}\finishitem}
\newcommand{\defunx}[3]{\startitem\deffx{#1}{#2}{#3}{Function}\finishitem}
\newcommand{\defmacrox}[3]{\startitem\deffx{#1}{#2}{#3}{Macro}\finishitem}
\newcommand{\defunxx}[4]{\startitem\deffxx{#1}{#2}{#3}{#4}{Function}\finishitem}
\newcommand{\defmacroxx}[4]{\startitem\deffxx{#1}{#2}{#3}{#4}{Macro}\finishitem}

\newcommand{\deff}[3]{\deffnoindex{#1}{#2}{#3}\lispindex{#1}}
\newcommand{\deffnoindex}[3]{\hbox to \hsize{{\tt #1} {\it #2}\hfill [{\it #3}]}}
\newcommand{\deffx}[4]{\deff{#1}{#2}{#4}\moreargs{#1}{#3}}
\newcommand{\deffxx}[5]{\deffx{#1}{#2}{#3}{#5}\moreargs{#1}{#4}}

\newcommand{\moreargs}[2]{\hbox to \hsize{\phantom{\lisp #1} {\it #2}\hfill}}

\newcommand{\lispindex}[1]{\index{{\lisp #1}}}

 \newenvironment{note}[1]{\pagebreak[2]\bigskip
                          \hrule\smallskip\small
                         {\setlength{\parindent}{0pt}
                           \par{\bf #1:}}}{\smallskip\hrule\bigskip}



\makeindex

%%%%%%%%%%%%%%%%%%%%%%%%%%%%%%%%%%%%%%%%%%%%%%%%%%%%%%%%%%%%%%%%

%\thispagestyle{empty}

%\vspace*{2in}

%\title{The |Iterate| Manual}
%\author{Jonathan Amsterdam}

%\maketitle

%\thispagestyle{empty}

%\begin{center}
%\Large *** DRAFT ***
%\end{center}
%This manual is in beta-test. It will become an AI Memo in a couple of
%months.  I would appreciate any comments or
%corrections.  You can reach me at email address |jba@ai.ai.mit.edu|.

\begin{document}

\def\writtenby{Jonathan Amsterdam}
\def\contractno{This report describes research done at the Artificial
Intelligence Laboratory of the Massachusetts Institute of Technology.
Support for the laboratory's artificial intelligence research is
provided in part by the Advanced Research Projects Agency of the
Department of Defense under Office
of Naval Research contract N00014-85-K-0124.}

\AIMEMO{1236}{\Large The Iterate Manual \\[1in]}
{This is the manual for version 1.4 of \iter, a powerful iteration
macro for Common  
Lisp. \iter\ is similar to {\tt loop} but provides numerous additional
features, is well integrated with Lisp, and is extensible. 
\vfill }


% use ~ for \em
\catcode`\~=13
\newif\ifem
\def~{\ifem
         \/\endgroup\emfalse
      \else 
         \begingroup\em\emtrue 
      \fi}



%\pagebreak
%\setcounter{page}{1}

%\thispagestyle{empty}


\tableofcontents

\pagebreak
\pagestyle{headings}

\section{Introduction}

\begin{sloppypar}
This manual describes
\iter, a powerful iteration facility for Common Lisp.
\iter\ provides abstractions for many common iteration
patterns and allows for the definition of additional patterns.  \iter\ is
a macro that expands into ordinary Lisp at
compile-time, so it is more efficient than higher-order
functions like |map| and |reduce.|  
While it is similar to |loop|, \iter\ offers a more Lisp-like syntax
and enhanced extensibility.
(For a more complete
comparison of \iter\ with other iteration constructs, see MIT AI Lab Working
Paper No. 324, ~Don't Loop, Iterate.~)
\end{sloppypar}

An \iter\ form consists of the symbol |iter|\footnote{You can also use
|iterate|, but |iter| is preferred because it avoids potential conflicts with
possible future additions to Common Lisp, and because it saves
horizontal space when writing code.}
followed by one or more
forms, some of which may be \iter\ 
~clauses.~  Here is a simple example of \iter\ which collects the
numbers from 1 to 10 into a list, and returns the list.  The return
value is shown following the arrow.
\begin{program}
(iter (for i from 1 to 10)
      (collect i))          \yields (1 2 3 4 5 6 7 8 9 10)
\end{program}
This form contains two clauses: a |for| clause that steps the
variable |i| over the integers from 1 to 10, and a |collect|
clause that accumulates its argument into a list.  With a few
exceptions, all \iter\ clauses have the same format: alternating
symbols (called ~keywords~) and expressions (called
~arguments~).  The syntax and terminology are those of Common
Lisp's keyword lambda lists.  One difference is that \iter's keywords
do not have to begin with a colon---though they may, except for the
first symbol of a clause.  So you can also write  |(for i :from
1 :to 10)| if you prefer.

Any Lisp form can appear in the body of an \iter, where it will have
its usual meaning.  \iter\ walks the entire body, expanding macros,
and recognizing clauses at any level.  This example collects all the
odd numbers in a list:
\begin{program}
(iter (for el in list)
      (if (and (numberp el) (oddp el))
          (collect el)))
\end{program}

There are clauses for iterating over numbers, lists, arrays and other
objects, and for
collecting, summing, counting, maximizing and other useful operations.
\iter\ also supports the creation of new variable bindings, stepping
over multiple sequences at once, destructuring, and compiler
declarations of variable types.  The following example illustrates
some of these features: 
\begin{program}
(iter (for (key . item) in alist)
         (for i from 0)
         (declare (fixnum i))
         (collect (cons i key)))
\end{program}

This loop takes the keys of an alist and returns a new alist
associating the keys with their positions in the original list. The
compiler declaration for |i| will appear in the generated code
in the appropriate place.

%%%%%%%%%%%%%%%%%%%%%%%%%%%%%%%%%%%%%%%%%%%%%%%%%%%%%%%%%%%%%%%%
\section{Clauses}

Most of \iter's clauses will be familiar to |loop| programmers.  
(|loop| is an iteration macro that has been incorporated into Common
Lisp.  Seen Guy Steele's {\em Common Lisp, 2nd Edition}.)
In
nearly all cases they behave the same as their |loop| counterparts, so
a |loop| user can switch to \iter\ with little pain (and much gain).

All clauses with the standard keyword-argument syntax consist of two
parts: a ~required~ part, 
containing keywords that must be present and in the right order; and
an ~optional~ part, containing keywords that may be omitted and,
if present, may occur in any order.  In the descriptions below, the
parts are separated by the Lisp lambda-list keyword \opt.


\subsection{Drivers}

An iteration-driving clause
conceptually causes the iteration to go forward.  Driver clauses in
\iter\ allow iteration over numbers, lists, vectors, hashtables, packages,
files and streams.  Iteration-driving clauses must
appear at the top level of an \iter\ form; they cannot be nested
inside another clause.  The driver variable is updated at the point
where the driver clause occurs.  Before the clause is executed for the
first time, the value of the variable is undefined.

%Also, regardless of where the driver clause
%appears in the body, the driver variable is stepped at the top of the
%loop; hence it is stylistically preferable, though
%not required, to place driver clauses at the beginning of the \iter.

Multiple drivers may appear in a single \iter\ form, in which case all
of the driver variables are updated each time through the loop, in the
order in which the clauses appear.  The first driver to terminate will
terminate the entire loop.

In all cases, the value of the driver variable on exit from the loop,
including within the epilogue code (see the |finally| clause), is
undefined.

All the parameters of a driver clause are evaluated once, before the
loop begins.  Hence it is not possible to change the bounds or other
properties of an iteration by side-effect from within the loop.

With one exception, driver clauses begin with the word |for| (or
the synonym |as|) and mention an iteration variable, which is
given a binding within the \iter\ form.
The exception is |repeat|, which just executes a loop a
specified number of times:

\begin{clauses}

\clausex{repeat}{~n~}
Repeats the loop ~n~ times.  For example:
\begin{program}
(iter (repeat 100)
      (print "I will not talk in class."))
\end{program}
\cpar If $n \leq 0$, the
loop will never be executed.  If ~n~ is not an integer, the
actual number of executions will be $\lceil n \rceil$.

\end{clauses}

\pagebreak[3]

\subsubsection{Numerical Iteration}

\begin{clauses}
\clausex{for}{~var~ |\&sequence|} 
The general form for iterating over a sequence of numbers requires a
variable and, optionally, one or more keywords that provide the bounds
and step size of the iteration. The |\&sequence|
lambda-list keyword is a shorthand for these sequence keywords.  They are:
|from|, |upfrom|, |downfrom|, |to|, |downto|, |above|, |below| and |by|.
|from| provides the starting value for ~var~
and defaults to zero. |to| provides a final value and implies that the
successive values of ~var~ will be increasing; |downto| implies that
they will be decreasing.  The loop terminates when ~var~ passes
the final value (i.e. becomes smaller or larger than it, depending on
the direction of iteration); in other  words, the loop body
will never be 
executed for values of ~var~ past the final value.
|below| and |above| are similar to |to| and
|downto|, except that the loop terminates when ~var~ equals or passes
the final value.

\cpar If no final value is specified, the variable will be stepped
forever.  Using |from| or |upfrom| will result in increasing
values, while |downfrom| will give decreasing values.

\cpar On each iteration,
~var~ is incremented or decremented by the value of the sequence
keyword |by|, which defaults to 1.  It should always be a positive
number, even for downward iterations.

\cpar In the following examples, the sequence of numbers generated is shown
next to the clause.

\begin{program}
(for i upfrom 0) \yields\ 0 1 2 \ldots
(for i from 5)   \yields\ 5 6 7 \ldots    ; either from or upfrom is okay
(for i downfrom 0) \yields\ 0 -1 -2 \ldots
(for i from 1 to 3) \yields\ 1 2 3
(for i from 1 below 3) \yields\ 1 2
(for i from 1 to 3 by 2) \yields\ 1 3
(for i from 1 below 3 by 2) \yields\ 1
(for i from 5 downto 3) \yields\ 5 4 3
\end{program}

\end{clauses}

\subsubsection{Sequence Iteration}

There are a number of clauses for iterating over sequences.  In all of
them, the argument following |for| may be a list instead of a
symbol, in which case destructuring is performed.  See section
\ref{destructuring}. 

\begin{clauses}

\clause{|for| ~var~ |in| ~list~ \opt\ |by| ~step-function~}
\clindexx{for}{in}
~var~ is set to successive elements of list.  
~step-function,~ which 
defaults to |cdr|, is used to obtain the next sublist. 

\clause{|for| ~var~ |on| ~list~ \opt\ |by| ~step-function~}
\clindexx{for}{on}
~var~ is set to successive sublists of list. 
~step-function~ (default |cdr|) is used as in |for\dots in|.

\end{clauses}

\medskip

These two clauses use |atom| to test for the end
of a list.  Hence, given a list whose final |cdr| is not \nil,
they will silently ignore the last |cdr|.  Other choices are
|endp|,  which would signal an error, and |null|, which
would probably result in an error somewhere else.  If you wish to use
an end-test other than |atom|, set the variable
|iterate::*list-end-test*|\lispindex{*list-end-test*} to the name of
the desired function. 

\begin{clauses}
\clause{|for| ~var~ |in-vector| ~vector~ |\&sequence|}
\clindexx{for}{in-vector}
~var~ takes on successive elements from ~vector.~   The vector's
fill-pointer is observed.  Here and in subsequent clauses, the |\&sequence|
keywords include |with-index|, which takes a symbol as argument
and uses it for the index variable instead of an internally generated
symbol.  The other |\&sequence| keywords behave as in numerical
iteration, except that the default iteration bounds are the bounds of
the vector.  E.g. in |(for i in-vector v downto 3)|,
|i| will start off being bound to the last element in |v|,
and will  
be set to preceding elements down to and including the element with
index 3.

\clause{|for| ~var~ |in-sequence| ~seq~ |\&sequence|}
\clindexx{for}{in-sequence}
This uses Common Lisp's generalized sequence functions, |elt|
and |length|, to obtain elements and determine the length of
~seq.~   Hence it will work for any sequence, including lists, and will
observe the fill-pointers of vectors.

\clause{|for| ~var~ |in-string| ~string~ |\&sequence|}
\clindexx{for}{in-string}
~var~ is set to successive characters of ~string.~

\startitem
\Clause{|for| ~var~ |index-of-vector| ~vector~ |\&sequence|}
\clindexx{for}{index-of-vector}
\Clause{|for| ~var~ |index-of-sequence| ~sequence~ |\&sequence|}
\clindexx{for}{index-of-sequence}
\Clause{|for| ~var~ |index-of-string| ~string~ |\&sequence|}
\clindexx{for}{index-of-string}
\finishitem
~var~ is set to successive indices of the sequence.
These clauses avoid the overhead of accessing the sequence elements
for those applications where they do not need to be examined, or are
examined rarely.  They admit all the optional keywords of the other
sequence drivers except the (redundant) |with-index| keyword.

\clause{|for| ~(key value)~ |in-hashtable| ~table~}
\clindexx{for}{in-hashtable}

~key~ and ~value,~ which must appear as shown in a list
and may be destructuring templates, are set to the keys and values of
~table~.  If ~key~ is |nil|, then the hashtable's keys will be ignored;
similarly for ~value~.
The order in which 
elements of ~table~ will be retrieved is unpredictable.

\clause{|for| ~var~ |in-package| ~package~ \opt\ |external-only| ~ext~}
\clindexx{for}{in-package}
Iterates over all the symbols in ~package,~ or over only the
external symbols if ~ext~ is specified and non-|nil|.  ~ext~ is not
evaluated.  The same symbol may appear more than once.

\clause{|for| ~var~ |in-packages| ~packages~ \opt\ |having-access| ~symbol-types~}
\clindexx{for}{in-packages}
Iterates over all the symbols from the list of packages denoted by the
descriptor ~packages~ and having accessibility (or visibility) given by
~symbol-types~.  This defaults to the list |(:external :internal :inherited)|
and is not evaluated.  ~var~ must be a list of up to three variables: in each
iteration, these will be set to a symbol, its access-type and package (as per
|with-package-iterator| in ANSI-CL).  The same symbol may appear more than
once.

\end{clauses}

\begin{clauses}

\clause{|for| ~var~ |in-file| ~name~ \opt\ |using| ~reader~}
\clindexx{for}{in-file}
Opens the file ~name~ (which may be a string or pathname) for input,
and iterates 
over its contents. ~reader~ defaults to |read|, so by default ~var~
will be bound to the successive forms in the file.  The \iter\ body is
wrapped in an unwind-protect to ensure that the file is closed no
matter how the \iter\ is exited.

\clause{|for| ~var~ |in-stream| ~stream~ \opt\ |using| ~reader~}
\clindexx{for}{in-stream}
Like |for\dots in-file|, except that ~stream~ should be an existing
stream object that supports input operations.

\end{clauses}

\subsubsection{Generalized Drivers}

These are primarily useful for writing drivers that can also be used
as generators (see section \ref{generators}, below).
 
\begin{clauses}

\clause{|for| ~var~ |next| ~expr~}
\clindexx{for}{next}

~var~ is set to ~expr~ each time through the loop.  Destructuring is
performed.  When the clause is used as a generator, ~expr~ is the code
that is executed when |(next ~var~)| is 
encountered (see section \ref{generators}, below).
~expr~ should compute the first value for ~var~, as well
as all subsequent values, and is responsible for terminating the loop.
For compatibility with future versions of \iter, this termination
should be done with |terminate|\clindex{terminate}, which can be
considered a synonym for |finish| (see section \ref{control-flow}).

\cpar As an example, the following clauses are equivalent to |(for
i from 1 to 10)|:
\begin{program}
(initially (setq i 0))
(for i next (if (> i 10) (terminate) (incf i)))
\end{program}

\clause{|for| ~var~ |do-next| ~form~}
\clindexx{for}{do-next}
~form~ is evaluated each time through the loop.  Its value is ~not~
set to ~var~; that is ~form's~ job. ~var~ is only present so that
\iter\ knows it is a driver variable.  \linebreak |(for ~var~ next
~expr~)| is equivalent to |(for ~var~ do-next (dsetq ~var~ ~expr~))|.
(See section \ref{destructuring} for an explanation of |dsetq|.) 

\end{clauses}

\subsubsection{Generators}
\label{generators}

In all of the above clauses, the driver variable is updated on each
iteration.
Sometimes it is desirable to have greater control over updating.  
For instance, consider the problem of associating numbers, in
increasing order and with no gaps, with the
\nonnil\ elements of a list.  One obvious first pass at writing this is:

\begin{program}
(iter (for el in list)
      (for i upfrom 1)
      (if el (collect (cons el i))))
\end{program}
But on the list |(a b nil c)| this produces |((a . 1) (b . 2) (c .
4))| instead of the desired |((a . 1) (b . 2) (c . 3))|.  The problem
is that |i| is incremented each time through the loop, even when |el|
is |nil|.  

The problem could be solved elegantly if we could step |i| only when
we wished to.  This
can be accomplished for any \iter\ driver by writing |generate| (or
its synonym |generating|)
instead of |for|.  Doing so produces a ~generator~---a driver whose
values are yielded explicitly.  To obtain the next value of a
generator variable ~v~, write \linebreak |(next ~v~)|.  The value of a |next|
form is the next value of ~v~, as determined by its associated driver
clause.  |next| also has the side-effect of updating ~v~ to that
value.  If there is no next value, |next| will terminate the loop,
just as with a normal driver.


Using generators, we can now write our example like this:
\begin{program}
(iter (for el in list)
      (generate i upfrom 1)
      (if el (collect (cons el (next i)))))
\end{program}
Now |i| is updated only when |(next i)| is executed, and this occurs
only when |el| is \nonnil.

To better understand the relationship between ordinary drivers and
generators, observe that we can rewrite an ordinary driver using its
generator form immediately followed by |next|, as this example shows:
\begin{program}
(iter (generating i from 1 to 10)
      (next i)
      ...)
\end{program}
Provided that the loop body contains no |(next i)| forms, this will
behave just as if we had written |(for i from 1 to 10)|.

We can still refer to a driver variable ~v~ without using |next|; in
this case, its value is that given to it by the last evaluation of
|(next ~v~)|.  Before |(next ~v~)| has been called the first time, the
value of ~v~ is undefined.  

This semantics is more flexible than
one in which ~v~ begins the loop bound to its first value and calls
of |next| supply subsequent values, because it means the loop will not
terminate too soon if the generator's sequence is empty.  For
instance, consider the following code, which tags \nonnil\ elements of
a list using a list of tags, and also counts the null elements.
(We assume there are at least as many tags as \nonnil\ elements.)
\begin{program}
(let* ((counter 0)
       (tagged-list (iter (for el in list)
                          (generating tag in tag-list) 
                          (if (null el)
                              (incf counter)
                              (collect (cons el (next tag)))))))
  ...)
\end{program}
It may be that there are just as many tags as non-null elements of
|list|.  If all the elements of |list| are null, we still want the
counting to proceed, even though |tag-list| is \nil.  If |tag| had to be
assigned its first value before the loop begins, we would have had to
terminate the loop before the first iteration, since when |tag-list|
is \nil, |tag| 
has no first value.  With the existing semantics, however, |(next
tag)| will never execute, so the iteration will cover all the elements of
|list|. 
        
When the ``variable'' of a driver clause is actually a destructuring
template containing several variables, all the variables are eligible
for use with |next|.  As before, |(next ~v~)| evaluates to ~v's~ next
value; but the effect is to update all of the template's variables.
For instance, the following code will return the list |(a 2 c)|.
\begin{program}
(iter (generating (key . item) in '((a . 1) (b . 2) (c . 3)))
      (collect (next key))
      (collect (next item)))
\end{program}

Only driver clauses with variables can be made into generators.  This
includes all clauses mentioned so far except for |repeat|.  It does
~not~ include |for\dots previous|, |for\dots =|,
|for\dots initially\dots then| or |for\dots first\dots then| (see
below). 

\subsubsection{Previous Values of Driver Variables}

Often one would like to access the value of a variable on a previous
iteration.  \iter\ provides a special clause for accomplishing this.

\begin{clauses}

\clause{|for| ~pvar~ |previous| ~var~ \opt\ |initially| ~init~
|back| ~n~}
\clindexx{for}{previous}
Sets ~pvar~ to the previous value of ~var,~ which should be a driver
variable, a variable from another |for\dots previous| clause, or a
variable established by a |for\dots =|, 
|for\dots initially\dots then| or |for\dots first\dots then| clause
(see section \ref{setting}).
Initially, ~pvar~ is given the value ~init~ (which defaults to \nil).  
The ~init~ expression will be moved outside the loop body, so it
should not depend on anything computed within the loop.
~pvar~ retains the value of ~init~ until ~var~ is set to its second
value, at which point ~pvar~ is set to ~var's~ first value; and so on.  

\cpar The
argument ~n~ to |back| must be a constant, positive integer, and
defaults to 1.  It determines how many iterations back ~pvar~ should
track ~var~.  For example, when ~n~ is 2, then ~pvar~ will be assigned
~var's~ first value when ~var~ is set to its third value.

\cpar A |for\dots previous| clause may occur after or before its
associated driver clause. |for\dots previous| works with generators as
well as ordinary drivers.

\pagebreak[3]

\cpar Example:
\begin{program}
(iter (for el in '(1 2 3 4))
      (for p-el previous el)
      (for pp-el previous p-el initially 0)
      (collect pp-el))
\end{program}
This evaluates to |(0 0 1 2)|.  It could have been written more
economically as 
\begin{program}
(iter (for el in '(1 2 3 4))
      (for pp-el previous el back 2 initially 0)
      (collect pp-el))
\end{program}


\end{clauses}

\subsection{Variable Binding and Setting}
\label{setting}

Several clauses exist for establishing new variable bindings or for
setting variables in the loop.  They all support destructuring.

\begin{clauses}

\clausex{with}{~var~ \opt\ |=| ~value~}
Causes ~var~ to be bound to value before the loop body is entered.
If ~value~ is not supplied, ~var~ assumes a default
binding, which will be 
\nil\ in the absence of declarations.  Also, if ~value~ is not
supplied, no destructuring is performed; instead, ~var~ may be a list of
symbols, all of which are given default 
bindings.  If ~value~ is supplied, ~var~ is bound to it, with
destructuring.  

\cpar Because |with|
creates bindings whose scope includes the entire \iter\ form, it is
good style to put all |with| clauses at the beginning.

\cpar Successive occurrences of |with| result in sequential
bindings (as with 
|let*|).  There is no way to obtain parallel bindings; see
section \ref{bindings} for a rationale.  


\clause{|for| ~var~ |=| ~expr~}
\clindexx{for}{=}
On each iteration, ~expr~ is evaluated and ~var~ is set
to its value. 

\cpar This clause may appear to do the same thing as |for\dots next|.
In fact, they are quite different.  |for\dots =| provides only three
services: it sets up a binding for ~var~, sets it to ~expr~ on each
iteration, and makes it possible to use |for\dots previous| with
~var~.  |for\dots next| provides these services in addition to the
ability to turn the driver into a generator.
%Also, the code which sets ~var~ appears in the loop body in the same
%place as the |for\dots =| clause; the code for |for\dots next| appears
%at the top of the loop, as with other drivers (except when being used
%as a generator).

\clause{|for| ~var~ |initially| ~init-expr~ |then| ~then-expr~}
\clindexxx{for}{initially}{then}
Before the loop begins, ~var~ is set to ~init-expr;~ on all
iterations after the first it is set to ~then-expr.~  This clause
must occur at top-level.  ~init-expr~ will be moved outside the loop
body and ~then-expr~ will be moved to the end of the loop body, so
they are subject to code motion problems (see section
\ref{code-movement}). 

\cpar This clause may appear to be similar to |for\dots next|, but in
fact they differ significantly.  |for\dots initially\dots then| is
typically used to give ~var~ its first value before the loop begins,
and subsequent values on following iterations.  This is incompatible
with generators, whose first value and subsequent values must all be
computed by |(next ~var~)|.  Also, the update of ~var~ in 
|for\dots initially\dots then| does not occur at the location of the
clause. 
Use |for\dots initially\dots then| for
one-shot computations where its idiom is more convenient, but use
|for\dots next| for extending \iter\ with new drivers (see section
\ref{extend}).

\clause{|for| ~var~ |first| ~first-expr~ |then| ~then-expr~}
\clindexxx{for}{first}{then}
The first time through the loop, ~var~ is set to ~first-expr;~ on
subsequent iterations, it is set to ~then-expr.~  This differs from
|for\dots initially| in that ~var~ is set to ~first-expr~
inside the loop body, 
so ~first-expr~ may depend on the results of other clauses.  For
instance, 
\begin{program}
(iter (for num in list)
      (for i first num then (1+ i))
      ...)
\end{program}
will set |i| to the first element of |list| on the first
iteration, whereas 
\begin{program}
(iter (for num in list)
      (for i initially num then (1+ i))
      ...)
\end{program}
is probably erroneous; |i| will be bound to |num|'s
default binding (usually \nil) for the first iteration.

\end{clauses}

\begin{note}{Compatibility note}
|loop|'s |for\dots =| works like \iter's, but |loop| used the syntax
|for\dots =\dots then| to mean |for\dots initially\dots then|.  It was
felt that these two operations were sufficiently different to warrant
different keywords.

Also, the |for| in the above three clauses is misleading,
since none is true driver (e.g. none has a corresponding |generate|
form).  |setting| would have been a better choice, but |for| was used
to retain some compatibility with |loop|.  
\end{note}


%%%%%%%%%%%%%%%%%%%%%%%%%%%%%%%%
\subsection{Gathering Clauses}
Many of \iter's clauses accumulate values into a variable, or set a
variable under certain conditions.  At the end of the loop, this
variable contains the desired result.  All these clauses have an
optional |into| keyword, whose argument should be a symbol.  If the
|into| keyword is not supplied, the accumulation variable will be
internally generated and its value will be returned at the end of the
loop; if a variable is specified, that variable is used for the
accumulation, and is not returned as a result---it is up to the user
to return it explicitly, in the loop's epilogue code (see |finally|).
It is safe to examine the accumulation variable during the loop, but
it should not be modified.

These clauses all begin with a verb.  When the verb does
not conflict with an existing Common Lisp function, then it may be
used in either its infinitival or present-participle form (e.g. |sum|,
|summing|).  However, when there is a conflict with Common Lisp, only
the present-participle form may be used (e.g. |unioning|).  This is to
prevent \iter\ clauses from clashing with Common Lisp
functions. 

%although these clauses are described as ``producing a value,'' it is a
%mistake to think of the lisp list representing the clause as a
%value-producing form in the usual way.  clauses may legally be written
%where a value is expected, e.g. |(setq x (sum i))|, but the
%lisp value of a clause in such a context is undefined.

\subsubsection{Reductions}

~Reduction~ is an extremely common iteration pattern in which
the results of successive applications of a
binary operation are accumulated.
For example, a loop that computes the sum of the
elements of a list is performing a reduction with the addition
operation.  This could be written in Common Lisp as  |(reduce \#'+
list)| or with \iter\ as
\begin{program}
(iter (for el in list)
      (sum el))
\end{program}

\begin{clauses}

\clausex{sum}{~expr~ \opt\ |into| ~var~}
Each time through the loop, ~expr~ is evaluated and added to a
variable, which is bound initially to zero.  If ~expr~ has a type,
it is ~not~ used as the type of the sum variable, which is always
|number|.  To get the result variable to be of a more specific
type, use an explicit variable, as in
\begin{program}
(iter (for el in number-list)
      (sum el into x)
      (declare (fixnum x))
      (finally (return x)))
\end{program}

\clausex{multiply}{~expr~ \opt\ |into| ~var~}
Like |sum|, but the initial value of the result variable is $1$,
and the variable is updated by multiplying ~expr~ into it.

\clausex{counting}{~expr~ \opt\ |into| ~var~}
~expr~ is evaluated on each iteration.  If it is \nonnil, the
accumulation variable, initially zero, is incremented.

\startitem
\Clausex{maximize}{~expr~ \opt\ |into| ~var~}
\Clausex{minimize}{~expr~ \opt\ |into| ~var~}
\finishitem
~expr~ is evaluated on each iteration and its extremum (maximum or
minimum) is stored in the accumulation variable.  If ~expr~ is never
evaluated, then the result is \nil\ (if the accumulation variable is
untyped) or $0$ (if it has a numeric type).

\clausex{reducing}{~expr~ |by| ~func~ \opt\ |initial-value| ~init-val~ |into| ~var~}
This is a general way to perform reductions. ~func~ should be a
function of two arguments, the first of which will be the value
computed so far and
the second of which will be the value of ~expr~.  It should return the
new value.
|reducing| is roughly equivalent to the
Common Lisp |(reduce ~func~ ~sequence~ :key ~expr-function~)|, where
~expr-function~ is used to derive values from the successive elements
of ~sequence~.

\cpar If the |reducing| clause is never executed, the result is
undefined.

\cpar It is not necessary to provide an
initial value, but better code can be generated if one is supplied.
Regardless of its location in the \iter\ body, ~init-val~ will be
evaluated before the loop is entered, so it should not depend on any
value computed inside the \iter\ form.  

%\cpar if a ~var~ is not specified, you can get \iter\ to declare the
%type of the internal variable by putting a |the| expression around
%~func.~  see section \ref{types}.

\end{clauses}

%\begin{note}{implementation note}
%in principle, |maximize| and |minimize| can be thought of
%as reductions where the initial value is the smallest (or largest)
%value that the accumulation variable can assume.  because lisp's
%bignums can represent arbitrary integers, these clauses cannot be
%implemented as reductions in general.  if, however, the type of 
%~expr~ or ~var~ can be determined to be a fixnum
%or a float, \iter\ will implement the clause as a true reduction,
%using one of the constants |most-negative-fixnum|, |%most-positive-fixnum|, 
%|most-negative-short-float|, etc. as appropriate.
%\end{note}

\subsubsection{Accumulations}

All the predefined accumulation clauses add values to a sequence.  If
the sequence is a list, they all 
have the property that the partial list is kept in the correct order
and available for inspection at any point in the loop.

\begin{clauses}

\clausex{collect}{~expr~ \opt\ |into| ~var~ |at| ~place~ |result-type| ~type~}
Produces a sequence of the values of ~expr~ on each iteration. ~place~
indicates where the next value of ~expr~ is added to the list and may
be one of the symbols |start|, |beginning| (a synonym for |start|) or
|end|.  The symbol may be quoted, but need not be.  The default is
|end|.  For example,
\begin{program}
(iter (for i from 1 to 5)
      (collect i))
\end{program}
produces |(1 2 3 4 5)|, whereas
\begin{program}
(iter (for i from 1 to 5)
      (collect i at beginning))
\end{program}
produces |(5 4 3 2 1)| (and is likely to be faster in most Common Lisp
implementations).

\cpar If ~type~ is provided, it should be a subtype of |sequence|.
The default is |list|.  Specifying a type other than |list| will
result in |collect| returning a sequence of that type.  ~However,~ the
type of the sequence being constructed when inside the loop body is
undefined when a non-|list| type is specified.  (As with ~place~,
quoting ~type~ is optional.)

\clausex{adjoining}{~expr~ \opt\ |into| ~var~ |test| ~test~ |at| ~place~ |result-type| ~type~}
Like |collect|, but only adds the value of ~expr~ if it is not
already present.  ~test,~ which defaults to |\#'eql|, is
the test to be used with |member|.

\startitem
\Clausex{appending}{~expr~ \opt\ |into| ~var~ |at| ~place~}
\Clausex{nconcing}{\ ~expr~ \opt\ |into| ~var~ |at| ~place~}
\Clausex{unioning}{\ ~expr~ \opt\ |into| ~var~ |test| ~test~ |at| ~place~}
\Clausex{nunioning}{~expr~ \opt\ |into| ~var~ |test| ~test~ |at| ~place~}
\finishitem
These are like |collect|, but behave like the Common Lisp functions
|append|, |nconc|, |union| or |nunion|.  
As in Common Lisp, they work only on lists.  Also as in Common Lisp,
|unioning| and |nunioning| assume that the value of ~expr~ contains no
duplicates. 

\clausex{accumulate}{~expr~ |by| ~func~ \opt\ |initial-value| ~init-val~ |into| ~var~}
This is a general-purpose accumulation clause. ~func~ should be a
function of two arguments, the value of ~expr~ and the value
accumulated so far in the iteration, and it should return the updated
value.  If no initial value is supplied, \nil\ is used.  

%\cpar If a ~var~ is not specified, you can get \iter\ to declare the
%type of the internal variable by putting a |the| expression around
%~func.~  see section \ref{types}.

\cpar The differences between |accumulate| and |reducing| are slight.
One difference is that the functions take their arguments in a
different order.  Another is that in the absence of ~init-val~,
|accumulate| will use \nil, whereas |reducing| will generate different
code that avoids any dependence on the initial value.
The reason for having both clauses is that one usually
thinks of reductions (like |sum|) and accumulations (like |collect|)
as different beasts.

\end{clauses}


\subsubsection{Finders}

A ~finder~ is a clause whose value is an expression that meets some
condition.

\begin{clauses}

\clause{|finding| ~expr~ |such-that| ~test~ \opt\ |into| ~var~ |on-failure| ~failure-value~}
\clindexx{finding}{such-that}
If ~test~ (which is an expression) ever evaluates to \nonnil, the loop
is terminated, the 
epilogue code is run and the value of ~expr~ is returned.  Otherwise,
\nil\ (or ~failure-value,~ if provided) is returned.  If ~var~ is
provided, it will have either the \nonnil\ value of ~expr~ or
~failure-value~ when the epilogue code is run.

\cpar As a special case, if the ~test~ expression is a sharp-quoted
function, then it is applied to ~expr~ instead of being simply
evaluated.  E.g. |(finding x such-that \#'evenp)| is equivalent to
|(finding x such-that (evenp x))|.  

%\cpar although ~test~ need have nothing to do with ~%expr~ as in
%|(finding j such-that (> i 3))|, it usually 
%will: |(finding (length el) such-that (oddp (length el)))|.  to
%avoid performing the |length| computation twice, you could write
%|(finding (length el) such-that \#'oddp)| or |(finding (length
%el) such-that 'oddp)|; for these cases, \iter\ generates code that
%executes ~expr~ only once.  the code for |\#'oddp|
%is slightly different from that for {\lisp 'oddp}; see the discussion
%under {\lisp for\dots in} and {\lisp for\dots on}.

\cpar |On-failure| is a misnomer. Because it is always evaluated, it behaves
more like the default third argument to the |gethash| function. As a result,
|on-failure (error "Not found")| makes no sense. Instead, the clauses |leave|
or |thereis| can be used in conjunction with |finally| as follows:
\begin{program}
(iter (for x in '(1 2 3))
      (if (evenp x) (leave x))
      (finally (error "not found")))
\end{program}

\cpar This clause may appear multiple times when all defaults are
identical. It can also be used together with either |always|/|never| or
|thereis| if their defaults match. More specifically, |on-failure nil| is
compatible with |thereis|, while |on-failure t| is compatible with |always|
and |never| clauses.
\begin{program}
(iter (for i in '(7 -4 2 -3))
      (if (plusp i)
	  (finding    i  such-that (evenp i))
          (finding (- i) such-that (oddp i))))
\end{program}

\startitem
\Clause{|finding| ~expr~ |maximizing| ~m-expr~ \opt\ |into| ~var~}
\clindexx{finding}{maximizing}
\Clause{|finding| ~expr~ |minimizing| ~m-expr~ \opt\ |into| ~var~}
\clindexx{finding}{minimizing}
\finishitem
Computes the extremum (maximum or minimum) value of ~m-expr~ over all
iterations, and returns the value of ~expr~ corresponding to the
extremum.  ~expr~ is evaluated inside the loop at the time the new
extremum is established.  If ~m-expr~ is never evaluated (due to, for
example, being embedded in a conditional clause), then the returned
value depends on the type, if any, of ~expr~ (or ~var,~ if
one is supplied).  If there is no type, the returned
value will be \nil; if the type is numeric, the returned value will be
zero.

\cpar For these two clauses, ~var~ may be a list of two
symbols; in that case, the first is used to record ~expr~ and
the second, ~m-expr.~  

\cpar As with |finding\dots such-that|, if ~m-expr~ is a sharp-quoted
function, then it is called on ~expr~ instead of being evaluated.

\end{clauses}

\subsubsection{Boolean Tests}

\begin{clauses}

\clausex{first-iteration-p}{}
Returns |t| in the first cycle of the loop, otherwise \nil.

\clausex{first-time-p}{}
Returns |t| the first time the expression is evaluated, and then \nil\ forever.
This clause comes handy when printing (optional) elements separated
by a comma:

\begin{program}
(iter (for el in '(nil 1 2 nil 3))
      (when el
        (unless (first-time-p)
          (princ ", "))
        (princ el)))
\end{program}
produces |"1, 2, 3"|.

\end{clauses}

\subsubsection{Aggregated Boolean Tests}

\begin{clauses}

\clausex{always}{~expr~}
If ~expr~ ever evaluates to
\nil, then \nil\ is immediately returned; the epilogue code is not
executed.  If ~expr~ never evaluates to \nil, the epilogue code
is executed and the last value of ~expr~ (or |t| if ~expr~ was never
evaluated) is returned (whereas |loop| would constantly return |t|).

% mention last evaluated clause when multiple always clauses?

\clausex{never}{~expr~}
Like |(always (not ~expr~))|, except it does not influence the last
value returned by a possible other |always| clause. That is,
\begin{program}
(iter (repeat 2)
      (always 2)
      (never nil)) \yields 2 ; not t
\end{program}

\clausex{thereis}{~expr~}
If ~expr~ is ever \nonnil,
its value is immediately returned without running epilogue code.
Otherwise, the epilogue code is performed and \nil\ is returned.

This clause cannot be used together with |always| or |never|, because their
defaults are opposed (similarly, |(loop always 3 thereis nil)| refuses to
compile in some implementations of |loop|).

\end{clauses}


\subsection{Control Flow}
\label{control-flow}
Several clauses can be used to alter the usual flow of control in a loop.

Note: the clauses of this and subsequent sections don't adhere to \iter's
usual syntax, but instead use standard Common Lisp syntax.  Hence the
format for describing syntax subsequently is
like the standard format used in the Common Lisp manual, not like the
descriptions of clauses above.

\begin{clauses}

\clausex{finish}{}
Stops the loop and runs the epilogue code.

%for example:
%
%\begin{program}
%(iter (with answer = nil)
%         (initially (make-a-mess))
%         (for i from 1 to 10)
%         (when (correct? i)
%           (setq answer i)
%           (finish))
%         (finally (cleanup)))
%\end{program}
%
%this code will execute |cleanup| whether or not the test |(correct?
%i)| ever succeeds. 
%the (more elegant) formulation,
%\begin{program}
%(iter (initially (make-a-mess))
%         (for i from 1 to 10)
%         (finding i such-that (correct? i))
%         (finally (cleanup)))
%\end{program}
%would not execute |cleanup| if |(correct? i)| succeeded; it
%would do an immediate return.

\clausex{leave}{\opt\ ~value~}
Immediately returns ~value~ (default \nil) from the current \iter\
form, skipping the epilogue code.  Equivalent to using |return-from|.

\clausex{next-iteration}{}
Skips the remainder of the loop body and begins the next iteration of
the loop.

\clausex{while}{~expr~}
If ~expr~ ever evaluates to \nil, the loop is terminated and the
epilogue code executed.  Equivalent to |(if (not ~expr~) (finish))|.

\clausex{until}{~expr~}
Equivalent to |(if ~expr~ (finish))|.

\clausex{if-first-time}{~then~ \opt\ ~else~}
If this clause is being executed for the first time in this invocation
of the \iter\ form, then the ~then~ code is evaluated; otherwise the
~else~ code is evaluated.

\cpar |(for ~var~ first ~expr1~ then ~expr2~)| is almost equivalent to
\begin{program}
(if-first-time (dsetq ~var~ ~expr1~)
               (dsetq ~var~ ~expr2~))
\end{program}
The only difference is that the |for| version makes ~var~ available
for use with |for\dots previous|.

\end{clauses}

\subsection{Code Placement}
When fine control is desired over where code appears in a loop
generated by \iter, the following special clauses may be useful.
They are all subject to code-motion problems (see section
\ref{code-movement}).  

\begin{clauses}

\clausex{initially}{|\&rest| ~forms~}
The lisp ~forms~ are placed in the prologue section of the loop, where
they are executed once, before the loop body is entered.

\clausex{after-each}{|\&rest| ~forms~}
The ~forms~ are placed at the end of the loop body, where they
are executed after each iteration.  Unlike the other clauses in this
section, ~forms~ may contain \iter\ clauses.

\clausex{else}{|\&rest| ~forms~}
The lisp ~forms~ are placed in the epilogue section of the loop, where they
are executed if this |else| clause is never met during execution of the
loop and the loop terminates normally.

\clausex{finally}{|\&rest| ~forms~}
The lisp ~forms~ are placed in the epilogue section of the loop, where
they are executed after the loop has terminated normally.

\clausex{finally-protected}{|\&rest| ~forms~}
The lisp ~forms~ are placed in the second form of an unwind-protect
outside the loop.  They are always executed after the loop has
terminated, regardless of how the termination occurred.

\end{clauses}


%%%%%%%%%%%%%%%%%%%%%%%%%%%%%%%%%%%%%%%%%%%%%%%%%%%%%%%%%%%%%%%%
\section{Other Features}

\subsection{Multiple Accumulations}
\label{multiple}

\begin{sloppypar}
It is permitted to have more than one clause accumulate into the same
variable, as in the following:
\begin{program}
(iter (for i from 1 to 10)
      (collect i into nums)
      (collect (sqrt i) into nums)
      (finally (return nums)))
\end{program}
Clauses can only accumulate into the same variable if they are
compatible.  |collect|, |adjoining|, |appending|, |nconcing|,
|unioning| and |nunioning| are compatible with each other; |sum|,
|multiply| and |counting| are compatible; |always| and |never| are compatible;
|finding| \dots |such-that| is compatible with either |thereis| or |always|
and |never| when their defaults match; and |maximize| and |minimize| clauses
are compatible only with other |maximize| and |minimize| clauses,
respectively.

%note that the same variable ~cannot~ be both an accumulation
%variable and an ordinary variable; there can be only one variable with
%a given name within an \iter\ form. 
\end{sloppypar}

\subsection{Named Blocks}

Like Common Lisp |block|s, \iter\ forms can be given names.  The
name should be a single symbol, and it must be the first form in the
\iter.  The generated code behaves exactly like a named block; in
particular, |(return-from ~name~)| can be used to exit it:
\begin{program}
(iter fred
      (for i from 1 to 10)
      (iter barney
            (for j from i to 10)
            (if (> (* i j) 17)
                (return-from fred j))))
\end{program}
An \iter\ form that is not given a name is implicitly named \nil.

Sometimes one would like to write an expression in an inner \iter\ form,
but have it processed by an outer \iter\ form.  This is possible with
the |in| clause.  

\begin{clauses}

\clausex{in}{~name~ |\&rest| ~forms~}
Evaluates ~forms~ as if they were part of the \iter\ form named
~name~.  In other words, \iter\ clauses are processed by the \iter\
form named ~name,~ and not by any \iter\ forms that occur inside ~name.~

\cpar As an example, consider the problem of collecting a list of the
elements in a two-dimensional array.  The naive solution,
\begin{program}
(iter (for i below (array-dimension ar 0))
      (iter (for j below (array-dimension ar 1))
            (collect (aref ar i j))))
\end{program}
\noindent is wrong because the list created by the inner \iter\ is simply
ignored by the outer one.  But using |in| we can write:
\begin{program}
(iter outer (for i below (array-dimension ar 0))
      (iter (for j below (array-dimension ar 1))
            (in outer (collect (aref ar i j)))))
\end{program}
\noindent which has the desired result.

\end{clauses}

\subsection{Destructuring}
\label{destructuring}

In many places within \iter\ clauses where a variable is expected, a
list can be written instead.  In these cases, the value to be assigned
is ~destructured~ according to the pattern
described by the list.  As a simple example, the clause
\begin{program}
(for (key . item) in alist)
\end{program}
\noindent will result in |key| being set to the |car| of
each element in |alist|, and |item| being set to the |cdr|.  The
pattern list may be nested to arbitrary depth, and (as the example
shows) need not be terminated with \nil; the only requirement is that
each leaf be a bindable symbol (or \nil, in which case no binding is
generated for that piece of the structure).

Sometimes, you might like to do the equivalent of a
|multiple-value-setq| in a clause.  This 
``multiple-value destructuring'' can be expressed by writing 
\linebreak |(values $pat_1$ $pat_2 \ldots$)| for a destructuring
pattern, as in 
\begin{program}
(for (values (a . b) c d) = (three-valued-function ...))
\end{program}
\begin{sloppypar}
Note that the $pat_i$ can themselves be destructuring patterns (though
not multiple-value destructuring patterns).  You can't do multiple-value
destructuring in a |with| clause; instead wrap the whole \iter\
form in a |multiple-value-bind|.
\end{sloppypar}

\begin{note}{Rationale}
There are subtle interactions between variable declarations and
evaluation order that make the correct implementation of
multiple-value destructuring in a |with| somewhat tricky.
\end{note}

The destructuring feature of \iter\ is available as a separate
mechanism, using the |dsetq| macro:

\begin{clauses}
\defmacro{dsetq}{template expr}
Performs destructuring of ~expr~ using ~template~.   May be used
outside of an \iter\ form. Yields the primary value of ~expr~.

\end{clauses}

\subsection{On-line Help}

\begin{sloppypar}
There is a limited facility for on-line help, in the form of the 
|display-iterate-clauses| function.
\end{sloppypar}

\begin{clauses}

\defun{display-iterate-clauses}{\opt\ clause-spec}
Displays a list of \iter\ clauses.  If ~clause-spec~ is not
provided, all clauses are shown; if it is a symbol, all clauses
beginning with that symbol are shown; and if it is a list of symbols,
all clauses for which ~clause-spec~ is a prefix are shown.

\end{clauses}

\subsection{Parallel Binding and Stepping}
\label{bindings}

The parallel binding and stepping of variables is a feature that
\iter\ does ~not~ have.  This section attempts to provide a rationale.

We say that two variables are bound ~in parallel~ if neither
binding shadows the other.  This is the usual semantics of |let|
(as opposed to |let*|).  Similarly, we can say that iteration
variables are stepped in parallel if neither variable is updated
before the other, conceptually speaking; in other words, if the code
to update each variable can reference the old values of both variables.

|loop| allows parallel binding of variables and parallel stepping of
driver variables.  My view is that if you are depending on the 
serial/parallel distinction, you are doing something obscure.  If you
need to bind 
variables in parallel using |with|, then you must be using a
variable name that shadows a name in the existing lexical environment.
Don't do that.  The most common use for parallel stepping is to track
the values of variables on the previous iteration, but in fact this
does not require parallel stepping at all; the following will work:
\begin{program}
(iter (for current in list)
      (for prev previous current)
      ...)
\end{program}


%%%%%%%%%%%%%%%%%%%%%%%%%%%%%%%%%%%%%%%%%%%%%%%%%%%%%%%%%%%%%%%%
\section{Types and Declarations}
\label{types}

\subsection{Discussion}

Sometimes efficiency dictates that the types of variables be declared.
This type information needs to be communicated to \iter\ so it can
bind variables to appropriate values.  Furthermore, \iter\ must often
generate internal variables invisible to the user; there needs to be a
way for these to be declared.

As an example, consider this code, which will return the number of
odd elements in |number-list|:
\begin{program}
(iter (for el in number-list)
      (count (oddp el)))
\end{program}
In processing this form,
\iter\ will create an internal variable, let us call it |list17|, to
hold the successive |cdr|s of |number-list|, and  
will bind the variable to |number-list|.  It will also generate a
default binding for |el|; only inside the body of the loop will |el|
be set to the |car| of |list17|.  Finally, \iter\ will generate a
variable, call it |result|, to hold the result of the count, and will
bind it to zero.

When dealing with type declarations, \iter\ observes one simple rule:
~it will never generate a declaration unless requested to do so.~  The
reason is that such declarations might mask errors in compiled code by
avoiding error-checks; the resulting problems would be doubly hard to
track down because the declarations would be hidden from the
programmer.  Of course, a compiler might omit error-checks even in the
absence of declarations, though this behavior can usually be avoided,
e.g. by saying |(declaim (optimize (safety 3)))|.

So, the above \iter\ form will generate code with no declarations.
But say we wish to declare the types of |el| and the internal
variables |list17| and |result|.  How is this done?

Declaring the type of |el| is easy, since the programmer knows
the variable's name:
\begin{program}
(iter (for el in number-list)
      (declare (fixnum el))
      (counting (oddp el)))
\end{program}
\iter\ can read variable type declarations like this one.  Before
processing any clauses, it scans the entire top-level form for type
declarations 
and records the types, so that variable bindings can be performed
correctly.  In this case, |el| will be bound to zero
instead of \nil.  Also, \iter\ collects all the top-level declarations
and puts them at the begining of the generated code, so it is not
necessary to place all declarations at the beginning of an \iter\
form; instead, they can be written near the variables whose types they
declare.  

Since \iter\ is not part of the compiler, it will not know
about declarations that occur outside an \iter\ form; these
declarations must be repeated inside the form.

Here is another way we could have declared the type of |el|:
\begin{program}
(iter (for (the fixnum el) in number-list)
      (counting (oddp el)))
\end{program}
\iter\ extends the Common Lisp |the|\lispindex{the}
form to apply to variables as well as value-producing forms; anywhere
a variable is allowed---in a |with| clause, as the iteration
variable in a driver clause, as the |into| argument of an
accumulation clause, even inside a destructuring template---you can
write |(the ~type~ ~symbol~)| instead.

There is one crucial difference between using a |the| form and
actually declaring the variable: explicit declarations are always
placed in the generated code, but type information from a |the|
form is not turned into an actual declaration unless you tell \iter\
to do so using |iterate:declare-variables|.  See below.

\begin{sloppypar}
Declaring the types of internal variables is harder than declaring the
types of explicitly mentioned variables, since their names
are unknown.  You do it by declaring |iterate:declare-variables|
somewhere inside the top level of the \iter\ form.  (This will also
generate declarations for variables declared using |the|.)
\iter\ does not provide much selectivity here: it's all or none. 
And unfortunately, since \iter\ is not privy to compiler information
but instead reads declarations itself, it will not hear if you
|(declaim (iterate:declare-variables))|.  Instead, set the variable 
|iterate::*always-declare-variables*| to |t| at
compile-time, using |eval-when|.
\end{sloppypar}

To determine the appropriate types for internal variables, \iter\ uses
three sources of information:

\begin{itemize}

\item Often, the particular clause dictates a certain type for a
variable; \iter\ will use this information when available.  In the
current example, the variable |list17| will be given the type
|list|, since that is the only type that makes sense; and the
variable |result| will be given the type |fixnum|, on the
assumption that you will not be counting high enough to need bignums.
You can override this assumption only by using and explicitly declaring a
variable: 
\begin{verbatim}
(iter (declare (iterate:declare-variables))
      (for el in number-list)
      (count (oddp el) into my-result)
      (declare (integer my-result))
      (finally (return my-result)))
\end{verbatim}

\begin{sloppypar}
Other examples of the type assumptions that \iter\ makes are: type
|list| for |into| variables of collection clauses; type |list| for
expressions that are to be destructured; type |vector| for the
variable holding the vector in a |for\dots in-vector| clause, and
similarly for |string| and the |for\dots in-string| clause;
and the implementation-dependent type
|(type-of array-dimension-limit)| for the index and limit
variables generated by sequence iteration drivers like |for\dots
in-vector| and |for\dots in-string| (but not |for\dots in-sequence|,
because it may be used to iterate over a list). 
\end{sloppypar}

\item Sometimes, \iter\ will examine expressions and try to determine
their types in a simple-minded way.  If the expression is
self-evaluating (like a number, for instance), \iter\ knows that the
expression's type is the same as the type of the value it denotes, so
it can use that type.  If the expression is of the form |(the ~type~
~expr~)|, \iter\ is smart enough to extract ~type~ and use it.
However, the current version of \iter\ does not  
examine declarations of function result types or do any type
inference.  It will not determine, for 
example, that the type of |(+ 3 4)| is |fixnum|, or even
|number|.

\item In some cases, the type of an internal variable should match the
type of some other variable.  For instance, \iter\ generates an
internal variable for |(f x)| in the
clause |(for i from 1 to (f x))|, and in the absence of other
information will give it the same type as |i|.  If, however, the
expression had been written |(the fixnum (f x))|, then \iter\
would have given the internal variable the type |fixnum|
regardless of |i|'s type.  The type incompatibility errors that
could arise in this situation are not checked for.

\end{itemize}

Note that if you do declare |iterate:declare-variables|, then
\iter\ may declare user variables as well as internal ones if they do
not already have declarations, though only for variables that it
binds.  For instance, in this code:

\begin{program}
(iter (declare (iterate:declare-variables))
      (for i from 1 to 10)
      (collect i into var))
\end{program}
the variable |var| will be declared to be of type |list|.


\subsection{Summary}

\iter\ understands standard Common Lisp variable type declarations
that occur within an \iter\ form and
will pass them through to the generated code.  If the declaration
|(iterate:declare-variables)|\lispindex{declare-variables}
appears at the top level of an 
\iter\ form, or if 
|iterate::*always-declare-variables*|\lispindex{*always-declare-variables*} 
is \nonnil, then \iter\ will use the type information gleaned from user
declarations, self-evaluating expressions and |the| expressions,
combined with reasonable assumptions, to determine variable
types and declare them.


%%%%%%%%%%%%%%%%%%%%%%%%%%%%%%%%%%%%%%%%%%%%%%%%%%%%%%%%%%%%%%%%
\section{Problems with Code Movement}
\label{code-movement}

Some \iter\ clauses, or parts of clauses, result in code being
moved from the location of the clause to other parts of the loop.
Drivers behave this way, as do code-placement clauses like |initially|
and |finally|.  When using these clauses, there is a danger of writing
an expression that makes sense in its apparent location but will be
invalid or have a different meaning in another location.  For example:
\begin{program}
(iter (for i from 1 to 10)
      (let ((x 3))
        (initially (setq x 4))))
\end{program}
While it may appear that the |x| of |(initially (setq x 4))| is the
same as the |x| of |(let ((x 3)) \dots|, in fact they are not:
|initially| moves its code outside the loop body, so |x| would refer
to a global variable.  Here is another example of the same problem:
\begin{program}
(iter (for i from 1 to 10)
      (let ((x 3))
        (collect i into x)))
\end{program}
If this code were executed, |collect| would create a binding for its
|x| at the top level of the \iter\ form that the |let| will shadow.

Happily, \iter\ is smart enough to catch these errors; it
walks all problematical code to ensure that free variables are not
bound inside the loop body, and checks all variables it binds for the
same problem.

However, some errors cannot be caught:

\begin{program}
(iter (with x = 3)
      (for el in list)
      (setq x 1)
      (reducing el by \#'+ initial-value x))
\end{program}
|reducing| moves its |initial-value| argument to the initialization
part of the loop in order to produce more efficient code.  Since
\iter\ does not perform data-flow analysis, it cannot determine that
|x| is changed inside the loop; all it can establish is that |x| is
not bound internally.  Hence this code will not signal an
error and will use $3$ as the initial value of the reduction.

The following list summarizes all cases that are subject to these code
motion and variable-shadowing problems.
\begin{itemize}
\item Any variable for which \iter\ creates a binding, including those
used in |with| and the |into| keyword of many clauses.

\begin{sloppypar}
\item The special clauses which place code: |initially|, |after-each|, |else|,
|finally| and |finally-protected|. 
\end{sloppypar}

\item The variables of a |next| or |do-next| form.

\item The |initially| arguments of |for\dots initially\dots then| and
|for\dots previous|. 

\item The |then| argument of |for\dots initially\dots then|.

\item The |initial-value| arguments of |reducing| and |accumulate|.

\item The |on-failure| argument of |finding\dots such-that|.
\end{itemize}

%%%%%%%%%%%%%%%%%%%%%%%%%%%%%%%%%%%%%%%%%%%%%%%%%%%%%%%%%%%%%%%%
\section{Differences Between {\tt Iterate} and {\tt Loop}}
|loop| contains a great deal of complexity which \iter\ tries to
avoid.  Hence many esoteric features of |loop| don't exist in \iter.
Other features have been carried over, but in a cleaned-up form.  
And of course, many new features have been added; they are not
mentioned in this list.

\begin{itemize}

\item \iter's syntax is more Lisp-like than |loop|'s, having a higher
density of parens.

\item The current implementation of \iter, unlike the current version
of  |loop| (as documented in {\em Common Lisp, 2nd Ed.\/}), is
extensible (see section \ref{extend}).  

\item |loop| puts the updates of all driver variables at the top of
the loop; \iter\ leaves them where the driver clauses appear.

\item While for the most part \iter\ clauses that resemble |loop| clauses
behave similarly, there are some differences.  For instance, there is
no |for\dots =\dots then| in \iter; instead use 
|for\dots initially\dots then|.  

\item |loop| binds the variable |it| at certain times
to allow pseudo-English expressions like |when ~expr~ return it|. 
In \iter, you must bind ~expr~ to a variable yourself.  Note that
|when ~expr~ return it| is like |thereis ~expr~| except that the latter is an
accumulation clause and therefore competes with other accumulations
(remember section \ref{multiple} above).

% repeat different behaviour of |always| clause here?

\item |loop| has a special |return| clause, illustrated in the
previous item.  \iter\ doesn't need one,  since an ordinary Lisp
|return| has the same effect.

\item |loop| allows for parallel binding and stepping of iteration
variables.  \iter\ does not.  (See section \ref{bindings}.)

\item |loop| and \iter\ handle variable type declarations very
differently.  |loop| provides a special syntax for declaring variable
types, and does not examine declarations.  Moreover, the standard
implementation of |loop| will
generate declarations when none are requested.
\iter\ parses standard Common Lisp type declarations, and will never
declare a variable itself unless declarations are 
specifically requested.

\end{itemize}

%%%%%%%%%%%%%%%%%%%%%%%%%%%%%%%%%%%%%%%%%%%%%%%%%%%%%%%%%%%%%%%%
\section{Rolling Your Own}
\label{extend}

\subsection{Introduction}

\iter\ is extensible---you can write new clauses that embody new
iteration patterns.  You might want to write a new driver clause for a
data structure of your own, or you might want to write a clause that
collects or manipulates elements in a way not provided by \iter. 

This section describes how to write clauses for \iter.  Writing a
clause is like writing a macro.  In fact, writing a clause ~is~
writing a macro: since \iter\ code-walks its body and macroexpands,
you can add new abstractions to \iter\ with good old |defmacro|.

Actually, there are two extensions you can make to \iter\ that are
even easier than writing a macro.  They are adding a synonym for an
existing clause and defining a driver clause for an indexable
sequence.  These can be done with |defsynonym| and
|defclause-sequence|, respectively.  See section \ref{aids}, below.

The rest of this section explains how to write macros that expand into
\iter\ clauses.
Here's how you could add a simplified version of \iter's |multiply|
clause, if \iter\ didn't already have one:
\begin{program}
(defmacro multiply (expr)
  `(reducing ,expr by \#'* initial-value 1))
\end{program}

If you found yourself summing the square of an expression often, you
might want to write a macro for that.  A first cut might be
\begin{program}
(defmacro sum-of-squares (expr)
  `(sum (* ,expr ,expr)))
\end{program}
but if you are an experienced macro writer, you will realize that this
code will evaluate ~expr~ twice, which is probably a bad idea.  A
better version would use a temporary:
\begin{program}
(defmacro sum-of-squares (expr)
  (let ((temp (gensym)))
    `(let ((,temp ,expr))
       (sum (* ,temp ,temp)))))
\end{program}
Although this may seem complex, it is just the sort of thing you'd
have to go through to write any macro, which illustrates the point of
this section: if you can write macros, you can extend \iter.

Our macros don't use \iter's keyword-argument
syntax.  We could just use keywords with |defmacro|, but we would
still not be using \iter's clause indexing mechanism.  Unlike Lisp,
which uses just the first symbol of a form to determine what function
to call, \iter\ individuates clauses by the list of required keywords.
For instance, |for\dots in| and |for\dots in-vector| are different clauses
implemented by distinct Lisp functions.  

To buy into this indexing scheme, as well as the keyword-argument
syntax, use |defmacro-clause|:

\begin{clauses}

\defmacro{defmacro-clause}{arglist |\&body| body}
Defines a new \iter\ clause.  ~arglist~ is a list of symbols which are
alternating keywords and arguments.  \opt\ may be used, and the list
may be terminated by |\&sequence|.  ~body~ is an ordinary macro body,
as with |defmacro|.  If the first form of ~body~ is a string, it is
considered a documentation string and will be shown by
|display-iterate-clauses|. |defmacro-clause| will signal an error if 
defining the clause would result in an ambiguity.  E.g. you cannot
define the clause |for\dots from| because there would be no way to
distinguish it from a use of the |for| clause with optional keyword |from|.

\end{clauses}

\medskip

Here is |multiply| using |defmacro-clause|.  The keywords are capitalized
for readability.
\begin{program}
(defmacro-clause (MULTIPLY expr \opt\ INTO var)
  `(reducing ,expr by \#'* into ,var initial-value 1))
\end{program}
You don't have to worry about the case when |var| is not supplied; for
any clause with an |into| keyword, saying |into nil| is equivalent to
omitting the |into| entirely.

As another, more extended example, consider the fairly common
iteration pattern that 
involves finding the sequence element that maximizes (or minimizes) some
function.  \iter\ provides this as |finding\dots maximizing|, but it's
instructive to see how to write it.
Here, in pseudocode, is how you might write such a loop for
maximizing a function F:

\begin{quote}
\begin{tabbing}
set variable MAX-VAL to NIL; \\
set variable WINNER to NIL; \\
for\=\ each element EL in the sequence \\
\>  if\=\ MAX-VAL\=\ is NIL or F(EL) $>$ MAX-VAL then \\
\>\>    set MAX-VAL to F(EL); \\
\>\>    set WINNER to EL; \\
\>  end if; \\
end for; \\
return WINNER.
\end{tabbing}
\end{quote}

Here is the macro:
\begin{program}
(defmacro-clause (FINDING expr MAXIMIZING func \opt\ INTO var)
  (let ((max-val (gensym))
        (temp1 (gensym))
        (temp2 (gensym))
        (winner (or var iterate::*result-var*)))
    `(progn 
       (with ,max-val = nil)
       (with ,winner = nil)
       (cond 
        ((null ,max-val)
         (setq ,winner ,expr)
         (setq ,max-val (funcall ,func ,winner))
        (t
         (let* ((,temp1 ,expr)
                (,temp2 (funcall ,func ,temp1)))
           (when (> ,temp2 ,max-val)
             (setq ,max-val ,temp2)
             (setq ,winner ,temp1))))))
       (finally (leave ,winner)))))
\end{program}
Note that if no |into| variable is supplied, we use
|iterate::*result-var*|, which contains the internal variable into
which all clauses place their results.  If this variable is bound by
some clause, then \iter\ will return its value automatically;
otherwise, \nil\ will be returned.  

\subsection{Writing Drivers}

In principle, drivers can be implemented just as easily as other
\iter\ clauses.  In practice, they are a little harder to get right. 
As an example, consider writing a driver that
iterates over all the 
elements of a vector, ignoring its fill-pointer.  |for\dots in-vector|
won't work for this, because it observes the fill-pointer.  It's necessary to
use |array-dimension| instead of |length| to obtain the size of the
vector.  Here is one approach:

\begin{program}
(defmacro-clause (FOR var IN-WHOLE-VECTOR v)
  "All the elements of a vector (disregards fill-pointer)"
  (let ((vect (gensym))
        (index (gensym)))
    `(progn
       (with ,vect = ,v)
       (for ,index from 0 below (array-dimension ,vect 0))
       (for ,var = (aref ,vect ,index)))))
\end{program}
Note that we immediately put |v| in a variable, in case it is an
expression.  Again, this is just good Lisp macrology.  It also has a
subtle effect on the semantics of the driver: |v| is evaluated only
once, at the beginning of the loop, so changes to |v| in the loop have
no effect on the driver.  Similarly, the bounds for numerical iteration
e.g. the above |array-dimension| are also evaluated once only.  This is how
all of \iter's drivers work.


There is an important point concerning the |progn| in this code.  We
need the |progn|, of course, because we are returning several forms,
one of which is a driver.  But \iter\ drivers must occur at top-level.
Is this code in error?  No, because ~top-level~ is defined in \iter\
to include forms inside a |progn|.  This is just the definition of
top-level that Common Lisp uses, and for the same reason: to allow
macros to return multiple forms at top-level.

While our |for\dots in-whole-vector| clause will work, it is not
ideal.  In particular, it does not support generating.  Do do so, we
need to use |for\dots next| or |for\dots do-next|.  The job is
simplified by the |defmacro-driver| macro.

\begin{clauses}

\defmacro{defmacro-driver}{arglist |\&body| body}
Defines a driver clause in
both the |for| and |generate| forms, and provides a parameter
|generate| which ~body~ can examine to determine how it was invoked.
~arglist~ is as in |defmacro-clause|, and should begin with the symbol
|for|. 

\end{clauses}

With |defmacro-driver|, our driver looks like this:
\begin{program}
(defmacro-driver (FOR var IN-WHOLE-VECTOR v)
  "All the elements of a vector (disregards fill-pointer)"
   (let ((vect (gensym))
         (end (gensym))
         (index (gensym))
         (kwd (if generate 'generate 'for)))
     `(progn
        (with ,vect = ,v)
        (with ,end = (array-dimension ,vect 0))
        (with ,index = -1)
        (,kwd ,var next (progn (incf ,index)
                               (if (>= ,index ,end) (terminate))
                               (aref ,vect ,index))))))
\end{program}


We are still missing one thing: the |\&sequence| keywords.  
We can get them easily enough, by writing 
\begin{program}
(defmacro-driver (FOR var IN-WHOLE-VECTOR v \&sequence) 
  ...)
\end{program}
We can now refer to parameters |from|, |to|, |by|, etc. which contain
either the values for the corresponding keyword, or \nil\ if the
keyword was not supplied.  Implementing the right code for these
keywords is cumbersome but not difficult; it is left as an exercise.
But before you begin, see |defclause-sequence| below for an easier way.
       
\subsection{Extensibility Aids}
\label{aids}

This section documents assorted features that may be of use in
extending \iter.

\begin{clauses}

\defunexpvar{*result-var*}
Holds the variable that is used to return a value as a result of the
\iter\ form.  You may examine this and use it in a |with| clause, but
you should not change it.

\defmacro{defsynonym}{syn word}
 Makes ~syn~ a synonym for the existing \iter\ keyword ~word.~  Only
the first word in each clause can have synonyms. 

\defmacroxx{defclause-sequence}{element-name index-name}{|\&key|
access-fn size-fn sequence-type}{element-type
element-doc-string index-doc-string}  
Provides
a simple way to define sequence clauses.  Generates two
clauses, one for iterating over the sequence's elements, the other
for iterating over its indices.  The first symbol of both
clauses will have print-name |for|.
~element-name~  and ~index-name~ should be symbols.
~element-name~ is the second keyword of the element iterator (typically of
the form 
|in-~sequence-type~|), and ~index-name~ is the second keyword
of the index-iterator (typically of the form
|index-of-~sequence-type~|).  Either name may be 
\nil, in which case the corresponding clause is not defined.  If both
symbols are supplied, they should be in the same package.  The |for|
that begins the clauses will be in this package.

\cpar ~access-fn~ is the function to be used to
access elements of the sequence in the element iterator.  The function
should take two 
arguments, a sequence and an index, and return the appropriate element.
~size-fn~ should denote a function of one argument, a sequence, that
returns its size.  Both ~access-fn~ and ~size-fn~ are required for the
element iterator, but only ~size-fn~ is needed for the index iterator.

\cpar The ~sequence-type~ and ~element-type~ keywords are used to
suggest types for the variables 
used to hold the sequence and the
sequence elements, respectively.  The usual rules about \iter's
treatment of variable type declarations apply (see section \ref{types}).

\cpar ~element-doc-string~ and ~index-doc-string~ are
the documentation strings, for use with |display-iterate-clauses|. 

\cpar The generated element-iterator performs destructuring on the
element variable.

\cpar As an example, the above |for\dots in-whole-vector| example
could have been written:
\begin{program}
(defclause-sequence IN-WHOLE-VECTOR INDEX-OF-WHOLE-VECTOR
  :access-fn 'aref
  :size-fn \#'(lambda (v) (array-dimension v 0))
  :sequence-type 'vector
  :element-type t
  :element-doc-string 
     "Elements of a vector, disregarding fill-pointer"
  :index-doc-string 
     "Indices of vector, disregarding fill-pointer")
\end{program}

\end{clauses}

\subsection{Subtleties}

There are some subtleties to be aware of when writing \iter\ clauses.
First, the code returned by your macros may be |nconc|'ed into a list,
so you should always returned freshly consed lists, rather than
constants.  Second, \iter\ matches clauses by using |eq| on the first
symbol and |string=| on the subsequent ones, so the package of the
first symbol of a clause is relevant.  All of the clauses in this manual
have their first word in the \iter\ package.  
You can use the package system in the usual way to shadow
\iter\ clauses without replacing them.

%%% say more here, about the badness that only the first word of a
%%% clause is packagey.

\section{Non-portable Extensions to Iterate (Contribs)}
\label{contribs}

Currently, there is only one non-portable extension to iterate in the
distribution: iterate-pg. If you have made an extension that depends
on non-portable features, feel free to send them to |asf@boinkor.net|
for inclusion in the iterate distribution.

\subsection{An SQL query driver for iterate}

The pg package by Eric Marsden (see |http://cliki.net/pg|) provides an
interface to the PostgreSQL database. Using the \iterpg\ extension, it
is possible to handle the results of SQL queries using \iter.

This usage example should give you an idea of how to use it:

\begin{program}
(pg:with-pg-connection (c "somedb" "someuser")
  (iter (for (impl version date) in-relation "select * from version"
                                 on-connection *dbconn*)
        (collect version)))
\end{program}

To use the extension via |ASDF|, simply make your system depend on the
|iterate-pg| system instead of the |iterate| system. To load it
manually, use:

\begin{program}
  (asdf:oos 'asdf:load-op :iterate-pg)
\end{program}

%%%%%%%%%%%%%%%%%%%%%%%%%%%%%%%%%%%%%%%%%%%%%%%%%%%%%%%%%%%%%%%%
\section{Obtaining {\tt Iterate}}

\begin{sloppypar}
The information in this chapter is obsolete but included for
completeness's sake; Currently, the most up-to-date information on
\iter\ can be found at |http://boinkor.net/iterate.html|.
\end{sloppypar}

\begin{sloppypar}
\iter\ currently runs on Lisp Machines, and on
HP's, Sun3's and Sparcstations under Lucid.  
\iter\ source and binaries are available at the MIT AI Lab in the
subdirectories of |/src/local/lisplib/|.  The source file,
|iterate.lisp|, is also available for anonymous FTP in the directory
|/com/fpt/pub/| on the machine |TRIX.AI.MIT.EDU| (Internet number
128.52.32.6).  If you are unable to obtain |iterate| in one of these
ways, send mail to |jba@ai.mit.edu| and I will send you the source
file. 
\end{sloppypar}

\begin{sloppypar}
\iter\ resides in the |iterate| package (nickname |iter|).  Just say
\linebreak |(use-package :iterate)| to make all the necessary symbols
available. 
If a symbol is not exported, it appears in this manual with an
``|iterate::|'' prefix.
\end{sloppypar}

Send bug reports to |bug-iterate@ai.mit.edu|.  The |info-iterate|
mailing list will have notices of changes and problems; to have
yourself added, send mail to |info-iterate-request@ai.mit.edu|.


\medskip

\begin{flushleft}
 \bf Acknowledgements
\end{flushleft}
\smallskip

Richard Waters provided invaluable criticism which spurred me to improve
\iter\ greatly.  As early users, David Clemens, Oren Etzioni and Jeff
Siskind helped ferret out many bugs.

%\begin{theindex}
% The index files must be generated with the genindex program
% or from makeindex iterate-manual.idx which creates iterate-manual.ind.

%\baselineskip\normalbaselineskip
%\advance\baselineskip by -2pt

\input{iterate-manual.ind}

%\underline{Clauses}
%\input{iterate-manual.clindex}
%
%\indexspace
%\underline{Lisp}
%\input{iterate-manual.lispindex}

%\end{theindex}


\end{document}


% arch-tag: "c0f871a6-313c-11d8-abb9-000c76244c24"


%\underline{Clauses}
%% Fix ``.'' spacing in lists
\documentstyle[12pt,aimemo]{article}

\setlength{\textwidth}{6.0in}
\setlength{\oddsidemargin}{0.25in}
\setlength{\evensidemargin}{0.25in}
\setlength{\topmargin}{0pt}
\setlength{\headsep}{24pt}
\setlength{\textheight}{8.5in}

\let\origpar=\par

% somehow, LaTex fucks with things so that the plain Tex \obeyspaces
% doesn't work.  This is the same, except we use an mbox containing
% a space instead of just a space.

\newcommand{\spc}{\mbox{ }}
\newcommand{\tAb}{\mbox{        }}
\def\obspaces{\catcode`\ =\active\catcode`\^^I=\active}
{\obspaces \global\let =\spc\global\let^^I=\tAb}

\newenvironment{program}{\medskip\samepage\tt\obeylines\obspaces}{\medskip}

% use | for \tt
\catcode`\|=13
\newif\ifvbar
\def|{\ifvbar
         \endgroup\vbarfalse
      \else 
         \begingroup\tt\vbartrue 
      \fi}



\newcommand{\lisp}{\tt}
\newcommand{\iter}{|iterate|}
\newcommand{\iterpg}{|iterate-pg|}
\newcommand{\setf}{|setf|}
\newcommand{\nil}{|nil|}
\newcommand{\nonnil}{non-\nil}
\newcommand{\opt}{{\lisp \&optional}}
\newcommand{\yields}{$\Rightarrow$}



\newbox\Dots
\setbox\Dots=\hbox{\small ...}
\renewcommand{\dots}{\copy\Dots}

% Here's a way to define an environment which will treat all
% paragraphs inside it in the same manner.  It's, as near as I can
% figure, the way that Latex does things.  Using \everypar won't work;
% apparently Latex resets that a lot.  What you have to do is: at the
% top of the file do \let\origpar=\par.  In the
% begin part of the environment, do the parshape (or whatever), then say
% \def\par{{\origpar}}.  The extra braces are necessary.   That's all
% you have to do; at the end of the environment things will return to
% normal automatically.  Why does this all work?  I have no idea.
% Another important thing: there should be a blank line (i.e. a \par)
% between the end of the last paragraph and the \end{environment}.

\newlength{\clindent}
\setlength{\clindent}{0.5in}
\newlength{\clparindent}
\setlength{\clparindent}{\parindent}

\newenvironment{clauses}{\advance\linewidth -\parindent
                        \hangindent\clindent\relax
                         \hangafter=0
                          \parindent=0pt
                          \def\par{{\origpar}}}{}

\newcommand{\cpar}{\hspace \clparindent}

\newcommand{\startitem}{\bigskip\pagebreak[3]}
\newcommand{\finishitem}{\smallskip}

\newcommand{\clause}[1]{\startitem\Clause{#1}\finishitem}
\newcommand{\Clause}[1]{\hbox{#1}}
\newcommand{\clindex}[1]{\index{{\lisp #1}}}
\newcommand{\clindexx}[2]{\index{{\lisp #1\dots #2}}}
\newcommand{\clindexxx}[3]{\index{{\lisp #1\dots #2\dots #3}}}
\newcommand{\clausex}[2]{\startitem\Clausex{#1}{#2}\finishitem}
\newcommand{\Clausex}[2]{\Clause{|#1| #2}\clindex{#1}}

\newcommand{\defvar}[1]{\startitem\deff{#1}{}{Variable}\finishitem}
\newcommand{\defunexpvar}[1]{\startitem\deffnoindex{iterate::#1}{}{Variable}
                                        \lispindex{#1}\finishitem}
\newcommand{\defun}[2]{\startitem\deff{#1}{#2}{Function}\finishitem}
\newcommand{\defmacro}[2]{\startitem\deff{#1}{#2}{Macro}\finishitem}
\newcommand{\defunx}[3]{\startitem\deffx{#1}{#2}{#3}{Function}\finishitem}
\newcommand{\defmacrox}[3]{\startitem\deffx{#1}{#2}{#3}{Macro}\finishitem}
\newcommand{\defunxx}[4]{\startitem\deffxx{#1}{#2}{#3}{#4}{Function}\finishitem}
\newcommand{\defmacroxx}[4]{\startitem\deffxx{#1}{#2}{#3}{#4}{Macro}\finishitem}

\newcommand{\deff}[3]{\deffnoindex{#1}{#2}{#3}\lispindex{#1}}
\newcommand{\deffnoindex}[3]{\hbox to \hsize{{\tt #1} {\it #2}\hfill [{\it #3}]}}
\newcommand{\deffx}[4]{\deff{#1}{#2}{#4}\moreargs{#1}{#3}}
\newcommand{\deffxx}[5]{\deffx{#1}{#2}{#3}{#5}\moreargs{#1}{#4}}

\newcommand{\moreargs}[2]{\hbox to \hsize{\phantom{\lisp #1} {\it #2}\hfill}}

\newcommand{\lispindex}[1]{\index{{\lisp #1}}}

 \newenvironment{note}[1]{\pagebreak[2]\bigskip
                          \hrule\smallskip\small
                         {\setlength{\parindent}{0pt}
                           \par{\bf #1:}}}{\smallskip\hrule\bigskip}



\makeindex

%%%%%%%%%%%%%%%%%%%%%%%%%%%%%%%%%%%%%%%%%%%%%%%%%%%%%%%%%%%%%%%%

%\thispagestyle{empty}

%\vspace*{2in}

%\title{The |Iterate| Manual}
%\author{Jonathan Amsterdam}

%\maketitle

%\thispagestyle{empty}

%\begin{center}
%\Large *** DRAFT ***
%\end{center}
%This manual is in beta-test. It will become an AI Memo in a couple of
%months.  I would appreciate any comments or
%corrections.  You can reach me at email address |jba@ai.ai.mit.edu|.

\begin{document}

\def\writtenby{Jonathan Amsterdam}
\def\contractno{This report describes research done at the Artificial
Intelligence Laboratory of the Massachusetts Institute of Technology.
Support for the laboratory's artificial intelligence research is
provided in part by the Advanced Research Projects Agency of the
Department of Defense under Office
of Naval Research contract N00014-85-K-0124.}

\AIMEMO{1236}{\Large The Iterate Manual \\[1in]}
{This is the manual for version 1.4 of \iter, a powerful iteration
macro for Common  
Lisp. \iter\ is similar to {\tt loop} but provides numerous additional
features, is well integrated with Lisp, and is extensible. 
\vfill }


% use ~ for \em
\catcode`\~=13
\newif\ifem
\def~{\ifem
         \/\endgroup\emfalse
      \else 
         \begingroup\em\emtrue 
      \fi}



%\pagebreak
%\setcounter{page}{1}

%\thispagestyle{empty}


\tableofcontents

\pagebreak
\pagestyle{headings}

\section{Introduction}

\begin{sloppypar}
This manual describes
\iter, a powerful iteration facility for Common Lisp.
\iter\ provides abstractions for many common iteration
patterns and allows for the definition of additional patterns.  \iter\ is
a macro that expands into ordinary Lisp at
compile-time, so it is more efficient than higher-order
functions like |map| and |reduce.|  
While it is similar to |loop|, \iter\ offers a more Lisp-like syntax
and enhanced extensibility.
(For a more complete
comparison of \iter\ with other iteration constructs, see MIT AI Lab Working
Paper No. 324, ~Don't Loop, Iterate.~)
\end{sloppypar}

An \iter\ form consists of the symbol |iter|\footnote{You can also use
|iterate|, but |iter| is preferred because it avoids potential conflicts with
possible future additions to Common Lisp, and because it saves
horizontal space when writing code.}
followed by one or more
forms, some of which may be \iter\ 
~clauses.~  Here is a simple example of \iter\ which collects the
numbers from 1 to 10 into a list, and returns the list.  The return
value is shown following the arrow.
\begin{program}
(iter (for i from 1 to 10)
      (collect i))          \yields (1 2 3 4 5 6 7 8 9 10)
\end{program}
This form contains two clauses: a |for| clause that steps the
variable |i| over the integers from 1 to 10, and a |collect|
clause that accumulates its argument into a list.  With a few
exceptions, all \iter\ clauses have the same format: alternating
symbols (called ~keywords~) and expressions (called
~arguments~).  The syntax and terminology are those of Common
Lisp's keyword lambda lists.  One difference is that \iter's keywords
do not have to begin with a colon---though they may, except for the
first symbol of a clause.  So you can also write  |(for i :from
1 :to 10)| if you prefer.

Any Lisp form can appear in the body of an \iter, where it will have
its usual meaning.  \iter\ walks the entire body, expanding macros,
and recognizing clauses at any level.  This example collects all the
odd numbers in a list:
\begin{program}
(iter (for el in list)
      (if (and (numberp el) (oddp el))
          (collect el)))
\end{program}

There are clauses for iterating over numbers, lists, arrays and other
objects, and for
collecting, summing, counting, maximizing and other useful operations.
\iter\ also supports the creation of new variable bindings, stepping
over multiple sequences at once, destructuring, and compiler
declarations of variable types.  The following example illustrates
some of these features: 
\begin{program}
(iter (for (key . item) in alist)
         (for i from 0)
         (declare (fixnum i))
         (collect (cons i key)))
\end{program}

This loop takes the keys of an alist and returns a new alist
associating the keys with their positions in the original list. The
compiler declaration for |i| will appear in the generated code
in the appropriate place.

%%%%%%%%%%%%%%%%%%%%%%%%%%%%%%%%%%%%%%%%%%%%%%%%%%%%%%%%%%%%%%%%
\section{Clauses}

Most of \iter's clauses will be familiar to |loop| programmers.  
(|loop| is an iteration macro that has been incorporated into Common
Lisp.  Seen Guy Steele's {\em Common Lisp, 2nd Edition}.)
In
nearly all cases they behave the same as their |loop| counterparts, so
a |loop| user can switch to \iter\ with little pain (and much gain).

All clauses with the standard keyword-argument syntax consist of two
parts: a ~required~ part, 
containing keywords that must be present and in the right order; and
an ~optional~ part, containing keywords that may be omitted and,
if present, may occur in any order.  In the descriptions below, the
parts are separated by the Lisp lambda-list keyword \opt.


\subsection{Drivers}

An iteration-driving clause
conceptually causes the iteration to go forward.  Driver clauses in
\iter\ allow iteration over numbers, lists, vectors, hashtables, packages,
files and streams.  Iteration-driving clauses must
appear at the top level of an \iter\ form; they cannot be nested
inside another clause.  The driver variable is updated at the point
where the driver clause occurs.  Before the clause is executed for the
first time, the value of the variable is undefined.

%Also, regardless of where the driver clause
%appears in the body, the driver variable is stepped at the top of the
%loop; hence it is stylistically preferable, though
%not required, to place driver clauses at the beginning of the \iter.

Multiple drivers may appear in a single \iter\ form, in which case all
of the driver variables are updated each time through the loop, in the
order in which the clauses appear.  The first driver to terminate will
terminate the entire loop.

In all cases, the value of the driver variable on exit from the loop,
including within the epilogue code (see the |finally| clause), is
undefined.

All the parameters of a driver clause are evaluated once, before the
loop begins.  Hence it is not possible to change the bounds or other
properties of an iteration by side-effect from within the loop.

With one exception, driver clauses begin with the word |for| (or
the synonym |as|) and mention an iteration variable, which is
given a binding within the \iter\ form.
The exception is |repeat|, which just executes a loop a
specified number of times:

\begin{clauses}

\clausex{repeat}{~n~}
Repeats the loop ~n~ times.  For example:
\begin{program}
(iter (repeat 100)
      (print "I will not talk in class."))
\end{program}
\cpar If $n \leq 0$, the
loop will never be executed.  If ~n~ is not an integer, the
actual number of executions will be $\lceil n \rceil$.

\end{clauses}

\pagebreak[3]

\subsubsection{Numerical Iteration}

\begin{clauses}
\clausex{for}{~var~ |\&sequence|} 
The general form for iterating over a sequence of numbers requires a
variable and, optionally, one or more keywords that provide the bounds
and step size of the iteration. The |\&sequence|
lambda-list keyword is a shorthand for these sequence keywords.  They are:
|from|, |upfrom|, |downfrom|, |to|, |downto|, |above|, |below| and |by|.
|from| provides the starting value for ~var~
and defaults to zero. |to| provides a final value and implies that the
successive values of ~var~ will be increasing; |downto| implies that
they will be decreasing.  The loop terminates when ~var~ passes
the final value (i.e. becomes smaller or larger than it, depending on
the direction of iteration); in other  words, the loop body
will never be 
executed for values of ~var~ past the final value.
|below| and |above| are similar to |to| and
|downto|, except that the loop terminates when ~var~ equals or passes
the final value.

\cpar If no final value is specified, the variable will be stepped
forever.  Using |from| or |upfrom| will result in increasing
values, while |downfrom| will give decreasing values.

\cpar On each iteration,
~var~ is incremented or decremented by the value of the sequence
keyword |by|, which defaults to 1.  It should always be a positive
number, even for downward iterations.

\cpar In the following examples, the sequence of numbers generated is shown
next to the clause.

\begin{program}
(for i upfrom 0) \yields\ 0 1 2 \ldots
(for i from 5)   \yields\ 5 6 7 \ldots    ; either from or upfrom is okay
(for i downfrom 0) \yields\ 0 -1 -2 \ldots
(for i from 1 to 3) \yields\ 1 2 3
(for i from 1 below 3) \yields\ 1 2
(for i from 1 to 3 by 2) \yields\ 1 3
(for i from 1 below 3 by 2) \yields\ 1
(for i from 5 downto 3) \yields\ 5 4 3
\end{program}

\end{clauses}

\subsubsection{Sequence Iteration}

There are a number of clauses for iterating over sequences.  In all of
them, the argument following |for| may be a list instead of a
symbol, in which case destructuring is performed.  See section
\ref{destructuring}. 

\begin{clauses}

\clause{|for| ~var~ |in| ~list~ \opt\ |by| ~step-function~}
\clindexx{for}{in}
~var~ is set to successive elements of list.  
~step-function,~ which 
defaults to |cdr|, is used to obtain the next sublist. 

\clause{|for| ~var~ |on| ~list~ \opt\ |by| ~step-function~}
\clindexx{for}{on}
~var~ is set to successive sublists of list. 
~step-function~ (default |cdr|) is used as in |for\dots in|.

\end{clauses}

\medskip

These two clauses use |atom| to test for the end
of a list.  Hence, given a list whose final |cdr| is not \nil,
they will silently ignore the last |cdr|.  Other choices are
|endp|,  which would signal an error, and |null|, which
would probably result in an error somewhere else.  If you wish to use
an end-test other than |atom|, set the variable
|iterate::*list-end-test*|\lispindex{*list-end-test*} to the name of
the desired function. 

\begin{clauses}
\clause{|for| ~var~ |in-vector| ~vector~ |\&sequence|}
\clindexx{for}{in-vector}
~var~ takes on successive elements from ~vector.~   The vector's
fill-pointer is observed.  Here and in subsequent clauses, the |\&sequence|
keywords include |with-index|, which takes a symbol as argument
and uses it for the index variable instead of an internally generated
symbol.  The other |\&sequence| keywords behave as in numerical
iteration, except that the default iteration bounds are the bounds of
the vector.  E.g. in |(for i in-vector v downto 3)|,
|i| will start off being bound to the last element in |v|,
and will  
be set to preceding elements down to and including the element with
index 3.

\clause{|for| ~var~ |in-sequence| ~seq~ |\&sequence|}
\clindexx{for}{in-sequence}
This uses Common Lisp's generalized sequence functions, |elt|
and |length|, to obtain elements and determine the length of
~seq.~   Hence it will work for any sequence, including lists, and will
observe the fill-pointers of vectors.

\clause{|for| ~var~ |in-string| ~string~ |\&sequence|}
\clindexx{for}{in-string}
~var~ is set to successive characters of ~string.~

\startitem
\Clause{|for| ~var~ |index-of-vector| ~vector~ |\&sequence|}
\clindexx{for}{index-of-vector}
\Clause{|for| ~var~ |index-of-sequence| ~sequence~ |\&sequence|}
\clindexx{for}{index-of-sequence}
\Clause{|for| ~var~ |index-of-string| ~string~ |\&sequence|}
\clindexx{for}{index-of-string}
\finishitem
~var~ is set to successive indices of the sequence.
These clauses avoid the overhead of accessing the sequence elements
for those applications where they do not need to be examined, or are
examined rarely.  They admit all the optional keywords of the other
sequence drivers except the (redundant) |with-index| keyword.

\clause{|for| ~(key value)~ |in-hashtable| ~table~}
\clindexx{for}{in-hashtable}

~key~ and ~value,~ which must appear as shown in a list
and may be destructuring templates, are set to the keys and values of
~table~.  If ~key~ is |nil|, then the hashtable's keys will be ignored;
similarly for ~value~.
The order in which 
elements of ~table~ will be retrieved is unpredictable.

\clause{|for| ~var~ |in-package| ~package~ \opt\ |external-only| ~ext~}
\clindexx{for}{in-package}
Iterates over all the symbols in ~package,~ or over only the
external symbols if ~ext~ is specified and non-|nil|.  ~ext~ is not
evaluated.  The same symbol may appear more than once.

\clause{|for| ~var~ |in-packages| ~packages~ \opt\ |having-access| ~symbol-types~}
\clindexx{for}{in-packages}
Iterates over all the symbols from the list of packages denoted by the
descriptor ~packages~ and having accessibility (or visibility) given by
~symbol-types~.  This defaults to the list |(:external :internal :inherited)|
and is not evaluated.  ~var~ must be a list of up to three variables: in each
iteration, these will be set to a symbol, its access-type and package (as per
|with-package-iterator| in ANSI-CL).  The same symbol may appear more than
once.

\end{clauses}

\begin{clauses}

\clause{|for| ~var~ |in-file| ~name~ \opt\ |using| ~reader~}
\clindexx{for}{in-file}
Opens the file ~name~ (which may be a string or pathname) for input,
and iterates 
over its contents. ~reader~ defaults to |read|, so by default ~var~
will be bound to the successive forms in the file.  The \iter\ body is
wrapped in an unwind-protect to ensure that the file is closed no
matter how the \iter\ is exited.

\clause{|for| ~var~ |in-stream| ~stream~ \opt\ |using| ~reader~}
\clindexx{for}{in-stream}
Like |for\dots in-file|, except that ~stream~ should be an existing
stream object that supports input operations.

\end{clauses}

\subsubsection{Generalized Drivers}

These are primarily useful for writing drivers that can also be used
as generators (see section \ref{generators}, below).
 
\begin{clauses}

\clause{|for| ~var~ |next| ~expr~}
\clindexx{for}{next}

~var~ is set to ~expr~ each time through the loop.  Destructuring is
performed.  When the clause is used as a generator, ~expr~ is the code
that is executed when |(next ~var~)| is 
encountered (see section \ref{generators}, below).
~expr~ should compute the first value for ~var~, as well
as all subsequent values, and is responsible for terminating the loop.
For compatibility with future versions of \iter, this termination
should be done with |terminate|\clindex{terminate}, which can be
considered a synonym for |finish| (see section \ref{control-flow}).

\cpar As an example, the following clauses are equivalent to |(for
i from 1 to 10)|:
\begin{program}
(initially (setq i 0))
(for i next (if (> i 10) (terminate) (incf i)))
\end{program}

\clause{|for| ~var~ |do-next| ~form~}
\clindexx{for}{do-next}
~form~ is evaluated each time through the loop.  Its value is ~not~
set to ~var~; that is ~form's~ job. ~var~ is only present so that
\iter\ knows it is a driver variable.  \linebreak |(for ~var~ next
~expr~)| is equivalent to |(for ~var~ do-next (dsetq ~var~ ~expr~))|.
(See section \ref{destructuring} for an explanation of |dsetq|.) 

\end{clauses}

\subsubsection{Generators}
\label{generators}

In all of the above clauses, the driver variable is updated on each
iteration.
Sometimes it is desirable to have greater control over updating.  
For instance, consider the problem of associating numbers, in
increasing order and with no gaps, with the
\nonnil\ elements of a list.  One obvious first pass at writing this is:

\begin{program}
(iter (for el in list)
      (for i upfrom 1)
      (if el (collect (cons el i))))
\end{program}
But on the list |(a b nil c)| this produces |((a . 1) (b . 2) (c .
4))| instead of the desired |((a . 1) (b . 2) (c . 3))|.  The problem
is that |i| is incremented each time through the loop, even when |el|
is |nil|.  

The problem could be solved elegantly if we could step |i| only when
we wished to.  This
can be accomplished for any \iter\ driver by writing |generate| (or
its synonym |generating|)
instead of |for|.  Doing so produces a ~generator~---a driver whose
values are yielded explicitly.  To obtain the next value of a
generator variable ~v~, write \linebreak |(next ~v~)|.  The value of a |next|
form is the next value of ~v~, as determined by its associated driver
clause.  |next| also has the side-effect of updating ~v~ to that
value.  If there is no next value, |next| will terminate the loop,
just as with a normal driver.


Using generators, we can now write our example like this:
\begin{program}
(iter (for el in list)
      (generate i upfrom 1)
      (if el (collect (cons el (next i)))))
\end{program}
Now |i| is updated only when |(next i)| is executed, and this occurs
only when |el| is \nonnil.

To better understand the relationship between ordinary drivers and
generators, observe that we can rewrite an ordinary driver using its
generator form immediately followed by |next|, as this example shows:
\begin{program}
(iter (generating i from 1 to 10)
      (next i)
      ...)
\end{program}
Provided that the loop body contains no |(next i)| forms, this will
behave just as if we had written |(for i from 1 to 10)|.

We can still refer to a driver variable ~v~ without using |next|; in
this case, its value is that given to it by the last evaluation of
|(next ~v~)|.  Before |(next ~v~)| has been called the first time, the
value of ~v~ is undefined.  

This semantics is more flexible than
one in which ~v~ begins the loop bound to its first value and calls
of |next| supply subsequent values, because it means the loop will not
terminate too soon if the generator's sequence is empty.  For
instance, consider the following code, which tags \nonnil\ elements of
a list using a list of tags, and also counts the null elements.
(We assume there are at least as many tags as \nonnil\ elements.)
\begin{program}
(let* ((counter 0)
       (tagged-list (iter (for el in list)
                          (generating tag in tag-list) 
                          (if (null el)
                              (incf counter)
                              (collect (cons el (next tag)))))))
  ...)
\end{program}
It may be that there are just as many tags as non-null elements of
|list|.  If all the elements of |list| are null, we still want the
counting to proceed, even though |tag-list| is \nil.  If |tag| had to be
assigned its first value before the loop begins, we would have had to
terminate the loop before the first iteration, since when |tag-list|
is \nil, |tag| 
has no first value.  With the existing semantics, however, |(next
tag)| will never execute, so the iteration will cover all the elements of
|list|. 
        
When the ``variable'' of a driver clause is actually a destructuring
template containing several variables, all the variables are eligible
for use with |next|.  As before, |(next ~v~)| evaluates to ~v's~ next
value; but the effect is to update all of the template's variables.
For instance, the following code will return the list |(a 2 c)|.
\begin{program}
(iter (generating (key . item) in '((a . 1) (b . 2) (c . 3)))
      (collect (next key))
      (collect (next item)))
\end{program}

Only driver clauses with variables can be made into generators.  This
includes all clauses mentioned so far except for |repeat|.  It does
~not~ include |for\dots previous|, |for\dots =|,
|for\dots initially\dots then| or |for\dots first\dots then| (see
below). 

\subsubsection{Previous Values of Driver Variables}

Often one would like to access the value of a variable on a previous
iteration.  \iter\ provides a special clause for accomplishing this.

\begin{clauses}

\clause{|for| ~pvar~ |previous| ~var~ \opt\ |initially| ~init~
|back| ~n~}
\clindexx{for}{previous}
Sets ~pvar~ to the previous value of ~var,~ which should be a driver
variable, a variable from another |for\dots previous| clause, or a
variable established by a |for\dots =|, 
|for\dots initially\dots then| or |for\dots first\dots then| clause
(see section \ref{setting}).
Initially, ~pvar~ is given the value ~init~ (which defaults to \nil).  
The ~init~ expression will be moved outside the loop body, so it
should not depend on anything computed within the loop.
~pvar~ retains the value of ~init~ until ~var~ is set to its second
value, at which point ~pvar~ is set to ~var's~ first value; and so on.  

\cpar The
argument ~n~ to |back| must be a constant, positive integer, and
defaults to 1.  It determines how many iterations back ~pvar~ should
track ~var~.  For example, when ~n~ is 2, then ~pvar~ will be assigned
~var's~ first value when ~var~ is set to its third value.

\cpar A |for\dots previous| clause may occur after or before its
associated driver clause. |for\dots previous| works with generators as
well as ordinary drivers.

\pagebreak[3]

\cpar Example:
\begin{program}
(iter (for el in '(1 2 3 4))
      (for p-el previous el)
      (for pp-el previous p-el initially 0)
      (collect pp-el))
\end{program}
This evaluates to |(0 0 1 2)|.  It could have been written more
economically as 
\begin{program}
(iter (for el in '(1 2 3 4))
      (for pp-el previous el back 2 initially 0)
      (collect pp-el))
\end{program}


\end{clauses}

\subsection{Variable Binding and Setting}
\label{setting}

Several clauses exist for establishing new variable bindings or for
setting variables in the loop.  They all support destructuring.

\begin{clauses}

\clausex{with}{~var~ \opt\ |=| ~value~}
Causes ~var~ to be bound to value before the loop body is entered.
If ~value~ is not supplied, ~var~ assumes a default
binding, which will be 
\nil\ in the absence of declarations.  Also, if ~value~ is not
supplied, no destructuring is performed; instead, ~var~ may be a list of
symbols, all of which are given default 
bindings.  If ~value~ is supplied, ~var~ is bound to it, with
destructuring.  

\cpar Because |with|
creates bindings whose scope includes the entire \iter\ form, it is
good style to put all |with| clauses at the beginning.

\cpar Successive occurrences of |with| result in sequential
bindings (as with 
|let*|).  There is no way to obtain parallel bindings; see
section \ref{bindings} for a rationale.  


\clause{|for| ~var~ |=| ~expr~}
\clindexx{for}{=}
On each iteration, ~expr~ is evaluated and ~var~ is set
to its value. 

\cpar This clause may appear to do the same thing as |for\dots next|.
In fact, they are quite different.  |for\dots =| provides only three
services: it sets up a binding for ~var~, sets it to ~expr~ on each
iteration, and makes it possible to use |for\dots previous| with
~var~.  |for\dots next| provides these services in addition to the
ability to turn the driver into a generator.
%Also, the code which sets ~var~ appears in the loop body in the same
%place as the |for\dots =| clause; the code for |for\dots next| appears
%at the top of the loop, as with other drivers (except when being used
%as a generator).

\clause{|for| ~var~ |initially| ~init-expr~ |then| ~then-expr~}
\clindexxx{for}{initially}{then}
Before the loop begins, ~var~ is set to ~init-expr;~ on all
iterations after the first it is set to ~then-expr.~  This clause
must occur at top-level.  ~init-expr~ will be moved outside the loop
body and ~then-expr~ will be moved to the end of the loop body, so
they are subject to code motion problems (see section
\ref{code-movement}). 

\cpar This clause may appear to be similar to |for\dots next|, but in
fact they differ significantly.  |for\dots initially\dots then| is
typically used to give ~var~ its first value before the loop begins,
and subsequent values on following iterations.  This is incompatible
with generators, whose first value and subsequent values must all be
computed by |(next ~var~)|.  Also, the update of ~var~ in 
|for\dots initially\dots then| does not occur at the location of the
clause. 
Use |for\dots initially\dots then| for
one-shot computations where its idiom is more convenient, but use
|for\dots next| for extending \iter\ with new drivers (see section
\ref{extend}).

\clause{|for| ~var~ |first| ~first-expr~ |then| ~then-expr~}
\clindexxx{for}{first}{then}
The first time through the loop, ~var~ is set to ~first-expr;~ on
subsequent iterations, it is set to ~then-expr.~  This differs from
|for\dots initially| in that ~var~ is set to ~first-expr~
inside the loop body, 
so ~first-expr~ may depend on the results of other clauses.  For
instance, 
\begin{program}
(iter (for num in list)
      (for i first num then (1+ i))
      ...)
\end{program}
will set |i| to the first element of |list| on the first
iteration, whereas 
\begin{program}
(iter (for num in list)
      (for i initially num then (1+ i))
      ...)
\end{program}
is probably erroneous; |i| will be bound to |num|'s
default binding (usually \nil) for the first iteration.

\end{clauses}

\begin{note}{Compatibility note}
|loop|'s |for\dots =| works like \iter's, but |loop| used the syntax
|for\dots =\dots then| to mean |for\dots initially\dots then|.  It was
felt that these two operations were sufficiently different to warrant
different keywords.

Also, the |for| in the above three clauses is misleading,
since none is true driver (e.g. none has a corresponding |generate|
form).  |setting| would have been a better choice, but |for| was used
to retain some compatibility with |loop|.  
\end{note}


%%%%%%%%%%%%%%%%%%%%%%%%%%%%%%%%
\subsection{Gathering Clauses}
Many of \iter's clauses accumulate values into a variable, or set a
variable under certain conditions.  At the end of the loop, this
variable contains the desired result.  All these clauses have an
optional |into| keyword, whose argument should be a symbol.  If the
|into| keyword is not supplied, the accumulation variable will be
internally generated and its value will be returned at the end of the
loop; if a variable is specified, that variable is used for the
accumulation, and is not returned as a result---it is up to the user
to return it explicitly, in the loop's epilogue code (see |finally|).
It is safe to examine the accumulation variable during the loop, but
it should not be modified.

These clauses all begin with a verb.  When the verb does
not conflict with an existing Common Lisp function, then it may be
used in either its infinitival or present-participle form (e.g. |sum|,
|summing|).  However, when there is a conflict with Common Lisp, only
the present-participle form may be used (e.g. |unioning|).  This is to
prevent \iter\ clauses from clashing with Common Lisp
functions. 

%although these clauses are described as ``producing a value,'' it is a
%mistake to think of the lisp list representing the clause as a
%value-producing form in the usual way.  clauses may legally be written
%where a value is expected, e.g. |(setq x (sum i))|, but the
%lisp value of a clause in such a context is undefined.

\subsubsection{Reductions}

~Reduction~ is an extremely common iteration pattern in which
the results of successive applications of a
binary operation are accumulated.
For example, a loop that computes the sum of the
elements of a list is performing a reduction with the addition
operation.  This could be written in Common Lisp as  |(reduce \#'+
list)| or with \iter\ as
\begin{program}
(iter (for el in list)
      (sum el))
\end{program}

\begin{clauses}

\clausex{sum}{~expr~ \opt\ |into| ~var~}
Each time through the loop, ~expr~ is evaluated and added to a
variable, which is bound initially to zero.  If ~expr~ has a type,
it is ~not~ used as the type of the sum variable, which is always
|number|.  To get the result variable to be of a more specific
type, use an explicit variable, as in
\begin{program}
(iter (for el in number-list)
      (sum el into x)
      (declare (fixnum x))
      (finally (return x)))
\end{program}

\clausex{multiply}{~expr~ \opt\ |into| ~var~}
Like |sum|, but the initial value of the result variable is $1$,
and the variable is updated by multiplying ~expr~ into it.

\clausex{counting}{~expr~ \opt\ |into| ~var~}
~expr~ is evaluated on each iteration.  If it is \nonnil, the
accumulation variable, initially zero, is incremented.

\startitem
\Clausex{maximize}{~expr~ \opt\ |into| ~var~}
\Clausex{minimize}{~expr~ \opt\ |into| ~var~}
\finishitem
~expr~ is evaluated on each iteration and its extremum (maximum or
minimum) is stored in the accumulation variable.  If ~expr~ is never
evaluated, then the result is \nil\ (if the accumulation variable is
untyped) or $0$ (if it has a numeric type).

\clausex{reducing}{~expr~ |by| ~func~ \opt\ |initial-value| ~init-val~ |into| ~var~}
This is a general way to perform reductions. ~func~ should be a
function of two arguments, the first of which will be the value
computed so far and
the second of which will be the value of ~expr~.  It should return the
new value.
|reducing| is roughly equivalent to the
Common Lisp |(reduce ~func~ ~sequence~ :key ~expr-function~)|, where
~expr-function~ is used to derive values from the successive elements
of ~sequence~.

\cpar If the |reducing| clause is never executed, the result is
undefined.

\cpar It is not necessary to provide an
initial value, but better code can be generated if one is supplied.
Regardless of its location in the \iter\ body, ~init-val~ will be
evaluated before the loop is entered, so it should not depend on any
value computed inside the \iter\ form.  

%\cpar if a ~var~ is not specified, you can get \iter\ to declare the
%type of the internal variable by putting a |the| expression around
%~func.~  see section \ref{types}.

\end{clauses}

%\begin{note}{implementation note}
%in principle, |maximize| and |minimize| can be thought of
%as reductions where the initial value is the smallest (or largest)
%value that the accumulation variable can assume.  because lisp's
%bignums can represent arbitrary integers, these clauses cannot be
%implemented as reductions in general.  if, however, the type of 
%~expr~ or ~var~ can be determined to be a fixnum
%or a float, \iter\ will implement the clause as a true reduction,
%using one of the constants |most-negative-fixnum|, |%most-positive-fixnum|, 
%|most-negative-short-float|, etc. as appropriate.
%\end{note}

\subsubsection{Accumulations}

All the predefined accumulation clauses add values to a sequence.  If
the sequence is a list, they all 
have the property that the partial list is kept in the correct order
and available for inspection at any point in the loop.

\begin{clauses}

\clausex{collect}{~expr~ \opt\ |into| ~var~ |at| ~place~ |result-type| ~type~}
Produces a sequence of the values of ~expr~ on each iteration. ~place~
indicates where the next value of ~expr~ is added to the list and may
be one of the symbols |start|, |beginning| (a synonym for |start|) or
|end|.  The symbol may be quoted, but need not be.  The default is
|end|.  For example,
\begin{program}
(iter (for i from 1 to 5)
      (collect i))
\end{program}
produces |(1 2 3 4 5)|, whereas
\begin{program}
(iter (for i from 1 to 5)
      (collect i at beginning))
\end{program}
produces |(5 4 3 2 1)| (and is likely to be faster in most Common Lisp
implementations).

\cpar If ~type~ is provided, it should be a subtype of |sequence|.
The default is |list|.  Specifying a type other than |list| will
result in |collect| returning a sequence of that type.  ~However,~ the
type of the sequence being constructed when inside the loop body is
undefined when a non-|list| type is specified.  (As with ~place~,
quoting ~type~ is optional.)

\clausex{adjoining}{~expr~ \opt\ |into| ~var~ |test| ~test~ |at| ~place~ |result-type| ~type~}
Like |collect|, but only adds the value of ~expr~ if it is not
already present.  ~test,~ which defaults to |\#'eql|, is
the test to be used with |member|.

\startitem
\Clausex{appending}{~expr~ \opt\ |into| ~var~ |at| ~place~}
\Clausex{nconcing}{\ ~expr~ \opt\ |into| ~var~ |at| ~place~}
\Clausex{unioning}{\ ~expr~ \opt\ |into| ~var~ |test| ~test~ |at| ~place~}
\Clausex{nunioning}{~expr~ \opt\ |into| ~var~ |test| ~test~ |at| ~place~}
\finishitem
These are like |collect|, but behave like the Common Lisp functions
|append|, |nconc|, |union| or |nunion|.  
As in Common Lisp, they work only on lists.  Also as in Common Lisp,
|unioning| and |nunioning| assume that the value of ~expr~ contains no
duplicates. 

\clausex{accumulate}{~expr~ |by| ~func~ \opt\ |initial-value| ~init-val~ |into| ~var~}
This is a general-purpose accumulation clause. ~func~ should be a
function of two arguments, the value of ~expr~ and the value
accumulated so far in the iteration, and it should return the updated
value.  If no initial value is supplied, \nil\ is used.  

%\cpar If a ~var~ is not specified, you can get \iter\ to declare the
%type of the internal variable by putting a |the| expression around
%~func.~  see section \ref{types}.

\cpar The differences between |accumulate| and |reducing| are slight.
One difference is that the functions take their arguments in a
different order.  Another is that in the absence of ~init-val~,
|accumulate| will use \nil, whereas |reducing| will generate different
code that avoids any dependence on the initial value.
The reason for having both clauses is that one usually
thinks of reductions (like |sum|) and accumulations (like |collect|)
as different beasts.

\end{clauses}


\subsubsection{Finders}

A ~finder~ is a clause whose value is an expression that meets some
condition.

\begin{clauses}

\clause{|finding| ~expr~ |such-that| ~test~ \opt\ |into| ~var~ |on-failure| ~failure-value~}
\clindexx{finding}{such-that}
If ~test~ (which is an expression) ever evaluates to \nonnil, the loop
is terminated, the 
epilogue code is run and the value of ~expr~ is returned.  Otherwise,
\nil\ (or ~failure-value,~ if provided) is returned.  If ~var~ is
provided, it will have either the \nonnil\ value of ~expr~ or
~failure-value~ when the epilogue code is run.

\cpar As a special case, if the ~test~ expression is a sharp-quoted
function, then it is applied to ~expr~ instead of being simply
evaluated.  E.g. |(finding x such-that \#'evenp)| is equivalent to
|(finding x such-that (evenp x))|.  

%\cpar although ~test~ need have nothing to do with ~%expr~ as in
%|(finding j such-that (> i 3))|, it usually 
%will: |(finding (length el) such-that (oddp (length el)))|.  to
%avoid performing the |length| computation twice, you could write
%|(finding (length el) such-that \#'oddp)| or |(finding (length
%el) such-that 'oddp)|; for these cases, \iter\ generates code that
%executes ~expr~ only once.  the code for |\#'oddp|
%is slightly different from that for {\lisp 'oddp}; see the discussion
%under {\lisp for\dots in} and {\lisp for\dots on}.

\cpar |On-failure| is a misnomer. Because it is always evaluated, it behaves
more like the default third argument to the |gethash| function. As a result,
|on-failure (error "Not found")| makes no sense. Instead, the clauses |leave|
or |thereis| can be used in conjunction with |finally| as follows:
\begin{program}
(iter (for x in '(1 2 3))
      (if (evenp x) (leave x))
      (finally (error "not found")))
\end{program}

\cpar This clause may appear multiple times when all defaults are
identical. It can also be used together with either |always|/|never| or
|thereis| if their defaults match. More specifically, |on-failure nil| is
compatible with |thereis|, while |on-failure t| is compatible with |always|
and |never| clauses.
\begin{program}
(iter (for i in '(7 -4 2 -3))
      (if (plusp i)
	  (finding    i  such-that (evenp i))
          (finding (- i) such-that (oddp i))))
\end{program}

\startitem
\Clause{|finding| ~expr~ |maximizing| ~m-expr~ \opt\ |into| ~var~}
\clindexx{finding}{maximizing}
\Clause{|finding| ~expr~ |minimizing| ~m-expr~ \opt\ |into| ~var~}
\clindexx{finding}{minimizing}
\finishitem
Computes the extremum (maximum or minimum) value of ~m-expr~ over all
iterations, and returns the value of ~expr~ corresponding to the
extremum.  ~expr~ is evaluated inside the loop at the time the new
extremum is established.  If ~m-expr~ is never evaluated (due to, for
example, being embedded in a conditional clause), then the returned
value depends on the type, if any, of ~expr~ (or ~var,~ if
one is supplied).  If there is no type, the returned
value will be \nil; if the type is numeric, the returned value will be
zero.

\cpar For these two clauses, ~var~ may be a list of two
symbols; in that case, the first is used to record ~expr~ and
the second, ~m-expr.~  

\cpar As with |finding\dots such-that|, if ~m-expr~ is a sharp-quoted
function, then it is called on ~expr~ instead of being evaluated.

\end{clauses}

\subsubsection{Boolean Tests}

\begin{clauses}

\clausex{first-iteration-p}{}
Returns |t| in the first cycle of the loop, otherwise \nil.

\clausex{first-time-p}{}
Returns |t| the first time the expression is evaluated, and then \nil\ forever.
This clause comes handy when printing (optional) elements separated
by a comma:

\begin{program}
(iter (for el in '(nil 1 2 nil 3))
      (when el
        (unless (first-time-p)
          (princ ", "))
        (princ el)))
\end{program}
produces |"1, 2, 3"|.

\end{clauses}

\subsubsection{Aggregated Boolean Tests}

\begin{clauses}

\clausex{always}{~expr~}
If ~expr~ ever evaluates to
\nil, then \nil\ is immediately returned; the epilogue code is not
executed.  If ~expr~ never evaluates to \nil, the epilogue code
is executed and the last value of ~expr~ (or |t| if ~expr~ was never
evaluated) is returned (whereas |loop| would constantly return |t|).

% mention last evaluated clause when multiple always clauses?

\clausex{never}{~expr~}
Like |(always (not ~expr~))|, except it does not influence the last
value returned by a possible other |always| clause. That is,
\begin{program}
(iter (repeat 2)
      (always 2)
      (never nil)) \yields 2 ; not t
\end{program}

\clausex{thereis}{~expr~}
If ~expr~ is ever \nonnil,
its value is immediately returned without running epilogue code.
Otherwise, the epilogue code is performed and \nil\ is returned.

This clause cannot be used together with |always| or |never|, because their
defaults are opposed (similarly, |(loop always 3 thereis nil)| refuses to
compile in some implementations of |loop|).

\end{clauses}


\subsection{Control Flow}
\label{control-flow}
Several clauses can be used to alter the usual flow of control in a loop.

Note: the clauses of this and subsequent sections don't adhere to \iter's
usual syntax, but instead use standard Common Lisp syntax.  Hence the
format for describing syntax subsequently is
like the standard format used in the Common Lisp manual, not like the
descriptions of clauses above.

\begin{clauses}

\clausex{finish}{}
Stops the loop and runs the epilogue code.

%for example:
%
%\begin{program}
%(iter (with answer = nil)
%         (initially (make-a-mess))
%         (for i from 1 to 10)
%         (when (correct? i)
%           (setq answer i)
%           (finish))
%         (finally (cleanup)))
%\end{program}
%
%this code will execute |cleanup| whether or not the test |(correct?
%i)| ever succeeds. 
%the (more elegant) formulation,
%\begin{program}
%(iter (initially (make-a-mess))
%         (for i from 1 to 10)
%         (finding i such-that (correct? i))
%         (finally (cleanup)))
%\end{program}
%would not execute |cleanup| if |(correct? i)| succeeded; it
%would do an immediate return.

\clausex{leave}{\opt\ ~value~}
Immediately returns ~value~ (default \nil) from the current \iter\
form, skipping the epilogue code.  Equivalent to using |return-from|.

\clausex{next-iteration}{}
Skips the remainder of the loop body and begins the next iteration of
the loop.

\clausex{while}{~expr~}
If ~expr~ ever evaluates to \nil, the loop is terminated and the
epilogue code executed.  Equivalent to |(if (not ~expr~) (finish))|.

\clausex{until}{~expr~}
Equivalent to |(if ~expr~ (finish))|.

\clausex{if-first-time}{~then~ \opt\ ~else~}
If this clause is being executed for the first time in this invocation
of the \iter\ form, then the ~then~ code is evaluated; otherwise the
~else~ code is evaluated.

\cpar |(for ~var~ first ~expr1~ then ~expr2~)| is almost equivalent to
\begin{program}
(if-first-time (dsetq ~var~ ~expr1~)
               (dsetq ~var~ ~expr2~))
\end{program}
The only difference is that the |for| version makes ~var~ available
for use with |for\dots previous|.

\end{clauses}

\subsection{Code Placement}
When fine control is desired over where code appears in a loop
generated by \iter, the following special clauses may be useful.
They are all subject to code-motion problems (see section
\ref{code-movement}).  

\begin{clauses}

\clausex{initially}{|\&rest| ~forms~}
The lisp ~forms~ are placed in the prologue section of the loop, where
they are executed once, before the loop body is entered.

\clausex{after-each}{|\&rest| ~forms~}
The ~forms~ are placed at the end of the loop body, where they
are executed after each iteration.  Unlike the other clauses in this
section, ~forms~ may contain \iter\ clauses.

\clausex{else}{|\&rest| ~forms~}
The lisp ~forms~ are placed in the epilogue section of the loop, where they
are executed if this |else| clause is never met during execution of the
loop and the loop terminates normally.

\clausex{finally}{|\&rest| ~forms~}
The lisp ~forms~ are placed in the epilogue section of the loop, where
they are executed after the loop has terminated normally.

\clausex{finally-protected}{|\&rest| ~forms~}
The lisp ~forms~ are placed in the second form of an unwind-protect
outside the loop.  They are always executed after the loop has
terminated, regardless of how the termination occurred.

\end{clauses}


%%%%%%%%%%%%%%%%%%%%%%%%%%%%%%%%%%%%%%%%%%%%%%%%%%%%%%%%%%%%%%%%
\section{Other Features}

\subsection{Multiple Accumulations}
\label{multiple}

\begin{sloppypar}
It is permitted to have more than one clause accumulate into the same
variable, as in the following:
\begin{program}
(iter (for i from 1 to 10)
      (collect i into nums)
      (collect (sqrt i) into nums)
      (finally (return nums)))
\end{program}
Clauses can only accumulate into the same variable if they are
compatible.  |collect|, |adjoining|, |appending|, |nconcing|,
|unioning| and |nunioning| are compatible with each other; |sum|,
|multiply| and |counting| are compatible; |always| and |never| are compatible;
|finding| \dots |such-that| is compatible with either |thereis| or |always|
and |never| when their defaults match; and |maximize| and |minimize| clauses
are compatible only with other |maximize| and |minimize| clauses,
respectively.

%note that the same variable ~cannot~ be both an accumulation
%variable and an ordinary variable; there can be only one variable with
%a given name within an \iter\ form. 
\end{sloppypar}

\subsection{Named Blocks}

Like Common Lisp |block|s, \iter\ forms can be given names.  The
name should be a single symbol, and it must be the first form in the
\iter.  The generated code behaves exactly like a named block; in
particular, |(return-from ~name~)| can be used to exit it:
\begin{program}
(iter fred
      (for i from 1 to 10)
      (iter barney
            (for j from i to 10)
            (if (> (* i j) 17)
                (return-from fred j))))
\end{program}
An \iter\ form that is not given a name is implicitly named \nil.

Sometimes one would like to write an expression in an inner \iter\ form,
but have it processed by an outer \iter\ form.  This is possible with
the |in| clause.  

\begin{clauses}

\clausex{in}{~name~ |\&rest| ~forms~}
Evaluates ~forms~ as if they were part of the \iter\ form named
~name~.  In other words, \iter\ clauses are processed by the \iter\
form named ~name,~ and not by any \iter\ forms that occur inside ~name.~

\cpar As an example, consider the problem of collecting a list of the
elements in a two-dimensional array.  The naive solution,
\begin{program}
(iter (for i below (array-dimension ar 0))
      (iter (for j below (array-dimension ar 1))
            (collect (aref ar i j))))
\end{program}
\noindent is wrong because the list created by the inner \iter\ is simply
ignored by the outer one.  But using |in| we can write:
\begin{program}
(iter outer (for i below (array-dimension ar 0))
      (iter (for j below (array-dimension ar 1))
            (in outer (collect (aref ar i j)))))
\end{program}
\noindent which has the desired result.

\end{clauses}

\subsection{Destructuring}
\label{destructuring}

In many places within \iter\ clauses where a variable is expected, a
list can be written instead.  In these cases, the value to be assigned
is ~destructured~ according to the pattern
described by the list.  As a simple example, the clause
\begin{program}
(for (key . item) in alist)
\end{program}
\noindent will result in |key| being set to the |car| of
each element in |alist|, and |item| being set to the |cdr|.  The
pattern list may be nested to arbitrary depth, and (as the example
shows) need not be terminated with \nil; the only requirement is that
each leaf be a bindable symbol (or \nil, in which case no binding is
generated for that piece of the structure).

Sometimes, you might like to do the equivalent of a
|multiple-value-setq| in a clause.  This 
``multiple-value destructuring'' can be expressed by writing 
\linebreak |(values $pat_1$ $pat_2 \ldots$)| for a destructuring
pattern, as in 
\begin{program}
(for (values (a . b) c d) = (three-valued-function ...))
\end{program}
\begin{sloppypar}
Note that the $pat_i$ can themselves be destructuring patterns (though
not multiple-value destructuring patterns).  You can't do multiple-value
destructuring in a |with| clause; instead wrap the whole \iter\
form in a |multiple-value-bind|.
\end{sloppypar}

\begin{note}{Rationale}
There are subtle interactions between variable declarations and
evaluation order that make the correct implementation of
multiple-value destructuring in a |with| somewhat tricky.
\end{note}

The destructuring feature of \iter\ is available as a separate
mechanism, using the |dsetq| macro:

\begin{clauses}
\defmacro{dsetq}{template expr}
Performs destructuring of ~expr~ using ~template~.   May be used
outside of an \iter\ form. Yields the primary value of ~expr~.

\end{clauses}

\subsection{On-line Help}

\begin{sloppypar}
There is a limited facility for on-line help, in the form of the 
|display-iterate-clauses| function.
\end{sloppypar}

\begin{clauses}

\defun{display-iterate-clauses}{\opt\ clause-spec}
Displays a list of \iter\ clauses.  If ~clause-spec~ is not
provided, all clauses are shown; if it is a symbol, all clauses
beginning with that symbol are shown; and if it is a list of symbols,
all clauses for which ~clause-spec~ is a prefix are shown.

\end{clauses}

\subsection{Parallel Binding and Stepping}
\label{bindings}

The parallel binding and stepping of variables is a feature that
\iter\ does ~not~ have.  This section attempts to provide a rationale.

We say that two variables are bound ~in parallel~ if neither
binding shadows the other.  This is the usual semantics of |let|
(as opposed to |let*|).  Similarly, we can say that iteration
variables are stepped in parallel if neither variable is updated
before the other, conceptually speaking; in other words, if the code
to update each variable can reference the old values of both variables.

|loop| allows parallel binding of variables and parallel stepping of
driver variables.  My view is that if you are depending on the 
serial/parallel distinction, you are doing something obscure.  If you
need to bind 
variables in parallel using |with|, then you must be using a
variable name that shadows a name in the existing lexical environment.
Don't do that.  The most common use for parallel stepping is to track
the values of variables on the previous iteration, but in fact this
does not require parallel stepping at all; the following will work:
\begin{program}
(iter (for current in list)
      (for prev previous current)
      ...)
\end{program}


%%%%%%%%%%%%%%%%%%%%%%%%%%%%%%%%%%%%%%%%%%%%%%%%%%%%%%%%%%%%%%%%
\section{Types and Declarations}
\label{types}

\subsection{Discussion}

Sometimes efficiency dictates that the types of variables be declared.
This type information needs to be communicated to \iter\ so it can
bind variables to appropriate values.  Furthermore, \iter\ must often
generate internal variables invisible to the user; there needs to be a
way for these to be declared.

As an example, consider this code, which will return the number of
odd elements in |number-list|:
\begin{program}
(iter (for el in number-list)
      (count (oddp el)))
\end{program}
In processing this form,
\iter\ will create an internal variable, let us call it |list17|, to
hold the successive |cdr|s of |number-list|, and  
will bind the variable to |number-list|.  It will also generate a
default binding for |el|; only inside the body of the loop will |el|
be set to the |car| of |list17|.  Finally, \iter\ will generate a
variable, call it |result|, to hold the result of the count, and will
bind it to zero.

When dealing with type declarations, \iter\ observes one simple rule:
~it will never generate a declaration unless requested to do so.~  The
reason is that such declarations might mask errors in compiled code by
avoiding error-checks; the resulting problems would be doubly hard to
track down because the declarations would be hidden from the
programmer.  Of course, a compiler might omit error-checks even in the
absence of declarations, though this behavior can usually be avoided,
e.g. by saying |(declaim (optimize (safety 3)))|.

So, the above \iter\ form will generate code with no declarations.
But say we wish to declare the types of |el| and the internal
variables |list17| and |result|.  How is this done?

Declaring the type of |el| is easy, since the programmer knows
the variable's name:
\begin{program}
(iter (for el in number-list)
      (declare (fixnum el))
      (counting (oddp el)))
\end{program}
\iter\ can read variable type declarations like this one.  Before
processing any clauses, it scans the entire top-level form for type
declarations 
and records the types, so that variable bindings can be performed
correctly.  In this case, |el| will be bound to zero
instead of \nil.  Also, \iter\ collects all the top-level declarations
and puts them at the begining of the generated code, so it is not
necessary to place all declarations at the beginning of an \iter\
form; instead, they can be written near the variables whose types they
declare.  

Since \iter\ is not part of the compiler, it will not know
about declarations that occur outside an \iter\ form; these
declarations must be repeated inside the form.

Here is another way we could have declared the type of |el|:
\begin{program}
(iter (for (the fixnum el) in number-list)
      (counting (oddp el)))
\end{program}
\iter\ extends the Common Lisp |the|\lispindex{the}
form to apply to variables as well as value-producing forms; anywhere
a variable is allowed---in a |with| clause, as the iteration
variable in a driver clause, as the |into| argument of an
accumulation clause, even inside a destructuring template---you can
write |(the ~type~ ~symbol~)| instead.

There is one crucial difference between using a |the| form and
actually declaring the variable: explicit declarations are always
placed in the generated code, but type information from a |the|
form is not turned into an actual declaration unless you tell \iter\
to do so using |iterate:declare-variables|.  See below.

\begin{sloppypar}
Declaring the types of internal variables is harder than declaring the
types of explicitly mentioned variables, since their names
are unknown.  You do it by declaring |iterate:declare-variables|
somewhere inside the top level of the \iter\ form.  (This will also
generate declarations for variables declared using |the|.)
\iter\ does not provide much selectivity here: it's all or none. 
And unfortunately, since \iter\ is not privy to compiler information
but instead reads declarations itself, it will not hear if you
|(declaim (iterate:declare-variables))|.  Instead, set the variable 
|iterate::*always-declare-variables*| to |t| at
compile-time, using |eval-when|.
\end{sloppypar}

To determine the appropriate types for internal variables, \iter\ uses
three sources of information:

\begin{itemize}

\item Often, the particular clause dictates a certain type for a
variable; \iter\ will use this information when available.  In the
current example, the variable |list17| will be given the type
|list|, since that is the only type that makes sense; and the
variable |result| will be given the type |fixnum|, on the
assumption that you will not be counting high enough to need bignums.
You can override this assumption only by using and explicitly declaring a
variable: 
\begin{verbatim}
(iter (declare (iterate:declare-variables))
      (for el in number-list)
      (count (oddp el) into my-result)
      (declare (integer my-result))
      (finally (return my-result)))
\end{verbatim}

\begin{sloppypar}
Other examples of the type assumptions that \iter\ makes are: type
|list| for |into| variables of collection clauses; type |list| for
expressions that are to be destructured; type |vector| for the
variable holding the vector in a |for\dots in-vector| clause, and
similarly for |string| and the |for\dots in-string| clause;
and the implementation-dependent type
|(type-of array-dimension-limit)| for the index and limit
variables generated by sequence iteration drivers like |for\dots
in-vector| and |for\dots in-string| (but not |for\dots in-sequence|,
because it may be used to iterate over a list). 
\end{sloppypar}

\item Sometimes, \iter\ will examine expressions and try to determine
their types in a simple-minded way.  If the expression is
self-evaluating (like a number, for instance), \iter\ knows that the
expression's type is the same as the type of the value it denotes, so
it can use that type.  If the expression is of the form |(the ~type~
~expr~)|, \iter\ is smart enough to extract ~type~ and use it.
However, the current version of \iter\ does not  
examine declarations of function result types or do any type
inference.  It will not determine, for 
example, that the type of |(+ 3 4)| is |fixnum|, or even
|number|.

\item In some cases, the type of an internal variable should match the
type of some other variable.  For instance, \iter\ generates an
internal variable for |(f x)| in the
clause |(for i from 1 to (f x))|, and in the absence of other
information will give it the same type as |i|.  If, however, the
expression had been written |(the fixnum (f x))|, then \iter\
would have given the internal variable the type |fixnum|
regardless of |i|'s type.  The type incompatibility errors that
could arise in this situation are not checked for.

\end{itemize}

Note that if you do declare |iterate:declare-variables|, then
\iter\ may declare user variables as well as internal ones if they do
not already have declarations, though only for variables that it
binds.  For instance, in this code:

\begin{program}
(iter (declare (iterate:declare-variables))
      (for i from 1 to 10)
      (collect i into var))
\end{program}
the variable |var| will be declared to be of type |list|.


\subsection{Summary}

\iter\ understands standard Common Lisp variable type declarations
that occur within an \iter\ form and
will pass them through to the generated code.  If the declaration
|(iterate:declare-variables)|\lispindex{declare-variables}
appears at the top level of an 
\iter\ form, or if 
|iterate::*always-declare-variables*|\lispindex{*always-declare-variables*} 
is \nonnil, then \iter\ will use the type information gleaned from user
declarations, self-evaluating expressions and |the| expressions,
combined with reasonable assumptions, to determine variable
types and declare them.


%%%%%%%%%%%%%%%%%%%%%%%%%%%%%%%%%%%%%%%%%%%%%%%%%%%%%%%%%%%%%%%%
\section{Problems with Code Movement}
\label{code-movement}

Some \iter\ clauses, or parts of clauses, result in code being
moved from the location of the clause to other parts of the loop.
Drivers behave this way, as do code-placement clauses like |initially|
and |finally|.  When using these clauses, there is a danger of writing
an expression that makes sense in its apparent location but will be
invalid or have a different meaning in another location.  For example:
\begin{program}
(iter (for i from 1 to 10)
      (let ((x 3))
        (initially (setq x 4))))
\end{program}
While it may appear that the |x| of |(initially (setq x 4))| is the
same as the |x| of |(let ((x 3)) \dots|, in fact they are not:
|initially| moves its code outside the loop body, so |x| would refer
to a global variable.  Here is another example of the same problem:
\begin{program}
(iter (for i from 1 to 10)
      (let ((x 3))
        (collect i into x)))
\end{program}
If this code were executed, |collect| would create a binding for its
|x| at the top level of the \iter\ form that the |let| will shadow.

Happily, \iter\ is smart enough to catch these errors; it
walks all problematical code to ensure that free variables are not
bound inside the loop body, and checks all variables it binds for the
same problem.

However, some errors cannot be caught:

\begin{program}
(iter (with x = 3)
      (for el in list)
      (setq x 1)
      (reducing el by \#'+ initial-value x))
\end{program}
|reducing| moves its |initial-value| argument to the initialization
part of the loop in order to produce more efficient code.  Since
\iter\ does not perform data-flow analysis, it cannot determine that
|x| is changed inside the loop; all it can establish is that |x| is
not bound internally.  Hence this code will not signal an
error and will use $3$ as the initial value of the reduction.

The following list summarizes all cases that are subject to these code
motion and variable-shadowing problems.
\begin{itemize}
\item Any variable for which \iter\ creates a binding, including those
used in |with| and the |into| keyword of many clauses.

\begin{sloppypar}
\item The special clauses which place code: |initially|, |after-each|, |else|,
|finally| and |finally-protected|. 
\end{sloppypar}

\item The variables of a |next| or |do-next| form.

\item The |initially| arguments of |for\dots initially\dots then| and
|for\dots previous|. 

\item The |then| argument of |for\dots initially\dots then|.

\item The |initial-value| arguments of |reducing| and |accumulate|.

\item The |on-failure| argument of |finding\dots such-that|.
\end{itemize}

%%%%%%%%%%%%%%%%%%%%%%%%%%%%%%%%%%%%%%%%%%%%%%%%%%%%%%%%%%%%%%%%
\section{Differences Between {\tt Iterate} and {\tt Loop}}
|loop| contains a great deal of complexity which \iter\ tries to
avoid.  Hence many esoteric features of |loop| don't exist in \iter.
Other features have been carried over, but in a cleaned-up form.  
And of course, many new features have been added; they are not
mentioned in this list.

\begin{itemize}

\item \iter's syntax is more Lisp-like than |loop|'s, having a higher
density of parens.

\item The current implementation of \iter, unlike the current version
of  |loop| (as documented in {\em Common Lisp, 2nd Ed.\/}), is
extensible (see section \ref{extend}).  

\item |loop| puts the updates of all driver variables at the top of
the loop; \iter\ leaves them where the driver clauses appear.

\item While for the most part \iter\ clauses that resemble |loop| clauses
behave similarly, there are some differences.  For instance, there is
no |for\dots =\dots then| in \iter; instead use 
|for\dots initially\dots then|.  

\item |loop| binds the variable |it| at certain times
to allow pseudo-English expressions like |when ~expr~ return it|. 
In \iter, you must bind ~expr~ to a variable yourself.  Note that
|when ~expr~ return it| is like |thereis ~expr~| except that the latter is an
accumulation clause and therefore competes with other accumulations
(remember section \ref{multiple} above).

% repeat different behaviour of |always| clause here?

\item |loop| has a special |return| clause, illustrated in the
previous item.  \iter\ doesn't need one,  since an ordinary Lisp
|return| has the same effect.

\item |loop| allows for parallel binding and stepping of iteration
variables.  \iter\ does not.  (See section \ref{bindings}.)

\item |loop| and \iter\ handle variable type declarations very
differently.  |loop| provides a special syntax for declaring variable
types, and does not examine declarations.  Moreover, the standard
implementation of |loop| will
generate declarations when none are requested.
\iter\ parses standard Common Lisp type declarations, and will never
declare a variable itself unless declarations are 
specifically requested.

\end{itemize}

%%%%%%%%%%%%%%%%%%%%%%%%%%%%%%%%%%%%%%%%%%%%%%%%%%%%%%%%%%%%%%%%
\section{Rolling Your Own}
\label{extend}

\subsection{Introduction}

\iter\ is extensible---you can write new clauses that embody new
iteration patterns.  You might want to write a new driver clause for a
data structure of your own, or you might want to write a clause that
collects or manipulates elements in a way not provided by \iter. 

This section describes how to write clauses for \iter.  Writing a
clause is like writing a macro.  In fact, writing a clause ~is~
writing a macro: since \iter\ code-walks its body and macroexpands,
you can add new abstractions to \iter\ with good old |defmacro|.

Actually, there are two extensions you can make to \iter\ that are
even easier than writing a macro.  They are adding a synonym for an
existing clause and defining a driver clause for an indexable
sequence.  These can be done with |defsynonym| and
|defclause-sequence|, respectively.  See section \ref{aids}, below.

The rest of this section explains how to write macros that expand into
\iter\ clauses.
Here's how you could add a simplified version of \iter's |multiply|
clause, if \iter\ didn't already have one:
\begin{program}
(defmacro multiply (expr)
  `(reducing ,expr by \#'* initial-value 1))
\end{program}

If you found yourself summing the square of an expression often, you
might want to write a macro for that.  A first cut might be
\begin{program}
(defmacro sum-of-squares (expr)
  `(sum (* ,expr ,expr)))
\end{program}
but if you are an experienced macro writer, you will realize that this
code will evaluate ~expr~ twice, which is probably a bad idea.  A
better version would use a temporary:
\begin{program}
(defmacro sum-of-squares (expr)
  (let ((temp (gensym)))
    `(let ((,temp ,expr))
       (sum (* ,temp ,temp)))))
\end{program}
Although this may seem complex, it is just the sort of thing you'd
have to go through to write any macro, which illustrates the point of
this section: if you can write macros, you can extend \iter.

Our macros don't use \iter's keyword-argument
syntax.  We could just use keywords with |defmacro|, but we would
still not be using \iter's clause indexing mechanism.  Unlike Lisp,
which uses just the first symbol of a form to determine what function
to call, \iter\ individuates clauses by the list of required keywords.
For instance, |for\dots in| and |for\dots in-vector| are different clauses
implemented by distinct Lisp functions.  

To buy into this indexing scheme, as well as the keyword-argument
syntax, use |defmacro-clause|:

\begin{clauses}

\defmacro{defmacro-clause}{arglist |\&body| body}
Defines a new \iter\ clause.  ~arglist~ is a list of symbols which are
alternating keywords and arguments.  \opt\ may be used, and the list
may be terminated by |\&sequence|.  ~body~ is an ordinary macro body,
as with |defmacro|.  If the first form of ~body~ is a string, it is
considered a documentation string and will be shown by
|display-iterate-clauses|. |defmacro-clause| will signal an error if 
defining the clause would result in an ambiguity.  E.g. you cannot
define the clause |for\dots from| because there would be no way to
distinguish it from a use of the |for| clause with optional keyword |from|.

\end{clauses}

\medskip

Here is |multiply| using |defmacro-clause|.  The keywords are capitalized
for readability.
\begin{program}
(defmacro-clause (MULTIPLY expr \opt\ INTO var)
  `(reducing ,expr by \#'* into ,var initial-value 1))
\end{program}
You don't have to worry about the case when |var| is not supplied; for
any clause with an |into| keyword, saying |into nil| is equivalent to
omitting the |into| entirely.

As another, more extended example, consider the fairly common
iteration pattern that 
involves finding the sequence element that maximizes (or minimizes) some
function.  \iter\ provides this as |finding\dots maximizing|, but it's
instructive to see how to write it.
Here, in pseudocode, is how you might write such a loop for
maximizing a function F:

\begin{quote}
\begin{tabbing}
set variable MAX-VAL to NIL; \\
set variable WINNER to NIL; \\
for\=\ each element EL in the sequence \\
\>  if\=\ MAX-VAL\=\ is NIL or F(EL) $>$ MAX-VAL then \\
\>\>    set MAX-VAL to F(EL); \\
\>\>    set WINNER to EL; \\
\>  end if; \\
end for; \\
return WINNER.
\end{tabbing}
\end{quote}

Here is the macro:
\begin{program}
(defmacro-clause (FINDING expr MAXIMIZING func \opt\ INTO var)
  (let ((max-val (gensym))
        (temp1 (gensym))
        (temp2 (gensym))
        (winner (or var iterate::*result-var*)))
    `(progn 
       (with ,max-val = nil)
       (with ,winner = nil)
       (cond 
        ((null ,max-val)
         (setq ,winner ,expr)
         (setq ,max-val (funcall ,func ,winner))
        (t
         (let* ((,temp1 ,expr)
                (,temp2 (funcall ,func ,temp1)))
           (when (> ,temp2 ,max-val)
             (setq ,max-val ,temp2)
             (setq ,winner ,temp1))))))
       (finally (leave ,winner)))))
\end{program}
Note that if no |into| variable is supplied, we use
|iterate::*result-var*|, which contains the internal variable into
which all clauses place their results.  If this variable is bound by
some clause, then \iter\ will return its value automatically;
otherwise, \nil\ will be returned.  

\subsection{Writing Drivers}

In principle, drivers can be implemented just as easily as other
\iter\ clauses.  In practice, they are a little harder to get right. 
As an example, consider writing a driver that
iterates over all the 
elements of a vector, ignoring its fill-pointer.  |for\dots in-vector|
won't work for this, because it observes the fill-pointer.  It's necessary to
use |array-dimension| instead of |length| to obtain the size of the
vector.  Here is one approach:

\begin{program}
(defmacro-clause (FOR var IN-WHOLE-VECTOR v)
  "All the elements of a vector (disregards fill-pointer)"
  (let ((vect (gensym))
        (index (gensym)))
    `(progn
       (with ,vect = ,v)
       (for ,index from 0 below (array-dimension ,vect 0))
       (for ,var = (aref ,vect ,index)))))
\end{program}
Note that we immediately put |v| in a variable, in case it is an
expression.  Again, this is just good Lisp macrology.  It also has a
subtle effect on the semantics of the driver: |v| is evaluated only
once, at the beginning of the loop, so changes to |v| in the loop have
no effect on the driver.  Similarly, the bounds for numerical iteration
e.g. the above |array-dimension| are also evaluated once only.  This is how
all of \iter's drivers work.


There is an important point concerning the |progn| in this code.  We
need the |progn|, of course, because we are returning several forms,
one of which is a driver.  But \iter\ drivers must occur at top-level.
Is this code in error?  No, because ~top-level~ is defined in \iter\
to include forms inside a |progn|.  This is just the definition of
top-level that Common Lisp uses, and for the same reason: to allow
macros to return multiple forms at top-level.

While our |for\dots in-whole-vector| clause will work, it is not
ideal.  In particular, it does not support generating.  Do do so, we
need to use |for\dots next| or |for\dots do-next|.  The job is
simplified by the |defmacro-driver| macro.

\begin{clauses}

\defmacro{defmacro-driver}{arglist |\&body| body}
Defines a driver clause in
both the |for| and |generate| forms, and provides a parameter
|generate| which ~body~ can examine to determine how it was invoked.
~arglist~ is as in |defmacro-clause|, and should begin with the symbol
|for|. 

\end{clauses}

With |defmacro-driver|, our driver looks like this:
\begin{program}
(defmacro-driver (FOR var IN-WHOLE-VECTOR v)
  "All the elements of a vector (disregards fill-pointer)"
   (let ((vect (gensym))
         (end (gensym))
         (index (gensym))
         (kwd (if generate 'generate 'for)))
     `(progn
        (with ,vect = ,v)
        (with ,end = (array-dimension ,vect 0))
        (with ,index = -1)
        (,kwd ,var next (progn (incf ,index)
                               (if (>= ,index ,end) (terminate))
                               (aref ,vect ,index))))))
\end{program}


We are still missing one thing: the |\&sequence| keywords.  
We can get them easily enough, by writing 
\begin{program}
(defmacro-driver (FOR var IN-WHOLE-VECTOR v \&sequence) 
  ...)
\end{program}
We can now refer to parameters |from|, |to|, |by|, etc. which contain
either the values for the corresponding keyword, or \nil\ if the
keyword was not supplied.  Implementing the right code for these
keywords is cumbersome but not difficult; it is left as an exercise.
But before you begin, see |defclause-sequence| below for an easier way.
       
\subsection{Extensibility Aids}
\label{aids}

This section documents assorted features that may be of use in
extending \iter.

\begin{clauses}

\defunexpvar{*result-var*}
Holds the variable that is used to return a value as a result of the
\iter\ form.  You may examine this and use it in a |with| clause, but
you should not change it.

\defmacro{defsynonym}{syn word}
 Makes ~syn~ a synonym for the existing \iter\ keyword ~word.~  Only
the first word in each clause can have synonyms. 

\defmacroxx{defclause-sequence}{element-name index-name}{|\&key|
access-fn size-fn sequence-type}{element-type
element-doc-string index-doc-string}  
Provides
a simple way to define sequence clauses.  Generates two
clauses, one for iterating over the sequence's elements, the other
for iterating over its indices.  The first symbol of both
clauses will have print-name |for|.
~element-name~  and ~index-name~ should be symbols.
~element-name~ is the second keyword of the element iterator (typically of
the form 
|in-~sequence-type~|), and ~index-name~ is the second keyword
of the index-iterator (typically of the form
|index-of-~sequence-type~|).  Either name may be 
\nil, in which case the corresponding clause is not defined.  If both
symbols are supplied, they should be in the same package.  The |for|
that begins the clauses will be in this package.

\cpar ~access-fn~ is the function to be used to
access elements of the sequence in the element iterator.  The function
should take two 
arguments, a sequence and an index, and return the appropriate element.
~size-fn~ should denote a function of one argument, a sequence, that
returns its size.  Both ~access-fn~ and ~size-fn~ are required for the
element iterator, but only ~size-fn~ is needed for the index iterator.

\cpar The ~sequence-type~ and ~element-type~ keywords are used to
suggest types for the variables 
used to hold the sequence and the
sequence elements, respectively.  The usual rules about \iter's
treatment of variable type declarations apply (see section \ref{types}).

\cpar ~element-doc-string~ and ~index-doc-string~ are
the documentation strings, for use with |display-iterate-clauses|. 

\cpar The generated element-iterator performs destructuring on the
element variable.

\cpar As an example, the above |for\dots in-whole-vector| example
could have been written:
\begin{program}
(defclause-sequence IN-WHOLE-VECTOR INDEX-OF-WHOLE-VECTOR
  :access-fn 'aref
  :size-fn \#'(lambda (v) (array-dimension v 0))
  :sequence-type 'vector
  :element-type t
  :element-doc-string 
     "Elements of a vector, disregarding fill-pointer"
  :index-doc-string 
     "Indices of vector, disregarding fill-pointer")
\end{program}

\end{clauses}

\subsection{Subtleties}

There are some subtleties to be aware of when writing \iter\ clauses.
First, the code returned by your macros may be |nconc|'ed into a list,
so you should always returned freshly consed lists, rather than
constants.  Second, \iter\ matches clauses by using |eq| on the first
symbol and |string=| on the subsequent ones, so the package of the
first symbol of a clause is relevant.  All of the clauses in this manual
have their first word in the \iter\ package.  
You can use the package system in the usual way to shadow
\iter\ clauses without replacing them.

%%% say more here, about the badness that only the first word of a
%%% clause is packagey.

\section{Non-portable Extensions to Iterate (Contribs)}
\label{contribs}

Currently, there is only one non-portable extension to iterate in the
distribution: iterate-pg. If you have made an extension that depends
on non-portable features, feel free to send them to |asf@boinkor.net|
for inclusion in the iterate distribution.

\subsection{An SQL query driver for iterate}

The pg package by Eric Marsden (see |http://cliki.net/pg|) provides an
interface to the PostgreSQL database. Using the \iterpg\ extension, it
is possible to handle the results of SQL queries using \iter.

This usage example should give you an idea of how to use it:

\begin{program}
(pg:with-pg-connection (c "somedb" "someuser")
  (iter (for (impl version date) in-relation "select * from version"
                                 on-connection *dbconn*)
        (collect version)))
\end{program}

To use the extension via |ASDF|, simply make your system depend on the
|iterate-pg| system instead of the |iterate| system. To load it
manually, use:

\begin{program}
  (asdf:oos 'asdf:load-op :iterate-pg)
\end{program}

%%%%%%%%%%%%%%%%%%%%%%%%%%%%%%%%%%%%%%%%%%%%%%%%%%%%%%%%%%%%%%%%
\section{Obtaining {\tt Iterate}}

\begin{sloppypar}
The information in this chapter is obsolete but included for
completeness's sake; Currently, the most up-to-date information on
\iter\ can be found at |http://boinkor.net/iterate.html|.
\end{sloppypar}

\begin{sloppypar}
\iter\ currently runs on Lisp Machines, and on
HP's, Sun3's and Sparcstations under Lucid.  
\iter\ source and binaries are available at the MIT AI Lab in the
subdirectories of |/src/local/lisplib/|.  The source file,
|iterate.lisp|, is also available for anonymous FTP in the directory
|/com/fpt/pub/| on the machine |TRIX.AI.MIT.EDU| (Internet number
128.52.32.6).  If you are unable to obtain |iterate| in one of these
ways, send mail to |jba@ai.mit.edu| and I will send you the source
file. 
\end{sloppypar}

\begin{sloppypar}
\iter\ resides in the |iterate| package (nickname |iter|).  Just say
\linebreak |(use-package :iterate)| to make all the necessary symbols
available. 
If a symbol is not exported, it appears in this manual with an
``|iterate::|'' prefix.
\end{sloppypar}

Send bug reports to |bug-iterate@ai.mit.edu|.  The |info-iterate|
mailing list will have notices of changes and problems; to have
yourself added, send mail to |info-iterate-request@ai.mit.edu|.


\medskip

\begin{flushleft}
 \bf Acknowledgements
\end{flushleft}
\smallskip

Richard Waters provided invaluable criticism which spurred me to improve
\iter\ greatly.  As early users, David Clemens, Oren Etzioni and Jeff
Siskind helped ferret out many bugs.

%\begin{theindex}
% The index files must be generated with the genindex program
% or from makeindex iterate-manual.idx which creates iterate-manual.ind.

%\baselineskip\normalbaselineskip
%\advance\baselineskip by -2pt

\input{iterate-manual.ind}

%\underline{Clauses}
%\input{iterate-manual.clindex}
%
%\indexspace
%\underline{Lisp}
%\input{iterate-manual.lispindex}

%\end{theindex}


\end{document}


% arch-tag: "c0f871a6-313c-11d8-abb9-000c76244c24"

%
%\indexspace
%\underline{Lisp}
%% Fix ``.'' spacing in lists
\documentstyle[12pt,aimemo]{article}

\setlength{\textwidth}{6.0in}
\setlength{\oddsidemargin}{0.25in}
\setlength{\evensidemargin}{0.25in}
\setlength{\topmargin}{0pt}
\setlength{\headsep}{24pt}
\setlength{\textheight}{8.5in}

\let\origpar=\par

% somehow, LaTex fucks with things so that the plain Tex \obeyspaces
% doesn't work.  This is the same, except we use an mbox containing
% a space instead of just a space.

\newcommand{\spc}{\mbox{ }}
\newcommand{\tAb}{\mbox{        }}
\def\obspaces{\catcode`\ =\active\catcode`\^^I=\active}
{\obspaces \global\let =\spc\global\let^^I=\tAb}

\newenvironment{program}{\medskip\samepage\tt\obeylines\obspaces}{\medskip}

% use | for \tt
\catcode`\|=13
\newif\ifvbar
\def|{\ifvbar
         \endgroup\vbarfalse
      \else 
         \begingroup\tt\vbartrue 
      \fi}



\newcommand{\lisp}{\tt}
\newcommand{\iter}{|iterate|}
\newcommand{\iterpg}{|iterate-pg|}
\newcommand{\setf}{|setf|}
\newcommand{\nil}{|nil|}
\newcommand{\nonnil}{non-\nil}
\newcommand{\opt}{{\lisp \&optional}}
\newcommand{\yields}{$\Rightarrow$}



\newbox\Dots
\setbox\Dots=\hbox{\small ...}
\renewcommand{\dots}{\copy\Dots}

% Here's a way to define an environment which will treat all
% paragraphs inside it in the same manner.  It's, as near as I can
% figure, the way that Latex does things.  Using \everypar won't work;
% apparently Latex resets that a lot.  What you have to do is: at the
% top of the file do \let\origpar=\par.  In the
% begin part of the environment, do the parshape (or whatever), then say
% \def\par{{\origpar}}.  The extra braces are necessary.   That's all
% you have to do; at the end of the environment things will return to
% normal automatically.  Why does this all work?  I have no idea.
% Another important thing: there should be a blank line (i.e. a \par)
% between the end of the last paragraph and the \end{environment}.

\newlength{\clindent}
\setlength{\clindent}{0.5in}
\newlength{\clparindent}
\setlength{\clparindent}{\parindent}

\newenvironment{clauses}{\advance\linewidth -\parindent
                        \hangindent\clindent\relax
                         \hangafter=0
                          \parindent=0pt
                          \def\par{{\origpar}}}{}

\newcommand{\cpar}{\hspace \clparindent}

\newcommand{\startitem}{\bigskip\pagebreak[3]}
\newcommand{\finishitem}{\smallskip}

\newcommand{\clause}[1]{\startitem\Clause{#1}\finishitem}
\newcommand{\Clause}[1]{\hbox{#1}}
\newcommand{\clindex}[1]{\index{{\lisp #1}}}
\newcommand{\clindexx}[2]{\index{{\lisp #1\dots #2}}}
\newcommand{\clindexxx}[3]{\index{{\lisp #1\dots #2\dots #3}}}
\newcommand{\clausex}[2]{\startitem\Clausex{#1}{#2}\finishitem}
\newcommand{\Clausex}[2]{\Clause{|#1| #2}\clindex{#1}}

\newcommand{\defvar}[1]{\startitem\deff{#1}{}{Variable}\finishitem}
\newcommand{\defunexpvar}[1]{\startitem\deffnoindex{iterate::#1}{}{Variable}
                                        \lispindex{#1}\finishitem}
\newcommand{\defun}[2]{\startitem\deff{#1}{#2}{Function}\finishitem}
\newcommand{\defmacro}[2]{\startitem\deff{#1}{#2}{Macro}\finishitem}
\newcommand{\defunx}[3]{\startitem\deffx{#1}{#2}{#3}{Function}\finishitem}
\newcommand{\defmacrox}[3]{\startitem\deffx{#1}{#2}{#3}{Macro}\finishitem}
\newcommand{\defunxx}[4]{\startitem\deffxx{#1}{#2}{#3}{#4}{Function}\finishitem}
\newcommand{\defmacroxx}[4]{\startitem\deffxx{#1}{#2}{#3}{#4}{Macro}\finishitem}

\newcommand{\deff}[3]{\deffnoindex{#1}{#2}{#3}\lispindex{#1}}
\newcommand{\deffnoindex}[3]{\hbox to \hsize{{\tt #1} {\it #2}\hfill [{\it #3}]}}
\newcommand{\deffx}[4]{\deff{#1}{#2}{#4}\moreargs{#1}{#3}}
\newcommand{\deffxx}[5]{\deffx{#1}{#2}{#3}{#5}\moreargs{#1}{#4}}

\newcommand{\moreargs}[2]{\hbox to \hsize{\phantom{\lisp #1} {\it #2}\hfill}}

\newcommand{\lispindex}[1]{\index{{\lisp #1}}}

 \newenvironment{note}[1]{\pagebreak[2]\bigskip
                          \hrule\smallskip\small
                         {\setlength{\parindent}{0pt}
                           \par{\bf #1:}}}{\smallskip\hrule\bigskip}



\makeindex

%%%%%%%%%%%%%%%%%%%%%%%%%%%%%%%%%%%%%%%%%%%%%%%%%%%%%%%%%%%%%%%%

%\thispagestyle{empty}

%\vspace*{2in}

%\title{The |Iterate| Manual}
%\author{Jonathan Amsterdam}

%\maketitle

%\thispagestyle{empty}

%\begin{center}
%\Large *** DRAFT ***
%\end{center}
%This manual is in beta-test. It will become an AI Memo in a couple of
%months.  I would appreciate any comments or
%corrections.  You can reach me at email address |jba@ai.ai.mit.edu|.

\begin{document}

\def\writtenby{Jonathan Amsterdam}
\def\contractno{This report describes research done at the Artificial
Intelligence Laboratory of the Massachusetts Institute of Technology.
Support for the laboratory's artificial intelligence research is
provided in part by the Advanced Research Projects Agency of the
Department of Defense under Office
of Naval Research contract N00014-85-K-0124.}

\AIMEMO{1236}{\Large The Iterate Manual \\[1in]}
{This is the manual for version 1.4 of \iter, a powerful iteration
macro for Common  
Lisp. \iter\ is similar to {\tt loop} but provides numerous additional
features, is well integrated with Lisp, and is extensible. 
\vfill }


% use ~ for \em
\catcode`\~=13
\newif\ifem
\def~{\ifem
         \/\endgroup\emfalse
      \else 
         \begingroup\em\emtrue 
      \fi}



%\pagebreak
%\setcounter{page}{1}

%\thispagestyle{empty}


\tableofcontents

\pagebreak
\pagestyle{headings}

\section{Introduction}

\begin{sloppypar}
This manual describes
\iter, a powerful iteration facility for Common Lisp.
\iter\ provides abstractions for many common iteration
patterns and allows for the definition of additional patterns.  \iter\ is
a macro that expands into ordinary Lisp at
compile-time, so it is more efficient than higher-order
functions like |map| and |reduce.|  
While it is similar to |loop|, \iter\ offers a more Lisp-like syntax
and enhanced extensibility.
(For a more complete
comparison of \iter\ with other iteration constructs, see MIT AI Lab Working
Paper No. 324, ~Don't Loop, Iterate.~)
\end{sloppypar}

An \iter\ form consists of the symbol |iter|\footnote{You can also use
|iterate|, but |iter| is preferred because it avoids potential conflicts with
possible future additions to Common Lisp, and because it saves
horizontal space when writing code.}
followed by one or more
forms, some of which may be \iter\ 
~clauses.~  Here is a simple example of \iter\ which collects the
numbers from 1 to 10 into a list, and returns the list.  The return
value is shown following the arrow.
\begin{program}
(iter (for i from 1 to 10)
      (collect i))          \yields (1 2 3 4 5 6 7 8 9 10)
\end{program}
This form contains two clauses: a |for| clause that steps the
variable |i| over the integers from 1 to 10, and a |collect|
clause that accumulates its argument into a list.  With a few
exceptions, all \iter\ clauses have the same format: alternating
symbols (called ~keywords~) and expressions (called
~arguments~).  The syntax and terminology are those of Common
Lisp's keyword lambda lists.  One difference is that \iter's keywords
do not have to begin with a colon---though they may, except for the
first symbol of a clause.  So you can also write  |(for i :from
1 :to 10)| if you prefer.

Any Lisp form can appear in the body of an \iter, where it will have
its usual meaning.  \iter\ walks the entire body, expanding macros,
and recognizing clauses at any level.  This example collects all the
odd numbers in a list:
\begin{program}
(iter (for el in list)
      (if (and (numberp el) (oddp el))
          (collect el)))
\end{program}

There are clauses for iterating over numbers, lists, arrays and other
objects, and for
collecting, summing, counting, maximizing and other useful operations.
\iter\ also supports the creation of new variable bindings, stepping
over multiple sequences at once, destructuring, and compiler
declarations of variable types.  The following example illustrates
some of these features: 
\begin{program}
(iter (for (key . item) in alist)
         (for i from 0)
         (declare (fixnum i))
         (collect (cons i key)))
\end{program}

This loop takes the keys of an alist and returns a new alist
associating the keys with their positions in the original list. The
compiler declaration for |i| will appear in the generated code
in the appropriate place.

%%%%%%%%%%%%%%%%%%%%%%%%%%%%%%%%%%%%%%%%%%%%%%%%%%%%%%%%%%%%%%%%
\section{Clauses}

Most of \iter's clauses will be familiar to |loop| programmers.  
(|loop| is an iteration macro that has been incorporated into Common
Lisp.  Seen Guy Steele's {\em Common Lisp, 2nd Edition}.)
In
nearly all cases they behave the same as their |loop| counterparts, so
a |loop| user can switch to \iter\ with little pain (and much gain).

All clauses with the standard keyword-argument syntax consist of two
parts: a ~required~ part, 
containing keywords that must be present and in the right order; and
an ~optional~ part, containing keywords that may be omitted and,
if present, may occur in any order.  In the descriptions below, the
parts are separated by the Lisp lambda-list keyword \opt.


\subsection{Drivers}

An iteration-driving clause
conceptually causes the iteration to go forward.  Driver clauses in
\iter\ allow iteration over numbers, lists, vectors, hashtables, packages,
files and streams.  Iteration-driving clauses must
appear at the top level of an \iter\ form; they cannot be nested
inside another clause.  The driver variable is updated at the point
where the driver clause occurs.  Before the clause is executed for the
first time, the value of the variable is undefined.

%Also, regardless of where the driver clause
%appears in the body, the driver variable is stepped at the top of the
%loop; hence it is stylistically preferable, though
%not required, to place driver clauses at the beginning of the \iter.

Multiple drivers may appear in a single \iter\ form, in which case all
of the driver variables are updated each time through the loop, in the
order in which the clauses appear.  The first driver to terminate will
terminate the entire loop.

In all cases, the value of the driver variable on exit from the loop,
including within the epilogue code (see the |finally| clause), is
undefined.

All the parameters of a driver clause are evaluated once, before the
loop begins.  Hence it is not possible to change the bounds or other
properties of an iteration by side-effect from within the loop.

With one exception, driver clauses begin with the word |for| (or
the synonym |as|) and mention an iteration variable, which is
given a binding within the \iter\ form.
The exception is |repeat|, which just executes a loop a
specified number of times:

\begin{clauses}

\clausex{repeat}{~n~}
Repeats the loop ~n~ times.  For example:
\begin{program}
(iter (repeat 100)
      (print "I will not talk in class."))
\end{program}
\cpar If $n \leq 0$, the
loop will never be executed.  If ~n~ is not an integer, the
actual number of executions will be $\lceil n \rceil$.

\end{clauses}

\pagebreak[3]

\subsubsection{Numerical Iteration}

\begin{clauses}
\clausex{for}{~var~ |\&sequence|} 
The general form for iterating over a sequence of numbers requires a
variable and, optionally, one or more keywords that provide the bounds
and step size of the iteration. The |\&sequence|
lambda-list keyword is a shorthand for these sequence keywords.  They are:
|from|, |upfrom|, |downfrom|, |to|, |downto|, |above|, |below| and |by|.
|from| provides the starting value for ~var~
and defaults to zero. |to| provides a final value and implies that the
successive values of ~var~ will be increasing; |downto| implies that
they will be decreasing.  The loop terminates when ~var~ passes
the final value (i.e. becomes smaller or larger than it, depending on
the direction of iteration); in other  words, the loop body
will never be 
executed for values of ~var~ past the final value.
|below| and |above| are similar to |to| and
|downto|, except that the loop terminates when ~var~ equals or passes
the final value.

\cpar If no final value is specified, the variable will be stepped
forever.  Using |from| or |upfrom| will result in increasing
values, while |downfrom| will give decreasing values.

\cpar On each iteration,
~var~ is incremented or decremented by the value of the sequence
keyword |by|, which defaults to 1.  It should always be a positive
number, even for downward iterations.

\cpar In the following examples, the sequence of numbers generated is shown
next to the clause.

\begin{program}
(for i upfrom 0) \yields\ 0 1 2 \ldots
(for i from 5)   \yields\ 5 6 7 \ldots    ; either from or upfrom is okay
(for i downfrom 0) \yields\ 0 -1 -2 \ldots
(for i from 1 to 3) \yields\ 1 2 3
(for i from 1 below 3) \yields\ 1 2
(for i from 1 to 3 by 2) \yields\ 1 3
(for i from 1 below 3 by 2) \yields\ 1
(for i from 5 downto 3) \yields\ 5 4 3
\end{program}

\end{clauses}

\subsubsection{Sequence Iteration}

There are a number of clauses for iterating over sequences.  In all of
them, the argument following |for| may be a list instead of a
symbol, in which case destructuring is performed.  See section
\ref{destructuring}. 

\begin{clauses}

\clause{|for| ~var~ |in| ~list~ \opt\ |by| ~step-function~}
\clindexx{for}{in}
~var~ is set to successive elements of list.  
~step-function,~ which 
defaults to |cdr|, is used to obtain the next sublist. 

\clause{|for| ~var~ |on| ~list~ \opt\ |by| ~step-function~}
\clindexx{for}{on}
~var~ is set to successive sublists of list. 
~step-function~ (default |cdr|) is used as in |for\dots in|.

\end{clauses}

\medskip

These two clauses use |atom| to test for the end
of a list.  Hence, given a list whose final |cdr| is not \nil,
they will silently ignore the last |cdr|.  Other choices are
|endp|,  which would signal an error, and |null|, which
would probably result in an error somewhere else.  If you wish to use
an end-test other than |atom|, set the variable
|iterate::*list-end-test*|\lispindex{*list-end-test*} to the name of
the desired function. 

\begin{clauses}
\clause{|for| ~var~ |in-vector| ~vector~ |\&sequence|}
\clindexx{for}{in-vector}
~var~ takes on successive elements from ~vector.~   The vector's
fill-pointer is observed.  Here and in subsequent clauses, the |\&sequence|
keywords include |with-index|, which takes a symbol as argument
and uses it for the index variable instead of an internally generated
symbol.  The other |\&sequence| keywords behave as in numerical
iteration, except that the default iteration bounds are the bounds of
the vector.  E.g. in |(for i in-vector v downto 3)|,
|i| will start off being bound to the last element in |v|,
and will  
be set to preceding elements down to and including the element with
index 3.

\clause{|for| ~var~ |in-sequence| ~seq~ |\&sequence|}
\clindexx{for}{in-sequence}
This uses Common Lisp's generalized sequence functions, |elt|
and |length|, to obtain elements and determine the length of
~seq.~   Hence it will work for any sequence, including lists, and will
observe the fill-pointers of vectors.

\clause{|for| ~var~ |in-string| ~string~ |\&sequence|}
\clindexx{for}{in-string}
~var~ is set to successive characters of ~string.~

\startitem
\Clause{|for| ~var~ |index-of-vector| ~vector~ |\&sequence|}
\clindexx{for}{index-of-vector}
\Clause{|for| ~var~ |index-of-sequence| ~sequence~ |\&sequence|}
\clindexx{for}{index-of-sequence}
\Clause{|for| ~var~ |index-of-string| ~string~ |\&sequence|}
\clindexx{for}{index-of-string}
\finishitem
~var~ is set to successive indices of the sequence.
These clauses avoid the overhead of accessing the sequence elements
for those applications where they do not need to be examined, or are
examined rarely.  They admit all the optional keywords of the other
sequence drivers except the (redundant) |with-index| keyword.

\clause{|for| ~(key value)~ |in-hashtable| ~table~}
\clindexx{for}{in-hashtable}

~key~ and ~value,~ which must appear as shown in a list
and may be destructuring templates, are set to the keys and values of
~table~.  If ~key~ is |nil|, then the hashtable's keys will be ignored;
similarly for ~value~.
The order in which 
elements of ~table~ will be retrieved is unpredictable.

\clause{|for| ~var~ |in-package| ~package~ \opt\ |external-only| ~ext~}
\clindexx{for}{in-package}
Iterates over all the symbols in ~package,~ or over only the
external symbols if ~ext~ is specified and non-|nil|.  ~ext~ is not
evaluated.  The same symbol may appear more than once.

\clause{|for| ~var~ |in-packages| ~packages~ \opt\ |having-access| ~symbol-types~}
\clindexx{for}{in-packages}
Iterates over all the symbols from the list of packages denoted by the
descriptor ~packages~ and having accessibility (or visibility) given by
~symbol-types~.  This defaults to the list |(:external :internal :inherited)|
and is not evaluated.  ~var~ must be a list of up to three variables: in each
iteration, these will be set to a symbol, its access-type and package (as per
|with-package-iterator| in ANSI-CL).  The same symbol may appear more than
once.

\end{clauses}

\begin{clauses}

\clause{|for| ~var~ |in-file| ~name~ \opt\ |using| ~reader~}
\clindexx{for}{in-file}
Opens the file ~name~ (which may be a string or pathname) for input,
and iterates 
over its contents. ~reader~ defaults to |read|, so by default ~var~
will be bound to the successive forms in the file.  The \iter\ body is
wrapped in an unwind-protect to ensure that the file is closed no
matter how the \iter\ is exited.

\clause{|for| ~var~ |in-stream| ~stream~ \opt\ |using| ~reader~}
\clindexx{for}{in-stream}
Like |for\dots in-file|, except that ~stream~ should be an existing
stream object that supports input operations.

\end{clauses}

\subsubsection{Generalized Drivers}

These are primarily useful for writing drivers that can also be used
as generators (see section \ref{generators}, below).
 
\begin{clauses}

\clause{|for| ~var~ |next| ~expr~}
\clindexx{for}{next}

~var~ is set to ~expr~ each time through the loop.  Destructuring is
performed.  When the clause is used as a generator, ~expr~ is the code
that is executed when |(next ~var~)| is 
encountered (see section \ref{generators}, below).
~expr~ should compute the first value for ~var~, as well
as all subsequent values, and is responsible for terminating the loop.
For compatibility with future versions of \iter, this termination
should be done with |terminate|\clindex{terminate}, which can be
considered a synonym for |finish| (see section \ref{control-flow}).

\cpar As an example, the following clauses are equivalent to |(for
i from 1 to 10)|:
\begin{program}
(initially (setq i 0))
(for i next (if (> i 10) (terminate) (incf i)))
\end{program}

\clause{|for| ~var~ |do-next| ~form~}
\clindexx{for}{do-next}
~form~ is evaluated each time through the loop.  Its value is ~not~
set to ~var~; that is ~form's~ job. ~var~ is only present so that
\iter\ knows it is a driver variable.  \linebreak |(for ~var~ next
~expr~)| is equivalent to |(for ~var~ do-next (dsetq ~var~ ~expr~))|.
(See section \ref{destructuring} for an explanation of |dsetq|.) 

\end{clauses}

\subsubsection{Generators}
\label{generators}

In all of the above clauses, the driver variable is updated on each
iteration.
Sometimes it is desirable to have greater control over updating.  
For instance, consider the problem of associating numbers, in
increasing order and with no gaps, with the
\nonnil\ elements of a list.  One obvious first pass at writing this is:

\begin{program}
(iter (for el in list)
      (for i upfrom 1)
      (if el (collect (cons el i))))
\end{program}
But on the list |(a b nil c)| this produces |((a . 1) (b . 2) (c .
4))| instead of the desired |((a . 1) (b . 2) (c . 3))|.  The problem
is that |i| is incremented each time through the loop, even when |el|
is |nil|.  

The problem could be solved elegantly if we could step |i| only when
we wished to.  This
can be accomplished for any \iter\ driver by writing |generate| (or
its synonym |generating|)
instead of |for|.  Doing so produces a ~generator~---a driver whose
values are yielded explicitly.  To obtain the next value of a
generator variable ~v~, write \linebreak |(next ~v~)|.  The value of a |next|
form is the next value of ~v~, as determined by its associated driver
clause.  |next| also has the side-effect of updating ~v~ to that
value.  If there is no next value, |next| will terminate the loop,
just as with a normal driver.


Using generators, we can now write our example like this:
\begin{program}
(iter (for el in list)
      (generate i upfrom 1)
      (if el (collect (cons el (next i)))))
\end{program}
Now |i| is updated only when |(next i)| is executed, and this occurs
only when |el| is \nonnil.

To better understand the relationship between ordinary drivers and
generators, observe that we can rewrite an ordinary driver using its
generator form immediately followed by |next|, as this example shows:
\begin{program}
(iter (generating i from 1 to 10)
      (next i)
      ...)
\end{program}
Provided that the loop body contains no |(next i)| forms, this will
behave just as if we had written |(for i from 1 to 10)|.

We can still refer to a driver variable ~v~ without using |next|; in
this case, its value is that given to it by the last evaluation of
|(next ~v~)|.  Before |(next ~v~)| has been called the first time, the
value of ~v~ is undefined.  

This semantics is more flexible than
one in which ~v~ begins the loop bound to its first value and calls
of |next| supply subsequent values, because it means the loop will not
terminate too soon if the generator's sequence is empty.  For
instance, consider the following code, which tags \nonnil\ elements of
a list using a list of tags, and also counts the null elements.
(We assume there are at least as many tags as \nonnil\ elements.)
\begin{program}
(let* ((counter 0)
       (tagged-list (iter (for el in list)
                          (generating tag in tag-list) 
                          (if (null el)
                              (incf counter)
                              (collect (cons el (next tag)))))))
  ...)
\end{program}
It may be that there are just as many tags as non-null elements of
|list|.  If all the elements of |list| are null, we still want the
counting to proceed, even though |tag-list| is \nil.  If |tag| had to be
assigned its first value before the loop begins, we would have had to
terminate the loop before the first iteration, since when |tag-list|
is \nil, |tag| 
has no first value.  With the existing semantics, however, |(next
tag)| will never execute, so the iteration will cover all the elements of
|list|. 
        
When the ``variable'' of a driver clause is actually a destructuring
template containing several variables, all the variables are eligible
for use with |next|.  As before, |(next ~v~)| evaluates to ~v's~ next
value; but the effect is to update all of the template's variables.
For instance, the following code will return the list |(a 2 c)|.
\begin{program}
(iter (generating (key . item) in '((a . 1) (b . 2) (c . 3)))
      (collect (next key))
      (collect (next item)))
\end{program}

Only driver clauses with variables can be made into generators.  This
includes all clauses mentioned so far except for |repeat|.  It does
~not~ include |for\dots previous|, |for\dots =|,
|for\dots initially\dots then| or |for\dots first\dots then| (see
below). 

\subsubsection{Previous Values of Driver Variables}

Often one would like to access the value of a variable on a previous
iteration.  \iter\ provides a special clause for accomplishing this.

\begin{clauses}

\clause{|for| ~pvar~ |previous| ~var~ \opt\ |initially| ~init~
|back| ~n~}
\clindexx{for}{previous}
Sets ~pvar~ to the previous value of ~var,~ which should be a driver
variable, a variable from another |for\dots previous| clause, or a
variable established by a |for\dots =|, 
|for\dots initially\dots then| or |for\dots first\dots then| clause
(see section \ref{setting}).
Initially, ~pvar~ is given the value ~init~ (which defaults to \nil).  
The ~init~ expression will be moved outside the loop body, so it
should not depend on anything computed within the loop.
~pvar~ retains the value of ~init~ until ~var~ is set to its second
value, at which point ~pvar~ is set to ~var's~ first value; and so on.  

\cpar The
argument ~n~ to |back| must be a constant, positive integer, and
defaults to 1.  It determines how many iterations back ~pvar~ should
track ~var~.  For example, when ~n~ is 2, then ~pvar~ will be assigned
~var's~ first value when ~var~ is set to its third value.

\cpar A |for\dots previous| clause may occur after or before its
associated driver clause. |for\dots previous| works with generators as
well as ordinary drivers.

\pagebreak[3]

\cpar Example:
\begin{program}
(iter (for el in '(1 2 3 4))
      (for p-el previous el)
      (for pp-el previous p-el initially 0)
      (collect pp-el))
\end{program}
This evaluates to |(0 0 1 2)|.  It could have been written more
economically as 
\begin{program}
(iter (for el in '(1 2 3 4))
      (for pp-el previous el back 2 initially 0)
      (collect pp-el))
\end{program}


\end{clauses}

\subsection{Variable Binding and Setting}
\label{setting}

Several clauses exist for establishing new variable bindings or for
setting variables in the loop.  They all support destructuring.

\begin{clauses}

\clausex{with}{~var~ \opt\ |=| ~value~}
Causes ~var~ to be bound to value before the loop body is entered.
If ~value~ is not supplied, ~var~ assumes a default
binding, which will be 
\nil\ in the absence of declarations.  Also, if ~value~ is not
supplied, no destructuring is performed; instead, ~var~ may be a list of
symbols, all of which are given default 
bindings.  If ~value~ is supplied, ~var~ is bound to it, with
destructuring.  

\cpar Because |with|
creates bindings whose scope includes the entire \iter\ form, it is
good style to put all |with| clauses at the beginning.

\cpar Successive occurrences of |with| result in sequential
bindings (as with 
|let*|).  There is no way to obtain parallel bindings; see
section \ref{bindings} for a rationale.  


\clause{|for| ~var~ |=| ~expr~}
\clindexx{for}{=}
On each iteration, ~expr~ is evaluated and ~var~ is set
to its value. 

\cpar This clause may appear to do the same thing as |for\dots next|.
In fact, they are quite different.  |for\dots =| provides only three
services: it sets up a binding for ~var~, sets it to ~expr~ on each
iteration, and makes it possible to use |for\dots previous| with
~var~.  |for\dots next| provides these services in addition to the
ability to turn the driver into a generator.
%Also, the code which sets ~var~ appears in the loop body in the same
%place as the |for\dots =| clause; the code for |for\dots next| appears
%at the top of the loop, as with other drivers (except when being used
%as a generator).

\clause{|for| ~var~ |initially| ~init-expr~ |then| ~then-expr~}
\clindexxx{for}{initially}{then}
Before the loop begins, ~var~ is set to ~init-expr;~ on all
iterations after the first it is set to ~then-expr.~  This clause
must occur at top-level.  ~init-expr~ will be moved outside the loop
body and ~then-expr~ will be moved to the end of the loop body, so
they are subject to code motion problems (see section
\ref{code-movement}). 

\cpar This clause may appear to be similar to |for\dots next|, but in
fact they differ significantly.  |for\dots initially\dots then| is
typically used to give ~var~ its first value before the loop begins,
and subsequent values on following iterations.  This is incompatible
with generators, whose first value and subsequent values must all be
computed by |(next ~var~)|.  Also, the update of ~var~ in 
|for\dots initially\dots then| does not occur at the location of the
clause. 
Use |for\dots initially\dots then| for
one-shot computations where its idiom is more convenient, but use
|for\dots next| for extending \iter\ with new drivers (see section
\ref{extend}).

\clause{|for| ~var~ |first| ~first-expr~ |then| ~then-expr~}
\clindexxx{for}{first}{then}
The first time through the loop, ~var~ is set to ~first-expr;~ on
subsequent iterations, it is set to ~then-expr.~  This differs from
|for\dots initially| in that ~var~ is set to ~first-expr~
inside the loop body, 
so ~first-expr~ may depend on the results of other clauses.  For
instance, 
\begin{program}
(iter (for num in list)
      (for i first num then (1+ i))
      ...)
\end{program}
will set |i| to the first element of |list| on the first
iteration, whereas 
\begin{program}
(iter (for num in list)
      (for i initially num then (1+ i))
      ...)
\end{program}
is probably erroneous; |i| will be bound to |num|'s
default binding (usually \nil) for the first iteration.

\end{clauses}

\begin{note}{Compatibility note}
|loop|'s |for\dots =| works like \iter's, but |loop| used the syntax
|for\dots =\dots then| to mean |for\dots initially\dots then|.  It was
felt that these two operations were sufficiently different to warrant
different keywords.

Also, the |for| in the above three clauses is misleading,
since none is true driver (e.g. none has a corresponding |generate|
form).  |setting| would have been a better choice, but |for| was used
to retain some compatibility with |loop|.  
\end{note}


%%%%%%%%%%%%%%%%%%%%%%%%%%%%%%%%
\subsection{Gathering Clauses}
Many of \iter's clauses accumulate values into a variable, or set a
variable under certain conditions.  At the end of the loop, this
variable contains the desired result.  All these clauses have an
optional |into| keyword, whose argument should be a symbol.  If the
|into| keyword is not supplied, the accumulation variable will be
internally generated and its value will be returned at the end of the
loop; if a variable is specified, that variable is used for the
accumulation, and is not returned as a result---it is up to the user
to return it explicitly, in the loop's epilogue code (see |finally|).
It is safe to examine the accumulation variable during the loop, but
it should not be modified.

These clauses all begin with a verb.  When the verb does
not conflict with an existing Common Lisp function, then it may be
used in either its infinitival or present-participle form (e.g. |sum|,
|summing|).  However, when there is a conflict with Common Lisp, only
the present-participle form may be used (e.g. |unioning|).  This is to
prevent \iter\ clauses from clashing with Common Lisp
functions. 

%although these clauses are described as ``producing a value,'' it is a
%mistake to think of the lisp list representing the clause as a
%value-producing form in the usual way.  clauses may legally be written
%where a value is expected, e.g. |(setq x (sum i))|, but the
%lisp value of a clause in such a context is undefined.

\subsubsection{Reductions}

~Reduction~ is an extremely common iteration pattern in which
the results of successive applications of a
binary operation are accumulated.
For example, a loop that computes the sum of the
elements of a list is performing a reduction with the addition
operation.  This could be written in Common Lisp as  |(reduce \#'+
list)| or with \iter\ as
\begin{program}
(iter (for el in list)
      (sum el))
\end{program}

\begin{clauses}

\clausex{sum}{~expr~ \opt\ |into| ~var~}
Each time through the loop, ~expr~ is evaluated and added to a
variable, which is bound initially to zero.  If ~expr~ has a type,
it is ~not~ used as the type of the sum variable, which is always
|number|.  To get the result variable to be of a more specific
type, use an explicit variable, as in
\begin{program}
(iter (for el in number-list)
      (sum el into x)
      (declare (fixnum x))
      (finally (return x)))
\end{program}

\clausex{multiply}{~expr~ \opt\ |into| ~var~}
Like |sum|, but the initial value of the result variable is $1$,
and the variable is updated by multiplying ~expr~ into it.

\clausex{counting}{~expr~ \opt\ |into| ~var~}
~expr~ is evaluated on each iteration.  If it is \nonnil, the
accumulation variable, initially zero, is incremented.

\startitem
\Clausex{maximize}{~expr~ \opt\ |into| ~var~}
\Clausex{minimize}{~expr~ \opt\ |into| ~var~}
\finishitem
~expr~ is evaluated on each iteration and its extremum (maximum or
minimum) is stored in the accumulation variable.  If ~expr~ is never
evaluated, then the result is \nil\ (if the accumulation variable is
untyped) or $0$ (if it has a numeric type).

\clausex{reducing}{~expr~ |by| ~func~ \opt\ |initial-value| ~init-val~ |into| ~var~}
This is a general way to perform reductions. ~func~ should be a
function of two arguments, the first of which will be the value
computed so far and
the second of which will be the value of ~expr~.  It should return the
new value.
|reducing| is roughly equivalent to the
Common Lisp |(reduce ~func~ ~sequence~ :key ~expr-function~)|, where
~expr-function~ is used to derive values from the successive elements
of ~sequence~.

\cpar If the |reducing| clause is never executed, the result is
undefined.

\cpar It is not necessary to provide an
initial value, but better code can be generated if one is supplied.
Regardless of its location in the \iter\ body, ~init-val~ will be
evaluated before the loop is entered, so it should not depend on any
value computed inside the \iter\ form.  

%\cpar if a ~var~ is not specified, you can get \iter\ to declare the
%type of the internal variable by putting a |the| expression around
%~func.~  see section \ref{types}.

\end{clauses}

%\begin{note}{implementation note}
%in principle, |maximize| and |minimize| can be thought of
%as reductions where the initial value is the smallest (or largest)
%value that the accumulation variable can assume.  because lisp's
%bignums can represent arbitrary integers, these clauses cannot be
%implemented as reductions in general.  if, however, the type of 
%~expr~ or ~var~ can be determined to be a fixnum
%or a float, \iter\ will implement the clause as a true reduction,
%using one of the constants |most-negative-fixnum|, |%most-positive-fixnum|, 
%|most-negative-short-float|, etc. as appropriate.
%\end{note}

\subsubsection{Accumulations}

All the predefined accumulation clauses add values to a sequence.  If
the sequence is a list, they all 
have the property that the partial list is kept in the correct order
and available for inspection at any point in the loop.

\begin{clauses}

\clausex{collect}{~expr~ \opt\ |into| ~var~ |at| ~place~ |result-type| ~type~}
Produces a sequence of the values of ~expr~ on each iteration. ~place~
indicates where the next value of ~expr~ is added to the list and may
be one of the symbols |start|, |beginning| (a synonym for |start|) or
|end|.  The symbol may be quoted, but need not be.  The default is
|end|.  For example,
\begin{program}
(iter (for i from 1 to 5)
      (collect i))
\end{program}
produces |(1 2 3 4 5)|, whereas
\begin{program}
(iter (for i from 1 to 5)
      (collect i at beginning))
\end{program}
produces |(5 4 3 2 1)| (and is likely to be faster in most Common Lisp
implementations).

\cpar If ~type~ is provided, it should be a subtype of |sequence|.
The default is |list|.  Specifying a type other than |list| will
result in |collect| returning a sequence of that type.  ~However,~ the
type of the sequence being constructed when inside the loop body is
undefined when a non-|list| type is specified.  (As with ~place~,
quoting ~type~ is optional.)

\clausex{adjoining}{~expr~ \opt\ |into| ~var~ |test| ~test~ |at| ~place~ |result-type| ~type~}
Like |collect|, but only adds the value of ~expr~ if it is not
already present.  ~test,~ which defaults to |\#'eql|, is
the test to be used with |member|.

\startitem
\Clausex{appending}{~expr~ \opt\ |into| ~var~ |at| ~place~}
\Clausex{nconcing}{\ ~expr~ \opt\ |into| ~var~ |at| ~place~}
\Clausex{unioning}{\ ~expr~ \opt\ |into| ~var~ |test| ~test~ |at| ~place~}
\Clausex{nunioning}{~expr~ \opt\ |into| ~var~ |test| ~test~ |at| ~place~}
\finishitem
These are like |collect|, but behave like the Common Lisp functions
|append|, |nconc|, |union| or |nunion|.  
As in Common Lisp, they work only on lists.  Also as in Common Lisp,
|unioning| and |nunioning| assume that the value of ~expr~ contains no
duplicates. 

\clausex{accumulate}{~expr~ |by| ~func~ \opt\ |initial-value| ~init-val~ |into| ~var~}
This is a general-purpose accumulation clause. ~func~ should be a
function of two arguments, the value of ~expr~ and the value
accumulated so far in the iteration, and it should return the updated
value.  If no initial value is supplied, \nil\ is used.  

%\cpar If a ~var~ is not specified, you can get \iter\ to declare the
%type of the internal variable by putting a |the| expression around
%~func.~  see section \ref{types}.

\cpar The differences between |accumulate| and |reducing| are slight.
One difference is that the functions take their arguments in a
different order.  Another is that in the absence of ~init-val~,
|accumulate| will use \nil, whereas |reducing| will generate different
code that avoids any dependence on the initial value.
The reason for having both clauses is that one usually
thinks of reductions (like |sum|) and accumulations (like |collect|)
as different beasts.

\end{clauses}


\subsubsection{Finders}

A ~finder~ is a clause whose value is an expression that meets some
condition.

\begin{clauses}

\clause{|finding| ~expr~ |such-that| ~test~ \opt\ |into| ~var~ |on-failure| ~failure-value~}
\clindexx{finding}{such-that}
If ~test~ (which is an expression) ever evaluates to \nonnil, the loop
is terminated, the 
epilogue code is run and the value of ~expr~ is returned.  Otherwise,
\nil\ (or ~failure-value,~ if provided) is returned.  If ~var~ is
provided, it will have either the \nonnil\ value of ~expr~ or
~failure-value~ when the epilogue code is run.

\cpar As a special case, if the ~test~ expression is a sharp-quoted
function, then it is applied to ~expr~ instead of being simply
evaluated.  E.g. |(finding x such-that \#'evenp)| is equivalent to
|(finding x such-that (evenp x))|.  

%\cpar although ~test~ need have nothing to do with ~%expr~ as in
%|(finding j such-that (> i 3))|, it usually 
%will: |(finding (length el) such-that (oddp (length el)))|.  to
%avoid performing the |length| computation twice, you could write
%|(finding (length el) such-that \#'oddp)| or |(finding (length
%el) such-that 'oddp)|; for these cases, \iter\ generates code that
%executes ~expr~ only once.  the code for |\#'oddp|
%is slightly different from that for {\lisp 'oddp}; see the discussion
%under {\lisp for\dots in} and {\lisp for\dots on}.

\cpar |On-failure| is a misnomer. Because it is always evaluated, it behaves
more like the default third argument to the |gethash| function. As a result,
|on-failure (error "Not found")| makes no sense. Instead, the clauses |leave|
or |thereis| can be used in conjunction with |finally| as follows:
\begin{program}
(iter (for x in '(1 2 3))
      (if (evenp x) (leave x))
      (finally (error "not found")))
\end{program}

\cpar This clause may appear multiple times when all defaults are
identical. It can also be used together with either |always|/|never| or
|thereis| if their defaults match. More specifically, |on-failure nil| is
compatible with |thereis|, while |on-failure t| is compatible with |always|
and |never| clauses.
\begin{program}
(iter (for i in '(7 -4 2 -3))
      (if (plusp i)
	  (finding    i  such-that (evenp i))
          (finding (- i) such-that (oddp i))))
\end{program}

\startitem
\Clause{|finding| ~expr~ |maximizing| ~m-expr~ \opt\ |into| ~var~}
\clindexx{finding}{maximizing}
\Clause{|finding| ~expr~ |minimizing| ~m-expr~ \opt\ |into| ~var~}
\clindexx{finding}{minimizing}
\finishitem
Computes the extremum (maximum or minimum) value of ~m-expr~ over all
iterations, and returns the value of ~expr~ corresponding to the
extremum.  ~expr~ is evaluated inside the loop at the time the new
extremum is established.  If ~m-expr~ is never evaluated (due to, for
example, being embedded in a conditional clause), then the returned
value depends on the type, if any, of ~expr~ (or ~var,~ if
one is supplied).  If there is no type, the returned
value will be \nil; if the type is numeric, the returned value will be
zero.

\cpar For these two clauses, ~var~ may be a list of two
symbols; in that case, the first is used to record ~expr~ and
the second, ~m-expr.~  

\cpar As with |finding\dots such-that|, if ~m-expr~ is a sharp-quoted
function, then it is called on ~expr~ instead of being evaluated.

\end{clauses}

\subsubsection{Boolean Tests}

\begin{clauses}

\clausex{first-iteration-p}{}
Returns |t| in the first cycle of the loop, otherwise \nil.

\clausex{first-time-p}{}
Returns |t| the first time the expression is evaluated, and then \nil\ forever.
This clause comes handy when printing (optional) elements separated
by a comma:

\begin{program}
(iter (for el in '(nil 1 2 nil 3))
      (when el
        (unless (first-time-p)
          (princ ", "))
        (princ el)))
\end{program}
produces |"1, 2, 3"|.

\end{clauses}

\subsubsection{Aggregated Boolean Tests}

\begin{clauses}

\clausex{always}{~expr~}
If ~expr~ ever evaluates to
\nil, then \nil\ is immediately returned; the epilogue code is not
executed.  If ~expr~ never evaluates to \nil, the epilogue code
is executed and the last value of ~expr~ (or |t| if ~expr~ was never
evaluated) is returned (whereas |loop| would constantly return |t|).

% mention last evaluated clause when multiple always clauses?

\clausex{never}{~expr~}
Like |(always (not ~expr~))|, except it does not influence the last
value returned by a possible other |always| clause. That is,
\begin{program}
(iter (repeat 2)
      (always 2)
      (never nil)) \yields 2 ; not t
\end{program}

\clausex{thereis}{~expr~}
If ~expr~ is ever \nonnil,
its value is immediately returned without running epilogue code.
Otherwise, the epilogue code is performed and \nil\ is returned.

This clause cannot be used together with |always| or |never|, because their
defaults are opposed (similarly, |(loop always 3 thereis nil)| refuses to
compile in some implementations of |loop|).

\end{clauses}


\subsection{Control Flow}
\label{control-flow}
Several clauses can be used to alter the usual flow of control in a loop.

Note: the clauses of this and subsequent sections don't adhere to \iter's
usual syntax, but instead use standard Common Lisp syntax.  Hence the
format for describing syntax subsequently is
like the standard format used in the Common Lisp manual, not like the
descriptions of clauses above.

\begin{clauses}

\clausex{finish}{}
Stops the loop and runs the epilogue code.

%for example:
%
%\begin{program}
%(iter (with answer = nil)
%         (initially (make-a-mess))
%         (for i from 1 to 10)
%         (when (correct? i)
%           (setq answer i)
%           (finish))
%         (finally (cleanup)))
%\end{program}
%
%this code will execute |cleanup| whether or not the test |(correct?
%i)| ever succeeds. 
%the (more elegant) formulation,
%\begin{program}
%(iter (initially (make-a-mess))
%         (for i from 1 to 10)
%         (finding i such-that (correct? i))
%         (finally (cleanup)))
%\end{program}
%would not execute |cleanup| if |(correct? i)| succeeded; it
%would do an immediate return.

\clausex{leave}{\opt\ ~value~}
Immediately returns ~value~ (default \nil) from the current \iter\
form, skipping the epilogue code.  Equivalent to using |return-from|.

\clausex{next-iteration}{}
Skips the remainder of the loop body and begins the next iteration of
the loop.

\clausex{while}{~expr~}
If ~expr~ ever evaluates to \nil, the loop is terminated and the
epilogue code executed.  Equivalent to |(if (not ~expr~) (finish))|.

\clausex{until}{~expr~}
Equivalent to |(if ~expr~ (finish))|.

\clausex{if-first-time}{~then~ \opt\ ~else~}
If this clause is being executed for the first time in this invocation
of the \iter\ form, then the ~then~ code is evaluated; otherwise the
~else~ code is evaluated.

\cpar |(for ~var~ first ~expr1~ then ~expr2~)| is almost equivalent to
\begin{program}
(if-first-time (dsetq ~var~ ~expr1~)
               (dsetq ~var~ ~expr2~))
\end{program}
The only difference is that the |for| version makes ~var~ available
for use with |for\dots previous|.

\end{clauses}

\subsection{Code Placement}
When fine control is desired over where code appears in a loop
generated by \iter, the following special clauses may be useful.
They are all subject to code-motion problems (see section
\ref{code-movement}).  

\begin{clauses}

\clausex{initially}{|\&rest| ~forms~}
The lisp ~forms~ are placed in the prologue section of the loop, where
they are executed once, before the loop body is entered.

\clausex{after-each}{|\&rest| ~forms~}
The ~forms~ are placed at the end of the loop body, where they
are executed after each iteration.  Unlike the other clauses in this
section, ~forms~ may contain \iter\ clauses.

\clausex{else}{|\&rest| ~forms~}
The lisp ~forms~ are placed in the epilogue section of the loop, where they
are executed if this |else| clause is never met during execution of the
loop and the loop terminates normally.

\clausex{finally}{|\&rest| ~forms~}
The lisp ~forms~ are placed in the epilogue section of the loop, where
they are executed after the loop has terminated normally.

\clausex{finally-protected}{|\&rest| ~forms~}
The lisp ~forms~ are placed in the second form of an unwind-protect
outside the loop.  They are always executed after the loop has
terminated, regardless of how the termination occurred.

\end{clauses}


%%%%%%%%%%%%%%%%%%%%%%%%%%%%%%%%%%%%%%%%%%%%%%%%%%%%%%%%%%%%%%%%
\section{Other Features}

\subsection{Multiple Accumulations}
\label{multiple}

\begin{sloppypar}
It is permitted to have more than one clause accumulate into the same
variable, as in the following:
\begin{program}
(iter (for i from 1 to 10)
      (collect i into nums)
      (collect (sqrt i) into nums)
      (finally (return nums)))
\end{program}
Clauses can only accumulate into the same variable if they are
compatible.  |collect|, |adjoining|, |appending|, |nconcing|,
|unioning| and |nunioning| are compatible with each other; |sum|,
|multiply| and |counting| are compatible; |always| and |never| are compatible;
|finding| \dots |such-that| is compatible with either |thereis| or |always|
and |never| when their defaults match; and |maximize| and |minimize| clauses
are compatible only with other |maximize| and |minimize| clauses,
respectively.

%note that the same variable ~cannot~ be both an accumulation
%variable and an ordinary variable; there can be only one variable with
%a given name within an \iter\ form. 
\end{sloppypar}

\subsection{Named Blocks}

Like Common Lisp |block|s, \iter\ forms can be given names.  The
name should be a single symbol, and it must be the first form in the
\iter.  The generated code behaves exactly like a named block; in
particular, |(return-from ~name~)| can be used to exit it:
\begin{program}
(iter fred
      (for i from 1 to 10)
      (iter barney
            (for j from i to 10)
            (if (> (* i j) 17)
                (return-from fred j))))
\end{program}
An \iter\ form that is not given a name is implicitly named \nil.

Sometimes one would like to write an expression in an inner \iter\ form,
but have it processed by an outer \iter\ form.  This is possible with
the |in| clause.  

\begin{clauses}

\clausex{in}{~name~ |\&rest| ~forms~}
Evaluates ~forms~ as if they were part of the \iter\ form named
~name~.  In other words, \iter\ clauses are processed by the \iter\
form named ~name,~ and not by any \iter\ forms that occur inside ~name.~

\cpar As an example, consider the problem of collecting a list of the
elements in a two-dimensional array.  The naive solution,
\begin{program}
(iter (for i below (array-dimension ar 0))
      (iter (for j below (array-dimension ar 1))
            (collect (aref ar i j))))
\end{program}
\noindent is wrong because the list created by the inner \iter\ is simply
ignored by the outer one.  But using |in| we can write:
\begin{program}
(iter outer (for i below (array-dimension ar 0))
      (iter (for j below (array-dimension ar 1))
            (in outer (collect (aref ar i j)))))
\end{program}
\noindent which has the desired result.

\end{clauses}

\subsection{Destructuring}
\label{destructuring}

In many places within \iter\ clauses where a variable is expected, a
list can be written instead.  In these cases, the value to be assigned
is ~destructured~ according to the pattern
described by the list.  As a simple example, the clause
\begin{program}
(for (key . item) in alist)
\end{program}
\noindent will result in |key| being set to the |car| of
each element in |alist|, and |item| being set to the |cdr|.  The
pattern list may be nested to arbitrary depth, and (as the example
shows) need not be terminated with \nil; the only requirement is that
each leaf be a bindable symbol (or \nil, in which case no binding is
generated for that piece of the structure).

Sometimes, you might like to do the equivalent of a
|multiple-value-setq| in a clause.  This 
``multiple-value destructuring'' can be expressed by writing 
\linebreak |(values $pat_1$ $pat_2 \ldots$)| for a destructuring
pattern, as in 
\begin{program}
(for (values (a . b) c d) = (three-valued-function ...))
\end{program}
\begin{sloppypar}
Note that the $pat_i$ can themselves be destructuring patterns (though
not multiple-value destructuring patterns).  You can't do multiple-value
destructuring in a |with| clause; instead wrap the whole \iter\
form in a |multiple-value-bind|.
\end{sloppypar}

\begin{note}{Rationale}
There are subtle interactions between variable declarations and
evaluation order that make the correct implementation of
multiple-value destructuring in a |with| somewhat tricky.
\end{note}

The destructuring feature of \iter\ is available as a separate
mechanism, using the |dsetq| macro:

\begin{clauses}
\defmacro{dsetq}{template expr}
Performs destructuring of ~expr~ using ~template~.   May be used
outside of an \iter\ form. Yields the primary value of ~expr~.

\end{clauses}

\subsection{On-line Help}

\begin{sloppypar}
There is a limited facility for on-line help, in the form of the 
|display-iterate-clauses| function.
\end{sloppypar}

\begin{clauses}

\defun{display-iterate-clauses}{\opt\ clause-spec}
Displays a list of \iter\ clauses.  If ~clause-spec~ is not
provided, all clauses are shown; if it is a symbol, all clauses
beginning with that symbol are shown; and if it is a list of symbols,
all clauses for which ~clause-spec~ is a prefix are shown.

\end{clauses}

\subsection{Parallel Binding and Stepping}
\label{bindings}

The parallel binding and stepping of variables is a feature that
\iter\ does ~not~ have.  This section attempts to provide a rationale.

We say that two variables are bound ~in parallel~ if neither
binding shadows the other.  This is the usual semantics of |let|
(as opposed to |let*|).  Similarly, we can say that iteration
variables are stepped in parallel if neither variable is updated
before the other, conceptually speaking; in other words, if the code
to update each variable can reference the old values of both variables.

|loop| allows parallel binding of variables and parallel stepping of
driver variables.  My view is that if you are depending on the 
serial/parallel distinction, you are doing something obscure.  If you
need to bind 
variables in parallel using |with|, then you must be using a
variable name that shadows a name in the existing lexical environment.
Don't do that.  The most common use for parallel stepping is to track
the values of variables on the previous iteration, but in fact this
does not require parallel stepping at all; the following will work:
\begin{program}
(iter (for current in list)
      (for prev previous current)
      ...)
\end{program}


%%%%%%%%%%%%%%%%%%%%%%%%%%%%%%%%%%%%%%%%%%%%%%%%%%%%%%%%%%%%%%%%
\section{Types and Declarations}
\label{types}

\subsection{Discussion}

Sometimes efficiency dictates that the types of variables be declared.
This type information needs to be communicated to \iter\ so it can
bind variables to appropriate values.  Furthermore, \iter\ must often
generate internal variables invisible to the user; there needs to be a
way for these to be declared.

As an example, consider this code, which will return the number of
odd elements in |number-list|:
\begin{program}
(iter (for el in number-list)
      (count (oddp el)))
\end{program}
In processing this form,
\iter\ will create an internal variable, let us call it |list17|, to
hold the successive |cdr|s of |number-list|, and  
will bind the variable to |number-list|.  It will also generate a
default binding for |el|; only inside the body of the loop will |el|
be set to the |car| of |list17|.  Finally, \iter\ will generate a
variable, call it |result|, to hold the result of the count, and will
bind it to zero.

When dealing with type declarations, \iter\ observes one simple rule:
~it will never generate a declaration unless requested to do so.~  The
reason is that such declarations might mask errors in compiled code by
avoiding error-checks; the resulting problems would be doubly hard to
track down because the declarations would be hidden from the
programmer.  Of course, a compiler might omit error-checks even in the
absence of declarations, though this behavior can usually be avoided,
e.g. by saying |(declaim (optimize (safety 3)))|.

So, the above \iter\ form will generate code with no declarations.
But say we wish to declare the types of |el| and the internal
variables |list17| and |result|.  How is this done?

Declaring the type of |el| is easy, since the programmer knows
the variable's name:
\begin{program}
(iter (for el in number-list)
      (declare (fixnum el))
      (counting (oddp el)))
\end{program}
\iter\ can read variable type declarations like this one.  Before
processing any clauses, it scans the entire top-level form for type
declarations 
and records the types, so that variable bindings can be performed
correctly.  In this case, |el| will be bound to zero
instead of \nil.  Also, \iter\ collects all the top-level declarations
and puts them at the begining of the generated code, so it is not
necessary to place all declarations at the beginning of an \iter\
form; instead, they can be written near the variables whose types they
declare.  

Since \iter\ is not part of the compiler, it will not know
about declarations that occur outside an \iter\ form; these
declarations must be repeated inside the form.

Here is another way we could have declared the type of |el|:
\begin{program}
(iter (for (the fixnum el) in number-list)
      (counting (oddp el)))
\end{program}
\iter\ extends the Common Lisp |the|\lispindex{the}
form to apply to variables as well as value-producing forms; anywhere
a variable is allowed---in a |with| clause, as the iteration
variable in a driver clause, as the |into| argument of an
accumulation clause, even inside a destructuring template---you can
write |(the ~type~ ~symbol~)| instead.

There is one crucial difference between using a |the| form and
actually declaring the variable: explicit declarations are always
placed in the generated code, but type information from a |the|
form is not turned into an actual declaration unless you tell \iter\
to do so using |iterate:declare-variables|.  See below.

\begin{sloppypar}
Declaring the types of internal variables is harder than declaring the
types of explicitly mentioned variables, since their names
are unknown.  You do it by declaring |iterate:declare-variables|
somewhere inside the top level of the \iter\ form.  (This will also
generate declarations for variables declared using |the|.)
\iter\ does not provide much selectivity here: it's all or none. 
And unfortunately, since \iter\ is not privy to compiler information
but instead reads declarations itself, it will not hear if you
|(declaim (iterate:declare-variables))|.  Instead, set the variable 
|iterate::*always-declare-variables*| to |t| at
compile-time, using |eval-when|.
\end{sloppypar}

To determine the appropriate types for internal variables, \iter\ uses
three sources of information:

\begin{itemize}

\item Often, the particular clause dictates a certain type for a
variable; \iter\ will use this information when available.  In the
current example, the variable |list17| will be given the type
|list|, since that is the only type that makes sense; and the
variable |result| will be given the type |fixnum|, on the
assumption that you will not be counting high enough to need bignums.
You can override this assumption only by using and explicitly declaring a
variable: 
\begin{verbatim}
(iter (declare (iterate:declare-variables))
      (for el in number-list)
      (count (oddp el) into my-result)
      (declare (integer my-result))
      (finally (return my-result)))
\end{verbatim}

\begin{sloppypar}
Other examples of the type assumptions that \iter\ makes are: type
|list| for |into| variables of collection clauses; type |list| for
expressions that are to be destructured; type |vector| for the
variable holding the vector in a |for\dots in-vector| clause, and
similarly for |string| and the |for\dots in-string| clause;
and the implementation-dependent type
|(type-of array-dimension-limit)| for the index and limit
variables generated by sequence iteration drivers like |for\dots
in-vector| and |for\dots in-string| (but not |for\dots in-sequence|,
because it may be used to iterate over a list). 
\end{sloppypar}

\item Sometimes, \iter\ will examine expressions and try to determine
their types in a simple-minded way.  If the expression is
self-evaluating (like a number, for instance), \iter\ knows that the
expression's type is the same as the type of the value it denotes, so
it can use that type.  If the expression is of the form |(the ~type~
~expr~)|, \iter\ is smart enough to extract ~type~ and use it.
However, the current version of \iter\ does not  
examine declarations of function result types or do any type
inference.  It will not determine, for 
example, that the type of |(+ 3 4)| is |fixnum|, or even
|number|.

\item In some cases, the type of an internal variable should match the
type of some other variable.  For instance, \iter\ generates an
internal variable for |(f x)| in the
clause |(for i from 1 to (f x))|, and in the absence of other
information will give it the same type as |i|.  If, however, the
expression had been written |(the fixnum (f x))|, then \iter\
would have given the internal variable the type |fixnum|
regardless of |i|'s type.  The type incompatibility errors that
could arise in this situation are not checked for.

\end{itemize}

Note that if you do declare |iterate:declare-variables|, then
\iter\ may declare user variables as well as internal ones if they do
not already have declarations, though only for variables that it
binds.  For instance, in this code:

\begin{program}
(iter (declare (iterate:declare-variables))
      (for i from 1 to 10)
      (collect i into var))
\end{program}
the variable |var| will be declared to be of type |list|.


\subsection{Summary}

\iter\ understands standard Common Lisp variable type declarations
that occur within an \iter\ form and
will pass them through to the generated code.  If the declaration
|(iterate:declare-variables)|\lispindex{declare-variables}
appears at the top level of an 
\iter\ form, or if 
|iterate::*always-declare-variables*|\lispindex{*always-declare-variables*} 
is \nonnil, then \iter\ will use the type information gleaned from user
declarations, self-evaluating expressions and |the| expressions,
combined with reasonable assumptions, to determine variable
types and declare them.


%%%%%%%%%%%%%%%%%%%%%%%%%%%%%%%%%%%%%%%%%%%%%%%%%%%%%%%%%%%%%%%%
\section{Problems with Code Movement}
\label{code-movement}

Some \iter\ clauses, or parts of clauses, result in code being
moved from the location of the clause to other parts of the loop.
Drivers behave this way, as do code-placement clauses like |initially|
and |finally|.  When using these clauses, there is a danger of writing
an expression that makes sense in its apparent location but will be
invalid or have a different meaning in another location.  For example:
\begin{program}
(iter (for i from 1 to 10)
      (let ((x 3))
        (initially (setq x 4))))
\end{program}
While it may appear that the |x| of |(initially (setq x 4))| is the
same as the |x| of |(let ((x 3)) \dots|, in fact they are not:
|initially| moves its code outside the loop body, so |x| would refer
to a global variable.  Here is another example of the same problem:
\begin{program}
(iter (for i from 1 to 10)
      (let ((x 3))
        (collect i into x)))
\end{program}
If this code were executed, |collect| would create a binding for its
|x| at the top level of the \iter\ form that the |let| will shadow.

Happily, \iter\ is smart enough to catch these errors; it
walks all problematical code to ensure that free variables are not
bound inside the loop body, and checks all variables it binds for the
same problem.

However, some errors cannot be caught:

\begin{program}
(iter (with x = 3)
      (for el in list)
      (setq x 1)
      (reducing el by \#'+ initial-value x))
\end{program}
|reducing| moves its |initial-value| argument to the initialization
part of the loop in order to produce more efficient code.  Since
\iter\ does not perform data-flow analysis, it cannot determine that
|x| is changed inside the loop; all it can establish is that |x| is
not bound internally.  Hence this code will not signal an
error and will use $3$ as the initial value of the reduction.

The following list summarizes all cases that are subject to these code
motion and variable-shadowing problems.
\begin{itemize}
\item Any variable for which \iter\ creates a binding, including those
used in |with| and the |into| keyword of many clauses.

\begin{sloppypar}
\item The special clauses which place code: |initially|, |after-each|, |else|,
|finally| and |finally-protected|. 
\end{sloppypar}

\item The variables of a |next| or |do-next| form.

\item The |initially| arguments of |for\dots initially\dots then| and
|for\dots previous|. 

\item The |then| argument of |for\dots initially\dots then|.

\item The |initial-value| arguments of |reducing| and |accumulate|.

\item The |on-failure| argument of |finding\dots such-that|.
\end{itemize}

%%%%%%%%%%%%%%%%%%%%%%%%%%%%%%%%%%%%%%%%%%%%%%%%%%%%%%%%%%%%%%%%
\section{Differences Between {\tt Iterate} and {\tt Loop}}
|loop| contains a great deal of complexity which \iter\ tries to
avoid.  Hence many esoteric features of |loop| don't exist in \iter.
Other features have been carried over, but in a cleaned-up form.  
And of course, many new features have been added; they are not
mentioned in this list.

\begin{itemize}

\item \iter's syntax is more Lisp-like than |loop|'s, having a higher
density of parens.

\item The current implementation of \iter, unlike the current version
of  |loop| (as documented in {\em Common Lisp, 2nd Ed.\/}), is
extensible (see section \ref{extend}).  

\item |loop| puts the updates of all driver variables at the top of
the loop; \iter\ leaves them where the driver clauses appear.

\item While for the most part \iter\ clauses that resemble |loop| clauses
behave similarly, there are some differences.  For instance, there is
no |for\dots =\dots then| in \iter; instead use 
|for\dots initially\dots then|.  

\item |loop| binds the variable |it| at certain times
to allow pseudo-English expressions like |when ~expr~ return it|. 
In \iter, you must bind ~expr~ to a variable yourself.  Note that
|when ~expr~ return it| is like |thereis ~expr~| except that the latter is an
accumulation clause and therefore competes with other accumulations
(remember section \ref{multiple} above).

% repeat different behaviour of |always| clause here?

\item |loop| has a special |return| clause, illustrated in the
previous item.  \iter\ doesn't need one,  since an ordinary Lisp
|return| has the same effect.

\item |loop| allows for parallel binding and stepping of iteration
variables.  \iter\ does not.  (See section \ref{bindings}.)

\item |loop| and \iter\ handle variable type declarations very
differently.  |loop| provides a special syntax for declaring variable
types, and does not examine declarations.  Moreover, the standard
implementation of |loop| will
generate declarations when none are requested.
\iter\ parses standard Common Lisp type declarations, and will never
declare a variable itself unless declarations are 
specifically requested.

\end{itemize}

%%%%%%%%%%%%%%%%%%%%%%%%%%%%%%%%%%%%%%%%%%%%%%%%%%%%%%%%%%%%%%%%
\section{Rolling Your Own}
\label{extend}

\subsection{Introduction}

\iter\ is extensible---you can write new clauses that embody new
iteration patterns.  You might want to write a new driver clause for a
data structure of your own, or you might want to write a clause that
collects or manipulates elements in a way not provided by \iter. 

This section describes how to write clauses for \iter.  Writing a
clause is like writing a macro.  In fact, writing a clause ~is~
writing a macro: since \iter\ code-walks its body and macroexpands,
you can add new abstractions to \iter\ with good old |defmacro|.

Actually, there are two extensions you can make to \iter\ that are
even easier than writing a macro.  They are adding a synonym for an
existing clause and defining a driver clause for an indexable
sequence.  These can be done with |defsynonym| and
|defclause-sequence|, respectively.  See section \ref{aids}, below.

The rest of this section explains how to write macros that expand into
\iter\ clauses.
Here's how you could add a simplified version of \iter's |multiply|
clause, if \iter\ didn't already have one:
\begin{program}
(defmacro multiply (expr)
  `(reducing ,expr by \#'* initial-value 1))
\end{program}

If you found yourself summing the square of an expression often, you
might want to write a macro for that.  A first cut might be
\begin{program}
(defmacro sum-of-squares (expr)
  `(sum (* ,expr ,expr)))
\end{program}
but if you are an experienced macro writer, you will realize that this
code will evaluate ~expr~ twice, which is probably a bad idea.  A
better version would use a temporary:
\begin{program}
(defmacro sum-of-squares (expr)
  (let ((temp (gensym)))
    `(let ((,temp ,expr))
       (sum (* ,temp ,temp)))))
\end{program}
Although this may seem complex, it is just the sort of thing you'd
have to go through to write any macro, which illustrates the point of
this section: if you can write macros, you can extend \iter.

Our macros don't use \iter's keyword-argument
syntax.  We could just use keywords with |defmacro|, but we would
still not be using \iter's clause indexing mechanism.  Unlike Lisp,
which uses just the first symbol of a form to determine what function
to call, \iter\ individuates clauses by the list of required keywords.
For instance, |for\dots in| and |for\dots in-vector| are different clauses
implemented by distinct Lisp functions.  

To buy into this indexing scheme, as well as the keyword-argument
syntax, use |defmacro-clause|:

\begin{clauses}

\defmacro{defmacro-clause}{arglist |\&body| body}
Defines a new \iter\ clause.  ~arglist~ is a list of symbols which are
alternating keywords and arguments.  \opt\ may be used, and the list
may be terminated by |\&sequence|.  ~body~ is an ordinary macro body,
as with |defmacro|.  If the first form of ~body~ is a string, it is
considered a documentation string and will be shown by
|display-iterate-clauses|. |defmacro-clause| will signal an error if 
defining the clause would result in an ambiguity.  E.g. you cannot
define the clause |for\dots from| because there would be no way to
distinguish it from a use of the |for| clause with optional keyword |from|.

\end{clauses}

\medskip

Here is |multiply| using |defmacro-clause|.  The keywords are capitalized
for readability.
\begin{program}
(defmacro-clause (MULTIPLY expr \opt\ INTO var)
  `(reducing ,expr by \#'* into ,var initial-value 1))
\end{program}
You don't have to worry about the case when |var| is not supplied; for
any clause with an |into| keyword, saying |into nil| is equivalent to
omitting the |into| entirely.

As another, more extended example, consider the fairly common
iteration pattern that 
involves finding the sequence element that maximizes (or minimizes) some
function.  \iter\ provides this as |finding\dots maximizing|, but it's
instructive to see how to write it.
Here, in pseudocode, is how you might write such a loop for
maximizing a function F:

\begin{quote}
\begin{tabbing}
set variable MAX-VAL to NIL; \\
set variable WINNER to NIL; \\
for\=\ each element EL in the sequence \\
\>  if\=\ MAX-VAL\=\ is NIL or F(EL) $>$ MAX-VAL then \\
\>\>    set MAX-VAL to F(EL); \\
\>\>    set WINNER to EL; \\
\>  end if; \\
end for; \\
return WINNER.
\end{tabbing}
\end{quote}

Here is the macro:
\begin{program}
(defmacro-clause (FINDING expr MAXIMIZING func \opt\ INTO var)
  (let ((max-val (gensym))
        (temp1 (gensym))
        (temp2 (gensym))
        (winner (or var iterate::*result-var*)))
    `(progn 
       (with ,max-val = nil)
       (with ,winner = nil)
       (cond 
        ((null ,max-val)
         (setq ,winner ,expr)
         (setq ,max-val (funcall ,func ,winner))
        (t
         (let* ((,temp1 ,expr)
                (,temp2 (funcall ,func ,temp1)))
           (when (> ,temp2 ,max-val)
             (setq ,max-val ,temp2)
             (setq ,winner ,temp1))))))
       (finally (leave ,winner)))))
\end{program}
Note that if no |into| variable is supplied, we use
|iterate::*result-var*|, which contains the internal variable into
which all clauses place their results.  If this variable is bound by
some clause, then \iter\ will return its value automatically;
otherwise, \nil\ will be returned.  

\subsection{Writing Drivers}

In principle, drivers can be implemented just as easily as other
\iter\ clauses.  In practice, they are a little harder to get right. 
As an example, consider writing a driver that
iterates over all the 
elements of a vector, ignoring its fill-pointer.  |for\dots in-vector|
won't work for this, because it observes the fill-pointer.  It's necessary to
use |array-dimension| instead of |length| to obtain the size of the
vector.  Here is one approach:

\begin{program}
(defmacro-clause (FOR var IN-WHOLE-VECTOR v)
  "All the elements of a vector (disregards fill-pointer)"
  (let ((vect (gensym))
        (index (gensym)))
    `(progn
       (with ,vect = ,v)
       (for ,index from 0 below (array-dimension ,vect 0))
       (for ,var = (aref ,vect ,index)))))
\end{program}
Note that we immediately put |v| in a variable, in case it is an
expression.  Again, this is just good Lisp macrology.  It also has a
subtle effect on the semantics of the driver: |v| is evaluated only
once, at the beginning of the loop, so changes to |v| in the loop have
no effect on the driver.  Similarly, the bounds for numerical iteration
e.g. the above |array-dimension| are also evaluated once only.  This is how
all of \iter's drivers work.


There is an important point concerning the |progn| in this code.  We
need the |progn|, of course, because we are returning several forms,
one of which is a driver.  But \iter\ drivers must occur at top-level.
Is this code in error?  No, because ~top-level~ is defined in \iter\
to include forms inside a |progn|.  This is just the definition of
top-level that Common Lisp uses, and for the same reason: to allow
macros to return multiple forms at top-level.

While our |for\dots in-whole-vector| clause will work, it is not
ideal.  In particular, it does not support generating.  Do do so, we
need to use |for\dots next| or |for\dots do-next|.  The job is
simplified by the |defmacro-driver| macro.

\begin{clauses}

\defmacro{defmacro-driver}{arglist |\&body| body}
Defines a driver clause in
both the |for| and |generate| forms, and provides a parameter
|generate| which ~body~ can examine to determine how it was invoked.
~arglist~ is as in |defmacro-clause|, and should begin with the symbol
|for|. 

\end{clauses}

With |defmacro-driver|, our driver looks like this:
\begin{program}
(defmacro-driver (FOR var IN-WHOLE-VECTOR v)
  "All the elements of a vector (disregards fill-pointer)"
   (let ((vect (gensym))
         (end (gensym))
         (index (gensym))
         (kwd (if generate 'generate 'for)))
     `(progn
        (with ,vect = ,v)
        (with ,end = (array-dimension ,vect 0))
        (with ,index = -1)
        (,kwd ,var next (progn (incf ,index)
                               (if (>= ,index ,end) (terminate))
                               (aref ,vect ,index))))))
\end{program}


We are still missing one thing: the |\&sequence| keywords.  
We can get them easily enough, by writing 
\begin{program}
(defmacro-driver (FOR var IN-WHOLE-VECTOR v \&sequence) 
  ...)
\end{program}
We can now refer to parameters |from|, |to|, |by|, etc. which contain
either the values for the corresponding keyword, or \nil\ if the
keyword was not supplied.  Implementing the right code for these
keywords is cumbersome but not difficult; it is left as an exercise.
But before you begin, see |defclause-sequence| below for an easier way.
       
\subsection{Extensibility Aids}
\label{aids}

This section documents assorted features that may be of use in
extending \iter.

\begin{clauses}

\defunexpvar{*result-var*}
Holds the variable that is used to return a value as a result of the
\iter\ form.  You may examine this and use it in a |with| clause, but
you should not change it.

\defmacro{defsynonym}{syn word}
 Makes ~syn~ a synonym for the existing \iter\ keyword ~word.~  Only
the first word in each clause can have synonyms. 

\defmacroxx{defclause-sequence}{element-name index-name}{|\&key|
access-fn size-fn sequence-type}{element-type
element-doc-string index-doc-string}  
Provides
a simple way to define sequence clauses.  Generates two
clauses, one for iterating over the sequence's elements, the other
for iterating over its indices.  The first symbol of both
clauses will have print-name |for|.
~element-name~  and ~index-name~ should be symbols.
~element-name~ is the second keyword of the element iterator (typically of
the form 
|in-~sequence-type~|), and ~index-name~ is the second keyword
of the index-iterator (typically of the form
|index-of-~sequence-type~|).  Either name may be 
\nil, in which case the corresponding clause is not defined.  If both
symbols are supplied, they should be in the same package.  The |for|
that begins the clauses will be in this package.

\cpar ~access-fn~ is the function to be used to
access elements of the sequence in the element iterator.  The function
should take two 
arguments, a sequence and an index, and return the appropriate element.
~size-fn~ should denote a function of one argument, a sequence, that
returns its size.  Both ~access-fn~ and ~size-fn~ are required for the
element iterator, but only ~size-fn~ is needed for the index iterator.

\cpar The ~sequence-type~ and ~element-type~ keywords are used to
suggest types for the variables 
used to hold the sequence and the
sequence elements, respectively.  The usual rules about \iter's
treatment of variable type declarations apply (see section \ref{types}).

\cpar ~element-doc-string~ and ~index-doc-string~ are
the documentation strings, for use with |display-iterate-clauses|. 

\cpar The generated element-iterator performs destructuring on the
element variable.

\cpar As an example, the above |for\dots in-whole-vector| example
could have been written:
\begin{program}
(defclause-sequence IN-WHOLE-VECTOR INDEX-OF-WHOLE-VECTOR
  :access-fn 'aref
  :size-fn \#'(lambda (v) (array-dimension v 0))
  :sequence-type 'vector
  :element-type t
  :element-doc-string 
     "Elements of a vector, disregarding fill-pointer"
  :index-doc-string 
     "Indices of vector, disregarding fill-pointer")
\end{program}

\end{clauses}

\subsection{Subtleties}

There are some subtleties to be aware of when writing \iter\ clauses.
First, the code returned by your macros may be |nconc|'ed into a list,
so you should always returned freshly consed lists, rather than
constants.  Second, \iter\ matches clauses by using |eq| on the first
symbol and |string=| on the subsequent ones, so the package of the
first symbol of a clause is relevant.  All of the clauses in this manual
have their first word in the \iter\ package.  
You can use the package system in the usual way to shadow
\iter\ clauses without replacing them.

%%% say more here, about the badness that only the first word of a
%%% clause is packagey.

\section{Non-portable Extensions to Iterate (Contribs)}
\label{contribs}

Currently, there is only one non-portable extension to iterate in the
distribution: iterate-pg. If you have made an extension that depends
on non-portable features, feel free to send them to |asf@boinkor.net|
for inclusion in the iterate distribution.

\subsection{An SQL query driver for iterate}

The pg package by Eric Marsden (see |http://cliki.net/pg|) provides an
interface to the PostgreSQL database. Using the \iterpg\ extension, it
is possible to handle the results of SQL queries using \iter.

This usage example should give you an idea of how to use it:

\begin{program}
(pg:with-pg-connection (c "somedb" "someuser")
  (iter (for (impl version date) in-relation "select * from version"
                                 on-connection *dbconn*)
        (collect version)))
\end{program}

To use the extension via |ASDF|, simply make your system depend on the
|iterate-pg| system instead of the |iterate| system. To load it
manually, use:

\begin{program}
  (asdf:oos 'asdf:load-op :iterate-pg)
\end{program}

%%%%%%%%%%%%%%%%%%%%%%%%%%%%%%%%%%%%%%%%%%%%%%%%%%%%%%%%%%%%%%%%
\section{Obtaining {\tt Iterate}}

\begin{sloppypar}
The information in this chapter is obsolete but included for
completeness's sake; Currently, the most up-to-date information on
\iter\ can be found at |http://boinkor.net/iterate.html|.
\end{sloppypar}

\begin{sloppypar}
\iter\ currently runs on Lisp Machines, and on
HP's, Sun3's and Sparcstations under Lucid.  
\iter\ source and binaries are available at the MIT AI Lab in the
subdirectories of |/src/local/lisplib/|.  The source file,
|iterate.lisp|, is also available for anonymous FTP in the directory
|/com/fpt/pub/| on the machine |TRIX.AI.MIT.EDU| (Internet number
128.52.32.6).  If you are unable to obtain |iterate| in one of these
ways, send mail to |jba@ai.mit.edu| and I will send you the source
file. 
\end{sloppypar}

\begin{sloppypar}
\iter\ resides in the |iterate| package (nickname |iter|).  Just say
\linebreak |(use-package :iterate)| to make all the necessary symbols
available. 
If a symbol is not exported, it appears in this manual with an
``|iterate::|'' prefix.
\end{sloppypar}

Send bug reports to |bug-iterate@ai.mit.edu|.  The |info-iterate|
mailing list will have notices of changes and problems; to have
yourself added, send mail to |info-iterate-request@ai.mit.edu|.


\medskip

\begin{flushleft}
 \bf Acknowledgements
\end{flushleft}
\smallskip

Richard Waters provided invaluable criticism which spurred me to improve
\iter\ greatly.  As early users, David Clemens, Oren Etzioni and Jeff
Siskind helped ferret out many bugs.

%\begin{theindex}
% The index files must be generated with the genindex program
% or from makeindex iterate-manual.idx which creates iterate-manual.ind.

%\baselineskip\normalbaselineskip
%\advance\baselineskip by -2pt

\input{iterate-manual.ind}

%\underline{Clauses}
%\input{iterate-manual.clindex}
%
%\indexspace
%\underline{Lisp}
%\input{iterate-manual.lispindex}

%\end{theindex}


\end{document}


% arch-tag: "c0f871a6-313c-11d8-abb9-000c76244c24"


%\end{theindex}


\end{document}


% arch-tag: "c0f871a6-313c-11d8-abb9-000c76244c24"


%\underline{Clauses}
%% Fix ``.'' spacing in lists
\documentstyle[12pt,aimemo]{article}

\setlength{\textwidth}{6.0in}
\setlength{\oddsidemargin}{0.25in}
\setlength{\evensidemargin}{0.25in}
\setlength{\topmargin}{0pt}
\setlength{\headsep}{24pt}
\setlength{\textheight}{8.5in}

\let\origpar=\par

% somehow, LaTex fucks with things so that the plain Tex \obeyspaces
% doesn't work.  This is the same, except we use an mbox containing
% a space instead of just a space.

\newcommand{\spc}{\mbox{ }}
\newcommand{\tAb}{\mbox{        }}
\def\obspaces{\catcode`\ =\active\catcode`\^^I=\active}
{\obspaces \global\let =\spc\global\let^^I=\tAb}

\newenvironment{program}{\medskip\samepage\tt\obeylines\obspaces}{\medskip}

% use | for \tt
\catcode`\|=13
\newif\ifvbar
\def|{\ifvbar
         \endgroup\vbarfalse
      \else 
         \begingroup\tt\vbartrue 
      \fi}



\newcommand{\lisp}{\tt}
\newcommand{\iter}{|iterate|}
\newcommand{\iterpg}{|iterate-pg|}
\newcommand{\setf}{|setf|}
\newcommand{\nil}{|nil|}
\newcommand{\nonnil}{non-\nil}
\newcommand{\opt}{{\lisp \&optional}}
\newcommand{\yields}{$\Rightarrow$}



\newbox\Dots
\setbox\Dots=\hbox{\small ...}
\renewcommand{\dots}{\copy\Dots}

% Here's a way to define an environment which will treat all
% paragraphs inside it in the same manner.  It's, as near as I can
% figure, the way that Latex does things.  Using \everypar won't work;
% apparently Latex resets that a lot.  What you have to do is: at the
% top of the file do \let\origpar=\par.  In the
% begin part of the environment, do the parshape (or whatever), then say
% \def\par{{\origpar}}.  The extra braces are necessary.   That's all
% you have to do; at the end of the environment things will return to
% normal automatically.  Why does this all work?  I have no idea.
% Another important thing: there should be a blank line (i.e. a \par)
% between the end of the last paragraph and the \end{environment}.

\newlength{\clindent}
\setlength{\clindent}{0.5in}
\newlength{\clparindent}
\setlength{\clparindent}{\parindent}

\newenvironment{clauses}{\advance\linewidth -\parindent
                        \hangindent\clindent\relax
                         \hangafter=0
                          \parindent=0pt
                          \def\par{{\origpar}}}{}

\newcommand{\cpar}{\hspace \clparindent}

\newcommand{\startitem}{\bigskip\pagebreak[3]}
\newcommand{\finishitem}{\smallskip}

\newcommand{\clause}[1]{\startitem\Clause{#1}\finishitem}
\newcommand{\Clause}[1]{\hbox{#1}}
\newcommand{\clindex}[1]{\index{{\lisp #1}}}
\newcommand{\clindexx}[2]{\index{{\lisp #1\dots #2}}}
\newcommand{\clindexxx}[3]{\index{{\lisp #1\dots #2\dots #3}}}
\newcommand{\clausex}[2]{\startitem\Clausex{#1}{#2}\finishitem}
\newcommand{\Clausex}[2]{\Clause{|#1| #2}\clindex{#1}}

\newcommand{\defvar}[1]{\startitem\deff{#1}{}{Variable}\finishitem}
\newcommand{\defunexpvar}[1]{\startitem\deffnoindex{iterate::#1}{}{Variable}
                                        \lispindex{#1}\finishitem}
\newcommand{\defun}[2]{\startitem\deff{#1}{#2}{Function}\finishitem}
\newcommand{\defmacro}[2]{\startitem\deff{#1}{#2}{Macro}\finishitem}
\newcommand{\defunx}[3]{\startitem\deffx{#1}{#2}{#3}{Function}\finishitem}
\newcommand{\defmacrox}[3]{\startitem\deffx{#1}{#2}{#3}{Macro}\finishitem}
\newcommand{\defunxx}[4]{\startitem\deffxx{#1}{#2}{#3}{#4}{Function}\finishitem}
\newcommand{\defmacroxx}[4]{\startitem\deffxx{#1}{#2}{#3}{#4}{Macro}\finishitem}

\newcommand{\deff}[3]{\deffnoindex{#1}{#2}{#3}\lispindex{#1}}
\newcommand{\deffnoindex}[3]{\hbox to \hsize{{\tt #1} {\it #2}\hfill [{\it #3}]}}
\newcommand{\deffx}[4]{\deff{#1}{#2}{#4}\moreargs{#1}{#3}}
\newcommand{\deffxx}[5]{\deffx{#1}{#2}{#3}{#5}\moreargs{#1}{#4}}

\newcommand{\moreargs}[2]{\hbox to \hsize{\phantom{\lisp #1} {\it #2}\hfill}}

\newcommand{\lispindex}[1]{\index{{\lisp #1}}}

 \newenvironment{note}[1]{\pagebreak[2]\bigskip
                          \hrule\smallskip\small
                         {\setlength{\parindent}{0pt}
                           \par{\bf #1:}}}{\smallskip\hrule\bigskip}



\makeindex

%%%%%%%%%%%%%%%%%%%%%%%%%%%%%%%%%%%%%%%%%%%%%%%%%%%%%%%%%%%%%%%%

%\thispagestyle{empty}

%\vspace*{2in}

%\title{The |Iterate| Manual}
%\author{Jonathan Amsterdam}

%\maketitle

%\thispagestyle{empty}

%\begin{center}
%\Large *** DRAFT ***
%\end{center}
%This manual is in beta-test. It will become an AI Memo in a couple of
%months.  I would appreciate any comments or
%corrections.  You can reach me at email address |jba@ai.ai.mit.edu|.

\begin{document}

\def\writtenby{Jonathan Amsterdam}
\def\contractno{This report describes research done at the Artificial
Intelligence Laboratory of the Massachusetts Institute of Technology.
Support for the laboratory's artificial intelligence research is
provided in part by the Advanced Research Projects Agency of the
Department of Defense under Office
of Naval Research contract N00014-85-K-0124.}

\AIMEMO{1236}{\Large The Iterate Manual \\[1in]}
{This is the manual for version 1.4 of \iter, a powerful iteration
macro for Common  
Lisp. \iter\ is similar to {\tt loop} but provides numerous additional
features, is well integrated with Lisp, and is extensible. 
\vfill }


% use ~ for \em
\catcode`\~=13
\newif\ifem
\def~{\ifem
         \/\endgroup\emfalse
      \else 
         \begingroup\em\emtrue 
      \fi}



%\pagebreak
%\setcounter{page}{1}

%\thispagestyle{empty}


\tableofcontents

\pagebreak
\pagestyle{headings}

\section{Introduction}

\begin{sloppypar}
This manual describes
\iter, a powerful iteration facility for Common Lisp.
\iter\ provides abstractions for many common iteration
patterns and allows for the definition of additional patterns.  \iter\ is
a macro that expands into ordinary Lisp at
compile-time, so it is more efficient than higher-order
functions like |map| and |reduce.|  
While it is similar to |loop|, \iter\ offers a more Lisp-like syntax
and enhanced extensibility.
(For a more complete
comparison of \iter\ with other iteration constructs, see MIT AI Lab Working
Paper No. 324, ~Don't Loop, Iterate.~)
\end{sloppypar}

An \iter\ form consists of the symbol |iter|\footnote{You can also use
|iterate|, but |iter| is preferred because it avoids potential conflicts with
possible future additions to Common Lisp, and because it saves
horizontal space when writing code.}
followed by one or more
forms, some of which may be \iter\ 
~clauses.~  Here is a simple example of \iter\ which collects the
numbers from 1 to 10 into a list, and returns the list.  The return
value is shown following the arrow.
\begin{program}
(iter (for i from 1 to 10)
      (collect i))          \yields (1 2 3 4 5 6 7 8 9 10)
\end{program}
This form contains two clauses: a |for| clause that steps the
variable |i| over the integers from 1 to 10, and a |collect|
clause that accumulates its argument into a list.  With a few
exceptions, all \iter\ clauses have the same format: alternating
symbols (called ~keywords~) and expressions (called
~arguments~).  The syntax and terminology are those of Common
Lisp's keyword lambda lists.  One difference is that \iter's keywords
do not have to begin with a colon---though they may, except for the
first symbol of a clause.  So you can also write  |(for i :from
1 :to 10)| if you prefer.

Any Lisp form can appear in the body of an \iter, where it will have
its usual meaning.  \iter\ walks the entire body, expanding macros,
and recognizing clauses at any level.  This example collects all the
odd numbers in a list:
\begin{program}
(iter (for el in list)
      (if (and (numberp el) (oddp el))
          (collect el)))
\end{program}

There are clauses for iterating over numbers, lists, arrays and other
objects, and for
collecting, summing, counting, maximizing and other useful operations.
\iter\ also supports the creation of new variable bindings, stepping
over multiple sequences at once, destructuring, and compiler
declarations of variable types.  The following example illustrates
some of these features: 
\begin{program}
(iter (for (key . item) in alist)
         (for i from 0)
         (declare (fixnum i))
         (collect (cons i key)))
\end{program}

This loop takes the keys of an alist and returns a new alist
associating the keys with their positions in the original list. The
compiler declaration for |i| will appear in the generated code
in the appropriate place.

%%%%%%%%%%%%%%%%%%%%%%%%%%%%%%%%%%%%%%%%%%%%%%%%%%%%%%%%%%%%%%%%
\section{Clauses}

Most of \iter's clauses will be familiar to |loop| programmers.  
(|loop| is an iteration macro that has been incorporated into Common
Lisp.  Seen Guy Steele's {\em Common Lisp, 2nd Edition}.)
In
nearly all cases they behave the same as their |loop| counterparts, so
a |loop| user can switch to \iter\ with little pain (and much gain).

All clauses with the standard keyword-argument syntax consist of two
parts: a ~required~ part, 
containing keywords that must be present and in the right order; and
an ~optional~ part, containing keywords that may be omitted and,
if present, may occur in any order.  In the descriptions below, the
parts are separated by the Lisp lambda-list keyword \opt.


\subsection{Drivers}

An iteration-driving clause
conceptually causes the iteration to go forward.  Driver clauses in
\iter\ allow iteration over numbers, lists, vectors, hashtables, packages,
files and streams.  Iteration-driving clauses must
appear at the top level of an \iter\ form; they cannot be nested
inside another clause.  The driver variable is updated at the point
where the driver clause occurs.  Before the clause is executed for the
first time, the value of the variable is undefined.

%Also, regardless of where the driver clause
%appears in the body, the driver variable is stepped at the top of the
%loop; hence it is stylistically preferable, though
%not required, to place driver clauses at the beginning of the \iter.

Multiple drivers may appear in a single \iter\ form, in which case all
of the driver variables are updated each time through the loop, in the
order in which the clauses appear.  The first driver to terminate will
terminate the entire loop.

In all cases, the value of the driver variable on exit from the loop,
including within the epilogue code (see the |finally| clause), is
undefined.

All the parameters of a driver clause are evaluated once, before the
loop begins.  Hence it is not possible to change the bounds or other
properties of an iteration by side-effect from within the loop.

With one exception, driver clauses begin with the word |for| (or
the synonym |as|) and mention an iteration variable, which is
given a binding within the \iter\ form.
The exception is |repeat|, which just executes a loop a
specified number of times:

\begin{clauses}

\clausex{repeat}{~n~}
Repeats the loop ~n~ times.  For example:
\begin{program}
(iter (repeat 100)
      (print "I will not talk in class."))
\end{program}
\cpar If $n \leq 0$, the
loop will never be executed.  If ~n~ is not an integer, the
actual number of executions will be $\lceil n \rceil$.

\end{clauses}

\pagebreak[3]

\subsubsection{Numerical Iteration}

\begin{clauses}
\clausex{for}{~var~ |\&sequence|} 
The general form for iterating over a sequence of numbers requires a
variable and, optionally, one or more keywords that provide the bounds
and step size of the iteration. The |\&sequence|
lambda-list keyword is a shorthand for these sequence keywords.  They are:
|from|, |upfrom|, |downfrom|, |to|, |downto|, |above|, |below| and |by|.
|from| provides the starting value for ~var~
and defaults to zero. |to| provides a final value and implies that the
successive values of ~var~ will be increasing; |downto| implies that
they will be decreasing.  The loop terminates when ~var~ passes
the final value (i.e. becomes smaller or larger than it, depending on
the direction of iteration); in other  words, the loop body
will never be 
executed for values of ~var~ past the final value.
|below| and |above| are similar to |to| and
|downto|, except that the loop terminates when ~var~ equals or passes
the final value.

\cpar If no final value is specified, the variable will be stepped
forever.  Using |from| or |upfrom| will result in increasing
values, while |downfrom| will give decreasing values.

\cpar On each iteration,
~var~ is incremented or decremented by the value of the sequence
keyword |by|, which defaults to 1.  It should always be a positive
number, even for downward iterations.

\cpar In the following examples, the sequence of numbers generated is shown
next to the clause.

\begin{program}
(for i upfrom 0) \yields\ 0 1 2 \ldots
(for i from 5)   \yields\ 5 6 7 \ldots    ; either from or upfrom is okay
(for i downfrom 0) \yields\ 0 -1 -2 \ldots
(for i from 1 to 3) \yields\ 1 2 3
(for i from 1 below 3) \yields\ 1 2
(for i from 1 to 3 by 2) \yields\ 1 3
(for i from 1 below 3 by 2) \yields\ 1
(for i from 5 downto 3) \yields\ 5 4 3
\end{program}

\end{clauses}

\subsubsection{Sequence Iteration}

There are a number of clauses for iterating over sequences.  In all of
them, the argument following |for| may be a list instead of a
symbol, in which case destructuring is performed.  See section
\ref{destructuring}. 

\begin{clauses}

\clause{|for| ~var~ |in| ~list~ \opt\ |by| ~step-function~}
\clindexx{for}{in}
~var~ is set to successive elements of list.  
~step-function,~ which 
defaults to |cdr|, is used to obtain the next sublist. 

\clause{|for| ~var~ |on| ~list~ \opt\ |by| ~step-function~}
\clindexx{for}{on}
~var~ is set to successive sublists of list. 
~step-function~ (default |cdr|) is used as in |for\dots in|.

\end{clauses}

\medskip

These two clauses use |atom| to test for the end
of a list.  Hence, given a list whose final |cdr| is not \nil,
they will silently ignore the last |cdr|.  Other choices are
|endp|,  which would signal an error, and |null|, which
would probably result in an error somewhere else.  If you wish to use
an end-test other than |atom|, set the variable
|iterate::*list-end-test*|\lispindex{*list-end-test*} to the name of
the desired function. 

\begin{clauses}
\clause{|for| ~var~ |in-vector| ~vector~ |\&sequence|}
\clindexx{for}{in-vector}
~var~ takes on successive elements from ~vector.~   The vector's
fill-pointer is observed.  Here and in subsequent clauses, the |\&sequence|
keywords include |with-index|, which takes a symbol as argument
and uses it for the index variable instead of an internally generated
symbol.  The other |\&sequence| keywords behave as in numerical
iteration, except that the default iteration bounds are the bounds of
the vector.  E.g. in |(for i in-vector v downto 3)|,
|i| will start off being bound to the last element in |v|,
and will  
be set to preceding elements down to and including the element with
index 3.

\clause{|for| ~var~ |in-sequence| ~seq~ |\&sequence|}
\clindexx{for}{in-sequence}
This uses Common Lisp's generalized sequence functions, |elt|
and |length|, to obtain elements and determine the length of
~seq.~   Hence it will work for any sequence, including lists, and will
observe the fill-pointers of vectors.

\clause{|for| ~var~ |in-string| ~string~ |\&sequence|}
\clindexx{for}{in-string}
~var~ is set to successive characters of ~string.~

\startitem
\Clause{|for| ~var~ |index-of-vector| ~vector~ |\&sequence|}
\clindexx{for}{index-of-vector}
\Clause{|for| ~var~ |index-of-sequence| ~sequence~ |\&sequence|}
\clindexx{for}{index-of-sequence}
\Clause{|for| ~var~ |index-of-string| ~string~ |\&sequence|}
\clindexx{for}{index-of-string}
\finishitem
~var~ is set to successive indices of the sequence.
These clauses avoid the overhead of accessing the sequence elements
for those applications where they do not need to be examined, or are
examined rarely.  They admit all the optional keywords of the other
sequence drivers except the (redundant) |with-index| keyword.

\clause{|for| ~(key value)~ |in-hashtable| ~table~}
\clindexx{for}{in-hashtable}

~key~ and ~value,~ which must appear as shown in a list
and may be destructuring templates, are set to the keys and values of
~table~.  If ~key~ is |nil|, then the hashtable's keys will be ignored;
similarly for ~value~.
The order in which 
elements of ~table~ will be retrieved is unpredictable.

\clause{|for| ~var~ |in-package| ~package~ \opt\ |external-only| ~ext~}
\clindexx{for}{in-package}
Iterates over all the symbols in ~package,~ or over only the
external symbols if ~ext~ is specified and non-|nil|.  ~ext~ is not
evaluated.  The same symbol may appear more than once.

\clause{|for| ~var~ |in-packages| ~packages~ \opt\ |having-access| ~symbol-types~}
\clindexx{for}{in-packages}
Iterates over all the symbols from the list of packages denoted by the
descriptor ~packages~ and having accessibility (or visibility) given by
~symbol-types~.  This defaults to the list |(:external :internal :inherited)|
and is not evaluated.  ~var~ must be a list of up to three variables: in each
iteration, these will be set to a symbol, its access-type and package (as per
|with-package-iterator| in ANSI-CL).  The same symbol may appear more than
once.

\end{clauses}

\begin{clauses}

\clause{|for| ~var~ |in-file| ~name~ \opt\ |using| ~reader~}
\clindexx{for}{in-file}
Opens the file ~name~ (which may be a string or pathname) for input,
and iterates 
over its contents. ~reader~ defaults to |read|, so by default ~var~
will be bound to the successive forms in the file.  The \iter\ body is
wrapped in an unwind-protect to ensure that the file is closed no
matter how the \iter\ is exited.

\clause{|for| ~var~ |in-stream| ~stream~ \opt\ |using| ~reader~}
\clindexx{for}{in-stream}
Like |for\dots in-file|, except that ~stream~ should be an existing
stream object that supports input operations.

\end{clauses}

\subsubsection{Generalized Drivers}

These are primarily useful for writing drivers that can also be used
as generators (see section \ref{generators}, below).
 
\begin{clauses}

\clause{|for| ~var~ |next| ~expr~}
\clindexx{for}{next}

~var~ is set to ~expr~ each time through the loop.  Destructuring is
performed.  When the clause is used as a generator, ~expr~ is the code
that is executed when |(next ~var~)| is 
encountered (see section \ref{generators}, below).
~expr~ should compute the first value for ~var~, as well
as all subsequent values, and is responsible for terminating the loop.
For compatibility with future versions of \iter, this termination
should be done with |terminate|\clindex{terminate}, which can be
considered a synonym for |finish| (see section \ref{control-flow}).

\cpar As an example, the following clauses are equivalent to |(for
i from 1 to 10)|:
\begin{program}
(initially (setq i 0))
(for i next (if (> i 10) (terminate) (incf i)))
\end{program}

\clause{|for| ~var~ |do-next| ~form~}
\clindexx{for}{do-next}
~form~ is evaluated each time through the loop.  Its value is ~not~
set to ~var~; that is ~form's~ job. ~var~ is only present so that
\iter\ knows it is a driver variable.  \linebreak |(for ~var~ next
~expr~)| is equivalent to |(for ~var~ do-next (dsetq ~var~ ~expr~))|.
(See section \ref{destructuring} for an explanation of |dsetq|.) 

\end{clauses}

\subsubsection{Generators}
\label{generators}

In all of the above clauses, the driver variable is updated on each
iteration.
Sometimes it is desirable to have greater control over updating.  
For instance, consider the problem of associating numbers, in
increasing order and with no gaps, with the
\nonnil\ elements of a list.  One obvious first pass at writing this is:

\begin{program}
(iter (for el in list)
      (for i upfrom 1)
      (if el (collect (cons el i))))
\end{program}
But on the list |(a b nil c)| this produces |((a . 1) (b . 2) (c .
4))| instead of the desired |((a . 1) (b . 2) (c . 3))|.  The problem
is that |i| is incremented each time through the loop, even when |el|
is |nil|.  

The problem could be solved elegantly if we could step |i| only when
we wished to.  This
can be accomplished for any \iter\ driver by writing |generate| (or
its synonym |generating|)
instead of |for|.  Doing so produces a ~generator~---a driver whose
values are yielded explicitly.  To obtain the next value of a
generator variable ~v~, write \linebreak |(next ~v~)|.  The value of a |next|
form is the next value of ~v~, as determined by its associated driver
clause.  |next| also has the side-effect of updating ~v~ to that
value.  If there is no next value, |next| will terminate the loop,
just as with a normal driver.


Using generators, we can now write our example like this:
\begin{program}
(iter (for el in list)
      (generate i upfrom 1)
      (if el (collect (cons el (next i)))))
\end{program}
Now |i| is updated only when |(next i)| is executed, and this occurs
only when |el| is \nonnil.

To better understand the relationship between ordinary drivers and
generators, observe that we can rewrite an ordinary driver using its
generator form immediately followed by |next|, as this example shows:
\begin{program}
(iter (generating i from 1 to 10)
      (next i)
      ...)
\end{program}
Provided that the loop body contains no |(next i)| forms, this will
behave just as if we had written |(for i from 1 to 10)|.

We can still refer to a driver variable ~v~ without using |next|; in
this case, its value is that given to it by the last evaluation of
|(next ~v~)|.  Before |(next ~v~)| has been called the first time, the
value of ~v~ is undefined.  

This semantics is more flexible than
one in which ~v~ begins the loop bound to its first value and calls
of |next| supply subsequent values, because it means the loop will not
terminate too soon if the generator's sequence is empty.  For
instance, consider the following code, which tags \nonnil\ elements of
a list using a list of tags, and also counts the null elements.
(We assume there are at least as many tags as \nonnil\ elements.)
\begin{program}
(let* ((counter 0)
       (tagged-list (iter (for el in list)
                          (generating tag in tag-list) 
                          (if (null el)
                              (incf counter)
                              (collect (cons el (next tag)))))))
  ...)
\end{program}
It may be that there are just as many tags as non-null elements of
|list|.  If all the elements of |list| are null, we still want the
counting to proceed, even though |tag-list| is \nil.  If |tag| had to be
assigned its first value before the loop begins, we would have had to
terminate the loop before the first iteration, since when |tag-list|
is \nil, |tag| 
has no first value.  With the existing semantics, however, |(next
tag)| will never execute, so the iteration will cover all the elements of
|list|. 
        
When the ``variable'' of a driver clause is actually a destructuring
template containing several variables, all the variables are eligible
for use with |next|.  As before, |(next ~v~)| evaluates to ~v's~ next
value; but the effect is to update all of the template's variables.
For instance, the following code will return the list |(a 2 c)|.
\begin{program}
(iter (generating (key . item) in '((a . 1) (b . 2) (c . 3)))
      (collect (next key))
      (collect (next item)))
\end{program}

Only driver clauses with variables can be made into generators.  This
includes all clauses mentioned so far except for |repeat|.  It does
~not~ include |for\dots previous|, |for\dots =|,
|for\dots initially\dots then| or |for\dots first\dots then| (see
below). 

\subsubsection{Previous Values of Driver Variables}

Often one would like to access the value of a variable on a previous
iteration.  \iter\ provides a special clause for accomplishing this.

\begin{clauses}

\clause{|for| ~pvar~ |previous| ~var~ \opt\ |initially| ~init~
|back| ~n~}
\clindexx{for}{previous}
Sets ~pvar~ to the previous value of ~var,~ which should be a driver
variable, a variable from another |for\dots previous| clause, or a
variable established by a |for\dots =|, 
|for\dots initially\dots then| or |for\dots first\dots then| clause
(see section \ref{setting}).
Initially, ~pvar~ is given the value ~init~ (which defaults to \nil).  
The ~init~ expression will be moved outside the loop body, so it
should not depend on anything computed within the loop.
~pvar~ retains the value of ~init~ until ~var~ is set to its second
value, at which point ~pvar~ is set to ~var's~ first value; and so on.  

\cpar The
argument ~n~ to |back| must be a constant, positive integer, and
defaults to 1.  It determines how many iterations back ~pvar~ should
track ~var~.  For example, when ~n~ is 2, then ~pvar~ will be assigned
~var's~ first value when ~var~ is set to its third value.

\cpar A |for\dots previous| clause may occur after or before its
associated driver clause. |for\dots previous| works with generators as
well as ordinary drivers.

\pagebreak[3]

\cpar Example:
\begin{program}
(iter (for el in '(1 2 3 4))
      (for p-el previous el)
      (for pp-el previous p-el initially 0)
      (collect pp-el))
\end{program}
This evaluates to |(0 0 1 2)|.  It could have been written more
economically as 
\begin{program}
(iter (for el in '(1 2 3 4))
      (for pp-el previous el back 2 initially 0)
      (collect pp-el))
\end{program}


\end{clauses}

\subsection{Variable Binding and Setting}
\label{setting}

Several clauses exist for establishing new variable bindings or for
setting variables in the loop.  They all support destructuring.

\begin{clauses}

\clausex{with}{~var~ \opt\ |=| ~value~}
Causes ~var~ to be bound to value before the loop body is entered.
If ~value~ is not supplied, ~var~ assumes a default
binding, which will be 
\nil\ in the absence of declarations.  Also, if ~value~ is not
supplied, no destructuring is performed; instead, ~var~ may be a list of
symbols, all of which are given default 
bindings.  If ~value~ is supplied, ~var~ is bound to it, with
destructuring.  

\cpar Because |with|
creates bindings whose scope includes the entire \iter\ form, it is
good style to put all |with| clauses at the beginning.

\cpar Successive occurrences of |with| result in sequential
bindings (as with 
|let*|).  There is no way to obtain parallel bindings; see
section \ref{bindings} for a rationale.  


\clause{|for| ~var~ |=| ~expr~}
\clindexx{for}{=}
On each iteration, ~expr~ is evaluated and ~var~ is set
to its value. 

\cpar This clause may appear to do the same thing as |for\dots next|.
In fact, they are quite different.  |for\dots =| provides only three
services: it sets up a binding for ~var~, sets it to ~expr~ on each
iteration, and makes it possible to use |for\dots previous| with
~var~.  |for\dots next| provides these services in addition to the
ability to turn the driver into a generator.
%Also, the code which sets ~var~ appears in the loop body in the same
%place as the |for\dots =| clause; the code for |for\dots next| appears
%at the top of the loop, as with other drivers (except when being used
%as a generator).

\clause{|for| ~var~ |initially| ~init-expr~ |then| ~then-expr~}
\clindexxx{for}{initially}{then}
Before the loop begins, ~var~ is set to ~init-expr;~ on all
iterations after the first it is set to ~then-expr.~  This clause
must occur at top-level.  ~init-expr~ will be moved outside the loop
body and ~then-expr~ will be moved to the end of the loop body, so
they are subject to code motion problems (see section
\ref{code-movement}). 

\cpar This clause may appear to be similar to |for\dots next|, but in
fact they differ significantly.  |for\dots initially\dots then| is
typically used to give ~var~ its first value before the loop begins,
and subsequent values on following iterations.  This is incompatible
with generators, whose first value and subsequent values must all be
computed by |(next ~var~)|.  Also, the update of ~var~ in 
|for\dots initially\dots then| does not occur at the location of the
clause. 
Use |for\dots initially\dots then| for
one-shot computations where its idiom is more convenient, but use
|for\dots next| for extending \iter\ with new drivers (see section
\ref{extend}).

\clause{|for| ~var~ |first| ~first-expr~ |then| ~then-expr~}
\clindexxx{for}{first}{then}
The first time through the loop, ~var~ is set to ~first-expr;~ on
subsequent iterations, it is set to ~then-expr.~  This differs from
|for\dots initially| in that ~var~ is set to ~first-expr~
inside the loop body, 
so ~first-expr~ may depend on the results of other clauses.  For
instance, 
\begin{program}
(iter (for num in list)
      (for i first num then (1+ i))
      ...)
\end{program}
will set |i| to the first element of |list| on the first
iteration, whereas 
\begin{program}
(iter (for num in list)
      (for i initially num then (1+ i))
      ...)
\end{program}
is probably erroneous; |i| will be bound to |num|'s
default binding (usually \nil) for the first iteration.

\end{clauses}

\begin{note}{Compatibility note}
|loop|'s |for\dots =| works like \iter's, but |loop| used the syntax
|for\dots =\dots then| to mean |for\dots initially\dots then|.  It was
felt that these two operations were sufficiently different to warrant
different keywords.

Also, the |for| in the above three clauses is misleading,
since none is true driver (e.g. none has a corresponding |generate|
form).  |setting| would have been a better choice, but |for| was used
to retain some compatibility with |loop|.  
\end{note}


%%%%%%%%%%%%%%%%%%%%%%%%%%%%%%%%
\subsection{Gathering Clauses}
Many of \iter's clauses accumulate values into a variable, or set a
variable under certain conditions.  At the end of the loop, this
variable contains the desired result.  All these clauses have an
optional |into| keyword, whose argument should be a symbol.  If the
|into| keyword is not supplied, the accumulation variable will be
internally generated and its value will be returned at the end of the
loop; if a variable is specified, that variable is used for the
accumulation, and is not returned as a result---it is up to the user
to return it explicitly, in the loop's epilogue code (see |finally|).
It is safe to examine the accumulation variable during the loop, but
it should not be modified.

These clauses all begin with a verb.  When the verb does
not conflict with an existing Common Lisp function, then it may be
used in either its infinitival or present-participle form (e.g. |sum|,
|summing|).  However, when there is a conflict with Common Lisp, only
the present-participle form may be used (e.g. |unioning|).  This is to
prevent \iter\ clauses from clashing with Common Lisp
functions. 

%although these clauses are described as ``producing a value,'' it is a
%mistake to think of the lisp list representing the clause as a
%value-producing form in the usual way.  clauses may legally be written
%where a value is expected, e.g. |(setq x (sum i))|, but the
%lisp value of a clause in such a context is undefined.

\subsubsection{Reductions}

~Reduction~ is an extremely common iteration pattern in which
the results of successive applications of a
binary operation are accumulated.
For example, a loop that computes the sum of the
elements of a list is performing a reduction with the addition
operation.  This could be written in Common Lisp as  |(reduce \#'+
list)| or with \iter\ as
\begin{program}
(iter (for el in list)
      (sum el))
\end{program}

\begin{clauses}

\clausex{sum}{~expr~ \opt\ |into| ~var~}
Each time through the loop, ~expr~ is evaluated and added to a
variable, which is bound initially to zero.  If ~expr~ has a type,
it is ~not~ used as the type of the sum variable, which is always
|number|.  To get the result variable to be of a more specific
type, use an explicit variable, as in
\begin{program}
(iter (for el in number-list)
      (sum el into x)
      (declare (fixnum x))
      (finally (return x)))
\end{program}

\clausex{multiply}{~expr~ \opt\ |into| ~var~}
Like |sum|, but the initial value of the result variable is $1$,
and the variable is updated by multiplying ~expr~ into it.

\clausex{counting}{~expr~ \opt\ |into| ~var~}
~expr~ is evaluated on each iteration.  If it is \nonnil, the
accumulation variable, initially zero, is incremented.

\startitem
\Clausex{maximize}{~expr~ \opt\ |into| ~var~}
\Clausex{minimize}{~expr~ \opt\ |into| ~var~}
\finishitem
~expr~ is evaluated on each iteration and its extremum (maximum or
minimum) is stored in the accumulation variable.  If ~expr~ is never
evaluated, then the result is \nil\ (if the accumulation variable is
untyped) or $0$ (if it has a numeric type).

\clausex{reducing}{~expr~ |by| ~func~ \opt\ |initial-value| ~init-val~ |into| ~var~}
This is a general way to perform reductions. ~func~ should be a
function of two arguments, the first of which will be the value
computed so far and
the second of which will be the value of ~expr~.  It should return the
new value.
|reducing| is roughly equivalent to the
Common Lisp |(reduce ~func~ ~sequence~ :key ~expr-function~)|, where
~expr-function~ is used to derive values from the successive elements
of ~sequence~.

\cpar If the |reducing| clause is never executed, the result is
undefined.

\cpar It is not necessary to provide an
initial value, but better code can be generated if one is supplied.
Regardless of its location in the \iter\ body, ~init-val~ will be
evaluated before the loop is entered, so it should not depend on any
value computed inside the \iter\ form.  

%\cpar if a ~var~ is not specified, you can get \iter\ to declare the
%type of the internal variable by putting a |the| expression around
%~func.~  see section \ref{types}.

\end{clauses}

%\begin{note}{implementation note}
%in principle, |maximize| and |minimize| can be thought of
%as reductions where the initial value is the smallest (or largest)
%value that the accumulation variable can assume.  because lisp's
%bignums can represent arbitrary integers, these clauses cannot be
%implemented as reductions in general.  if, however, the type of 
%~expr~ or ~var~ can be determined to be a fixnum
%or a float, \iter\ will implement the clause as a true reduction,
%using one of the constants |most-negative-fixnum|, |%most-positive-fixnum|, 
%|most-negative-short-float|, etc. as appropriate.
%\end{note}

\subsubsection{Accumulations}

All the predefined accumulation clauses add values to a sequence.  If
the sequence is a list, they all 
have the property that the partial list is kept in the correct order
and available for inspection at any point in the loop.

\begin{clauses}

\clausex{collect}{~expr~ \opt\ |into| ~var~ |at| ~place~ |result-type| ~type~}
Produces a sequence of the values of ~expr~ on each iteration. ~place~
indicates where the next value of ~expr~ is added to the list and may
be one of the symbols |start|, |beginning| (a synonym for |start|) or
|end|.  The symbol may be quoted, but need not be.  The default is
|end|.  For example,
\begin{program}
(iter (for i from 1 to 5)
      (collect i))
\end{program}
produces |(1 2 3 4 5)|, whereas
\begin{program}
(iter (for i from 1 to 5)
      (collect i at beginning))
\end{program}
produces |(5 4 3 2 1)| (and is likely to be faster in most Common Lisp
implementations).

\cpar If ~type~ is provided, it should be a subtype of |sequence|.
The default is |list|.  Specifying a type other than |list| will
result in |collect| returning a sequence of that type.  ~However,~ the
type of the sequence being constructed when inside the loop body is
undefined when a non-|list| type is specified.  (As with ~place~,
quoting ~type~ is optional.)

\clausex{adjoining}{~expr~ \opt\ |into| ~var~ |test| ~test~ |at| ~place~ |result-type| ~type~}
Like |collect|, but only adds the value of ~expr~ if it is not
already present.  ~test,~ which defaults to |\#'eql|, is
the test to be used with |member|.

\startitem
\Clausex{appending}{~expr~ \opt\ |into| ~var~ |at| ~place~}
\Clausex{nconcing}{\ ~expr~ \opt\ |into| ~var~ |at| ~place~}
\Clausex{unioning}{\ ~expr~ \opt\ |into| ~var~ |test| ~test~ |at| ~place~}
\Clausex{nunioning}{~expr~ \opt\ |into| ~var~ |test| ~test~ |at| ~place~}
\finishitem
These are like |collect|, but behave like the Common Lisp functions
|append|, |nconc|, |union| or |nunion|.  
As in Common Lisp, they work only on lists.  Also as in Common Lisp,
|unioning| and |nunioning| assume that the value of ~expr~ contains no
duplicates. 

\clausex{accumulate}{~expr~ |by| ~func~ \opt\ |initial-value| ~init-val~ |into| ~var~}
This is a general-purpose accumulation clause. ~func~ should be a
function of two arguments, the value of ~expr~ and the value
accumulated so far in the iteration, and it should return the updated
value.  If no initial value is supplied, \nil\ is used.  

%\cpar If a ~var~ is not specified, you can get \iter\ to declare the
%type of the internal variable by putting a |the| expression around
%~func.~  see section \ref{types}.

\cpar The differences between |accumulate| and |reducing| are slight.
One difference is that the functions take their arguments in a
different order.  Another is that in the absence of ~init-val~,
|accumulate| will use \nil, whereas |reducing| will generate different
code that avoids any dependence on the initial value.
The reason for having both clauses is that one usually
thinks of reductions (like |sum|) and accumulations (like |collect|)
as different beasts.

\end{clauses}


\subsubsection{Finders}

A ~finder~ is a clause whose value is an expression that meets some
condition.

\begin{clauses}

\clause{|finding| ~expr~ |such-that| ~test~ \opt\ |into| ~var~ |on-failure| ~failure-value~}
\clindexx{finding}{such-that}
If ~test~ (which is an expression) ever evaluates to \nonnil, the loop
is terminated, the 
epilogue code is run and the value of ~expr~ is returned.  Otherwise,
\nil\ (or ~failure-value,~ if provided) is returned.  If ~var~ is
provided, it will have either the \nonnil\ value of ~expr~ or
~failure-value~ when the epilogue code is run.

\cpar As a special case, if the ~test~ expression is a sharp-quoted
function, then it is applied to ~expr~ instead of being simply
evaluated.  E.g. |(finding x such-that \#'evenp)| is equivalent to
|(finding x such-that (evenp x))|.  

%\cpar although ~test~ need have nothing to do with ~%expr~ as in
%|(finding j such-that (> i 3))|, it usually 
%will: |(finding (length el) such-that (oddp (length el)))|.  to
%avoid performing the |length| computation twice, you could write
%|(finding (length el) such-that \#'oddp)| or |(finding (length
%el) such-that 'oddp)|; for these cases, \iter\ generates code that
%executes ~expr~ only once.  the code for |\#'oddp|
%is slightly different from that for {\lisp 'oddp}; see the discussion
%under {\lisp for\dots in} and {\lisp for\dots on}.

\cpar |On-failure| is a misnomer. Because it is always evaluated, it behaves
more like the default third argument to the |gethash| function. As a result,
|on-failure (error "Not found")| makes no sense. Instead, the clauses |leave|
or |thereis| can be used in conjunction with |finally| as follows:
\begin{program}
(iter (for x in '(1 2 3))
      (if (evenp x) (leave x))
      (finally (error "not found")))
\end{program}

\cpar This clause may appear multiple times when all defaults are
identical. It can also be used together with either |always|/|never| or
|thereis| if their defaults match. More specifically, |on-failure nil| is
compatible with |thereis|, while |on-failure t| is compatible with |always|
and |never| clauses.
\begin{program}
(iter (for i in '(7 -4 2 -3))
      (if (plusp i)
	  (finding    i  such-that (evenp i))
          (finding (- i) such-that (oddp i))))
\end{program}

\startitem
\Clause{|finding| ~expr~ |maximizing| ~m-expr~ \opt\ |into| ~var~}
\clindexx{finding}{maximizing}
\Clause{|finding| ~expr~ |minimizing| ~m-expr~ \opt\ |into| ~var~}
\clindexx{finding}{minimizing}
\finishitem
Computes the extremum (maximum or minimum) value of ~m-expr~ over all
iterations, and returns the value of ~expr~ corresponding to the
extremum.  ~expr~ is evaluated inside the loop at the time the new
extremum is established.  If ~m-expr~ is never evaluated (due to, for
example, being embedded in a conditional clause), then the returned
value depends on the type, if any, of ~expr~ (or ~var,~ if
one is supplied).  If there is no type, the returned
value will be \nil; if the type is numeric, the returned value will be
zero.

\cpar For these two clauses, ~var~ may be a list of two
symbols; in that case, the first is used to record ~expr~ and
the second, ~m-expr.~  

\cpar As with |finding\dots such-that|, if ~m-expr~ is a sharp-quoted
function, then it is called on ~expr~ instead of being evaluated.

\end{clauses}

\subsubsection{Boolean Tests}

\begin{clauses}

\clausex{first-iteration-p}{}
Returns |t| in the first cycle of the loop, otherwise \nil.

\clausex{first-time-p}{}
Returns |t| the first time the expression is evaluated, and then \nil\ forever.
This clause comes handy when printing (optional) elements separated
by a comma:

\begin{program}
(iter (for el in '(nil 1 2 nil 3))
      (when el
        (unless (first-time-p)
          (princ ", "))
        (princ el)))
\end{program}
produces |"1, 2, 3"|.

\end{clauses}

\subsubsection{Aggregated Boolean Tests}

\begin{clauses}

\clausex{always}{~expr~}
If ~expr~ ever evaluates to
\nil, then \nil\ is immediately returned; the epilogue code is not
executed.  If ~expr~ never evaluates to \nil, the epilogue code
is executed and the last value of ~expr~ (or |t| if ~expr~ was never
evaluated) is returned (whereas |loop| would constantly return |t|).

% mention last evaluated clause when multiple always clauses?

\clausex{never}{~expr~}
Like |(always (not ~expr~))|, except it does not influence the last
value returned by a possible other |always| clause. That is,
\begin{program}
(iter (repeat 2)
      (always 2)
      (never nil)) \yields 2 ; not t
\end{program}

\clausex{thereis}{~expr~}
If ~expr~ is ever \nonnil,
its value is immediately returned without running epilogue code.
Otherwise, the epilogue code is performed and \nil\ is returned.

This clause cannot be used together with |always| or |never|, because their
defaults are opposed (similarly, |(loop always 3 thereis nil)| refuses to
compile in some implementations of |loop|).

\end{clauses}


\subsection{Control Flow}
\label{control-flow}
Several clauses can be used to alter the usual flow of control in a loop.

Note: the clauses of this and subsequent sections don't adhere to \iter's
usual syntax, but instead use standard Common Lisp syntax.  Hence the
format for describing syntax subsequently is
like the standard format used in the Common Lisp manual, not like the
descriptions of clauses above.

\begin{clauses}

\clausex{finish}{}
Stops the loop and runs the epilogue code.

%for example:
%
%\begin{program}
%(iter (with answer = nil)
%         (initially (make-a-mess))
%         (for i from 1 to 10)
%         (when (correct? i)
%           (setq answer i)
%           (finish))
%         (finally (cleanup)))
%\end{program}
%
%this code will execute |cleanup| whether or not the test |(correct?
%i)| ever succeeds. 
%the (more elegant) formulation,
%\begin{program}
%(iter (initially (make-a-mess))
%         (for i from 1 to 10)
%         (finding i such-that (correct? i))
%         (finally (cleanup)))
%\end{program}
%would not execute |cleanup| if |(correct? i)| succeeded; it
%would do an immediate return.

\clausex{leave}{\opt\ ~value~}
Immediately returns ~value~ (default \nil) from the current \iter\
form, skipping the epilogue code.  Equivalent to using |return-from|.

\clausex{next-iteration}{}
Skips the remainder of the loop body and begins the next iteration of
the loop.

\clausex{while}{~expr~}
If ~expr~ ever evaluates to \nil, the loop is terminated and the
epilogue code executed.  Equivalent to |(if (not ~expr~) (finish))|.

\clausex{until}{~expr~}
Equivalent to |(if ~expr~ (finish))|.

\clausex{if-first-time}{~then~ \opt\ ~else~}
If this clause is being executed for the first time in this invocation
of the \iter\ form, then the ~then~ code is evaluated; otherwise the
~else~ code is evaluated.

\cpar |(for ~var~ first ~expr1~ then ~expr2~)| is almost equivalent to
\begin{program}
(if-first-time (dsetq ~var~ ~expr1~)
               (dsetq ~var~ ~expr2~))
\end{program}
The only difference is that the |for| version makes ~var~ available
for use with |for\dots previous|.

\end{clauses}

\subsection{Code Placement}
When fine control is desired over where code appears in a loop
generated by \iter, the following special clauses may be useful.
They are all subject to code-motion problems (see section
\ref{code-movement}).  

\begin{clauses}

\clausex{initially}{|\&rest| ~forms~}
The lisp ~forms~ are placed in the prologue section of the loop, where
they are executed once, before the loop body is entered.

\clausex{after-each}{|\&rest| ~forms~}
The ~forms~ are placed at the end of the loop body, where they
are executed after each iteration.  Unlike the other clauses in this
section, ~forms~ may contain \iter\ clauses.

\clausex{else}{|\&rest| ~forms~}
The lisp ~forms~ are placed in the epilogue section of the loop, where they
are executed if this |else| clause is never met during execution of the
loop and the loop terminates normally.

\clausex{finally}{|\&rest| ~forms~}
The lisp ~forms~ are placed in the epilogue section of the loop, where
they are executed after the loop has terminated normally.

\clausex{finally-protected}{|\&rest| ~forms~}
The lisp ~forms~ are placed in the second form of an unwind-protect
outside the loop.  They are always executed after the loop has
terminated, regardless of how the termination occurred.

\end{clauses}


%%%%%%%%%%%%%%%%%%%%%%%%%%%%%%%%%%%%%%%%%%%%%%%%%%%%%%%%%%%%%%%%
\section{Other Features}

\subsection{Multiple Accumulations}
\label{multiple}

\begin{sloppypar}
It is permitted to have more than one clause accumulate into the same
variable, as in the following:
\begin{program}
(iter (for i from 1 to 10)
      (collect i into nums)
      (collect (sqrt i) into nums)
      (finally (return nums)))
\end{program}
Clauses can only accumulate into the same variable if they are
compatible.  |collect|, |adjoining|, |appending|, |nconcing|,
|unioning| and |nunioning| are compatible with each other; |sum|,
|multiply| and |counting| are compatible; |always| and |never| are compatible;
|finding| \dots |such-that| is compatible with either |thereis| or |always|
and |never| when their defaults match; and |maximize| and |minimize| clauses
are compatible only with other |maximize| and |minimize| clauses,
respectively.

%note that the same variable ~cannot~ be both an accumulation
%variable and an ordinary variable; there can be only one variable with
%a given name within an \iter\ form. 
\end{sloppypar}

\subsection{Named Blocks}

Like Common Lisp |block|s, \iter\ forms can be given names.  The
name should be a single symbol, and it must be the first form in the
\iter.  The generated code behaves exactly like a named block; in
particular, |(return-from ~name~)| can be used to exit it:
\begin{program}
(iter fred
      (for i from 1 to 10)
      (iter barney
            (for j from i to 10)
            (if (> (* i j) 17)
                (return-from fred j))))
\end{program}
An \iter\ form that is not given a name is implicitly named \nil.

Sometimes one would like to write an expression in an inner \iter\ form,
but have it processed by an outer \iter\ form.  This is possible with
the |in| clause.  

\begin{clauses}

\clausex{in}{~name~ |\&rest| ~forms~}
Evaluates ~forms~ as if they were part of the \iter\ form named
~name~.  In other words, \iter\ clauses are processed by the \iter\
form named ~name,~ and not by any \iter\ forms that occur inside ~name.~

\cpar As an example, consider the problem of collecting a list of the
elements in a two-dimensional array.  The naive solution,
\begin{program}
(iter (for i below (array-dimension ar 0))
      (iter (for j below (array-dimension ar 1))
            (collect (aref ar i j))))
\end{program}
\noindent is wrong because the list created by the inner \iter\ is simply
ignored by the outer one.  But using |in| we can write:
\begin{program}
(iter outer (for i below (array-dimension ar 0))
      (iter (for j below (array-dimension ar 1))
            (in outer (collect (aref ar i j)))))
\end{program}
\noindent which has the desired result.

\end{clauses}

\subsection{Destructuring}
\label{destructuring}

In many places within \iter\ clauses where a variable is expected, a
list can be written instead.  In these cases, the value to be assigned
is ~destructured~ according to the pattern
described by the list.  As a simple example, the clause
\begin{program}
(for (key . item) in alist)
\end{program}
\noindent will result in |key| being set to the |car| of
each element in |alist|, and |item| being set to the |cdr|.  The
pattern list may be nested to arbitrary depth, and (as the example
shows) need not be terminated with \nil; the only requirement is that
each leaf be a bindable symbol (or \nil, in which case no binding is
generated for that piece of the structure).

Sometimes, you might like to do the equivalent of a
|multiple-value-setq| in a clause.  This 
``multiple-value destructuring'' can be expressed by writing 
\linebreak |(values $pat_1$ $pat_2 \ldots$)| for a destructuring
pattern, as in 
\begin{program}
(for (values (a . b) c d) = (three-valued-function ...))
\end{program}
\begin{sloppypar}
Note that the $pat_i$ can themselves be destructuring patterns (though
not multiple-value destructuring patterns).  You can't do multiple-value
destructuring in a |with| clause; instead wrap the whole \iter\
form in a |multiple-value-bind|.
\end{sloppypar}

\begin{note}{Rationale}
There are subtle interactions between variable declarations and
evaluation order that make the correct implementation of
multiple-value destructuring in a |with| somewhat tricky.
\end{note}

The destructuring feature of \iter\ is available as a separate
mechanism, using the |dsetq| macro:

\begin{clauses}
\defmacro{dsetq}{template expr}
Performs destructuring of ~expr~ using ~template~.   May be used
outside of an \iter\ form. Yields the primary value of ~expr~.

\end{clauses}

\subsection{On-line Help}

\begin{sloppypar}
There is a limited facility for on-line help, in the form of the 
|display-iterate-clauses| function.
\end{sloppypar}

\begin{clauses}

\defun{display-iterate-clauses}{\opt\ clause-spec}
Displays a list of \iter\ clauses.  If ~clause-spec~ is not
provided, all clauses are shown; if it is a symbol, all clauses
beginning with that symbol are shown; and if it is a list of symbols,
all clauses for which ~clause-spec~ is a prefix are shown.

\end{clauses}

\subsection{Parallel Binding and Stepping}
\label{bindings}

The parallel binding and stepping of variables is a feature that
\iter\ does ~not~ have.  This section attempts to provide a rationale.

We say that two variables are bound ~in parallel~ if neither
binding shadows the other.  This is the usual semantics of |let|
(as opposed to |let*|).  Similarly, we can say that iteration
variables are stepped in parallel if neither variable is updated
before the other, conceptually speaking; in other words, if the code
to update each variable can reference the old values of both variables.

|loop| allows parallel binding of variables and parallel stepping of
driver variables.  My view is that if you are depending on the 
serial/parallel distinction, you are doing something obscure.  If you
need to bind 
variables in parallel using |with|, then you must be using a
variable name that shadows a name in the existing lexical environment.
Don't do that.  The most common use for parallel stepping is to track
the values of variables on the previous iteration, but in fact this
does not require parallel stepping at all; the following will work:
\begin{program}
(iter (for current in list)
      (for prev previous current)
      ...)
\end{program}


%%%%%%%%%%%%%%%%%%%%%%%%%%%%%%%%%%%%%%%%%%%%%%%%%%%%%%%%%%%%%%%%
\section{Types and Declarations}
\label{types}

\subsection{Discussion}

Sometimes efficiency dictates that the types of variables be declared.
This type information needs to be communicated to \iter\ so it can
bind variables to appropriate values.  Furthermore, \iter\ must often
generate internal variables invisible to the user; there needs to be a
way for these to be declared.

As an example, consider this code, which will return the number of
odd elements in |number-list|:
\begin{program}
(iter (for el in number-list)
      (count (oddp el)))
\end{program}
In processing this form,
\iter\ will create an internal variable, let us call it |list17|, to
hold the successive |cdr|s of |number-list|, and  
will bind the variable to |number-list|.  It will also generate a
default binding for |el|; only inside the body of the loop will |el|
be set to the |car| of |list17|.  Finally, \iter\ will generate a
variable, call it |result|, to hold the result of the count, and will
bind it to zero.

When dealing with type declarations, \iter\ observes one simple rule:
~it will never generate a declaration unless requested to do so.~  The
reason is that such declarations might mask errors in compiled code by
avoiding error-checks; the resulting problems would be doubly hard to
track down because the declarations would be hidden from the
programmer.  Of course, a compiler might omit error-checks even in the
absence of declarations, though this behavior can usually be avoided,
e.g. by saying |(declaim (optimize (safety 3)))|.

So, the above \iter\ form will generate code with no declarations.
But say we wish to declare the types of |el| and the internal
variables |list17| and |result|.  How is this done?

Declaring the type of |el| is easy, since the programmer knows
the variable's name:
\begin{program}
(iter (for el in number-list)
      (declare (fixnum el))
      (counting (oddp el)))
\end{program}
\iter\ can read variable type declarations like this one.  Before
processing any clauses, it scans the entire top-level form for type
declarations 
and records the types, so that variable bindings can be performed
correctly.  In this case, |el| will be bound to zero
instead of \nil.  Also, \iter\ collects all the top-level declarations
and puts them at the begining of the generated code, so it is not
necessary to place all declarations at the beginning of an \iter\
form; instead, they can be written near the variables whose types they
declare.  

Since \iter\ is not part of the compiler, it will not know
about declarations that occur outside an \iter\ form; these
declarations must be repeated inside the form.

Here is another way we could have declared the type of |el|:
\begin{program}
(iter (for (the fixnum el) in number-list)
      (counting (oddp el)))
\end{program}
\iter\ extends the Common Lisp |the|\lispindex{the}
form to apply to variables as well as value-producing forms; anywhere
a variable is allowed---in a |with| clause, as the iteration
variable in a driver clause, as the |into| argument of an
accumulation clause, even inside a destructuring template---you can
write |(the ~type~ ~symbol~)| instead.

There is one crucial difference between using a |the| form and
actually declaring the variable: explicit declarations are always
placed in the generated code, but type information from a |the|
form is not turned into an actual declaration unless you tell \iter\
to do so using |iterate:declare-variables|.  See below.

\begin{sloppypar}
Declaring the types of internal variables is harder than declaring the
types of explicitly mentioned variables, since their names
are unknown.  You do it by declaring |iterate:declare-variables|
somewhere inside the top level of the \iter\ form.  (This will also
generate declarations for variables declared using |the|.)
\iter\ does not provide much selectivity here: it's all or none. 
And unfortunately, since \iter\ is not privy to compiler information
but instead reads declarations itself, it will not hear if you
|(declaim (iterate:declare-variables))|.  Instead, set the variable 
|iterate::*always-declare-variables*| to |t| at
compile-time, using |eval-when|.
\end{sloppypar}

To determine the appropriate types for internal variables, \iter\ uses
three sources of information:

\begin{itemize}

\item Often, the particular clause dictates a certain type for a
variable; \iter\ will use this information when available.  In the
current example, the variable |list17| will be given the type
|list|, since that is the only type that makes sense; and the
variable |result| will be given the type |fixnum|, on the
assumption that you will not be counting high enough to need bignums.
You can override this assumption only by using and explicitly declaring a
variable: 
\begin{verbatim}
(iter (declare (iterate:declare-variables))
      (for el in number-list)
      (count (oddp el) into my-result)
      (declare (integer my-result))
      (finally (return my-result)))
\end{verbatim}

\begin{sloppypar}
Other examples of the type assumptions that \iter\ makes are: type
|list| for |into| variables of collection clauses; type |list| for
expressions that are to be destructured; type |vector| for the
variable holding the vector in a |for\dots in-vector| clause, and
similarly for |string| and the |for\dots in-string| clause;
and the implementation-dependent type
|(type-of array-dimension-limit)| for the index and limit
variables generated by sequence iteration drivers like |for\dots
in-vector| and |for\dots in-string| (but not |for\dots in-sequence|,
because it may be used to iterate over a list). 
\end{sloppypar}

\item Sometimes, \iter\ will examine expressions and try to determine
their types in a simple-minded way.  If the expression is
self-evaluating (like a number, for instance), \iter\ knows that the
expression's type is the same as the type of the value it denotes, so
it can use that type.  If the expression is of the form |(the ~type~
~expr~)|, \iter\ is smart enough to extract ~type~ and use it.
However, the current version of \iter\ does not  
examine declarations of function result types or do any type
inference.  It will not determine, for 
example, that the type of |(+ 3 4)| is |fixnum|, or even
|number|.

\item In some cases, the type of an internal variable should match the
type of some other variable.  For instance, \iter\ generates an
internal variable for |(f x)| in the
clause |(for i from 1 to (f x))|, and in the absence of other
information will give it the same type as |i|.  If, however, the
expression had been written |(the fixnum (f x))|, then \iter\
would have given the internal variable the type |fixnum|
regardless of |i|'s type.  The type incompatibility errors that
could arise in this situation are not checked for.

\end{itemize}

Note that if you do declare |iterate:declare-variables|, then
\iter\ may declare user variables as well as internal ones if they do
not already have declarations, though only for variables that it
binds.  For instance, in this code:

\begin{program}
(iter (declare (iterate:declare-variables))
      (for i from 1 to 10)
      (collect i into var))
\end{program}
the variable |var| will be declared to be of type |list|.


\subsection{Summary}

\iter\ understands standard Common Lisp variable type declarations
that occur within an \iter\ form and
will pass them through to the generated code.  If the declaration
|(iterate:declare-variables)|\lispindex{declare-variables}
appears at the top level of an 
\iter\ form, or if 
|iterate::*always-declare-variables*|\lispindex{*always-declare-variables*} 
is \nonnil, then \iter\ will use the type information gleaned from user
declarations, self-evaluating expressions and |the| expressions,
combined with reasonable assumptions, to determine variable
types and declare them.


%%%%%%%%%%%%%%%%%%%%%%%%%%%%%%%%%%%%%%%%%%%%%%%%%%%%%%%%%%%%%%%%
\section{Problems with Code Movement}
\label{code-movement}

Some \iter\ clauses, or parts of clauses, result in code being
moved from the location of the clause to other parts of the loop.
Drivers behave this way, as do code-placement clauses like |initially|
and |finally|.  When using these clauses, there is a danger of writing
an expression that makes sense in its apparent location but will be
invalid or have a different meaning in another location.  For example:
\begin{program}
(iter (for i from 1 to 10)
      (let ((x 3))
        (initially (setq x 4))))
\end{program}
While it may appear that the |x| of |(initially (setq x 4))| is the
same as the |x| of |(let ((x 3)) \dots|, in fact they are not:
|initially| moves its code outside the loop body, so |x| would refer
to a global variable.  Here is another example of the same problem:
\begin{program}
(iter (for i from 1 to 10)
      (let ((x 3))
        (collect i into x)))
\end{program}
If this code were executed, |collect| would create a binding for its
|x| at the top level of the \iter\ form that the |let| will shadow.

Happily, \iter\ is smart enough to catch these errors; it
walks all problematical code to ensure that free variables are not
bound inside the loop body, and checks all variables it binds for the
same problem.

However, some errors cannot be caught:

\begin{program}
(iter (with x = 3)
      (for el in list)
      (setq x 1)
      (reducing el by \#'+ initial-value x))
\end{program}
|reducing| moves its |initial-value| argument to the initialization
part of the loop in order to produce more efficient code.  Since
\iter\ does not perform data-flow analysis, it cannot determine that
|x| is changed inside the loop; all it can establish is that |x| is
not bound internally.  Hence this code will not signal an
error and will use $3$ as the initial value of the reduction.

The following list summarizes all cases that are subject to these code
motion and variable-shadowing problems.
\begin{itemize}
\item Any variable for which \iter\ creates a binding, including those
used in |with| and the |into| keyword of many clauses.

\begin{sloppypar}
\item The special clauses which place code: |initially|, |after-each|, |else|,
|finally| and |finally-protected|. 
\end{sloppypar}

\item The variables of a |next| or |do-next| form.

\item The |initially| arguments of |for\dots initially\dots then| and
|for\dots previous|. 

\item The |then| argument of |for\dots initially\dots then|.

\item The |initial-value| arguments of |reducing| and |accumulate|.

\item The |on-failure| argument of |finding\dots such-that|.
\end{itemize}

%%%%%%%%%%%%%%%%%%%%%%%%%%%%%%%%%%%%%%%%%%%%%%%%%%%%%%%%%%%%%%%%
\section{Differences Between {\tt Iterate} and {\tt Loop}}
|loop| contains a great deal of complexity which \iter\ tries to
avoid.  Hence many esoteric features of |loop| don't exist in \iter.
Other features have been carried over, but in a cleaned-up form.  
And of course, many new features have been added; they are not
mentioned in this list.

\begin{itemize}

\item \iter's syntax is more Lisp-like than |loop|'s, having a higher
density of parens.

\item The current implementation of \iter, unlike the current version
of  |loop| (as documented in {\em Common Lisp, 2nd Ed.\/}), is
extensible (see section \ref{extend}).  

\item |loop| puts the updates of all driver variables at the top of
the loop; \iter\ leaves them where the driver clauses appear.

\item While for the most part \iter\ clauses that resemble |loop| clauses
behave similarly, there are some differences.  For instance, there is
no |for\dots =\dots then| in \iter; instead use 
|for\dots initially\dots then|.  

\item |loop| binds the variable |it| at certain times
to allow pseudo-English expressions like |when ~expr~ return it|. 
In \iter, you must bind ~expr~ to a variable yourself.  Note that
|when ~expr~ return it| is like |thereis ~expr~| except that the latter is an
accumulation clause and therefore competes with other accumulations
(remember section \ref{multiple} above).

% repeat different behaviour of |always| clause here?

\item |loop| has a special |return| clause, illustrated in the
previous item.  \iter\ doesn't need one,  since an ordinary Lisp
|return| has the same effect.

\item |loop| allows for parallel binding and stepping of iteration
variables.  \iter\ does not.  (See section \ref{bindings}.)

\item |loop| and \iter\ handle variable type declarations very
differently.  |loop| provides a special syntax for declaring variable
types, and does not examine declarations.  Moreover, the standard
implementation of |loop| will
generate declarations when none are requested.
\iter\ parses standard Common Lisp type declarations, and will never
declare a variable itself unless declarations are 
specifically requested.

\end{itemize}

%%%%%%%%%%%%%%%%%%%%%%%%%%%%%%%%%%%%%%%%%%%%%%%%%%%%%%%%%%%%%%%%
\section{Rolling Your Own}
\label{extend}

\subsection{Introduction}

\iter\ is extensible---you can write new clauses that embody new
iteration patterns.  You might want to write a new driver clause for a
data structure of your own, or you might want to write a clause that
collects or manipulates elements in a way not provided by \iter. 

This section describes how to write clauses for \iter.  Writing a
clause is like writing a macro.  In fact, writing a clause ~is~
writing a macro: since \iter\ code-walks its body and macroexpands,
you can add new abstractions to \iter\ with good old |defmacro|.

Actually, there are two extensions you can make to \iter\ that are
even easier than writing a macro.  They are adding a synonym for an
existing clause and defining a driver clause for an indexable
sequence.  These can be done with |defsynonym| and
|defclause-sequence|, respectively.  See section \ref{aids}, below.

The rest of this section explains how to write macros that expand into
\iter\ clauses.
Here's how you could add a simplified version of \iter's |multiply|
clause, if \iter\ didn't already have one:
\begin{program}
(defmacro multiply (expr)
  `(reducing ,expr by \#'* initial-value 1))
\end{program}

If you found yourself summing the square of an expression often, you
might want to write a macro for that.  A first cut might be
\begin{program}
(defmacro sum-of-squares (expr)
  `(sum (* ,expr ,expr)))
\end{program}
but if you are an experienced macro writer, you will realize that this
code will evaluate ~expr~ twice, which is probably a bad idea.  A
better version would use a temporary:
\begin{program}
(defmacro sum-of-squares (expr)
  (let ((temp (gensym)))
    `(let ((,temp ,expr))
       (sum (* ,temp ,temp)))))
\end{program}
Although this may seem complex, it is just the sort of thing you'd
have to go through to write any macro, which illustrates the point of
this section: if you can write macros, you can extend \iter.

Our macros don't use \iter's keyword-argument
syntax.  We could just use keywords with |defmacro|, but we would
still not be using \iter's clause indexing mechanism.  Unlike Lisp,
which uses just the first symbol of a form to determine what function
to call, \iter\ individuates clauses by the list of required keywords.
For instance, |for\dots in| and |for\dots in-vector| are different clauses
implemented by distinct Lisp functions.  

To buy into this indexing scheme, as well as the keyword-argument
syntax, use |defmacro-clause|:

\begin{clauses}

\defmacro{defmacro-clause}{arglist |\&body| body}
Defines a new \iter\ clause.  ~arglist~ is a list of symbols which are
alternating keywords and arguments.  \opt\ may be used, and the list
may be terminated by |\&sequence|.  ~body~ is an ordinary macro body,
as with |defmacro|.  If the first form of ~body~ is a string, it is
considered a documentation string and will be shown by
|display-iterate-clauses|. |defmacro-clause| will signal an error if 
defining the clause would result in an ambiguity.  E.g. you cannot
define the clause |for\dots from| because there would be no way to
distinguish it from a use of the |for| clause with optional keyword |from|.

\end{clauses}

\medskip

Here is |multiply| using |defmacro-clause|.  The keywords are capitalized
for readability.
\begin{program}
(defmacro-clause (MULTIPLY expr \opt\ INTO var)
  `(reducing ,expr by \#'* into ,var initial-value 1))
\end{program}
You don't have to worry about the case when |var| is not supplied; for
any clause with an |into| keyword, saying |into nil| is equivalent to
omitting the |into| entirely.

As another, more extended example, consider the fairly common
iteration pattern that 
involves finding the sequence element that maximizes (or minimizes) some
function.  \iter\ provides this as |finding\dots maximizing|, but it's
instructive to see how to write it.
Here, in pseudocode, is how you might write such a loop for
maximizing a function F:

\begin{quote}
\begin{tabbing}
set variable MAX-VAL to NIL; \\
set variable WINNER to NIL; \\
for\=\ each element EL in the sequence \\
\>  if\=\ MAX-VAL\=\ is NIL or F(EL) $>$ MAX-VAL then \\
\>\>    set MAX-VAL to F(EL); \\
\>\>    set WINNER to EL; \\
\>  end if; \\
end for; \\
return WINNER.
\end{tabbing}
\end{quote}

Here is the macro:
\begin{program}
(defmacro-clause (FINDING expr MAXIMIZING func \opt\ INTO var)
  (let ((max-val (gensym))
        (temp1 (gensym))
        (temp2 (gensym))
        (winner (or var iterate::*result-var*)))
    `(progn 
       (with ,max-val = nil)
       (with ,winner = nil)
       (cond 
        ((null ,max-val)
         (setq ,winner ,expr)
         (setq ,max-val (funcall ,func ,winner))
        (t
         (let* ((,temp1 ,expr)
                (,temp2 (funcall ,func ,temp1)))
           (when (> ,temp2 ,max-val)
             (setq ,max-val ,temp2)
             (setq ,winner ,temp1))))))
       (finally (leave ,winner)))))
\end{program}
Note that if no |into| variable is supplied, we use
|iterate::*result-var*|, which contains the internal variable into
which all clauses place their results.  If this variable is bound by
some clause, then \iter\ will return its value automatically;
otherwise, \nil\ will be returned.  

\subsection{Writing Drivers}

In principle, drivers can be implemented just as easily as other
\iter\ clauses.  In practice, they are a little harder to get right. 
As an example, consider writing a driver that
iterates over all the 
elements of a vector, ignoring its fill-pointer.  |for\dots in-vector|
won't work for this, because it observes the fill-pointer.  It's necessary to
use |array-dimension| instead of |length| to obtain the size of the
vector.  Here is one approach:

\begin{program}
(defmacro-clause (FOR var IN-WHOLE-VECTOR v)
  "All the elements of a vector (disregards fill-pointer)"
  (let ((vect (gensym))
        (index (gensym)))
    `(progn
       (with ,vect = ,v)
       (for ,index from 0 below (array-dimension ,vect 0))
       (for ,var = (aref ,vect ,index)))))
\end{program}
Note that we immediately put |v| in a variable, in case it is an
expression.  Again, this is just good Lisp macrology.  It also has a
subtle effect on the semantics of the driver: |v| is evaluated only
once, at the beginning of the loop, so changes to |v| in the loop have
no effect on the driver.  Similarly, the bounds for numerical iteration
e.g. the above |array-dimension| are also evaluated once only.  This is how
all of \iter's drivers work.


There is an important point concerning the |progn| in this code.  We
need the |progn|, of course, because we are returning several forms,
one of which is a driver.  But \iter\ drivers must occur at top-level.
Is this code in error?  No, because ~top-level~ is defined in \iter\
to include forms inside a |progn|.  This is just the definition of
top-level that Common Lisp uses, and for the same reason: to allow
macros to return multiple forms at top-level.

While our |for\dots in-whole-vector| clause will work, it is not
ideal.  In particular, it does not support generating.  Do do so, we
need to use |for\dots next| or |for\dots do-next|.  The job is
simplified by the |defmacro-driver| macro.

\begin{clauses}

\defmacro{defmacro-driver}{arglist |\&body| body}
Defines a driver clause in
both the |for| and |generate| forms, and provides a parameter
|generate| which ~body~ can examine to determine how it was invoked.
~arglist~ is as in |defmacro-clause|, and should begin with the symbol
|for|. 

\end{clauses}

With |defmacro-driver|, our driver looks like this:
\begin{program}
(defmacro-driver (FOR var IN-WHOLE-VECTOR v)
  "All the elements of a vector (disregards fill-pointer)"
   (let ((vect (gensym))
         (end (gensym))
         (index (gensym))
         (kwd (if generate 'generate 'for)))
     `(progn
        (with ,vect = ,v)
        (with ,end = (array-dimension ,vect 0))
        (with ,index = -1)
        (,kwd ,var next (progn (incf ,index)
                               (if (>= ,index ,end) (terminate))
                               (aref ,vect ,index))))))
\end{program}


We are still missing one thing: the |\&sequence| keywords.  
We can get them easily enough, by writing 
\begin{program}
(defmacro-driver (FOR var IN-WHOLE-VECTOR v \&sequence) 
  ...)
\end{program}
We can now refer to parameters |from|, |to|, |by|, etc. which contain
either the values for the corresponding keyword, or \nil\ if the
keyword was not supplied.  Implementing the right code for these
keywords is cumbersome but not difficult; it is left as an exercise.
But before you begin, see |defclause-sequence| below for an easier way.
       
\subsection{Extensibility Aids}
\label{aids}

This section documents assorted features that may be of use in
extending \iter.

\begin{clauses}

\defunexpvar{*result-var*}
Holds the variable that is used to return a value as a result of the
\iter\ form.  You may examine this and use it in a |with| clause, but
you should not change it.

\defmacro{defsynonym}{syn word}
 Makes ~syn~ a synonym for the existing \iter\ keyword ~word.~  Only
the first word in each clause can have synonyms. 

\defmacroxx{defclause-sequence}{element-name index-name}{|\&key|
access-fn size-fn sequence-type}{element-type
element-doc-string index-doc-string}  
Provides
a simple way to define sequence clauses.  Generates two
clauses, one for iterating over the sequence's elements, the other
for iterating over its indices.  The first symbol of both
clauses will have print-name |for|.
~element-name~  and ~index-name~ should be symbols.
~element-name~ is the second keyword of the element iterator (typically of
the form 
|in-~sequence-type~|), and ~index-name~ is the second keyword
of the index-iterator (typically of the form
|index-of-~sequence-type~|).  Either name may be 
\nil, in which case the corresponding clause is not defined.  If both
symbols are supplied, they should be in the same package.  The |for|
that begins the clauses will be in this package.

\cpar ~access-fn~ is the function to be used to
access elements of the sequence in the element iterator.  The function
should take two 
arguments, a sequence and an index, and return the appropriate element.
~size-fn~ should denote a function of one argument, a sequence, that
returns its size.  Both ~access-fn~ and ~size-fn~ are required for the
element iterator, but only ~size-fn~ is needed for the index iterator.

\cpar The ~sequence-type~ and ~element-type~ keywords are used to
suggest types for the variables 
used to hold the sequence and the
sequence elements, respectively.  The usual rules about \iter's
treatment of variable type declarations apply (see section \ref{types}).

\cpar ~element-doc-string~ and ~index-doc-string~ are
the documentation strings, for use with |display-iterate-clauses|. 

\cpar The generated element-iterator performs destructuring on the
element variable.

\cpar As an example, the above |for\dots in-whole-vector| example
could have been written:
\begin{program}
(defclause-sequence IN-WHOLE-VECTOR INDEX-OF-WHOLE-VECTOR
  :access-fn 'aref
  :size-fn \#'(lambda (v) (array-dimension v 0))
  :sequence-type 'vector
  :element-type t
  :element-doc-string 
     "Elements of a vector, disregarding fill-pointer"
  :index-doc-string 
     "Indices of vector, disregarding fill-pointer")
\end{program}

\end{clauses}

\subsection{Subtleties}

There are some subtleties to be aware of when writing \iter\ clauses.
First, the code returned by your macros may be |nconc|'ed into a list,
so you should always returned freshly consed lists, rather than
constants.  Second, \iter\ matches clauses by using |eq| on the first
symbol and |string=| on the subsequent ones, so the package of the
first symbol of a clause is relevant.  All of the clauses in this manual
have their first word in the \iter\ package.  
You can use the package system in the usual way to shadow
\iter\ clauses without replacing them.

%%% say more here, about the badness that only the first word of a
%%% clause is packagey.

\section{Non-portable Extensions to Iterate (Contribs)}
\label{contribs}

Currently, there is only one non-portable extension to iterate in the
distribution: iterate-pg. If you have made an extension that depends
on non-portable features, feel free to send them to |asf@boinkor.net|
for inclusion in the iterate distribution.

\subsection{An SQL query driver for iterate}

The pg package by Eric Marsden (see |http://cliki.net/pg|) provides an
interface to the PostgreSQL database. Using the \iterpg\ extension, it
is possible to handle the results of SQL queries using \iter.

This usage example should give you an idea of how to use it:

\begin{program}
(pg:with-pg-connection (c "somedb" "someuser")
  (iter (for (impl version date) in-relation "select * from version"
                                 on-connection *dbconn*)
        (collect version)))
\end{program}

To use the extension via |ASDF|, simply make your system depend on the
|iterate-pg| system instead of the |iterate| system. To load it
manually, use:

\begin{program}
  (asdf:oos 'asdf:load-op :iterate-pg)
\end{program}

%%%%%%%%%%%%%%%%%%%%%%%%%%%%%%%%%%%%%%%%%%%%%%%%%%%%%%%%%%%%%%%%
\section{Obtaining {\tt Iterate}}

\begin{sloppypar}
The information in this chapter is obsolete but included for
completeness's sake; Currently, the most up-to-date information on
\iter\ can be found at |http://boinkor.net/iterate.html|.
\end{sloppypar}

\begin{sloppypar}
\iter\ currently runs on Lisp Machines, and on
HP's, Sun3's and Sparcstations under Lucid.  
\iter\ source and binaries are available at the MIT AI Lab in the
subdirectories of |/src/local/lisplib/|.  The source file,
|iterate.lisp|, is also available for anonymous FTP in the directory
|/com/fpt/pub/| on the machine |TRIX.AI.MIT.EDU| (Internet number
128.52.32.6).  If you are unable to obtain |iterate| in one of these
ways, send mail to |jba@ai.mit.edu| and I will send you the source
file. 
\end{sloppypar}

\begin{sloppypar}
\iter\ resides in the |iterate| package (nickname |iter|).  Just say
\linebreak |(use-package :iterate)| to make all the necessary symbols
available. 
If a symbol is not exported, it appears in this manual with an
``|iterate::|'' prefix.
\end{sloppypar}

Send bug reports to |bug-iterate@ai.mit.edu|.  The |info-iterate|
mailing list will have notices of changes and problems; to have
yourself added, send mail to |info-iterate-request@ai.mit.edu|.


\medskip

\begin{flushleft}
 \bf Acknowledgements
\end{flushleft}
\smallskip

Richard Waters provided invaluable criticism which spurred me to improve
\iter\ greatly.  As early users, David Clemens, Oren Etzioni and Jeff
Siskind helped ferret out many bugs.

%\begin{theindex}
% The index files must be generated with the genindex program
% or from makeindex iterate-manual.idx which creates iterate-manual.ind.

%\baselineskip\normalbaselineskip
%\advance\baselineskip by -2pt

% Fix ``.'' spacing in lists
\documentstyle[12pt,aimemo]{article}

\setlength{\textwidth}{6.0in}
\setlength{\oddsidemargin}{0.25in}
\setlength{\evensidemargin}{0.25in}
\setlength{\topmargin}{0pt}
\setlength{\headsep}{24pt}
\setlength{\textheight}{8.5in}

\let\origpar=\par

% somehow, LaTex fucks with things so that the plain Tex \obeyspaces
% doesn't work.  This is the same, except we use an mbox containing
% a space instead of just a space.

\newcommand{\spc}{\mbox{ }}
\newcommand{\tAb}{\mbox{        }}
\def\obspaces{\catcode`\ =\active\catcode`\^^I=\active}
{\obspaces \global\let =\spc\global\let^^I=\tAb}

\newenvironment{program}{\medskip\samepage\tt\obeylines\obspaces}{\medskip}

% use | for \tt
\catcode`\|=13
\newif\ifvbar
\def|{\ifvbar
         \endgroup\vbarfalse
      \else 
         \begingroup\tt\vbartrue 
      \fi}



\newcommand{\lisp}{\tt}
\newcommand{\iter}{|iterate|}
\newcommand{\iterpg}{|iterate-pg|}
\newcommand{\setf}{|setf|}
\newcommand{\nil}{|nil|}
\newcommand{\nonnil}{non-\nil}
\newcommand{\opt}{{\lisp \&optional}}
\newcommand{\yields}{$\Rightarrow$}



\newbox\Dots
\setbox\Dots=\hbox{\small ...}
\renewcommand{\dots}{\copy\Dots}

% Here's a way to define an environment which will treat all
% paragraphs inside it in the same manner.  It's, as near as I can
% figure, the way that Latex does things.  Using \everypar won't work;
% apparently Latex resets that a lot.  What you have to do is: at the
% top of the file do \let\origpar=\par.  In the
% begin part of the environment, do the parshape (or whatever), then say
% \def\par{{\origpar}}.  The extra braces are necessary.   That's all
% you have to do; at the end of the environment things will return to
% normal automatically.  Why does this all work?  I have no idea.
% Another important thing: there should be a blank line (i.e. a \par)
% between the end of the last paragraph and the \end{environment}.

\newlength{\clindent}
\setlength{\clindent}{0.5in}
\newlength{\clparindent}
\setlength{\clparindent}{\parindent}

\newenvironment{clauses}{\advance\linewidth -\parindent
                        \hangindent\clindent\relax
                         \hangafter=0
                          \parindent=0pt
                          \def\par{{\origpar}}}{}

\newcommand{\cpar}{\hspace \clparindent}

\newcommand{\startitem}{\bigskip\pagebreak[3]}
\newcommand{\finishitem}{\smallskip}

\newcommand{\clause}[1]{\startitem\Clause{#1}\finishitem}
\newcommand{\Clause}[1]{\hbox{#1}}
\newcommand{\clindex}[1]{\index{{\lisp #1}}}
\newcommand{\clindexx}[2]{\index{{\lisp #1\dots #2}}}
\newcommand{\clindexxx}[3]{\index{{\lisp #1\dots #2\dots #3}}}
\newcommand{\clausex}[2]{\startitem\Clausex{#1}{#2}\finishitem}
\newcommand{\Clausex}[2]{\Clause{|#1| #2}\clindex{#1}}

\newcommand{\defvar}[1]{\startitem\deff{#1}{}{Variable}\finishitem}
\newcommand{\defunexpvar}[1]{\startitem\deffnoindex{iterate::#1}{}{Variable}
                                        \lispindex{#1}\finishitem}
\newcommand{\defun}[2]{\startitem\deff{#1}{#2}{Function}\finishitem}
\newcommand{\defmacro}[2]{\startitem\deff{#1}{#2}{Macro}\finishitem}
\newcommand{\defunx}[3]{\startitem\deffx{#1}{#2}{#3}{Function}\finishitem}
\newcommand{\defmacrox}[3]{\startitem\deffx{#1}{#2}{#3}{Macro}\finishitem}
\newcommand{\defunxx}[4]{\startitem\deffxx{#1}{#2}{#3}{#4}{Function}\finishitem}
\newcommand{\defmacroxx}[4]{\startitem\deffxx{#1}{#2}{#3}{#4}{Macro}\finishitem}

\newcommand{\deff}[3]{\deffnoindex{#1}{#2}{#3}\lispindex{#1}}
\newcommand{\deffnoindex}[3]{\hbox to \hsize{{\tt #1} {\it #2}\hfill [{\it #3}]}}
\newcommand{\deffx}[4]{\deff{#1}{#2}{#4}\moreargs{#1}{#3}}
\newcommand{\deffxx}[5]{\deffx{#1}{#2}{#3}{#5}\moreargs{#1}{#4}}

\newcommand{\moreargs}[2]{\hbox to \hsize{\phantom{\lisp #1} {\it #2}\hfill}}

\newcommand{\lispindex}[1]{\index{{\lisp #1}}}

 \newenvironment{note}[1]{\pagebreak[2]\bigskip
                          \hrule\smallskip\small
                         {\setlength{\parindent}{0pt}
                           \par{\bf #1:}}}{\smallskip\hrule\bigskip}



\makeindex

%%%%%%%%%%%%%%%%%%%%%%%%%%%%%%%%%%%%%%%%%%%%%%%%%%%%%%%%%%%%%%%%

%\thispagestyle{empty}

%\vspace*{2in}

%\title{The |Iterate| Manual}
%\author{Jonathan Amsterdam}

%\maketitle

%\thispagestyle{empty}

%\begin{center}
%\Large *** DRAFT ***
%\end{center}
%This manual is in beta-test. It will become an AI Memo in a couple of
%months.  I would appreciate any comments or
%corrections.  You can reach me at email address |jba@ai.ai.mit.edu|.

\begin{document}

\def\writtenby{Jonathan Amsterdam}
\def\contractno{This report describes research done at the Artificial
Intelligence Laboratory of the Massachusetts Institute of Technology.
Support for the laboratory's artificial intelligence research is
provided in part by the Advanced Research Projects Agency of the
Department of Defense under Office
of Naval Research contract N00014-85-K-0124.}

\AIMEMO{1236}{\Large The Iterate Manual \\[1in]}
{This is the manual for version 1.4 of \iter, a powerful iteration
macro for Common  
Lisp. \iter\ is similar to {\tt loop} but provides numerous additional
features, is well integrated with Lisp, and is extensible. 
\vfill }


% use ~ for \em
\catcode`\~=13
\newif\ifem
\def~{\ifem
         \/\endgroup\emfalse
      \else 
         \begingroup\em\emtrue 
      \fi}



%\pagebreak
%\setcounter{page}{1}

%\thispagestyle{empty}


\tableofcontents

\pagebreak
\pagestyle{headings}

\section{Introduction}

\begin{sloppypar}
This manual describes
\iter, a powerful iteration facility for Common Lisp.
\iter\ provides abstractions for many common iteration
patterns and allows for the definition of additional patterns.  \iter\ is
a macro that expands into ordinary Lisp at
compile-time, so it is more efficient than higher-order
functions like |map| and |reduce.|  
While it is similar to |loop|, \iter\ offers a more Lisp-like syntax
and enhanced extensibility.
(For a more complete
comparison of \iter\ with other iteration constructs, see MIT AI Lab Working
Paper No. 324, ~Don't Loop, Iterate.~)
\end{sloppypar}

An \iter\ form consists of the symbol |iter|\footnote{You can also use
|iterate|, but |iter| is preferred because it avoids potential conflicts with
possible future additions to Common Lisp, and because it saves
horizontal space when writing code.}
followed by one or more
forms, some of which may be \iter\ 
~clauses.~  Here is a simple example of \iter\ which collects the
numbers from 1 to 10 into a list, and returns the list.  The return
value is shown following the arrow.
\begin{program}
(iter (for i from 1 to 10)
      (collect i))          \yields (1 2 3 4 5 6 7 8 9 10)
\end{program}
This form contains two clauses: a |for| clause that steps the
variable |i| over the integers from 1 to 10, and a |collect|
clause that accumulates its argument into a list.  With a few
exceptions, all \iter\ clauses have the same format: alternating
symbols (called ~keywords~) and expressions (called
~arguments~).  The syntax and terminology are those of Common
Lisp's keyword lambda lists.  One difference is that \iter's keywords
do not have to begin with a colon---though they may, except for the
first symbol of a clause.  So you can also write  |(for i :from
1 :to 10)| if you prefer.

Any Lisp form can appear in the body of an \iter, where it will have
its usual meaning.  \iter\ walks the entire body, expanding macros,
and recognizing clauses at any level.  This example collects all the
odd numbers in a list:
\begin{program}
(iter (for el in list)
      (if (and (numberp el) (oddp el))
          (collect el)))
\end{program}

There are clauses for iterating over numbers, lists, arrays and other
objects, and for
collecting, summing, counting, maximizing and other useful operations.
\iter\ also supports the creation of new variable bindings, stepping
over multiple sequences at once, destructuring, and compiler
declarations of variable types.  The following example illustrates
some of these features: 
\begin{program}
(iter (for (key . item) in alist)
         (for i from 0)
         (declare (fixnum i))
         (collect (cons i key)))
\end{program}

This loop takes the keys of an alist and returns a new alist
associating the keys with their positions in the original list. The
compiler declaration for |i| will appear in the generated code
in the appropriate place.

%%%%%%%%%%%%%%%%%%%%%%%%%%%%%%%%%%%%%%%%%%%%%%%%%%%%%%%%%%%%%%%%
\section{Clauses}

Most of \iter's clauses will be familiar to |loop| programmers.  
(|loop| is an iteration macro that has been incorporated into Common
Lisp.  Seen Guy Steele's {\em Common Lisp, 2nd Edition}.)
In
nearly all cases they behave the same as their |loop| counterparts, so
a |loop| user can switch to \iter\ with little pain (and much gain).

All clauses with the standard keyword-argument syntax consist of two
parts: a ~required~ part, 
containing keywords that must be present and in the right order; and
an ~optional~ part, containing keywords that may be omitted and,
if present, may occur in any order.  In the descriptions below, the
parts are separated by the Lisp lambda-list keyword \opt.


\subsection{Drivers}

An iteration-driving clause
conceptually causes the iteration to go forward.  Driver clauses in
\iter\ allow iteration over numbers, lists, vectors, hashtables, packages,
files and streams.  Iteration-driving clauses must
appear at the top level of an \iter\ form; they cannot be nested
inside another clause.  The driver variable is updated at the point
where the driver clause occurs.  Before the clause is executed for the
first time, the value of the variable is undefined.

%Also, regardless of where the driver clause
%appears in the body, the driver variable is stepped at the top of the
%loop; hence it is stylistically preferable, though
%not required, to place driver clauses at the beginning of the \iter.

Multiple drivers may appear in a single \iter\ form, in which case all
of the driver variables are updated each time through the loop, in the
order in which the clauses appear.  The first driver to terminate will
terminate the entire loop.

In all cases, the value of the driver variable on exit from the loop,
including within the epilogue code (see the |finally| clause), is
undefined.

All the parameters of a driver clause are evaluated once, before the
loop begins.  Hence it is not possible to change the bounds or other
properties of an iteration by side-effect from within the loop.

With one exception, driver clauses begin with the word |for| (or
the synonym |as|) and mention an iteration variable, which is
given a binding within the \iter\ form.
The exception is |repeat|, which just executes a loop a
specified number of times:

\begin{clauses}

\clausex{repeat}{~n~}
Repeats the loop ~n~ times.  For example:
\begin{program}
(iter (repeat 100)
      (print "I will not talk in class."))
\end{program}
\cpar If $n \leq 0$, the
loop will never be executed.  If ~n~ is not an integer, the
actual number of executions will be $\lceil n \rceil$.

\end{clauses}

\pagebreak[3]

\subsubsection{Numerical Iteration}

\begin{clauses}
\clausex{for}{~var~ |\&sequence|} 
The general form for iterating over a sequence of numbers requires a
variable and, optionally, one or more keywords that provide the bounds
and step size of the iteration. The |\&sequence|
lambda-list keyword is a shorthand for these sequence keywords.  They are:
|from|, |upfrom|, |downfrom|, |to|, |downto|, |above|, |below| and |by|.
|from| provides the starting value for ~var~
and defaults to zero. |to| provides a final value and implies that the
successive values of ~var~ will be increasing; |downto| implies that
they will be decreasing.  The loop terminates when ~var~ passes
the final value (i.e. becomes smaller or larger than it, depending on
the direction of iteration); in other  words, the loop body
will never be 
executed for values of ~var~ past the final value.
|below| and |above| are similar to |to| and
|downto|, except that the loop terminates when ~var~ equals or passes
the final value.

\cpar If no final value is specified, the variable will be stepped
forever.  Using |from| or |upfrom| will result in increasing
values, while |downfrom| will give decreasing values.

\cpar On each iteration,
~var~ is incremented or decremented by the value of the sequence
keyword |by|, which defaults to 1.  It should always be a positive
number, even for downward iterations.

\cpar In the following examples, the sequence of numbers generated is shown
next to the clause.

\begin{program}
(for i upfrom 0) \yields\ 0 1 2 \ldots
(for i from 5)   \yields\ 5 6 7 \ldots    ; either from or upfrom is okay
(for i downfrom 0) \yields\ 0 -1 -2 \ldots
(for i from 1 to 3) \yields\ 1 2 3
(for i from 1 below 3) \yields\ 1 2
(for i from 1 to 3 by 2) \yields\ 1 3
(for i from 1 below 3 by 2) \yields\ 1
(for i from 5 downto 3) \yields\ 5 4 3
\end{program}

\end{clauses}

\subsubsection{Sequence Iteration}

There are a number of clauses for iterating over sequences.  In all of
them, the argument following |for| may be a list instead of a
symbol, in which case destructuring is performed.  See section
\ref{destructuring}. 

\begin{clauses}

\clause{|for| ~var~ |in| ~list~ \opt\ |by| ~step-function~}
\clindexx{for}{in}
~var~ is set to successive elements of list.  
~step-function,~ which 
defaults to |cdr|, is used to obtain the next sublist. 

\clause{|for| ~var~ |on| ~list~ \opt\ |by| ~step-function~}
\clindexx{for}{on}
~var~ is set to successive sublists of list. 
~step-function~ (default |cdr|) is used as in |for\dots in|.

\end{clauses}

\medskip

These two clauses use |atom| to test for the end
of a list.  Hence, given a list whose final |cdr| is not \nil,
they will silently ignore the last |cdr|.  Other choices are
|endp|,  which would signal an error, and |null|, which
would probably result in an error somewhere else.  If you wish to use
an end-test other than |atom|, set the variable
|iterate::*list-end-test*|\lispindex{*list-end-test*} to the name of
the desired function. 

\begin{clauses}
\clause{|for| ~var~ |in-vector| ~vector~ |\&sequence|}
\clindexx{for}{in-vector}
~var~ takes on successive elements from ~vector.~   The vector's
fill-pointer is observed.  Here and in subsequent clauses, the |\&sequence|
keywords include |with-index|, which takes a symbol as argument
and uses it for the index variable instead of an internally generated
symbol.  The other |\&sequence| keywords behave as in numerical
iteration, except that the default iteration bounds are the bounds of
the vector.  E.g. in |(for i in-vector v downto 3)|,
|i| will start off being bound to the last element in |v|,
and will  
be set to preceding elements down to and including the element with
index 3.

\clause{|for| ~var~ |in-sequence| ~seq~ |\&sequence|}
\clindexx{for}{in-sequence}
This uses Common Lisp's generalized sequence functions, |elt|
and |length|, to obtain elements and determine the length of
~seq.~   Hence it will work for any sequence, including lists, and will
observe the fill-pointers of vectors.

\clause{|for| ~var~ |in-string| ~string~ |\&sequence|}
\clindexx{for}{in-string}
~var~ is set to successive characters of ~string.~

\startitem
\Clause{|for| ~var~ |index-of-vector| ~vector~ |\&sequence|}
\clindexx{for}{index-of-vector}
\Clause{|for| ~var~ |index-of-sequence| ~sequence~ |\&sequence|}
\clindexx{for}{index-of-sequence}
\Clause{|for| ~var~ |index-of-string| ~string~ |\&sequence|}
\clindexx{for}{index-of-string}
\finishitem
~var~ is set to successive indices of the sequence.
These clauses avoid the overhead of accessing the sequence elements
for those applications where they do not need to be examined, or are
examined rarely.  They admit all the optional keywords of the other
sequence drivers except the (redundant) |with-index| keyword.

\clause{|for| ~(key value)~ |in-hashtable| ~table~}
\clindexx{for}{in-hashtable}

~key~ and ~value,~ which must appear as shown in a list
and may be destructuring templates, are set to the keys and values of
~table~.  If ~key~ is |nil|, then the hashtable's keys will be ignored;
similarly for ~value~.
The order in which 
elements of ~table~ will be retrieved is unpredictable.

\clause{|for| ~var~ |in-package| ~package~ \opt\ |external-only| ~ext~}
\clindexx{for}{in-package}
Iterates over all the symbols in ~package,~ or over only the
external symbols if ~ext~ is specified and non-|nil|.  ~ext~ is not
evaluated.  The same symbol may appear more than once.

\clause{|for| ~var~ |in-packages| ~packages~ \opt\ |having-access| ~symbol-types~}
\clindexx{for}{in-packages}
Iterates over all the symbols from the list of packages denoted by the
descriptor ~packages~ and having accessibility (or visibility) given by
~symbol-types~.  This defaults to the list |(:external :internal :inherited)|
and is not evaluated.  ~var~ must be a list of up to three variables: in each
iteration, these will be set to a symbol, its access-type and package (as per
|with-package-iterator| in ANSI-CL).  The same symbol may appear more than
once.

\end{clauses}

\begin{clauses}

\clause{|for| ~var~ |in-file| ~name~ \opt\ |using| ~reader~}
\clindexx{for}{in-file}
Opens the file ~name~ (which may be a string or pathname) for input,
and iterates 
over its contents. ~reader~ defaults to |read|, so by default ~var~
will be bound to the successive forms in the file.  The \iter\ body is
wrapped in an unwind-protect to ensure that the file is closed no
matter how the \iter\ is exited.

\clause{|for| ~var~ |in-stream| ~stream~ \opt\ |using| ~reader~}
\clindexx{for}{in-stream}
Like |for\dots in-file|, except that ~stream~ should be an existing
stream object that supports input operations.

\end{clauses}

\subsubsection{Generalized Drivers}

These are primarily useful for writing drivers that can also be used
as generators (see section \ref{generators}, below).
 
\begin{clauses}

\clause{|for| ~var~ |next| ~expr~}
\clindexx{for}{next}

~var~ is set to ~expr~ each time through the loop.  Destructuring is
performed.  When the clause is used as a generator, ~expr~ is the code
that is executed when |(next ~var~)| is 
encountered (see section \ref{generators}, below).
~expr~ should compute the first value for ~var~, as well
as all subsequent values, and is responsible for terminating the loop.
For compatibility with future versions of \iter, this termination
should be done with |terminate|\clindex{terminate}, which can be
considered a synonym for |finish| (see section \ref{control-flow}).

\cpar As an example, the following clauses are equivalent to |(for
i from 1 to 10)|:
\begin{program}
(initially (setq i 0))
(for i next (if (> i 10) (terminate) (incf i)))
\end{program}

\clause{|for| ~var~ |do-next| ~form~}
\clindexx{for}{do-next}
~form~ is evaluated each time through the loop.  Its value is ~not~
set to ~var~; that is ~form's~ job. ~var~ is only present so that
\iter\ knows it is a driver variable.  \linebreak |(for ~var~ next
~expr~)| is equivalent to |(for ~var~ do-next (dsetq ~var~ ~expr~))|.
(See section \ref{destructuring} for an explanation of |dsetq|.) 

\end{clauses}

\subsubsection{Generators}
\label{generators}

In all of the above clauses, the driver variable is updated on each
iteration.
Sometimes it is desirable to have greater control over updating.  
For instance, consider the problem of associating numbers, in
increasing order and with no gaps, with the
\nonnil\ elements of a list.  One obvious first pass at writing this is:

\begin{program}
(iter (for el in list)
      (for i upfrom 1)
      (if el (collect (cons el i))))
\end{program}
But on the list |(a b nil c)| this produces |((a . 1) (b . 2) (c .
4))| instead of the desired |((a . 1) (b . 2) (c . 3))|.  The problem
is that |i| is incremented each time through the loop, even when |el|
is |nil|.  

The problem could be solved elegantly if we could step |i| only when
we wished to.  This
can be accomplished for any \iter\ driver by writing |generate| (or
its synonym |generating|)
instead of |for|.  Doing so produces a ~generator~---a driver whose
values are yielded explicitly.  To obtain the next value of a
generator variable ~v~, write \linebreak |(next ~v~)|.  The value of a |next|
form is the next value of ~v~, as determined by its associated driver
clause.  |next| also has the side-effect of updating ~v~ to that
value.  If there is no next value, |next| will terminate the loop,
just as with a normal driver.


Using generators, we can now write our example like this:
\begin{program}
(iter (for el in list)
      (generate i upfrom 1)
      (if el (collect (cons el (next i)))))
\end{program}
Now |i| is updated only when |(next i)| is executed, and this occurs
only when |el| is \nonnil.

To better understand the relationship between ordinary drivers and
generators, observe that we can rewrite an ordinary driver using its
generator form immediately followed by |next|, as this example shows:
\begin{program}
(iter (generating i from 1 to 10)
      (next i)
      ...)
\end{program}
Provided that the loop body contains no |(next i)| forms, this will
behave just as if we had written |(for i from 1 to 10)|.

We can still refer to a driver variable ~v~ without using |next|; in
this case, its value is that given to it by the last evaluation of
|(next ~v~)|.  Before |(next ~v~)| has been called the first time, the
value of ~v~ is undefined.  

This semantics is more flexible than
one in which ~v~ begins the loop bound to its first value and calls
of |next| supply subsequent values, because it means the loop will not
terminate too soon if the generator's sequence is empty.  For
instance, consider the following code, which tags \nonnil\ elements of
a list using a list of tags, and also counts the null elements.
(We assume there are at least as many tags as \nonnil\ elements.)
\begin{program}
(let* ((counter 0)
       (tagged-list (iter (for el in list)
                          (generating tag in tag-list) 
                          (if (null el)
                              (incf counter)
                              (collect (cons el (next tag)))))))
  ...)
\end{program}
It may be that there are just as many tags as non-null elements of
|list|.  If all the elements of |list| are null, we still want the
counting to proceed, even though |tag-list| is \nil.  If |tag| had to be
assigned its first value before the loop begins, we would have had to
terminate the loop before the first iteration, since when |tag-list|
is \nil, |tag| 
has no first value.  With the existing semantics, however, |(next
tag)| will never execute, so the iteration will cover all the elements of
|list|. 
        
When the ``variable'' of a driver clause is actually a destructuring
template containing several variables, all the variables are eligible
for use with |next|.  As before, |(next ~v~)| evaluates to ~v's~ next
value; but the effect is to update all of the template's variables.
For instance, the following code will return the list |(a 2 c)|.
\begin{program}
(iter (generating (key . item) in '((a . 1) (b . 2) (c . 3)))
      (collect (next key))
      (collect (next item)))
\end{program}

Only driver clauses with variables can be made into generators.  This
includes all clauses mentioned so far except for |repeat|.  It does
~not~ include |for\dots previous|, |for\dots =|,
|for\dots initially\dots then| or |for\dots first\dots then| (see
below). 

\subsubsection{Previous Values of Driver Variables}

Often one would like to access the value of a variable on a previous
iteration.  \iter\ provides a special clause for accomplishing this.

\begin{clauses}

\clause{|for| ~pvar~ |previous| ~var~ \opt\ |initially| ~init~
|back| ~n~}
\clindexx{for}{previous}
Sets ~pvar~ to the previous value of ~var,~ which should be a driver
variable, a variable from another |for\dots previous| clause, or a
variable established by a |for\dots =|, 
|for\dots initially\dots then| or |for\dots first\dots then| clause
(see section \ref{setting}).
Initially, ~pvar~ is given the value ~init~ (which defaults to \nil).  
The ~init~ expression will be moved outside the loop body, so it
should not depend on anything computed within the loop.
~pvar~ retains the value of ~init~ until ~var~ is set to its second
value, at which point ~pvar~ is set to ~var's~ first value; and so on.  

\cpar The
argument ~n~ to |back| must be a constant, positive integer, and
defaults to 1.  It determines how many iterations back ~pvar~ should
track ~var~.  For example, when ~n~ is 2, then ~pvar~ will be assigned
~var's~ first value when ~var~ is set to its third value.

\cpar A |for\dots previous| clause may occur after or before its
associated driver clause. |for\dots previous| works with generators as
well as ordinary drivers.

\pagebreak[3]

\cpar Example:
\begin{program}
(iter (for el in '(1 2 3 4))
      (for p-el previous el)
      (for pp-el previous p-el initially 0)
      (collect pp-el))
\end{program}
This evaluates to |(0 0 1 2)|.  It could have been written more
economically as 
\begin{program}
(iter (for el in '(1 2 3 4))
      (for pp-el previous el back 2 initially 0)
      (collect pp-el))
\end{program}


\end{clauses}

\subsection{Variable Binding and Setting}
\label{setting}

Several clauses exist for establishing new variable bindings or for
setting variables in the loop.  They all support destructuring.

\begin{clauses}

\clausex{with}{~var~ \opt\ |=| ~value~}
Causes ~var~ to be bound to value before the loop body is entered.
If ~value~ is not supplied, ~var~ assumes a default
binding, which will be 
\nil\ in the absence of declarations.  Also, if ~value~ is not
supplied, no destructuring is performed; instead, ~var~ may be a list of
symbols, all of which are given default 
bindings.  If ~value~ is supplied, ~var~ is bound to it, with
destructuring.  

\cpar Because |with|
creates bindings whose scope includes the entire \iter\ form, it is
good style to put all |with| clauses at the beginning.

\cpar Successive occurrences of |with| result in sequential
bindings (as with 
|let*|).  There is no way to obtain parallel bindings; see
section \ref{bindings} for a rationale.  


\clause{|for| ~var~ |=| ~expr~}
\clindexx{for}{=}
On each iteration, ~expr~ is evaluated and ~var~ is set
to its value. 

\cpar This clause may appear to do the same thing as |for\dots next|.
In fact, they are quite different.  |for\dots =| provides only three
services: it sets up a binding for ~var~, sets it to ~expr~ on each
iteration, and makes it possible to use |for\dots previous| with
~var~.  |for\dots next| provides these services in addition to the
ability to turn the driver into a generator.
%Also, the code which sets ~var~ appears in the loop body in the same
%place as the |for\dots =| clause; the code for |for\dots next| appears
%at the top of the loop, as with other drivers (except when being used
%as a generator).

\clause{|for| ~var~ |initially| ~init-expr~ |then| ~then-expr~}
\clindexxx{for}{initially}{then}
Before the loop begins, ~var~ is set to ~init-expr;~ on all
iterations after the first it is set to ~then-expr.~  This clause
must occur at top-level.  ~init-expr~ will be moved outside the loop
body and ~then-expr~ will be moved to the end of the loop body, so
they are subject to code motion problems (see section
\ref{code-movement}). 

\cpar This clause may appear to be similar to |for\dots next|, but in
fact they differ significantly.  |for\dots initially\dots then| is
typically used to give ~var~ its first value before the loop begins,
and subsequent values on following iterations.  This is incompatible
with generators, whose first value and subsequent values must all be
computed by |(next ~var~)|.  Also, the update of ~var~ in 
|for\dots initially\dots then| does not occur at the location of the
clause. 
Use |for\dots initially\dots then| for
one-shot computations where its idiom is more convenient, but use
|for\dots next| for extending \iter\ with new drivers (see section
\ref{extend}).

\clause{|for| ~var~ |first| ~first-expr~ |then| ~then-expr~}
\clindexxx{for}{first}{then}
The first time through the loop, ~var~ is set to ~first-expr;~ on
subsequent iterations, it is set to ~then-expr.~  This differs from
|for\dots initially| in that ~var~ is set to ~first-expr~
inside the loop body, 
so ~first-expr~ may depend on the results of other clauses.  For
instance, 
\begin{program}
(iter (for num in list)
      (for i first num then (1+ i))
      ...)
\end{program}
will set |i| to the first element of |list| on the first
iteration, whereas 
\begin{program}
(iter (for num in list)
      (for i initially num then (1+ i))
      ...)
\end{program}
is probably erroneous; |i| will be bound to |num|'s
default binding (usually \nil) for the first iteration.

\end{clauses}

\begin{note}{Compatibility note}
|loop|'s |for\dots =| works like \iter's, but |loop| used the syntax
|for\dots =\dots then| to mean |for\dots initially\dots then|.  It was
felt that these two operations were sufficiently different to warrant
different keywords.

Also, the |for| in the above three clauses is misleading,
since none is true driver (e.g. none has a corresponding |generate|
form).  |setting| would have been a better choice, but |for| was used
to retain some compatibility with |loop|.  
\end{note}


%%%%%%%%%%%%%%%%%%%%%%%%%%%%%%%%
\subsection{Gathering Clauses}
Many of \iter's clauses accumulate values into a variable, or set a
variable under certain conditions.  At the end of the loop, this
variable contains the desired result.  All these clauses have an
optional |into| keyword, whose argument should be a symbol.  If the
|into| keyword is not supplied, the accumulation variable will be
internally generated and its value will be returned at the end of the
loop; if a variable is specified, that variable is used for the
accumulation, and is not returned as a result---it is up to the user
to return it explicitly, in the loop's epilogue code (see |finally|).
It is safe to examine the accumulation variable during the loop, but
it should not be modified.

These clauses all begin with a verb.  When the verb does
not conflict with an existing Common Lisp function, then it may be
used in either its infinitival or present-participle form (e.g. |sum|,
|summing|).  However, when there is a conflict with Common Lisp, only
the present-participle form may be used (e.g. |unioning|).  This is to
prevent \iter\ clauses from clashing with Common Lisp
functions. 

%although these clauses are described as ``producing a value,'' it is a
%mistake to think of the lisp list representing the clause as a
%value-producing form in the usual way.  clauses may legally be written
%where a value is expected, e.g. |(setq x (sum i))|, but the
%lisp value of a clause in such a context is undefined.

\subsubsection{Reductions}

~Reduction~ is an extremely common iteration pattern in which
the results of successive applications of a
binary operation are accumulated.
For example, a loop that computes the sum of the
elements of a list is performing a reduction with the addition
operation.  This could be written in Common Lisp as  |(reduce \#'+
list)| or with \iter\ as
\begin{program}
(iter (for el in list)
      (sum el))
\end{program}

\begin{clauses}

\clausex{sum}{~expr~ \opt\ |into| ~var~}
Each time through the loop, ~expr~ is evaluated and added to a
variable, which is bound initially to zero.  If ~expr~ has a type,
it is ~not~ used as the type of the sum variable, which is always
|number|.  To get the result variable to be of a more specific
type, use an explicit variable, as in
\begin{program}
(iter (for el in number-list)
      (sum el into x)
      (declare (fixnum x))
      (finally (return x)))
\end{program}

\clausex{multiply}{~expr~ \opt\ |into| ~var~}
Like |sum|, but the initial value of the result variable is $1$,
and the variable is updated by multiplying ~expr~ into it.

\clausex{counting}{~expr~ \opt\ |into| ~var~}
~expr~ is evaluated on each iteration.  If it is \nonnil, the
accumulation variable, initially zero, is incremented.

\startitem
\Clausex{maximize}{~expr~ \opt\ |into| ~var~}
\Clausex{minimize}{~expr~ \opt\ |into| ~var~}
\finishitem
~expr~ is evaluated on each iteration and its extremum (maximum or
minimum) is stored in the accumulation variable.  If ~expr~ is never
evaluated, then the result is \nil\ (if the accumulation variable is
untyped) or $0$ (if it has a numeric type).

\clausex{reducing}{~expr~ |by| ~func~ \opt\ |initial-value| ~init-val~ |into| ~var~}
This is a general way to perform reductions. ~func~ should be a
function of two arguments, the first of which will be the value
computed so far and
the second of which will be the value of ~expr~.  It should return the
new value.
|reducing| is roughly equivalent to the
Common Lisp |(reduce ~func~ ~sequence~ :key ~expr-function~)|, where
~expr-function~ is used to derive values from the successive elements
of ~sequence~.

\cpar If the |reducing| clause is never executed, the result is
undefined.

\cpar It is not necessary to provide an
initial value, but better code can be generated if one is supplied.
Regardless of its location in the \iter\ body, ~init-val~ will be
evaluated before the loop is entered, so it should not depend on any
value computed inside the \iter\ form.  

%\cpar if a ~var~ is not specified, you can get \iter\ to declare the
%type of the internal variable by putting a |the| expression around
%~func.~  see section \ref{types}.

\end{clauses}

%\begin{note}{implementation note}
%in principle, |maximize| and |minimize| can be thought of
%as reductions where the initial value is the smallest (or largest)
%value that the accumulation variable can assume.  because lisp's
%bignums can represent arbitrary integers, these clauses cannot be
%implemented as reductions in general.  if, however, the type of 
%~expr~ or ~var~ can be determined to be a fixnum
%or a float, \iter\ will implement the clause as a true reduction,
%using one of the constants |most-negative-fixnum|, |%most-positive-fixnum|, 
%|most-negative-short-float|, etc. as appropriate.
%\end{note}

\subsubsection{Accumulations}

All the predefined accumulation clauses add values to a sequence.  If
the sequence is a list, they all 
have the property that the partial list is kept in the correct order
and available for inspection at any point in the loop.

\begin{clauses}

\clausex{collect}{~expr~ \opt\ |into| ~var~ |at| ~place~ |result-type| ~type~}
Produces a sequence of the values of ~expr~ on each iteration. ~place~
indicates where the next value of ~expr~ is added to the list and may
be one of the symbols |start|, |beginning| (a synonym for |start|) or
|end|.  The symbol may be quoted, but need not be.  The default is
|end|.  For example,
\begin{program}
(iter (for i from 1 to 5)
      (collect i))
\end{program}
produces |(1 2 3 4 5)|, whereas
\begin{program}
(iter (for i from 1 to 5)
      (collect i at beginning))
\end{program}
produces |(5 4 3 2 1)| (and is likely to be faster in most Common Lisp
implementations).

\cpar If ~type~ is provided, it should be a subtype of |sequence|.
The default is |list|.  Specifying a type other than |list| will
result in |collect| returning a sequence of that type.  ~However,~ the
type of the sequence being constructed when inside the loop body is
undefined when a non-|list| type is specified.  (As with ~place~,
quoting ~type~ is optional.)

\clausex{adjoining}{~expr~ \opt\ |into| ~var~ |test| ~test~ |at| ~place~ |result-type| ~type~}
Like |collect|, but only adds the value of ~expr~ if it is not
already present.  ~test,~ which defaults to |\#'eql|, is
the test to be used with |member|.

\startitem
\Clausex{appending}{~expr~ \opt\ |into| ~var~ |at| ~place~}
\Clausex{nconcing}{\ ~expr~ \opt\ |into| ~var~ |at| ~place~}
\Clausex{unioning}{\ ~expr~ \opt\ |into| ~var~ |test| ~test~ |at| ~place~}
\Clausex{nunioning}{~expr~ \opt\ |into| ~var~ |test| ~test~ |at| ~place~}
\finishitem
These are like |collect|, but behave like the Common Lisp functions
|append|, |nconc|, |union| or |nunion|.  
As in Common Lisp, they work only on lists.  Also as in Common Lisp,
|unioning| and |nunioning| assume that the value of ~expr~ contains no
duplicates. 

\clausex{accumulate}{~expr~ |by| ~func~ \opt\ |initial-value| ~init-val~ |into| ~var~}
This is a general-purpose accumulation clause. ~func~ should be a
function of two arguments, the value of ~expr~ and the value
accumulated so far in the iteration, and it should return the updated
value.  If no initial value is supplied, \nil\ is used.  

%\cpar If a ~var~ is not specified, you can get \iter\ to declare the
%type of the internal variable by putting a |the| expression around
%~func.~  see section \ref{types}.

\cpar The differences between |accumulate| and |reducing| are slight.
One difference is that the functions take their arguments in a
different order.  Another is that in the absence of ~init-val~,
|accumulate| will use \nil, whereas |reducing| will generate different
code that avoids any dependence on the initial value.
The reason for having both clauses is that one usually
thinks of reductions (like |sum|) and accumulations (like |collect|)
as different beasts.

\end{clauses}


\subsubsection{Finders}

A ~finder~ is a clause whose value is an expression that meets some
condition.

\begin{clauses}

\clause{|finding| ~expr~ |such-that| ~test~ \opt\ |into| ~var~ |on-failure| ~failure-value~}
\clindexx{finding}{such-that}
If ~test~ (which is an expression) ever evaluates to \nonnil, the loop
is terminated, the 
epilogue code is run and the value of ~expr~ is returned.  Otherwise,
\nil\ (or ~failure-value,~ if provided) is returned.  If ~var~ is
provided, it will have either the \nonnil\ value of ~expr~ or
~failure-value~ when the epilogue code is run.

\cpar As a special case, if the ~test~ expression is a sharp-quoted
function, then it is applied to ~expr~ instead of being simply
evaluated.  E.g. |(finding x such-that \#'evenp)| is equivalent to
|(finding x such-that (evenp x))|.  

%\cpar although ~test~ need have nothing to do with ~%expr~ as in
%|(finding j such-that (> i 3))|, it usually 
%will: |(finding (length el) such-that (oddp (length el)))|.  to
%avoid performing the |length| computation twice, you could write
%|(finding (length el) such-that \#'oddp)| or |(finding (length
%el) such-that 'oddp)|; for these cases, \iter\ generates code that
%executes ~expr~ only once.  the code for |\#'oddp|
%is slightly different from that for {\lisp 'oddp}; see the discussion
%under {\lisp for\dots in} and {\lisp for\dots on}.

\cpar |On-failure| is a misnomer. Because it is always evaluated, it behaves
more like the default third argument to the |gethash| function. As a result,
|on-failure (error "Not found")| makes no sense. Instead, the clauses |leave|
or |thereis| can be used in conjunction with |finally| as follows:
\begin{program}
(iter (for x in '(1 2 3))
      (if (evenp x) (leave x))
      (finally (error "not found")))
\end{program}

\cpar This clause may appear multiple times when all defaults are
identical. It can also be used together with either |always|/|never| or
|thereis| if their defaults match. More specifically, |on-failure nil| is
compatible with |thereis|, while |on-failure t| is compatible with |always|
and |never| clauses.
\begin{program}
(iter (for i in '(7 -4 2 -3))
      (if (plusp i)
	  (finding    i  such-that (evenp i))
          (finding (- i) such-that (oddp i))))
\end{program}

\startitem
\Clause{|finding| ~expr~ |maximizing| ~m-expr~ \opt\ |into| ~var~}
\clindexx{finding}{maximizing}
\Clause{|finding| ~expr~ |minimizing| ~m-expr~ \opt\ |into| ~var~}
\clindexx{finding}{minimizing}
\finishitem
Computes the extremum (maximum or minimum) value of ~m-expr~ over all
iterations, and returns the value of ~expr~ corresponding to the
extremum.  ~expr~ is evaluated inside the loop at the time the new
extremum is established.  If ~m-expr~ is never evaluated (due to, for
example, being embedded in a conditional clause), then the returned
value depends on the type, if any, of ~expr~ (or ~var,~ if
one is supplied).  If there is no type, the returned
value will be \nil; if the type is numeric, the returned value will be
zero.

\cpar For these two clauses, ~var~ may be a list of two
symbols; in that case, the first is used to record ~expr~ and
the second, ~m-expr.~  

\cpar As with |finding\dots such-that|, if ~m-expr~ is a sharp-quoted
function, then it is called on ~expr~ instead of being evaluated.

\end{clauses}

\subsubsection{Boolean Tests}

\begin{clauses}

\clausex{first-iteration-p}{}
Returns |t| in the first cycle of the loop, otherwise \nil.

\clausex{first-time-p}{}
Returns |t| the first time the expression is evaluated, and then \nil\ forever.
This clause comes handy when printing (optional) elements separated
by a comma:

\begin{program}
(iter (for el in '(nil 1 2 nil 3))
      (when el
        (unless (first-time-p)
          (princ ", "))
        (princ el)))
\end{program}
produces |"1, 2, 3"|.

\end{clauses}

\subsubsection{Aggregated Boolean Tests}

\begin{clauses}

\clausex{always}{~expr~}
If ~expr~ ever evaluates to
\nil, then \nil\ is immediately returned; the epilogue code is not
executed.  If ~expr~ never evaluates to \nil, the epilogue code
is executed and the last value of ~expr~ (or |t| if ~expr~ was never
evaluated) is returned (whereas |loop| would constantly return |t|).

% mention last evaluated clause when multiple always clauses?

\clausex{never}{~expr~}
Like |(always (not ~expr~))|, except it does not influence the last
value returned by a possible other |always| clause. That is,
\begin{program}
(iter (repeat 2)
      (always 2)
      (never nil)) \yields 2 ; not t
\end{program}

\clausex{thereis}{~expr~}
If ~expr~ is ever \nonnil,
its value is immediately returned without running epilogue code.
Otherwise, the epilogue code is performed and \nil\ is returned.

This clause cannot be used together with |always| or |never|, because their
defaults are opposed (similarly, |(loop always 3 thereis nil)| refuses to
compile in some implementations of |loop|).

\end{clauses}


\subsection{Control Flow}
\label{control-flow}
Several clauses can be used to alter the usual flow of control in a loop.

Note: the clauses of this and subsequent sections don't adhere to \iter's
usual syntax, but instead use standard Common Lisp syntax.  Hence the
format for describing syntax subsequently is
like the standard format used in the Common Lisp manual, not like the
descriptions of clauses above.

\begin{clauses}

\clausex{finish}{}
Stops the loop and runs the epilogue code.

%for example:
%
%\begin{program}
%(iter (with answer = nil)
%         (initially (make-a-mess))
%         (for i from 1 to 10)
%         (when (correct? i)
%           (setq answer i)
%           (finish))
%         (finally (cleanup)))
%\end{program}
%
%this code will execute |cleanup| whether or not the test |(correct?
%i)| ever succeeds. 
%the (more elegant) formulation,
%\begin{program}
%(iter (initially (make-a-mess))
%         (for i from 1 to 10)
%         (finding i such-that (correct? i))
%         (finally (cleanup)))
%\end{program}
%would not execute |cleanup| if |(correct? i)| succeeded; it
%would do an immediate return.

\clausex{leave}{\opt\ ~value~}
Immediately returns ~value~ (default \nil) from the current \iter\
form, skipping the epilogue code.  Equivalent to using |return-from|.

\clausex{next-iteration}{}
Skips the remainder of the loop body and begins the next iteration of
the loop.

\clausex{while}{~expr~}
If ~expr~ ever evaluates to \nil, the loop is terminated and the
epilogue code executed.  Equivalent to |(if (not ~expr~) (finish))|.

\clausex{until}{~expr~}
Equivalent to |(if ~expr~ (finish))|.

\clausex{if-first-time}{~then~ \opt\ ~else~}
If this clause is being executed for the first time in this invocation
of the \iter\ form, then the ~then~ code is evaluated; otherwise the
~else~ code is evaluated.

\cpar |(for ~var~ first ~expr1~ then ~expr2~)| is almost equivalent to
\begin{program}
(if-first-time (dsetq ~var~ ~expr1~)
               (dsetq ~var~ ~expr2~))
\end{program}
The only difference is that the |for| version makes ~var~ available
for use with |for\dots previous|.

\end{clauses}

\subsection{Code Placement}
When fine control is desired over where code appears in a loop
generated by \iter, the following special clauses may be useful.
They are all subject to code-motion problems (see section
\ref{code-movement}).  

\begin{clauses}

\clausex{initially}{|\&rest| ~forms~}
The lisp ~forms~ are placed in the prologue section of the loop, where
they are executed once, before the loop body is entered.

\clausex{after-each}{|\&rest| ~forms~}
The ~forms~ are placed at the end of the loop body, where they
are executed after each iteration.  Unlike the other clauses in this
section, ~forms~ may contain \iter\ clauses.

\clausex{else}{|\&rest| ~forms~}
The lisp ~forms~ are placed in the epilogue section of the loop, where they
are executed if this |else| clause is never met during execution of the
loop and the loop terminates normally.

\clausex{finally}{|\&rest| ~forms~}
The lisp ~forms~ are placed in the epilogue section of the loop, where
they are executed after the loop has terminated normally.

\clausex{finally-protected}{|\&rest| ~forms~}
The lisp ~forms~ are placed in the second form of an unwind-protect
outside the loop.  They are always executed after the loop has
terminated, regardless of how the termination occurred.

\end{clauses}


%%%%%%%%%%%%%%%%%%%%%%%%%%%%%%%%%%%%%%%%%%%%%%%%%%%%%%%%%%%%%%%%
\section{Other Features}

\subsection{Multiple Accumulations}
\label{multiple}

\begin{sloppypar}
It is permitted to have more than one clause accumulate into the same
variable, as in the following:
\begin{program}
(iter (for i from 1 to 10)
      (collect i into nums)
      (collect (sqrt i) into nums)
      (finally (return nums)))
\end{program}
Clauses can only accumulate into the same variable if they are
compatible.  |collect|, |adjoining|, |appending|, |nconcing|,
|unioning| and |nunioning| are compatible with each other; |sum|,
|multiply| and |counting| are compatible; |always| and |never| are compatible;
|finding| \dots |such-that| is compatible with either |thereis| or |always|
and |never| when their defaults match; and |maximize| and |minimize| clauses
are compatible only with other |maximize| and |minimize| clauses,
respectively.

%note that the same variable ~cannot~ be both an accumulation
%variable and an ordinary variable; there can be only one variable with
%a given name within an \iter\ form. 
\end{sloppypar}

\subsection{Named Blocks}

Like Common Lisp |block|s, \iter\ forms can be given names.  The
name should be a single symbol, and it must be the first form in the
\iter.  The generated code behaves exactly like a named block; in
particular, |(return-from ~name~)| can be used to exit it:
\begin{program}
(iter fred
      (for i from 1 to 10)
      (iter barney
            (for j from i to 10)
            (if (> (* i j) 17)
                (return-from fred j))))
\end{program}
An \iter\ form that is not given a name is implicitly named \nil.

Sometimes one would like to write an expression in an inner \iter\ form,
but have it processed by an outer \iter\ form.  This is possible with
the |in| clause.  

\begin{clauses}

\clausex{in}{~name~ |\&rest| ~forms~}
Evaluates ~forms~ as if they were part of the \iter\ form named
~name~.  In other words, \iter\ clauses are processed by the \iter\
form named ~name,~ and not by any \iter\ forms that occur inside ~name.~

\cpar As an example, consider the problem of collecting a list of the
elements in a two-dimensional array.  The naive solution,
\begin{program}
(iter (for i below (array-dimension ar 0))
      (iter (for j below (array-dimension ar 1))
            (collect (aref ar i j))))
\end{program}
\noindent is wrong because the list created by the inner \iter\ is simply
ignored by the outer one.  But using |in| we can write:
\begin{program}
(iter outer (for i below (array-dimension ar 0))
      (iter (for j below (array-dimension ar 1))
            (in outer (collect (aref ar i j)))))
\end{program}
\noindent which has the desired result.

\end{clauses}

\subsection{Destructuring}
\label{destructuring}

In many places within \iter\ clauses where a variable is expected, a
list can be written instead.  In these cases, the value to be assigned
is ~destructured~ according to the pattern
described by the list.  As a simple example, the clause
\begin{program}
(for (key . item) in alist)
\end{program}
\noindent will result in |key| being set to the |car| of
each element in |alist|, and |item| being set to the |cdr|.  The
pattern list may be nested to arbitrary depth, and (as the example
shows) need not be terminated with \nil; the only requirement is that
each leaf be a bindable symbol (or \nil, in which case no binding is
generated for that piece of the structure).

Sometimes, you might like to do the equivalent of a
|multiple-value-setq| in a clause.  This 
``multiple-value destructuring'' can be expressed by writing 
\linebreak |(values $pat_1$ $pat_2 \ldots$)| for a destructuring
pattern, as in 
\begin{program}
(for (values (a . b) c d) = (three-valued-function ...))
\end{program}
\begin{sloppypar}
Note that the $pat_i$ can themselves be destructuring patterns (though
not multiple-value destructuring patterns).  You can't do multiple-value
destructuring in a |with| clause; instead wrap the whole \iter\
form in a |multiple-value-bind|.
\end{sloppypar}

\begin{note}{Rationale}
There are subtle interactions between variable declarations and
evaluation order that make the correct implementation of
multiple-value destructuring in a |with| somewhat tricky.
\end{note}

The destructuring feature of \iter\ is available as a separate
mechanism, using the |dsetq| macro:

\begin{clauses}
\defmacro{dsetq}{template expr}
Performs destructuring of ~expr~ using ~template~.   May be used
outside of an \iter\ form. Yields the primary value of ~expr~.

\end{clauses}

\subsection{On-line Help}

\begin{sloppypar}
There is a limited facility for on-line help, in the form of the 
|display-iterate-clauses| function.
\end{sloppypar}

\begin{clauses}

\defun{display-iterate-clauses}{\opt\ clause-spec}
Displays a list of \iter\ clauses.  If ~clause-spec~ is not
provided, all clauses are shown; if it is a symbol, all clauses
beginning with that symbol are shown; and if it is a list of symbols,
all clauses for which ~clause-spec~ is a prefix are shown.

\end{clauses}

\subsection{Parallel Binding and Stepping}
\label{bindings}

The parallel binding and stepping of variables is a feature that
\iter\ does ~not~ have.  This section attempts to provide a rationale.

We say that two variables are bound ~in parallel~ if neither
binding shadows the other.  This is the usual semantics of |let|
(as opposed to |let*|).  Similarly, we can say that iteration
variables are stepped in parallel if neither variable is updated
before the other, conceptually speaking; in other words, if the code
to update each variable can reference the old values of both variables.

|loop| allows parallel binding of variables and parallel stepping of
driver variables.  My view is that if you are depending on the 
serial/parallel distinction, you are doing something obscure.  If you
need to bind 
variables in parallel using |with|, then you must be using a
variable name that shadows a name in the existing lexical environment.
Don't do that.  The most common use for parallel stepping is to track
the values of variables on the previous iteration, but in fact this
does not require parallel stepping at all; the following will work:
\begin{program}
(iter (for current in list)
      (for prev previous current)
      ...)
\end{program}


%%%%%%%%%%%%%%%%%%%%%%%%%%%%%%%%%%%%%%%%%%%%%%%%%%%%%%%%%%%%%%%%
\section{Types and Declarations}
\label{types}

\subsection{Discussion}

Sometimes efficiency dictates that the types of variables be declared.
This type information needs to be communicated to \iter\ so it can
bind variables to appropriate values.  Furthermore, \iter\ must often
generate internal variables invisible to the user; there needs to be a
way for these to be declared.

As an example, consider this code, which will return the number of
odd elements in |number-list|:
\begin{program}
(iter (for el in number-list)
      (count (oddp el)))
\end{program}
In processing this form,
\iter\ will create an internal variable, let us call it |list17|, to
hold the successive |cdr|s of |number-list|, and  
will bind the variable to |number-list|.  It will also generate a
default binding for |el|; only inside the body of the loop will |el|
be set to the |car| of |list17|.  Finally, \iter\ will generate a
variable, call it |result|, to hold the result of the count, and will
bind it to zero.

When dealing with type declarations, \iter\ observes one simple rule:
~it will never generate a declaration unless requested to do so.~  The
reason is that such declarations might mask errors in compiled code by
avoiding error-checks; the resulting problems would be doubly hard to
track down because the declarations would be hidden from the
programmer.  Of course, a compiler might omit error-checks even in the
absence of declarations, though this behavior can usually be avoided,
e.g. by saying |(declaim (optimize (safety 3)))|.

So, the above \iter\ form will generate code with no declarations.
But say we wish to declare the types of |el| and the internal
variables |list17| and |result|.  How is this done?

Declaring the type of |el| is easy, since the programmer knows
the variable's name:
\begin{program}
(iter (for el in number-list)
      (declare (fixnum el))
      (counting (oddp el)))
\end{program}
\iter\ can read variable type declarations like this one.  Before
processing any clauses, it scans the entire top-level form for type
declarations 
and records the types, so that variable bindings can be performed
correctly.  In this case, |el| will be bound to zero
instead of \nil.  Also, \iter\ collects all the top-level declarations
and puts them at the begining of the generated code, so it is not
necessary to place all declarations at the beginning of an \iter\
form; instead, they can be written near the variables whose types they
declare.  

Since \iter\ is not part of the compiler, it will not know
about declarations that occur outside an \iter\ form; these
declarations must be repeated inside the form.

Here is another way we could have declared the type of |el|:
\begin{program}
(iter (for (the fixnum el) in number-list)
      (counting (oddp el)))
\end{program}
\iter\ extends the Common Lisp |the|\lispindex{the}
form to apply to variables as well as value-producing forms; anywhere
a variable is allowed---in a |with| clause, as the iteration
variable in a driver clause, as the |into| argument of an
accumulation clause, even inside a destructuring template---you can
write |(the ~type~ ~symbol~)| instead.

There is one crucial difference between using a |the| form and
actually declaring the variable: explicit declarations are always
placed in the generated code, but type information from a |the|
form is not turned into an actual declaration unless you tell \iter\
to do so using |iterate:declare-variables|.  See below.

\begin{sloppypar}
Declaring the types of internal variables is harder than declaring the
types of explicitly mentioned variables, since their names
are unknown.  You do it by declaring |iterate:declare-variables|
somewhere inside the top level of the \iter\ form.  (This will also
generate declarations for variables declared using |the|.)
\iter\ does not provide much selectivity here: it's all or none. 
And unfortunately, since \iter\ is not privy to compiler information
but instead reads declarations itself, it will not hear if you
|(declaim (iterate:declare-variables))|.  Instead, set the variable 
|iterate::*always-declare-variables*| to |t| at
compile-time, using |eval-when|.
\end{sloppypar}

To determine the appropriate types for internal variables, \iter\ uses
three sources of information:

\begin{itemize}

\item Often, the particular clause dictates a certain type for a
variable; \iter\ will use this information when available.  In the
current example, the variable |list17| will be given the type
|list|, since that is the only type that makes sense; and the
variable |result| will be given the type |fixnum|, on the
assumption that you will not be counting high enough to need bignums.
You can override this assumption only by using and explicitly declaring a
variable: 
\begin{verbatim}
(iter (declare (iterate:declare-variables))
      (for el in number-list)
      (count (oddp el) into my-result)
      (declare (integer my-result))
      (finally (return my-result)))
\end{verbatim}

\begin{sloppypar}
Other examples of the type assumptions that \iter\ makes are: type
|list| for |into| variables of collection clauses; type |list| for
expressions that are to be destructured; type |vector| for the
variable holding the vector in a |for\dots in-vector| clause, and
similarly for |string| and the |for\dots in-string| clause;
and the implementation-dependent type
|(type-of array-dimension-limit)| for the index and limit
variables generated by sequence iteration drivers like |for\dots
in-vector| and |for\dots in-string| (but not |for\dots in-sequence|,
because it may be used to iterate over a list). 
\end{sloppypar}

\item Sometimes, \iter\ will examine expressions and try to determine
their types in a simple-minded way.  If the expression is
self-evaluating (like a number, for instance), \iter\ knows that the
expression's type is the same as the type of the value it denotes, so
it can use that type.  If the expression is of the form |(the ~type~
~expr~)|, \iter\ is smart enough to extract ~type~ and use it.
However, the current version of \iter\ does not  
examine declarations of function result types or do any type
inference.  It will not determine, for 
example, that the type of |(+ 3 4)| is |fixnum|, or even
|number|.

\item In some cases, the type of an internal variable should match the
type of some other variable.  For instance, \iter\ generates an
internal variable for |(f x)| in the
clause |(for i from 1 to (f x))|, and in the absence of other
information will give it the same type as |i|.  If, however, the
expression had been written |(the fixnum (f x))|, then \iter\
would have given the internal variable the type |fixnum|
regardless of |i|'s type.  The type incompatibility errors that
could arise in this situation are not checked for.

\end{itemize}

Note that if you do declare |iterate:declare-variables|, then
\iter\ may declare user variables as well as internal ones if they do
not already have declarations, though only for variables that it
binds.  For instance, in this code:

\begin{program}
(iter (declare (iterate:declare-variables))
      (for i from 1 to 10)
      (collect i into var))
\end{program}
the variable |var| will be declared to be of type |list|.


\subsection{Summary}

\iter\ understands standard Common Lisp variable type declarations
that occur within an \iter\ form and
will pass them through to the generated code.  If the declaration
|(iterate:declare-variables)|\lispindex{declare-variables}
appears at the top level of an 
\iter\ form, or if 
|iterate::*always-declare-variables*|\lispindex{*always-declare-variables*} 
is \nonnil, then \iter\ will use the type information gleaned from user
declarations, self-evaluating expressions and |the| expressions,
combined with reasonable assumptions, to determine variable
types and declare them.


%%%%%%%%%%%%%%%%%%%%%%%%%%%%%%%%%%%%%%%%%%%%%%%%%%%%%%%%%%%%%%%%
\section{Problems with Code Movement}
\label{code-movement}

Some \iter\ clauses, or parts of clauses, result in code being
moved from the location of the clause to other parts of the loop.
Drivers behave this way, as do code-placement clauses like |initially|
and |finally|.  When using these clauses, there is a danger of writing
an expression that makes sense in its apparent location but will be
invalid or have a different meaning in another location.  For example:
\begin{program}
(iter (for i from 1 to 10)
      (let ((x 3))
        (initially (setq x 4))))
\end{program}
While it may appear that the |x| of |(initially (setq x 4))| is the
same as the |x| of |(let ((x 3)) \dots|, in fact they are not:
|initially| moves its code outside the loop body, so |x| would refer
to a global variable.  Here is another example of the same problem:
\begin{program}
(iter (for i from 1 to 10)
      (let ((x 3))
        (collect i into x)))
\end{program}
If this code were executed, |collect| would create a binding for its
|x| at the top level of the \iter\ form that the |let| will shadow.

Happily, \iter\ is smart enough to catch these errors; it
walks all problematical code to ensure that free variables are not
bound inside the loop body, and checks all variables it binds for the
same problem.

However, some errors cannot be caught:

\begin{program}
(iter (with x = 3)
      (for el in list)
      (setq x 1)
      (reducing el by \#'+ initial-value x))
\end{program}
|reducing| moves its |initial-value| argument to the initialization
part of the loop in order to produce more efficient code.  Since
\iter\ does not perform data-flow analysis, it cannot determine that
|x| is changed inside the loop; all it can establish is that |x| is
not bound internally.  Hence this code will not signal an
error and will use $3$ as the initial value of the reduction.

The following list summarizes all cases that are subject to these code
motion and variable-shadowing problems.
\begin{itemize}
\item Any variable for which \iter\ creates a binding, including those
used in |with| and the |into| keyword of many clauses.

\begin{sloppypar}
\item The special clauses which place code: |initially|, |after-each|, |else|,
|finally| and |finally-protected|. 
\end{sloppypar}

\item The variables of a |next| or |do-next| form.

\item The |initially| arguments of |for\dots initially\dots then| and
|for\dots previous|. 

\item The |then| argument of |for\dots initially\dots then|.

\item The |initial-value| arguments of |reducing| and |accumulate|.

\item The |on-failure| argument of |finding\dots such-that|.
\end{itemize}

%%%%%%%%%%%%%%%%%%%%%%%%%%%%%%%%%%%%%%%%%%%%%%%%%%%%%%%%%%%%%%%%
\section{Differences Between {\tt Iterate} and {\tt Loop}}
|loop| contains a great deal of complexity which \iter\ tries to
avoid.  Hence many esoteric features of |loop| don't exist in \iter.
Other features have been carried over, but in a cleaned-up form.  
And of course, many new features have been added; they are not
mentioned in this list.

\begin{itemize}

\item \iter's syntax is more Lisp-like than |loop|'s, having a higher
density of parens.

\item The current implementation of \iter, unlike the current version
of  |loop| (as documented in {\em Common Lisp, 2nd Ed.\/}), is
extensible (see section \ref{extend}).  

\item |loop| puts the updates of all driver variables at the top of
the loop; \iter\ leaves them where the driver clauses appear.

\item While for the most part \iter\ clauses that resemble |loop| clauses
behave similarly, there are some differences.  For instance, there is
no |for\dots =\dots then| in \iter; instead use 
|for\dots initially\dots then|.  

\item |loop| binds the variable |it| at certain times
to allow pseudo-English expressions like |when ~expr~ return it|. 
In \iter, you must bind ~expr~ to a variable yourself.  Note that
|when ~expr~ return it| is like |thereis ~expr~| except that the latter is an
accumulation clause and therefore competes with other accumulations
(remember section \ref{multiple} above).

% repeat different behaviour of |always| clause here?

\item |loop| has a special |return| clause, illustrated in the
previous item.  \iter\ doesn't need one,  since an ordinary Lisp
|return| has the same effect.

\item |loop| allows for parallel binding and stepping of iteration
variables.  \iter\ does not.  (See section \ref{bindings}.)

\item |loop| and \iter\ handle variable type declarations very
differently.  |loop| provides a special syntax for declaring variable
types, and does not examine declarations.  Moreover, the standard
implementation of |loop| will
generate declarations when none are requested.
\iter\ parses standard Common Lisp type declarations, and will never
declare a variable itself unless declarations are 
specifically requested.

\end{itemize}

%%%%%%%%%%%%%%%%%%%%%%%%%%%%%%%%%%%%%%%%%%%%%%%%%%%%%%%%%%%%%%%%
\section{Rolling Your Own}
\label{extend}

\subsection{Introduction}

\iter\ is extensible---you can write new clauses that embody new
iteration patterns.  You might want to write a new driver clause for a
data structure of your own, or you might want to write a clause that
collects or manipulates elements in a way not provided by \iter. 

This section describes how to write clauses for \iter.  Writing a
clause is like writing a macro.  In fact, writing a clause ~is~
writing a macro: since \iter\ code-walks its body and macroexpands,
you can add new abstractions to \iter\ with good old |defmacro|.

Actually, there are two extensions you can make to \iter\ that are
even easier than writing a macro.  They are adding a synonym for an
existing clause and defining a driver clause for an indexable
sequence.  These can be done with |defsynonym| and
|defclause-sequence|, respectively.  See section \ref{aids}, below.

The rest of this section explains how to write macros that expand into
\iter\ clauses.
Here's how you could add a simplified version of \iter's |multiply|
clause, if \iter\ didn't already have one:
\begin{program}
(defmacro multiply (expr)
  `(reducing ,expr by \#'* initial-value 1))
\end{program}

If you found yourself summing the square of an expression often, you
might want to write a macro for that.  A first cut might be
\begin{program}
(defmacro sum-of-squares (expr)
  `(sum (* ,expr ,expr)))
\end{program}
but if you are an experienced macro writer, you will realize that this
code will evaluate ~expr~ twice, which is probably a bad idea.  A
better version would use a temporary:
\begin{program}
(defmacro sum-of-squares (expr)
  (let ((temp (gensym)))
    `(let ((,temp ,expr))
       (sum (* ,temp ,temp)))))
\end{program}
Although this may seem complex, it is just the sort of thing you'd
have to go through to write any macro, which illustrates the point of
this section: if you can write macros, you can extend \iter.

Our macros don't use \iter's keyword-argument
syntax.  We could just use keywords with |defmacro|, but we would
still not be using \iter's clause indexing mechanism.  Unlike Lisp,
which uses just the first symbol of a form to determine what function
to call, \iter\ individuates clauses by the list of required keywords.
For instance, |for\dots in| and |for\dots in-vector| are different clauses
implemented by distinct Lisp functions.  

To buy into this indexing scheme, as well as the keyword-argument
syntax, use |defmacro-clause|:

\begin{clauses}

\defmacro{defmacro-clause}{arglist |\&body| body}
Defines a new \iter\ clause.  ~arglist~ is a list of symbols which are
alternating keywords and arguments.  \opt\ may be used, and the list
may be terminated by |\&sequence|.  ~body~ is an ordinary macro body,
as with |defmacro|.  If the first form of ~body~ is a string, it is
considered a documentation string and will be shown by
|display-iterate-clauses|. |defmacro-clause| will signal an error if 
defining the clause would result in an ambiguity.  E.g. you cannot
define the clause |for\dots from| because there would be no way to
distinguish it from a use of the |for| clause with optional keyword |from|.

\end{clauses}

\medskip

Here is |multiply| using |defmacro-clause|.  The keywords are capitalized
for readability.
\begin{program}
(defmacro-clause (MULTIPLY expr \opt\ INTO var)
  `(reducing ,expr by \#'* into ,var initial-value 1))
\end{program}
You don't have to worry about the case when |var| is not supplied; for
any clause with an |into| keyword, saying |into nil| is equivalent to
omitting the |into| entirely.

As another, more extended example, consider the fairly common
iteration pattern that 
involves finding the sequence element that maximizes (or minimizes) some
function.  \iter\ provides this as |finding\dots maximizing|, but it's
instructive to see how to write it.
Here, in pseudocode, is how you might write such a loop for
maximizing a function F:

\begin{quote}
\begin{tabbing}
set variable MAX-VAL to NIL; \\
set variable WINNER to NIL; \\
for\=\ each element EL in the sequence \\
\>  if\=\ MAX-VAL\=\ is NIL or F(EL) $>$ MAX-VAL then \\
\>\>    set MAX-VAL to F(EL); \\
\>\>    set WINNER to EL; \\
\>  end if; \\
end for; \\
return WINNER.
\end{tabbing}
\end{quote}

Here is the macro:
\begin{program}
(defmacro-clause (FINDING expr MAXIMIZING func \opt\ INTO var)
  (let ((max-val (gensym))
        (temp1 (gensym))
        (temp2 (gensym))
        (winner (or var iterate::*result-var*)))
    `(progn 
       (with ,max-val = nil)
       (with ,winner = nil)
       (cond 
        ((null ,max-val)
         (setq ,winner ,expr)
         (setq ,max-val (funcall ,func ,winner))
        (t
         (let* ((,temp1 ,expr)
                (,temp2 (funcall ,func ,temp1)))
           (when (> ,temp2 ,max-val)
             (setq ,max-val ,temp2)
             (setq ,winner ,temp1))))))
       (finally (leave ,winner)))))
\end{program}
Note that if no |into| variable is supplied, we use
|iterate::*result-var*|, which contains the internal variable into
which all clauses place their results.  If this variable is bound by
some clause, then \iter\ will return its value automatically;
otherwise, \nil\ will be returned.  

\subsection{Writing Drivers}

In principle, drivers can be implemented just as easily as other
\iter\ clauses.  In practice, they are a little harder to get right. 
As an example, consider writing a driver that
iterates over all the 
elements of a vector, ignoring its fill-pointer.  |for\dots in-vector|
won't work for this, because it observes the fill-pointer.  It's necessary to
use |array-dimension| instead of |length| to obtain the size of the
vector.  Here is one approach:

\begin{program}
(defmacro-clause (FOR var IN-WHOLE-VECTOR v)
  "All the elements of a vector (disregards fill-pointer)"
  (let ((vect (gensym))
        (index (gensym)))
    `(progn
       (with ,vect = ,v)
       (for ,index from 0 below (array-dimension ,vect 0))
       (for ,var = (aref ,vect ,index)))))
\end{program}
Note that we immediately put |v| in a variable, in case it is an
expression.  Again, this is just good Lisp macrology.  It also has a
subtle effect on the semantics of the driver: |v| is evaluated only
once, at the beginning of the loop, so changes to |v| in the loop have
no effect on the driver.  Similarly, the bounds for numerical iteration
e.g. the above |array-dimension| are also evaluated once only.  This is how
all of \iter's drivers work.


There is an important point concerning the |progn| in this code.  We
need the |progn|, of course, because we are returning several forms,
one of which is a driver.  But \iter\ drivers must occur at top-level.
Is this code in error?  No, because ~top-level~ is defined in \iter\
to include forms inside a |progn|.  This is just the definition of
top-level that Common Lisp uses, and for the same reason: to allow
macros to return multiple forms at top-level.

While our |for\dots in-whole-vector| clause will work, it is not
ideal.  In particular, it does not support generating.  Do do so, we
need to use |for\dots next| or |for\dots do-next|.  The job is
simplified by the |defmacro-driver| macro.

\begin{clauses}

\defmacro{defmacro-driver}{arglist |\&body| body}
Defines a driver clause in
both the |for| and |generate| forms, and provides a parameter
|generate| which ~body~ can examine to determine how it was invoked.
~arglist~ is as in |defmacro-clause|, and should begin with the symbol
|for|. 

\end{clauses}

With |defmacro-driver|, our driver looks like this:
\begin{program}
(defmacro-driver (FOR var IN-WHOLE-VECTOR v)
  "All the elements of a vector (disregards fill-pointer)"
   (let ((vect (gensym))
         (end (gensym))
         (index (gensym))
         (kwd (if generate 'generate 'for)))
     `(progn
        (with ,vect = ,v)
        (with ,end = (array-dimension ,vect 0))
        (with ,index = -1)
        (,kwd ,var next (progn (incf ,index)
                               (if (>= ,index ,end) (terminate))
                               (aref ,vect ,index))))))
\end{program}


We are still missing one thing: the |\&sequence| keywords.  
We can get them easily enough, by writing 
\begin{program}
(defmacro-driver (FOR var IN-WHOLE-VECTOR v \&sequence) 
  ...)
\end{program}
We can now refer to parameters |from|, |to|, |by|, etc. which contain
either the values for the corresponding keyword, or \nil\ if the
keyword was not supplied.  Implementing the right code for these
keywords is cumbersome but not difficult; it is left as an exercise.
But before you begin, see |defclause-sequence| below for an easier way.
       
\subsection{Extensibility Aids}
\label{aids}

This section documents assorted features that may be of use in
extending \iter.

\begin{clauses}

\defunexpvar{*result-var*}
Holds the variable that is used to return a value as a result of the
\iter\ form.  You may examine this and use it in a |with| clause, but
you should not change it.

\defmacro{defsynonym}{syn word}
 Makes ~syn~ a synonym for the existing \iter\ keyword ~word.~  Only
the first word in each clause can have synonyms. 

\defmacroxx{defclause-sequence}{element-name index-name}{|\&key|
access-fn size-fn sequence-type}{element-type
element-doc-string index-doc-string}  
Provides
a simple way to define sequence clauses.  Generates two
clauses, one for iterating over the sequence's elements, the other
for iterating over its indices.  The first symbol of both
clauses will have print-name |for|.
~element-name~  and ~index-name~ should be symbols.
~element-name~ is the second keyword of the element iterator (typically of
the form 
|in-~sequence-type~|), and ~index-name~ is the second keyword
of the index-iterator (typically of the form
|index-of-~sequence-type~|).  Either name may be 
\nil, in which case the corresponding clause is not defined.  If both
symbols are supplied, they should be in the same package.  The |for|
that begins the clauses will be in this package.

\cpar ~access-fn~ is the function to be used to
access elements of the sequence in the element iterator.  The function
should take two 
arguments, a sequence and an index, and return the appropriate element.
~size-fn~ should denote a function of one argument, a sequence, that
returns its size.  Both ~access-fn~ and ~size-fn~ are required for the
element iterator, but only ~size-fn~ is needed for the index iterator.

\cpar The ~sequence-type~ and ~element-type~ keywords are used to
suggest types for the variables 
used to hold the sequence and the
sequence elements, respectively.  The usual rules about \iter's
treatment of variable type declarations apply (see section \ref{types}).

\cpar ~element-doc-string~ and ~index-doc-string~ are
the documentation strings, for use with |display-iterate-clauses|. 

\cpar The generated element-iterator performs destructuring on the
element variable.

\cpar As an example, the above |for\dots in-whole-vector| example
could have been written:
\begin{program}
(defclause-sequence IN-WHOLE-VECTOR INDEX-OF-WHOLE-VECTOR
  :access-fn 'aref
  :size-fn \#'(lambda (v) (array-dimension v 0))
  :sequence-type 'vector
  :element-type t
  :element-doc-string 
     "Elements of a vector, disregarding fill-pointer"
  :index-doc-string 
     "Indices of vector, disregarding fill-pointer")
\end{program}

\end{clauses}

\subsection{Subtleties}

There are some subtleties to be aware of when writing \iter\ clauses.
First, the code returned by your macros may be |nconc|'ed into a list,
so you should always returned freshly consed lists, rather than
constants.  Second, \iter\ matches clauses by using |eq| on the first
symbol and |string=| on the subsequent ones, so the package of the
first symbol of a clause is relevant.  All of the clauses in this manual
have their first word in the \iter\ package.  
You can use the package system in the usual way to shadow
\iter\ clauses without replacing them.

%%% say more here, about the badness that only the first word of a
%%% clause is packagey.

\section{Non-portable Extensions to Iterate (Contribs)}
\label{contribs}

Currently, there is only one non-portable extension to iterate in the
distribution: iterate-pg. If you have made an extension that depends
on non-portable features, feel free to send them to |asf@boinkor.net|
for inclusion in the iterate distribution.

\subsection{An SQL query driver for iterate}

The pg package by Eric Marsden (see |http://cliki.net/pg|) provides an
interface to the PostgreSQL database. Using the \iterpg\ extension, it
is possible to handle the results of SQL queries using \iter.

This usage example should give you an idea of how to use it:

\begin{program}
(pg:with-pg-connection (c "somedb" "someuser")
  (iter (for (impl version date) in-relation "select * from version"
                                 on-connection *dbconn*)
        (collect version)))
\end{program}

To use the extension via |ASDF|, simply make your system depend on the
|iterate-pg| system instead of the |iterate| system. To load it
manually, use:

\begin{program}
  (asdf:oos 'asdf:load-op :iterate-pg)
\end{program}

%%%%%%%%%%%%%%%%%%%%%%%%%%%%%%%%%%%%%%%%%%%%%%%%%%%%%%%%%%%%%%%%
\section{Obtaining {\tt Iterate}}

\begin{sloppypar}
The information in this chapter is obsolete but included for
completeness's sake; Currently, the most up-to-date information on
\iter\ can be found at |http://boinkor.net/iterate.html|.
\end{sloppypar}

\begin{sloppypar}
\iter\ currently runs on Lisp Machines, and on
HP's, Sun3's and Sparcstations under Lucid.  
\iter\ source and binaries are available at the MIT AI Lab in the
subdirectories of |/src/local/lisplib/|.  The source file,
|iterate.lisp|, is also available for anonymous FTP in the directory
|/com/fpt/pub/| on the machine |TRIX.AI.MIT.EDU| (Internet number
128.52.32.6).  If you are unable to obtain |iterate| in one of these
ways, send mail to |jba@ai.mit.edu| and I will send you the source
file. 
\end{sloppypar}

\begin{sloppypar}
\iter\ resides in the |iterate| package (nickname |iter|).  Just say
\linebreak |(use-package :iterate)| to make all the necessary symbols
available. 
If a symbol is not exported, it appears in this manual with an
``|iterate::|'' prefix.
\end{sloppypar}

Send bug reports to |bug-iterate@ai.mit.edu|.  The |info-iterate|
mailing list will have notices of changes and problems; to have
yourself added, send mail to |info-iterate-request@ai.mit.edu|.


\medskip

\begin{flushleft}
 \bf Acknowledgements
\end{flushleft}
\smallskip

Richard Waters provided invaluable criticism which spurred me to improve
\iter\ greatly.  As early users, David Clemens, Oren Etzioni and Jeff
Siskind helped ferret out many bugs.

%\begin{theindex}
% The index files must be generated with the genindex program
% or from makeindex iterate-manual.idx which creates iterate-manual.ind.

%\baselineskip\normalbaselineskip
%\advance\baselineskip by -2pt

\input{iterate-manual.ind}

%\underline{Clauses}
%\input{iterate-manual.clindex}
%
%\indexspace
%\underline{Lisp}
%\input{iterate-manual.lispindex}

%\end{theindex}


\end{document}


% arch-tag: "c0f871a6-313c-11d8-abb9-000c76244c24"


%\underline{Clauses}
%% Fix ``.'' spacing in lists
\documentstyle[12pt,aimemo]{article}

\setlength{\textwidth}{6.0in}
\setlength{\oddsidemargin}{0.25in}
\setlength{\evensidemargin}{0.25in}
\setlength{\topmargin}{0pt}
\setlength{\headsep}{24pt}
\setlength{\textheight}{8.5in}

\let\origpar=\par

% somehow, LaTex fucks with things so that the plain Tex \obeyspaces
% doesn't work.  This is the same, except we use an mbox containing
% a space instead of just a space.

\newcommand{\spc}{\mbox{ }}
\newcommand{\tAb}{\mbox{        }}
\def\obspaces{\catcode`\ =\active\catcode`\^^I=\active}
{\obspaces \global\let =\spc\global\let^^I=\tAb}

\newenvironment{program}{\medskip\samepage\tt\obeylines\obspaces}{\medskip}

% use | for \tt
\catcode`\|=13
\newif\ifvbar
\def|{\ifvbar
         \endgroup\vbarfalse
      \else 
         \begingroup\tt\vbartrue 
      \fi}



\newcommand{\lisp}{\tt}
\newcommand{\iter}{|iterate|}
\newcommand{\iterpg}{|iterate-pg|}
\newcommand{\setf}{|setf|}
\newcommand{\nil}{|nil|}
\newcommand{\nonnil}{non-\nil}
\newcommand{\opt}{{\lisp \&optional}}
\newcommand{\yields}{$\Rightarrow$}



\newbox\Dots
\setbox\Dots=\hbox{\small ...}
\renewcommand{\dots}{\copy\Dots}

% Here's a way to define an environment which will treat all
% paragraphs inside it in the same manner.  It's, as near as I can
% figure, the way that Latex does things.  Using \everypar won't work;
% apparently Latex resets that a lot.  What you have to do is: at the
% top of the file do \let\origpar=\par.  In the
% begin part of the environment, do the parshape (or whatever), then say
% \def\par{{\origpar}}.  The extra braces are necessary.   That's all
% you have to do; at the end of the environment things will return to
% normal automatically.  Why does this all work?  I have no idea.
% Another important thing: there should be a blank line (i.e. a \par)
% between the end of the last paragraph and the \end{environment}.

\newlength{\clindent}
\setlength{\clindent}{0.5in}
\newlength{\clparindent}
\setlength{\clparindent}{\parindent}

\newenvironment{clauses}{\advance\linewidth -\parindent
                        \hangindent\clindent\relax
                         \hangafter=0
                          \parindent=0pt
                          \def\par{{\origpar}}}{}

\newcommand{\cpar}{\hspace \clparindent}

\newcommand{\startitem}{\bigskip\pagebreak[3]}
\newcommand{\finishitem}{\smallskip}

\newcommand{\clause}[1]{\startitem\Clause{#1}\finishitem}
\newcommand{\Clause}[1]{\hbox{#1}}
\newcommand{\clindex}[1]{\index{{\lisp #1}}}
\newcommand{\clindexx}[2]{\index{{\lisp #1\dots #2}}}
\newcommand{\clindexxx}[3]{\index{{\lisp #1\dots #2\dots #3}}}
\newcommand{\clausex}[2]{\startitem\Clausex{#1}{#2}\finishitem}
\newcommand{\Clausex}[2]{\Clause{|#1| #2}\clindex{#1}}

\newcommand{\defvar}[1]{\startitem\deff{#1}{}{Variable}\finishitem}
\newcommand{\defunexpvar}[1]{\startitem\deffnoindex{iterate::#1}{}{Variable}
                                        \lispindex{#1}\finishitem}
\newcommand{\defun}[2]{\startitem\deff{#1}{#2}{Function}\finishitem}
\newcommand{\defmacro}[2]{\startitem\deff{#1}{#2}{Macro}\finishitem}
\newcommand{\defunx}[3]{\startitem\deffx{#1}{#2}{#3}{Function}\finishitem}
\newcommand{\defmacrox}[3]{\startitem\deffx{#1}{#2}{#3}{Macro}\finishitem}
\newcommand{\defunxx}[4]{\startitem\deffxx{#1}{#2}{#3}{#4}{Function}\finishitem}
\newcommand{\defmacroxx}[4]{\startitem\deffxx{#1}{#2}{#3}{#4}{Macro}\finishitem}

\newcommand{\deff}[3]{\deffnoindex{#1}{#2}{#3}\lispindex{#1}}
\newcommand{\deffnoindex}[3]{\hbox to \hsize{{\tt #1} {\it #2}\hfill [{\it #3}]}}
\newcommand{\deffx}[4]{\deff{#1}{#2}{#4}\moreargs{#1}{#3}}
\newcommand{\deffxx}[5]{\deffx{#1}{#2}{#3}{#5}\moreargs{#1}{#4}}

\newcommand{\moreargs}[2]{\hbox to \hsize{\phantom{\lisp #1} {\it #2}\hfill}}

\newcommand{\lispindex}[1]{\index{{\lisp #1}}}

 \newenvironment{note}[1]{\pagebreak[2]\bigskip
                          \hrule\smallskip\small
                         {\setlength{\parindent}{0pt}
                           \par{\bf #1:}}}{\smallskip\hrule\bigskip}



\makeindex

%%%%%%%%%%%%%%%%%%%%%%%%%%%%%%%%%%%%%%%%%%%%%%%%%%%%%%%%%%%%%%%%

%\thispagestyle{empty}

%\vspace*{2in}

%\title{The |Iterate| Manual}
%\author{Jonathan Amsterdam}

%\maketitle

%\thispagestyle{empty}

%\begin{center}
%\Large *** DRAFT ***
%\end{center}
%This manual is in beta-test. It will become an AI Memo in a couple of
%months.  I would appreciate any comments or
%corrections.  You can reach me at email address |jba@ai.ai.mit.edu|.

\begin{document}

\def\writtenby{Jonathan Amsterdam}
\def\contractno{This report describes research done at the Artificial
Intelligence Laboratory of the Massachusetts Institute of Technology.
Support for the laboratory's artificial intelligence research is
provided in part by the Advanced Research Projects Agency of the
Department of Defense under Office
of Naval Research contract N00014-85-K-0124.}

\AIMEMO{1236}{\Large The Iterate Manual \\[1in]}
{This is the manual for version 1.4 of \iter, a powerful iteration
macro for Common  
Lisp. \iter\ is similar to {\tt loop} but provides numerous additional
features, is well integrated with Lisp, and is extensible. 
\vfill }


% use ~ for \em
\catcode`\~=13
\newif\ifem
\def~{\ifem
         \/\endgroup\emfalse
      \else 
         \begingroup\em\emtrue 
      \fi}



%\pagebreak
%\setcounter{page}{1}

%\thispagestyle{empty}


\tableofcontents

\pagebreak
\pagestyle{headings}

\section{Introduction}

\begin{sloppypar}
This manual describes
\iter, a powerful iteration facility for Common Lisp.
\iter\ provides abstractions for many common iteration
patterns and allows for the definition of additional patterns.  \iter\ is
a macro that expands into ordinary Lisp at
compile-time, so it is more efficient than higher-order
functions like |map| and |reduce.|  
While it is similar to |loop|, \iter\ offers a more Lisp-like syntax
and enhanced extensibility.
(For a more complete
comparison of \iter\ with other iteration constructs, see MIT AI Lab Working
Paper No. 324, ~Don't Loop, Iterate.~)
\end{sloppypar}

An \iter\ form consists of the symbol |iter|\footnote{You can also use
|iterate|, but |iter| is preferred because it avoids potential conflicts with
possible future additions to Common Lisp, and because it saves
horizontal space when writing code.}
followed by one or more
forms, some of which may be \iter\ 
~clauses.~  Here is a simple example of \iter\ which collects the
numbers from 1 to 10 into a list, and returns the list.  The return
value is shown following the arrow.
\begin{program}
(iter (for i from 1 to 10)
      (collect i))          \yields (1 2 3 4 5 6 7 8 9 10)
\end{program}
This form contains two clauses: a |for| clause that steps the
variable |i| over the integers from 1 to 10, and a |collect|
clause that accumulates its argument into a list.  With a few
exceptions, all \iter\ clauses have the same format: alternating
symbols (called ~keywords~) and expressions (called
~arguments~).  The syntax and terminology are those of Common
Lisp's keyword lambda lists.  One difference is that \iter's keywords
do not have to begin with a colon---though they may, except for the
first symbol of a clause.  So you can also write  |(for i :from
1 :to 10)| if you prefer.

Any Lisp form can appear in the body of an \iter, where it will have
its usual meaning.  \iter\ walks the entire body, expanding macros,
and recognizing clauses at any level.  This example collects all the
odd numbers in a list:
\begin{program}
(iter (for el in list)
      (if (and (numberp el) (oddp el))
          (collect el)))
\end{program}

There are clauses for iterating over numbers, lists, arrays and other
objects, and for
collecting, summing, counting, maximizing and other useful operations.
\iter\ also supports the creation of new variable bindings, stepping
over multiple sequences at once, destructuring, and compiler
declarations of variable types.  The following example illustrates
some of these features: 
\begin{program}
(iter (for (key . item) in alist)
         (for i from 0)
         (declare (fixnum i))
         (collect (cons i key)))
\end{program}

This loop takes the keys of an alist and returns a new alist
associating the keys with their positions in the original list. The
compiler declaration for |i| will appear in the generated code
in the appropriate place.

%%%%%%%%%%%%%%%%%%%%%%%%%%%%%%%%%%%%%%%%%%%%%%%%%%%%%%%%%%%%%%%%
\section{Clauses}

Most of \iter's clauses will be familiar to |loop| programmers.  
(|loop| is an iteration macro that has been incorporated into Common
Lisp.  Seen Guy Steele's {\em Common Lisp, 2nd Edition}.)
In
nearly all cases they behave the same as their |loop| counterparts, so
a |loop| user can switch to \iter\ with little pain (and much gain).

All clauses with the standard keyword-argument syntax consist of two
parts: a ~required~ part, 
containing keywords that must be present and in the right order; and
an ~optional~ part, containing keywords that may be omitted and,
if present, may occur in any order.  In the descriptions below, the
parts are separated by the Lisp lambda-list keyword \opt.


\subsection{Drivers}

An iteration-driving clause
conceptually causes the iteration to go forward.  Driver clauses in
\iter\ allow iteration over numbers, lists, vectors, hashtables, packages,
files and streams.  Iteration-driving clauses must
appear at the top level of an \iter\ form; they cannot be nested
inside another clause.  The driver variable is updated at the point
where the driver clause occurs.  Before the clause is executed for the
first time, the value of the variable is undefined.

%Also, regardless of where the driver clause
%appears in the body, the driver variable is stepped at the top of the
%loop; hence it is stylistically preferable, though
%not required, to place driver clauses at the beginning of the \iter.

Multiple drivers may appear in a single \iter\ form, in which case all
of the driver variables are updated each time through the loop, in the
order in which the clauses appear.  The first driver to terminate will
terminate the entire loop.

In all cases, the value of the driver variable on exit from the loop,
including within the epilogue code (see the |finally| clause), is
undefined.

All the parameters of a driver clause are evaluated once, before the
loop begins.  Hence it is not possible to change the bounds or other
properties of an iteration by side-effect from within the loop.

With one exception, driver clauses begin with the word |for| (or
the synonym |as|) and mention an iteration variable, which is
given a binding within the \iter\ form.
The exception is |repeat|, which just executes a loop a
specified number of times:

\begin{clauses}

\clausex{repeat}{~n~}
Repeats the loop ~n~ times.  For example:
\begin{program}
(iter (repeat 100)
      (print "I will not talk in class."))
\end{program}
\cpar If $n \leq 0$, the
loop will never be executed.  If ~n~ is not an integer, the
actual number of executions will be $\lceil n \rceil$.

\end{clauses}

\pagebreak[3]

\subsubsection{Numerical Iteration}

\begin{clauses}
\clausex{for}{~var~ |\&sequence|} 
The general form for iterating over a sequence of numbers requires a
variable and, optionally, one or more keywords that provide the bounds
and step size of the iteration. The |\&sequence|
lambda-list keyword is a shorthand for these sequence keywords.  They are:
|from|, |upfrom|, |downfrom|, |to|, |downto|, |above|, |below| and |by|.
|from| provides the starting value for ~var~
and defaults to zero. |to| provides a final value and implies that the
successive values of ~var~ will be increasing; |downto| implies that
they will be decreasing.  The loop terminates when ~var~ passes
the final value (i.e. becomes smaller or larger than it, depending on
the direction of iteration); in other  words, the loop body
will never be 
executed for values of ~var~ past the final value.
|below| and |above| are similar to |to| and
|downto|, except that the loop terminates when ~var~ equals or passes
the final value.

\cpar If no final value is specified, the variable will be stepped
forever.  Using |from| or |upfrom| will result in increasing
values, while |downfrom| will give decreasing values.

\cpar On each iteration,
~var~ is incremented or decremented by the value of the sequence
keyword |by|, which defaults to 1.  It should always be a positive
number, even for downward iterations.

\cpar In the following examples, the sequence of numbers generated is shown
next to the clause.

\begin{program}
(for i upfrom 0) \yields\ 0 1 2 \ldots
(for i from 5)   \yields\ 5 6 7 \ldots    ; either from or upfrom is okay
(for i downfrom 0) \yields\ 0 -1 -2 \ldots
(for i from 1 to 3) \yields\ 1 2 3
(for i from 1 below 3) \yields\ 1 2
(for i from 1 to 3 by 2) \yields\ 1 3
(for i from 1 below 3 by 2) \yields\ 1
(for i from 5 downto 3) \yields\ 5 4 3
\end{program}

\end{clauses}

\subsubsection{Sequence Iteration}

There are a number of clauses for iterating over sequences.  In all of
them, the argument following |for| may be a list instead of a
symbol, in which case destructuring is performed.  See section
\ref{destructuring}. 

\begin{clauses}

\clause{|for| ~var~ |in| ~list~ \opt\ |by| ~step-function~}
\clindexx{for}{in}
~var~ is set to successive elements of list.  
~step-function,~ which 
defaults to |cdr|, is used to obtain the next sublist. 

\clause{|for| ~var~ |on| ~list~ \opt\ |by| ~step-function~}
\clindexx{for}{on}
~var~ is set to successive sublists of list. 
~step-function~ (default |cdr|) is used as in |for\dots in|.

\end{clauses}

\medskip

These two clauses use |atom| to test for the end
of a list.  Hence, given a list whose final |cdr| is not \nil,
they will silently ignore the last |cdr|.  Other choices are
|endp|,  which would signal an error, and |null|, which
would probably result in an error somewhere else.  If you wish to use
an end-test other than |atom|, set the variable
|iterate::*list-end-test*|\lispindex{*list-end-test*} to the name of
the desired function. 

\begin{clauses}
\clause{|for| ~var~ |in-vector| ~vector~ |\&sequence|}
\clindexx{for}{in-vector}
~var~ takes on successive elements from ~vector.~   The vector's
fill-pointer is observed.  Here and in subsequent clauses, the |\&sequence|
keywords include |with-index|, which takes a symbol as argument
and uses it for the index variable instead of an internally generated
symbol.  The other |\&sequence| keywords behave as in numerical
iteration, except that the default iteration bounds are the bounds of
the vector.  E.g. in |(for i in-vector v downto 3)|,
|i| will start off being bound to the last element in |v|,
and will  
be set to preceding elements down to and including the element with
index 3.

\clause{|for| ~var~ |in-sequence| ~seq~ |\&sequence|}
\clindexx{for}{in-sequence}
This uses Common Lisp's generalized sequence functions, |elt|
and |length|, to obtain elements and determine the length of
~seq.~   Hence it will work for any sequence, including lists, and will
observe the fill-pointers of vectors.

\clause{|for| ~var~ |in-string| ~string~ |\&sequence|}
\clindexx{for}{in-string}
~var~ is set to successive characters of ~string.~

\startitem
\Clause{|for| ~var~ |index-of-vector| ~vector~ |\&sequence|}
\clindexx{for}{index-of-vector}
\Clause{|for| ~var~ |index-of-sequence| ~sequence~ |\&sequence|}
\clindexx{for}{index-of-sequence}
\Clause{|for| ~var~ |index-of-string| ~string~ |\&sequence|}
\clindexx{for}{index-of-string}
\finishitem
~var~ is set to successive indices of the sequence.
These clauses avoid the overhead of accessing the sequence elements
for those applications where they do not need to be examined, or are
examined rarely.  They admit all the optional keywords of the other
sequence drivers except the (redundant) |with-index| keyword.

\clause{|for| ~(key value)~ |in-hashtable| ~table~}
\clindexx{for}{in-hashtable}

~key~ and ~value,~ which must appear as shown in a list
and may be destructuring templates, are set to the keys and values of
~table~.  If ~key~ is |nil|, then the hashtable's keys will be ignored;
similarly for ~value~.
The order in which 
elements of ~table~ will be retrieved is unpredictable.

\clause{|for| ~var~ |in-package| ~package~ \opt\ |external-only| ~ext~}
\clindexx{for}{in-package}
Iterates over all the symbols in ~package,~ or over only the
external symbols if ~ext~ is specified and non-|nil|.  ~ext~ is not
evaluated.  The same symbol may appear more than once.

\clause{|for| ~var~ |in-packages| ~packages~ \opt\ |having-access| ~symbol-types~}
\clindexx{for}{in-packages}
Iterates over all the symbols from the list of packages denoted by the
descriptor ~packages~ and having accessibility (or visibility) given by
~symbol-types~.  This defaults to the list |(:external :internal :inherited)|
and is not evaluated.  ~var~ must be a list of up to three variables: in each
iteration, these will be set to a symbol, its access-type and package (as per
|with-package-iterator| in ANSI-CL).  The same symbol may appear more than
once.

\end{clauses}

\begin{clauses}

\clause{|for| ~var~ |in-file| ~name~ \opt\ |using| ~reader~}
\clindexx{for}{in-file}
Opens the file ~name~ (which may be a string or pathname) for input,
and iterates 
over its contents. ~reader~ defaults to |read|, so by default ~var~
will be bound to the successive forms in the file.  The \iter\ body is
wrapped in an unwind-protect to ensure that the file is closed no
matter how the \iter\ is exited.

\clause{|for| ~var~ |in-stream| ~stream~ \opt\ |using| ~reader~}
\clindexx{for}{in-stream}
Like |for\dots in-file|, except that ~stream~ should be an existing
stream object that supports input operations.

\end{clauses}

\subsubsection{Generalized Drivers}

These are primarily useful for writing drivers that can also be used
as generators (see section \ref{generators}, below).
 
\begin{clauses}

\clause{|for| ~var~ |next| ~expr~}
\clindexx{for}{next}

~var~ is set to ~expr~ each time through the loop.  Destructuring is
performed.  When the clause is used as a generator, ~expr~ is the code
that is executed when |(next ~var~)| is 
encountered (see section \ref{generators}, below).
~expr~ should compute the first value for ~var~, as well
as all subsequent values, and is responsible for terminating the loop.
For compatibility with future versions of \iter, this termination
should be done with |terminate|\clindex{terminate}, which can be
considered a synonym for |finish| (see section \ref{control-flow}).

\cpar As an example, the following clauses are equivalent to |(for
i from 1 to 10)|:
\begin{program}
(initially (setq i 0))
(for i next (if (> i 10) (terminate) (incf i)))
\end{program}

\clause{|for| ~var~ |do-next| ~form~}
\clindexx{for}{do-next}
~form~ is evaluated each time through the loop.  Its value is ~not~
set to ~var~; that is ~form's~ job. ~var~ is only present so that
\iter\ knows it is a driver variable.  \linebreak |(for ~var~ next
~expr~)| is equivalent to |(for ~var~ do-next (dsetq ~var~ ~expr~))|.
(See section \ref{destructuring} for an explanation of |dsetq|.) 

\end{clauses}

\subsubsection{Generators}
\label{generators}

In all of the above clauses, the driver variable is updated on each
iteration.
Sometimes it is desirable to have greater control over updating.  
For instance, consider the problem of associating numbers, in
increasing order and with no gaps, with the
\nonnil\ elements of a list.  One obvious first pass at writing this is:

\begin{program}
(iter (for el in list)
      (for i upfrom 1)
      (if el (collect (cons el i))))
\end{program}
But on the list |(a b nil c)| this produces |((a . 1) (b . 2) (c .
4))| instead of the desired |((a . 1) (b . 2) (c . 3))|.  The problem
is that |i| is incremented each time through the loop, even when |el|
is |nil|.  

The problem could be solved elegantly if we could step |i| only when
we wished to.  This
can be accomplished for any \iter\ driver by writing |generate| (or
its synonym |generating|)
instead of |for|.  Doing so produces a ~generator~---a driver whose
values are yielded explicitly.  To obtain the next value of a
generator variable ~v~, write \linebreak |(next ~v~)|.  The value of a |next|
form is the next value of ~v~, as determined by its associated driver
clause.  |next| also has the side-effect of updating ~v~ to that
value.  If there is no next value, |next| will terminate the loop,
just as with a normal driver.


Using generators, we can now write our example like this:
\begin{program}
(iter (for el in list)
      (generate i upfrom 1)
      (if el (collect (cons el (next i)))))
\end{program}
Now |i| is updated only when |(next i)| is executed, and this occurs
only when |el| is \nonnil.

To better understand the relationship between ordinary drivers and
generators, observe that we can rewrite an ordinary driver using its
generator form immediately followed by |next|, as this example shows:
\begin{program}
(iter (generating i from 1 to 10)
      (next i)
      ...)
\end{program}
Provided that the loop body contains no |(next i)| forms, this will
behave just as if we had written |(for i from 1 to 10)|.

We can still refer to a driver variable ~v~ without using |next|; in
this case, its value is that given to it by the last evaluation of
|(next ~v~)|.  Before |(next ~v~)| has been called the first time, the
value of ~v~ is undefined.  

This semantics is more flexible than
one in which ~v~ begins the loop bound to its first value and calls
of |next| supply subsequent values, because it means the loop will not
terminate too soon if the generator's sequence is empty.  For
instance, consider the following code, which tags \nonnil\ elements of
a list using a list of tags, and also counts the null elements.
(We assume there are at least as many tags as \nonnil\ elements.)
\begin{program}
(let* ((counter 0)
       (tagged-list (iter (for el in list)
                          (generating tag in tag-list) 
                          (if (null el)
                              (incf counter)
                              (collect (cons el (next tag)))))))
  ...)
\end{program}
It may be that there are just as many tags as non-null elements of
|list|.  If all the elements of |list| are null, we still want the
counting to proceed, even though |tag-list| is \nil.  If |tag| had to be
assigned its first value before the loop begins, we would have had to
terminate the loop before the first iteration, since when |tag-list|
is \nil, |tag| 
has no first value.  With the existing semantics, however, |(next
tag)| will never execute, so the iteration will cover all the elements of
|list|. 
        
When the ``variable'' of a driver clause is actually a destructuring
template containing several variables, all the variables are eligible
for use with |next|.  As before, |(next ~v~)| evaluates to ~v's~ next
value; but the effect is to update all of the template's variables.
For instance, the following code will return the list |(a 2 c)|.
\begin{program}
(iter (generating (key . item) in '((a . 1) (b . 2) (c . 3)))
      (collect (next key))
      (collect (next item)))
\end{program}

Only driver clauses with variables can be made into generators.  This
includes all clauses mentioned so far except for |repeat|.  It does
~not~ include |for\dots previous|, |for\dots =|,
|for\dots initially\dots then| or |for\dots first\dots then| (see
below). 

\subsubsection{Previous Values of Driver Variables}

Often one would like to access the value of a variable on a previous
iteration.  \iter\ provides a special clause for accomplishing this.

\begin{clauses}

\clause{|for| ~pvar~ |previous| ~var~ \opt\ |initially| ~init~
|back| ~n~}
\clindexx{for}{previous}
Sets ~pvar~ to the previous value of ~var,~ which should be a driver
variable, a variable from another |for\dots previous| clause, or a
variable established by a |for\dots =|, 
|for\dots initially\dots then| or |for\dots first\dots then| clause
(see section \ref{setting}).
Initially, ~pvar~ is given the value ~init~ (which defaults to \nil).  
The ~init~ expression will be moved outside the loop body, so it
should not depend on anything computed within the loop.
~pvar~ retains the value of ~init~ until ~var~ is set to its second
value, at which point ~pvar~ is set to ~var's~ first value; and so on.  

\cpar The
argument ~n~ to |back| must be a constant, positive integer, and
defaults to 1.  It determines how many iterations back ~pvar~ should
track ~var~.  For example, when ~n~ is 2, then ~pvar~ will be assigned
~var's~ first value when ~var~ is set to its third value.

\cpar A |for\dots previous| clause may occur after or before its
associated driver clause. |for\dots previous| works with generators as
well as ordinary drivers.

\pagebreak[3]

\cpar Example:
\begin{program}
(iter (for el in '(1 2 3 4))
      (for p-el previous el)
      (for pp-el previous p-el initially 0)
      (collect pp-el))
\end{program}
This evaluates to |(0 0 1 2)|.  It could have been written more
economically as 
\begin{program}
(iter (for el in '(1 2 3 4))
      (for pp-el previous el back 2 initially 0)
      (collect pp-el))
\end{program}


\end{clauses}

\subsection{Variable Binding and Setting}
\label{setting}

Several clauses exist for establishing new variable bindings or for
setting variables in the loop.  They all support destructuring.

\begin{clauses}

\clausex{with}{~var~ \opt\ |=| ~value~}
Causes ~var~ to be bound to value before the loop body is entered.
If ~value~ is not supplied, ~var~ assumes a default
binding, which will be 
\nil\ in the absence of declarations.  Also, if ~value~ is not
supplied, no destructuring is performed; instead, ~var~ may be a list of
symbols, all of which are given default 
bindings.  If ~value~ is supplied, ~var~ is bound to it, with
destructuring.  

\cpar Because |with|
creates bindings whose scope includes the entire \iter\ form, it is
good style to put all |with| clauses at the beginning.

\cpar Successive occurrences of |with| result in sequential
bindings (as with 
|let*|).  There is no way to obtain parallel bindings; see
section \ref{bindings} for a rationale.  


\clause{|for| ~var~ |=| ~expr~}
\clindexx{for}{=}
On each iteration, ~expr~ is evaluated and ~var~ is set
to its value. 

\cpar This clause may appear to do the same thing as |for\dots next|.
In fact, they are quite different.  |for\dots =| provides only three
services: it sets up a binding for ~var~, sets it to ~expr~ on each
iteration, and makes it possible to use |for\dots previous| with
~var~.  |for\dots next| provides these services in addition to the
ability to turn the driver into a generator.
%Also, the code which sets ~var~ appears in the loop body in the same
%place as the |for\dots =| clause; the code for |for\dots next| appears
%at the top of the loop, as with other drivers (except when being used
%as a generator).

\clause{|for| ~var~ |initially| ~init-expr~ |then| ~then-expr~}
\clindexxx{for}{initially}{then}
Before the loop begins, ~var~ is set to ~init-expr;~ on all
iterations after the first it is set to ~then-expr.~  This clause
must occur at top-level.  ~init-expr~ will be moved outside the loop
body and ~then-expr~ will be moved to the end of the loop body, so
they are subject to code motion problems (see section
\ref{code-movement}). 

\cpar This clause may appear to be similar to |for\dots next|, but in
fact they differ significantly.  |for\dots initially\dots then| is
typically used to give ~var~ its first value before the loop begins,
and subsequent values on following iterations.  This is incompatible
with generators, whose first value and subsequent values must all be
computed by |(next ~var~)|.  Also, the update of ~var~ in 
|for\dots initially\dots then| does not occur at the location of the
clause. 
Use |for\dots initially\dots then| for
one-shot computations where its idiom is more convenient, but use
|for\dots next| for extending \iter\ with new drivers (see section
\ref{extend}).

\clause{|for| ~var~ |first| ~first-expr~ |then| ~then-expr~}
\clindexxx{for}{first}{then}
The first time through the loop, ~var~ is set to ~first-expr;~ on
subsequent iterations, it is set to ~then-expr.~  This differs from
|for\dots initially| in that ~var~ is set to ~first-expr~
inside the loop body, 
so ~first-expr~ may depend on the results of other clauses.  For
instance, 
\begin{program}
(iter (for num in list)
      (for i first num then (1+ i))
      ...)
\end{program}
will set |i| to the first element of |list| on the first
iteration, whereas 
\begin{program}
(iter (for num in list)
      (for i initially num then (1+ i))
      ...)
\end{program}
is probably erroneous; |i| will be bound to |num|'s
default binding (usually \nil) for the first iteration.

\end{clauses}

\begin{note}{Compatibility note}
|loop|'s |for\dots =| works like \iter's, but |loop| used the syntax
|for\dots =\dots then| to mean |for\dots initially\dots then|.  It was
felt that these two operations were sufficiently different to warrant
different keywords.

Also, the |for| in the above three clauses is misleading,
since none is true driver (e.g. none has a corresponding |generate|
form).  |setting| would have been a better choice, but |for| was used
to retain some compatibility with |loop|.  
\end{note}


%%%%%%%%%%%%%%%%%%%%%%%%%%%%%%%%
\subsection{Gathering Clauses}
Many of \iter's clauses accumulate values into a variable, or set a
variable under certain conditions.  At the end of the loop, this
variable contains the desired result.  All these clauses have an
optional |into| keyword, whose argument should be a symbol.  If the
|into| keyword is not supplied, the accumulation variable will be
internally generated and its value will be returned at the end of the
loop; if a variable is specified, that variable is used for the
accumulation, and is not returned as a result---it is up to the user
to return it explicitly, in the loop's epilogue code (see |finally|).
It is safe to examine the accumulation variable during the loop, but
it should not be modified.

These clauses all begin with a verb.  When the verb does
not conflict with an existing Common Lisp function, then it may be
used in either its infinitival or present-participle form (e.g. |sum|,
|summing|).  However, when there is a conflict with Common Lisp, only
the present-participle form may be used (e.g. |unioning|).  This is to
prevent \iter\ clauses from clashing with Common Lisp
functions. 

%although these clauses are described as ``producing a value,'' it is a
%mistake to think of the lisp list representing the clause as a
%value-producing form in the usual way.  clauses may legally be written
%where a value is expected, e.g. |(setq x (sum i))|, but the
%lisp value of a clause in such a context is undefined.

\subsubsection{Reductions}

~Reduction~ is an extremely common iteration pattern in which
the results of successive applications of a
binary operation are accumulated.
For example, a loop that computes the sum of the
elements of a list is performing a reduction with the addition
operation.  This could be written in Common Lisp as  |(reduce \#'+
list)| or with \iter\ as
\begin{program}
(iter (for el in list)
      (sum el))
\end{program}

\begin{clauses}

\clausex{sum}{~expr~ \opt\ |into| ~var~}
Each time through the loop, ~expr~ is evaluated and added to a
variable, which is bound initially to zero.  If ~expr~ has a type,
it is ~not~ used as the type of the sum variable, which is always
|number|.  To get the result variable to be of a more specific
type, use an explicit variable, as in
\begin{program}
(iter (for el in number-list)
      (sum el into x)
      (declare (fixnum x))
      (finally (return x)))
\end{program}

\clausex{multiply}{~expr~ \opt\ |into| ~var~}
Like |sum|, but the initial value of the result variable is $1$,
and the variable is updated by multiplying ~expr~ into it.

\clausex{counting}{~expr~ \opt\ |into| ~var~}
~expr~ is evaluated on each iteration.  If it is \nonnil, the
accumulation variable, initially zero, is incremented.

\startitem
\Clausex{maximize}{~expr~ \opt\ |into| ~var~}
\Clausex{minimize}{~expr~ \opt\ |into| ~var~}
\finishitem
~expr~ is evaluated on each iteration and its extremum (maximum or
minimum) is stored in the accumulation variable.  If ~expr~ is never
evaluated, then the result is \nil\ (if the accumulation variable is
untyped) or $0$ (if it has a numeric type).

\clausex{reducing}{~expr~ |by| ~func~ \opt\ |initial-value| ~init-val~ |into| ~var~}
This is a general way to perform reductions. ~func~ should be a
function of two arguments, the first of which will be the value
computed so far and
the second of which will be the value of ~expr~.  It should return the
new value.
|reducing| is roughly equivalent to the
Common Lisp |(reduce ~func~ ~sequence~ :key ~expr-function~)|, where
~expr-function~ is used to derive values from the successive elements
of ~sequence~.

\cpar If the |reducing| clause is never executed, the result is
undefined.

\cpar It is not necessary to provide an
initial value, but better code can be generated if one is supplied.
Regardless of its location in the \iter\ body, ~init-val~ will be
evaluated before the loop is entered, so it should not depend on any
value computed inside the \iter\ form.  

%\cpar if a ~var~ is not specified, you can get \iter\ to declare the
%type of the internal variable by putting a |the| expression around
%~func.~  see section \ref{types}.

\end{clauses}

%\begin{note}{implementation note}
%in principle, |maximize| and |minimize| can be thought of
%as reductions where the initial value is the smallest (or largest)
%value that the accumulation variable can assume.  because lisp's
%bignums can represent arbitrary integers, these clauses cannot be
%implemented as reductions in general.  if, however, the type of 
%~expr~ or ~var~ can be determined to be a fixnum
%or a float, \iter\ will implement the clause as a true reduction,
%using one of the constants |most-negative-fixnum|, |%most-positive-fixnum|, 
%|most-negative-short-float|, etc. as appropriate.
%\end{note}

\subsubsection{Accumulations}

All the predefined accumulation clauses add values to a sequence.  If
the sequence is a list, they all 
have the property that the partial list is kept in the correct order
and available for inspection at any point in the loop.

\begin{clauses}

\clausex{collect}{~expr~ \opt\ |into| ~var~ |at| ~place~ |result-type| ~type~}
Produces a sequence of the values of ~expr~ on each iteration. ~place~
indicates where the next value of ~expr~ is added to the list and may
be one of the symbols |start|, |beginning| (a synonym for |start|) or
|end|.  The symbol may be quoted, but need not be.  The default is
|end|.  For example,
\begin{program}
(iter (for i from 1 to 5)
      (collect i))
\end{program}
produces |(1 2 3 4 5)|, whereas
\begin{program}
(iter (for i from 1 to 5)
      (collect i at beginning))
\end{program}
produces |(5 4 3 2 1)| (and is likely to be faster in most Common Lisp
implementations).

\cpar If ~type~ is provided, it should be a subtype of |sequence|.
The default is |list|.  Specifying a type other than |list| will
result in |collect| returning a sequence of that type.  ~However,~ the
type of the sequence being constructed when inside the loop body is
undefined when a non-|list| type is specified.  (As with ~place~,
quoting ~type~ is optional.)

\clausex{adjoining}{~expr~ \opt\ |into| ~var~ |test| ~test~ |at| ~place~ |result-type| ~type~}
Like |collect|, but only adds the value of ~expr~ if it is not
already present.  ~test,~ which defaults to |\#'eql|, is
the test to be used with |member|.

\startitem
\Clausex{appending}{~expr~ \opt\ |into| ~var~ |at| ~place~}
\Clausex{nconcing}{\ ~expr~ \opt\ |into| ~var~ |at| ~place~}
\Clausex{unioning}{\ ~expr~ \opt\ |into| ~var~ |test| ~test~ |at| ~place~}
\Clausex{nunioning}{~expr~ \opt\ |into| ~var~ |test| ~test~ |at| ~place~}
\finishitem
These are like |collect|, but behave like the Common Lisp functions
|append|, |nconc|, |union| or |nunion|.  
As in Common Lisp, they work only on lists.  Also as in Common Lisp,
|unioning| and |nunioning| assume that the value of ~expr~ contains no
duplicates. 

\clausex{accumulate}{~expr~ |by| ~func~ \opt\ |initial-value| ~init-val~ |into| ~var~}
This is a general-purpose accumulation clause. ~func~ should be a
function of two arguments, the value of ~expr~ and the value
accumulated so far in the iteration, and it should return the updated
value.  If no initial value is supplied, \nil\ is used.  

%\cpar If a ~var~ is not specified, you can get \iter\ to declare the
%type of the internal variable by putting a |the| expression around
%~func.~  see section \ref{types}.

\cpar The differences between |accumulate| and |reducing| are slight.
One difference is that the functions take their arguments in a
different order.  Another is that in the absence of ~init-val~,
|accumulate| will use \nil, whereas |reducing| will generate different
code that avoids any dependence on the initial value.
The reason for having both clauses is that one usually
thinks of reductions (like |sum|) and accumulations (like |collect|)
as different beasts.

\end{clauses}


\subsubsection{Finders}

A ~finder~ is a clause whose value is an expression that meets some
condition.

\begin{clauses}

\clause{|finding| ~expr~ |such-that| ~test~ \opt\ |into| ~var~ |on-failure| ~failure-value~}
\clindexx{finding}{such-that}
If ~test~ (which is an expression) ever evaluates to \nonnil, the loop
is terminated, the 
epilogue code is run and the value of ~expr~ is returned.  Otherwise,
\nil\ (or ~failure-value,~ if provided) is returned.  If ~var~ is
provided, it will have either the \nonnil\ value of ~expr~ or
~failure-value~ when the epilogue code is run.

\cpar As a special case, if the ~test~ expression is a sharp-quoted
function, then it is applied to ~expr~ instead of being simply
evaluated.  E.g. |(finding x such-that \#'evenp)| is equivalent to
|(finding x such-that (evenp x))|.  

%\cpar although ~test~ need have nothing to do with ~%expr~ as in
%|(finding j such-that (> i 3))|, it usually 
%will: |(finding (length el) such-that (oddp (length el)))|.  to
%avoid performing the |length| computation twice, you could write
%|(finding (length el) such-that \#'oddp)| or |(finding (length
%el) such-that 'oddp)|; for these cases, \iter\ generates code that
%executes ~expr~ only once.  the code for |\#'oddp|
%is slightly different from that for {\lisp 'oddp}; see the discussion
%under {\lisp for\dots in} and {\lisp for\dots on}.

\cpar |On-failure| is a misnomer. Because it is always evaluated, it behaves
more like the default third argument to the |gethash| function. As a result,
|on-failure (error "Not found")| makes no sense. Instead, the clauses |leave|
or |thereis| can be used in conjunction with |finally| as follows:
\begin{program}
(iter (for x in '(1 2 3))
      (if (evenp x) (leave x))
      (finally (error "not found")))
\end{program}

\cpar This clause may appear multiple times when all defaults are
identical. It can also be used together with either |always|/|never| or
|thereis| if their defaults match. More specifically, |on-failure nil| is
compatible with |thereis|, while |on-failure t| is compatible with |always|
and |never| clauses.
\begin{program}
(iter (for i in '(7 -4 2 -3))
      (if (plusp i)
	  (finding    i  such-that (evenp i))
          (finding (- i) such-that (oddp i))))
\end{program}

\startitem
\Clause{|finding| ~expr~ |maximizing| ~m-expr~ \opt\ |into| ~var~}
\clindexx{finding}{maximizing}
\Clause{|finding| ~expr~ |minimizing| ~m-expr~ \opt\ |into| ~var~}
\clindexx{finding}{minimizing}
\finishitem
Computes the extremum (maximum or minimum) value of ~m-expr~ over all
iterations, and returns the value of ~expr~ corresponding to the
extremum.  ~expr~ is evaluated inside the loop at the time the new
extremum is established.  If ~m-expr~ is never evaluated (due to, for
example, being embedded in a conditional clause), then the returned
value depends on the type, if any, of ~expr~ (or ~var,~ if
one is supplied).  If there is no type, the returned
value will be \nil; if the type is numeric, the returned value will be
zero.

\cpar For these two clauses, ~var~ may be a list of two
symbols; in that case, the first is used to record ~expr~ and
the second, ~m-expr.~  

\cpar As with |finding\dots such-that|, if ~m-expr~ is a sharp-quoted
function, then it is called on ~expr~ instead of being evaluated.

\end{clauses}

\subsubsection{Boolean Tests}

\begin{clauses}

\clausex{first-iteration-p}{}
Returns |t| in the first cycle of the loop, otherwise \nil.

\clausex{first-time-p}{}
Returns |t| the first time the expression is evaluated, and then \nil\ forever.
This clause comes handy when printing (optional) elements separated
by a comma:

\begin{program}
(iter (for el in '(nil 1 2 nil 3))
      (when el
        (unless (first-time-p)
          (princ ", "))
        (princ el)))
\end{program}
produces |"1, 2, 3"|.

\end{clauses}

\subsubsection{Aggregated Boolean Tests}

\begin{clauses}

\clausex{always}{~expr~}
If ~expr~ ever evaluates to
\nil, then \nil\ is immediately returned; the epilogue code is not
executed.  If ~expr~ never evaluates to \nil, the epilogue code
is executed and the last value of ~expr~ (or |t| if ~expr~ was never
evaluated) is returned (whereas |loop| would constantly return |t|).

% mention last evaluated clause when multiple always clauses?

\clausex{never}{~expr~}
Like |(always (not ~expr~))|, except it does not influence the last
value returned by a possible other |always| clause. That is,
\begin{program}
(iter (repeat 2)
      (always 2)
      (never nil)) \yields 2 ; not t
\end{program}

\clausex{thereis}{~expr~}
If ~expr~ is ever \nonnil,
its value is immediately returned without running epilogue code.
Otherwise, the epilogue code is performed and \nil\ is returned.

This clause cannot be used together with |always| or |never|, because their
defaults are opposed (similarly, |(loop always 3 thereis nil)| refuses to
compile in some implementations of |loop|).

\end{clauses}


\subsection{Control Flow}
\label{control-flow}
Several clauses can be used to alter the usual flow of control in a loop.

Note: the clauses of this and subsequent sections don't adhere to \iter's
usual syntax, but instead use standard Common Lisp syntax.  Hence the
format for describing syntax subsequently is
like the standard format used in the Common Lisp manual, not like the
descriptions of clauses above.

\begin{clauses}

\clausex{finish}{}
Stops the loop and runs the epilogue code.

%for example:
%
%\begin{program}
%(iter (with answer = nil)
%         (initially (make-a-mess))
%         (for i from 1 to 10)
%         (when (correct? i)
%           (setq answer i)
%           (finish))
%         (finally (cleanup)))
%\end{program}
%
%this code will execute |cleanup| whether or not the test |(correct?
%i)| ever succeeds. 
%the (more elegant) formulation,
%\begin{program}
%(iter (initially (make-a-mess))
%         (for i from 1 to 10)
%         (finding i such-that (correct? i))
%         (finally (cleanup)))
%\end{program}
%would not execute |cleanup| if |(correct? i)| succeeded; it
%would do an immediate return.

\clausex{leave}{\opt\ ~value~}
Immediately returns ~value~ (default \nil) from the current \iter\
form, skipping the epilogue code.  Equivalent to using |return-from|.

\clausex{next-iteration}{}
Skips the remainder of the loop body and begins the next iteration of
the loop.

\clausex{while}{~expr~}
If ~expr~ ever evaluates to \nil, the loop is terminated and the
epilogue code executed.  Equivalent to |(if (not ~expr~) (finish))|.

\clausex{until}{~expr~}
Equivalent to |(if ~expr~ (finish))|.

\clausex{if-first-time}{~then~ \opt\ ~else~}
If this clause is being executed for the first time in this invocation
of the \iter\ form, then the ~then~ code is evaluated; otherwise the
~else~ code is evaluated.

\cpar |(for ~var~ first ~expr1~ then ~expr2~)| is almost equivalent to
\begin{program}
(if-first-time (dsetq ~var~ ~expr1~)
               (dsetq ~var~ ~expr2~))
\end{program}
The only difference is that the |for| version makes ~var~ available
for use with |for\dots previous|.

\end{clauses}

\subsection{Code Placement}
When fine control is desired over where code appears in a loop
generated by \iter, the following special clauses may be useful.
They are all subject to code-motion problems (see section
\ref{code-movement}).  

\begin{clauses}

\clausex{initially}{|\&rest| ~forms~}
The lisp ~forms~ are placed in the prologue section of the loop, where
they are executed once, before the loop body is entered.

\clausex{after-each}{|\&rest| ~forms~}
The ~forms~ are placed at the end of the loop body, where they
are executed after each iteration.  Unlike the other clauses in this
section, ~forms~ may contain \iter\ clauses.

\clausex{else}{|\&rest| ~forms~}
The lisp ~forms~ are placed in the epilogue section of the loop, where they
are executed if this |else| clause is never met during execution of the
loop and the loop terminates normally.

\clausex{finally}{|\&rest| ~forms~}
The lisp ~forms~ are placed in the epilogue section of the loop, where
they are executed after the loop has terminated normally.

\clausex{finally-protected}{|\&rest| ~forms~}
The lisp ~forms~ are placed in the second form of an unwind-protect
outside the loop.  They are always executed after the loop has
terminated, regardless of how the termination occurred.

\end{clauses}


%%%%%%%%%%%%%%%%%%%%%%%%%%%%%%%%%%%%%%%%%%%%%%%%%%%%%%%%%%%%%%%%
\section{Other Features}

\subsection{Multiple Accumulations}
\label{multiple}

\begin{sloppypar}
It is permitted to have more than one clause accumulate into the same
variable, as in the following:
\begin{program}
(iter (for i from 1 to 10)
      (collect i into nums)
      (collect (sqrt i) into nums)
      (finally (return nums)))
\end{program}
Clauses can only accumulate into the same variable if they are
compatible.  |collect|, |adjoining|, |appending|, |nconcing|,
|unioning| and |nunioning| are compatible with each other; |sum|,
|multiply| and |counting| are compatible; |always| and |never| are compatible;
|finding| \dots |such-that| is compatible with either |thereis| or |always|
and |never| when their defaults match; and |maximize| and |minimize| clauses
are compatible only with other |maximize| and |minimize| clauses,
respectively.

%note that the same variable ~cannot~ be both an accumulation
%variable and an ordinary variable; there can be only one variable with
%a given name within an \iter\ form. 
\end{sloppypar}

\subsection{Named Blocks}

Like Common Lisp |block|s, \iter\ forms can be given names.  The
name should be a single symbol, and it must be the first form in the
\iter.  The generated code behaves exactly like a named block; in
particular, |(return-from ~name~)| can be used to exit it:
\begin{program}
(iter fred
      (for i from 1 to 10)
      (iter barney
            (for j from i to 10)
            (if (> (* i j) 17)
                (return-from fred j))))
\end{program}
An \iter\ form that is not given a name is implicitly named \nil.

Sometimes one would like to write an expression in an inner \iter\ form,
but have it processed by an outer \iter\ form.  This is possible with
the |in| clause.  

\begin{clauses}

\clausex{in}{~name~ |\&rest| ~forms~}
Evaluates ~forms~ as if they were part of the \iter\ form named
~name~.  In other words, \iter\ clauses are processed by the \iter\
form named ~name,~ and not by any \iter\ forms that occur inside ~name.~

\cpar As an example, consider the problem of collecting a list of the
elements in a two-dimensional array.  The naive solution,
\begin{program}
(iter (for i below (array-dimension ar 0))
      (iter (for j below (array-dimension ar 1))
            (collect (aref ar i j))))
\end{program}
\noindent is wrong because the list created by the inner \iter\ is simply
ignored by the outer one.  But using |in| we can write:
\begin{program}
(iter outer (for i below (array-dimension ar 0))
      (iter (for j below (array-dimension ar 1))
            (in outer (collect (aref ar i j)))))
\end{program}
\noindent which has the desired result.

\end{clauses}

\subsection{Destructuring}
\label{destructuring}

In many places within \iter\ clauses where a variable is expected, a
list can be written instead.  In these cases, the value to be assigned
is ~destructured~ according to the pattern
described by the list.  As a simple example, the clause
\begin{program}
(for (key . item) in alist)
\end{program}
\noindent will result in |key| being set to the |car| of
each element in |alist|, and |item| being set to the |cdr|.  The
pattern list may be nested to arbitrary depth, and (as the example
shows) need not be terminated with \nil; the only requirement is that
each leaf be a bindable symbol (or \nil, in which case no binding is
generated for that piece of the structure).

Sometimes, you might like to do the equivalent of a
|multiple-value-setq| in a clause.  This 
``multiple-value destructuring'' can be expressed by writing 
\linebreak |(values $pat_1$ $pat_2 \ldots$)| for a destructuring
pattern, as in 
\begin{program}
(for (values (a . b) c d) = (three-valued-function ...))
\end{program}
\begin{sloppypar}
Note that the $pat_i$ can themselves be destructuring patterns (though
not multiple-value destructuring patterns).  You can't do multiple-value
destructuring in a |with| clause; instead wrap the whole \iter\
form in a |multiple-value-bind|.
\end{sloppypar}

\begin{note}{Rationale}
There are subtle interactions between variable declarations and
evaluation order that make the correct implementation of
multiple-value destructuring in a |with| somewhat tricky.
\end{note}

The destructuring feature of \iter\ is available as a separate
mechanism, using the |dsetq| macro:

\begin{clauses}
\defmacro{dsetq}{template expr}
Performs destructuring of ~expr~ using ~template~.   May be used
outside of an \iter\ form. Yields the primary value of ~expr~.

\end{clauses}

\subsection{On-line Help}

\begin{sloppypar}
There is a limited facility for on-line help, in the form of the 
|display-iterate-clauses| function.
\end{sloppypar}

\begin{clauses}

\defun{display-iterate-clauses}{\opt\ clause-spec}
Displays a list of \iter\ clauses.  If ~clause-spec~ is not
provided, all clauses are shown; if it is a symbol, all clauses
beginning with that symbol are shown; and if it is a list of symbols,
all clauses for which ~clause-spec~ is a prefix are shown.

\end{clauses}

\subsection{Parallel Binding and Stepping}
\label{bindings}

The parallel binding and stepping of variables is a feature that
\iter\ does ~not~ have.  This section attempts to provide a rationale.

We say that two variables are bound ~in parallel~ if neither
binding shadows the other.  This is the usual semantics of |let|
(as opposed to |let*|).  Similarly, we can say that iteration
variables are stepped in parallel if neither variable is updated
before the other, conceptually speaking; in other words, if the code
to update each variable can reference the old values of both variables.

|loop| allows parallel binding of variables and parallel stepping of
driver variables.  My view is that if you are depending on the 
serial/parallel distinction, you are doing something obscure.  If you
need to bind 
variables in parallel using |with|, then you must be using a
variable name that shadows a name in the existing lexical environment.
Don't do that.  The most common use for parallel stepping is to track
the values of variables on the previous iteration, but in fact this
does not require parallel stepping at all; the following will work:
\begin{program}
(iter (for current in list)
      (for prev previous current)
      ...)
\end{program}


%%%%%%%%%%%%%%%%%%%%%%%%%%%%%%%%%%%%%%%%%%%%%%%%%%%%%%%%%%%%%%%%
\section{Types and Declarations}
\label{types}

\subsection{Discussion}

Sometimes efficiency dictates that the types of variables be declared.
This type information needs to be communicated to \iter\ so it can
bind variables to appropriate values.  Furthermore, \iter\ must often
generate internal variables invisible to the user; there needs to be a
way for these to be declared.

As an example, consider this code, which will return the number of
odd elements in |number-list|:
\begin{program}
(iter (for el in number-list)
      (count (oddp el)))
\end{program}
In processing this form,
\iter\ will create an internal variable, let us call it |list17|, to
hold the successive |cdr|s of |number-list|, and  
will bind the variable to |number-list|.  It will also generate a
default binding for |el|; only inside the body of the loop will |el|
be set to the |car| of |list17|.  Finally, \iter\ will generate a
variable, call it |result|, to hold the result of the count, and will
bind it to zero.

When dealing with type declarations, \iter\ observes one simple rule:
~it will never generate a declaration unless requested to do so.~  The
reason is that such declarations might mask errors in compiled code by
avoiding error-checks; the resulting problems would be doubly hard to
track down because the declarations would be hidden from the
programmer.  Of course, a compiler might omit error-checks even in the
absence of declarations, though this behavior can usually be avoided,
e.g. by saying |(declaim (optimize (safety 3)))|.

So, the above \iter\ form will generate code with no declarations.
But say we wish to declare the types of |el| and the internal
variables |list17| and |result|.  How is this done?

Declaring the type of |el| is easy, since the programmer knows
the variable's name:
\begin{program}
(iter (for el in number-list)
      (declare (fixnum el))
      (counting (oddp el)))
\end{program}
\iter\ can read variable type declarations like this one.  Before
processing any clauses, it scans the entire top-level form for type
declarations 
and records the types, so that variable bindings can be performed
correctly.  In this case, |el| will be bound to zero
instead of \nil.  Also, \iter\ collects all the top-level declarations
and puts them at the begining of the generated code, so it is not
necessary to place all declarations at the beginning of an \iter\
form; instead, they can be written near the variables whose types they
declare.  

Since \iter\ is not part of the compiler, it will not know
about declarations that occur outside an \iter\ form; these
declarations must be repeated inside the form.

Here is another way we could have declared the type of |el|:
\begin{program}
(iter (for (the fixnum el) in number-list)
      (counting (oddp el)))
\end{program}
\iter\ extends the Common Lisp |the|\lispindex{the}
form to apply to variables as well as value-producing forms; anywhere
a variable is allowed---in a |with| clause, as the iteration
variable in a driver clause, as the |into| argument of an
accumulation clause, even inside a destructuring template---you can
write |(the ~type~ ~symbol~)| instead.

There is one crucial difference between using a |the| form and
actually declaring the variable: explicit declarations are always
placed in the generated code, but type information from a |the|
form is not turned into an actual declaration unless you tell \iter\
to do so using |iterate:declare-variables|.  See below.

\begin{sloppypar}
Declaring the types of internal variables is harder than declaring the
types of explicitly mentioned variables, since their names
are unknown.  You do it by declaring |iterate:declare-variables|
somewhere inside the top level of the \iter\ form.  (This will also
generate declarations for variables declared using |the|.)
\iter\ does not provide much selectivity here: it's all or none. 
And unfortunately, since \iter\ is not privy to compiler information
but instead reads declarations itself, it will not hear if you
|(declaim (iterate:declare-variables))|.  Instead, set the variable 
|iterate::*always-declare-variables*| to |t| at
compile-time, using |eval-when|.
\end{sloppypar}

To determine the appropriate types for internal variables, \iter\ uses
three sources of information:

\begin{itemize}

\item Often, the particular clause dictates a certain type for a
variable; \iter\ will use this information when available.  In the
current example, the variable |list17| will be given the type
|list|, since that is the only type that makes sense; and the
variable |result| will be given the type |fixnum|, on the
assumption that you will not be counting high enough to need bignums.
You can override this assumption only by using and explicitly declaring a
variable: 
\begin{verbatim}
(iter (declare (iterate:declare-variables))
      (for el in number-list)
      (count (oddp el) into my-result)
      (declare (integer my-result))
      (finally (return my-result)))
\end{verbatim}

\begin{sloppypar}
Other examples of the type assumptions that \iter\ makes are: type
|list| for |into| variables of collection clauses; type |list| for
expressions that are to be destructured; type |vector| for the
variable holding the vector in a |for\dots in-vector| clause, and
similarly for |string| and the |for\dots in-string| clause;
and the implementation-dependent type
|(type-of array-dimension-limit)| for the index and limit
variables generated by sequence iteration drivers like |for\dots
in-vector| and |for\dots in-string| (but not |for\dots in-sequence|,
because it may be used to iterate over a list). 
\end{sloppypar}

\item Sometimes, \iter\ will examine expressions and try to determine
their types in a simple-minded way.  If the expression is
self-evaluating (like a number, for instance), \iter\ knows that the
expression's type is the same as the type of the value it denotes, so
it can use that type.  If the expression is of the form |(the ~type~
~expr~)|, \iter\ is smart enough to extract ~type~ and use it.
However, the current version of \iter\ does not  
examine declarations of function result types or do any type
inference.  It will not determine, for 
example, that the type of |(+ 3 4)| is |fixnum|, or even
|number|.

\item In some cases, the type of an internal variable should match the
type of some other variable.  For instance, \iter\ generates an
internal variable for |(f x)| in the
clause |(for i from 1 to (f x))|, and in the absence of other
information will give it the same type as |i|.  If, however, the
expression had been written |(the fixnum (f x))|, then \iter\
would have given the internal variable the type |fixnum|
regardless of |i|'s type.  The type incompatibility errors that
could arise in this situation are not checked for.

\end{itemize}

Note that if you do declare |iterate:declare-variables|, then
\iter\ may declare user variables as well as internal ones if they do
not already have declarations, though only for variables that it
binds.  For instance, in this code:

\begin{program}
(iter (declare (iterate:declare-variables))
      (for i from 1 to 10)
      (collect i into var))
\end{program}
the variable |var| will be declared to be of type |list|.


\subsection{Summary}

\iter\ understands standard Common Lisp variable type declarations
that occur within an \iter\ form and
will pass them through to the generated code.  If the declaration
|(iterate:declare-variables)|\lispindex{declare-variables}
appears at the top level of an 
\iter\ form, or if 
|iterate::*always-declare-variables*|\lispindex{*always-declare-variables*} 
is \nonnil, then \iter\ will use the type information gleaned from user
declarations, self-evaluating expressions and |the| expressions,
combined with reasonable assumptions, to determine variable
types and declare them.


%%%%%%%%%%%%%%%%%%%%%%%%%%%%%%%%%%%%%%%%%%%%%%%%%%%%%%%%%%%%%%%%
\section{Problems with Code Movement}
\label{code-movement}

Some \iter\ clauses, or parts of clauses, result in code being
moved from the location of the clause to other parts of the loop.
Drivers behave this way, as do code-placement clauses like |initially|
and |finally|.  When using these clauses, there is a danger of writing
an expression that makes sense in its apparent location but will be
invalid or have a different meaning in another location.  For example:
\begin{program}
(iter (for i from 1 to 10)
      (let ((x 3))
        (initially (setq x 4))))
\end{program}
While it may appear that the |x| of |(initially (setq x 4))| is the
same as the |x| of |(let ((x 3)) \dots|, in fact they are not:
|initially| moves its code outside the loop body, so |x| would refer
to a global variable.  Here is another example of the same problem:
\begin{program}
(iter (for i from 1 to 10)
      (let ((x 3))
        (collect i into x)))
\end{program}
If this code were executed, |collect| would create a binding for its
|x| at the top level of the \iter\ form that the |let| will shadow.

Happily, \iter\ is smart enough to catch these errors; it
walks all problematical code to ensure that free variables are not
bound inside the loop body, and checks all variables it binds for the
same problem.

However, some errors cannot be caught:

\begin{program}
(iter (with x = 3)
      (for el in list)
      (setq x 1)
      (reducing el by \#'+ initial-value x))
\end{program}
|reducing| moves its |initial-value| argument to the initialization
part of the loop in order to produce more efficient code.  Since
\iter\ does not perform data-flow analysis, it cannot determine that
|x| is changed inside the loop; all it can establish is that |x| is
not bound internally.  Hence this code will not signal an
error and will use $3$ as the initial value of the reduction.

The following list summarizes all cases that are subject to these code
motion and variable-shadowing problems.
\begin{itemize}
\item Any variable for which \iter\ creates a binding, including those
used in |with| and the |into| keyword of many clauses.

\begin{sloppypar}
\item The special clauses which place code: |initially|, |after-each|, |else|,
|finally| and |finally-protected|. 
\end{sloppypar}

\item The variables of a |next| or |do-next| form.

\item The |initially| arguments of |for\dots initially\dots then| and
|for\dots previous|. 

\item The |then| argument of |for\dots initially\dots then|.

\item The |initial-value| arguments of |reducing| and |accumulate|.

\item The |on-failure| argument of |finding\dots such-that|.
\end{itemize}

%%%%%%%%%%%%%%%%%%%%%%%%%%%%%%%%%%%%%%%%%%%%%%%%%%%%%%%%%%%%%%%%
\section{Differences Between {\tt Iterate} and {\tt Loop}}
|loop| contains a great deal of complexity which \iter\ tries to
avoid.  Hence many esoteric features of |loop| don't exist in \iter.
Other features have been carried over, but in a cleaned-up form.  
And of course, many new features have been added; they are not
mentioned in this list.

\begin{itemize}

\item \iter's syntax is more Lisp-like than |loop|'s, having a higher
density of parens.

\item The current implementation of \iter, unlike the current version
of  |loop| (as documented in {\em Common Lisp, 2nd Ed.\/}), is
extensible (see section \ref{extend}).  

\item |loop| puts the updates of all driver variables at the top of
the loop; \iter\ leaves them where the driver clauses appear.

\item While for the most part \iter\ clauses that resemble |loop| clauses
behave similarly, there are some differences.  For instance, there is
no |for\dots =\dots then| in \iter; instead use 
|for\dots initially\dots then|.  

\item |loop| binds the variable |it| at certain times
to allow pseudo-English expressions like |when ~expr~ return it|. 
In \iter, you must bind ~expr~ to a variable yourself.  Note that
|when ~expr~ return it| is like |thereis ~expr~| except that the latter is an
accumulation clause and therefore competes with other accumulations
(remember section \ref{multiple} above).

% repeat different behaviour of |always| clause here?

\item |loop| has a special |return| clause, illustrated in the
previous item.  \iter\ doesn't need one,  since an ordinary Lisp
|return| has the same effect.

\item |loop| allows for parallel binding and stepping of iteration
variables.  \iter\ does not.  (See section \ref{bindings}.)

\item |loop| and \iter\ handle variable type declarations very
differently.  |loop| provides a special syntax for declaring variable
types, and does not examine declarations.  Moreover, the standard
implementation of |loop| will
generate declarations when none are requested.
\iter\ parses standard Common Lisp type declarations, and will never
declare a variable itself unless declarations are 
specifically requested.

\end{itemize}

%%%%%%%%%%%%%%%%%%%%%%%%%%%%%%%%%%%%%%%%%%%%%%%%%%%%%%%%%%%%%%%%
\section{Rolling Your Own}
\label{extend}

\subsection{Introduction}

\iter\ is extensible---you can write new clauses that embody new
iteration patterns.  You might want to write a new driver clause for a
data structure of your own, or you might want to write a clause that
collects or manipulates elements in a way not provided by \iter. 

This section describes how to write clauses for \iter.  Writing a
clause is like writing a macro.  In fact, writing a clause ~is~
writing a macro: since \iter\ code-walks its body and macroexpands,
you can add new abstractions to \iter\ with good old |defmacro|.

Actually, there are two extensions you can make to \iter\ that are
even easier than writing a macro.  They are adding a synonym for an
existing clause and defining a driver clause for an indexable
sequence.  These can be done with |defsynonym| and
|defclause-sequence|, respectively.  See section \ref{aids}, below.

The rest of this section explains how to write macros that expand into
\iter\ clauses.
Here's how you could add a simplified version of \iter's |multiply|
clause, if \iter\ didn't already have one:
\begin{program}
(defmacro multiply (expr)
  `(reducing ,expr by \#'* initial-value 1))
\end{program}

If you found yourself summing the square of an expression often, you
might want to write a macro for that.  A first cut might be
\begin{program}
(defmacro sum-of-squares (expr)
  `(sum (* ,expr ,expr)))
\end{program}
but if you are an experienced macro writer, you will realize that this
code will evaluate ~expr~ twice, which is probably a bad idea.  A
better version would use a temporary:
\begin{program}
(defmacro sum-of-squares (expr)
  (let ((temp (gensym)))
    `(let ((,temp ,expr))
       (sum (* ,temp ,temp)))))
\end{program}
Although this may seem complex, it is just the sort of thing you'd
have to go through to write any macro, which illustrates the point of
this section: if you can write macros, you can extend \iter.

Our macros don't use \iter's keyword-argument
syntax.  We could just use keywords with |defmacro|, but we would
still not be using \iter's clause indexing mechanism.  Unlike Lisp,
which uses just the first symbol of a form to determine what function
to call, \iter\ individuates clauses by the list of required keywords.
For instance, |for\dots in| and |for\dots in-vector| are different clauses
implemented by distinct Lisp functions.  

To buy into this indexing scheme, as well as the keyword-argument
syntax, use |defmacro-clause|:

\begin{clauses}

\defmacro{defmacro-clause}{arglist |\&body| body}
Defines a new \iter\ clause.  ~arglist~ is a list of symbols which are
alternating keywords and arguments.  \opt\ may be used, and the list
may be terminated by |\&sequence|.  ~body~ is an ordinary macro body,
as with |defmacro|.  If the first form of ~body~ is a string, it is
considered a documentation string and will be shown by
|display-iterate-clauses|. |defmacro-clause| will signal an error if 
defining the clause would result in an ambiguity.  E.g. you cannot
define the clause |for\dots from| because there would be no way to
distinguish it from a use of the |for| clause with optional keyword |from|.

\end{clauses}

\medskip

Here is |multiply| using |defmacro-clause|.  The keywords are capitalized
for readability.
\begin{program}
(defmacro-clause (MULTIPLY expr \opt\ INTO var)
  `(reducing ,expr by \#'* into ,var initial-value 1))
\end{program}
You don't have to worry about the case when |var| is not supplied; for
any clause with an |into| keyword, saying |into nil| is equivalent to
omitting the |into| entirely.

As another, more extended example, consider the fairly common
iteration pattern that 
involves finding the sequence element that maximizes (or minimizes) some
function.  \iter\ provides this as |finding\dots maximizing|, but it's
instructive to see how to write it.
Here, in pseudocode, is how you might write such a loop for
maximizing a function F:

\begin{quote}
\begin{tabbing}
set variable MAX-VAL to NIL; \\
set variable WINNER to NIL; \\
for\=\ each element EL in the sequence \\
\>  if\=\ MAX-VAL\=\ is NIL or F(EL) $>$ MAX-VAL then \\
\>\>    set MAX-VAL to F(EL); \\
\>\>    set WINNER to EL; \\
\>  end if; \\
end for; \\
return WINNER.
\end{tabbing}
\end{quote}

Here is the macro:
\begin{program}
(defmacro-clause (FINDING expr MAXIMIZING func \opt\ INTO var)
  (let ((max-val (gensym))
        (temp1 (gensym))
        (temp2 (gensym))
        (winner (or var iterate::*result-var*)))
    `(progn 
       (with ,max-val = nil)
       (with ,winner = nil)
       (cond 
        ((null ,max-val)
         (setq ,winner ,expr)
         (setq ,max-val (funcall ,func ,winner))
        (t
         (let* ((,temp1 ,expr)
                (,temp2 (funcall ,func ,temp1)))
           (when (> ,temp2 ,max-val)
             (setq ,max-val ,temp2)
             (setq ,winner ,temp1))))))
       (finally (leave ,winner)))))
\end{program}
Note that if no |into| variable is supplied, we use
|iterate::*result-var*|, which contains the internal variable into
which all clauses place their results.  If this variable is bound by
some clause, then \iter\ will return its value automatically;
otherwise, \nil\ will be returned.  

\subsection{Writing Drivers}

In principle, drivers can be implemented just as easily as other
\iter\ clauses.  In practice, they are a little harder to get right. 
As an example, consider writing a driver that
iterates over all the 
elements of a vector, ignoring its fill-pointer.  |for\dots in-vector|
won't work for this, because it observes the fill-pointer.  It's necessary to
use |array-dimension| instead of |length| to obtain the size of the
vector.  Here is one approach:

\begin{program}
(defmacro-clause (FOR var IN-WHOLE-VECTOR v)
  "All the elements of a vector (disregards fill-pointer)"
  (let ((vect (gensym))
        (index (gensym)))
    `(progn
       (with ,vect = ,v)
       (for ,index from 0 below (array-dimension ,vect 0))
       (for ,var = (aref ,vect ,index)))))
\end{program}
Note that we immediately put |v| in a variable, in case it is an
expression.  Again, this is just good Lisp macrology.  It also has a
subtle effect on the semantics of the driver: |v| is evaluated only
once, at the beginning of the loop, so changes to |v| in the loop have
no effect on the driver.  Similarly, the bounds for numerical iteration
e.g. the above |array-dimension| are also evaluated once only.  This is how
all of \iter's drivers work.


There is an important point concerning the |progn| in this code.  We
need the |progn|, of course, because we are returning several forms,
one of which is a driver.  But \iter\ drivers must occur at top-level.
Is this code in error?  No, because ~top-level~ is defined in \iter\
to include forms inside a |progn|.  This is just the definition of
top-level that Common Lisp uses, and for the same reason: to allow
macros to return multiple forms at top-level.

While our |for\dots in-whole-vector| clause will work, it is not
ideal.  In particular, it does not support generating.  Do do so, we
need to use |for\dots next| or |for\dots do-next|.  The job is
simplified by the |defmacro-driver| macro.

\begin{clauses}

\defmacro{defmacro-driver}{arglist |\&body| body}
Defines a driver clause in
both the |for| and |generate| forms, and provides a parameter
|generate| which ~body~ can examine to determine how it was invoked.
~arglist~ is as in |defmacro-clause|, and should begin with the symbol
|for|. 

\end{clauses}

With |defmacro-driver|, our driver looks like this:
\begin{program}
(defmacro-driver (FOR var IN-WHOLE-VECTOR v)
  "All the elements of a vector (disregards fill-pointer)"
   (let ((vect (gensym))
         (end (gensym))
         (index (gensym))
         (kwd (if generate 'generate 'for)))
     `(progn
        (with ,vect = ,v)
        (with ,end = (array-dimension ,vect 0))
        (with ,index = -1)
        (,kwd ,var next (progn (incf ,index)
                               (if (>= ,index ,end) (terminate))
                               (aref ,vect ,index))))))
\end{program}


We are still missing one thing: the |\&sequence| keywords.  
We can get them easily enough, by writing 
\begin{program}
(defmacro-driver (FOR var IN-WHOLE-VECTOR v \&sequence) 
  ...)
\end{program}
We can now refer to parameters |from|, |to|, |by|, etc. which contain
either the values for the corresponding keyword, or \nil\ if the
keyword was not supplied.  Implementing the right code for these
keywords is cumbersome but not difficult; it is left as an exercise.
But before you begin, see |defclause-sequence| below for an easier way.
       
\subsection{Extensibility Aids}
\label{aids}

This section documents assorted features that may be of use in
extending \iter.

\begin{clauses}

\defunexpvar{*result-var*}
Holds the variable that is used to return a value as a result of the
\iter\ form.  You may examine this and use it in a |with| clause, but
you should not change it.

\defmacro{defsynonym}{syn word}
 Makes ~syn~ a synonym for the existing \iter\ keyword ~word.~  Only
the first word in each clause can have synonyms. 

\defmacroxx{defclause-sequence}{element-name index-name}{|\&key|
access-fn size-fn sequence-type}{element-type
element-doc-string index-doc-string}  
Provides
a simple way to define sequence clauses.  Generates two
clauses, one for iterating over the sequence's elements, the other
for iterating over its indices.  The first symbol of both
clauses will have print-name |for|.
~element-name~  and ~index-name~ should be symbols.
~element-name~ is the second keyword of the element iterator (typically of
the form 
|in-~sequence-type~|), and ~index-name~ is the second keyword
of the index-iterator (typically of the form
|index-of-~sequence-type~|).  Either name may be 
\nil, in which case the corresponding clause is not defined.  If both
symbols are supplied, they should be in the same package.  The |for|
that begins the clauses will be in this package.

\cpar ~access-fn~ is the function to be used to
access elements of the sequence in the element iterator.  The function
should take two 
arguments, a sequence and an index, and return the appropriate element.
~size-fn~ should denote a function of one argument, a sequence, that
returns its size.  Both ~access-fn~ and ~size-fn~ are required for the
element iterator, but only ~size-fn~ is needed for the index iterator.

\cpar The ~sequence-type~ and ~element-type~ keywords are used to
suggest types for the variables 
used to hold the sequence and the
sequence elements, respectively.  The usual rules about \iter's
treatment of variable type declarations apply (see section \ref{types}).

\cpar ~element-doc-string~ and ~index-doc-string~ are
the documentation strings, for use with |display-iterate-clauses|. 

\cpar The generated element-iterator performs destructuring on the
element variable.

\cpar As an example, the above |for\dots in-whole-vector| example
could have been written:
\begin{program}
(defclause-sequence IN-WHOLE-VECTOR INDEX-OF-WHOLE-VECTOR
  :access-fn 'aref
  :size-fn \#'(lambda (v) (array-dimension v 0))
  :sequence-type 'vector
  :element-type t
  :element-doc-string 
     "Elements of a vector, disregarding fill-pointer"
  :index-doc-string 
     "Indices of vector, disregarding fill-pointer")
\end{program}

\end{clauses}

\subsection{Subtleties}

There are some subtleties to be aware of when writing \iter\ clauses.
First, the code returned by your macros may be |nconc|'ed into a list,
so you should always returned freshly consed lists, rather than
constants.  Second, \iter\ matches clauses by using |eq| on the first
symbol and |string=| on the subsequent ones, so the package of the
first symbol of a clause is relevant.  All of the clauses in this manual
have their first word in the \iter\ package.  
You can use the package system in the usual way to shadow
\iter\ clauses without replacing them.

%%% say more here, about the badness that only the first word of a
%%% clause is packagey.

\section{Non-portable Extensions to Iterate (Contribs)}
\label{contribs}

Currently, there is only one non-portable extension to iterate in the
distribution: iterate-pg. If you have made an extension that depends
on non-portable features, feel free to send them to |asf@boinkor.net|
for inclusion in the iterate distribution.

\subsection{An SQL query driver for iterate}

The pg package by Eric Marsden (see |http://cliki.net/pg|) provides an
interface to the PostgreSQL database. Using the \iterpg\ extension, it
is possible to handle the results of SQL queries using \iter.

This usage example should give you an idea of how to use it:

\begin{program}
(pg:with-pg-connection (c "somedb" "someuser")
  (iter (for (impl version date) in-relation "select * from version"
                                 on-connection *dbconn*)
        (collect version)))
\end{program}

To use the extension via |ASDF|, simply make your system depend on the
|iterate-pg| system instead of the |iterate| system. To load it
manually, use:

\begin{program}
  (asdf:oos 'asdf:load-op :iterate-pg)
\end{program}

%%%%%%%%%%%%%%%%%%%%%%%%%%%%%%%%%%%%%%%%%%%%%%%%%%%%%%%%%%%%%%%%
\section{Obtaining {\tt Iterate}}

\begin{sloppypar}
The information in this chapter is obsolete but included for
completeness's sake; Currently, the most up-to-date information on
\iter\ can be found at |http://boinkor.net/iterate.html|.
\end{sloppypar}

\begin{sloppypar}
\iter\ currently runs on Lisp Machines, and on
HP's, Sun3's and Sparcstations under Lucid.  
\iter\ source and binaries are available at the MIT AI Lab in the
subdirectories of |/src/local/lisplib/|.  The source file,
|iterate.lisp|, is also available for anonymous FTP in the directory
|/com/fpt/pub/| on the machine |TRIX.AI.MIT.EDU| (Internet number
128.52.32.6).  If you are unable to obtain |iterate| in one of these
ways, send mail to |jba@ai.mit.edu| and I will send you the source
file. 
\end{sloppypar}

\begin{sloppypar}
\iter\ resides in the |iterate| package (nickname |iter|).  Just say
\linebreak |(use-package :iterate)| to make all the necessary symbols
available. 
If a symbol is not exported, it appears in this manual with an
``|iterate::|'' prefix.
\end{sloppypar}

Send bug reports to |bug-iterate@ai.mit.edu|.  The |info-iterate|
mailing list will have notices of changes and problems; to have
yourself added, send mail to |info-iterate-request@ai.mit.edu|.


\medskip

\begin{flushleft}
 \bf Acknowledgements
\end{flushleft}
\smallskip

Richard Waters provided invaluable criticism which spurred me to improve
\iter\ greatly.  As early users, David Clemens, Oren Etzioni and Jeff
Siskind helped ferret out many bugs.

%\begin{theindex}
% The index files must be generated with the genindex program
% or from makeindex iterate-manual.idx which creates iterate-manual.ind.

%\baselineskip\normalbaselineskip
%\advance\baselineskip by -2pt

\input{iterate-manual.ind}

%\underline{Clauses}
%\input{iterate-manual.clindex}
%
%\indexspace
%\underline{Lisp}
%\input{iterate-manual.lispindex}

%\end{theindex}


\end{document}


% arch-tag: "c0f871a6-313c-11d8-abb9-000c76244c24"

%
%\indexspace
%\underline{Lisp}
%% Fix ``.'' spacing in lists
\documentstyle[12pt,aimemo]{article}

\setlength{\textwidth}{6.0in}
\setlength{\oddsidemargin}{0.25in}
\setlength{\evensidemargin}{0.25in}
\setlength{\topmargin}{0pt}
\setlength{\headsep}{24pt}
\setlength{\textheight}{8.5in}

\let\origpar=\par

% somehow, LaTex fucks with things so that the plain Tex \obeyspaces
% doesn't work.  This is the same, except we use an mbox containing
% a space instead of just a space.

\newcommand{\spc}{\mbox{ }}
\newcommand{\tAb}{\mbox{        }}
\def\obspaces{\catcode`\ =\active\catcode`\^^I=\active}
{\obspaces \global\let =\spc\global\let^^I=\tAb}

\newenvironment{program}{\medskip\samepage\tt\obeylines\obspaces}{\medskip}

% use | for \tt
\catcode`\|=13
\newif\ifvbar
\def|{\ifvbar
         \endgroup\vbarfalse
      \else 
         \begingroup\tt\vbartrue 
      \fi}



\newcommand{\lisp}{\tt}
\newcommand{\iter}{|iterate|}
\newcommand{\iterpg}{|iterate-pg|}
\newcommand{\setf}{|setf|}
\newcommand{\nil}{|nil|}
\newcommand{\nonnil}{non-\nil}
\newcommand{\opt}{{\lisp \&optional}}
\newcommand{\yields}{$\Rightarrow$}



\newbox\Dots
\setbox\Dots=\hbox{\small ...}
\renewcommand{\dots}{\copy\Dots}

% Here's a way to define an environment which will treat all
% paragraphs inside it in the same manner.  It's, as near as I can
% figure, the way that Latex does things.  Using \everypar won't work;
% apparently Latex resets that a lot.  What you have to do is: at the
% top of the file do \let\origpar=\par.  In the
% begin part of the environment, do the parshape (or whatever), then say
% \def\par{{\origpar}}.  The extra braces are necessary.   That's all
% you have to do; at the end of the environment things will return to
% normal automatically.  Why does this all work?  I have no idea.
% Another important thing: there should be a blank line (i.e. a \par)
% between the end of the last paragraph and the \end{environment}.

\newlength{\clindent}
\setlength{\clindent}{0.5in}
\newlength{\clparindent}
\setlength{\clparindent}{\parindent}

\newenvironment{clauses}{\advance\linewidth -\parindent
                        \hangindent\clindent\relax
                         \hangafter=0
                          \parindent=0pt
                          \def\par{{\origpar}}}{}

\newcommand{\cpar}{\hspace \clparindent}

\newcommand{\startitem}{\bigskip\pagebreak[3]}
\newcommand{\finishitem}{\smallskip}

\newcommand{\clause}[1]{\startitem\Clause{#1}\finishitem}
\newcommand{\Clause}[1]{\hbox{#1}}
\newcommand{\clindex}[1]{\index{{\lisp #1}}}
\newcommand{\clindexx}[2]{\index{{\lisp #1\dots #2}}}
\newcommand{\clindexxx}[3]{\index{{\lisp #1\dots #2\dots #3}}}
\newcommand{\clausex}[2]{\startitem\Clausex{#1}{#2}\finishitem}
\newcommand{\Clausex}[2]{\Clause{|#1| #2}\clindex{#1}}

\newcommand{\defvar}[1]{\startitem\deff{#1}{}{Variable}\finishitem}
\newcommand{\defunexpvar}[1]{\startitem\deffnoindex{iterate::#1}{}{Variable}
                                        \lispindex{#1}\finishitem}
\newcommand{\defun}[2]{\startitem\deff{#1}{#2}{Function}\finishitem}
\newcommand{\defmacro}[2]{\startitem\deff{#1}{#2}{Macro}\finishitem}
\newcommand{\defunx}[3]{\startitem\deffx{#1}{#2}{#3}{Function}\finishitem}
\newcommand{\defmacrox}[3]{\startitem\deffx{#1}{#2}{#3}{Macro}\finishitem}
\newcommand{\defunxx}[4]{\startitem\deffxx{#1}{#2}{#3}{#4}{Function}\finishitem}
\newcommand{\defmacroxx}[4]{\startitem\deffxx{#1}{#2}{#3}{#4}{Macro}\finishitem}

\newcommand{\deff}[3]{\deffnoindex{#1}{#2}{#3}\lispindex{#1}}
\newcommand{\deffnoindex}[3]{\hbox to \hsize{{\tt #1} {\it #2}\hfill [{\it #3}]}}
\newcommand{\deffx}[4]{\deff{#1}{#2}{#4}\moreargs{#1}{#3}}
\newcommand{\deffxx}[5]{\deffx{#1}{#2}{#3}{#5}\moreargs{#1}{#4}}

\newcommand{\moreargs}[2]{\hbox to \hsize{\phantom{\lisp #1} {\it #2}\hfill}}

\newcommand{\lispindex}[1]{\index{{\lisp #1}}}

 \newenvironment{note}[1]{\pagebreak[2]\bigskip
                          \hrule\smallskip\small
                         {\setlength{\parindent}{0pt}
                           \par{\bf #1:}}}{\smallskip\hrule\bigskip}



\makeindex

%%%%%%%%%%%%%%%%%%%%%%%%%%%%%%%%%%%%%%%%%%%%%%%%%%%%%%%%%%%%%%%%

%\thispagestyle{empty}

%\vspace*{2in}

%\title{The |Iterate| Manual}
%\author{Jonathan Amsterdam}

%\maketitle

%\thispagestyle{empty}

%\begin{center}
%\Large *** DRAFT ***
%\end{center}
%This manual is in beta-test. It will become an AI Memo in a couple of
%months.  I would appreciate any comments or
%corrections.  You can reach me at email address |jba@ai.ai.mit.edu|.

\begin{document}

\def\writtenby{Jonathan Amsterdam}
\def\contractno{This report describes research done at the Artificial
Intelligence Laboratory of the Massachusetts Institute of Technology.
Support for the laboratory's artificial intelligence research is
provided in part by the Advanced Research Projects Agency of the
Department of Defense under Office
of Naval Research contract N00014-85-K-0124.}

\AIMEMO{1236}{\Large The Iterate Manual \\[1in]}
{This is the manual for version 1.4 of \iter, a powerful iteration
macro for Common  
Lisp. \iter\ is similar to {\tt loop} but provides numerous additional
features, is well integrated with Lisp, and is extensible. 
\vfill }


% use ~ for \em
\catcode`\~=13
\newif\ifem
\def~{\ifem
         \/\endgroup\emfalse
      \else 
         \begingroup\em\emtrue 
      \fi}



%\pagebreak
%\setcounter{page}{1}

%\thispagestyle{empty}


\tableofcontents

\pagebreak
\pagestyle{headings}

\section{Introduction}

\begin{sloppypar}
This manual describes
\iter, a powerful iteration facility for Common Lisp.
\iter\ provides abstractions for many common iteration
patterns and allows for the definition of additional patterns.  \iter\ is
a macro that expands into ordinary Lisp at
compile-time, so it is more efficient than higher-order
functions like |map| and |reduce.|  
While it is similar to |loop|, \iter\ offers a more Lisp-like syntax
and enhanced extensibility.
(For a more complete
comparison of \iter\ with other iteration constructs, see MIT AI Lab Working
Paper No. 324, ~Don't Loop, Iterate.~)
\end{sloppypar}

An \iter\ form consists of the symbol |iter|\footnote{You can also use
|iterate|, but |iter| is preferred because it avoids potential conflicts with
possible future additions to Common Lisp, and because it saves
horizontal space when writing code.}
followed by one or more
forms, some of which may be \iter\ 
~clauses.~  Here is a simple example of \iter\ which collects the
numbers from 1 to 10 into a list, and returns the list.  The return
value is shown following the arrow.
\begin{program}
(iter (for i from 1 to 10)
      (collect i))          \yields (1 2 3 4 5 6 7 8 9 10)
\end{program}
This form contains two clauses: a |for| clause that steps the
variable |i| over the integers from 1 to 10, and a |collect|
clause that accumulates its argument into a list.  With a few
exceptions, all \iter\ clauses have the same format: alternating
symbols (called ~keywords~) and expressions (called
~arguments~).  The syntax and terminology are those of Common
Lisp's keyword lambda lists.  One difference is that \iter's keywords
do not have to begin with a colon---though they may, except for the
first symbol of a clause.  So you can also write  |(for i :from
1 :to 10)| if you prefer.

Any Lisp form can appear in the body of an \iter, where it will have
its usual meaning.  \iter\ walks the entire body, expanding macros,
and recognizing clauses at any level.  This example collects all the
odd numbers in a list:
\begin{program}
(iter (for el in list)
      (if (and (numberp el) (oddp el))
          (collect el)))
\end{program}

There are clauses for iterating over numbers, lists, arrays and other
objects, and for
collecting, summing, counting, maximizing and other useful operations.
\iter\ also supports the creation of new variable bindings, stepping
over multiple sequences at once, destructuring, and compiler
declarations of variable types.  The following example illustrates
some of these features: 
\begin{program}
(iter (for (key . item) in alist)
         (for i from 0)
         (declare (fixnum i))
         (collect (cons i key)))
\end{program}

This loop takes the keys of an alist and returns a new alist
associating the keys with their positions in the original list. The
compiler declaration for |i| will appear in the generated code
in the appropriate place.

%%%%%%%%%%%%%%%%%%%%%%%%%%%%%%%%%%%%%%%%%%%%%%%%%%%%%%%%%%%%%%%%
\section{Clauses}

Most of \iter's clauses will be familiar to |loop| programmers.  
(|loop| is an iteration macro that has been incorporated into Common
Lisp.  Seen Guy Steele's {\em Common Lisp, 2nd Edition}.)
In
nearly all cases they behave the same as their |loop| counterparts, so
a |loop| user can switch to \iter\ with little pain (and much gain).

All clauses with the standard keyword-argument syntax consist of two
parts: a ~required~ part, 
containing keywords that must be present and in the right order; and
an ~optional~ part, containing keywords that may be omitted and,
if present, may occur in any order.  In the descriptions below, the
parts are separated by the Lisp lambda-list keyword \opt.


\subsection{Drivers}

An iteration-driving clause
conceptually causes the iteration to go forward.  Driver clauses in
\iter\ allow iteration over numbers, lists, vectors, hashtables, packages,
files and streams.  Iteration-driving clauses must
appear at the top level of an \iter\ form; they cannot be nested
inside another clause.  The driver variable is updated at the point
where the driver clause occurs.  Before the clause is executed for the
first time, the value of the variable is undefined.

%Also, regardless of where the driver clause
%appears in the body, the driver variable is stepped at the top of the
%loop; hence it is stylistically preferable, though
%not required, to place driver clauses at the beginning of the \iter.

Multiple drivers may appear in a single \iter\ form, in which case all
of the driver variables are updated each time through the loop, in the
order in which the clauses appear.  The first driver to terminate will
terminate the entire loop.

In all cases, the value of the driver variable on exit from the loop,
including within the epilogue code (see the |finally| clause), is
undefined.

All the parameters of a driver clause are evaluated once, before the
loop begins.  Hence it is not possible to change the bounds or other
properties of an iteration by side-effect from within the loop.

With one exception, driver clauses begin with the word |for| (or
the synonym |as|) and mention an iteration variable, which is
given a binding within the \iter\ form.
The exception is |repeat|, which just executes a loop a
specified number of times:

\begin{clauses}

\clausex{repeat}{~n~}
Repeats the loop ~n~ times.  For example:
\begin{program}
(iter (repeat 100)
      (print "I will not talk in class."))
\end{program}
\cpar If $n \leq 0$, the
loop will never be executed.  If ~n~ is not an integer, the
actual number of executions will be $\lceil n \rceil$.

\end{clauses}

\pagebreak[3]

\subsubsection{Numerical Iteration}

\begin{clauses}
\clausex{for}{~var~ |\&sequence|} 
The general form for iterating over a sequence of numbers requires a
variable and, optionally, one or more keywords that provide the bounds
and step size of the iteration. The |\&sequence|
lambda-list keyword is a shorthand for these sequence keywords.  They are:
|from|, |upfrom|, |downfrom|, |to|, |downto|, |above|, |below| and |by|.
|from| provides the starting value for ~var~
and defaults to zero. |to| provides a final value and implies that the
successive values of ~var~ will be increasing; |downto| implies that
they will be decreasing.  The loop terminates when ~var~ passes
the final value (i.e. becomes smaller or larger than it, depending on
the direction of iteration); in other  words, the loop body
will never be 
executed for values of ~var~ past the final value.
|below| and |above| are similar to |to| and
|downto|, except that the loop terminates when ~var~ equals or passes
the final value.

\cpar If no final value is specified, the variable will be stepped
forever.  Using |from| or |upfrom| will result in increasing
values, while |downfrom| will give decreasing values.

\cpar On each iteration,
~var~ is incremented or decremented by the value of the sequence
keyword |by|, which defaults to 1.  It should always be a positive
number, even for downward iterations.

\cpar In the following examples, the sequence of numbers generated is shown
next to the clause.

\begin{program}
(for i upfrom 0) \yields\ 0 1 2 \ldots
(for i from 5)   \yields\ 5 6 7 \ldots    ; either from or upfrom is okay
(for i downfrom 0) \yields\ 0 -1 -2 \ldots
(for i from 1 to 3) \yields\ 1 2 3
(for i from 1 below 3) \yields\ 1 2
(for i from 1 to 3 by 2) \yields\ 1 3
(for i from 1 below 3 by 2) \yields\ 1
(for i from 5 downto 3) \yields\ 5 4 3
\end{program}

\end{clauses}

\subsubsection{Sequence Iteration}

There are a number of clauses for iterating over sequences.  In all of
them, the argument following |for| may be a list instead of a
symbol, in which case destructuring is performed.  See section
\ref{destructuring}. 

\begin{clauses}

\clause{|for| ~var~ |in| ~list~ \opt\ |by| ~step-function~}
\clindexx{for}{in}
~var~ is set to successive elements of list.  
~step-function,~ which 
defaults to |cdr|, is used to obtain the next sublist. 

\clause{|for| ~var~ |on| ~list~ \opt\ |by| ~step-function~}
\clindexx{for}{on}
~var~ is set to successive sublists of list. 
~step-function~ (default |cdr|) is used as in |for\dots in|.

\end{clauses}

\medskip

These two clauses use |atom| to test for the end
of a list.  Hence, given a list whose final |cdr| is not \nil,
they will silently ignore the last |cdr|.  Other choices are
|endp|,  which would signal an error, and |null|, which
would probably result in an error somewhere else.  If you wish to use
an end-test other than |atom|, set the variable
|iterate::*list-end-test*|\lispindex{*list-end-test*} to the name of
the desired function. 

\begin{clauses}
\clause{|for| ~var~ |in-vector| ~vector~ |\&sequence|}
\clindexx{for}{in-vector}
~var~ takes on successive elements from ~vector.~   The vector's
fill-pointer is observed.  Here and in subsequent clauses, the |\&sequence|
keywords include |with-index|, which takes a symbol as argument
and uses it for the index variable instead of an internally generated
symbol.  The other |\&sequence| keywords behave as in numerical
iteration, except that the default iteration bounds are the bounds of
the vector.  E.g. in |(for i in-vector v downto 3)|,
|i| will start off being bound to the last element in |v|,
and will  
be set to preceding elements down to and including the element with
index 3.

\clause{|for| ~var~ |in-sequence| ~seq~ |\&sequence|}
\clindexx{for}{in-sequence}
This uses Common Lisp's generalized sequence functions, |elt|
and |length|, to obtain elements and determine the length of
~seq.~   Hence it will work for any sequence, including lists, and will
observe the fill-pointers of vectors.

\clause{|for| ~var~ |in-string| ~string~ |\&sequence|}
\clindexx{for}{in-string}
~var~ is set to successive characters of ~string.~

\startitem
\Clause{|for| ~var~ |index-of-vector| ~vector~ |\&sequence|}
\clindexx{for}{index-of-vector}
\Clause{|for| ~var~ |index-of-sequence| ~sequence~ |\&sequence|}
\clindexx{for}{index-of-sequence}
\Clause{|for| ~var~ |index-of-string| ~string~ |\&sequence|}
\clindexx{for}{index-of-string}
\finishitem
~var~ is set to successive indices of the sequence.
These clauses avoid the overhead of accessing the sequence elements
for those applications where they do not need to be examined, or are
examined rarely.  They admit all the optional keywords of the other
sequence drivers except the (redundant) |with-index| keyword.

\clause{|for| ~(key value)~ |in-hashtable| ~table~}
\clindexx{for}{in-hashtable}

~key~ and ~value,~ which must appear as shown in a list
and may be destructuring templates, are set to the keys and values of
~table~.  If ~key~ is |nil|, then the hashtable's keys will be ignored;
similarly for ~value~.
The order in which 
elements of ~table~ will be retrieved is unpredictable.

\clause{|for| ~var~ |in-package| ~package~ \opt\ |external-only| ~ext~}
\clindexx{for}{in-package}
Iterates over all the symbols in ~package,~ or over only the
external symbols if ~ext~ is specified and non-|nil|.  ~ext~ is not
evaluated.  The same symbol may appear more than once.

\clause{|for| ~var~ |in-packages| ~packages~ \opt\ |having-access| ~symbol-types~}
\clindexx{for}{in-packages}
Iterates over all the symbols from the list of packages denoted by the
descriptor ~packages~ and having accessibility (or visibility) given by
~symbol-types~.  This defaults to the list |(:external :internal :inherited)|
and is not evaluated.  ~var~ must be a list of up to three variables: in each
iteration, these will be set to a symbol, its access-type and package (as per
|with-package-iterator| in ANSI-CL).  The same symbol may appear more than
once.

\end{clauses}

\begin{clauses}

\clause{|for| ~var~ |in-file| ~name~ \opt\ |using| ~reader~}
\clindexx{for}{in-file}
Opens the file ~name~ (which may be a string or pathname) for input,
and iterates 
over its contents. ~reader~ defaults to |read|, so by default ~var~
will be bound to the successive forms in the file.  The \iter\ body is
wrapped in an unwind-protect to ensure that the file is closed no
matter how the \iter\ is exited.

\clause{|for| ~var~ |in-stream| ~stream~ \opt\ |using| ~reader~}
\clindexx{for}{in-stream}
Like |for\dots in-file|, except that ~stream~ should be an existing
stream object that supports input operations.

\end{clauses}

\subsubsection{Generalized Drivers}

These are primarily useful for writing drivers that can also be used
as generators (see section \ref{generators}, below).
 
\begin{clauses}

\clause{|for| ~var~ |next| ~expr~}
\clindexx{for}{next}

~var~ is set to ~expr~ each time through the loop.  Destructuring is
performed.  When the clause is used as a generator, ~expr~ is the code
that is executed when |(next ~var~)| is 
encountered (see section \ref{generators}, below).
~expr~ should compute the first value for ~var~, as well
as all subsequent values, and is responsible for terminating the loop.
For compatibility with future versions of \iter, this termination
should be done with |terminate|\clindex{terminate}, which can be
considered a synonym for |finish| (see section \ref{control-flow}).

\cpar As an example, the following clauses are equivalent to |(for
i from 1 to 10)|:
\begin{program}
(initially (setq i 0))
(for i next (if (> i 10) (terminate) (incf i)))
\end{program}

\clause{|for| ~var~ |do-next| ~form~}
\clindexx{for}{do-next}
~form~ is evaluated each time through the loop.  Its value is ~not~
set to ~var~; that is ~form's~ job. ~var~ is only present so that
\iter\ knows it is a driver variable.  \linebreak |(for ~var~ next
~expr~)| is equivalent to |(for ~var~ do-next (dsetq ~var~ ~expr~))|.
(See section \ref{destructuring} for an explanation of |dsetq|.) 

\end{clauses}

\subsubsection{Generators}
\label{generators}

In all of the above clauses, the driver variable is updated on each
iteration.
Sometimes it is desirable to have greater control over updating.  
For instance, consider the problem of associating numbers, in
increasing order and with no gaps, with the
\nonnil\ elements of a list.  One obvious first pass at writing this is:

\begin{program}
(iter (for el in list)
      (for i upfrom 1)
      (if el (collect (cons el i))))
\end{program}
But on the list |(a b nil c)| this produces |((a . 1) (b . 2) (c .
4))| instead of the desired |((a . 1) (b . 2) (c . 3))|.  The problem
is that |i| is incremented each time through the loop, even when |el|
is |nil|.  

The problem could be solved elegantly if we could step |i| only when
we wished to.  This
can be accomplished for any \iter\ driver by writing |generate| (or
its synonym |generating|)
instead of |for|.  Doing so produces a ~generator~---a driver whose
values are yielded explicitly.  To obtain the next value of a
generator variable ~v~, write \linebreak |(next ~v~)|.  The value of a |next|
form is the next value of ~v~, as determined by its associated driver
clause.  |next| also has the side-effect of updating ~v~ to that
value.  If there is no next value, |next| will terminate the loop,
just as with a normal driver.


Using generators, we can now write our example like this:
\begin{program}
(iter (for el in list)
      (generate i upfrom 1)
      (if el (collect (cons el (next i)))))
\end{program}
Now |i| is updated only when |(next i)| is executed, and this occurs
only when |el| is \nonnil.

To better understand the relationship between ordinary drivers and
generators, observe that we can rewrite an ordinary driver using its
generator form immediately followed by |next|, as this example shows:
\begin{program}
(iter (generating i from 1 to 10)
      (next i)
      ...)
\end{program}
Provided that the loop body contains no |(next i)| forms, this will
behave just as if we had written |(for i from 1 to 10)|.

We can still refer to a driver variable ~v~ without using |next|; in
this case, its value is that given to it by the last evaluation of
|(next ~v~)|.  Before |(next ~v~)| has been called the first time, the
value of ~v~ is undefined.  

This semantics is more flexible than
one in which ~v~ begins the loop bound to its first value and calls
of |next| supply subsequent values, because it means the loop will not
terminate too soon if the generator's sequence is empty.  For
instance, consider the following code, which tags \nonnil\ elements of
a list using a list of tags, and also counts the null elements.
(We assume there are at least as many tags as \nonnil\ elements.)
\begin{program}
(let* ((counter 0)
       (tagged-list (iter (for el in list)
                          (generating tag in tag-list) 
                          (if (null el)
                              (incf counter)
                              (collect (cons el (next tag)))))))
  ...)
\end{program}
It may be that there are just as many tags as non-null elements of
|list|.  If all the elements of |list| are null, we still want the
counting to proceed, even though |tag-list| is \nil.  If |tag| had to be
assigned its first value before the loop begins, we would have had to
terminate the loop before the first iteration, since when |tag-list|
is \nil, |tag| 
has no first value.  With the existing semantics, however, |(next
tag)| will never execute, so the iteration will cover all the elements of
|list|. 
        
When the ``variable'' of a driver clause is actually a destructuring
template containing several variables, all the variables are eligible
for use with |next|.  As before, |(next ~v~)| evaluates to ~v's~ next
value; but the effect is to update all of the template's variables.
For instance, the following code will return the list |(a 2 c)|.
\begin{program}
(iter (generating (key . item) in '((a . 1) (b . 2) (c . 3)))
      (collect (next key))
      (collect (next item)))
\end{program}

Only driver clauses with variables can be made into generators.  This
includes all clauses mentioned so far except for |repeat|.  It does
~not~ include |for\dots previous|, |for\dots =|,
|for\dots initially\dots then| or |for\dots first\dots then| (see
below). 

\subsubsection{Previous Values of Driver Variables}

Often one would like to access the value of a variable on a previous
iteration.  \iter\ provides a special clause for accomplishing this.

\begin{clauses}

\clause{|for| ~pvar~ |previous| ~var~ \opt\ |initially| ~init~
|back| ~n~}
\clindexx{for}{previous}
Sets ~pvar~ to the previous value of ~var,~ which should be a driver
variable, a variable from another |for\dots previous| clause, or a
variable established by a |for\dots =|, 
|for\dots initially\dots then| or |for\dots first\dots then| clause
(see section \ref{setting}).
Initially, ~pvar~ is given the value ~init~ (which defaults to \nil).  
The ~init~ expression will be moved outside the loop body, so it
should not depend on anything computed within the loop.
~pvar~ retains the value of ~init~ until ~var~ is set to its second
value, at which point ~pvar~ is set to ~var's~ first value; and so on.  

\cpar The
argument ~n~ to |back| must be a constant, positive integer, and
defaults to 1.  It determines how many iterations back ~pvar~ should
track ~var~.  For example, when ~n~ is 2, then ~pvar~ will be assigned
~var's~ first value when ~var~ is set to its third value.

\cpar A |for\dots previous| clause may occur after or before its
associated driver clause. |for\dots previous| works with generators as
well as ordinary drivers.

\pagebreak[3]

\cpar Example:
\begin{program}
(iter (for el in '(1 2 3 4))
      (for p-el previous el)
      (for pp-el previous p-el initially 0)
      (collect pp-el))
\end{program}
This evaluates to |(0 0 1 2)|.  It could have been written more
economically as 
\begin{program}
(iter (for el in '(1 2 3 4))
      (for pp-el previous el back 2 initially 0)
      (collect pp-el))
\end{program}


\end{clauses}

\subsection{Variable Binding and Setting}
\label{setting}

Several clauses exist for establishing new variable bindings or for
setting variables in the loop.  They all support destructuring.

\begin{clauses}

\clausex{with}{~var~ \opt\ |=| ~value~}
Causes ~var~ to be bound to value before the loop body is entered.
If ~value~ is not supplied, ~var~ assumes a default
binding, which will be 
\nil\ in the absence of declarations.  Also, if ~value~ is not
supplied, no destructuring is performed; instead, ~var~ may be a list of
symbols, all of which are given default 
bindings.  If ~value~ is supplied, ~var~ is bound to it, with
destructuring.  

\cpar Because |with|
creates bindings whose scope includes the entire \iter\ form, it is
good style to put all |with| clauses at the beginning.

\cpar Successive occurrences of |with| result in sequential
bindings (as with 
|let*|).  There is no way to obtain parallel bindings; see
section \ref{bindings} for a rationale.  


\clause{|for| ~var~ |=| ~expr~}
\clindexx{for}{=}
On each iteration, ~expr~ is evaluated and ~var~ is set
to its value. 

\cpar This clause may appear to do the same thing as |for\dots next|.
In fact, they are quite different.  |for\dots =| provides only three
services: it sets up a binding for ~var~, sets it to ~expr~ on each
iteration, and makes it possible to use |for\dots previous| with
~var~.  |for\dots next| provides these services in addition to the
ability to turn the driver into a generator.
%Also, the code which sets ~var~ appears in the loop body in the same
%place as the |for\dots =| clause; the code for |for\dots next| appears
%at the top of the loop, as with other drivers (except when being used
%as a generator).

\clause{|for| ~var~ |initially| ~init-expr~ |then| ~then-expr~}
\clindexxx{for}{initially}{then}
Before the loop begins, ~var~ is set to ~init-expr;~ on all
iterations after the first it is set to ~then-expr.~  This clause
must occur at top-level.  ~init-expr~ will be moved outside the loop
body and ~then-expr~ will be moved to the end of the loop body, so
they are subject to code motion problems (see section
\ref{code-movement}). 

\cpar This clause may appear to be similar to |for\dots next|, but in
fact they differ significantly.  |for\dots initially\dots then| is
typically used to give ~var~ its first value before the loop begins,
and subsequent values on following iterations.  This is incompatible
with generators, whose first value and subsequent values must all be
computed by |(next ~var~)|.  Also, the update of ~var~ in 
|for\dots initially\dots then| does not occur at the location of the
clause. 
Use |for\dots initially\dots then| for
one-shot computations where its idiom is more convenient, but use
|for\dots next| for extending \iter\ with new drivers (see section
\ref{extend}).

\clause{|for| ~var~ |first| ~first-expr~ |then| ~then-expr~}
\clindexxx{for}{first}{then}
The first time through the loop, ~var~ is set to ~first-expr;~ on
subsequent iterations, it is set to ~then-expr.~  This differs from
|for\dots initially| in that ~var~ is set to ~first-expr~
inside the loop body, 
so ~first-expr~ may depend on the results of other clauses.  For
instance, 
\begin{program}
(iter (for num in list)
      (for i first num then (1+ i))
      ...)
\end{program}
will set |i| to the first element of |list| on the first
iteration, whereas 
\begin{program}
(iter (for num in list)
      (for i initially num then (1+ i))
      ...)
\end{program}
is probably erroneous; |i| will be bound to |num|'s
default binding (usually \nil) for the first iteration.

\end{clauses}

\begin{note}{Compatibility note}
|loop|'s |for\dots =| works like \iter's, but |loop| used the syntax
|for\dots =\dots then| to mean |for\dots initially\dots then|.  It was
felt that these two operations were sufficiently different to warrant
different keywords.

Also, the |for| in the above three clauses is misleading,
since none is true driver (e.g. none has a corresponding |generate|
form).  |setting| would have been a better choice, but |for| was used
to retain some compatibility with |loop|.  
\end{note}


%%%%%%%%%%%%%%%%%%%%%%%%%%%%%%%%
\subsection{Gathering Clauses}
Many of \iter's clauses accumulate values into a variable, or set a
variable under certain conditions.  At the end of the loop, this
variable contains the desired result.  All these clauses have an
optional |into| keyword, whose argument should be a symbol.  If the
|into| keyword is not supplied, the accumulation variable will be
internally generated and its value will be returned at the end of the
loop; if a variable is specified, that variable is used for the
accumulation, and is not returned as a result---it is up to the user
to return it explicitly, in the loop's epilogue code (see |finally|).
It is safe to examine the accumulation variable during the loop, but
it should not be modified.

These clauses all begin with a verb.  When the verb does
not conflict with an existing Common Lisp function, then it may be
used in either its infinitival or present-participle form (e.g. |sum|,
|summing|).  However, when there is a conflict with Common Lisp, only
the present-participle form may be used (e.g. |unioning|).  This is to
prevent \iter\ clauses from clashing with Common Lisp
functions. 

%although these clauses are described as ``producing a value,'' it is a
%mistake to think of the lisp list representing the clause as a
%value-producing form in the usual way.  clauses may legally be written
%where a value is expected, e.g. |(setq x (sum i))|, but the
%lisp value of a clause in such a context is undefined.

\subsubsection{Reductions}

~Reduction~ is an extremely common iteration pattern in which
the results of successive applications of a
binary operation are accumulated.
For example, a loop that computes the sum of the
elements of a list is performing a reduction with the addition
operation.  This could be written in Common Lisp as  |(reduce \#'+
list)| or with \iter\ as
\begin{program}
(iter (for el in list)
      (sum el))
\end{program}

\begin{clauses}

\clausex{sum}{~expr~ \opt\ |into| ~var~}
Each time through the loop, ~expr~ is evaluated and added to a
variable, which is bound initially to zero.  If ~expr~ has a type,
it is ~not~ used as the type of the sum variable, which is always
|number|.  To get the result variable to be of a more specific
type, use an explicit variable, as in
\begin{program}
(iter (for el in number-list)
      (sum el into x)
      (declare (fixnum x))
      (finally (return x)))
\end{program}

\clausex{multiply}{~expr~ \opt\ |into| ~var~}
Like |sum|, but the initial value of the result variable is $1$,
and the variable is updated by multiplying ~expr~ into it.

\clausex{counting}{~expr~ \opt\ |into| ~var~}
~expr~ is evaluated on each iteration.  If it is \nonnil, the
accumulation variable, initially zero, is incremented.

\startitem
\Clausex{maximize}{~expr~ \opt\ |into| ~var~}
\Clausex{minimize}{~expr~ \opt\ |into| ~var~}
\finishitem
~expr~ is evaluated on each iteration and its extremum (maximum or
minimum) is stored in the accumulation variable.  If ~expr~ is never
evaluated, then the result is \nil\ (if the accumulation variable is
untyped) or $0$ (if it has a numeric type).

\clausex{reducing}{~expr~ |by| ~func~ \opt\ |initial-value| ~init-val~ |into| ~var~}
This is a general way to perform reductions. ~func~ should be a
function of two arguments, the first of which will be the value
computed so far and
the second of which will be the value of ~expr~.  It should return the
new value.
|reducing| is roughly equivalent to the
Common Lisp |(reduce ~func~ ~sequence~ :key ~expr-function~)|, where
~expr-function~ is used to derive values from the successive elements
of ~sequence~.

\cpar If the |reducing| clause is never executed, the result is
undefined.

\cpar It is not necessary to provide an
initial value, but better code can be generated if one is supplied.
Regardless of its location in the \iter\ body, ~init-val~ will be
evaluated before the loop is entered, so it should not depend on any
value computed inside the \iter\ form.  

%\cpar if a ~var~ is not specified, you can get \iter\ to declare the
%type of the internal variable by putting a |the| expression around
%~func.~  see section \ref{types}.

\end{clauses}

%\begin{note}{implementation note}
%in principle, |maximize| and |minimize| can be thought of
%as reductions where the initial value is the smallest (or largest)
%value that the accumulation variable can assume.  because lisp's
%bignums can represent arbitrary integers, these clauses cannot be
%implemented as reductions in general.  if, however, the type of 
%~expr~ or ~var~ can be determined to be a fixnum
%or a float, \iter\ will implement the clause as a true reduction,
%using one of the constants |most-negative-fixnum|, |%most-positive-fixnum|, 
%|most-negative-short-float|, etc. as appropriate.
%\end{note}

\subsubsection{Accumulations}

All the predefined accumulation clauses add values to a sequence.  If
the sequence is a list, they all 
have the property that the partial list is kept in the correct order
and available for inspection at any point in the loop.

\begin{clauses}

\clausex{collect}{~expr~ \opt\ |into| ~var~ |at| ~place~ |result-type| ~type~}
Produces a sequence of the values of ~expr~ on each iteration. ~place~
indicates where the next value of ~expr~ is added to the list and may
be one of the symbols |start|, |beginning| (a synonym for |start|) or
|end|.  The symbol may be quoted, but need not be.  The default is
|end|.  For example,
\begin{program}
(iter (for i from 1 to 5)
      (collect i))
\end{program}
produces |(1 2 3 4 5)|, whereas
\begin{program}
(iter (for i from 1 to 5)
      (collect i at beginning))
\end{program}
produces |(5 4 3 2 1)| (and is likely to be faster in most Common Lisp
implementations).

\cpar If ~type~ is provided, it should be a subtype of |sequence|.
The default is |list|.  Specifying a type other than |list| will
result in |collect| returning a sequence of that type.  ~However,~ the
type of the sequence being constructed when inside the loop body is
undefined when a non-|list| type is specified.  (As with ~place~,
quoting ~type~ is optional.)

\clausex{adjoining}{~expr~ \opt\ |into| ~var~ |test| ~test~ |at| ~place~ |result-type| ~type~}
Like |collect|, but only adds the value of ~expr~ if it is not
already present.  ~test,~ which defaults to |\#'eql|, is
the test to be used with |member|.

\startitem
\Clausex{appending}{~expr~ \opt\ |into| ~var~ |at| ~place~}
\Clausex{nconcing}{\ ~expr~ \opt\ |into| ~var~ |at| ~place~}
\Clausex{unioning}{\ ~expr~ \opt\ |into| ~var~ |test| ~test~ |at| ~place~}
\Clausex{nunioning}{~expr~ \opt\ |into| ~var~ |test| ~test~ |at| ~place~}
\finishitem
These are like |collect|, but behave like the Common Lisp functions
|append|, |nconc|, |union| or |nunion|.  
As in Common Lisp, they work only on lists.  Also as in Common Lisp,
|unioning| and |nunioning| assume that the value of ~expr~ contains no
duplicates. 

\clausex{accumulate}{~expr~ |by| ~func~ \opt\ |initial-value| ~init-val~ |into| ~var~}
This is a general-purpose accumulation clause. ~func~ should be a
function of two arguments, the value of ~expr~ and the value
accumulated so far in the iteration, and it should return the updated
value.  If no initial value is supplied, \nil\ is used.  

%\cpar If a ~var~ is not specified, you can get \iter\ to declare the
%type of the internal variable by putting a |the| expression around
%~func.~  see section \ref{types}.

\cpar The differences between |accumulate| and |reducing| are slight.
One difference is that the functions take their arguments in a
different order.  Another is that in the absence of ~init-val~,
|accumulate| will use \nil, whereas |reducing| will generate different
code that avoids any dependence on the initial value.
The reason for having both clauses is that one usually
thinks of reductions (like |sum|) and accumulations (like |collect|)
as different beasts.

\end{clauses}


\subsubsection{Finders}

A ~finder~ is a clause whose value is an expression that meets some
condition.

\begin{clauses}

\clause{|finding| ~expr~ |such-that| ~test~ \opt\ |into| ~var~ |on-failure| ~failure-value~}
\clindexx{finding}{such-that}
If ~test~ (which is an expression) ever evaluates to \nonnil, the loop
is terminated, the 
epilogue code is run and the value of ~expr~ is returned.  Otherwise,
\nil\ (or ~failure-value,~ if provided) is returned.  If ~var~ is
provided, it will have either the \nonnil\ value of ~expr~ or
~failure-value~ when the epilogue code is run.

\cpar As a special case, if the ~test~ expression is a sharp-quoted
function, then it is applied to ~expr~ instead of being simply
evaluated.  E.g. |(finding x such-that \#'evenp)| is equivalent to
|(finding x such-that (evenp x))|.  

%\cpar although ~test~ need have nothing to do with ~%expr~ as in
%|(finding j such-that (> i 3))|, it usually 
%will: |(finding (length el) such-that (oddp (length el)))|.  to
%avoid performing the |length| computation twice, you could write
%|(finding (length el) such-that \#'oddp)| or |(finding (length
%el) such-that 'oddp)|; for these cases, \iter\ generates code that
%executes ~expr~ only once.  the code for |\#'oddp|
%is slightly different from that for {\lisp 'oddp}; see the discussion
%under {\lisp for\dots in} and {\lisp for\dots on}.

\cpar |On-failure| is a misnomer. Because it is always evaluated, it behaves
more like the default third argument to the |gethash| function. As a result,
|on-failure (error "Not found")| makes no sense. Instead, the clauses |leave|
or |thereis| can be used in conjunction with |finally| as follows:
\begin{program}
(iter (for x in '(1 2 3))
      (if (evenp x) (leave x))
      (finally (error "not found")))
\end{program}

\cpar This clause may appear multiple times when all defaults are
identical. It can also be used together with either |always|/|never| or
|thereis| if their defaults match. More specifically, |on-failure nil| is
compatible with |thereis|, while |on-failure t| is compatible with |always|
and |never| clauses.
\begin{program}
(iter (for i in '(7 -4 2 -3))
      (if (plusp i)
	  (finding    i  such-that (evenp i))
          (finding (- i) such-that (oddp i))))
\end{program}

\startitem
\Clause{|finding| ~expr~ |maximizing| ~m-expr~ \opt\ |into| ~var~}
\clindexx{finding}{maximizing}
\Clause{|finding| ~expr~ |minimizing| ~m-expr~ \opt\ |into| ~var~}
\clindexx{finding}{minimizing}
\finishitem
Computes the extremum (maximum or minimum) value of ~m-expr~ over all
iterations, and returns the value of ~expr~ corresponding to the
extremum.  ~expr~ is evaluated inside the loop at the time the new
extremum is established.  If ~m-expr~ is never evaluated (due to, for
example, being embedded in a conditional clause), then the returned
value depends on the type, if any, of ~expr~ (or ~var,~ if
one is supplied).  If there is no type, the returned
value will be \nil; if the type is numeric, the returned value will be
zero.

\cpar For these two clauses, ~var~ may be a list of two
symbols; in that case, the first is used to record ~expr~ and
the second, ~m-expr.~  

\cpar As with |finding\dots such-that|, if ~m-expr~ is a sharp-quoted
function, then it is called on ~expr~ instead of being evaluated.

\end{clauses}

\subsubsection{Boolean Tests}

\begin{clauses}

\clausex{first-iteration-p}{}
Returns |t| in the first cycle of the loop, otherwise \nil.

\clausex{first-time-p}{}
Returns |t| the first time the expression is evaluated, and then \nil\ forever.
This clause comes handy when printing (optional) elements separated
by a comma:

\begin{program}
(iter (for el in '(nil 1 2 nil 3))
      (when el
        (unless (first-time-p)
          (princ ", "))
        (princ el)))
\end{program}
produces |"1, 2, 3"|.

\end{clauses}

\subsubsection{Aggregated Boolean Tests}

\begin{clauses}

\clausex{always}{~expr~}
If ~expr~ ever evaluates to
\nil, then \nil\ is immediately returned; the epilogue code is not
executed.  If ~expr~ never evaluates to \nil, the epilogue code
is executed and the last value of ~expr~ (or |t| if ~expr~ was never
evaluated) is returned (whereas |loop| would constantly return |t|).

% mention last evaluated clause when multiple always clauses?

\clausex{never}{~expr~}
Like |(always (not ~expr~))|, except it does not influence the last
value returned by a possible other |always| clause. That is,
\begin{program}
(iter (repeat 2)
      (always 2)
      (never nil)) \yields 2 ; not t
\end{program}

\clausex{thereis}{~expr~}
If ~expr~ is ever \nonnil,
its value is immediately returned without running epilogue code.
Otherwise, the epilogue code is performed and \nil\ is returned.

This clause cannot be used together with |always| or |never|, because their
defaults are opposed (similarly, |(loop always 3 thereis nil)| refuses to
compile in some implementations of |loop|).

\end{clauses}


\subsection{Control Flow}
\label{control-flow}
Several clauses can be used to alter the usual flow of control in a loop.

Note: the clauses of this and subsequent sections don't adhere to \iter's
usual syntax, but instead use standard Common Lisp syntax.  Hence the
format for describing syntax subsequently is
like the standard format used in the Common Lisp manual, not like the
descriptions of clauses above.

\begin{clauses}

\clausex{finish}{}
Stops the loop and runs the epilogue code.

%for example:
%
%\begin{program}
%(iter (with answer = nil)
%         (initially (make-a-mess))
%         (for i from 1 to 10)
%         (when (correct? i)
%           (setq answer i)
%           (finish))
%         (finally (cleanup)))
%\end{program}
%
%this code will execute |cleanup| whether or not the test |(correct?
%i)| ever succeeds. 
%the (more elegant) formulation,
%\begin{program}
%(iter (initially (make-a-mess))
%         (for i from 1 to 10)
%         (finding i such-that (correct? i))
%         (finally (cleanup)))
%\end{program}
%would not execute |cleanup| if |(correct? i)| succeeded; it
%would do an immediate return.

\clausex{leave}{\opt\ ~value~}
Immediately returns ~value~ (default \nil) from the current \iter\
form, skipping the epilogue code.  Equivalent to using |return-from|.

\clausex{next-iteration}{}
Skips the remainder of the loop body and begins the next iteration of
the loop.

\clausex{while}{~expr~}
If ~expr~ ever evaluates to \nil, the loop is terminated and the
epilogue code executed.  Equivalent to |(if (not ~expr~) (finish))|.

\clausex{until}{~expr~}
Equivalent to |(if ~expr~ (finish))|.

\clausex{if-first-time}{~then~ \opt\ ~else~}
If this clause is being executed for the first time in this invocation
of the \iter\ form, then the ~then~ code is evaluated; otherwise the
~else~ code is evaluated.

\cpar |(for ~var~ first ~expr1~ then ~expr2~)| is almost equivalent to
\begin{program}
(if-first-time (dsetq ~var~ ~expr1~)
               (dsetq ~var~ ~expr2~))
\end{program}
The only difference is that the |for| version makes ~var~ available
for use with |for\dots previous|.

\end{clauses}

\subsection{Code Placement}
When fine control is desired over where code appears in a loop
generated by \iter, the following special clauses may be useful.
They are all subject to code-motion problems (see section
\ref{code-movement}).  

\begin{clauses}

\clausex{initially}{|\&rest| ~forms~}
The lisp ~forms~ are placed in the prologue section of the loop, where
they are executed once, before the loop body is entered.

\clausex{after-each}{|\&rest| ~forms~}
The ~forms~ are placed at the end of the loop body, where they
are executed after each iteration.  Unlike the other clauses in this
section, ~forms~ may contain \iter\ clauses.

\clausex{else}{|\&rest| ~forms~}
The lisp ~forms~ are placed in the epilogue section of the loop, where they
are executed if this |else| clause is never met during execution of the
loop and the loop terminates normally.

\clausex{finally}{|\&rest| ~forms~}
The lisp ~forms~ are placed in the epilogue section of the loop, where
they are executed after the loop has terminated normally.

\clausex{finally-protected}{|\&rest| ~forms~}
The lisp ~forms~ are placed in the second form of an unwind-protect
outside the loop.  They are always executed after the loop has
terminated, regardless of how the termination occurred.

\end{clauses}


%%%%%%%%%%%%%%%%%%%%%%%%%%%%%%%%%%%%%%%%%%%%%%%%%%%%%%%%%%%%%%%%
\section{Other Features}

\subsection{Multiple Accumulations}
\label{multiple}

\begin{sloppypar}
It is permitted to have more than one clause accumulate into the same
variable, as in the following:
\begin{program}
(iter (for i from 1 to 10)
      (collect i into nums)
      (collect (sqrt i) into nums)
      (finally (return nums)))
\end{program}
Clauses can only accumulate into the same variable if they are
compatible.  |collect|, |adjoining|, |appending|, |nconcing|,
|unioning| and |nunioning| are compatible with each other; |sum|,
|multiply| and |counting| are compatible; |always| and |never| are compatible;
|finding| \dots |such-that| is compatible with either |thereis| or |always|
and |never| when their defaults match; and |maximize| and |minimize| clauses
are compatible only with other |maximize| and |minimize| clauses,
respectively.

%note that the same variable ~cannot~ be both an accumulation
%variable and an ordinary variable; there can be only one variable with
%a given name within an \iter\ form. 
\end{sloppypar}

\subsection{Named Blocks}

Like Common Lisp |block|s, \iter\ forms can be given names.  The
name should be a single symbol, and it must be the first form in the
\iter.  The generated code behaves exactly like a named block; in
particular, |(return-from ~name~)| can be used to exit it:
\begin{program}
(iter fred
      (for i from 1 to 10)
      (iter barney
            (for j from i to 10)
            (if (> (* i j) 17)
                (return-from fred j))))
\end{program}
An \iter\ form that is not given a name is implicitly named \nil.

Sometimes one would like to write an expression in an inner \iter\ form,
but have it processed by an outer \iter\ form.  This is possible with
the |in| clause.  

\begin{clauses}

\clausex{in}{~name~ |\&rest| ~forms~}
Evaluates ~forms~ as if they were part of the \iter\ form named
~name~.  In other words, \iter\ clauses are processed by the \iter\
form named ~name,~ and not by any \iter\ forms that occur inside ~name.~

\cpar As an example, consider the problem of collecting a list of the
elements in a two-dimensional array.  The naive solution,
\begin{program}
(iter (for i below (array-dimension ar 0))
      (iter (for j below (array-dimension ar 1))
            (collect (aref ar i j))))
\end{program}
\noindent is wrong because the list created by the inner \iter\ is simply
ignored by the outer one.  But using |in| we can write:
\begin{program}
(iter outer (for i below (array-dimension ar 0))
      (iter (for j below (array-dimension ar 1))
            (in outer (collect (aref ar i j)))))
\end{program}
\noindent which has the desired result.

\end{clauses}

\subsection{Destructuring}
\label{destructuring}

In many places within \iter\ clauses where a variable is expected, a
list can be written instead.  In these cases, the value to be assigned
is ~destructured~ according to the pattern
described by the list.  As a simple example, the clause
\begin{program}
(for (key . item) in alist)
\end{program}
\noindent will result in |key| being set to the |car| of
each element in |alist|, and |item| being set to the |cdr|.  The
pattern list may be nested to arbitrary depth, and (as the example
shows) need not be terminated with \nil; the only requirement is that
each leaf be a bindable symbol (or \nil, in which case no binding is
generated for that piece of the structure).

Sometimes, you might like to do the equivalent of a
|multiple-value-setq| in a clause.  This 
``multiple-value destructuring'' can be expressed by writing 
\linebreak |(values $pat_1$ $pat_2 \ldots$)| for a destructuring
pattern, as in 
\begin{program}
(for (values (a . b) c d) = (three-valued-function ...))
\end{program}
\begin{sloppypar}
Note that the $pat_i$ can themselves be destructuring patterns (though
not multiple-value destructuring patterns).  You can't do multiple-value
destructuring in a |with| clause; instead wrap the whole \iter\
form in a |multiple-value-bind|.
\end{sloppypar}

\begin{note}{Rationale}
There are subtle interactions between variable declarations and
evaluation order that make the correct implementation of
multiple-value destructuring in a |with| somewhat tricky.
\end{note}

The destructuring feature of \iter\ is available as a separate
mechanism, using the |dsetq| macro:

\begin{clauses}
\defmacro{dsetq}{template expr}
Performs destructuring of ~expr~ using ~template~.   May be used
outside of an \iter\ form. Yields the primary value of ~expr~.

\end{clauses}

\subsection{On-line Help}

\begin{sloppypar}
There is a limited facility for on-line help, in the form of the 
|display-iterate-clauses| function.
\end{sloppypar}

\begin{clauses}

\defun{display-iterate-clauses}{\opt\ clause-spec}
Displays a list of \iter\ clauses.  If ~clause-spec~ is not
provided, all clauses are shown; if it is a symbol, all clauses
beginning with that symbol are shown; and if it is a list of symbols,
all clauses for which ~clause-spec~ is a prefix are shown.

\end{clauses}

\subsection{Parallel Binding and Stepping}
\label{bindings}

The parallel binding and stepping of variables is a feature that
\iter\ does ~not~ have.  This section attempts to provide a rationale.

We say that two variables are bound ~in parallel~ if neither
binding shadows the other.  This is the usual semantics of |let|
(as opposed to |let*|).  Similarly, we can say that iteration
variables are stepped in parallel if neither variable is updated
before the other, conceptually speaking; in other words, if the code
to update each variable can reference the old values of both variables.

|loop| allows parallel binding of variables and parallel stepping of
driver variables.  My view is that if you are depending on the 
serial/parallel distinction, you are doing something obscure.  If you
need to bind 
variables in parallel using |with|, then you must be using a
variable name that shadows a name in the existing lexical environment.
Don't do that.  The most common use for parallel stepping is to track
the values of variables on the previous iteration, but in fact this
does not require parallel stepping at all; the following will work:
\begin{program}
(iter (for current in list)
      (for prev previous current)
      ...)
\end{program}


%%%%%%%%%%%%%%%%%%%%%%%%%%%%%%%%%%%%%%%%%%%%%%%%%%%%%%%%%%%%%%%%
\section{Types and Declarations}
\label{types}

\subsection{Discussion}

Sometimes efficiency dictates that the types of variables be declared.
This type information needs to be communicated to \iter\ so it can
bind variables to appropriate values.  Furthermore, \iter\ must often
generate internal variables invisible to the user; there needs to be a
way for these to be declared.

As an example, consider this code, which will return the number of
odd elements in |number-list|:
\begin{program}
(iter (for el in number-list)
      (count (oddp el)))
\end{program}
In processing this form,
\iter\ will create an internal variable, let us call it |list17|, to
hold the successive |cdr|s of |number-list|, and  
will bind the variable to |number-list|.  It will also generate a
default binding for |el|; only inside the body of the loop will |el|
be set to the |car| of |list17|.  Finally, \iter\ will generate a
variable, call it |result|, to hold the result of the count, and will
bind it to zero.

When dealing with type declarations, \iter\ observes one simple rule:
~it will never generate a declaration unless requested to do so.~  The
reason is that such declarations might mask errors in compiled code by
avoiding error-checks; the resulting problems would be doubly hard to
track down because the declarations would be hidden from the
programmer.  Of course, a compiler might omit error-checks even in the
absence of declarations, though this behavior can usually be avoided,
e.g. by saying |(declaim (optimize (safety 3)))|.

So, the above \iter\ form will generate code with no declarations.
But say we wish to declare the types of |el| and the internal
variables |list17| and |result|.  How is this done?

Declaring the type of |el| is easy, since the programmer knows
the variable's name:
\begin{program}
(iter (for el in number-list)
      (declare (fixnum el))
      (counting (oddp el)))
\end{program}
\iter\ can read variable type declarations like this one.  Before
processing any clauses, it scans the entire top-level form for type
declarations 
and records the types, so that variable bindings can be performed
correctly.  In this case, |el| will be bound to zero
instead of \nil.  Also, \iter\ collects all the top-level declarations
and puts them at the begining of the generated code, so it is not
necessary to place all declarations at the beginning of an \iter\
form; instead, they can be written near the variables whose types they
declare.  

Since \iter\ is not part of the compiler, it will not know
about declarations that occur outside an \iter\ form; these
declarations must be repeated inside the form.

Here is another way we could have declared the type of |el|:
\begin{program}
(iter (for (the fixnum el) in number-list)
      (counting (oddp el)))
\end{program}
\iter\ extends the Common Lisp |the|\lispindex{the}
form to apply to variables as well as value-producing forms; anywhere
a variable is allowed---in a |with| clause, as the iteration
variable in a driver clause, as the |into| argument of an
accumulation clause, even inside a destructuring template---you can
write |(the ~type~ ~symbol~)| instead.

There is one crucial difference between using a |the| form and
actually declaring the variable: explicit declarations are always
placed in the generated code, but type information from a |the|
form is not turned into an actual declaration unless you tell \iter\
to do so using |iterate:declare-variables|.  See below.

\begin{sloppypar}
Declaring the types of internal variables is harder than declaring the
types of explicitly mentioned variables, since their names
are unknown.  You do it by declaring |iterate:declare-variables|
somewhere inside the top level of the \iter\ form.  (This will also
generate declarations for variables declared using |the|.)
\iter\ does not provide much selectivity here: it's all or none. 
And unfortunately, since \iter\ is not privy to compiler information
but instead reads declarations itself, it will not hear if you
|(declaim (iterate:declare-variables))|.  Instead, set the variable 
|iterate::*always-declare-variables*| to |t| at
compile-time, using |eval-when|.
\end{sloppypar}

To determine the appropriate types for internal variables, \iter\ uses
three sources of information:

\begin{itemize}

\item Often, the particular clause dictates a certain type for a
variable; \iter\ will use this information when available.  In the
current example, the variable |list17| will be given the type
|list|, since that is the only type that makes sense; and the
variable |result| will be given the type |fixnum|, on the
assumption that you will not be counting high enough to need bignums.
You can override this assumption only by using and explicitly declaring a
variable: 
\begin{verbatim}
(iter (declare (iterate:declare-variables))
      (for el in number-list)
      (count (oddp el) into my-result)
      (declare (integer my-result))
      (finally (return my-result)))
\end{verbatim}

\begin{sloppypar}
Other examples of the type assumptions that \iter\ makes are: type
|list| for |into| variables of collection clauses; type |list| for
expressions that are to be destructured; type |vector| for the
variable holding the vector in a |for\dots in-vector| clause, and
similarly for |string| and the |for\dots in-string| clause;
and the implementation-dependent type
|(type-of array-dimension-limit)| for the index and limit
variables generated by sequence iteration drivers like |for\dots
in-vector| and |for\dots in-string| (but not |for\dots in-sequence|,
because it may be used to iterate over a list). 
\end{sloppypar}

\item Sometimes, \iter\ will examine expressions and try to determine
their types in a simple-minded way.  If the expression is
self-evaluating (like a number, for instance), \iter\ knows that the
expression's type is the same as the type of the value it denotes, so
it can use that type.  If the expression is of the form |(the ~type~
~expr~)|, \iter\ is smart enough to extract ~type~ and use it.
However, the current version of \iter\ does not  
examine declarations of function result types or do any type
inference.  It will not determine, for 
example, that the type of |(+ 3 4)| is |fixnum|, or even
|number|.

\item In some cases, the type of an internal variable should match the
type of some other variable.  For instance, \iter\ generates an
internal variable for |(f x)| in the
clause |(for i from 1 to (f x))|, and in the absence of other
information will give it the same type as |i|.  If, however, the
expression had been written |(the fixnum (f x))|, then \iter\
would have given the internal variable the type |fixnum|
regardless of |i|'s type.  The type incompatibility errors that
could arise in this situation are not checked for.

\end{itemize}

Note that if you do declare |iterate:declare-variables|, then
\iter\ may declare user variables as well as internal ones if they do
not already have declarations, though only for variables that it
binds.  For instance, in this code:

\begin{program}
(iter (declare (iterate:declare-variables))
      (for i from 1 to 10)
      (collect i into var))
\end{program}
the variable |var| will be declared to be of type |list|.


\subsection{Summary}

\iter\ understands standard Common Lisp variable type declarations
that occur within an \iter\ form and
will pass them through to the generated code.  If the declaration
|(iterate:declare-variables)|\lispindex{declare-variables}
appears at the top level of an 
\iter\ form, or if 
|iterate::*always-declare-variables*|\lispindex{*always-declare-variables*} 
is \nonnil, then \iter\ will use the type information gleaned from user
declarations, self-evaluating expressions and |the| expressions,
combined with reasonable assumptions, to determine variable
types and declare them.


%%%%%%%%%%%%%%%%%%%%%%%%%%%%%%%%%%%%%%%%%%%%%%%%%%%%%%%%%%%%%%%%
\section{Problems with Code Movement}
\label{code-movement}

Some \iter\ clauses, or parts of clauses, result in code being
moved from the location of the clause to other parts of the loop.
Drivers behave this way, as do code-placement clauses like |initially|
and |finally|.  When using these clauses, there is a danger of writing
an expression that makes sense in its apparent location but will be
invalid or have a different meaning in another location.  For example:
\begin{program}
(iter (for i from 1 to 10)
      (let ((x 3))
        (initially (setq x 4))))
\end{program}
While it may appear that the |x| of |(initially (setq x 4))| is the
same as the |x| of |(let ((x 3)) \dots|, in fact they are not:
|initially| moves its code outside the loop body, so |x| would refer
to a global variable.  Here is another example of the same problem:
\begin{program}
(iter (for i from 1 to 10)
      (let ((x 3))
        (collect i into x)))
\end{program}
If this code were executed, |collect| would create a binding for its
|x| at the top level of the \iter\ form that the |let| will shadow.

Happily, \iter\ is smart enough to catch these errors; it
walks all problematical code to ensure that free variables are not
bound inside the loop body, and checks all variables it binds for the
same problem.

However, some errors cannot be caught:

\begin{program}
(iter (with x = 3)
      (for el in list)
      (setq x 1)
      (reducing el by \#'+ initial-value x))
\end{program}
|reducing| moves its |initial-value| argument to the initialization
part of the loop in order to produce more efficient code.  Since
\iter\ does not perform data-flow analysis, it cannot determine that
|x| is changed inside the loop; all it can establish is that |x| is
not bound internally.  Hence this code will not signal an
error and will use $3$ as the initial value of the reduction.

The following list summarizes all cases that are subject to these code
motion and variable-shadowing problems.
\begin{itemize}
\item Any variable for which \iter\ creates a binding, including those
used in |with| and the |into| keyword of many clauses.

\begin{sloppypar}
\item The special clauses which place code: |initially|, |after-each|, |else|,
|finally| and |finally-protected|. 
\end{sloppypar}

\item The variables of a |next| or |do-next| form.

\item The |initially| arguments of |for\dots initially\dots then| and
|for\dots previous|. 

\item The |then| argument of |for\dots initially\dots then|.

\item The |initial-value| arguments of |reducing| and |accumulate|.

\item The |on-failure| argument of |finding\dots such-that|.
\end{itemize}

%%%%%%%%%%%%%%%%%%%%%%%%%%%%%%%%%%%%%%%%%%%%%%%%%%%%%%%%%%%%%%%%
\section{Differences Between {\tt Iterate} and {\tt Loop}}
|loop| contains a great deal of complexity which \iter\ tries to
avoid.  Hence many esoteric features of |loop| don't exist in \iter.
Other features have been carried over, but in a cleaned-up form.  
And of course, many new features have been added; they are not
mentioned in this list.

\begin{itemize}

\item \iter's syntax is more Lisp-like than |loop|'s, having a higher
density of parens.

\item The current implementation of \iter, unlike the current version
of  |loop| (as documented in {\em Common Lisp, 2nd Ed.\/}), is
extensible (see section \ref{extend}).  

\item |loop| puts the updates of all driver variables at the top of
the loop; \iter\ leaves them where the driver clauses appear.

\item While for the most part \iter\ clauses that resemble |loop| clauses
behave similarly, there are some differences.  For instance, there is
no |for\dots =\dots then| in \iter; instead use 
|for\dots initially\dots then|.  

\item |loop| binds the variable |it| at certain times
to allow pseudo-English expressions like |when ~expr~ return it|. 
In \iter, you must bind ~expr~ to a variable yourself.  Note that
|when ~expr~ return it| is like |thereis ~expr~| except that the latter is an
accumulation clause and therefore competes with other accumulations
(remember section \ref{multiple} above).

% repeat different behaviour of |always| clause here?

\item |loop| has a special |return| clause, illustrated in the
previous item.  \iter\ doesn't need one,  since an ordinary Lisp
|return| has the same effect.

\item |loop| allows for parallel binding and stepping of iteration
variables.  \iter\ does not.  (See section \ref{bindings}.)

\item |loop| and \iter\ handle variable type declarations very
differently.  |loop| provides a special syntax for declaring variable
types, and does not examine declarations.  Moreover, the standard
implementation of |loop| will
generate declarations when none are requested.
\iter\ parses standard Common Lisp type declarations, and will never
declare a variable itself unless declarations are 
specifically requested.

\end{itemize}

%%%%%%%%%%%%%%%%%%%%%%%%%%%%%%%%%%%%%%%%%%%%%%%%%%%%%%%%%%%%%%%%
\section{Rolling Your Own}
\label{extend}

\subsection{Introduction}

\iter\ is extensible---you can write new clauses that embody new
iteration patterns.  You might want to write a new driver clause for a
data structure of your own, or you might want to write a clause that
collects or manipulates elements in a way not provided by \iter. 

This section describes how to write clauses for \iter.  Writing a
clause is like writing a macro.  In fact, writing a clause ~is~
writing a macro: since \iter\ code-walks its body and macroexpands,
you can add new abstractions to \iter\ with good old |defmacro|.

Actually, there are two extensions you can make to \iter\ that are
even easier than writing a macro.  They are adding a synonym for an
existing clause and defining a driver clause for an indexable
sequence.  These can be done with |defsynonym| and
|defclause-sequence|, respectively.  See section \ref{aids}, below.

The rest of this section explains how to write macros that expand into
\iter\ clauses.
Here's how you could add a simplified version of \iter's |multiply|
clause, if \iter\ didn't already have one:
\begin{program}
(defmacro multiply (expr)
  `(reducing ,expr by \#'* initial-value 1))
\end{program}

If you found yourself summing the square of an expression often, you
might want to write a macro for that.  A first cut might be
\begin{program}
(defmacro sum-of-squares (expr)
  `(sum (* ,expr ,expr)))
\end{program}
but if you are an experienced macro writer, you will realize that this
code will evaluate ~expr~ twice, which is probably a bad idea.  A
better version would use a temporary:
\begin{program}
(defmacro sum-of-squares (expr)
  (let ((temp (gensym)))
    `(let ((,temp ,expr))
       (sum (* ,temp ,temp)))))
\end{program}
Although this may seem complex, it is just the sort of thing you'd
have to go through to write any macro, which illustrates the point of
this section: if you can write macros, you can extend \iter.

Our macros don't use \iter's keyword-argument
syntax.  We could just use keywords with |defmacro|, but we would
still not be using \iter's clause indexing mechanism.  Unlike Lisp,
which uses just the first symbol of a form to determine what function
to call, \iter\ individuates clauses by the list of required keywords.
For instance, |for\dots in| and |for\dots in-vector| are different clauses
implemented by distinct Lisp functions.  

To buy into this indexing scheme, as well as the keyword-argument
syntax, use |defmacro-clause|:

\begin{clauses}

\defmacro{defmacro-clause}{arglist |\&body| body}
Defines a new \iter\ clause.  ~arglist~ is a list of symbols which are
alternating keywords and arguments.  \opt\ may be used, and the list
may be terminated by |\&sequence|.  ~body~ is an ordinary macro body,
as with |defmacro|.  If the first form of ~body~ is a string, it is
considered a documentation string and will be shown by
|display-iterate-clauses|. |defmacro-clause| will signal an error if 
defining the clause would result in an ambiguity.  E.g. you cannot
define the clause |for\dots from| because there would be no way to
distinguish it from a use of the |for| clause with optional keyword |from|.

\end{clauses}

\medskip

Here is |multiply| using |defmacro-clause|.  The keywords are capitalized
for readability.
\begin{program}
(defmacro-clause (MULTIPLY expr \opt\ INTO var)
  `(reducing ,expr by \#'* into ,var initial-value 1))
\end{program}
You don't have to worry about the case when |var| is not supplied; for
any clause with an |into| keyword, saying |into nil| is equivalent to
omitting the |into| entirely.

As another, more extended example, consider the fairly common
iteration pattern that 
involves finding the sequence element that maximizes (or minimizes) some
function.  \iter\ provides this as |finding\dots maximizing|, but it's
instructive to see how to write it.
Here, in pseudocode, is how you might write such a loop for
maximizing a function F:

\begin{quote}
\begin{tabbing}
set variable MAX-VAL to NIL; \\
set variable WINNER to NIL; \\
for\=\ each element EL in the sequence \\
\>  if\=\ MAX-VAL\=\ is NIL or F(EL) $>$ MAX-VAL then \\
\>\>    set MAX-VAL to F(EL); \\
\>\>    set WINNER to EL; \\
\>  end if; \\
end for; \\
return WINNER.
\end{tabbing}
\end{quote}

Here is the macro:
\begin{program}
(defmacro-clause (FINDING expr MAXIMIZING func \opt\ INTO var)
  (let ((max-val (gensym))
        (temp1 (gensym))
        (temp2 (gensym))
        (winner (or var iterate::*result-var*)))
    `(progn 
       (with ,max-val = nil)
       (with ,winner = nil)
       (cond 
        ((null ,max-val)
         (setq ,winner ,expr)
         (setq ,max-val (funcall ,func ,winner))
        (t
         (let* ((,temp1 ,expr)
                (,temp2 (funcall ,func ,temp1)))
           (when (> ,temp2 ,max-val)
             (setq ,max-val ,temp2)
             (setq ,winner ,temp1))))))
       (finally (leave ,winner)))))
\end{program}
Note that if no |into| variable is supplied, we use
|iterate::*result-var*|, which contains the internal variable into
which all clauses place their results.  If this variable is bound by
some clause, then \iter\ will return its value automatically;
otherwise, \nil\ will be returned.  

\subsection{Writing Drivers}

In principle, drivers can be implemented just as easily as other
\iter\ clauses.  In practice, they are a little harder to get right. 
As an example, consider writing a driver that
iterates over all the 
elements of a vector, ignoring its fill-pointer.  |for\dots in-vector|
won't work for this, because it observes the fill-pointer.  It's necessary to
use |array-dimension| instead of |length| to obtain the size of the
vector.  Here is one approach:

\begin{program}
(defmacro-clause (FOR var IN-WHOLE-VECTOR v)
  "All the elements of a vector (disregards fill-pointer)"
  (let ((vect (gensym))
        (index (gensym)))
    `(progn
       (with ,vect = ,v)
       (for ,index from 0 below (array-dimension ,vect 0))
       (for ,var = (aref ,vect ,index)))))
\end{program}
Note that we immediately put |v| in a variable, in case it is an
expression.  Again, this is just good Lisp macrology.  It also has a
subtle effect on the semantics of the driver: |v| is evaluated only
once, at the beginning of the loop, so changes to |v| in the loop have
no effect on the driver.  Similarly, the bounds for numerical iteration
e.g. the above |array-dimension| are also evaluated once only.  This is how
all of \iter's drivers work.


There is an important point concerning the |progn| in this code.  We
need the |progn|, of course, because we are returning several forms,
one of which is a driver.  But \iter\ drivers must occur at top-level.
Is this code in error?  No, because ~top-level~ is defined in \iter\
to include forms inside a |progn|.  This is just the definition of
top-level that Common Lisp uses, and for the same reason: to allow
macros to return multiple forms at top-level.

While our |for\dots in-whole-vector| clause will work, it is not
ideal.  In particular, it does not support generating.  Do do so, we
need to use |for\dots next| or |for\dots do-next|.  The job is
simplified by the |defmacro-driver| macro.

\begin{clauses}

\defmacro{defmacro-driver}{arglist |\&body| body}
Defines a driver clause in
both the |for| and |generate| forms, and provides a parameter
|generate| which ~body~ can examine to determine how it was invoked.
~arglist~ is as in |defmacro-clause|, and should begin with the symbol
|for|. 

\end{clauses}

With |defmacro-driver|, our driver looks like this:
\begin{program}
(defmacro-driver (FOR var IN-WHOLE-VECTOR v)
  "All the elements of a vector (disregards fill-pointer)"
   (let ((vect (gensym))
         (end (gensym))
         (index (gensym))
         (kwd (if generate 'generate 'for)))
     `(progn
        (with ,vect = ,v)
        (with ,end = (array-dimension ,vect 0))
        (with ,index = -1)
        (,kwd ,var next (progn (incf ,index)
                               (if (>= ,index ,end) (terminate))
                               (aref ,vect ,index))))))
\end{program}


We are still missing one thing: the |\&sequence| keywords.  
We can get them easily enough, by writing 
\begin{program}
(defmacro-driver (FOR var IN-WHOLE-VECTOR v \&sequence) 
  ...)
\end{program}
We can now refer to parameters |from|, |to|, |by|, etc. which contain
either the values for the corresponding keyword, or \nil\ if the
keyword was not supplied.  Implementing the right code for these
keywords is cumbersome but not difficult; it is left as an exercise.
But before you begin, see |defclause-sequence| below for an easier way.
       
\subsection{Extensibility Aids}
\label{aids}

This section documents assorted features that may be of use in
extending \iter.

\begin{clauses}

\defunexpvar{*result-var*}
Holds the variable that is used to return a value as a result of the
\iter\ form.  You may examine this and use it in a |with| clause, but
you should not change it.

\defmacro{defsynonym}{syn word}
 Makes ~syn~ a synonym for the existing \iter\ keyword ~word.~  Only
the first word in each clause can have synonyms. 

\defmacroxx{defclause-sequence}{element-name index-name}{|\&key|
access-fn size-fn sequence-type}{element-type
element-doc-string index-doc-string}  
Provides
a simple way to define sequence clauses.  Generates two
clauses, one for iterating over the sequence's elements, the other
for iterating over its indices.  The first symbol of both
clauses will have print-name |for|.
~element-name~  and ~index-name~ should be symbols.
~element-name~ is the second keyword of the element iterator (typically of
the form 
|in-~sequence-type~|), and ~index-name~ is the second keyword
of the index-iterator (typically of the form
|index-of-~sequence-type~|).  Either name may be 
\nil, in which case the corresponding clause is not defined.  If both
symbols are supplied, they should be in the same package.  The |for|
that begins the clauses will be in this package.

\cpar ~access-fn~ is the function to be used to
access elements of the sequence in the element iterator.  The function
should take two 
arguments, a sequence and an index, and return the appropriate element.
~size-fn~ should denote a function of one argument, a sequence, that
returns its size.  Both ~access-fn~ and ~size-fn~ are required for the
element iterator, but only ~size-fn~ is needed for the index iterator.

\cpar The ~sequence-type~ and ~element-type~ keywords are used to
suggest types for the variables 
used to hold the sequence and the
sequence elements, respectively.  The usual rules about \iter's
treatment of variable type declarations apply (see section \ref{types}).

\cpar ~element-doc-string~ and ~index-doc-string~ are
the documentation strings, for use with |display-iterate-clauses|. 

\cpar The generated element-iterator performs destructuring on the
element variable.

\cpar As an example, the above |for\dots in-whole-vector| example
could have been written:
\begin{program}
(defclause-sequence IN-WHOLE-VECTOR INDEX-OF-WHOLE-VECTOR
  :access-fn 'aref
  :size-fn \#'(lambda (v) (array-dimension v 0))
  :sequence-type 'vector
  :element-type t
  :element-doc-string 
     "Elements of a vector, disregarding fill-pointer"
  :index-doc-string 
     "Indices of vector, disregarding fill-pointer")
\end{program}

\end{clauses}

\subsection{Subtleties}

There are some subtleties to be aware of when writing \iter\ clauses.
First, the code returned by your macros may be |nconc|'ed into a list,
so you should always returned freshly consed lists, rather than
constants.  Second, \iter\ matches clauses by using |eq| on the first
symbol and |string=| on the subsequent ones, so the package of the
first symbol of a clause is relevant.  All of the clauses in this manual
have their first word in the \iter\ package.  
You can use the package system in the usual way to shadow
\iter\ clauses without replacing them.

%%% say more here, about the badness that only the first word of a
%%% clause is packagey.

\section{Non-portable Extensions to Iterate (Contribs)}
\label{contribs}

Currently, there is only one non-portable extension to iterate in the
distribution: iterate-pg. If you have made an extension that depends
on non-portable features, feel free to send them to |asf@boinkor.net|
for inclusion in the iterate distribution.

\subsection{An SQL query driver for iterate}

The pg package by Eric Marsden (see |http://cliki.net/pg|) provides an
interface to the PostgreSQL database. Using the \iterpg\ extension, it
is possible to handle the results of SQL queries using \iter.

This usage example should give you an idea of how to use it:

\begin{program}
(pg:with-pg-connection (c "somedb" "someuser")
  (iter (for (impl version date) in-relation "select * from version"
                                 on-connection *dbconn*)
        (collect version)))
\end{program}

To use the extension via |ASDF|, simply make your system depend on the
|iterate-pg| system instead of the |iterate| system. To load it
manually, use:

\begin{program}
  (asdf:oos 'asdf:load-op :iterate-pg)
\end{program}

%%%%%%%%%%%%%%%%%%%%%%%%%%%%%%%%%%%%%%%%%%%%%%%%%%%%%%%%%%%%%%%%
\section{Obtaining {\tt Iterate}}

\begin{sloppypar}
The information in this chapter is obsolete but included for
completeness's sake; Currently, the most up-to-date information on
\iter\ can be found at |http://boinkor.net/iterate.html|.
\end{sloppypar}

\begin{sloppypar}
\iter\ currently runs on Lisp Machines, and on
HP's, Sun3's and Sparcstations under Lucid.  
\iter\ source and binaries are available at the MIT AI Lab in the
subdirectories of |/src/local/lisplib/|.  The source file,
|iterate.lisp|, is also available for anonymous FTP in the directory
|/com/fpt/pub/| on the machine |TRIX.AI.MIT.EDU| (Internet number
128.52.32.6).  If you are unable to obtain |iterate| in one of these
ways, send mail to |jba@ai.mit.edu| and I will send you the source
file. 
\end{sloppypar}

\begin{sloppypar}
\iter\ resides in the |iterate| package (nickname |iter|).  Just say
\linebreak |(use-package :iterate)| to make all the necessary symbols
available. 
If a symbol is not exported, it appears in this manual with an
``|iterate::|'' prefix.
\end{sloppypar}

Send bug reports to |bug-iterate@ai.mit.edu|.  The |info-iterate|
mailing list will have notices of changes and problems; to have
yourself added, send mail to |info-iterate-request@ai.mit.edu|.


\medskip

\begin{flushleft}
 \bf Acknowledgements
\end{flushleft}
\smallskip

Richard Waters provided invaluable criticism which spurred me to improve
\iter\ greatly.  As early users, David Clemens, Oren Etzioni and Jeff
Siskind helped ferret out many bugs.

%\begin{theindex}
% The index files must be generated with the genindex program
% or from makeindex iterate-manual.idx which creates iterate-manual.ind.

%\baselineskip\normalbaselineskip
%\advance\baselineskip by -2pt

\input{iterate-manual.ind}

%\underline{Clauses}
%\input{iterate-manual.clindex}
%
%\indexspace
%\underline{Lisp}
%\input{iterate-manual.lispindex}

%\end{theindex}


\end{document}


% arch-tag: "c0f871a6-313c-11d8-abb9-000c76244c24"


%\end{theindex}


\end{document}


% arch-tag: "c0f871a6-313c-11d8-abb9-000c76244c24"

%
%\indexspace
%\underline{Lisp}
%% Fix ``.'' spacing in lists
\documentstyle[12pt,aimemo]{article}

\setlength{\textwidth}{6.0in}
\setlength{\oddsidemargin}{0.25in}
\setlength{\evensidemargin}{0.25in}
\setlength{\topmargin}{0pt}
\setlength{\headsep}{24pt}
\setlength{\textheight}{8.5in}

\let\origpar=\par

% somehow, LaTex fucks with things so that the plain Tex \obeyspaces
% doesn't work.  This is the same, except we use an mbox containing
% a space instead of just a space.

\newcommand{\spc}{\mbox{ }}
\newcommand{\tAb}{\mbox{        }}
\def\obspaces{\catcode`\ =\active\catcode`\^^I=\active}
{\obspaces \global\let =\spc\global\let^^I=\tAb}

\newenvironment{program}{\medskip\samepage\tt\obeylines\obspaces}{\medskip}

% use | for \tt
\catcode`\|=13
\newif\ifvbar
\def|{\ifvbar
         \endgroup\vbarfalse
      \else 
         \begingroup\tt\vbartrue 
      \fi}



\newcommand{\lisp}{\tt}
\newcommand{\iter}{|iterate|}
\newcommand{\iterpg}{|iterate-pg|}
\newcommand{\setf}{|setf|}
\newcommand{\nil}{|nil|}
\newcommand{\nonnil}{non-\nil}
\newcommand{\opt}{{\lisp \&optional}}
\newcommand{\yields}{$\Rightarrow$}



\newbox\Dots
\setbox\Dots=\hbox{\small ...}
\renewcommand{\dots}{\copy\Dots}

% Here's a way to define an environment which will treat all
% paragraphs inside it in the same manner.  It's, as near as I can
% figure, the way that Latex does things.  Using \everypar won't work;
% apparently Latex resets that a lot.  What you have to do is: at the
% top of the file do \let\origpar=\par.  In the
% begin part of the environment, do the parshape (or whatever), then say
% \def\par{{\origpar}}.  The extra braces are necessary.   That's all
% you have to do; at the end of the environment things will return to
% normal automatically.  Why does this all work?  I have no idea.
% Another important thing: there should be a blank line (i.e. a \par)
% between the end of the last paragraph and the \end{environment}.

\newlength{\clindent}
\setlength{\clindent}{0.5in}
\newlength{\clparindent}
\setlength{\clparindent}{\parindent}

\newenvironment{clauses}{\advance\linewidth -\parindent
                        \hangindent\clindent\relax
                         \hangafter=0
                          \parindent=0pt
                          \def\par{{\origpar}}}{}

\newcommand{\cpar}{\hspace \clparindent}

\newcommand{\startitem}{\bigskip\pagebreak[3]}
\newcommand{\finishitem}{\smallskip}

\newcommand{\clause}[1]{\startitem\Clause{#1}\finishitem}
\newcommand{\Clause}[1]{\hbox{#1}}
\newcommand{\clindex}[1]{\index{{\lisp #1}}}
\newcommand{\clindexx}[2]{\index{{\lisp #1\dots #2}}}
\newcommand{\clindexxx}[3]{\index{{\lisp #1\dots #2\dots #3}}}
\newcommand{\clausex}[2]{\startitem\Clausex{#1}{#2}\finishitem}
\newcommand{\Clausex}[2]{\Clause{|#1| #2}\clindex{#1}}

\newcommand{\defvar}[1]{\startitem\deff{#1}{}{Variable}\finishitem}
\newcommand{\defunexpvar}[1]{\startitem\deffnoindex{iterate::#1}{}{Variable}
                                        \lispindex{#1}\finishitem}
\newcommand{\defun}[2]{\startitem\deff{#1}{#2}{Function}\finishitem}
\newcommand{\defmacro}[2]{\startitem\deff{#1}{#2}{Macro}\finishitem}
\newcommand{\defunx}[3]{\startitem\deffx{#1}{#2}{#3}{Function}\finishitem}
\newcommand{\defmacrox}[3]{\startitem\deffx{#1}{#2}{#3}{Macro}\finishitem}
\newcommand{\defunxx}[4]{\startitem\deffxx{#1}{#2}{#3}{#4}{Function}\finishitem}
\newcommand{\defmacroxx}[4]{\startitem\deffxx{#1}{#2}{#3}{#4}{Macro}\finishitem}

\newcommand{\deff}[3]{\deffnoindex{#1}{#2}{#3}\lispindex{#1}}
\newcommand{\deffnoindex}[3]{\hbox to \hsize{{\tt #1} {\it #2}\hfill [{\it #3}]}}
\newcommand{\deffx}[4]{\deff{#1}{#2}{#4}\moreargs{#1}{#3}}
\newcommand{\deffxx}[5]{\deffx{#1}{#2}{#3}{#5}\moreargs{#1}{#4}}

\newcommand{\moreargs}[2]{\hbox to \hsize{\phantom{\lisp #1} {\it #2}\hfill}}

\newcommand{\lispindex}[1]{\index{{\lisp #1}}}

 \newenvironment{note}[1]{\pagebreak[2]\bigskip
                          \hrule\smallskip\small
                         {\setlength{\parindent}{0pt}
                           \par{\bf #1:}}}{\smallskip\hrule\bigskip}



\makeindex

%%%%%%%%%%%%%%%%%%%%%%%%%%%%%%%%%%%%%%%%%%%%%%%%%%%%%%%%%%%%%%%%

%\thispagestyle{empty}

%\vspace*{2in}

%\title{The |Iterate| Manual}
%\author{Jonathan Amsterdam}

%\maketitle

%\thispagestyle{empty}

%\begin{center}
%\Large *** DRAFT ***
%\end{center}
%This manual is in beta-test. It will become an AI Memo in a couple of
%months.  I would appreciate any comments or
%corrections.  You can reach me at email address |jba@ai.ai.mit.edu|.

\begin{document}

\def\writtenby{Jonathan Amsterdam}
\def\contractno{This report describes research done at the Artificial
Intelligence Laboratory of the Massachusetts Institute of Technology.
Support for the laboratory's artificial intelligence research is
provided in part by the Advanced Research Projects Agency of the
Department of Defense under Office
of Naval Research contract N00014-85-K-0124.}

\AIMEMO{1236}{\Large The Iterate Manual \\[1in]}
{This is the manual for version 1.4 of \iter, a powerful iteration
macro for Common  
Lisp. \iter\ is similar to {\tt loop} but provides numerous additional
features, is well integrated with Lisp, and is extensible. 
\vfill }


% use ~ for \em
\catcode`\~=13
\newif\ifem
\def~{\ifem
         \/\endgroup\emfalse
      \else 
         \begingroup\em\emtrue 
      \fi}



%\pagebreak
%\setcounter{page}{1}

%\thispagestyle{empty}


\tableofcontents

\pagebreak
\pagestyle{headings}

\section{Introduction}

\begin{sloppypar}
This manual describes
\iter, a powerful iteration facility for Common Lisp.
\iter\ provides abstractions for many common iteration
patterns and allows for the definition of additional patterns.  \iter\ is
a macro that expands into ordinary Lisp at
compile-time, so it is more efficient than higher-order
functions like |map| and |reduce.|  
While it is similar to |loop|, \iter\ offers a more Lisp-like syntax
and enhanced extensibility.
(For a more complete
comparison of \iter\ with other iteration constructs, see MIT AI Lab Working
Paper No. 324, ~Don't Loop, Iterate.~)
\end{sloppypar}

An \iter\ form consists of the symbol |iter|\footnote{You can also use
|iterate|, but |iter| is preferred because it avoids potential conflicts with
possible future additions to Common Lisp, and because it saves
horizontal space when writing code.}
followed by one or more
forms, some of which may be \iter\ 
~clauses.~  Here is a simple example of \iter\ which collects the
numbers from 1 to 10 into a list, and returns the list.  The return
value is shown following the arrow.
\begin{program}
(iter (for i from 1 to 10)
      (collect i))          \yields (1 2 3 4 5 6 7 8 9 10)
\end{program}
This form contains two clauses: a |for| clause that steps the
variable |i| over the integers from 1 to 10, and a |collect|
clause that accumulates its argument into a list.  With a few
exceptions, all \iter\ clauses have the same format: alternating
symbols (called ~keywords~) and expressions (called
~arguments~).  The syntax and terminology are those of Common
Lisp's keyword lambda lists.  One difference is that \iter's keywords
do not have to begin with a colon---though they may, except for the
first symbol of a clause.  So you can also write  |(for i :from
1 :to 10)| if you prefer.

Any Lisp form can appear in the body of an \iter, where it will have
its usual meaning.  \iter\ walks the entire body, expanding macros,
and recognizing clauses at any level.  This example collects all the
odd numbers in a list:
\begin{program}
(iter (for el in list)
      (if (and (numberp el) (oddp el))
          (collect el)))
\end{program}

There are clauses for iterating over numbers, lists, arrays and other
objects, and for
collecting, summing, counting, maximizing and other useful operations.
\iter\ also supports the creation of new variable bindings, stepping
over multiple sequences at once, destructuring, and compiler
declarations of variable types.  The following example illustrates
some of these features: 
\begin{program}
(iter (for (key . item) in alist)
         (for i from 0)
         (declare (fixnum i))
         (collect (cons i key)))
\end{program}

This loop takes the keys of an alist and returns a new alist
associating the keys with their positions in the original list. The
compiler declaration for |i| will appear in the generated code
in the appropriate place.

%%%%%%%%%%%%%%%%%%%%%%%%%%%%%%%%%%%%%%%%%%%%%%%%%%%%%%%%%%%%%%%%
\section{Clauses}

Most of \iter's clauses will be familiar to |loop| programmers.  
(|loop| is an iteration macro that has been incorporated into Common
Lisp.  Seen Guy Steele's {\em Common Lisp, 2nd Edition}.)
In
nearly all cases they behave the same as their |loop| counterparts, so
a |loop| user can switch to \iter\ with little pain (and much gain).

All clauses with the standard keyword-argument syntax consist of two
parts: a ~required~ part, 
containing keywords that must be present and in the right order; and
an ~optional~ part, containing keywords that may be omitted and,
if present, may occur in any order.  In the descriptions below, the
parts are separated by the Lisp lambda-list keyword \opt.


\subsection{Drivers}

An iteration-driving clause
conceptually causes the iteration to go forward.  Driver clauses in
\iter\ allow iteration over numbers, lists, vectors, hashtables, packages,
files and streams.  Iteration-driving clauses must
appear at the top level of an \iter\ form; they cannot be nested
inside another clause.  The driver variable is updated at the point
where the driver clause occurs.  Before the clause is executed for the
first time, the value of the variable is undefined.

%Also, regardless of where the driver clause
%appears in the body, the driver variable is stepped at the top of the
%loop; hence it is stylistically preferable, though
%not required, to place driver clauses at the beginning of the \iter.

Multiple drivers may appear in a single \iter\ form, in which case all
of the driver variables are updated each time through the loop, in the
order in which the clauses appear.  The first driver to terminate will
terminate the entire loop.

In all cases, the value of the driver variable on exit from the loop,
including within the epilogue code (see the |finally| clause), is
undefined.

All the parameters of a driver clause are evaluated once, before the
loop begins.  Hence it is not possible to change the bounds or other
properties of an iteration by side-effect from within the loop.

With one exception, driver clauses begin with the word |for| (or
the synonym |as|) and mention an iteration variable, which is
given a binding within the \iter\ form.
The exception is |repeat|, which just executes a loop a
specified number of times:

\begin{clauses}

\clausex{repeat}{~n~}
Repeats the loop ~n~ times.  For example:
\begin{program}
(iter (repeat 100)
      (print "I will not talk in class."))
\end{program}
\cpar If $n \leq 0$, the
loop will never be executed.  If ~n~ is not an integer, the
actual number of executions will be $\lceil n \rceil$.

\end{clauses}

\pagebreak[3]

\subsubsection{Numerical Iteration}

\begin{clauses}
\clausex{for}{~var~ |\&sequence|} 
The general form for iterating over a sequence of numbers requires a
variable and, optionally, one or more keywords that provide the bounds
and step size of the iteration. The |\&sequence|
lambda-list keyword is a shorthand for these sequence keywords.  They are:
|from|, |upfrom|, |downfrom|, |to|, |downto|, |above|, |below| and |by|.
|from| provides the starting value for ~var~
and defaults to zero. |to| provides a final value and implies that the
successive values of ~var~ will be increasing; |downto| implies that
they will be decreasing.  The loop terminates when ~var~ passes
the final value (i.e. becomes smaller or larger than it, depending on
the direction of iteration); in other  words, the loop body
will never be 
executed for values of ~var~ past the final value.
|below| and |above| are similar to |to| and
|downto|, except that the loop terminates when ~var~ equals or passes
the final value.

\cpar If no final value is specified, the variable will be stepped
forever.  Using |from| or |upfrom| will result in increasing
values, while |downfrom| will give decreasing values.

\cpar On each iteration,
~var~ is incremented or decremented by the value of the sequence
keyword |by|, which defaults to 1.  It should always be a positive
number, even for downward iterations.

\cpar In the following examples, the sequence of numbers generated is shown
next to the clause.

\begin{program}
(for i upfrom 0) \yields\ 0 1 2 \ldots
(for i from 5)   \yields\ 5 6 7 \ldots    ; either from or upfrom is okay
(for i downfrom 0) \yields\ 0 -1 -2 \ldots
(for i from 1 to 3) \yields\ 1 2 3
(for i from 1 below 3) \yields\ 1 2
(for i from 1 to 3 by 2) \yields\ 1 3
(for i from 1 below 3 by 2) \yields\ 1
(for i from 5 downto 3) \yields\ 5 4 3
\end{program}

\end{clauses}

\subsubsection{Sequence Iteration}

There are a number of clauses for iterating over sequences.  In all of
them, the argument following |for| may be a list instead of a
symbol, in which case destructuring is performed.  See section
\ref{destructuring}. 

\begin{clauses}

\clause{|for| ~var~ |in| ~list~ \opt\ |by| ~step-function~}
\clindexx{for}{in}
~var~ is set to successive elements of list.  
~step-function,~ which 
defaults to |cdr|, is used to obtain the next sublist. 

\clause{|for| ~var~ |on| ~list~ \opt\ |by| ~step-function~}
\clindexx{for}{on}
~var~ is set to successive sublists of list. 
~step-function~ (default |cdr|) is used as in |for\dots in|.

\end{clauses}

\medskip

These two clauses use |atom| to test for the end
of a list.  Hence, given a list whose final |cdr| is not \nil,
they will silently ignore the last |cdr|.  Other choices are
|endp|,  which would signal an error, and |null|, which
would probably result in an error somewhere else.  If you wish to use
an end-test other than |atom|, set the variable
|iterate::*list-end-test*|\lispindex{*list-end-test*} to the name of
the desired function. 

\begin{clauses}
\clause{|for| ~var~ |in-vector| ~vector~ |\&sequence|}
\clindexx{for}{in-vector}
~var~ takes on successive elements from ~vector.~   The vector's
fill-pointer is observed.  Here and in subsequent clauses, the |\&sequence|
keywords include |with-index|, which takes a symbol as argument
and uses it for the index variable instead of an internally generated
symbol.  The other |\&sequence| keywords behave as in numerical
iteration, except that the default iteration bounds are the bounds of
the vector.  E.g. in |(for i in-vector v downto 3)|,
|i| will start off being bound to the last element in |v|,
and will  
be set to preceding elements down to and including the element with
index 3.

\clause{|for| ~var~ |in-sequence| ~seq~ |\&sequence|}
\clindexx{for}{in-sequence}
This uses Common Lisp's generalized sequence functions, |elt|
and |length|, to obtain elements and determine the length of
~seq.~   Hence it will work for any sequence, including lists, and will
observe the fill-pointers of vectors.

\clause{|for| ~var~ |in-string| ~string~ |\&sequence|}
\clindexx{for}{in-string}
~var~ is set to successive characters of ~string.~

\startitem
\Clause{|for| ~var~ |index-of-vector| ~vector~ |\&sequence|}
\clindexx{for}{index-of-vector}
\Clause{|for| ~var~ |index-of-sequence| ~sequence~ |\&sequence|}
\clindexx{for}{index-of-sequence}
\Clause{|for| ~var~ |index-of-string| ~string~ |\&sequence|}
\clindexx{for}{index-of-string}
\finishitem
~var~ is set to successive indices of the sequence.
These clauses avoid the overhead of accessing the sequence elements
for those applications where they do not need to be examined, or are
examined rarely.  They admit all the optional keywords of the other
sequence drivers except the (redundant) |with-index| keyword.

\clause{|for| ~(key value)~ |in-hashtable| ~table~}
\clindexx{for}{in-hashtable}

~key~ and ~value,~ which must appear as shown in a list
and may be destructuring templates, are set to the keys and values of
~table~.  If ~key~ is |nil|, then the hashtable's keys will be ignored;
similarly for ~value~.
The order in which 
elements of ~table~ will be retrieved is unpredictable.

\clause{|for| ~var~ |in-package| ~package~ \opt\ |external-only| ~ext~}
\clindexx{for}{in-package}
Iterates over all the symbols in ~package,~ or over only the
external symbols if ~ext~ is specified and non-|nil|.  ~ext~ is not
evaluated.  The same symbol may appear more than once.

\clause{|for| ~var~ |in-packages| ~packages~ \opt\ |having-access| ~symbol-types~}
\clindexx{for}{in-packages}
Iterates over all the symbols from the list of packages denoted by the
descriptor ~packages~ and having accessibility (or visibility) given by
~symbol-types~.  This defaults to the list |(:external :internal :inherited)|
and is not evaluated.  ~var~ must be a list of up to three variables: in each
iteration, these will be set to a symbol, its access-type and package (as per
|with-package-iterator| in ANSI-CL).  The same symbol may appear more than
once.

\end{clauses}

\begin{clauses}

\clause{|for| ~var~ |in-file| ~name~ \opt\ |using| ~reader~}
\clindexx{for}{in-file}
Opens the file ~name~ (which may be a string or pathname) for input,
and iterates 
over its contents. ~reader~ defaults to |read|, so by default ~var~
will be bound to the successive forms in the file.  The \iter\ body is
wrapped in an unwind-protect to ensure that the file is closed no
matter how the \iter\ is exited.

\clause{|for| ~var~ |in-stream| ~stream~ \opt\ |using| ~reader~}
\clindexx{for}{in-stream}
Like |for\dots in-file|, except that ~stream~ should be an existing
stream object that supports input operations.

\end{clauses}

\subsubsection{Generalized Drivers}

These are primarily useful for writing drivers that can also be used
as generators (see section \ref{generators}, below).
 
\begin{clauses}

\clause{|for| ~var~ |next| ~expr~}
\clindexx{for}{next}

~var~ is set to ~expr~ each time through the loop.  Destructuring is
performed.  When the clause is used as a generator, ~expr~ is the code
that is executed when |(next ~var~)| is 
encountered (see section \ref{generators}, below).
~expr~ should compute the first value for ~var~, as well
as all subsequent values, and is responsible for terminating the loop.
For compatibility with future versions of \iter, this termination
should be done with |terminate|\clindex{terminate}, which can be
considered a synonym for |finish| (see section \ref{control-flow}).

\cpar As an example, the following clauses are equivalent to |(for
i from 1 to 10)|:
\begin{program}
(initially (setq i 0))
(for i next (if (> i 10) (terminate) (incf i)))
\end{program}

\clause{|for| ~var~ |do-next| ~form~}
\clindexx{for}{do-next}
~form~ is evaluated each time through the loop.  Its value is ~not~
set to ~var~; that is ~form's~ job. ~var~ is only present so that
\iter\ knows it is a driver variable.  \linebreak |(for ~var~ next
~expr~)| is equivalent to |(for ~var~ do-next (dsetq ~var~ ~expr~))|.
(See section \ref{destructuring} for an explanation of |dsetq|.) 

\end{clauses}

\subsubsection{Generators}
\label{generators}

In all of the above clauses, the driver variable is updated on each
iteration.
Sometimes it is desirable to have greater control over updating.  
For instance, consider the problem of associating numbers, in
increasing order and with no gaps, with the
\nonnil\ elements of a list.  One obvious first pass at writing this is:

\begin{program}
(iter (for el in list)
      (for i upfrom 1)
      (if el (collect (cons el i))))
\end{program}
But on the list |(a b nil c)| this produces |((a . 1) (b . 2) (c .
4))| instead of the desired |((a . 1) (b . 2) (c . 3))|.  The problem
is that |i| is incremented each time through the loop, even when |el|
is |nil|.  

The problem could be solved elegantly if we could step |i| only when
we wished to.  This
can be accomplished for any \iter\ driver by writing |generate| (or
its synonym |generating|)
instead of |for|.  Doing so produces a ~generator~---a driver whose
values are yielded explicitly.  To obtain the next value of a
generator variable ~v~, write \linebreak |(next ~v~)|.  The value of a |next|
form is the next value of ~v~, as determined by its associated driver
clause.  |next| also has the side-effect of updating ~v~ to that
value.  If there is no next value, |next| will terminate the loop,
just as with a normal driver.


Using generators, we can now write our example like this:
\begin{program}
(iter (for el in list)
      (generate i upfrom 1)
      (if el (collect (cons el (next i)))))
\end{program}
Now |i| is updated only when |(next i)| is executed, and this occurs
only when |el| is \nonnil.

To better understand the relationship between ordinary drivers and
generators, observe that we can rewrite an ordinary driver using its
generator form immediately followed by |next|, as this example shows:
\begin{program}
(iter (generating i from 1 to 10)
      (next i)
      ...)
\end{program}
Provided that the loop body contains no |(next i)| forms, this will
behave just as if we had written |(for i from 1 to 10)|.

We can still refer to a driver variable ~v~ without using |next|; in
this case, its value is that given to it by the last evaluation of
|(next ~v~)|.  Before |(next ~v~)| has been called the first time, the
value of ~v~ is undefined.  

This semantics is more flexible than
one in which ~v~ begins the loop bound to its first value and calls
of |next| supply subsequent values, because it means the loop will not
terminate too soon if the generator's sequence is empty.  For
instance, consider the following code, which tags \nonnil\ elements of
a list using a list of tags, and also counts the null elements.
(We assume there are at least as many tags as \nonnil\ elements.)
\begin{program}
(let* ((counter 0)
       (tagged-list (iter (for el in list)
                          (generating tag in tag-list) 
                          (if (null el)
                              (incf counter)
                              (collect (cons el (next tag)))))))
  ...)
\end{program}
It may be that there are just as many tags as non-null elements of
|list|.  If all the elements of |list| are null, we still want the
counting to proceed, even though |tag-list| is \nil.  If |tag| had to be
assigned its first value before the loop begins, we would have had to
terminate the loop before the first iteration, since when |tag-list|
is \nil, |tag| 
has no first value.  With the existing semantics, however, |(next
tag)| will never execute, so the iteration will cover all the elements of
|list|. 
        
When the ``variable'' of a driver clause is actually a destructuring
template containing several variables, all the variables are eligible
for use with |next|.  As before, |(next ~v~)| evaluates to ~v's~ next
value; but the effect is to update all of the template's variables.
For instance, the following code will return the list |(a 2 c)|.
\begin{program}
(iter (generating (key . item) in '((a . 1) (b . 2) (c . 3)))
      (collect (next key))
      (collect (next item)))
\end{program}

Only driver clauses with variables can be made into generators.  This
includes all clauses mentioned so far except for |repeat|.  It does
~not~ include |for\dots previous|, |for\dots =|,
|for\dots initially\dots then| or |for\dots first\dots then| (see
below). 

\subsubsection{Previous Values of Driver Variables}

Often one would like to access the value of a variable on a previous
iteration.  \iter\ provides a special clause for accomplishing this.

\begin{clauses}

\clause{|for| ~pvar~ |previous| ~var~ \opt\ |initially| ~init~
|back| ~n~}
\clindexx{for}{previous}
Sets ~pvar~ to the previous value of ~var,~ which should be a driver
variable, a variable from another |for\dots previous| clause, or a
variable established by a |for\dots =|, 
|for\dots initially\dots then| or |for\dots first\dots then| clause
(see section \ref{setting}).
Initially, ~pvar~ is given the value ~init~ (which defaults to \nil).  
The ~init~ expression will be moved outside the loop body, so it
should not depend on anything computed within the loop.
~pvar~ retains the value of ~init~ until ~var~ is set to its second
value, at which point ~pvar~ is set to ~var's~ first value; and so on.  

\cpar The
argument ~n~ to |back| must be a constant, positive integer, and
defaults to 1.  It determines how many iterations back ~pvar~ should
track ~var~.  For example, when ~n~ is 2, then ~pvar~ will be assigned
~var's~ first value when ~var~ is set to its third value.

\cpar A |for\dots previous| clause may occur after or before its
associated driver clause. |for\dots previous| works with generators as
well as ordinary drivers.

\pagebreak[3]

\cpar Example:
\begin{program}
(iter (for el in '(1 2 3 4))
      (for p-el previous el)
      (for pp-el previous p-el initially 0)
      (collect pp-el))
\end{program}
This evaluates to |(0 0 1 2)|.  It could have been written more
economically as 
\begin{program}
(iter (for el in '(1 2 3 4))
      (for pp-el previous el back 2 initially 0)
      (collect pp-el))
\end{program}


\end{clauses}

\subsection{Variable Binding and Setting}
\label{setting}

Several clauses exist for establishing new variable bindings or for
setting variables in the loop.  They all support destructuring.

\begin{clauses}

\clausex{with}{~var~ \opt\ |=| ~value~}
Causes ~var~ to be bound to value before the loop body is entered.
If ~value~ is not supplied, ~var~ assumes a default
binding, which will be 
\nil\ in the absence of declarations.  Also, if ~value~ is not
supplied, no destructuring is performed; instead, ~var~ may be a list of
symbols, all of which are given default 
bindings.  If ~value~ is supplied, ~var~ is bound to it, with
destructuring.  

\cpar Because |with|
creates bindings whose scope includes the entire \iter\ form, it is
good style to put all |with| clauses at the beginning.

\cpar Successive occurrences of |with| result in sequential
bindings (as with 
|let*|).  There is no way to obtain parallel bindings; see
section \ref{bindings} for a rationale.  


\clause{|for| ~var~ |=| ~expr~}
\clindexx{for}{=}
On each iteration, ~expr~ is evaluated and ~var~ is set
to its value. 

\cpar This clause may appear to do the same thing as |for\dots next|.
In fact, they are quite different.  |for\dots =| provides only three
services: it sets up a binding for ~var~, sets it to ~expr~ on each
iteration, and makes it possible to use |for\dots previous| with
~var~.  |for\dots next| provides these services in addition to the
ability to turn the driver into a generator.
%Also, the code which sets ~var~ appears in the loop body in the same
%place as the |for\dots =| clause; the code for |for\dots next| appears
%at the top of the loop, as with other drivers (except when being used
%as a generator).

\clause{|for| ~var~ |initially| ~init-expr~ |then| ~then-expr~}
\clindexxx{for}{initially}{then}
Before the loop begins, ~var~ is set to ~init-expr;~ on all
iterations after the first it is set to ~then-expr.~  This clause
must occur at top-level.  ~init-expr~ will be moved outside the loop
body and ~then-expr~ will be moved to the end of the loop body, so
they are subject to code motion problems (see section
\ref{code-movement}). 

\cpar This clause may appear to be similar to |for\dots next|, but in
fact they differ significantly.  |for\dots initially\dots then| is
typically used to give ~var~ its first value before the loop begins,
and subsequent values on following iterations.  This is incompatible
with generators, whose first value and subsequent values must all be
computed by |(next ~var~)|.  Also, the update of ~var~ in 
|for\dots initially\dots then| does not occur at the location of the
clause. 
Use |for\dots initially\dots then| for
one-shot computations where its idiom is more convenient, but use
|for\dots next| for extending \iter\ with new drivers (see section
\ref{extend}).

\clause{|for| ~var~ |first| ~first-expr~ |then| ~then-expr~}
\clindexxx{for}{first}{then}
The first time through the loop, ~var~ is set to ~first-expr;~ on
subsequent iterations, it is set to ~then-expr.~  This differs from
|for\dots initially| in that ~var~ is set to ~first-expr~
inside the loop body, 
so ~first-expr~ may depend on the results of other clauses.  For
instance, 
\begin{program}
(iter (for num in list)
      (for i first num then (1+ i))
      ...)
\end{program}
will set |i| to the first element of |list| on the first
iteration, whereas 
\begin{program}
(iter (for num in list)
      (for i initially num then (1+ i))
      ...)
\end{program}
is probably erroneous; |i| will be bound to |num|'s
default binding (usually \nil) for the first iteration.

\end{clauses}

\begin{note}{Compatibility note}
|loop|'s |for\dots =| works like \iter's, but |loop| used the syntax
|for\dots =\dots then| to mean |for\dots initially\dots then|.  It was
felt that these two operations were sufficiently different to warrant
different keywords.

Also, the |for| in the above three clauses is misleading,
since none is true driver (e.g. none has a corresponding |generate|
form).  |setting| would have been a better choice, but |for| was used
to retain some compatibility with |loop|.  
\end{note}


%%%%%%%%%%%%%%%%%%%%%%%%%%%%%%%%
\subsection{Gathering Clauses}
Many of \iter's clauses accumulate values into a variable, or set a
variable under certain conditions.  At the end of the loop, this
variable contains the desired result.  All these clauses have an
optional |into| keyword, whose argument should be a symbol.  If the
|into| keyword is not supplied, the accumulation variable will be
internally generated and its value will be returned at the end of the
loop; if a variable is specified, that variable is used for the
accumulation, and is not returned as a result---it is up to the user
to return it explicitly, in the loop's epilogue code (see |finally|).
It is safe to examine the accumulation variable during the loop, but
it should not be modified.

These clauses all begin with a verb.  When the verb does
not conflict with an existing Common Lisp function, then it may be
used in either its infinitival or present-participle form (e.g. |sum|,
|summing|).  However, when there is a conflict with Common Lisp, only
the present-participle form may be used (e.g. |unioning|).  This is to
prevent \iter\ clauses from clashing with Common Lisp
functions. 

%although these clauses are described as ``producing a value,'' it is a
%mistake to think of the lisp list representing the clause as a
%value-producing form in the usual way.  clauses may legally be written
%where a value is expected, e.g. |(setq x (sum i))|, but the
%lisp value of a clause in such a context is undefined.

\subsubsection{Reductions}

~Reduction~ is an extremely common iteration pattern in which
the results of successive applications of a
binary operation are accumulated.
For example, a loop that computes the sum of the
elements of a list is performing a reduction with the addition
operation.  This could be written in Common Lisp as  |(reduce \#'+
list)| or with \iter\ as
\begin{program}
(iter (for el in list)
      (sum el))
\end{program}

\begin{clauses}

\clausex{sum}{~expr~ \opt\ |into| ~var~}
Each time through the loop, ~expr~ is evaluated and added to a
variable, which is bound initially to zero.  If ~expr~ has a type,
it is ~not~ used as the type of the sum variable, which is always
|number|.  To get the result variable to be of a more specific
type, use an explicit variable, as in
\begin{program}
(iter (for el in number-list)
      (sum el into x)
      (declare (fixnum x))
      (finally (return x)))
\end{program}

\clausex{multiply}{~expr~ \opt\ |into| ~var~}
Like |sum|, but the initial value of the result variable is $1$,
and the variable is updated by multiplying ~expr~ into it.

\clausex{counting}{~expr~ \opt\ |into| ~var~}
~expr~ is evaluated on each iteration.  If it is \nonnil, the
accumulation variable, initially zero, is incremented.

\startitem
\Clausex{maximize}{~expr~ \opt\ |into| ~var~}
\Clausex{minimize}{~expr~ \opt\ |into| ~var~}
\finishitem
~expr~ is evaluated on each iteration and its extremum (maximum or
minimum) is stored in the accumulation variable.  If ~expr~ is never
evaluated, then the result is \nil\ (if the accumulation variable is
untyped) or $0$ (if it has a numeric type).

\clausex{reducing}{~expr~ |by| ~func~ \opt\ |initial-value| ~init-val~ |into| ~var~}
This is a general way to perform reductions. ~func~ should be a
function of two arguments, the first of which will be the value
computed so far and
the second of which will be the value of ~expr~.  It should return the
new value.
|reducing| is roughly equivalent to the
Common Lisp |(reduce ~func~ ~sequence~ :key ~expr-function~)|, where
~expr-function~ is used to derive values from the successive elements
of ~sequence~.

\cpar If the |reducing| clause is never executed, the result is
undefined.

\cpar It is not necessary to provide an
initial value, but better code can be generated if one is supplied.
Regardless of its location in the \iter\ body, ~init-val~ will be
evaluated before the loop is entered, so it should not depend on any
value computed inside the \iter\ form.  

%\cpar if a ~var~ is not specified, you can get \iter\ to declare the
%type of the internal variable by putting a |the| expression around
%~func.~  see section \ref{types}.

\end{clauses}

%\begin{note}{implementation note}
%in principle, |maximize| and |minimize| can be thought of
%as reductions where the initial value is the smallest (or largest)
%value that the accumulation variable can assume.  because lisp's
%bignums can represent arbitrary integers, these clauses cannot be
%implemented as reductions in general.  if, however, the type of 
%~expr~ or ~var~ can be determined to be a fixnum
%or a float, \iter\ will implement the clause as a true reduction,
%using one of the constants |most-negative-fixnum|, |%most-positive-fixnum|, 
%|most-negative-short-float|, etc. as appropriate.
%\end{note}

\subsubsection{Accumulations}

All the predefined accumulation clauses add values to a sequence.  If
the sequence is a list, they all 
have the property that the partial list is kept in the correct order
and available for inspection at any point in the loop.

\begin{clauses}

\clausex{collect}{~expr~ \opt\ |into| ~var~ |at| ~place~ |result-type| ~type~}
Produces a sequence of the values of ~expr~ on each iteration. ~place~
indicates where the next value of ~expr~ is added to the list and may
be one of the symbols |start|, |beginning| (a synonym for |start|) or
|end|.  The symbol may be quoted, but need not be.  The default is
|end|.  For example,
\begin{program}
(iter (for i from 1 to 5)
      (collect i))
\end{program}
produces |(1 2 3 4 5)|, whereas
\begin{program}
(iter (for i from 1 to 5)
      (collect i at beginning))
\end{program}
produces |(5 4 3 2 1)| (and is likely to be faster in most Common Lisp
implementations).

\cpar If ~type~ is provided, it should be a subtype of |sequence|.
The default is |list|.  Specifying a type other than |list| will
result in |collect| returning a sequence of that type.  ~However,~ the
type of the sequence being constructed when inside the loop body is
undefined when a non-|list| type is specified.  (As with ~place~,
quoting ~type~ is optional.)

\clausex{adjoining}{~expr~ \opt\ |into| ~var~ |test| ~test~ |at| ~place~ |result-type| ~type~}
Like |collect|, but only adds the value of ~expr~ if it is not
already present.  ~test,~ which defaults to |\#'eql|, is
the test to be used with |member|.

\startitem
\Clausex{appending}{~expr~ \opt\ |into| ~var~ |at| ~place~}
\Clausex{nconcing}{\ ~expr~ \opt\ |into| ~var~ |at| ~place~}
\Clausex{unioning}{\ ~expr~ \opt\ |into| ~var~ |test| ~test~ |at| ~place~}
\Clausex{nunioning}{~expr~ \opt\ |into| ~var~ |test| ~test~ |at| ~place~}
\finishitem
These are like |collect|, but behave like the Common Lisp functions
|append|, |nconc|, |union| or |nunion|.  
As in Common Lisp, they work only on lists.  Also as in Common Lisp,
|unioning| and |nunioning| assume that the value of ~expr~ contains no
duplicates. 

\clausex{accumulate}{~expr~ |by| ~func~ \opt\ |initial-value| ~init-val~ |into| ~var~}
This is a general-purpose accumulation clause. ~func~ should be a
function of two arguments, the value of ~expr~ and the value
accumulated so far in the iteration, and it should return the updated
value.  If no initial value is supplied, \nil\ is used.  

%\cpar If a ~var~ is not specified, you can get \iter\ to declare the
%type of the internal variable by putting a |the| expression around
%~func.~  see section \ref{types}.

\cpar The differences between |accumulate| and |reducing| are slight.
One difference is that the functions take their arguments in a
different order.  Another is that in the absence of ~init-val~,
|accumulate| will use \nil, whereas |reducing| will generate different
code that avoids any dependence on the initial value.
The reason for having both clauses is that one usually
thinks of reductions (like |sum|) and accumulations (like |collect|)
as different beasts.

\end{clauses}


\subsubsection{Finders}

A ~finder~ is a clause whose value is an expression that meets some
condition.

\begin{clauses}

\clause{|finding| ~expr~ |such-that| ~test~ \opt\ |into| ~var~ |on-failure| ~failure-value~}
\clindexx{finding}{such-that}
If ~test~ (which is an expression) ever evaluates to \nonnil, the loop
is terminated, the 
epilogue code is run and the value of ~expr~ is returned.  Otherwise,
\nil\ (or ~failure-value,~ if provided) is returned.  If ~var~ is
provided, it will have either the \nonnil\ value of ~expr~ or
~failure-value~ when the epilogue code is run.

\cpar As a special case, if the ~test~ expression is a sharp-quoted
function, then it is applied to ~expr~ instead of being simply
evaluated.  E.g. |(finding x such-that \#'evenp)| is equivalent to
|(finding x such-that (evenp x))|.  

%\cpar although ~test~ need have nothing to do with ~%expr~ as in
%|(finding j such-that (> i 3))|, it usually 
%will: |(finding (length el) such-that (oddp (length el)))|.  to
%avoid performing the |length| computation twice, you could write
%|(finding (length el) such-that \#'oddp)| or |(finding (length
%el) such-that 'oddp)|; for these cases, \iter\ generates code that
%executes ~expr~ only once.  the code for |\#'oddp|
%is slightly different from that for {\lisp 'oddp}; see the discussion
%under {\lisp for\dots in} and {\lisp for\dots on}.

\cpar |On-failure| is a misnomer. Because it is always evaluated, it behaves
more like the default third argument to the |gethash| function. As a result,
|on-failure (error "Not found")| makes no sense. Instead, the clauses |leave|
or |thereis| can be used in conjunction with |finally| as follows:
\begin{program}
(iter (for x in '(1 2 3))
      (if (evenp x) (leave x))
      (finally (error "not found")))
\end{program}

\cpar This clause may appear multiple times when all defaults are
identical. It can also be used together with either |always|/|never| or
|thereis| if their defaults match. More specifically, |on-failure nil| is
compatible with |thereis|, while |on-failure t| is compatible with |always|
and |never| clauses.
\begin{program}
(iter (for i in '(7 -4 2 -3))
      (if (plusp i)
	  (finding    i  such-that (evenp i))
          (finding (- i) such-that (oddp i))))
\end{program}

\startitem
\Clause{|finding| ~expr~ |maximizing| ~m-expr~ \opt\ |into| ~var~}
\clindexx{finding}{maximizing}
\Clause{|finding| ~expr~ |minimizing| ~m-expr~ \opt\ |into| ~var~}
\clindexx{finding}{minimizing}
\finishitem
Computes the extremum (maximum or minimum) value of ~m-expr~ over all
iterations, and returns the value of ~expr~ corresponding to the
extremum.  ~expr~ is evaluated inside the loop at the time the new
extremum is established.  If ~m-expr~ is never evaluated (due to, for
example, being embedded in a conditional clause), then the returned
value depends on the type, if any, of ~expr~ (or ~var,~ if
one is supplied).  If there is no type, the returned
value will be \nil; if the type is numeric, the returned value will be
zero.

\cpar For these two clauses, ~var~ may be a list of two
symbols; in that case, the first is used to record ~expr~ and
the second, ~m-expr.~  

\cpar As with |finding\dots such-that|, if ~m-expr~ is a sharp-quoted
function, then it is called on ~expr~ instead of being evaluated.

\end{clauses}

\subsubsection{Boolean Tests}

\begin{clauses}

\clausex{first-iteration-p}{}
Returns |t| in the first cycle of the loop, otherwise \nil.

\clausex{first-time-p}{}
Returns |t| the first time the expression is evaluated, and then \nil\ forever.
This clause comes handy when printing (optional) elements separated
by a comma:

\begin{program}
(iter (for el in '(nil 1 2 nil 3))
      (when el
        (unless (first-time-p)
          (princ ", "))
        (princ el)))
\end{program}
produces |"1, 2, 3"|.

\end{clauses}

\subsubsection{Aggregated Boolean Tests}

\begin{clauses}

\clausex{always}{~expr~}
If ~expr~ ever evaluates to
\nil, then \nil\ is immediately returned; the epilogue code is not
executed.  If ~expr~ never evaluates to \nil, the epilogue code
is executed and the last value of ~expr~ (or |t| if ~expr~ was never
evaluated) is returned (whereas |loop| would constantly return |t|).

% mention last evaluated clause when multiple always clauses?

\clausex{never}{~expr~}
Like |(always (not ~expr~))|, except it does not influence the last
value returned by a possible other |always| clause. That is,
\begin{program}
(iter (repeat 2)
      (always 2)
      (never nil)) \yields 2 ; not t
\end{program}

\clausex{thereis}{~expr~}
If ~expr~ is ever \nonnil,
its value is immediately returned without running epilogue code.
Otherwise, the epilogue code is performed and \nil\ is returned.

This clause cannot be used together with |always| or |never|, because their
defaults are opposed (similarly, |(loop always 3 thereis nil)| refuses to
compile in some implementations of |loop|).

\end{clauses}


\subsection{Control Flow}
\label{control-flow}
Several clauses can be used to alter the usual flow of control in a loop.

Note: the clauses of this and subsequent sections don't adhere to \iter's
usual syntax, but instead use standard Common Lisp syntax.  Hence the
format for describing syntax subsequently is
like the standard format used in the Common Lisp manual, not like the
descriptions of clauses above.

\begin{clauses}

\clausex{finish}{}
Stops the loop and runs the epilogue code.

%for example:
%
%\begin{program}
%(iter (with answer = nil)
%         (initially (make-a-mess))
%         (for i from 1 to 10)
%         (when (correct? i)
%           (setq answer i)
%           (finish))
%         (finally (cleanup)))
%\end{program}
%
%this code will execute |cleanup| whether or not the test |(correct?
%i)| ever succeeds. 
%the (more elegant) formulation,
%\begin{program}
%(iter (initially (make-a-mess))
%         (for i from 1 to 10)
%         (finding i such-that (correct? i))
%         (finally (cleanup)))
%\end{program}
%would not execute |cleanup| if |(correct? i)| succeeded; it
%would do an immediate return.

\clausex{leave}{\opt\ ~value~}
Immediately returns ~value~ (default \nil) from the current \iter\
form, skipping the epilogue code.  Equivalent to using |return-from|.

\clausex{next-iteration}{}
Skips the remainder of the loop body and begins the next iteration of
the loop.

\clausex{while}{~expr~}
If ~expr~ ever evaluates to \nil, the loop is terminated and the
epilogue code executed.  Equivalent to |(if (not ~expr~) (finish))|.

\clausex{until}{~expr~}
Equivalent to |(if ~expr~ (finish))|.

\clausex{if-first-time}{~then~ \opt\ ~else~}
If this clause is being executed for the first time in this invocation
of the \iter\ form, then the ~then~ code is evaluated; otherwise the
~else~ code is evaluated.

\cpar |(for ~var~ first ~expr1~ then ~expr2~)| is almost equivalent to
\begin{program}
(if-first-time (dsetq ~var~ ~expr1~)
               (dsetq ~var~ ~expr2~))
\end{program}
The only difference is that the |for| version makes ~var~ available
for use with |for\dots previous|.

\end{clauses}

\subsection{Code Placement}
When fine control is desired over where code appears in a loop
generated by \iter, the following special clauses may be useful.
They are all subject to code-motion problems (see section
\ref{code-movement}).  

\begin{clauses}

\clausex{initially}{|\&rest| ~forms~}
The lisp ~forms~ are placed in the prologue section of the loop, where
they are executed once, before the loop body is entered.

\clausex{after-each}{|\&rest| ~forms~}
The ~forms~ are placed at the end of the loop body, where they
are executed after each iteration.  Unlike the other clauses in this
section, ~forms~ may contain \iter\ clauses.

\clausex{else}{|\&rest| ~forms~}
The lisp ~forms~ are placed in the epilogue section of the loop, where they
are executed if this |else| clause is never met during execution of the
loop and the loop terminates normally.

\clausex{finally}{|\&rest| ~forms~}
The lisp ~forms~ are placed in the epilogue section of the loop, where
they are executed after the loop has terminated normally.

\clausex{finally-protected}{|\&rest| ~forms~}
The lisp ~forms~ are placed in the second form of an unwind-protect
outside the loop.  They are always executed after the loop has
terminated, regardless of how the termination occurred.

\end{clauses}


%%%%%%%%%%%%%%%%%%%%%%%%%%%%%%%%%%%%%%%%%%%%%%%%%%%%%%%%%%%%%%%%
\section{Other Features}

\subsection{Multiple Accumulations}
\label{multiple}

\begin{sloppypar}
It is permitted to have more than one clause accumulate into the same
variable, as in the following:
\begin{program}
(iter (for i from 1 to 10)
      (collect i into nums)
      (collect (sqrt i) into nums)
      (finally (return nums)))
\end{program}
Clauses can only accumulate into the same variable if they are
compatible.  |collect|, |adjoining|, |appending|, |nconcing|,
|unioning| and |nunioning| are compatible with each other; |sum|,
|multiply| and |counting| are compatible; |always| and |never| are compatible;
|finding| \dots |such-that| is compatible with either |thereis| or |always|
and |never| when their defaults match; and |maximize| and |minimize| clauses
are compatible only with other |maximize| and |minimize| clauses,
respectively.

%note that the same variable ~cannot~ be both an accumulation
%variable and an ordinary variable; there can be only one variable with
%a given name within an \iter\ form. 
\end{sloppypar}

\subsection{Named Blocks}

Like Common Lisp |block|s, \iter\ forms can be given names.  The
name should be a single symbol, and it must be the first form in the
\iter.  The generated code behaves exactly like a named block; in
particular, |(return-from ~name~)| can be used to exit it:
\begin{program}
(iter fred
      (for i from 1 to 10)
      (iter barney
            (for j from i to 10)
            (if (> (* i j) 17)
                (return-from fred j))))
\end{program}
An \iter\ form that is not given a name is implicitly named \nil.

Sometimes one would like to write an expression in an inner \iter\ form,
but have it processed by an outer \iter\ form.  This is possible with
the |in| clause.  

\begin{clauses}

\clausex{in}{~name~ |\&rest| ~forms~}
Evaluates ~forms~ as if they were part of the \iter\ form named
~name~.  In other words, \iter\ clauses are processed by the \iter\
form named ~name,~ and not by any \iter\ forms that occur inside ~name.~

\cpar As an example, consider the problem of collecting a list of the
elements in a two-dimensional array.  The naive solution,
\begin{program}
(iter (for i below (array-dimension ar 0))
      (iter (for j below (array-dimension ar 1))
            (collect (aref ar i j))))
\end{program}
\noindent is wrong because the list created by the inner \iter\ is simply
ignored by the outer one.  But using |in| we can write:
\begin{program}
(iter outer (for i below (array-dimension ar 0))
      (iter (for j below (array-dimension ar 1))
            (in outer (collect (aref ar i j)))))
\end{program}
\noindent which has the desired result.

\end{clauses}

\subsection{Destructuring}
\label{destructuring}

In many places within \iter\ clauses where a variable is expected, a
list can be written instead.  In these cases, the value to be assigned
is ~destructured~ according to the pattern
described by the list.  As a simple example, the clause
\begin{program}
(for (key . item) in alist)
\end{program}
\noindent will result in |key| being set to the |car| of
each element in |alist|, and |item| being set to the |cdr|.  The
pattern list may be nested to arbitrary depth, and (as the example
shows) need not be terminated with \nil; the only requirement is that
each leaf be a bindable symbol (or \nil, in which case no binding is
generated for that piece of the structure).

Sometimes, you might like to do the equivalent of a
|multiple-value-setq| in a clause.  This 
``multiple-value destructuring'' can be expressed by writing 
\linebreak |(values $pat_1$ $pat_2 \ldots$)| for a destructuring
pattern, as in 
\begin{program}
(for (values (a . b) c d) = (three-valued-function ...))
\end{program}
\begin{sloppypar}
Note that the $pat_i$ can themselves be destructuring patterns (though
not multiple-value destructuring patterns).  You can't do multiple-value
destructuring in a |with| clause; instead wrap the whole \iter\
form in a |multiple-value-bind|.
\end{sloppypar}

\begin{note}{Rationale}
There are subtle interactions between variable declarations and
evaluation order that make the correct implementation of
multiple-value destructuring in a |with| somewhat tricky.
\end{note}

The destructuring feature of \iter\ is available as a separate
mechanism, using the |dsetq| macro:

\begin{clauses}
\defmacro{dsetq}{template expr}
Performs destructuring of ~expr~ using ~template~.   May be used
outside of an \iter\ form. Yields the primary value of ~expr~.

\end{clauses}

\subsection{On-line Help}

\begin{sloppypar}
There is a limited facility for on-line help, in the form of the 
|display-iterate-clauses| function.
\end{sloppypar}

\begin{clauses}

\defun{display-iterate-clauses}{\opt\ clause-spec}
Displays a list of \iter\ clauses.  If ~clause-spec~ is not
provided, all clauses are shown; if it is a symbol, all clauses
beginning with that symbol are shown; and if it is a list of symbols,
all clauses for which ~clause-spec~ is a prefix are shown.

\end{clauses}

\subsection{Parallel Binding and Stepping}
\label{bindings}

The parallel binding and stepping of variables is a feature that
\iter\ does ~not~ have.  This section attempts to provide a rationale.

We say that two variables are bound ~in parallel~ if neither
binding shadows the other.  This is the usual semantics of |let|
(as opposed to |let*|).  Similarly, we can say that iteration
variables are stepped in parallel if neither variable is updated
before the other, conceptually speaking; in other words, if the code
to update each variable can reference the old values of both variables.

|loop| allows parallel binding of variables and parallel stepping of
driver variables.  My view is that if you are depending on the 
serial/parallel distinction, you are doing something obscure.  If you
need to bind 
variables in parallel using |with|, then you must be using a
variable name that shadows a name in the existing lexical environment.
Don't do that.  The most common use for parallel stepping is to track
the values of variables on the previous iteration, but in fact this
does not require parallel stepping at all; the following will work:
\begin{program}
(iter (for current in list)
      (for prev previous current)
      ...)
\end{program}


%%%%%%%%%%%%%%%%%%%%%%%%%%%%%%%%%%%%%%%%%%%%%%%%%%%%%%%%%%%%%%%%
\section{Types and Declarations}
\label{types}

\subsection{Discussion}

Sometimes efficiency dictates that the types of variables be declared.
This type information needs to be communicated to \iter\ so it can
bind variables to appropriate values.  Furthermore, \iter\ must often
generate internal variables invisible to the user; there needs to be a
way for these to be declared.

As an example, consider this code, which will return the number of
odd elements in |number-list|:
\begin{program}
(iter (for el in number-list)
      (count (oddp el)))
\end{program}
In processing this form,
\iter\ will create an internal variable, let us call it |list17|, to
hold the successive |cdr|s of |number-list|, and  
will bind the variable to |number-list|.  It will also generate a
default binding for |el|; only inside the body of the loop will |el|
be set to the |car| of |list17|.  Finally, \iter\ will generate a
variable, call it |result|, to hold the result of the count, and will
bind it to zero.

When dealing with type declarations, \iter\ observes one simple rule:
~it will never generate a declaration unless requested to do so.~  The
reason is that such declarations might mask errors in compiled code by
avoiding error-checks; the resulting problems would be doubly hard to
track down because the declarations would be hidden from the
programmer.  Of course, a compiler might omit error-checks even in the
absence of declarations, though this behavior can usually be avoided,
e.g. by saying |(declaim (optimize (safety 3)))|.

So, the above \iter\ form will generate code with no declarations.
But say we wish to declare the types of |el| and the internal
variables |list17| and |result|.  How is this done?

Declaring the type of |el| is easy, since the programmer knows
the variable's name:
\begin{program}
(iter (for el in number-list)
      (declare (fixnum el))
      (counting (oddp el)))
\end{program}
\iter\ can read variable type declarations like this one.  Before
processing any clauses, it scans the entire top-level form for type
declarations 
and records the types, so that variable bindings can be performed
correctly.  In this case, |el| will be bound to zero
instead of \nil.  Also, \iter\ collects all the top-level declarations
and puts them at the begining of the generated code, so it is not
necessary to place all declarations at the beginning of an \iter\
form; instead, they can be written near the variables whose types they
declare.  

Since \iter\ is not part of the compiler, it will not know
about declarations that occur outside an \iter\ form; these
declarations must be repeated inside the form.

Here is another way we could have declared the type of |el|:
\begin{program}
(iter (for (the fixnum el) in number-list)
      (counting (oddp el)))
\end{program}
\iter\ extends the Common Lisp |the|\lispindex{the}
form to apply to variables as well as value-producing forms; anywhere
a variable is allowed---in a |with| clause, as the iteration
variable in a driver clause, as the |into| argument of an
accumulation clause, even inside a destructuring template---you can
write |(the ~type~ ~symbol~)| instead.

There is one crucial difference between using a |the| form and
actually declaring the variable: explicit declarations are always
placed in the generated code, but type information from a |the|
form is not turned into an actual declaration unless you tell \iter\
to do so using |iterate:declare-variables|.  See below.

\begin{sloppypar}
Declaring the types of internal variables is harder than declaring the
types of explicitly mentioned variables, since their names
are unknown.  You do it by declaring |iterate:declare-variables|
somewhere inside the top level of the \iter\ form.  (This will also
generate declarations for variables declared using |the|.)
\iter\ does not provide much selectivity here: it's all or none. 
And unfortunately, since \iter\ is not privy to compiler information
but instead reads declarations itself, it will not hear if you
|(declaim (iterate:declare-variables))|.  Instead, set the variable 
|iterate::*always-declare-variables*| to |t| at
compile-time, using |eval-when|.
\end{sloppypar}

To determine the appropriate types for internal variables, \iter\ uses
three sources of information:

\begin{itemize}

\item Often, the particular clause dictates a certain type for a
variable; \iter\ will use this information when available.  In the
current example, the variable |list17| will be given the type
|list|, since that is the only type that makes sense; and the
variable |result| will be given the type |fixnum|, on the
assumption that you will not be counting high enough to need bignums.
You can override this assumption only by using and explicitly declaring a
variable: 
\begin{verbatim}
(iter (declare (iterate:declare-variables))
      (for el in number-list)
      (count (oddp el) into my-result)
      (declare (integer my-result))
      (finally (return my-result)))
\end{verbatim}

\begin{sloppypar}
Other examples of the type assumptions that \iter\ makes are: type
|list| for |into| variables of collection clauses; type |list| for
expressions that are to be destructured; type |vector| for the
variable holding the vector in a |for\dots in-vector| clause, and
similarly for |string| and the |for\dots in-string| clause;
and the implementation-dependent type
|(type-of array-dimension-limit)| for the index and limit
variables generated by sequence iteration drivers like |for\dots
in-vector| and |for\dots in-string| (but not |for\dots in-sequence|,
because it may be used to iterate over a list). 
\end{sloppypar}

\item Sometimes, \iter\ will examine expressions and try to determine
their types in a simple-minded way.  If the expression is
self-evaluating (like a number, for instance), \iter\ knows that the
expression's type is the same as the type of the value it denotes, so
it can use that type.  If the expression is of the form |(the ~type~
~expr~)|, \iter\ is smart enough to extract ~type~ and use it.
However, the current version of \iter\ does not  
examine declarations of function result types or do any type
inference.  It will not determine, for 
example, that the type of |(+ 3 4)| is |fixnum|, or even
|number|.

\item In some cases, the type of an internal variable should match the
type of some other variable.  For instance, \iter\ generates an
internal variable for |(f x)| in the
clause |(for i from 1 to (f x))|, and in the absence of other
information will give it the same type as |i|.  If, however, the
expression had been written |(the fixnum (f x))|, then \iter\
would have given the internal variable the type |fixnum|
regardless of |i|'s type.  The type incompatibility errors that
could arise in this situation are not checked for.

\end{itemize}

Note that if you do declare |iterate:declare-variables|, then
\iter\ may declare user variables as well as internal ones if they do
not already have declarations, though only for variables that it
binds.  For instance, in this code:

\begin{program}
(iter (declare (iterate:declare-variables))
      (for i from 1 to 10)
      (collect i into var))
\end{program}
the variable |var| will be declared to be of type |list|.


\subsection{Summary}

\iter\ understands standard Common Lisp variable type declarations
that occur within an \iter\ form and
will pass them through to the generated code.  If the declaration
|(iterate:declare-variables)|\lispindex{declare-variables}
appears at the top level of an 
\iter\ form, or if 
|iterate::*always-declare-variables*|\lispindex{*always-declare-variables*} 
is \nonnil, then \iter\ will use the type information gleaned from user
declarations, self-evaluating expressions and |the| expressions,
combined with reasonable assumptions, to determine variable
types and declare them.


%%%%%%%%%%%%%%%%%%%%%%%%%%%%%%%%%%%%%%%%%%%%%%%%%%%%%%%%%%%%%%%%
\section{Problems with Code Movement}
\label{code-movement}

Some \iter\ clauses, or parts of clauses, result in code being
moved from the location of the clause to other parts of the loop.
Drivers behave this way, as do code-placement clauses like |initially|
and |finally|.  When using these clauses, there is a danger of writing
an expression that makes sense in its apparent location but will be
invalid or have a different meaning in another location.  For example:
\begin{program}
(iter (for i from 1 to 10)
      (let ((x 3))
        (initially (setq x 4))))
\end{program}
While it may appear that the |x| of |(initially (setq x 4))| is the
same as the |x| of |(let ((x 3)) \dots|, in fact they are not:
|initially| moves its code outside the loop body, so |x| would refer
to a global variable.  Here is another example of the same problem:
\begin{program}
(iter (for i from 1 to 10)
      (let ((x 3))
        (collect i into x)))
\end{program}
If this code were executed, |collect| would create a binding for its
|x| at the top level of the \iter\ form that the |let| will shadow.

Happily, \iter\ is smart enough to catch these errors; it
walks all problematical code to ensure that free variables are not
bound inside the loop body, and checks all variables it binds for the
same problem.

However, some errors cannot be caught:

\begin{program}
(iter (with x = 3)
      (for el in list)
      (setq x 1)
      (reducing el by \#'+ initial-value x))
\end{program}
|reducing| moves its |initial-value| argument to the initialization
part of the loop in order to produce more efficient code.  Since
\iter\ does not perform data-flow analysis, it cannot determine that
|x| is changed inside the loop; all it can establish is that |x| is
not bound internally.  Hence this code will not signal an
error and will use $3$ as the initial value of the reduction.

The following list summarizes all cases that are subject to these code
motion and variable-shadowing problems.
\begin{itemize}
\item Any variable for which \iter\ creates a binding, including those
used in |with| and the |into| keyword of many clauses.

\begin{sloppypar}
\item The special clauses which place code: |initially|, |after-each|, |else|,
|finally| and |finally-protected|. 
\end{sloppypar}

\item The variables of a |next| or |do-next| form.

\item The |initially| arguments of |for\dots initially\dots then| and
|for\dots previous|. 

\item The |then| argument of |for\dots initially\dots then|.

\item The |initial-value| arguments of |reducing| and |accumulate|.

\item The |on-failure| argument of |finding\dots such-that|.
\end{itemize}

%%%%%%%%%%%%%%%%%%%%%%%%%%%%%%%%%%%%%%%%%%%%%%%%%%%%%%%%%%%%%%%%
\section{Differences Between {\tt Iterate} and {\tt Loop}}
|loop| contains a great deal of complexity which \iter\ tries to
avoid.  Hence many esoteric features of |loop| don't exist in \iter.
Other features have been carried over, but in a cleaned-up form.  
And of course, many new features have been added; they are not
mentioned in this list.

\begin{itemize}

\item \iter's syntax is more Lisp-like than |loop|'s, having a higher
density of parens.

\item The current implementation of \iter, unlike the current version
of  |loop| (as documented in {\em Common Lisp, 2nd Ed.\/}), is
extensible (see section \ref{extend}).  

\item |loop| puts the updates of all driver variables at the top of
the loop; \iter\ leaves them where the driver clauses appear.

\item While for the most part \iter\ clauses that resemble |loop| clauses
behave similarly, there are some differences.  For instance, there is
no |for\dots =\dots then| in \iter; instead use 
|for\dots initially\dots then|.  

\item |loop| binds the variable |it| at certain times
to allow pseudo-English expressions like |when ~expr~ return it|. 
In \iter, you must bind ~expr~ to a variable yourself.  Note that
|when ~expr~ return it| is like |thereis ~expr~| except that the latter is an
accumulation clause and therefore competes with other accumulations
(remember section \ref{multiple} above).

% repeat different behaviour of |always| clause here?

\item |loop| has a special |return| clause, illustrated in the
previous item.  \iter\ doesn't need one,  since an ordinary Lisp
|return| has the same effect.

\item |loop| allows for parallel binding and stepping of iteration
variables.  \iter\ does not.  (See section \ref{bindings}.)

\item |loop| and \iter\ handle variable type declarations very
differently.  |loop| provides a special syntax for declaring variable
types, and does not examine declarations.  Moreover, the standard
implementation of |loop| will
generate declarations when none are requested.
\iter\ parses standard Common Lisp type declarations, and will never
declare a variable itself unless declarations are 
specifically requested.

\end{itemize}

%%%%%%%%%%%%%%%%%%%%%%%%%%%%%%%%%%%%%%%%%%%%%%%%%%%%%%%%%%%%%%%%
\section{Rolling Your Own}
\label{extend}

\subsection{Introduction}

\iter\ is extensible---you can write new clauses that embody new
iteration patterns.  You might want to write a new driver clause for a
data structure of your own, or you might want to write a clause that
collects or manipulates elements in a way not provided by \iter. 

This section describes how to write clauses for \iter.  Writing a
clause is like writing a macro.  In fact, writing a clause ~is~
writing a macro: since \iter\ code-walks its body and macroexpands,
you can add new abstractions to \iter\ with good old |defmacro|.

Actually, there are two extensions you can make to \iter\ that are
even easier than writing a macro.  They are adding a synonym for an
existing clause and defining a driver clause for an indexable
sequence.  These can be done with |defsynonym| and
|defclause-sequence|, respectively.  See section \ref{aids}, below.

The rest of this section explains how to write macros that expand into
\iter\ clauses.
Here's how you could add a simplified version of \iter's |multiply|
clause, if \iter\ didn't already have one:
\begin{program}
(defmacro multiply (expr)
  `(reducing ,expr by \#'* initial-value 1))
\end{program}

If you found yourself summing the square of an expression often, you
might want to write a macro for that.  A first cut might be
\begin{program}
(defmacro sum-of-squares (expr)
  `(sum (* ,expr ,expr)))
\end{program}
but if you are an experienced macro writer, you will realize that this
code will evaluate ~expr~ twice, which is probably a bad idea.  A
better version would use a temporary:
\begin{program}
(defmacro sum-of-squares (expr)
  (let ((temp (gensym)))
    `(let ((,temp ,expr))
       (sum (* ,temp ,temp)))))
\end{program}
Although this may seem complex, it is just the sort of thing you'd
have to go through to write any macro, which illustrates the point of
this section: if you can write macros, you can extend \iter.

Our macros don't use \iter's keyword-argument
syntax.  We could just use keywords with |defmacro|, but we would
still not be using \iter's clause indexing mechanism.  Unlike Lisp,
which uses just the first symbol of a form to determine what function
to call, \iter\ individuates clauses by the list of required keywords.
For instance, |for\dots in| and |for\dots in-vector| are different clauses
implemented by distinct Lisp functions.  

To buy into this indexing scheme, as well as the keyword-argument
syntax, use |defmacro-clause|:

\begin{clauses}

\defmacro{defmacro-clause}{arglist |\&body| body}
Defines a new \iter\ clause.  ~arglist~ is a list of symbols which are
alternating keywords and arguments.  \opt\ may be used, and the list
may be terminated by |\&sequence|.  ~body~ is an ordinary macro body,
as with |defmacro|.  If the first form of ~body~ is a string, it is
considered a documentation string and will be shown by
|display-iterate-clauses|. |defmacro-clause| will signal an error if 
defining the clause would result in an ambiguity.  E.g. you cannot
define the clause |for\dots from| because there would be no way to
distinguish it from a use of the |for| clause with optional keyword |from|.

\end{clauses}

\medskip

Here is |multiply| using |defmacro-clause|.  The keywords are capitalized
for readability.
\begin{program}
(defmacro-clause (MULTIPLY expr \opt\ INTO var)
  `(reducing ,expr by \#'* into ,var initial-value 1))
\end{program}
You don't have to worry about the case when |var| is not supplied; for
any clause with an |into| keyword, saying |into nil| is equivalent to
omitting the |into| entirely.

As another, more extended example, consider the fairly common
iteration pattern that 
involves finding the sequence element that maximizes (or minimizes) some
function.  \iter\ provides this as |finding\dots maximizing|, but it's
instructive to see how to write it.
Here, in pseudocode, is how you might write such a loop for
maximizing a function F:

\begin{quote}
\begin{tabbing}
set variable MAX-VAL to NIL; \\
set variable WINNER to NIL; \\
for\=\ each element EL in the sequence \\
\>  if\=\ MAX-VAL\=\ is NIL or F(EL) $>$ MAX-VAL then \\
\>\>    set MAX-VAL to F(EL); \\
\>\>    set WINNER to EL; \\
\>  end if; \\
end for; \\
return WINNER.
\end{tabbing}
\end{quote}

Here is the macro:
\begin{program}
(defmacro-clause (FINDING expr MAXIMIZING func \opt\ INTO var)
  (let ((max-val (gensym))
        (temp1 (gensym))
        (temp2 (gensym))
        (winner (or var iterate::*result-var*)))
    `(progn 
       (with ,max-val = nil)
       (with ,winner = nil)
       (cond 
        ((null ,max-val)
         (setq ,winner ,expr)
         (setq ,max-val (funcall ,func ,winner))
        (t
         (let* ((,temp1 ,expr)
                (,temp2 (funcall ,func ,temp1)))
           (when (> ,temp2 ,max-val)
             (setq ,max-val ,temp2)
             (setq ,winner ,temp1))))))
       (finally (leave ,winner)))))
\end{program}
Note that if no |into| variable is supplied, we use
|iterate::*result-var*|, which contains the internal variable into
which all clauses place their results.  If this variable is bound by
some clause, then \iter\ will return its value automatically;
otherwise, \nil\ will be returned.  

\subsection{Writing Drivers}

In principle, drivers can be implemented just as easily as other
\iter\ clauses.  In practice, they are a little harder to get right. 
As an example, consider writing a driver that
iterates over all the 
elements of a vector, ignoring its fill-pointer.  |for\dots in-vector|
won't work for this, because it observes the fill-pointer.  It's necessary to
use |array-dimension| instead of |length| to obtain the size of the
vector.  Here is one approach:

\begin{program}
(defmacro-clause (FOR var IN-WHOLE-VECTOR v)
  "All the elements of a vector (disregards fill-pointer)"
  (let ((vect (gensym))
        (index (gensym)))
    `(progn
       (with ,vect = ,v)
       (for ,index from 0 below (array-dimension ,vect 0))
       (for ,var = (aref ,vect ,index)))))
\end{program}
Note that we immediately put |v| in a variable, in case it is an
expression.  Again, this is just good Lisp macrology.  It also has a
subtle effect on the semantics of the driver: |v| is evaluated only
once, at the beginning of the loop, so changes to |v| in the loop have
no effect on the driver.  Similarly, the bounds for numerical iteration
e.g. the above |array-dimension| are also evaluated once only.  This is how
all of \iter's drivers work.


There is an important point concerning the |progn| in this code.  We
need the |progn|, of course, because we are returning several forms,
one of which is a driver.  But \iter\ drivers must occur at top-level.
Is this code in error?  No, because ~top-level~ is defined in \iter\
to include forms inside a |progn|.  This is just the definition of
top-level that Common Lisp uses, and for the same reason: to allow
macros to return multiple forms at top-level.

While our |for\dots in-whole-vector| clause will work, it is not
ideal.  In particular, it does not support generating.  Do do so, we
need to use |for\dots next| or |for\dots do-next|.  The job is
simplified by the |defmacro-driver| macro.

\begin{clauses}

\defmacro{defmacro-driver}{arglist |\&body| body}
Defines a driver clause in
both the |for| and |generate| forms, and provides a parameter
|generate| which ~body~ can examine to determine how it was invoked.
~arglist~ is as in |defmacro-clause|, and should begin with the symbol
|for|. 

\end{clauses}

With |defmacro-driver|, our driver looks like this:
\begin{program}
(defmacro-driver (FOR var IN-WHOLE-VECTOR v)
  "All the elements of a vector (disregards fill-pointer)"
   (let ((vect (gensym))
         (end (gensym))
         (index (gensym))
         (kwd (if generate 'generate 'for)))
     `(progn
        (with ,vect = ,v)
        (with ,end = (array-dimension ,vect 0))
        (with ,index = -1)
        (,kwd ,var next (progn (incf ,index)
                               (if (>= ,index ,end) (terminate))
                               (aref ,vect ,index))))))
\end{program}


We are still missing one thing: the |\&sequence| keywords.  
We can get them easily enough, by writing 
\begin{program}
(defmacro-driver (FOR var IN-WHOLE-VECTOR v \&sequence) 
  ...)
\end{program}
We can now refer to parameters |from|, |to|, |by|, etc. which contain
either the values for the corresponding keyword, or \nil\ if the
keyword was not supplied.  Implementing the right code for these
keywords is cumbersome but not difficult; it is left as an exercise.
But before you begin, see |defclause-sequence| below for an easier way.
       
\subsection{Extensibility Aids}
\label{aids}

This section documents assorted features that may be of use in
extending \iter.

\begin{clauses}

\defunexpvar{*result-var*}
Holds the variable that is used to return a value as a result of the
\iter\ form.  You may examine this and use it in a |with| clause, but
you should not change it.

\defmacro{defsynonym}{syn word}
 Makes ~syn~ a synonym for the existing \iter\ keyword ~word.~  Only
the first word in each clause can have synonyms. 

\defmacroxx{defclause-sequence}{element-name index-name}{|\&key|
access-fn size-fn sequence-type}{element-type
element-doc-string index-doc-string}  
Provides
a simple way to define sequence clauses.  Generates two
clauses, one for iterating over the sequence's elements, the other
for iterating over its indices.  The first symbol of both
clauses will have print-name |for|.
~element-name~  and ~index-name~ should be symbols.
~element-name~ is the second keyword of the element iterator (typically of
the form 
|in-~sequence-type~|), and ~index-name~ is the second keyword
of the index-iterator (typically of the form
|index-of-~sequence-type~|).  Either name may be 
\nil, in which case the corresponding clause is not defined.  If both
symbols are supplied, they should be in the same package.  The |for|
that begins the clauses will be in this package.

\cpar ~access-fn~ is the function to be used to
access elements of the sequence in the element iterator.  The function
should take two 
arguments, a sequence and an index, and return the appropriate element.
~size-fn~ should denote a function of one argument, a sequence, that
returns its size.  Both ~access-fn~ and ~size-fn~ are required for the
element iterator, but only ~size-fn~ is needed for the index iterator.

\cpar The ~sequence-type~ and ~element-type~ keywords are used to
suggest types for the variables 
used to hold the sequence and the
sequence elements, respectively.  The usual rules about \iter's
treatment of variable type declarations apply (see section \ref{types}).

\cpar ~element-doc-string~ and ~index-doc-string~ are
the documentation strings, for use with |display-iterate-clauses|. 

\cpar The generated element-iterator performs destructuring on the
element variable.

\cpar As an example, the above |for\dots in-whole-vector| example
could have been written:
\begin{program}
(defclause-sequence IN-WHOLE-VECTOR INDEX-OF-WHOLE-VECTOR
  :access-fn 'aref
  :size-fn \#'(lambda (v) (array-dimension v 0))
  :sequence-type 'vector
  :element-type t
  :element-doc-string 
     "Elements of a vector, disregarding fill-pointer"
  :index-doc-string 
     "Indices of vector, disregarding fill-pointer")
\end{program}

\end{clauses}

\subsection{Subtleties}

There are some subtleties to be aware of when writing \iter\ clauses.
First, the code returned by your macros may be |nconc|'ed into a list,
so you should always returned freshly consed lists, rather than
constants.  Second, \iter\ matches clauses by using |eq| on the first
symbol and |string=| on the subsequent ones, so the package of the
first symbol of a clause is relevant.  All of the clauses in this manual
have their first word in the \iter\ package.  
You can use the package system in the usual way to shadow
\iter\ clauses without replacing them.

%%% say more here, about the badness that only the first word of a
%%% clause is packagey.

\section{Non-portable Extensions to Iterate (Contribs)}
\label{contribs}

Currently, there is only one non-portable extension to iterate in the
distribution: iterate-pg. If you have made an extension that depends
on non-portable features, feel free to send them to |asf@boinkor.net|
for inclusion in the iterate distribution.

\subsection{An SQL query driver for iterate}

The pg package by Eric Marsden (see |http://cliki.net/pg|) provides an
interface to the PostgreSQL database. Using the \iterpg\ extension, it
is possible to handle the results of SQL queries using \iter.

This usage example should give you an idea of how to use it:

\begin{program}
(pg:with-pg-connection (c "somedb" "someuser")
  (iter (for (impl version date) in-relation "select * from version"
                                 on-connection *dbconn*)
        (collect version)))
\end{program}

To use the extension via |ASDF|, simply make your system depend on the
|iterate-pg| system instead of the |iterate| system. To load it
manually, use:

\begin{program}
  (asdf:oos 'asdf:load-op :iterate-pg)
\end{program}

%%%%%%%%%%%%%%%%%%%%%%%%%%%%%%%%%%%%%%%%%%%%%%%%%%%%%%%%%%%%%%%%
\section{Obtaining {\tt Iterate}}

\begin{sloppypar}
The information in this chapter is obsolete but included for
completeness's sake; Currently, the most up-to-date information on
\iter\ can be found at |http://boinkor.net/iterate.html|.
\end{sloppypar}

\begin{sloppypar}
\iter\ currently runs on Lisp Machines, and on
HP's, Sun3's and Sparcstations under Lucid.  
\iter\ source and binaries are available at the MIT AI Lab in the
subdirectories of |/src/local/lisplib/|.  The source file,
|iterate.lisp|, is also available for anonymous FTP in the directory
|/com/fpt/pub/| on the machine |TRIX.AI.MIT.EDU| (Internet number
128.52.32.6).  If you are unable to obtain |iterate| in one of these
ways, send mail to |jba@ai.mit.edu| and I will send you the source
file. 
\end{sloppypar}

\begin{sloppypar}
\iter\ resides in the |iterate| package (nickname |iter|).  Just say
\linebreak |(use-package :iterate)| to make all the necessary symbols
available. 
If a symbol is not exported, it appears in this manual with an
``|iterate::|'' prefix.
\end{sloppypar}

Send bug reports to |bug-iterate@ai.mit.edu|.  The |info-iterate|
mailing list will have notices of changes and problems; to have
yourself added, send mail to |info-iterate-request@ai.mit.edu|.


\medskip

\begin{flushleft}
 \bf Acknowledgements
\end{flushleft}
\smallskip

Richard Waters provided invaluable criticism which spurred me to improve
\iter\ greatly.  As early users, David Clemens, Oren Etzioni and Jeff
Siskind helped ferret out many bugs.

%\begin{theindex}
% The index files must be generated with the genindex program
% or from makeindex iterate-manual.idx which creates iterate-manual.ind.

%\baselineskip\normalbaselineskip
%\advance\baselineskip by -2pt

% Fix ``.'' spacing in lists
\documentstyle[12pt,aimemo]{article}

\setlength{\textwidth}{6.0in}
\setlength{\oddsidemargin}{0.25in}
\setlength{\evensidemargin}{0.25in}
\setlength{\topmargin}{0pt}
\setlength{\headsep}{24pt}
\setlength{\textheight}{8.5in}

\let\origpar=\par

% somehow, LaTex fucks with things so that the plain Tex \obeyspaces
% doesn't work.  This is the same, except we use an mbox containing
% a space instead of just a space.

\newcommand{\spc}{\mbox{ }}
\newcommand{\tAb}{\mbox{        }}
\def\obspaces{\catcode`\ =\active\catcode`\^^I=\active}
{\obspaces \global\let =\spc\global\let^^I=\tAb}

\newenvironment{program}{\medskip\samepage\tt\obeylines\obspaces}{\medskip}

% use | for \tt
\catcode`\|=13
\newif\ifvbar
\def|{\ifvbar
         \endgroup\vbarfalse
      \else 
         \begingroup\tt\vbartrue 
      \fi}



\newcommand{\lisp}{\tt}
\newcommand{\iter}{|iterate|}
\newcommand{\iterpg}{|iterate-pg|}
\newcommand{\setf}{|setf|}
\newcommand{\nil}{|nil|}
\newcommand{\nonnil}{non-\nil}
\newcommand{\opt}{{\lisp \&optional}}
\newcommand{\yields}{$\Rightarrow$}



\newbox\Dots
\setbox\Dots=\hbox{\small ...}
\renewcommand{\dots}{\copy\Dots}

% Here's a way to define an environment which will treat all
% paragraphs inside it in the same manner.  It's, as near as I can
% figure, the way that Latex does things.  Using \everypar won't work;
% apparently Latex resets that a lot.  What you have to do is: at the
% top of the file do \let\origpar=\par.  In the
% begin part of the environment, do the parshape (or whatever), then say
% \def\par{{\origpar}}.  The extra braces are necessary.   That's all
% you have to do; at the end of the environment things will return to
% normal automatically.  Why does this all work?  I have no idea.
% Another important thing: there should be a blank line (i.e. a \par)
% between the end of the last paragraph and the \end{environment}.

\newlength{\clindent}
\setlength{\clindent}{0.5in}
\newlength{\clparindent}
\setlength{\clparindent}{\parindent}

\newenvironment{clauses}{\advance\linewidth -\parindent
                        \hangindent\clindent\relax
                         \hangafter=0
                          \parindent=0pt
                          \def\par{{\origpar}}}{}

\newcommand{\cpar}{\hspace \clparindent}

\newcommand{\startitem}{\bigskip\pagebreak[3]}
\newcommand{\finishitem}{\smallskip}

\newcommand{\clause}[1]{\startitem\Clause{#1}\finishitem}
\newcommand{\Clause}[1]{\hbox{#1}}
\newcommand{\clindex}[1]{\index{{\lisp #1}}}
\newcommand{\clindexx}[2]{\index{{\lisp #1\dots #2}}}
\newcommand{\clindexxx}[3]{\index{{\lisp #1\dots #2\dots #3}}}
\newcommand{\clausex}[2]{\startitem\Clausex{#1}{#2}\finishitem}
\newcommand{\Clausex}[2]{\Clause{|#1| #2}\clindex{#1}}

\newcommand{\defvar}[1]{\startitem\deff{#1}{}{Variable}\finishitem}
\newcommand{\defunexpvar}[1]{\startitem\deffnoindex{iterate::#1}{}{Variable}
                                        \lispindex{#1}\finishitem}
\newcommand{\defun}[2]{\startitem\deff{#1}{#2}{Function}\finishitem}
\newcommand{\defmacro}[2]{\startitem\deff{#1}{#2}{Macro}\finishitem}
\newcommand{\defunx}[3]{\startitem\deffx{#1}{#2}{#3}{Function}\finishitem}
\newcommand{\defmacrox}[3]{\startitem\deffx{#1}{#2}{#3}{Macro}\finishitem}
\newcommand{\defunxx}[4]{\startitem\deffxx{#1}{#2}{#3}{#4}{Function}\finishitem}
\newcommand{\defmacroxx}[4]{\startitem\deffxx{#1}{#2}{#3}{#4}{Macro}\finishitem}

\newcommand{\deff}[3]{\deffnoindex{#1}{#2}{#3}\lispindex{#1}}
\newcommand{\deffnoindex}[3]{\hbox to \hsize{{\tt #1} {\it #2}\hfill [{\it #3}]}}
\newcommand{\deffx}[4]{\deff{#1}{#2}{#4}\moreargs{#1}{#3}}
\newcommand{\deffxx}[5]{\deffx{#1}{#2}{#3}{#5}\moreargs{#1}{#4}}

\newcommand{\moreargs}[2]{\hbox to \hsize{\phantom{\lisp #1} {\it #2}\hfill}}

\newcommand{\lispindex}[1]{\index{{\lisp #1}}}

 \newenvironment{note}[1]{\pagebreak[2]\bigskip
                          \hrule\smallskip\small
                         {\setlength{\parindent}{0pt}
                           \par{\bf #1:}}}{\smallskip\hrule\bigskip}



\makeindex

%%%%%%%%%%%%%%%%%%%%%%%%%%%%%%%%%%%%%%%%%%%%%%%%%%%%%%%%%%%%%%%%

%\thispagestyle{empty}

%\vspace*{2in}

%\title{The |Iterate| Manual}
%\author{Jonathan Amsterdam}

%\maketitle

%\thispagestyle{empty}

%\begin{center}
%\Large *** DRAFT ***
%\end{center}
%This manual is in beta-test. It will become an AI Memo in a couple of
%months.  I would appreciate any comments or
%corrections.  You can reach me at email address |jba@ai.ai.mit.edu|.

\begin{document}

\def\writtenby{Jonathan Amsterdam}
\def\contractno{This report describes research done at the Artificial
Intelligence Laboratory of the Massachusetts Institute of Technology.
Support for the laboratory's artificial intelligence research is
provided in part by the Advanced Research Projects Agency of the
Department of Defense under Office
of Naval Research contract N00014-85-K-0124.}

\AIMEMO{1236}{\Large The Iterate Manual \\[1in]}
{This is the manual for version 1.4 of \iter, a powerful iteration
macro for Common  
Lisp. \iter\ is similar to {\tt loop} but provides numerous additional
features, is well integrated with Lisp, and is extensible. 
\vfill }


% use ~ for \em
\catcode`\~=13
\newif\ifem
\def~{\ifem
         \/\endgroup\emfalse
      \else 
         \begingroup\em\emtrue 
      \fi}



%\pagebreak
%\setcounter{page}{1}

%\thispagestyle{empty}


\tableofcontents

\pagebreak
\pagestyle{headings}

\section{Introduction}

\begin{sloppypar}
This manual describes
\iter, a powerful iteration facility for Common Lisp.
\iter\ provides abstractions for many common iteration
patterns and allows for the definition of additional patterns.  \iter\ is
a macro that expands into ordinary Lisp at
compile-time, so it is more efficient than higher-order
functions like |map| and |reduce.|  
While it is similar to |loop|, \iter\ offers a more Lisp-like syntax
and enhanced extensibility.
(For a more complete
comparison of \iter\ with other iteration constructs, see MIT AI Lab Working
Paper No. 324, ~Don't Loop, Iterate.~)
\end{sloppypar}

An \iter\ form consists of the symbol |iter|\footnote{You can also use
|iterate|, but |iter| is preferred because it avoids potential conflicts with
possible future additions to Common Lisp, and because it saves
horizontal space when writing code.}
followed by one or more
forms, some of which may be \iter\ 
~clauses.~  Here is a simple example of \iter\ which collects the
numbers from 1 to 10 into a list, and returns the list.  The return
value is shown following the arrow.
\begin{program}
(iter (for i from 1 to 10)
      (collect i))          \yields (1 2 3 4 5 6 7 8 9 10)
\end{program}
This form contains two clauses: a |for| clause that steps the
variable |i| over the integers from 1 to 10, and a |collect|
clause that accumulates its argument into a list.  With a few
exceptions, all \iter\ clauses have the same format: alternating
symbols (called ~keywords~) and expressions (called
~arguments~).  The syntax and terminology are those of Common
Lisp's keyword lambda lists.  One difference is that \iter's keywords
do not have to begin with a colon---though they may, except for the
first symbol of a clause.  So you can also write  |(for i :from
1 :to 10)| if you prefer.

Any Lisp form can appear in the body of an \iter, where it will have
its usual meaning.  \iter\ walks the entire body, expanding macros,
and recognizing clauses at any level.  This example collects all the
odd numbers in a list:
\begin{program}
(iter (for el in list)
      (if (and (numberp el) (oddp el))
          (collect el)))
\end{program}

There are clauses for iterating over numbers, lists, arrays and other
objects, and for
collecting, summing, counting, maximizing and other useful operations.
\iter\ also supports the creation of new variable bindings, stepping
over multiple sequences at once, destructuring, and compiler
declarations of variable types.  The following example illustrates
some of these features: 
\begin{program}
(iter (for (key . item) in alist)
         (for i from 0)
         (declare (fixnum i))
         (collect (cons i key)))
\end{program}

This loop takes the keys of an alist and returns a new alist
associating the keys with their positions in the original list. The
compiler declaration for |i| will appear in the generated code
in the appropriate place.

%%%%%%%%%%%%%%%%%%%%%%%%%%%%%%%%%%%%%%%%%%%%%%%%%%%%%%%%%%%%%%%%
\section{Clauses}

Most of \iter's clauses will be familiar to |loop| programmers.  
(|loop| is an iteration macro that has been incorporated into Common
Lisp.  Seen Guy Steele's {\em Common Lisp, 2nd Edition}.)
In
nearly all cases they behave the same as their |loop| counterparts, so
a |loop| user can switch to \iter\ with little pain (and much gain).

All clauses with the standard keyword-argument syntax consist of two
parts: a ~required~ part, 
containing keywords that must be present and in the right order; and
an ~optional~ part, containing keywords that may be omitted and,
if present, may occur in any order.  In the descriptions below, the
parts are separated by the Lisp lambda-list keyword \opt.


\subsection{Drivers}

An iteration-driving clause
conceptually causes the iteration to go forward.  Driver clauses in
\iter\ allow iteration over numbers, lists, vectors, hashtables, packages,
files and streams.  Iteration-driving clauses must
appear at the top level of an \iter\ form; they cannot be nested
inside another clause.  The driver variable is updated at the point
where the driver clause occurs.  Before the clause is executed for the
first time, the value of the variable is undefined.

%Also, regardless of where the driver clause
%appears in the body, the driver variable is stepped at the top of the
%loop; hence it is stylistically preferable, though
%not required, to place driver clauses at the beginning of the \iter.

Multiple drivers may appear in a single \iter\ form, in which case all
of the driver variables are updated each time through the loop, in the
order in which the clauses appear.  The first driver to terminate will
terminate the entire loop.

In all cases, the value of the driver variable on exit from the loop,
including within the epilogue code (see the |finally| clause), is
undefined.

All the parameters of a driver clause are evaluated once, before the
loop begins.  Hence it is not possible to change the bounds or other
properties of an iteration by side-effect from within the loop.

With one exception, driver clauses begin with the word |for| (or
the synonym |as|) and mention an iteration variable, which is
given a binding within the \iter\ form.
The exception is |repeat|, which just executes a loop a
specified number of times:

\begin{clauses}

\clausex{repeat}{~n~}
Repeats the loop ~n~ times.  For example:
\begin{program}
(iter (repeat 100)
      (print "I will not talk in class."))
\end{program}
\cpar If $n \leq 0$, the
loop will never be executed.  If ~n~ is not an integer, the
actual number of executions will be $\lceil n \rceil$.

\end{clauses}

\pagebreak[3]

\subsubsection{Numerical Iteration}

\begin{clauses}
\clausex{for}{~var~ |\&sequence|} 
The general form for iterating over a sequence of numbers requires a
variable and, optionally, one or more keywords that provide the bounds
and step size of the iteration. The |\&sequence|
lambda-list keyword is a shorthand for these sequence keywords.  They are:
|from|, |upfrom|, |downfrom|, |to|, |downto|, |above|, |below| and |by|.
|from| provides the starting value for ~var~
and defaults to zero. |to| provides a final value and implies that the
successive values of ~var~ will be increasing; |downto| implies that
they will be decreasing.  The loop terminates when ~var~ passes
the final value (i.e. becomes smaller or larger than it, depending on
the direction of iteration); in other  words, the loop body
will never be 
executed for values of ~var~ past the final value.
|below| and |above| are similar to |to| and
|downto|, except that the loop terminates when ~var~ equals or passes
the final value.

\cpar If no final value is specified, the variable will be stepped
forever.  Using |from| or |upfrom| will result in increasing
values, while |downfrom| will give decreasing values.

\cpar On each iteration,
~var~ is incremented or decremented by the value of the sequence
keyword |by|, which defaults to 1.  It should always be a positive
number, even for downward iterations.

\cpar In the following examples, the sequence of numbers generated is shown
next to the clause.

\begin{program}
(for i upfrom 0) \yields\ 0 1 2 \ldots
(for i from 5)   \yields\ 5 6 7 \ldots    ; either from or upfrom is okay
(for i downfrom 0) \yields\ 0 -1 -2 \ldots
(for i from 1 to 3) \yields\ 1 2 3
(for i from 1 below 3) \yields\ 1 2
(for i from 1 to 3 by 2) \yields\ 1 3
(for i from 1 below 3 by 2) \yields\ 1
(for i from 5 downto 3) \yields\ 5 4 3
\end{program}

\end{clauses}

\subsubsection{Sequence Iteration}

There are a number of clauses for iterating over sequences.  In all of
them, the argument following |for| may be a list instead of a
symbol, in which case destructuring is performed.  See section
\ref{destructuring}. 

\begin{clauses}

\clause{|for| ~var~ |in| ~list~ \opt\ |by| ~step-function~}
\clindexx{for}{in}
~var~ is set to successive elements of list.  
~step-function,~ which 
defaults to |cdr|, is used to obtain the next sublist. 

\clause{|for| ~var~ |on| ~list~ \opt\ |by| ~step-function~}
\clindexx{for}{on}
~var~ is set to successive sublists of list. 
~step-function~ (default |cdr|) is used as in |for\dots in|.

\end{clauses}

\medskip

These two clauses use |atom| to test for the end
of a list.  Hence, given a list whose final |cdr| is not \nil,
they will silently ignore the last |cdr|.  Other choices are
|endp|,  which would signal an error, and |null|, which
would probably result in an error somewhere else.  If you wish to use
an end-test other than |atom|, set the variable
|iterate::*list-end-test*|\lispindex{*list-end-test*} to the name of
the desired function. 

\begin{clauses}
\clause{|for| ~var~ |in-vector| ~vector~ |\&sequence|}
\clindexx{for}{in-vector}
~var~ takes on successive elements from ~vector.~   The vector's
fill-pointer is observed.  Here and in subsequent clauses, the |\&sequence|
keywords include |with-index|, which takes a symbol as argument
and uses it for the index variable instead of an internally generated
symbol.  The other |\&sequence| keywords behave as in numerical
iteration, except that the default iteration bounds are the bounds of
the vector.  E.g. in |(for i in-vector v downto 3)|,
|i| will start off being bound to the last element in |v|,
and will  
be set to preceding elements down to and including the element with
index 3.

\clause{|for| ~var~ |in-sequence| ~seq~ |\&sequence|}
\clindexx{for}{in-sequence}
This uses Common Lisp's generalized sequence functions, |elt|
and |length|, to obtain elements and determine the length of
~seq.~   Hence it will work for any sequence, including lists, and will
observe the fill-pointers of vectors.

\clause{|for| ~var~ |in-string| ~string~ |\&sequence|}
\clindexx{for}{in-string}
~var~ is set to successive characters of ~string.~

\startitem
\Clause{|for| ~var~ |index-of-vector| ~vector~ |\&sequence|}
\clindexx{for}{index-of-vector}
\Clause{|for| ~var~ |index-of-sequence| ~sequence~ |\&sequence|}
\clindexx{for}{index-of-sequence}
\Clause{|for| ~var~ |index-of-string| ~string~ |\&sequence|}
\clindexx{for}{index-of-string}
\finishitem
~var~ is set to successive indices of the sequence.
These clauses avoid the overhead of accessing the sequence elements
for those applications where they do not need to be examined, or are
examined rarely.  They admit all the optional keywords of the other
sequence drivers except the (redundant) |with-index| keyword.

\clause{|for| ~(key value)~ |in-hashtable| ~table~}
\clindexx{for}{in-hashtable}

~key~ and ~value,~ which must appear as shown in a list
and may be destructuring templates, are set to the keys and values of
~table~.  If ~key~ is |nil|, then the hashtable's keys will be ignored;
similarly for ~value~.
The order in which 
elements of ~table~ will be retrieved is unpredictable.

\clause{|for| ~var~ |in-package| ~package~ \opt\ |external-only| ~ext~}
\clindexx{for}{in-package}
Iterates over all the symbols in ~package,~ or over only the
external symbols if ~ext~ is specified and non-|nil|.  ~ext~ is not
evaluated.  The same symbol may appear more than once.

\clause{|for| ~var~ |in-packages| ~packages~ \opt\ |having-access| ~symbol-types~}
\clindexx{for}{in-packages}
Iterates over all the symbols from the list of packages denoted by the
descriptor ~packages~ and having accessibility (or visibility) given by
~symbol-types~.  This defaults to the list |(:external :internal :inherited)|
and is not evaluated.  ~var~ must be a list of up to three variables: in each
iteration, these will be set to a symbol, its access-type and package (as per
|with-package-iterator| in ANSI-CL).  The same symbol may appear more than
once.

\end{clauses}

\begin{clauses}

\clause{|for| ~var~ |in-file| ~name~ \opt\ |using| ~reader~}
\clindexx{for}{in-file}
Opens the file ~name~ (which may be a string or pathname) for input,
and iterates 
over its contents. ~reader~ defaults to |read|, so by default ~var~
will be bound to the successive forms in the file.  The \iter\ body is
wrapped in an unwind-protect to ensure that the file is closed no
matter how the \iter\ is exited.

\clause{|for| ~var~ |in-stream| ~stream~ \opt\ |using| ~reader~}
\clindexx{for}{in-stream}
Like |for\dots in-file|, except that ~stream~ should be an existing
stream object that supports input operations.

\end{clauses}

\subsubsection{Generalized Drivers}

These are primarily useful for writing drivers that can also be used
as generators (see section \ref{generators}, below).
 
\begin{clauses}

\clause{|for| ~var~ |next| ~expr~}
\clindexx{for}{next}

~var~ is set to ~expr~ each time through the loop.  Destructuring is
performed.  When the clause is used as a generator, ~expr~ is the code
that is executed when |(next ~var~)| is 
encountered (see section \ref{generators}, below).
~expr~ should compute the first value for ~var~, as well
as all subsequent values, and is responsible for terminating the loop.
For compatibility with future versions of \iter, this termination
should be done with |terminate|\clindex{terminate}, which can be
considered a synonym for |finish| (see section \ref{control-flow}).

\cpar As an example, the following clauses are equivalent to |(for
i from 1 to 10)|:
\begin{program}
(initially (setq i 0))
(for i next (if (> i 10) (terminate) (incf i)))
\end{program}

\clause{|for| ~var~ |do-next| ~form~}
\clindexx{for}{do-next}
~form~ is evaluated each time through the loop.  Its value is ~not~
set to ~var~; that is ~form's~ job. ~var~ is only present so that
\iter\ knows it is a driver variable.  \linebreak |(for ~var~ next
~expr~)| is equivalent to |(for ~var~ do-next (dsetq ~var~ ~expr~))|.
(See section \ref{destructuring} for an explanation of |dsetq|.) 

\end{clauses}

\subsubsection{Generators}
\label{generators}

In all of the above clauses, the driver variable is updated on each
iteration.
Sometimes it is desirable to have greater control over updating.  
For instance, consider the problem of associating numbers, in
increasing order and with no gaps, with the
\nonnil\ elements of a list.  One obvious first pass at writing this is:

\begin{program}
(iter (for el in list)
      (for i upfrom 1)
      (if el (collect (cons el i))))
\end{program}
But on the list |(a b nil c)| this produces |((a . 1) (b . 2) (c .
4))| instead of the desired |((a . 1) (b . 2) (c . 3))|.  The problem
is that |i| is incremented each time through the loop, even when |el|
is |nil|.  

The problem could be solved elegantly if we could step |i| only when
we wished to.  This
can be accomplished for any \iter\ driver by writing |generate| (or
its synonym |generating|)
instead of |for|.  Doing so produces a ~generator~---a driver whose
values are yielded explicitly.  To obtain the next value of a
generator variable ~v~, write \linebreak |(next ~v~)|.  The value of a |next|
form is the next value of ~v~, as determined by its associated driver
clause.  |next| also has the side-effect of updating ~v~ to that
value.  If there is no next value, |next| will terminate the loop,
just as with a normal driver.


Using generators, we can now write our example like this:
\begin{program}
(iter (for el in list)
      (generate i upfrom 1)
      (if el (collect (cons el (next i)))))
\end{program}
Now |i| is updated only when |(next i)| is executed, and this occurs
only when |el| is \nonnil.

To better understand the relationship between ordinary drivers and
generators, observe that we can rewrite an ordinary driver using its
generator form immediately followed by |next|, as this example shows:
\begin{program}
(iter (generating i from 1 to 10)
      (next i)
      ...)
\end{program}
Provided that the loop body contains no |(next i)| forms, this will
behave just as if we had written |(for i from 1 to 10)|.

We can still refer to a driver variable ~v~ without using |next|; in
this case, its value is that given to it by the last evaluation of
|(next ~v~)|.  Before |(next ~v~)| has been called the first time, the
value of ~v~ is undefined.  

This semantics is more flexible than
one in which ~v~ begins the loop bound to its first value and calls
of |next| supply subsequent values, because it means the loop will not
terminate too soon if the generator's sequence is empty.  For
instance, consider the following code, which tags \nonnil\ elements of
a list using a list of tags, and also counts the null elements.
(We assume there are at least as many tags as \nonnil\ elements.)
\begin{program}
(let* ((counter 0)
       (tagged-list (iter (for el in list)
                          (generating tag in tag-list) 
                          (if (null el)
                              (incf counter)
                              (collect (cons el (next tag)))))))
  ...)
\end{program}
It may be that there are just as many tags as non-null elements of
|list|.  If all the elements of |list| are null, we still want the
counting to proceed, even though |tag-list| is \nil.  If |tag| had to be
assigned its first value before the loop begins, we would have had to
terminate the loop before the first iteration, since when |tag-list|
is \nil, |tag| 
has no first value.  With the existing semantics, however, |(next
tag)| will never execute, so the iteration will cover all the elements of
|list|. 
        
When the ``variable'' of a driver clause is actually a destructuring
template containing several variables, all the variables are eligible
for use with |next|.  As before, |(next ~v~)| evaluates to ~v's~ next
value; but the effect is to update all of the template's variables.
For instance, the following code will return the list |(a 2 c)|.
\begin{program}
(iter (generating (key . item) in '((a . 1) (b . 2) (c . 3)))
      (collect (next key))
      (collect (next item)))
\end{program}

Only driver clauses with variables can be made into generators.  This
includes all clauses mentioned so far except for |repeat|.  It does
~not~ include |for\dots previous|, |for\dots =|,
|for\dots initially\dots then| or |for\dots first\dots then| (see
below). 

\subsubsection{Previous Values of Driver Variables}

Often one would like to access the value of a variable on a previous
iteration.  \iter\ provides a special clause for accomplishing this.

\begin{clauses}

\clause{|for| ~pvar~ |previous| ~var~ \opt\ |initially| ~init~
|back| ~n~}
\clindexx{for}{previous}
Sets ~pvar~ to the previous value of ~var,~ which should be a driver
variable, a variable from another |for\dots previous| clause, or a
variable established by a |for\dots =|, 
|for\dots initially\dots then| or |for\dots first\dots then| clause
(see section \ref{setting}).
Initially, ~pvar~ is given the value ~init~ (which defaults to \nil).  
The ~init~ expression will be moved outside the loop body, so it
should not depend on anything computed within the loop.
~pvar~ retains the value of ~init~ until ~var~ is set to its second
value, at which point ~pvar~ is set to ~var's~ first value; and so on.  

\cpar The
argument ~n~ to |back| must be a constant, positive integer, and
defaults to 1.  It determines how many iterations back ~pvar~ should
track ~var~.  For example, when ~n~ is 2, then ~pvar~ will be assigned
~var's~ first value when ~var~ is set to its third value.

\cpar A |for\dots previous| clause may occur after or before its
associated driver clause. |for\dots previous| works with generators as
well as ordinary drivers.

\pagebreak[3]

\cpar Example:
\begin{program}
(iter (for el in '(1 2 3 4))
      (for p-el previous el)
      (for pp-el previous p-el initially 0)
      (collect pp-el))
\end{program}
This evaluates to |(0 0 1 2)|.  It could have been written more
economically as 
\begin{program}
(iter (for el in '(1 2 3 4))
      (for pp-el previous el back 2 initially 0)
      (collect pp-el))
\end{program}


\end{clauses}

\subsection{Variable Binding and Setting}
\label{setting}

Several clauses exist for establishing new variable bindings or for
setting variables in the loop.  They all support destructuring.

\begin{clauses}

\clausex{with}{~var~ \opt\ |=| ~value~}
Causes ~var~ to be bound to value before the loop body is entered.
If ~value~ is not supplied, ~var~ assumes a default
binding, which will be 
\nil\ in the absence of declarations.  Also, if ~value~ is not
supplied, no destructuring is performed; instead, ~var~ may be a list of
symbols, all of which are given default 
bindings.  If ~value~ is supplied, ~var~ is bound to it, with
destructuring.  

\cpar Because |with|
creates bindings whose scope includes the entire \iter\ form, it is
good style to put all |with| clauses at the beginning.

\cpar Successive occurrences of |with| result in sequential
bindings (as with 
|let*|).  There is no way to obtain parallel bindings; see
section \ref{bindings} for a rationale.  


\clause{|for| ~var~ |=| ~expr~}
\clindexx{for}{=}
On each iteration, ~expr~ is evaluated and ~var~ is set
to its value. 

\cpar This clause may appear to do the same thing as |for\dots next|.
In fact, they are quite different.  |for\dots =| provides only three
services: it sets up a binding for ~var~, sets it to ~expr~ on each
iteration, and makes it possible to use |for\dots previous| with
~var~.  |for\dots next| provides these services in addition to the
ability to turn the driver into a generator.
%Also, the code which sets ~var~ appears in the loop body in the same
%place as the |for\dots =| clause; the code for |for\dots next| appears
%at the top of the loop, as with other drivers (except when being used
%as a generator).

\clause{|for| ~var~ |initially| ~init-expr~ |then| ~then-expr~}
\clindexxx{for}{initially}{then}
Before the loop begins, ~var~ is set to ~init-expr;~ on all
iterations after the first it is set to ~then-expr.~  This clause
must occur at top-level.  ~init-expr~ will be moved outside the loop
body and ~then-expr~ will be moved to the end of the loop body, so
they are subject to code motion problems (see section
\ref{code-movement}). 

\cpar This clause may appear to be similar to |for\dots next|, but in
fact they differ significantly.  |for\dots initially\dots then| is
typically used to give ~var~ its first value before the loop begins,
and subsequent values on following iterations.  This is incompatible
with generators, whose first value and subsequent values must all be
computed by |(next ~var~)|.  Also, the update of ~var~ in 
|for\dots initially\dots then| does not occur at the location of the
clause. 
Use |for\dots initially\dots then| for
one-shot computations where its idiom is more convenient, but use
|for\dots next| for extending \iter\ with new drivers (see section
\ref{extend}).

\clause{|for| ~var~ |first| ~first-expr~ |then| ~then-expr~}
\clindexxx{for}{first}{then}
The first time through the loop, ~var~ is set to ~first-expr;~ on
subsequent iterations, it is set to ~then-expr.~  This differs from
|for\dots initially| in that ~var~ is set to ~first-expr~
inside the loop body, 
so ~first-expr~ may depend on the results of other clauses.  For
instance, 
\begin{program}
(iter (for num in list)
      (for i first num then (1+ i))
      ...)
\end{program}
will set |i| to the first element of |list| on the first
iteration, whereas 
\begin{program}
(iter (for num in list)
      (for i initially num then (1+ i))
      ...)
\end{program}
is probably erroneous; |i| will be bound to |num|'s
default binding (usually \nil) for the first iteration.

\end{clauses}

\begin{note}{Compatibility note}
|loop|'s |for\dots =| works like \iter's, but |loop| used the syntax
|for\dots =\dots then| to mean |for\dots initially\dots then|.  It was
felt that these two operations were sufficiently different to warrant
different keywords.

Also, the |for| in the above three clauses is misleading,
since none is true driver (e.g. none has a corresponding |generate|
form).  |setting| would have been a better choice, but |for| was used
to retain some compatibility with |loop|.  
\end{note}


%%%%%%%%%%%%%%%%%%%%%%%%%%%%%%%%
\subsection{Gathering Clauses}
Many of \iter's clauses accumulate values into a variable, or set a
variable under certain conditions.  At the end of the loop, this
variable contains the desired result.  All these clauses have an
optional |into| keyword, whose argument should be a symbol.  If the
|into| keyword is not supplied, the accumulation variable will be
internally generated and its value will be returned at the end of the
loop; if a variable is specified, that variable is used for the
accumulation, and is not returned as a result---it is up to the user
to return it explicitly, in the loop's epilogue code (see |finally|).
It is safe to examine the accumulation variable during the loop, but
it should not be modified.

These clauses all begin with a verb.  When the verb does
not conflict with an existing Common Lisp function, then it may be
used in either its infinitival or present-participle form (e.g. |sum|,
|summing|).  However, when there is a conflict with Common Lisp, only
the present-participle form may be used (e.g. |unioning|).  This is to
prevent \iter\ clauses from clashing with Common Lisp
functions. 

%although these clauses are described as ``producing a value,'' it is a
%mistake to think of the lisp list representing the clause as a
%value-producing form in the usual way.  clauses may legally be written
%where a value is expected, e.g. |(setq x (sum i))|, but the
%lisp value of a clause in such a context is undefined.

\subsubsection{Reductions}

~Reduction~ is an extremely common iteration pattern in which
the results of successive applications of a
binary operation are accumulated.
For example, a loop that computes the sum of the
elements of a list is performing a reduction with the addition
operation.  This could be written in Common Lisp as  |(reduce \#'+
list)| or with \iter\ as
\begin{program}
(iter (for el in list)
      (sum el))
\end{program}

\begin{clauses}

\clausex{sum}{~expr~ \opt\ |into| ~var~}
Each time through the loop, ~expr~ is evaluated and added to a
variable, which is bound initially to zero.  If ~expr~ has a type,
it is ~not~ used as the type of the sum variable, which is always
|number|.  To get the result variable to be of a more specific
type, use an explicit variable, as in
\begin{program}
(iter (for el in number-list)
      (sum el into x)
      (declare (fixnum x))
      (finally (return x)))
\end{program}

\clausex{multiply}{~expr~ \opt\ |into| ~var~}
Like |sum|, but the initial value of the result variable is $1$,
and the variable is updated by multiplying ~expr~ into it.

\clausex{counting}{~expr~ \opt\ |into| ~var~}
~expr~ is evaluated on each iteration.  If it is \nonnil, the
accumulation variable, initially zero, is incremented.

\startitem
\Clausex{maximize}{~expr~ \opt\ |into| ~var~}
\Clausex{minimize}{~expr~ \opt\ |into| ~var~}
\finishitem
~expr~ is evaluated on each iteration and its extremum (maximum or
minimum) is stored in the accumulation variable.  If ~expr~ is never
evaluated, then the result is \nil\ (if the accumulation variable is
untyped) or $0$ (if it has a numeric type).

\clausex{reducing}{~expr~ |by| ~func~ \opt\ |initial-value| ~init-val~ |into| ~var~}
This is a general way to perform reductions. ~func~ should be a
function of two arguments, the first of which will be the value
computed so far and
the second of which will be the value of ~expr~.  It should return the
new value.
|reducing| is roughly equivalent to the
Common Lisp |(reduce ~func~ ~sequence~ :key ~expr-function~)|, where
~expr-function~ is used to derive values from the successive elements
of ~sequence~.

\cpar If the |reducing| clause is never executed, the result is
undefined.

\cpar It is not necessary to provide an
initial value, but better code can be generated if one is supplied.
Regardless of its location in the \iter\ body, ~init-val~ will be
evaluated before the loop is entered, so it should not depend on any
value computed inside the \iter\ form.  

%\cpar if a ~var~ is not specified, you can get \iter\ to declare the
%type of the internal variable by putting a |the| expression around
%~func.~  see section \ref{types}.

\end{clauses}

%\begin{note}{implementation note}
%in principle, |maximize| and |minimize| can be thought of
%as reductions where the initial value is the smallest (or largest)
%value that the accumulation variable can assume.  because lisp's
%bignums can represent arbitrary integers, these clauses cannot be
%implemented as reductions in general.  if, however, the type of 
%~expr~ or ~var~ can be determined to be a fixnum
%or a float, \iter\ will implement the clause as a true reduction,
%using one of the constants |most-negative-fixnum|, |%most-positive-fixnum|, 
%|most-negative-short-float|, etc. as appropriate.
%\end{note}

\subsubsection{Accumulations}

All the predefined accumulation clauses add values to a sequence.  If
the sequence is a list, they all 
have the property that the partial list is kept in the correct order
and available for inspection at any point in the loop.

\begin{clauses}

\clausex{collect}{~expr~ \opt\ |into| ~var~ |at| ~place~ |result-type| ~type~}
Produces a sequence of the values of ~expr~ on each iteration. ~place~
indicates where the next value of ~expr~ is added to the list and may
be one of the symbols |start|, |beginning| (a synonym for |start|) or
|end|.  The symbol may be quoted, but need not be.  The default is
|end|.  For example,
\begin{program}
(iter (for i from 1 to 5)
      (collect i))
\end{program}
produces |(1 2 3 4 5)|, whereas
\begin{program}
(iter (for i from 1 to 5)
      (collect i at beginning))
\end{program}
produces |(5 4 3 2 1)| (and is likely to be faster in most Common Lisp
implementations).

\cpar If ~type~ is provided, it should be a subtype of |sequence|.
The default is |list|.  Specifying a type other than |list| will
result in |collect| returning a sequence of that type.  ~However,~ the
type of the sequence being constructed when inside the loop body is
undefined when a non-|list| type is specified.  (As with ~place~,
quoting ~type~ is optional.)

\clausex{adjoining}{~expr~ \opt\ |into| ~var~ |test| ~test~ |at| ~place~ |result-type| ~type~}
Like |collect|, but only adds the value of ~expr~ if it is not
already present.  ~test,~ which defaults to |\#'eql|, is
the test to be used with |member|.

\startitem
\Clausex{appending}{~expr~ \opt\ |into| ~var~ |at| ~place~}
\Clausex{nconcing}{\ ~expr~ \opt\ |into| ~var~ |at| ~place~}
\Clausex{unioning}{\ ~expr~ \opt\ |into| ~var~ |test| ~test~ |at| ~place~}
\Clausex{nunioning}{~expr~ \opt\ |into| ~var~ |test| ~test~ |at| ~place~}
\finishitem
These are like |collect|, but behave like the Common Lisp functions
|append|, |nconc|, |union| or |nunion|.  
As in Common Lisp, they work only on lists.  Also as in Common Lisp,
|unioning| and |nunioning| assume that the value of ~expr~ contains no
duplicates. 

\clausex{accumulate}{~expr~ |by| ~func~ \opt\ |initial-value| ~init-val~ |into| ~var~}
This is a general-purpose accumulation clause. ~func~ should be a
function of two arguments, the value of ~expr~ and the value
accumulated so far in the iteration, and it should return the updated
value.  If no initial value is supplied, \nil\ is used.  

%\cpar If a ~var~ is not specified, you can get \iter\ to declare the
%type of the internal variable by putting a |the| expression around
%~func.~  see section \ref{types}.

\cpar The differences between |accumulate| and |reducing| are slight.
One difference is that the functions take their arguments in a
different order.  Another is that in the absence of ~init-val~,
|accumulate| will use \nil, whereas |reducing| will generate different
code that avoids any dependence on the initial value.
The reason for having both clauses is that one usually
thinks of reductions (like |sum|) and accumulations (like |collect|)
as different beasts.

\end{clauses}


\subsubsection{Finders}

A ~finder~ is a clause whose value is an expression that meets some
condition.

\begin{clauses}

\clause{|finding| ~expr~ |such-that| ~test~ \opt\ |into| ~var~ |on-failure| ~failure-value~}
\clindexx{finding}{such-that}
If ~test~ (which is an expression) ever evaluates to \nonnil, the loop
is terminated, the 
epilogue code is run and the value of ~expr~ is returned.  Otherwise,
\nil\ (or ~failure-value,~ if provided) is returned.  If ~var~ is
provided, it will have either the \nonnil\ value of ~expr~ or
~failure-value~ when the epilogue code is run.

\cpar As a special case, if the ~test~ expression is a sharp-quoted
function, then it is applied to ~expr~ instead of being simply
evaluated.  E.g. |(finding x such-that \#'evenp)| is equivalent to
|(finding x such-that (evenp x))|.  

%\cpar although ~test~ need have nothing to do with ~%expr~ as in
%|(finding j such-that (> i 3))|, it usually 
%will: |(finding (length el) such-that (oddp (length el)))|.  to
%avoid performing the |length| computation twice, you could write
%|(finding (length el) such-that \#'oddp)| or |(finding (length
%el) such-that 'oddp)|; for these cases, \iter\ generates code that
%executes ~expr~ only once.  the code for |\#'oddp|
%is slightly different from that for {\lisp 'oddp}; see the discussion
%under {\lisp for\dots in} and {\lisp for\dots on}.

\cpar |On-failure| is a misnomer. Because it is always evaluated, it behaves
more like the default third argument to the |gethash| function. As a result,
|on-failure (error "Not found")| makes no sense. Instead, the clauses |leave|
or |thereis| can be used in conjunction with |finally| as follows:
\begin{program}
(iter (for x in '(1 2 3))
      (if (evenp x) (leave x))
      (finally (error "not found")))
\end{program}

\cpar This clause may appear multiple times when all defaults are
identical. It can also be used together with either |always|/|never| or
|thereis| if their defaults match. More specifically, |on-failure nil| is
compatible with |thereis|, while |on-failure t| is compatible with |always|
and |never| clauses.
\begin{program}
(iter (for i in '(7 -4 2 -3))
      (if (plusp i)
	  (finding    i  such-that (evenp i))
          (finding (- i) such-that (oddp i))))
\end{program}

\startitem
\Clause{|finding| ~expr~ |maximizing| ~m-expr~ \opt\ |into| ~var~}
\clindexx{finding}{maximizing}
\Clause{|finding| ~expr~ |minimizing| ~m-expr~ \opt\ |into| ~var~}
\clindexx{finding}{minimizing}
\finishitem
Computes the extremum (maximum or minimum) value of ~m-expr~ over all
iterations, and returns the value of ~expr~ corresponding to the
extremum.  ~expr~ is evaluated inside the loop at the time the new
extremum is established.  If ~m-expr~ is never evaluated (due to, for
example, being embedded in a conditional clause), then the returned
value depends on the type, if any, of ~expr~ (or ~var,~ if
one is supplied).  If there is no type, the returned
value will be \nil; if the type is numeric, the returned value will be
zero.

\cpar For these two clauses, ~var~ may be a list of two
symbols; in that case, the first is used to record ~expr~ and
the second, ~m-expr.~  

\cpar As with |finding\dots such-that|, if ~m-expr~ is a sharp-quoted
function, then it is called on ~expr~ instead of being evaluated.

\end{clauses}

\subsubsection{Boolean Tests}

\begin{clauses}

\clausex{first-iteration-p}{}
Returns |t| in the first cycle of the loop, otherwise \nil.

\clausex{first-time-p}{}
Returns |t| the first time the expression is evaluated, and then \nil\ forever.
This clause comes handy when printing (optional) elements separated
by a comma:

\begin{program}
(iter (for el in '(nil 1 2 nil 3))
      (when el
        (unless (first-time-p)
          (princ ", "))
        (princ el)))
\end{program}
produces |"1, 2, 3"|.

\end{clauses}

\subsubsection{Aggregated Boolean Tests}

\begin{clauses}

\clausex{always}{~expr~}
If ~expr~ ever evaluates to
\nil, then \nil\ is immediately returned; the epilogue code is not
executed.  If ~expr~ never evaluates to \nil, the epilogue code
is executed and the last value of ~expr~ (or |t| if ~expr~ was never
evaluated) is returned (whereas |loop| would constantly return |t|).

% mention last evaluated clause when multiple always clauses?

\clausex{never}{~expr~}
Like |(always (not ~expr~))|, except it does not influence the last
value returned by a possible other |always| clause. That is,
\begin{program}
(iter (repeat 2)
      (always 2)
      (never nil)) \yields 2 ; not t
\end{program}

\clausex{thereis}{~expr~}
If ~expr~ is ever \nonnil,
its value is immediately returned without running epilogue code.
Otherwise, the epilogue code is performed and \nil\ is returned.

This clause cannot be used together with |always| or |never|, because their
defaults are opposed (similarly, |(loop always 3 thereis nil)| refuses to
compile in some implementations of |loop|).

\end{clauses}


\subsection{Control Flow}
\label{control-flow}
Several clauses can be used to alter the usual flow of control in a loop.

Note: the clauses of this and subsequent sections don't adhere to \iter's
usual syntax, but instead use standard Common Lisp syntax.  Hence the
format for describing syntax subsequently is
like the standard format used in the Common Lisp manual, not like the
descriptions of clauses above.

\begin{clauses}

\clausex{finish}{}
Stops the loop and runs the epilogue code.

%for example:
%
%\begin{program}
%(iter (with answer = nil)
%         (initially (make-a-mess))
%         (for i from 1 to 10)
%         (when (correct? i)
%           (setq answer i)
%           (finish))
%         (finally (cleanup)))
%\end{program}
%
%this code will execute |cleanup| whether or not the test |(correct?
%i)| ever succeeds. 
%the (more elegant) formulation,
%\begin{program}
%(iter (initially (make-a-mess))
%         (for i from 1 to 10)
%         (finding i such-that (correct? i))
%         (finally (cleanup)))
%\end{program}
%would not execute |cleanup| if |(correct? i)| succeeded; it
%would do an immediate return.

\clausex{leave}{\opt\ ~value~}
Immediately returns ~value~ (default \nil) from the current \iter\
form, skipping the epilogue code.  Equivalent to using |return-from|.

\clausex{next-iteration}{}
Skips the remainder of the loop body and begins the next iteration of
the loop.

\clausex{while}{~expr~}
If ~expr~ ever evaluates to \nil, the loop is terminated and the
epilogue code executed.  Equivalent to |(if (not ~expr~) (finish))|.

\clausex{until}{~expr~}
Equivalent to |(if ~expr~ (finish))|.

\clausex{if-first-time}{~then~ \opt\ ~else~}
If this clause is being executed for the first time in this invocation
of the \iter\ form, then the ~then~ code is evaluated; otherwise the
~else~ code is evaluated.

\cpar |(for ~var~ first ~expr1~ then ~expr2~)| is almost equivalent to
\begin{program}
(if-first-time (dsetq ~var~ ~expr1~)
               (dsetq ~var~ ~expr2~))
\end{program}
The only difference is that the |for| version makes ~var~ available
for use with |for\dots previous|.

\end{clauses}

\subsection{Code Placement}
When fine control is desired over where code appears in a loop
generated by \iter, the following special clauses may be useful.
They are all subject to code-motion problems (see section
\ref{code-movement}).  

\begin{clauses}

\clausex{initially}{|\&rest| ~forms~}
The lisp ~forms~ are placed in the prologue section of the loop, where
they are executed once, before the loop body is entered.

\clausex{after-each}{|\&rest| ~forms~}
The ~forms~ are placed at the end of the loop body, where they
are executed after each iteration.  Unlike the other clauses in this
section, ~forms~ may contain \iter\ clauses.

\clausex{else}{|\&rest| ~forms~}
The lisp ~forms~ are placed in the epilogue section of the loop, where they
are executed if this |else| clause is never met during execution of the
loop and the loop terminates normally.

\clausex{finally}{|\&rest| ~forms~}
The lisp ~forms~ are placed in the epilogue section of the loop, where
they are executed after the loop has terminated normally.

\clausex{finally-protected}{|\&rest| ~forms~}
The lisp ~forms~ are placed in the second form of an unwind-protect
outside the loop.  They are always executed after the loop has
terminated, regardless of how the termination occurred.

\end{clauses}


%%%%%%%%%%%%%%%%%%%%%%%%%%%%%%%%%%%%%%%%%%%%%%%%%%%%%%%%%%%%%%%%
\section{Other Features}

\subsection{Multiple Accumulations}
\label{multiple}

\begin{sloppypar}
It is permitted to have more than one clause accumulate into the same
variable, as in the following:
\begin{program}
(iter (for i from 1 to 10)
      (collect i into nums)
      (collect (sqrt i) into nums)
      (finally (return nums)))
\end{program}
Clauses can only accumulate into the same variable if they are
compatible.  |collect|, |adjoining|, |appending|, |nconcing|,
|unioning| and |nunioning| are compatible with each other; |sum|,
|multiply| and |counting| are compatible; |always| and |never| are compatible;
|finding| \dots |such-that| is compatible with either |thereis| or |always|
and |never| when their defaults match; and |maximize| and |minimize| clauses
are compatible only with other |maximize| and |minimize| clauses,
respectively.

%note that the same variable ~cannot~ be both an accumulation
%variable and an ordinary variable; there can be only one variable with
%a given name within an \iter\ form. 
\end{sloppypar}

\subsection{Named Blocks}

Like Common Lisp |block|s, \iter\ forms can be given names.  The
name should be a single symbol, and it must be the first form in the
\iter.  The generated code behaves exactly like a named block; in
particular, |(return-from ~name~)| can be used to exit it:
\begin{program}
(iter fred
      (for i from 1 to 10)
      (iter barney
            (for j from i to 10)
            (if (> (* i j) 17)
                (return-from fred j))))
\end{program}
An \iter\ form that is not given a name is implicitly named \nil.

Sometimes one would like to write an expression in an inner \iter\ form,
but have it processed by an outer \iter\ form.  This is possible with
the |in| clause.  

\begin{clauses}

\clausex{in}{~name~ |\&rest| ~forms~}
Evaluates ~forms~ as if they were part of the \iter\ form named
~name~.  In other words, \iter\ clauses are processed by the \iter\
form named ~name,~ and not by any \iter\ forms that occur inside ~name.~

\cpar As an example, consider the problem of collecting a list of the
elements in a two-dimensional array.  The naive solution,
\begin{program}
(iter (for i below (array-dimension ar 0))
      (iter (for j below (array-dimension ar 1))
            (collect (aref ar i j))))
\end{program}
\noindent is wrong because the list created by the inner \iter\ is simply
ignored by the outer one.  But using |in| we can write:
\begin{program}
(iter outer (for i below (array-dimension ar 0))
      (iter (for j below (array-dimension ar 1))
            (in outer (collect (aref ar i j)))))
\end{program}
\noindent which has the desired result.

\end{clauses}

\subsection{Destructuring}
\label{destructuring}

In many places within \iter\ clauses where a variable is expected, a
list can be written instead.  In these cases, the value to be assigned
is ~destructured~ according to the pattern
described by the list.  As a simple example, the clause
\begin{program}
(for (key . item) in alist)
\end{program}
\noindent will result in |key| being set to the |car| of
each element in |alist|, and |item| being set to the |cdr|.  The
pattern list may be nested to arbitrary depth, and (as the example
shows) need not be terminated with \nil; the only requirement is that
each leaf be a bindable symbol (or \nil, in which case no binding is
generated for that piece of the structure).

Sometimes, you might like to do the equivalent of a
|multiple-value-setq| in a clause.  This 
``multiple-value destructuring'' can be expressed by writing 
\linebreak |(values $pat_1$ $pat_2 \ldots$)| for a destructuring
pattern, as in 
\begin{program}
(for (values (a . b) c d) = (three-valued-function ...))
\end{program}
\begin{sloppypar}
Note that the $pat_i$ can themselves be destructuring patterns (though
not multiple-value destructuring patterns).  You can't do multiple-value
destructuring in a |with| clause; instead wrap the whole \iter\
form in a |multiple-value-bind|.
\end{sloppypar}

\begin{note}{Rationale}
There are subtle interactions between variable declarations and
evaluation order that make the correct implementation of
multiple-value destructuring in a |with| somewhat tricky.
\end{note}

The destructuring feature of \iter\ is available as a separate
mechanism, using the |dsetq| macro:

\begin{clauses}
\defmacro{dsetq}{template expr}
Performs destructuring of ~expr~ using ~template~.   May be used
outside of an \iter\ form. Yields the primary value of ~expr~.

\end{clauses}

\subsection{On-line Help}

\begin{sloppypar}
There is a limited facility for on-line help, in the form of the 
|display-iterate-clauses| function.
\end{sloppypar}

\begin{clauses}

\defun{display-iterate-clauses}{\opt\ clause-spec}
Displays a list of \iter\ clauses.  If ~clause-spec~ is not
provided, all clauses are shown; if it is a symbol, all clauses
beginning with that symbol are shown; and if it is a list of symbols,
all clauses for which ~clause-spec~ is a prefix are shown.

\end{clauses}

\subsection{Parallel Binding and Stepping}
\label{bindings}

The parallel binding and stepping of variables is a feature that
\iter\ does ~not~ have.  This section attempts to provide a rationale.

We say that two variables are bound ~in parallel~ if neither
binding shadows the other.  This is the usual semantics of |let|
(as opposed to |let*|).  Similarly, we can say that iteration
variables are stepped in parallel if neither variable is updated
before the other, conceptually speaking; in other words, if the code
to update each variable can reference the old values of both variables.

|loop| allows parallel binding of variables and parallel stepping of
driver variables.  My view is that if you are depending on the 
serial/parallel distinction, you are doing something obscure.  If you
need to bind 
variables in parallel using |with|, then you must be using a
variable name that shadows a name in the existing lexical environment.
Don't do that.  The most common use for parallel stepping is to track
the values of variables on the previous iteration, but in fact this
does not require parallel stepping at all; the following will work:
\begin{program}
(iter (for current in list)
      (for prev previous current)
      ...)
\end{program}


%%%%%%%%%%%%%%%%%%%%%%%%%%%%%%%%%%%%%%%%%%%%%%%%%%%%%%%%%%%%%%%%
\section{Types and Declarations}
\label{types}

\subsection{Discussion}

Sometimes efficiency dictates that the types of variables be declared.
This type information needs to be communicated to \iter\ so it can
bind variables to appropriate values.  Furthermore, \iter\ must often
generate internal variables invisible to the user; there needs to be a
way for these to be declared.

As an example, consider this code, which will return the number of
odd elements in |number-list|:
\begin{program}
(iter (for el in number-list)
      (count (oddp el)))
\end{program}
In processing this form,
\iter\ will create an internal variable, let us call it |list17|, to
hold the successive |cdr|s of |number-list|, and  
will bind the variable to |number-list|.  It will also generate a
default binding for |el|; only inside the body of the loop will |el|
be set to the |car| of |list17|.  Finally, \iter\ will generate a
variable, call it |result|, to hold the result of the count, and will
bind it to zero.

When dealing with type declarations, \iter\ observes one simple rule:
~it will never generate a declaration unless requested to do so.~  The
reason is that such declarations might mask errors in compiled code by
avoiding error-checks; the resulting problems would be doubly hard to
track down because the declarations would be hidden from the
programmer.  Of course, a compiler might omit error-checks even in the
absence of declarations, though this behavior can usually be avoided,
e.g. by saying |(declaim (optimize (safety 3)))|.

So, the above \iter\ form will generate code with no declarations.
But say we wish to declare the types of |el| and the internal
variables |list17| and |result|.  How is this done?

Declaring the type of |el| is easy, since the programmer knows
the variable's name:
\begin{program}
(iter (for el in number-list)
      (declare (fixnum el))
      (counting (oddp el)))
\end{program}
\iter\ can read variable type declarations like this one.  Before
processing any clauses, it scans the entire top-level form for type
declarations 
and records the types, so that variable bindings can be performed
correctly.  In this case, |el| will be bound to zero
instead of \nil.  Also, \iter\ collects all the top-level declarations
and puts them at the begining of the generated code, so it is not
necessary to place all declarations at the beginning of an \iter\
form; instead, they can be written near the variables whose types they
declare.  

Since \iter\ is not part of the compiler, it will not know
about declarations that occur outside an \iter\ form; these
declarations must be repeated inside the form.

Here is another way we could have declared the type of |el|:
\begin{program}
(iter (for (the fixnum el) in number-list)
      (counting (oddp el)))
\end{program}
\iter\ extends the Common Lisp |the|\lispindex{the}
form to apply to variables as well as value-producing forms; anywhere
a variable is allowed---in a |with| clause, as the iteration
variable in a driver clause, as the |into| argument of an
accumulation clause, even inside a destructuring template---you can
write |(the ~type~ ~symbol~)| instead.

There is one crucial difference between using a |the| form and
actually declaring the variable: explicit declarations are always
placed in the generated code, but type information from a |the|
form is not turned into an actual declaration unless you tell \iter\
to do so using |iterate:declare-variables|.  See below.

\begin{sloppypar}
Declaring the types of internal variables is harder than declaring the
types of explicitly mentioned variables, since their names
are unknown.  You do it by declaring |iterate:declare-variables|
somewhere inside the top level of the \iter\ form.  (This will also
generate declarations for variables declared using |the|.)
\iter\ does not provide much selectivity here: it's all or none. 
And unfortunately, since \iter\ is not privy to compiler information
but instead reads declarations itself, it will not hear if you
|(declaim (iterate:declare-variables))|.  Instead, set the variable 
|iterate::*always-declare-variables*| to |t| at
compile-time, using |eval-when|.
\end{sloppypar}

To determine the appropriate types for internal variables, \iter\ uses
three sources of information:

\begin{itemize}

\item Often, the particular clause dictates a certain type for a
variable; \iter\ will use this information when available.  In the
current example, the variable |list17| will be given the type
|list|, since that is the only type that makes sense; and the
variable |result| will be given the type |fixnum|, on the
assumption that you will not be counting high enough to need bignums.
You can override this assumption only by using and explicitly declaring a
variable: 
\begin{verbatim}
(iter (declare (iterate:declare-variables))
      (for el in number-list)
      (count (oddp el) into my-result)
      (declare (integer my-result))
      (finally (return my-result)))
\end{verbatim}

\begin{sloppypar}
Other examples of the type assumptions that \iter\ makes are: type
|list| for |into| variables of collection clauses; type |list| for
expressions that are to be destructured; type |vector| for the
variable holding the vector in a |for\dots in-vector| clause, and
similarly for |string| and the |for\dots in-string| clause;
and the implementation-dependent type
|(type-of array-dimension-limit)| for the index and limit
variables generated by sequence iteration drivers like |for\dots
in-vector| and |for\dots in-string| (but not |for\dots in-sequence|,
because it may be used to iterate over a list). 
\end{sloppypar}

\item Sometimes, \iter\ will examine expressions and try to determine
their types in a simple-minded way.  If the expression is
self-evaluating (like a number, for instance), \iter\ knows that the
expression's type is the same as the type of the value it denotes, so
it can use that type.  If the expression is of the form |(the ~type~
~expr~)|, \iter\ is smart enough to extract ~type~ and use it.
However, the current version of \iter\ does not  
examine declarations of function result types or do any type
inference.  It will not determine, for 
example, that the type of |(+ 3 4)| is |fixnum|, or even
|number|.

\item In some cases, the type of an internal variable should match the
type of some other variable.  For instance, \iter\ generates an
internal variable for |(f x)| in the
clause |(for i from 1 to (f x))|, and in the absence of other
information will give it the same type as |i|.  If, however, the
expression had been written |(the fixnum (f x))|, then \iter\
would have given the internal variable the type |fixnum|
regardless of |i|'s type.  The type incompatibility errors that
could arise in this situation are not checked for.

\end{itemize}

Note that if you do declare |iterate:declare-variables|, then
\iter\ may declare user variables as well as internal ones if they do
not already have declarations, though only for variables that it
binds.  For instance, in this code:

\begin{program}
(iter (declare (iterate:declare-variables))
      (for i from 1 to 10)
      (collect i into var))
\end{program}
the variable |var| will be declared to be of type |list|.


\subsection{Summary}

\iter\ understands standard Common Lisp variable type declarations
that occur within an \iter\ form and
will pass them through to the generated code.  If the declaration
|(iterate:declare-variables)|\lispindex{declare-variables}
appears at the top level of an 
\iter\ form, or if 
|iterate::*always-declare-variables*|\lispindex{*always-declare-variables*} 
is \nonnil, then \iter\ will use the type information gleaned from user
declarations, self-evaluating expressions and |the| expressions,
combined with reasonable assumptions, to determine variable
types and declare them.


%%%%%%%%%%%%%%%%%%%%%%%%%%%%%%%%%%%%%%%%%%%%%%%%%%%%%%%%%%%%%%%%
\section{Problems with Code Movement}
\label{code-movement}

Some \iter\ clauses, or parts of clauses, result in code being
moved from the location of the clause to other parts of the loop.
Drivers behave this way, as do code-placement clauses like |initially|
and |finally|.  When using these clauses, there is a danger of writing
an expression that makes sense in its apparent location but will be
invalid or have a different meaning in another location.  For example:
\begin{program}
(iter (for i from 1 to 10)
      (let ((x 3))
        (initially (setq x 4))))
\end{program}
While it may appear that the |x| of |(initially (setq x 4))| is the
same as the |x| of |(let ((x 3)) \dots|, in fact they are not:
|initially| moves its code outside the loop body, so |x| would refer
to a global variable.  Here is another example of the same problem:
\begin{program}
(iter (for i from 1 to 10)
      (let ((x 3))
        (collect i into x)))
\end{program}
If this code were executed, |collect| would create a binding for its
|x| at the top level of the \iter\ form that the |let| will shadow.

Happily, \iter\ is smart enough to catch these errors; it
walks all problematical code to ensure that free variables are not
bound inside the loop body, and checks all variables it binds for the
same problem.

However, some errors cannot be caught:

\begin{program}
(iter (with x = 3)
      (for el in list)
      (setq x 1)
      (reducing el by \#'+ initial-value x))
\end{program}
|reducing| moves its |initial-value| argument to the initialization
part of the loop in order to produce more efficient code.  Since
\iter\ does not perform data-flow analysis, it cannot determine that
|x| is changed inside the loop; all it can establish is that |x| is
not bound internally.  Hence this code will not signal an
error and will use $3$ as the initial value of the reduction.

The following list summarizes all cases that are subject to these code
motion and variable-shadowing problems.
\begin{itemize}
\item Any variable for which \iter\ creates a binding, including those
used in |with| and the |into| keyword of many clauses.

\begin{sloppypar}
\item The special clauses which place code: |initially|, |after-each|, |else|,
|finally| and |finally-protected|. 
\end{sloppypar}

\item The variables of a |next| or |do-next| form.

\item The |initially| arguments of |for\dots initially\dots then| and
|for\dots previous|. 

\item The |then| argument of |for\dots initially\dots then|.

\item The |initial-value| arguments of |reducing| and |accumulate|.

\item The |on-failure| argument of |finding\dots such-that|.
\end{itemize}

%%%%%%%%%%%%%%%%%%%%%%%%%%%%%%%%%%%%%%%%%%%%%%%%%%%%%%%%%%%%%%%%
\section{Differences Between {\tt Iterate} and {\tt Loop}}
|loop| contains a great deal of complexity which \iter\ tries to
avoid.  Hence many esoteric features of |loop| don't exist in \iter.
Other features have been carried over, but in a cleaned-up form.  
And of course, many new features have been added; they are not
mentioned in this list.

\begin{itemize}

\item \iter's syntax is more Lisp-like than |loop|'s, having a higher
density of parens.

\item The current implementation of \iter, unlike the current version
of  |loop| (as documented in {\em Common Lisp, 2nd Ed.\/}), is
extensible (see section \ref{extend}).  

\item |loop| puts the updates of all driver variables at the top of
the loop; \iter\ leaves them where the driver clauses appear.

\item While for the most part \iter\ clauses that resemble |loop| clauses
behave similarly, there are some differences.  For instance, there is
no |for\dots =\dots then| in \iter; instead use 
|for\dots initially\dots then|.  

\item |loop| binds the variable |it| at certain times
to allow pseudo-English expressions like |when ~expr~ return it|. 
In \iter, you must bind ~expr~ to a variable yourself.  Note that
|when ~expr~ return it| is like |thereis ~expr~| except that the latter is an
accumulation clause and therefore competes with other accumulations
(remember section \ref{multiple} above).

% repeat different behaviour of |always| clause here?

\item |loop| has a special |return| clause, illustrated in the
previous item.  \iter\ doesn't need one,  since an ordinary Lisp
|return| has the same effect.

\item |loop| allows for parallel binding and stepping of iteration
variables.  \iter\ does not.  (See section \ref{bindings}.)

\item |loop| and \iter\ handle variable type declarations very
differently.  |loop| provides a special syntax for declaring variable
types, and does not examine declarations.  Moreover, the standard
implementation of |loop| will
generate declarations when none are requested.
\iter\ parses standard Common Lisp type declarations, and will never
declare a variable itself unless declarations are 
specifically requested.

\end{itemize}

%%%%%%%%%%%%%%%%%%%%%%%%%%%%%%%%%%%%%%%%%%%%%%%%%%%%%%%%%%%%%%%%
\section{Rolling Your Own}
\label{extend}

\subsection{Introduction}

\iter\ is extensible---you can write new clauses that embody new
iteration patterns.  You might want to write a new driver clause for a
data structure of your own, or you might want to write a clause that
collects or manipulates elements in a way not provided by \iter. 

This section describes how to write clauses for \iter.  Writing a
clause is like writing a macro.  In fact, writing a clause ~is~
writing a macro: since \iter\ code-walks its body and macroexpands,
you can add new abstractions to \iter\ with good old |defmacro|.

Actually, there are two extensions you can make to \iter\ that are
even easier than writing a macro.  They are adding a synonym for an
existing clause and defining a driver clause for an indexable
sequence.  These can be done with |defsynonym| and
|defclause-sequence|, respectively.  See section \ref{aids}, below.

The rest of this section explains how to write macros that expand into
\iter\ clauses.
Here's how you could add a simplified version of \iter's |multiply|
clause, if \iter\ didn't already have one:
\begin{program}
(defmacro multiply (expr)
  `(reducing ,expr by \#'* initial-value 1))
\end{program}

If you found yourself summing the square of an expression often, you
might want to write a macro for that.  A first cut might be
\begin{program}
(defmacro sum-of-squares (expr)
  `(sum (* ,expr ,expr)))
\end{program}
but if you are an experienced macro writer, you will realize that this
code will evaluate ~expr~ twice, which is probably a bad idea.  A
better version would use a temporary:
\begin{program}
(defmacro sum-of-squares (expr)
  (let ((temp (gensym)))
    `(let ((,temp ,expr))
       (sum (* ,temp ,temp)))))
\end{program}
Although this may seem complex, it is just the sort of thing you'd
have to go through to write any macro, which illustrates the point of
this section: if you can write macros, you can extend \iter.

Our macros don't use \iter's keyword-argument
syntax.  We could just use keywords with |defmacro|, but we would
still not be using \iter's clause indexing mechanism.  Unlike Lisp,
which uses just the first symbol of a form to determine what function
to call, \iter\ individuates clauses by the list of required keywords.
For instance, |for\dots in| and |for\dots in-vector| are different clauses
implemented by distinct Lisp functions.  

To buy into this indexing scheme, as well as the keyword-argument
syntax, use |defmacro-clause|:

\begin{clauses}

\defmacro{defmacro-clause}{arglist |\&body| body}
Defines a new \iter\ clause.  ~arglist~ is a list of symbols which are
alternating keywords and arguments.  \opt\ may be used, and the list
may be terminated by |\&sequence|.  ~body~ is an ordinary macro body,
as with |defmacro|.  If the first form of ~body~ is a string, it is
considered a documentation string and will be shown by
|display-iterate-clauses|. |defmacro-clause| will signal an error if 
defining the clause would result in an ambiguity.  E.g. you cannot
define the clause |for\dots from| because there would be no way to
distinguish it from a use of the |for| clause with optional keyword |from|.

\end{clauses}

\medskip

Here is |multiply| using |defmacro-clause|.  The keywords are capitalized
for readability.
\begin{program}
(defmacro-clause (MULTIPLY expr \opt\ INTO var)
  `(reducing ,expr by \#'* into ,var initial-value 1))
\end{program}
You don't have to worry about the case when |var| is not supplied; for
any clause with an |into| keyword, saying |into nil| is equivalent to
omitting the |into| entirely.

As another, more extended example, consider the fairly common
iteration pattern that 
involves finding the sequence element that maximizes (or minimizes) some
function.  \iter\ provides this as |finding\dots maximizing|, but it's
instructive to see how to write it.
Here, in pseudocode, is how you might write such a loop for
maximizing a function F:

\begin{quote}
\begin{tabbing}
set variable MAX-VAL to NIL; \\
set variable WINNER to NIL; \\
for\=\ each element EL in the sequence \\
\>  if\=\ MAX-VAL\=\ is NIL or F(EL) $>$ MAX-VAL then \\
\>\>    set MAX-VAL to F(EL); \\
\>\>    set WINNER to EL; \\
\>  end if; \\
end for; \\
return WINNER.
\end{tabbing}
\end{quote}

Here is the macro:
\begin{program}
(defmacro-clause (FINDING expr MAXIMIZING func \opt\ INTO var)
  (let ((max-val (gensym))
        (temp1 (gensym))
        (temp2 (gensym))
        (winner (or var iterate::*result-var*)))
    `(progn 
       (with ,max-val = nil)
       (with ,winner = nil)
       (cond 
        ((null ,max-val)
         (setq ,winner ,expr)
         (setq ,max-val (funcall ,func ,winner))
        (t
         (let* ((,temp1 ,expr)
                (,temp2 (funcall ,func ,temp1)))
           (when (> ,temp2 ,max-val)
             (setq ,max-val ,temp2)
             (setq ,winner ,temp1))))))
       (finally (leave ,winner)))))
\end{program}
Note that if no |into| variable is supplied, we use
|iterate::*result-var*|, which contains the internal variable into
which all clauses place their results.  If this variable is bound by
some clause, then \iter\ will return its value automatically;
otherwise, \nil\ will be returned.  

\subsection{Writing Drivers}

In principle, drivers can be implemented just as easily as other
\iter\ clauses.  In practice, they are a little harder to get right. 
As an example, consider writing a driver that
iterates over all the 
elements of a vector, ignoring its fill-pointer.  |for\dots in-vector|
won't work for this, because it observes the fill-pointer.  It's necessary to
use |array-dimension| instead of |length| to obtain the size of the
vector.  Here is one approach:

\begin{program}
(defmacro-clause (FOR var IN-WHOLE-VECTOR v)
  "All the elements of a vector (disregards fill-pointer)"
  (let ((vect (gensym))
        (index (gensym)))
    `(progn
       (with ,vect = ,v)
       (for ,index from 0 below (array-dimension ,vect 0))
       (for ,var = (aref ,vect ,index)))))
\end{program}
Note that we immediately put |v| in a variable, in case it is an
expression.  Again, this is just good Lisp macrology.  It also has a
subtle effect on the semantics of the driver: |v| is evaluated only
once, at the beginning of the loop, so changes to |v| in the loop have
no effect on the driver.  Similarly, the bounds for numerical iteration
e.g. the above |array-dimension| are also evaluated once only.  This is how
all of \iter's drivers work.


There is an important point concerning the |progn| in this code.  We
need the |progn|, of course, because we are returning several forms,
one of which is a driver.  But \iter\ drivers must occur at top-level.
Is this code in error?  No, because ~top-level~ is defined in \iter\
to include forms inside a |progn|.  This is just the definition of
top-level that Common Lisp uses, and for the same reason: to allow
macros to return multiple forms at top-level.

While our |for\dots in-whole-vector| clause will work, it is not
ideal.  In particular, it does not support generating.  Do do so, we
need to use |for\dots next| or |for\dots do-next|.  The job is
simplified by the |defmacro-driver| macro.

\begin{clauses}

\defmacro{defmacro-driver}{arglist |\&body| body}
Defines a driver clause in
both the |for| and |generate| forms, and provides a parameter
|generate| which ~body~ can examine to determine how it was invoked.
~arglist~ is as in |defmacro-clause|, and should begin with the symbol
|for|. 

\end{clauses}

With |defmacro-driver|, our driver looks like this:
\begin{program}
(defmacro-driver (FOR var IN-WHOLE-VECTOR v)
  "All the elements of a vector (disregards fill-pointer)"
   (let ((vect (gensym))
         (end (gensym))
         (index (gensym))
         (kwd (if generate 'generate 'for)))
     `(progn
        (with ,vect = ,v)
        (with ,end = (array-dimension ,vect 0))
        (with ,index = -1)
        (,kwd ,var next (progn (incf ,index)
                               (if (>= ,index ,end) (terminate))
                               (aref ,vect ,index))))))
\end{program}


We are still missing one thing: the |\&sequence| keywords.  
We can get them easily enough, by writing 
\begin{program}
(defmacro-driver (FOR var IN-WHOLE-VECTOR v \&sequence) 
  ...)
\end{program}
We can now refer to parameters |from|, |to|, |by|, etc. which contain
either the values for the corresponding keyword, or \nil\ if the
keyword was not supplied.  Implementing the right code for these
keywords is cumbersome but not difficult; it is left as an exercise.
But before you begin, see |defclause-sequence| below for an easier way.
       
\subsection{Extensibility Aids}
\label{aids}

This section documents assorted features that may be of use in
extending \iter.

\begin{clauses}

\defunexpvar{*result-var*}
Holds the variable that is used to return a value as a result of the
\iter\ form.  You may examine this and use it in a |with| clause, but
you should not change it.

\defmacro{defsynonym}{syn word}
 Makes ~syn~ a synonym for the existing \iter\ keyword ~word.~  Only
the first word in each clause can have synonyms. 

\defmacroxx{defclause-sequence}{element-name index-name}{|\&key|
access-fn size-fn sequence-type}{element-type
element-doc-string index-doc-string}  
Provides
a simple way to define sequence clauses.  Generates two
clauses, one for iterating over the sequence's elements, the other
for iterating over its indices.  The first symbol of both
clauses will have print-name |for|.
~element-name~  and ~index-name~ should be symbols.
~element-name~ is the second keyword of the element iterator (typically of
the form 
|in-~sequence-type~|), and ~index-name~ is the second keyword
of the index-iterator (typically of the form
|index-of-~sequence-type~|).  Either name may be 
\nil, in which case the corresponding clause is not defined.  If both
symbols are supplied, they should be in the same package.  The |for|
that begins the clauses will be in this package.

\cpar ~access-fn~ is the function to be used to
access elements of the sequence in the element iterator.  The function
should take two 
arguments, a sequence and an index, and return the appropriate element.
~size-fn~ should denote a function of one argument, a sequence, that
returns its size.  Both ~access-fn~ and ~size-fn~ are required for the
element iterator, but only ~size-fn~ is needed for the index iterator.

\cpar The ~sequence-type~ and ~element-type~ keywords are used to
suggest types for the variables 
used to hold the sequence and the
sequence elements, respectively.  The usual rules about \iter's
treatment of variable type declarations apply (see section \ref{types}).

\cpar ~element-doc-string~ and ~index-doc-string~ are
the documentation strings, for use with |display-iterate-clauses|. 

\cpar The generated element-iterator performs destructuring on the
element variable.

\cpar As an example, the above |for\dots in-whole-vector| example
could have been written:
\begin{program}
(defclause-sequence IN-WHOLE-VECTOR INDEX-OF-WHOLE-VECTOR
  :access-fn 'aref
  :size-fn \#'(lambda (v) (array-dimension v 0))
  :sequence-type 'vector
  :element-type t
  :element-doc-string 
     "Elements of a vector, disregarding fill-pointer"
  :index-doc-string 
     "Indices of vector, disregarding fill-pointer")
\end{program}

\end{clauses}

\subsection{Subtleties}

There are some subtleties to be aware of when writing \iter\ clauses.
First, the code returned by your macros may be |nconc|'ed into a list,
so you should always returned freshly consed lists, rather than
constants.  Second, \iter\ matches clauses by using |eq| on the first
symbol and |string=| on the subsequent ones, so the package of the
first symbol of a clause is relevant.  All of the clauses in this manual
have their first word in the \iter\ package.  
You can use the package system in the usual way to shadow
\iter\ clauses without replacing them.

%%% say more here, about the badness that only the first word of a
%%% clause is packagey.

\section{Non-portable Extensions to Iterate (Contribs)}
\label{contribs}

Currently, there is only one non-portable extension to iterate in the
distribution: iterate-pg. If you have made an extension that depends
on non-portable features, feel free to send them to |asf@boinkor.net|
for inclusion in the iterate distribution.

\subsection{An SQL query driver for iterate}

The pg package by Eric Marsden (see |http://cliki.net/pg|) provides an
interface to the PostgreSQL database. Using the \iterpg\ extension, it
is possible to handle the results of SQL queries using \iter.

This usage example should give you an idea of how to use it:

\begin{program}
(pg:with-pg-connection (c "somedb" "someuser")
  (iter (for (impl version date) in-relation "select * from version"
                                 on-connection *dbconn*)
        (collect version)))
\end{program}

To use the extension via |ASDF|, simply make your system depend on the
|iterate-pg| system instead of the |iterate| system. To load it
manually, use:

\begin{program}
  (asdf:oos 'asdf:load-op :iterate-pg)
\end{program}

%%%%%%%%%%%%%%%%%%%%%%%%%%%%%%%%%%%%%%%%%%%%%%%%%%%%%%%%%%%%%%%%
\section{Obtaining {\tt Iterate}}

\begin{sloppypar}
The information in this chapter is obsolete but included for
completeness's sake; Currently, the most up-to-date information on
\iter\ can be found at |http://boinkor.net/iterate.html|.
\end{sloppypar}

\begin{sloppypar}
\iter\ currently runs on Lisp Machines, and on
HP's, Sun3's and Sparcstations under Lucid.  
\iter\ source and binaries are available at the MIT AI Lab in the
subdirectories of |/src/local/lisplib/|.  The source file,
|iterate.lisp|, is also available for anonymous FTP in the directory
|/com/fpt/pub/| on the machine |TRIX.AI.MIT.EDU| (Internet number
128.52.32.6).  If you are unable to obtain |iterate| in one of these
ways, send mail to |jba@ai.mit.edu| and I will send you the source
file. 
\end{sloppypar}

\begin{sloppypar}
\iter\ resides in the |iterate| package (nickname |iter|).  Just say
\linebreak |(use-package :iterate)| to make all the necessary symbols
available. 
If a symbol is not exported, it appears in this manual with an
``|iterate::|'' prefix.
\end{sloppypar}

Send bug reports to |bug-iterate@ai.mit.edu|.  The |info-iterate|
mailing list will have notices of changes and problems; to have
yourself added, send mail to |info-iterate-request@ai.mit.edu|.


\medskip

\begin{flushleft}
 \bf Acknowledgements
\end{flushleft}
\smallskip

Richard Waters provided invaluable criticism which spurred me to improve
\iter\ greatly.  As early users, David Clemens, Oren Etzioni and Jeff
Siskind helped ferret out many bugs.

%\begin{theindex}
% The index files must be generated with the genindex program
% or from makeindex iterate-manual.idx which creates iterate-manual.ind.

%\baselineskip\normalbaselineskip
%\advance\baselineskip by -2pt

\input{iterate-manual.ind}

%\underline{Clauses}
%\input{iterate-manual.clindex}
%
%\indexspace
%\underline{Lisp}
%\input{iterate-manual.lispindex}

%\end{theindex}


\end{document}


% arch-tag: "c0f871a6-313c-11d8-abb9-000c76244c24"


%\underline{Clauses}
%% Fix ``.'' spacing in lists
\documentstyle[12pt,aimemo]{article}

\setlength{\textwidth}{6.0in}
\setlength{\oddsidemargin}{0.25in}
\setlength{\evensidemargin}{0.25in}
\setlength{\topmargin}{0pt}
\setlength{\headsep}{24pt}
\setlength{\textheight}{8.5in}

\let\origpar=\par

% somehow, LaTex fucks with things so that the plain Tex \obeyspaces
% doesn't work.  This is the same, except we use an mbox containing
% a space instead of just a space.

\newcommand{\spc}{\mbox{ }}
\newcommand{\tAb}{\mbox{        }}
\def\obspaces{\catcode`\ =\active\catcode`\^^I=\active}
{\obspaces \global\let =\spc\global\let^^I=\tAb}

\newenvironment{program}{\medskip\samepage\tt\obeylines\obspaces}{\medskip}

% use | for \tt
\catcode`\|=13
\newif\ifvbar
\def|{\ifvbar
         \endgroup\vbarfalse
      \else 
         \begingroup\tt\vbartrue 
      \fi}



\newcommand{\lisp}{\tt}
\newcommand{\iter}{|iterate|}
\newcommand{\iterpg}{|iterate-pg|}
\newcommand{\setf}{|setf|}
\newcommand{\nil}{|nil|}
\newcommand{\nonnil}{non-\nil}
\newcommand{\opt}{{\lisp \&optional}}
\newcommand{\yields}{$\Rightarrow$}



\newbox\Dots
\setbox\Dots=\hbox{\small ...}
\renewcommand{\dots}{\copy\Dots}

% Here's a way to define an environment which will treat all
% paragraphs inside it in the same manner.  It's, as near as I can
% figure, the way that Latex does things.  Using \everypar won't work;
% apparently Latex resets that a lot.  What you have to do is: at the
% top of the file do \let\origpar=\par.  In the
% begin part of the environment, do the parshape (or whatever), then say
% \def\par{{\origpar}}.  The extra braces are necessary.   That's all
% you have to do; at the end of the environment things will return to
% normal automatically.  Why does this all work?  I have no idea.
% Another important thing: there should be a blank line (i.e. a \par)
% between the end of the last paragraph and the \end{environment}.

\newlength{\clindent}
\setlength{\clindent}{0.5in}
\newlength{\clparindent}
\setlength{\clparindent}{\parindent}

\newenvironment{clauses}{\advance\linewidth -\parindent
                        \hangindent\clindent\relax
                         \hangafter=0
                          \parindent=0pt
                          \def\par{{\origpar}}}{}

\newcommand{\cpar}{\hspace \clparindent}

\newcommand{\startitem}{\bigskip\pagebreak[3]}
\newcommand{\finishitem}{\smallskip}

\newcommand{\clause}[1]{\startitem\Clause{#1}\finishitem}
\newcommand{\Clause}[1]{\hbox{#1}}
\newcommand{\clindex}[1]{\index{{\lisp #1}}}
\newcommand{\clindexx}[2]{\index{{\lisp #1\dots #2}}}
\newcommand{\clindexxx}[3]{\index{{\lisp #1\dots #2\dots #3}}}
\newcommand{\clausex}[2]{\startitem\Clausex{#1}{#2}\finishitem}
\newcommand{\Clausex}[2]{\Clause{|#1| #2}\clindex{#1}}

\newcommand{\defvar}[1]{\startitem\deff{#1}{}{Variable}\finishitem}
\newcommand{\defunexpvar}[1]{\startitem\deffnoindex{iterate::#1}{}{Variable}
                                        \lispindex{#1}\finishitem}
\newcommand{\defun}[2]{\startitem\deff{#1}{#2}{Function}\finishitem}
\newcommand{\defmacro}[2]{\startitem\deff{#1}{#2}{Macro}\finishitem}
\newcommand{\defunx}[3]{\startitem\deffx{#1}{#2}{#3}{Function}\finishitem}
\newcommand{\defmacrox}[3]{\startitem\deffx{#1}{#2}{#3}{Macro}\finishitem}
\newcommand{\defunxx}[4]{\startitem\deffxx{#1}{#2}{#3}{#4}{Function}\finishitem}
\newcommand{\defmacroxx}[4]{\startitem\deffxx{#1}{#2}{#3}{#4}{Macro}\finishitem}

\newcommand{\deff}[3]{\deffnoindex{#1}{#2}{#3}\lispindex{#1}}
\newcommand{\deffnoindex}[3]{\hbox to \hsize{{\tt #1} {\it #2}\hfill [{\it #3}]}}
\newcommand{\deffx}[4]{\deff{#1}{#2}{#4}\moreargs{#1}{#3}}
\newcommand{\deffxx}[5]{\deffx{#1}{#2}{#3}{#5}\moreargs{#1}{#4}}

\newcommand{\moreargs}[2]{\hbox to \hsize{\phantom{\lisp #1} {\it #2}\hfill}}

\newcommand{\lispindex}[1]{\index{{\lisp #1}}}

 \newenvironment{note}[1]{\pagebreak[2]\bigskip
                          \hrule\smallskip\small
                         {\setlength{\parindent}{0pt}
                           \par{\bf #1:}}}{\smallskip\hrule\bigskip}



\makeindex

%%%%%%%%%%%%%%%%%%%%%%%%%%%%%%%%%%%%%%%%%%%%%%%%%%%%%%%%%%%%%%%%

%\thispagestyle{empty}

%\vspace*{2in}

%\title{The |Iterate| Manual}
%\author{Jonathan Amsterdam}

%\maketitle

%\thispagestyle{empty}

%\begin{center}
%\Large *** DRAFT ***
%\end{center}
%This manual is in beta-test. It will become an AI Memo in a couple of
%months.  I would appreciate any comments or
%corrections.  You can reach me at email address |jba@ai.ai.mit.edu|.

\begin{document}

\def\writtenby{Jonathan Amsterdam}
\def\contractno{This report describes research done at the Artificial
Intelligence Laboratory of the Massachusetts Institute of Technology.
Support for the laboratory's artificial intelligence research is
provided in part by the Advanced Research Projects Agency of the
Department of Defense under Office
of Naval Research contract N00014-85-K-0124.}

\AIMEMO{1236}{\Large The Iterate Manual \\[1in]}
{This is the manual for version 1.4 of \iter, a powerful iteration
macro for Common  
Lisp. \iter\ is similar to {\tt loop} but provides numerous additional
features, is well integrated with Lisp, and is extensible. 
\vfill }


% use ~ for \em
\catcode`\~=13
\newif\ifem
\def~{\ifem
         \/\endgroup\emfalse
      \else 
         \begingroup\em\emtrue 
      \fi}



%\pagebreak
%\setcounter{page}{1}

%\thispagestyle{empty}


\tableofcontents

\pagebreak
\pagestyle{headings}

\section{Introduction}

\begin{sloppypar}
This manual describes
\iter, a powerful iteration facility for Common Lisp.
\iter\ provides abstractions for many common iteration
patterns and allows for the definition of additional patterns.  \iter\ is
a macro that expands into ordinary Lisp at
compile-time, so it is more efficient than higher-order
functions like |map| and |reduce.|  
While it is similar to |loop|, \iter\ offers a more Lisp-like syntax
and enhanced extensibility.
(For a more complete
comparison of \iter\ with other iteration constructs, see MIT AI Lab Working
Paper No. 324, ~Don't Loop, Iterate.~)
\end{sloppypar}

An \iter\ form consists of the symbol |iter|\footnote{You can also use
|iterate|, but |iter| is preferred because it avoids potential conflicts with
possible future additions to Common Lisp, and because it saves
horizontal space when writing code.}
followed by one or more
forms, some of which may be \iter\ 
~clauses.~  Here is a simple example of \iter\ which collects the
numbers from 1 to 10 into a list, and returns the list.  The return
value is shown following the arrow.
\begin{program}
(iter (for i from 1 to 10)
      (collect i))          \yields (1 2 3 4 5 6 7 8 9 10)
\end{program}
This form contains two clauses: a |for| clause that steps the
variable |i| over the integers from 1 to 10, and a |collect|
clause that accumulates its argument into a list.  With a few
exceptions, all \iter\ clauses have the same format: alternating
symbols (called ~keywords~) and expressions (called
~arguments~).  The syntax and terminology are those of Common
Lisp's keyword lambda lists.  One difference is that \iter's keywords
do not have to begin with a colon---though they may, except for the
first symbol of a clause.  So you can also write  |(for i :from
1 :to 10)| if you prefer.

Any Lisp form can appear in the body of an \iter, where it will have
its usual meaning.  \iter\ walks the entire body, expanding macros,
and recognizing clauses at any level.  This example collects all the
odd numbers in a list:
\begin{program}
(iter (for el in list)
      (if (and (numberp el) (oddp el))
          (collect el)))
\end{program}

There are clauses for iterating over numbers, lists, arrays and other
objects, and for
collecting, summing, counting, maximizing and other useful operations.
\iter\ also supports the creation of new variable bindings, stepping
over multiple sequences at once, destructuring, and compiler
declarations of variable types.  The following example illustrates
some of these features: 
\begin{program}
(iter (for (key . item) in alist)
         (for i from 0)
         (declare (fixnum i))
         (collect (cons i key)))
\end{program}

This loop takes the keys of an alist and returns a new alist
associating the keys with their positions in the original list. The
compiler declaration for |i| will appear in the generated code
in the appropriate place.

%%%%%%%%%%%%%%%%%%%%%%%%%%%%%%%%%%%%%%%%%%%%%%%%%%%%%%%%%%%%%%%%
\section{Clauses}

Most of \iter's clauses will be familiar to |loop| programmers.  
(|loop| is an iteration macro that has been incorporated into Common
Lisp.  Seen Guy Steele's {\em Common Lisp, 2nd Edition}.)
In
nearly all cases they behave the same as their |loop| counterparts, so
a |loop| user can switch to \iter\ with little pain (and much gain).

All clauses with the standard keyword-argument syntax consist of two
parts: a ~required~ part, 
containing keywords that must be present and in the right order; and
an ~optional~ part, containing keywords that may be omitted and,
if present, may occur in any order.  In the descriptions below, the
parts are separated by the Lisp lambda-list keyword \opt.


\subsection{Drivers}

An iteration-driving clause
conceptually causes the iteration to go forward.  Driver clauses in
\iter\ allow iteration over numbers, lists, vectors, hashtables, packages,
files and streams.  Iteration-driving clauses must
appear at the top level of an \iter\ form; they cannot be nested
inside another clause.  The driver variable is updated at the point
where the driver clause occurs.  Before the clause is executed for the
first time, the value of the variable is undefined.

%Also, regardless of where the driver clause
%appears in the body, the driver variable is stepped at the top of the
%loop; hence it is stylistically preferable, though
%not required, to place driver clauses at the beginning of the \iter.

Multiple drivers may appear in a single \iter\ form, in which case all
of the driver variables are updated each time through the loop, in the
order in which the clauses appear.  The first driver to terminate will
terminate the entire loop.

In all cases, the value of the driver variable on exit from the loop,
including within the epilogue code (see the |finally| clause), is
undefined.

All the parameters of a driver clause are evaluated once, before the
loop begins.  Hence it is not possible to change the bounds or other
properties of an iteration by side-effect from within the loop.

With one exception, driver clauses begin with the word |for| (or
the synonym |as|) and mention an iteration variable, which is
given a binding within the \iter\ form.
The exception is |repeat|, which just executes a loop a
specified number of times:

\begin{clauses}

\clausex{repeat}{~n~}
Repeats the loop ~n~ times.  For example:
\begin{program}
(iter (repeat 100)
      (print "I will not talk in class."))
\end{program}
\cpar If $n \leq 0$, the
loop will never be executed.  If ~n~ is not an integer, the
actual number of executions will be $\lceil n \rceil$.

\end{clauses}

\pagebreak[3]

\subsubsection{Numerical Iteration}

\begin{clauses}
\clausex{for}{~var~ |\&sequence|} 
The general form for iterating over a sequence of numbers requires a
variable and, optionally, one or more keywords that provide the bounds
and step size of the iteration. The |\&sequence|
lambda-list keyword is a shorthand for these sequence keywords.  They are:
|from|, |upfrom|, |downfrom|, |to|, |downto|, |above|, |below| and |by|.
|from| provides the starting value for ~var~
and defaults to zero. |to| provides a final value and implies that the
successive values of ~var~ will be increasing; |downto| implies that
they will be decreasing.  The loop terminates when ~var~ passes
the final value (i.e. becomes smaller or larger than it, depending on
the direction of iteration); in other  words, the loop body
will never be 
executed for values of ~var~ past the final value.
|below| and |above| are similar to |to| and
|downto|, except that the loop terminates when ~var~ equals or passes
the final value.

\cpar If no final value is specified, the variable will be stepped
forever.  Using |from| or |upfrom| will result in increasing
values, while |downfrom| will give decreasing values.

\cpar On each iteration,
~var~ is incremented or decremented by the value of the sequence
keyword |by|, which defaults to 1.  It should always be a positive
number, even for downward iterations.

\cpar In the following examples, the sequence of numbers generated is shown
next to the clause.

\begin{program}
(for i upfrom 0) \yields\ 0 1 2 \ldots
(for i from 5)   \yields\ 5 6 7 \ldots    ; either from or upfrom is okay
(for i downfrom 0) \yields\ 0 -1 -2 \ldots
(for i from 1 to 3) \yields\ 1 2 3
(for i from 1 below 3) \yields\ 1 2
(for i from 1 to 3 by 2) \yields\ 1 3
(for i from 1 below 3 by 2) \yields\ 1
(for i from 5 downto 3) \yields\ 5 4 3
\end{program}

\end{clauses}

\subsubsection{Sequence Iteration}

There are a number of clauses for iterating over sequences.  In all of
them, the argument following |for| may be a list instead of a
symbol, in which case destructuring is performed.  See section
\ref{destructuring}. 

\begin{clauses}

\clause{|for| ~var~ |in| ~list~ \opt\ |by| ~step-function~}
\clindexx{for}{in}
~var~ is set to successive elements of list.  
~step-function,~ which 
defaults to |cdr|, is used to obtain the next sublist. 

\clause{|for| ~var~ |on| ~list~ \opt\ |by| ~step-function~}
\clindexx{for}{on}
~var~ is set to successive sublists of list. 
~step-function~ (default |cdr|) is used as in |for\dots in|.

\end{clauses}

\medskip

These two clauses use |atom| to test for the end
of a list.  Hence, given a list whose final |cdr| is not \nil,
they will silently ignore the last |cdr|.  Other choices are
|endp|,  which would signal an error, and |null|, which
would probably result in an error somewhere else.  If you wish to use
an end-test other than |atom|, set the variable
|iterate::*list-end-test*|\lispindex{*list-end-test*} to the name of
the desired function. 

\begin{clauses}
\clause{|for| ~var~ |in-vector| ~vector~ |\&sequence|}
\clindexx{for}{in-vector}
~var~ takes on successive elements from ~vector.~   The vector's
fill-pointer is observed.  Here and in subsequent clauses, the |\&sequence|
keywords include |with-index|, which takes a symbol as argument
and uses it for the index variable instead of an internally generated
symbol.  The other |\&sequence| keywords behave as in numerical
iteration, except that the default iteration bounds are the bounds of
the vector.  E.g. in |(for i in-vector v downto 3)|,
|i| will start off being bound to the last element in |v|,
and will  
be set to preceding elements down to and including the element with
index 3.

\clause{|for| ~var~ |in-sequence| ~seq~ |\&sequence|}
\clindexx{for}{in-sequence}
This uses Common Lisp's generalized sequence functions, |elt|
and |length|, to obtain elements and determine the length of
~seq.~   Hence it will work for any sequence, including lists, and will
observe the fill-pointers of vectors.

\clause{|for| ~var~ |in-string| ~string~ |\&sequence|}
\clindexx{for}{in-string}
~var~ is set to successive characters of ~string.~

\startitem
\Clause{|for| ~var~ |index-of-vector| ~vector~ |\&sequence|}
\clindexx{for}{index-of-vector}
\Clause{|for| ~var~ |index-of-sequence| ~sequence~ |\&sequence|}
\clindexx{for}{index-of-sequence}
\Clause{|for| ~var~ |index-of-string| ~string~ |\&sequence|}
\clindexx{for}{index-of-string}
\finishitem
~var~ is set to successive indices of the sequence.
These clauses avoid the overhead of accessing the sequence elements
for those applications where they do not need to be examined, or are
examined rarely.  They admit all the optional keywords of the other
sequence drivers except the (redundant) |with-index| keyword.

\clause{|for| ~(key value)~ |in-hashtable| ~table~}
\clindexx{for}{in-hashtable}

~key~ and ~value,~ which must appear as shown in a list
and may be destructuring templates, are set to the keys and values of
~table~.  If ~key~ is |nil|, then the hashtable's keys will be ignored;
similarly for ~value~.
The order in which 
elements of ~table~ will be retrieved is unpredictable.

\clause{|for| ~var~ |in-package| ~package~ \opt\ |external-only| ~ext~}
\clindexx{for}{in-package}
Iterates over all the symbols in ~package,~ or over only the
external symbols if ~ext~ is specified and non-|nil|.  ~ext~ is not
evaluated.  The same symbol may appear more than once.

\clause{|for| ~var~ |in-packages| ~packages~ \opt\ |having-access| ~symbol-types~}
\clindexx{for}{in-packages}
Iterates over all the symbols from the list of packages denoted by the
descriptor ~packages~ and having accessibility (or visibility) given by
~symbol-types~.  This defaults to the list |(:external :internal :inherited)|
and is not evaluated.  ~var~ must be a list of up to three variables: in each
iteration, these will be set to a symbol, its access-type and package (as per
|with-package-iterator| in ANSI-CL).  The same symbol may appear more than
once.

\end{clauses}

\begin{clauses}

\clause{|for| ~var~ |in-file| ~name~ \opt\ |using| ~reader~}
\clindexx{for}{in-file}
Opens the file ~name~ (which may be a string or pathname) for input,
and iterates 
over its contents. ~reader~ defaults to |read|, so by default ~var~
will be bound to the successive forms in the file.  The \iter\ body is
wrapped in an unwind-protect to ensure that the file is closed no
matter how the \iter\ is exited.

\clause{|for| ~var~ |in-stream| ~stream~ \opt\ |using| ~reader~}
\clindexx{for}{in-stream}
Like |for\dots in-file|, except that ~stream~ should be an existing
stream object that supports input operations.

\end{clauses}

\subsubsection{Generalized Drivers}

These are primarily useful for writing drivers that can also be used
as generators (see section \ref{generators}, below).
 
\begin{clauses}

\clause{|for| ~var~ |next| ~expr~}
\clindexx{for}{next}

~var~ is set to ~expr~ each time through the loop.  Destructuring is
performed.  When the clause is used as a generator, ~expr~ is the code
that is executed when |(next ~var~)| is 
encountered (see section \ref{generators}, below).
~expr~ should compute the first value for ~var~, as well
as all subsequent values, and is responsible for terminating the loop.
For compatibility with future versions of \iter, this termination
should be done with |terminate|\clindex{terminate}, which can be
considered a synonym for |finish| (see section \ref{control-flow}).

\cpar As an example, the following clauses are equivalent to |(for
i from 1 to 10)|:
\begin{program}
(initially (setq i 0))
(for i next (if (> i 10) (terminate) (incf i)))
\end{program}

\clause{|for| ~var~ |do-next| ~form~}
\clindexx{for}{do-next}
~form~ is evaluated each time through the loop.  Its value is ~not~
set to ~var~; that is ~form's~ job. ~var~ is only present so that
\iter\ knows it is a driver variable.  \linebreak |(for ~var~ next
~expr~)| is equivalent to |(for ~var~ do-next (dsetq ~var~ ~expr~))|.
(See section \ref{destructuring} for an explanation of |dsetq|.) 

\end{clauses}

\subsubsection{Generators}
\label{generators}

In all of the above clauses, the driver variable is updated on each
iteration.
Sometimes it is desirable to have greater control over updating.  
For instance, consider the problem of associating numbers, in
increasing order and with no gaps, with the
\nonnil\ elements of a list.  One obvious first pass at writing this is:

\begin{program}
(iter (for el in list)
      (for i upfrom 1)
      (if el (collect (cons el i))))
\end{program}
But on the list |(a b nil c)| this produces |((a . 1) (b . 2) (c .
4))| instead of the desired |((a . 1) (b . 2) (c . 3))|.  The problem
is that |i| is incremented each time through the loop, even when |el|
is |nil|.  

The problem could be solved elegantly if we could step |i| only when
we wished to.  This
can be accomplished for any \iter\ driver by writing |generate| (or
its synonym |generating|)
instead of |for|.  Doing so produces a ~generator~---a driver whose
values are yielded explicitly.  To obtain the next value of a
generator variable ~v~, write \linebreak |(next ~v~)|.  The value of a |next|
form is the next value of ~v~, as determined by its associated driver
clause.  |next| also has the side-effect of updating ~v~ to that
value.  If there is no next value, |next| will terminate the loop,
just as with a normal driver.


Using generators, we can now write our example like this:
\begin{program}
(iter (for el in list)
      (generate i upfrom 1)
      (if el (collect (cons el (next i)))))
\end{program}
Now |i| is updated only when |(next i)| is executed, and this occurs
only when |el| is \nonnil.

To better understand the relationship between ordinary drivers and
generators, observe that we can rewrite an ordinary driver using its
generator form immediately followed by |next|, as this example shows:
\begin{program}
(iter (generating i from 1 to 10)
      (next i)
      ...)
\end{program}
Provided that the loop body contains no |(next i)| forms, this will
behave just as if we had written |(for i from 1 to 10)|.

We can still refer to a driver variable ~v~ without using |next|; in
this case, its value is that given to it by the last evaluation of
|(next ~v~)|.  Before |(next ~v~)| has been called the first time, the
value of ~v~ is undefined.  

This semantics is more flexible than
one in which ~v~ begins the loop bound to its first value and calls
of |next| supply subsequent values, because it means the loop will not
terminate too soon if the generator's sequence is empty.  For
instance, consider the following code, which tags \nonnil\ elements of
a list using a list of tags, and also counts the null elements.
(We assume there are at least as many tags as \nonnil\ elements.)
\begin{program}
(let* ((counter 0)
       (tagged-list (iter (for el in list)
                          (generating tag in tag-list) 
                          (if (null el)
                              (incf counter)
                              (collect (cons el (next tag)))))))
  ...)
\end{program}
It may be that there are just as many tags as non-null elements of
|list|.  If all the elements of |list| are null, we still want the
counting to proceed, even though |tag-list| is \nil.  If |tag| had to be
assigned its first value before the loop begins, we would have had to
terminate the loop before the first iteration, since when |tag-list|
is \nil, |tag| 
has no first value.  With the existing semantics, however, |(next
tag)| will never execute, so the iteration will cover all the elements of
|list|. 
        
When the ``variable'' of a driver clause is actually a destructuring
template containing several variables, all the variables are eligible
for use with |next|.  As before, |(next ~v~)| evaluates to ~v's~ next
value; but the effect is to update all of the template's variables.
For instance, the following code will return the list |(a 2 c)|.
\begin{program}
(iter (generating (key . item) in '((a . 1) (b . 2) (c . 3)))
      (collect (next key))
      (collect (next item)))
\end{program}

Only driver clauses with variables can be made into generators.  This
includes all clauses mentioned so far except for |repeat|.  It does
~not~ include |for\dots previous|, |for\dots =|,
|for\dots initially\dots then| or |for\dots first\dots then| (see
below). 

\subsubsection{Previous Values of Driver Variables}

Often one would like to access the value of a variable on a previous
iteration.  \iter\ provides a special clause for accomplishing this.

\begin{clauses}

\clause{|for| ~pvar~ |previous| ~var~ \opt\ |initially| ~init~
|back| ~n~}
\clindexx{for}{previous}
Sets ~pvar~ to the previous value of ~var,~ which should be a driver
variable, a variable from another |for\dots previous| clause, or a
variable established by a |for\dots =|, 
|for\dots initially\dots then| or |for\dots first\dots then| clause
(see section \ref{setting}).
Initially, ~pvar~ is given the value ~init~ (which defaults to \nil).  
The ~init~ expression will be moved outside the loop body, so it
should not depend on anything computed within the loop.
~pvar~ retains the value of ~init~ until ~var~ is set to its second
value, at which point ~pvar~ is set to ~var's~ first value; and so on.  

\cpar The
argument ~n~ to |back| must be a constant, positive integer, and
defaults to 1.  It determines how many iterations back ~pvar~ should
track ~var~.  For example, when ~n~ is 2, then ~pvar~ will be assigned
~var's~ first value when ~var~ is set to its third value.

\cpar A |for\dots previous| clause may occur after or before its
associated driver clause. |for\dots previous| works with generators as
well as ordinary drivers.

\pagebreak[3]

\cpar Example:
\begin{program}
(iter (for el in '(1 2 3 4))
      (for p-el previous el)
      (for pp-el previous p-el initially 0)
      (collect pp-el))
\end{program}
This evaluates to |(0 0 1 2)|.  It could have been written more
economically as 
\begin{program}
(iter (for el in '(1 2 3 4))
      (for pp-el previous el back 2 initially 0)
      (collect pp-el))
\end{program}


\end{clauses}

\subsection{Variable Binding and Setting}
\label{setting}

Several clauses exist for establishing new variable bindings or for
setting variables in the loop.  They all support destructuring.

\begin{clauses}

\clausex{with}{~var~ \opt\ |=| ~value~}
Causes ~var~ to be bound to value before the loop body is entered.
If ~value~ is not supplied, ~var~ assumes a default
binding, which will be 
\nil\ in the absence of declarations.  Also, if ~value~ is not
supplied, no destructuring is performed; instead, ~var~ may be a list of
symbols, all of which are given default 
bindings.  If ~value~ is supplied, ~var~ is bound to it, with
destructuring.  

\cpar Because |with|
creates bindings whose scope includes the entire \iter\ form, it is
good style to put all |with| clauses at the beginning.

\cpar Successive occurrences of |with| result in sequential
bindings (as with 
|let*|).  There is no way to obtain parallel bindings; see
section \ref{bindings} for a rationale.  


\clause{|for| ~var~ |=| ~expr~}
\clindexx{for}{=}
On each iteration, ~expr~ is evaluated and ~var~ is set
to its value. 

\cpar This clause may appear to do the same thing as |for\dots next|.
In fact, they are quite different.  |for\dots =| provides only three
services: it sets up a binding for ~var~, sets it to ~expr~ on each
iteration, and makes it possible to use |for\dots previous| with
~var~.  |for\dots next| provides these services in addition to the
ability to turn the driver into a generator.
%Also, the code which sets ~var~ appears in the loop body in the same
%place as the |for\dots =| clause; the code for |for\dots next| appears
%at the top of the loop, as with other drivers (except when being used
%as a generator).

\clause{|for| ~var~ |initially| ~init-expr~ |then| ~then-expr~}
\clindexxx{for}{initially}{then}
Before the loop begins, ~var~ is set to ~init-expr;~ on all
iterations after the first it is set to ~then-expr.~  This clause
must occur at top-level.  ~init-expr~ will be moved outside the loop
body and ~then-expr~ will be moved to the end of the loop body, so
they are subject to code motion problems (see section
\ref{code-movement}). 

\cpar This clause may appear to be similar to |for\dots next|, but in
fact they differ significantly.  |for\dots initially\dots then| is
typically used to give ~var~ its first value before the loop begins,
and subsequent values on following iterations.  This is incompatible
with generators, whose first value and subsequent values must all be
computed by |(next ~var~)|.  Also, the update of ~var~ in 
|for\dots initially\dots then| does not occur at the location of the
clause. 
Use |for\dots initially\dots then| for
one-shot computations where its idiom is more convenient, but use
|for\dots next| for extending \iter\ with new drivers (see section
\ref{extend}).

\clause{|for| ~var~ |first| ~first-expr~ |then| ~then-expr~}
\clindexxx{for}{first}{then}
The first time through the loop, ~var~ is set to ~first-expr;~ on
subsequent iterations, it is set to ~then-expr.~  This differs from
|for\dots initially| in that ~var~ is set to ~first-expr~
inside the loop body, 
so ~first-expr~ may depend on the results of other clauses.  For
instance, 
\begin{program}
(iter (for num in list)
      (for i first num then (1+ i))
      ...)
\end{program}
will set |i| to the first element of |list| on the first
iteration, whereas 
\begin{program}
(iter (for num in list)
      (for i initially num then (1+ i))
      ...)
\end{program}
is probably erroneous; |i| will be bound to |num|'s
default binding (usually \nil) for the first iteration.

\end{clauses}

\begin{note}{Compatibility note}
|loop|'s |for\dots =| works like \iter's, but |loop| used the syntax
|for\dots =\dots then| to mean |for\dots initially\dots then|.  It was
felt that these two operations were sufficiently different to warrant
different keywords.

Also, the |for| in the above three clauses is misleading,
since none is true driver (e.g. none has a corresponding |generate|
form).  |setting| would have been a better choice, but |for| was used
to retain some compatibility with |loop|.  
\end{note}


%%%%%%%%%%%%%%%%%%%%%%%%%%%%%%%%
\subsection{Gathering Clauses}
Many of \iter's clauses accumulate values into a variable, or set a
variable under certain conditions.  At the end of the loop, this
variable contains the desired result.  All these clauses have an
optional |into| keyword, whose argument should be a symbol.  If the
|into| keyword is not supplied, the accumulation variable will be
internally generated and its value will be returned at the end of the
loop; if a variable is specified, that variable is used for the
accumulation, and is not returned as a result---it is up to the user
to return it explicitly, in the loop's epilogue code (see |finally|).
It is safe to examine the accumulation variable during the loop, but
it should not be modified.

These clauses all begin with a verb.  When the verb does
not conflict with an existing Common Lisp function, then it may be
used in either its infinitival or present-participle form (e.g. |sum|,
|summing|).  However, when there is a conflict with Common Lisp, only
the present-participle form may be used (e.g. |unioning|).  This is to
prevent \iter\ clauses from clashing with Common Lisp
functions. 

%although these clauses are described as ``producing a value,'' it is a
%mistake to think of the lisp list representing the clause as a
%value-producing form in the usual way.  clauses may legally be written
%where a value is expected, e.g. |(setq x (sum i))|, but the
%lisp value of a clause in such a context is undefined.

\subsubsection{Reductions}

~Reduction~ is an extremely common iteration pattern in which
the results of successive applications of a
binary operation are accumulated.
For example, a loop that computes the sum of the
elements of a list is performing a reduction with the addition
operation.  This could be written in Common Lisp as  |(reduce \#'+
list)| or with \iter\ as
\begin{program}
(iter (for el in list)
      (sum el))
\end{program}

\begin{clauses}

\clausex{sum}{~expr~ \opt\ |into| ~var~}
Each time through the loop, ~expr~ is evaluated and added to a
variable, which is bound initially to zero.  If ~expr~ has a type,
it is ~not~ used as the type of the sum variable, which is always
|number|.  To get the result variable to be of a more specific
type, use an explicit variable, as in
\begin{program}
(iter (for el in number-list)
      (sum el into x)
      (declare (fixnum x))
      (finally (return x)))
\end{program}

\clausex{multiply}{~expr~ \opt\ |into| ~var~}
Like |sum|, but the initial value of the result variable is $1$,
and the variable is updated by multiplying ~expr~ into it.

\clausex{counting}{~expr~ \opt\ |into| ~var~}
~expr~ is evaluated on each iteration.  If it is \nonnil, the
accumulation variable, initially zero, is incremented.

\startitem
\Clausex{maximize}{~expr~ \opt\ |into| ~var~}
\Clausex{minimize}{~expr~ \opt\ |into| ~var~}
\finishitem
~expr~ is evaluated on each iteration and its extremum (maximum or
minimum) is stored in the accumulation variable.  If ~expr~ is never
evaluated, then the result is \nil\ (if the accumulation variable is
untyped) or $0$ (if it has a numeric type).

\clausex{reducing}{~expr~ |by| ~func~ \opt\ |initial-value| ~init-val~ |into| ~var~}
This is a general way to perform reductions. ~func~ should be a
function of two arguments, the first of which will be the value
computed so far and
the second of which will be the value of ~expr~.  It should return the
new value.
|reducing| is roughly equivalent to the
Common Lisp |(reduce ~func~ ~sequence~ :key ~expr-function~)|, where
~expr-function~ is used to derive values from the successive elements
of ~sequence~.

\cpar If the |reducing| clause is never executed, the result is
undefined.

\cpar It is not necessary to provide an
initial value, but better code can be generated if one is supplied.
Regardless of its location in the \iter\ body, ~init-val~ will be
evaluated before the loop is entered, so it should not depend on any
value computed inside the \iter\ form.  

%\cpar if a ~var~ is not specified, you can get \iter\ to declare the
%type of the internal variable by putting a |the| expression around
%~func.~  see section \ref{types}.

\end{clauses}

%\begin{note}{implementation note}
%in principle, |maximize| and |minimize| can be thought of
%as reductions where the initial value is the smallest (or largest)
%value that the accumulation variable can assume.  because lisp's
%bignums can represent arbitrary integers, these clauses cannot be
%implemented as reductions in general.  if, however, the type of 
%~expr~ or ~var~ can be determined to be a fixnum
%or a float, \iter\ will implement the clause as a true reduction,
%using one of the constants |most-negative-fixnum|, |%most-positive-fixnum|, 
%|most-negative-short-float|, etc. as appropriate.
%\end{note}

\subsubsection{Accumulations}

All the predefined accumulation clauses add values to a sequence.  If
the sequence is a list, they all 
have the property that the partial list is kept in the correct order
and available for inspection at any point in the loop.

\begin{clauses}

\clausex{collect}{~expr~ \opt\ |into| ~var~ |at| ~place~ |result-type| ~type~}
Produces a sequence of the values of ~expr~ on each iteration. ~place~
indicates where the next value of ~expr~ is added to the list and may
be one of the symbols |start|, |beginning| (a synonym for |start|) or
|end|.  The symbol may be quoted, but need not be.  The default is
|end|.  For example,
\begin{program}
(iter (for i from 1 to 5)
      (collect i))
\end{program}
produces |(1 2 3 4 5)|, whereas
\begin{program}
(iter (for i from 1 to 5)
      (collect i at beginning))
\end{program}
produces |(5 4 3 2 1)| (and is likely to be faster in most Common Lisp
implementations).

\cpar If ~type~ is provided, it should be a subtype of |sequence|.
The default is |list|.  Specifying a type other than |list| will
result in |collect| returning a sequence of that type.  ~However,~ the
type of the sequence being constructed when inside the loop body is
undefined when a non-|list| type is specified.  (As with ~place~,
quoting ~type~ is optional.)

\clausex{adjoining}{~expr~ \opt\ |into| ~var~ |test| ~test~ |at| ~place~ |result-type| ~type~}
Like |collect|, but only adds the value of ~expr~ if it is not
already present.  ~test,~ which defaults to |\#'eql|, is
the test to be used with |member|.

\startitem
\Clausex{appending}{~expr~ \opt\ |into| ~var~ |at| ~place~}
\Clausex{nconcing}{\ ~expr~ \opt\ |into| ~var~ |at| ~place~}
\Clausex{unioning}{\ ~expr~ \opt\ |into| ~var~ |test| ~test~ |at| ~place~}
\Clausex{nunioning}{~expr~ \opt\ |into| ~var~ |test| ~test~ |at| ~place~}
\finishitem
These are like |collect|, but behave like the Common Lisp functions
|append|, |nconc|, |union| or |nunion|.  
As in Common Lisp, they work only on lists.  Also as in Common Lisp,
|unioning| and |nunioning| assume that the value of ~expr~ contains no
duplicates. 

\clausex{accumulate}{~expr~ |by| ~func~ \opt\ |initial-value| ~init-val~ |into| ~var~}
This is a general-purpose accumulation clause. ~func~ should be a
function of two arguments, the value of ~expr~ and the value
accumulated so far in the iteration, and it should return the updated
value.  If no initial value is supplied, \nil\ is used.  

%\cpar If a ~var~ is not specified, you can get \iter\ to declare the
%type of the internal variable by putting a |the| expression around
%~func.~  see section \ref{types}.

\cpar The differences between |accumulate| and |reducing| are slight.
One difference is that the functions take their arguments in a
different order.  Another is that in the absence of ~init-val~,
|accumulate| will use \nil, whereas |reducing| will generate different
code that avoids any dependence on the initial value.
The reason for having both clauses is that one usually
thinks of reductions (like |sum|) and accumulations (like |collect|)
as different beasts.

\end{clauses}


\subsubsection{Finders}

A ~finder~ is a clause whose value is an expression that meets some
condition.

\begin{clauses}

\clause{|finding| ~expr~ |such-that| ~test~ \opt\ |into| ~var~ |on-failure| ~failure-value~}
\clindexx{finding}{such-that}
If ~test~ (which is an expression) ever evaluates to \nonnil, the loop
is terminated, the 
epilogue code is run and the value of ~expr~ is returned.  Otherwise,
\nil\ (or ~failure-value,~ if provided) is returned.  If ~var~ is
provided, it will have either the \nonnil\ value of ~expr~ or
~failure-value~ when the epilogue code is run.

\cpar As a special case, if the ~test~ expression is a sharp-quoted
function, then it is applied to ~expr~ instead of being simply
evaluated.  E.g. |(finding x such-that \#'evenp)| is equivalent to
|(finding x such-that (evenp x))|.  

%\cpar although ~test~ need have nothing to do with ~%expr~ as in
%|(finding j such-that (> i 3))|, it usually 
%will: |(finding (length el) such-that (oddp (length el)))|.  to
%avoid performing the |length| computation twice, you could write
%|(finding (length el) such-that \#'oddp)| or |(finding (length
%el) such-that 'oddp)|; for these cases, \iter\ generates code that
%executes ~expr~ only once.  the code for |\#'oddp|
%is slightly different from that for {\lisp 'oddp}; see the discussion
%under {\lisp for\dots in} and {\lisp for\dots on}.

\cpar |On-failure| is a misnomer. Because it is always evaluated, it behaves
more like the default third argument to the |gethash| function. As a result,
|on-failure (error "Not found")| makes no sense. Instead, the clauses |leave|
or |thereis| can be used in conjunction with |finally| as follows:
\begin{program}
(iter (for x in '(1 2 3))
      (if (evenp x) (leave x))
      (finally (error "not found")))
\end{program}

\cpar This clause may appear multiple times when all defaults are
identical. It can also be used together with either |always|/|never| or
|thereis| if their defaults match. More specifically, |on-failure nil| is
compatible with |thereis|, while |on-failure t| is compatible with |always|
and |never| clauses.
\begin{program}
(iter (for i in '(7 -4 2 -3))
      (if (plusp i)
	  (finding    i  such-that (evenp i))
          (finding (- i) such-that (oddp i))))
\end{program}

\startitem
\Clause{|finding| ~expr~ |maximizing| ~m-expr~ \opt\ |into| ~var~}
\clindexx{finding}{maximizing}
\Clause{|finding| ~expr~ |minimizing| ~m-expr~ \opt\ |into| ~var~}
\clindexx{finding}{minimizing}
\finishitem
Computes the extremum (maximum or minimum) value of ~m-expr~ over all
iterations, and returns the value of ~expr~ corresponding to the
extremum.  ~expr~ is evaluated inside the loop at the time the new
extremum is established.  If ~m-expr~ is never evaluated (due to, for
example, being embedded in a conditional clause), then the returned
value depends on the type, if any, of ~expr~ (or ~var,~ if
one is supplied).  If there is no type, the returned
value will be \nil; if the type is numeric, the returned value will be
zero.

\cpar For these two clauses, ~var~ may be a list of two
symbols; in that case, the first is used to record ~expr~ and
the second, ~m-expr.~  

\cpar As with |finding\dots such-that|, if ~m-expr~ is a sharp-quoted
function, then it is called on ~expr~ instead of being evaluated.

\end{clauses}

\subsubsection{Boolean Tests}

\begin{clauses}

\clausex{first-iteration-p}{}
Returns |t| in the first cycle of the loop, otherwise \nil.

\clausex{first-time-p}{}
Returns |t| the first time the expression is evaluated, and then \nil\ forever.
This clause comes handy when printing (optional) elements separated
by a comma:

\begin{program}
(iter (for el in '(nil 1 2 nil 3))
      (when el
        (unless (first-time-p)
          (princ ", "))
        (princ el)))
\end{program}
produces |"1, 2, 3"|.

\end{clauses}

\subsubsection{Aggregated Boolean Tests}

\begin{clauses}

\clausex{always}{~expr~}
If ~expr~ ever evaluates to
\nil, then \nil\ is immediately returned; the epilogue code is not
executed.  If ~expr~ never evaluates to \nil, the epilogue code
is executed and the last value of ~expr~ (or |t| if ~expr~ was never
evaluated) is returned (whereas |loop| would constantly return |t|).

% mention last evaluated clause when multiple always clauses?

\clausex{never}{~expr~}
Like |(always (not ~expr~))|, except it does not influence the last
value returned by a possible other |always| clause. That is,
\begin{program}
(iter (repeat 2)
      (always 2)
      (never nil)) \yields 2 ; not t
\end{program}

\clausex{thereis}{~expr~}
If ~expr~ is ever \nonnil,
its value is immediately returned without running epilogue code.
Otherwise, the epilogue code is performed and \nil\ is returned.

This clause cannot be used together with |always| or |never|, because their
defaults are opposed (similarly, |(loop always 3 thereis nil)| refuses to
compile in some implementations of |loop|).

\end{clauses}


\subsection{Control Flow}
\label{control-flow}
Several clauses can be used to alter the usual flow of control in a loop.

Note: the clauses of this and subsequent sections don't adhere to \iter's
usual syntax, but instead use standard Common Lisp syntax.  Hence the
format for describing syntax subsequently is
like the standard format used in the Common Lisp manual, not like the
descriptions of clauses above.

\begin{clauses}

\clausex{finish}{}
Stops the loop and runs the epilogue code.

%for example:
%
%\begin{program}
%(iter (with answer = nil)
%         (initially (make-a-mess))
%         (for i from 1 to 10)
%         (when (correct? i)
%           (setq answer i)
%           (finish))
%         (finally (cleanup)))
%\end{program}
%
%this code will execute |cleanup| whether or not the test |(correct?
%i)| ever succeeds. 
%the (more elegant) formulation,
%\begin{program}
%(iter (initially (make-a-mess))
%         (for i from 1 to 10)
%         (finding i such-that (correct? i))
%         (finally (cleanup)))
%\end{program}
%would not execute |cleanup| if |(correct? i)| succeeded; it
%would do an immediate return.

\clausex{leave}{\opt\ ~value~}
Immediately returns ~value~ (default \nil) from the current \iter\
form, skipping the epilogue code.  Equivalent to using |return-from|.

\clausex{next-iteration}{}
Skips the remainder of the loop body and begins the next iteration of
the loop.

\clausex{while}{~expr~}
If ~expr~ ever evaluates to \nil, the loop is terminated and the
epilogue code executed.  Equivalent to |(if (not ~expr~) (finish))|.

\clausex{until}{~expr~}
Equivalent to |(if ~expr~ (finish))|.

\clausex{if-first-time}{~then~ \opt\ ~else~}
If this clause is being executed for the first time in this invocation
of the \iter\ form, then the ~then~ code is evaluated; otherwise the
~else~ code is evaluated.

\cpar |(for ~var~ first ~expr1~ then ~expr2~)| is almost equivalent to
\begin{program}
(if-first-time (dsetq ~var~ ~expr1~)
               (dsetq ~var~ ~expr2~))
\end{program}
The only difference is that the |for| version makes ~var~ available
for use with |for\dots previous|.

\end{clauses}

\subsection{Code Placement}
When fine control is desired over where code appears in a loop
generated by \iter, the following special clauses may be useful.
They are all subject to code-motion problems (see section
\ref{code-movement}).  

\begin{clauses}

\clausex{initially}{|\&rest| ~forms~}
The lisp ~forms~ are placed in the prologue section of the loop, where
they are executed once, before the loop body is entered.

\clausex{after-each}{|\&rest| ~forms~}
The ~forms~ are placed at the end of the loop body, where they
are executed after each iteration.  Unlike the other clauses in this
section, ~forms~ may contain \iter\ clauses.

\clausex{else}{|\&rest| ~forms~}
The lisp ~forms~ are placed in the epilogue section of the loop, where they
are executed if this |else| clause is never met during execution of the
loop and the loop terminates normally.

\clausex{finally}{|\&rest| ~forms~}
The lisp ~forms~ are placed in the epilogue section of the loop, where
they are executed after the loop has terminated normally.

\clausex{finally-protected}{|\&rest| ~forms~}
The lisp ~forms~ are placed in the second form of an unwind-protect
outside the loop.  They are always executed after the loop has
terminated, regardless of how the termination occurred.

\end{clauses}


%%%%%%%%%%%%%%%%%%%%%%%%%%%%%%%%%%%%%%%%%%%%%%%%%%%%%%%%%%%%%%%%
\section{Other Features}

\subsection{Multiple Accumulations}
\label{multiple}

\begin{sloppypar}
It is permitted to have more than one clause accumulate into the same
variable, as in the following:
\begin{program}
(iter (for i from 1 to 10)
      (collect i into nums)
      (collect (sqrt i) into nums)
      (finally (return nums)))
\end{program}
Clauses can only accumulate into the same variable if they are
compatible.  |collect|, |adjoining|, |appending|, |nconcing|,
|unioning| and |nunioning| are compatible with each other; |sum|,
|multiply| and |counting| are compatible; |always| and |never| are compatible;
|finding| \dots |such-that| is compatible with either |thereis| or |always|
and |never| when their defaults match; and |maximize| and |minimize| clauses
are compatible only with other |maximize| and |minimize| clauses,
respectively.

%note that the same variable ~cannot~ be both an accumulation
%variable and an ordinary variable; there can be only one variable with
%a given name within an \iter\ form. 
\end{sloppypar}

\subsection{Named Blocks}

Like Common Lisp |block|s, \iter\ forms can be given names.  The
name should be a single symbol, and it must be the first form in the
\iter.  The generated code behaves exactly like a named block; in
particular, |(return-from ~name~)| can be used to exit it:
\begin{program}
(iter fred
      (for i from 1 to 10)
      (iter barney
            (for j from i to 10)
            (if (> (* i j) 17)
                (return-from fred j))))
\end{program}
An \iter\ form that is not given a name is implicitly named \nil.

Sometimes one would like to write an expression in an inner \iter\ form,
but have it processed by an outer \iter\ form.  This is possible with
the |in| clause.  

\begin{clauses}

\clausex{in}{~name~ |\&rest| ~forms~}
Evaluates ~forms~ as if they were part of the \iter\ form named
~name~.  In other words, \iter\ clauses are processed by the \iter\
form named ~name,~ and not by any \iter\ forms that occur inside ~name.~

\cpar As an example, consider the problem of collecting a list of the
elements in a two-dimensional array.  The naive solution,
\begin{program}
(iter (for i below (array-dimension ar 0))
      (iter (for j below (array-dimension ar 1))
            (collect (aref ar i j))))
\end{program}
\noindent is wrong because the list created by the inner \iter\ is simply
ignored by the outer one.  But using |in| we can write:
\begin{program}
(iter outer (for i below (array-dimension ar 0))
      (iter (for j below (array-dimension ar 1))
            (in outer (collect (aref ar i j)))))
\end{program}
\noindent which has the desired result.

\end{clauses}

\subsection{Destructuring}
\label{destructuring}

In many places within \iter\ clauses where a variable is expected, a
list can be written instead.  In these cases, the value to be assigned
is ~destructured~ according to the pattern
described by the list.  As a simple example, the clause
\begin{program}
(for (key . item) in alist)
\end{program}
\noindent will result in |key| being set to the |car| of
each element in |alist|, and |item| being set to the |cdr|.  The
pattern list may be nested to arbitrary depth, and (as the example
shows) need not be terminated with \nil; the only requirement is that
each leaf be a bindable symbol (or \nil, in which case no binding is
generated for that piece of the structure).

Sometimes, you might like to do the equivalent of a
|multiple-value-setq| in a clause.  This 
``multiple-value destructuring'' can be expressed by writing 
\linebreak |(values $pat_1$ $pat_2 \ldots$)| for a destructuring
pattern, as in 
\begin{program}
(for (values (a . b) c d) = (three-valued-function ...))
\end{program}
\begin{sloppypar}
Note that the $pat_i$ can themselves be destructuring patterns (though
not multiple-value destructuring patterns).  You can't do multiple-value
destructuring in a |with| clause; instead wrap the whole \iter\
form in a |multiple-value-bind|.
\end{sloppypar}

\begin{note}{Rationale}
There are subtle interactions between variable declarations and
evaluation order that make the correct implementation of
multiple-value destructuring in a |with| somewhat tricky.
\end{note}

The destructuring feature of \iter\ is available as a separate
mechanism, using the |dsetq| macro:

\begin{clauses}
\defmacro{dsetq}{template expr}
Performs destructuring of ~expr~ using ~template~.   May be used
outside of an \iter\ form. Yields the primary value of ~expr~.

\end{clauses}

\subsection{On-line Help}

\begin{sloppypar}
There is a limited facility for on-line help, in the form of the 
|display-iterate-clauses| function.
\end{sloppypar}

\begin{clauses}

\defun{display-iterate-clauses}{\opt\ clause-spec}
Displays a list of \iter\ clauses.  If ~clause-spec~ is not
provided, all clauses are shown; if it is a symbol, all clauses
beginning with that symbol are shown; and if it is a list of symbols,
all clauses for which ~clause-spec~ is a prefix are shown.

\end{clauses}

\subsection{Parallel Binding and Stepping}
\label{bindings}

The parallel binding and stepping of variables is a feature that
\iter\ does ~not~ have.  This section attempts to provide a rationale.

We say that two variables are bound ~in parallel~ if neither
binding shadows the other.  This is the usual semantics of |let|
(as opposed to |let*|).  Similarly, we can say that iteration
variables are stepped in parallel if neither variable is updated
before the other, conceptually speaking; in other words, if the code
to update each variable can reference the old values of both variables.

|loop| allows parallel binding of variables and parallel stepping of
driver variables.  My view is that if you are depending on the 
serial/parallel distinction, you are doing something obscure.  If you
need to bind 
variables in parallel using |with|, then you must be using a
variable name that shadows a name in the existing lexical environment.
Don't do that.  The most common use for parallel stepping is to track
the values of variables on the previous iteration, but in fact this
does not require parallel stepping at all; the following will work:
\begin{program}
(iter (for current in list)
      (for prev previous current)
      ...)
\end{program}


%%%%%%%%%%%%%%%%%%%%%%%%%%%%%%%%%%%%%%%%%%%%%%%%%%%%%%%%%%%%%%%%
\section{Types and Declarations}
\label{types}

\subsection{Discussion}

Sometimes efficiency dictates that the types of variables be declared.
This type information needs to be communicated to \iter\ so it can
bind variables to appropriate values.  Furthermore, \iter\ must often
generate internal variables invisible to the user; there needs to be a
way for these to be declared.

As an example, consider this code, which will return the number of
odd elements in |number-list|:
\begin{program}
(iter (for el in number-list)
      (count (oddp el)))
\end{program}
In processing this form,
\iter\ will create an internal variable, let us call it |list17|, to
hold the successive |cdr|s of |number-list|, and  
will bind the variable to |number-list|.  It will also generate a
default binding for |el|; only inside the body of the loop will |el|
be set to the |car| of |list17|.  Finally, \iter\ will generate a
variable, call it |result|, to hold the result of the count, and will
bind it to zero.

When dealing with type declarations, \iter\ observes one simple rule:
~it will never generate a declaration unless requested to do so.~  The
reason is that such declarations might mask errors in compiled code by
avoiding error-checks; the resulting problems would be doubly hard to
track down because the declarations would be hidden from the
programmer.  Of course, a compiler might omit error-checks even in the
absence of declarations, though this behavior can usually be avoided,
e.g. by saying |(declaim (optimize (safety 3)))|.

So, the above \iter\ form will generate code with no declarations.
But say we wish to declare the types of |el| and the internal
variables |list17| and |result|.  How is this done?

Declaring the type of |el| is easy, since the programmer knows
the variable's name:
\begin{program}
(iter (for el in number-list)
      (declare (fixnum el))
      (counting (oddp el)))
\end{program}
\iter\ can read variable type declarations like this one.  Before
processing any clauses, it scans the entire top-level form for type
declarations 
and records the types, so that variable bindings can be performed
correctly.  In this case, |el| will be bound to zero
instead of \nil.  Also, \iter\ collects all the top-level declarations
and puts them at the begining of the generated code, so it is not
necessary to place all declarations at the beginning of an \iter\
form; instead, they can be written near the variables whose types they
declare.  

Since \iter\ is not part of the compiler, it will not know
about declarations that occur outside an \iter\ form; these
declarations must be repeated inside the form.

Here is another way we could have declared the type of |el|:
\begin{program}
(iter (for (the fixnum el) in number-list)
      (counting (oddp el)))
\end{program}
\iter\ extends the Common Lisp |the|\lispindex{the}
form to apply to variables as well as value-producing forms; anywhere
a variable is allowed---in a |with| clause, as the iteration
variable in a driver clause, as the |into| argument of an
accumulation clause, even inside a destructuring template---you can
write |(the ~type~ ~symbol~)| instead.

There is one crucial difference between using a |the| form and
actually declaring the variable: explicit declarations are always
placed in the generated code, but type information from a |the|
form is not turned into an actual declaration unless you tell \iter\
to do so using |iterate:declare-variables|.  See below.

\begin{sloppypar}
Declaring the types of internal variables is harder than declaring the
types of explicitly mentioned variables, since their names
are unknown.  You do it by declaring |iterate:declare-variables|
somewhere inside the top level of the \iter\ form.  (This will also
generate declarations for variables declared using |the|.)
\iter\ does not provide much selectivity here: it's all or none. 
And unfortunately, since \iter\ is not privy to compiler information
but instead reads declarations itself, it will not hear if you
|(declaim (iterate:declare-variables))|.  Instead, set the variable 
|iterate::*always-declare-variables*| to |t| at
compile-time, using |eval-when|.
\end{sloppypar}

To determine the appropriate types for internal variables, \iter\ uses
three sources of information:

\begin{itemize}

\item Often, the particular clause dictates a certain type for a
variable; \iter\ will use this information when available.  In the
current example, the variable |list17| will be given the type
|list|, since that is the only type that makes sense; and the
variable |result| will be given the type |fixnum|, on the
assumption that you will not be counting high enough to need bignums.
You can override this assumption only by using and explicitly declaring a
variable: 
\begin{verbatim}
(iter (declare (iterate:declare-variables))
      (for el in number-list)
      (count (oddp el) into my-result)
      (declare (integer my-result))
      (finally (return my-result)))
\end{verbatim}

\begin{sloppypar}
Other examples of the type assumptions that \iter\ makes are: type
|list| for |into| variables of collection clauses; type |list| for
expressions that are to be destructured; type |vector| for the
variable holding the vector in a |for\dots in-vector| clause, and
similarly for |string| and the |for\dots in-string| clause;
and the implementation-dependent type
|(type-of array-dimension-limit)| for the index and limit
variables generated by sequence iteration drivers like |for\dots
in-vector| and |for\dots in-string| (but not |for\dots in-sequence|,
because it may be used to iterate over a list). 
\end{sloppypar}

\item Sometimes, \iter\ will examine expressions and try to determine
their types in a simple-minded way.  If the expression is
self-evaluating (like a number, for instance), \iter\ knows that the
expression's type is the same as the type of the value it denotes, so
it can use that type.  If the expression is of the form |(the ~type~
~expr~)|, \iter\ is smart enough to extract ~type~ and use it.
However, the current version of \iter\ does not  
examine declarations of function result types or do any type
inference.  It will not determine, for 
example, that the type of |(+ 3 4)| is |fixnum|, or even
|number|.

\item In some cases, the type of an internal variable should match the
type of some other variable.  For instance, \iter\ generates an
internal variable for |(f x)| in the
clause |(for i from 1 to (f x))|, and in the absence of other
information will give it the same type as |i|.  If, however, the
expression had been written |(the fixnum (f x))|, then \iter\
would have given the internal variable the type |fixnum|
regardless of |i|'s type.  The type incompatibility errors that
could arise in this situation are not checked for.

\end{itemize}

Note that if you do declare |iterate:declare-variables|, then
\iter\ may declare user variables as well as internal ones if they do
not already have declarations, though only for variables that it
binds.  For instance, in this code:

\begin{program}
(iter (declare (iterate:declare-variables))
      (for i from 1 to 10)
      (collect i into var))
\end{program}
the variable |var| will be declared to be of type |list|.


\subsection{Summary}

\iter\ understands standard Common Lisp variable type declarations
that occur within an \iter\ form and
will pass them through to the generated code.  If the declaration
|(iterate:declare-variables)|\lispindex{declare-variables}
appears at the top level of an 
\iter\ form, or if 
|iterate::*always-declare-variables*|\lispindex{*always-declare-variables*} 
is \nonnil, then \iter\ will use the type information gleaned from user
declarations, self-evaluating expressions and |the| expressions,
combined with reasonable assumptions, to determine variable
types and declare them.


%%%%%%%%%%%%%%%%%%%%%%%%%%%%%%%%%%%%%%%%%%%%%%%%%%%%%%%%%%%%%%%%
\section{Problems with Code Movement}
\label{code-movement}

Some \iter\ clauses, or parts of clauses, result in code being
moved from the location of the clause to other parts of the loop.
Drivers behave this way, as do code-placement clauses like |initially|
and |finally|.  When using these clauses, there is a danger of writing
an expression that makes sense in its apparent location but will be
invalid or have a different meaning in another location.  For example:
\begin{program}
(iter (for i from 1 to 10)
      (let ((x 3))
        (initially (setq x 4))))
\end{program}
While it may appear that the |x| of |(initially (setq x 4))| is the
same as the |x| of |(let ((x 3)) \dots|, in fact they are not:
|initially| moves its code outside the loop body, so |x| would refer
to a global variable.  Here is another example of the same problem:
\begin{program}
(iter (for i from 1 to 10)
      (let ((x 3))
        (collect i into x)))
\end{program}
If this code were executed, |collect| would create a binding for its
|x| at the top level of the \iter\ form that the |let| will shadow.

Happily, \iter\ is smart enough to catch these errors; it
walks all problematical code to ensure that free variables are not
bound inside the loop body, and checks all variables it binds for the
same problem.

However, some errors cannot be caught:

\begin{program}
(iter (with x = 3)
      (for el in list)
      (setq x 1)
      (reducing el by \#'+ initial-value x))
\end{program}
|reducing| moves its |initial-value| argument to the initialization
part of the loop in order to produce more efficient code.  Since
\iter\ does not perform data-flow analysis, it cannot determine that
|x| is changed inside the loop; all it can establish is that |x| is
not bound internally.  Hence this code will not signal an
error and will use $3$ as the initial value of the reduction.

The following list summarizes all cases that are subject to these code
motion and variable-shadowing problems.
\begin{itemize}
\item Any variable for which \iter\ creates a binding, including those
used in |with| and the |into| keyword of many clauses.

\begin{sloppypar}
\item The special clauses which place code: |initially|, |after-each|, |else|,
|finally| and |finally-protected|. 
\end{sloppypar}

\item The variables of a |next| or |do-next| form.

\item The |initially| arguments of |for\dots initially\dots then| and
|for\dots previous|. 

\item The |then| argument of |for\dots initially\dots then|.

\item The |initial-value| arguments of |reducing| and |accumulate|.

\item The |on-failure| argument of |finding\dots such-that|.
\end{itemize}

%%%%%%%%%%%%%%%%%%%%%%%%%%%%%%%%%%%%%%%%%%%%%%%%%%%%%%%%%%%%%%%%
\section{Differences Between {\tt Iterate} and {\tt Loop}}
|loop| contains a great deal of complexity which \iter\ tries to
avoid.  Hence many esoteric features of |loop| don't exist in \iter.
Other features have been carried over, but in a cleaned-up form.  
And of course, many new features have been added; they are not
mentioned in this list.

\begin{itemize}

\item \iter's syntax is more Lisp-like than |loop|'s, having a higher
density of parens.

\item The current implementation of \iter, unlike the current version
of  |loop| (as documented in {\em Common Lisp, 2nd Ed.\/}), is
extensible (see section \ref{extend}).  

\item |loop| puts the updates of all driver variables at the top of
the loop; \iter\ leaves them where the driver clauses appear.

\item While for the most part \iter\ clauses that resemble |loop| clauses
behave similarly, there are some differences.  For instance, there is
no |for\dots =\dots then| in \iter; instead use 
|for\dots initially\dots then|.  

\item |loop| binds the variable |it| at certain times
to allow pseudo-English expressions like |when ~expr~ return it|. 
In \iter, you must bind ~expr~ to a variable yourself.  Note that
|when ~expr~ return it| is like |thereis ~expr~| except that the latter is an
accumulation clause and therefore competes with other accumulations
(remember section \ref{multiple} above).

% repeat different behaviour of |always| clause here?

\item |loop| has a special |return| clause, illustrated in the
previous item.  \iter\ doesn't need one,  since an ordinary Lisp
|return| has the same effect.

\item |loop| allows for parallel binding and stepping of iteration
variables.  \iter\ does not.  (See section \ref{bindings}.)

\item |loop| and \iter\ handle variable type declarations very
differently.  |loop| provides a special syntax for declaring variable
types, and does not examine declarations.  Moreover, the standard
implementation of |loop| will
generate declarations when none are requested.
\iter\ parses standard Common Lisp type declarations, and will never
declare a variable itself unless declarations are 
specifically requested.

\end{itemize}

%%%%%%%%%%%%%%%%%%%%%%%%%%%%%%%%%%%%%%%%%%%%%%%%%%%%%%%%%%%%%%%%
\section{Rolling Your Own}
\label{extend}

\subsection{Introduction}

\iter\ is extensible---you can write new clauses that embody new
iteration patterns.  You might want to write a new driver clause for a
data structure of your own, or you might want to write a clause that
collects or manipulates elements in a way not provided by \iter. 

This section describes how to write clauses for \iter.  Writing a
clause is like writing a macro.  In fact, writing a clause ~is~
writing a macro: since \iter\ code-walks its body and macroexpands,
you can add new abstractions to \iter\ with good old |defmacro|.

Actually, there are two extensions you can make to \iter\ that are
even easier than writing a macro.  They are adding a synonym for an
existing clause and defining a driver clause for an indexable
sequence.  These can be done with |defsynonym| and
|defclause-sequence|, respectively.  See section \ref{aids}, below.

The rest of this section explains how to write macros that expand into
\iter\ clauses.
Here's how you could add a simplified version of \iter's |multiply|
clause, if \iter\ didn't already have one:
\begin{program}
(defmacro multiply (expr)
  `(reducing ,expr by \#'* initial-value 1))
\end{program}

If you found yourself summing the square of an expression often, you
might want to write a macro for that.  A first cut might be
\begin{program}
(defmacro sum-of-squares (expr)
  `(sum (* ,expr ,expr)))
\end{program}
but if you are an experienced macro writer, you will realize that this
code will evaluate ~expr~ twice, which is probably a bad idea.  A
better version would use a temporary:
\begin{program}
(defmacro sum-of-squares (expr)
  (let ((temp (gensym)))
    `(let ((,temp ,expr))
       (sum (* ,temp ,temp)))))
\end{program}
Although this may seem complex, it is just the sort of thing you'd
have to go through to write any macro, which illustrates the point of
this section: if you can write macros, you can extend \iter.

Our macros don't use \iter's keyword-argument
syntax.  We could just use keywords with |defmacro|, but we would
still not be using \iter's clause indexing mechanism.  Unlike Lisp,
which uses just the first symbol of a form to determine what function
to call, \iter\ individuates clauses by the list of required keywords.
For instance, |for\dots in| and |for\dots in-vector| are different clauses
implemented by distinct Lisp functions.  

To buy into this indexing scheme, as well as the keyword-argument
syntax, use |defmacro-clause|:

\begin{clauses}

\defmacro{defmacro-clause}{arglist |\&body| body}
Defines a new \iter\ clause.  ~arglist~ is a list of symbols which are
alternating keywords and arguments.  \opt\ may be used, and the list
may be terminated by |\&sequence|.  ~body~ is an ordinary macro body,
as with |defmacro|.  If the first form of ~body~ is a string, it is
considered a documentation string and will be shown by
|display-iterate-clauses|. |defmacro-clause| will signal an error if 
defining the clause would result in an ambiguity.  E.g. you cannot
define the clause |for\dots from| because there would be no way to
distinguish it from a use of the |for| clause with optional keyword |from|.

\end{clauses}

\medskip

Here is |multiply| using |defmacro-clause|.  The keywords are capitalized
for readability.
\begin{program}
(defmacro-clause (MULTIPLY expr \opt\ INTO var)
  `(reducing ,expr by \#'* into ,var initial-value 1))
\end{program}
You don't have to worry about the case when |var| is not supplied; for
any clause with an |into| keyword, saying |into nil| is equivalent to
omitting the |into| entirely.

As another, more extended example, consider the fairly common
iteration pattern that 
involves finding the sequence element that maximizes (or minimizes) some
function.  \iter\ provides this as |finding\dots maximizing|, but it's
instructive to see how to write it.
Here, in pseudocode, is how you might write such a loop for
maximizing a function F:

\begin{quote}
\begin{tabbing}
set variable MAX-VAL to NIL; \\
set variable WINNER to NIL; \\
for\=\ each element EL in the sequence \\
\>  if\=\ MAX-VAL\=\ is NIL or F(EL) $>$ MAX-VAL then \\
\>\>    set MAX-VAL to F(EL); \\
\>\>    set WINNER to EL; \\
\>  end if; \\
end for; \\
return WINNER.
\end{tabbing}
\end{quote}

Here is the macro:
\begin{program}
(defmacro-clause (FINDING expr MAXIMIZING func \opt\ INTO var)
  (let ((max-val (gensym))
        (temp1 (gensym))
        (temp2 (gensym))
        (winner (or var iterate::*result-var*)))
    `(progn 
       (with ,max-val = nil)
       (with ,winner = nil)
       (cond 
        ((null ,max-val)
         (setq ,winner ,expr)
         (setq ,max-val (funcall ,func ,winner))
        (t
         (let* ((,temp1 ,expr)
                (,temp2 (funcall ,func ,temp1)))
           (when (> ,temp2 ,max-val)
             (setq ,max-val ,temp2)
             (setq ,winner ,temp1))))))
       (finally (leave ,winner)))))
\end{program}
Note that if no |into| variable is supplied, we use
|iterate::*result-var*|, which contains the internal variable into
which all clauses place their results.  If this variable is bound by
some clause, then \iter\ will return its value automatically;
otherwise, \nil\ will be returned.  

\subsection{Writing Drivers}

In principle, drivers can be implemented just as easily as other
\iter\ clauses.  In practice, they are a little harder to get right. 
As an example, consider writing a driver that
iterates over all the 
elements of a vector, ignoring its fill-pointer.  |for\dots in-vector|
won't work for this, because it observes the fill-pointer.  It's necessary to
use |array-dimension| instead of |length| to obtain the size of the
vector.  Here is one approach:

\begin{program}
(defmacro-clause (FOR var IN-WHOLE-VECTOR v)
  "All the elements of a vector (disregards fill-pointer)"
  (let ((vect (gensym))
        (index (gensym)))
    `(progn
       (with ,vect = ,v)
       (for ,index from 0 below (array-dimension ,vect 0))
       (for ,var = (aref ,vect ,index)))))
\end{program}
Note that we immediately put |v| in a variable, in case it is an
expression.  Again, this is just good Lisp macrology.  It also has a
subtle effect on the semantics of the driver: |v| is evaluated only
once, at the beginning of the loop, so changes to |v| in the loop have
no effect on the driver.  Similarly, the bounds for numerical iteration
e.g. the above |array-dimension| are also evaluated once only.  This is how
all of \iter's drivers work.


There is an important point concerning the |progn| in this code.  We
need the |progn|, of course, because we are returning several forms,
one of which is a driver.  But \iter\ drivers must occur at top-level.
Is this code in error?  No, because ~top-level~ is defined in \iter\
to include forms inside a |progn|.  This is just the definition of
top-level that Common Lisp uses, and for the same reason: to allow
macros to return multiple forms at top-level.

While our |for\dots in-whole-vector| clause will work, it is not
ideal.  In particular, it does not support generating.  Do do so, we
need to use |for\dots next| or |for\dots do-next|.  The job is
simplified by the |defmacro-driver| macro.

\begin{clauses}

\defmacro{defmacro-driver}{arglist |\&body| body}
Defines a driver clause in
both the |for| and |generate| forms, and provides a parameter
|generate| which ~body~ can examine to determine how it was invoked.
~arglist~ is as in |defmacro-clause|, and should begin with the symbol
|for|. 

\end{clauses}

With |defmacro-driver|, our driver looks like this:
\begin{program}
(defmacro-driver (FOR var IN-WHOLE-VECTOR v)
  "All the elements of a vector (disregards fill-pointer)"
   (let ((vect (gensym))
         (end (gensym))
         (index (gensym))
         (kwd (if generate 'generate 'for)))
     `(progn
        (with ,vect = ,v)
        (with ,end = (array-dimension ,vect 0))
        (with ,index = -1)
        (,kwd ,var next (progn (incf ,index)
                               (if (>= ,index ,end) (terminate))
                               (aref ,vect ,index))))))
\end{program}


We are still missing one thing: the |\&sequence| keywords.  
We can get them easily enough, by writing 
\begin{program}
(defmacro-driver (FOR var IN-WHOLE-VECTOR v \&sequence) 
  ...)
\end{program}
We can now refer to parameters |from|, |to|, |by|, etc. which contain
either the values for the corresponding keyword, or \nil\ if the
keyword was not supplied.  Implementing the right code for these
keywords is cumbersome but not difficult; it is left as an exercise.
But before you begin, see |defclause-sequence| below for an easier way.
       
\subsection{Extensibility Aids}
\label{aids}

This section documents assorted features that may be of use in
extending \iter.

\begin{clauses}

\defunexpvar{*result-var*}
Holds the variable that is used to return a value as a result of the
\iter\ form.  You may examine this and use it in a |with| clause, but
you should not change it.

\defmacro{defsynonym}{syn word}
 Makes ~syn~ a synonym for the existing \iter\ keyword ~word.~  Only
the first word in each clause can have synonyms. 

\defmacroxx{defclause-sequence}{element-name index-name}{|\&key|
access-fn size-fn sequence-type}{element-type
element-doc-string index-doc-string}  
Provides
a simple way to define sequence clauses.  Generates two
clauses, one for iterating over the sequence's elements, the other
for iterating over its indices.  The first symbol of both
clauses will have print-name |for|.
~element-name~  and ~index-name~ should be symbols.
~element-name~ is the second keyword of the element iterator (typically of
the form 
|in-~sequence-type~|), and ~index-name~ is the second keyword
of the index-iterator (typically of the form
|index-of-~sequence-type~|).  Either name may be 
\nil, in which case the corresponding clause is not defined.  If both
symbols are supplied, they should be in the same package.  The |for|
that begins the clauses will be in this package.

\cpar ~access-fn~ is the function to be used to
access elements of the sequence in the element iterator.  The function
should take two 
arguments, a sequence and an index, and return the appropriate element.
~size-fn~ should denote a function of one argument, a sequence, that
returns its size.  Both ~access-fn~ and ~size-fn~ are required for the
element iterator, but only ~size-fn~ is needed for the index iterator.

\cpar The ~sequence-type~ and ~element-type~ keywords are used to
suggest types for the variables 
used to hold the sequence and the
sequence elements, respectively.  The usual rules about \iter's
treatment of variable type declarations apply (see section \ref{types}).

\cpar ~element-doc-string~ and ~index-doc-string~ are
the documentation strings, for use with |display-iterate-clauses|. 

\cpar The generated element-iterator performs destructuring on the
element variable.

\cpar As an example, the above |for\dots in-whole-vector| example
could have been written:
\begin{program}
(defclause-sequence IN-WHOLE-VECTOR INDEX-OF-WHOLE-VECTOR
  :access-fn 'aref
  :size-fn \#'(lambda (v) (array-dimension v 0))
  :sequence-type 'vector
  :element-type t
  :element-doc-string 
     "Elements of a vector, disregarding fill-pointer"
  :index-doc-string 
     "Indices of vector, disregarding fill-pointer")
\end{program}

\end{clauses}

\subsection{Subtleties}

There are some subtleties to be aware of when writing \iter\ clauses.
First, the code returned by your macros may be |nconc|'ed into a list,
so you should always returned freshly consed lists, rather than
constants.  Second, \iter\ matches clauses by using |eq| on the first
symbol and |string=| on the subsequent ones, so the package of the
first symbol of a clause is relevant.  All of the clauses in this manual
have their first word in the \iter\ package.  
You can use the package system in the usual way to shadow
\iter\ clauses without replacing them.

%%% say more here, about the badness that only the first word of a
%%% clause is packagey.

\section{Non-portable Extensions to Iterate (Contribs)}
\label{contribs}

Currently, there is only one non-portable extension to iterate in the
distribution: iterate-pg. If you have made an extension that depends
on non-portable features, feel free to send them to |asf@boinkor.net|
for inclusion in the iterate distribution.

\subsection{An SQL query driver for iterate}

The pg package by Eric Marsden (see |http://cliki.net/pg|) provides an
interface to the PostgreSQL database. Using the \iterpg\ extension, it
is possible to handle the results of SQL queries using \iter.

This usage example should give you an idea of how to use it:

\begin{program}
(pg:with-pg-connection (c "somedb" "someuser")
  (iter (for (impl version date) in-relation "select * from version"
                                 on-connection *dbconn*)
        (collect version)))
\end{program}

To use the extension via |ASDF|, simply make your system depend on the
|iterate-pg| system instead of the |iterate| system. To load it
manually, use:

\begin{program}
  (asdf:oos 'asdf:load-op :iterate-pg)
\end{program}

%%%%%%%%%%%%%%%%%%%%%%%%%%%%%%%%%%%%%%%%%%%%%%%%%%%%%%%%%%%%%%%%
\section{Obtaining {\tt Iterate}}

\begin{sloppypar}
The information in this chapter is obsolete but included for
completeness's sake; Currently, the most up-to-date information on
\iter\ can be found at |http://boinkor.net/iterate.html|.
\end{sloppypar}

\begin{sloppypar}
\iter\ currently runs on Lisp Machines, and on
HP's, Sun3's and Sparcstations under Lucid.  
\iter\ source and binaries are available at the MIT AI Lab in the
subdirectories of |/src/local/lisplib/|.  The source file,
|iterate.lisp|, is also available for anonymous FTP in the directory
|/com/fpt/pub/| on the machine |TRIX.AI.MIT.EDU| (Internet number
128.52.32.6).  If you are unable to obtain |iterate| in one of these
ways, send mail to |jba@ai.mit.edu| and I will send you the source
file. 
\end{sloppypar}

\begin{sloppypar}
\iter\ resides in the |iterate| package (nickname |iter|).  Just say
\linebreak |(use-package :iterate)| to make all the necessary symbols
available. 
If a symbol is not exported, it appears in this manual with an
``|iterate::|'' prefix.
\end{sloppypar}

Send bug reports to |bug-iterate@ai.mit.edu|.  The |info-iterate|
mailing list will have notices of changes and problems; to have
yourself added, send mail to |info-iterate-request@ai.mit.edu|.


\medskip

\begin{flushleft}
 \bf Acknowledgements
\end{flushleft}
\smallskip

Richard Waters provided invaluable criticism which spurred me to improve
\iter\ greatly.  As early users, David Clemens, Oren Etzioni and Jeff
Siskind helped ferret out many bugs.

%\begin{theindex}
% The index files must be generated with the genindex program
% or from makeindex iterate-manual.idx which creates iterate-manual.ind.

%\baselineskip\normalbaselineskip
%\advance\baselineskip by -2pt

\input{iterate-manual.ind}

%\underline{Clauses}
%\input{iterate-manual.clindex}
%
%\indexspace
%\underline{Lisp}
%\input{iterate-manual.lispindex}

%\end{theindex}


\end{document}


% arch-tag: "c0f871a6-313c-11d8-abb9-000c76244c24"

%
%\indexspace
%\underline{Lisp}
%% Fix ``.'' spacing in lists
\documentstyle[12pt,aimemo]{article}

\setlength{\textwidth}{6.0in}
\setlength{\oddsidemargin}{0.25in}
\setlength{\evensidemargin}{0.25in}
\setlength{\topmargin}{0pt}
\setlength{\headsep}{24pt}
\setlength{\textheight}{8.5in}

\let\origpar=\par

% somehow, LaTex fucks with things so that the plain Tex \obeyspaces
% doesn't work.  This is the same, except we use an mbox containing
% a space instead of just a space.

\newcommand{\spc}{\mbox{ }}
\newcommand{\tAb}{\mbox{        }}
\def\obspaces{\catcode`\ =\active\catcode`\^^I=\active}
{\obspaces \global\let =\spc\global\let^^I=\tAb}

\newenvironment{program}{\medskip\samepage\tt\obeylines\obspaces}{\medskip}

% use | for \tt
\catcode`\|=13
\newif\ifvbar
\def|{\ifvbar
         \endgroup\vbarfalse
      \else 
         \begingroup\tt\vbartrue 
      \fi}



\newcommand{\lisp}{\tt}
\newcommand{\iter}{|iterate|}
\newcommand{\iterpg}{|iterate-pg|}
\newcommand{\setf}{|setf|}
\newcommand{\nil}{|nil|}
\newcommand{\nonnil}{non-\nil}
\newcommand{\opt}{{\lisp \&optional}}
\newcommand{\yields}{$\Rightarrow$}



\newbox\Dots
\setbox\Dots=\hbox{\small ...}
\renewcommand{\dots}{\copy\Dots}

% Here's a way to define an environment which will treat all
% paragraphs inside it in the same manner.  It's, as near as I can
% figure, the way that Latex does things.  Using \everypar won't work;
% apparently Latex resets that a lot.  What you have to do is: at the
% top of the file do \let\origpar=\par.  In the
% begin part of the environment, do the parshape (or whatever), then say
% \def\par{{\origpar}}.  The extra braces are necessary.   That's all
% you have to do; at the end of the environment things will return to
% normal automatically.  Why does this all work?  I have no idea.
% Another important thing: there should be a blank line (i.e. a \par)
% between the end of the last paragraph and the \end{environment}.

\newlength{\clindent}
\setlength{\clindent}{0.5in}
\newlength{\clparindent}
\setlength{\clparindent}{\parindent}

\newenvironment{clauses}{\advance\linewidth -\parindent
                        \hangindent\clindent\relax
                         \hangafter=0
                          \parindent=0pt
                          \def\par{{\origpar}}}{}

\newcommand{\cpar}{\hspace \clparindent}

\newcommand{\startitem}{\bigskip\pagebreak[3]}
\newcommand{\finishitem}{\smallskip}

\newcommand{\clause}[1]{\startitem\Clause{#1}\finishitem}
\newcommand{\Clause}[1]{\hbox{#1}}
\newcommand{\clindex}[1]{\index{{\lisp #1}}}
\newcommand{\clindexx}[2]{\index{{\lisp #1\dots #2}}}
\newcommand{\clindexxx}[3]{\index{{\lisp #1\dots #2\dots #3}}}
\newcommand{\clausex}[2]{\startitem\Clausex{#1}{#2}\finishitem}
\newcommand{\Clausex}[2]{\Clause{|#1| #2}\clindex{#1}}

\newcommand{\defvar}[1]{\startitem\deff{#1}{}{Variable}\finishitem}
\newcommand{\defunexpvar}[1]{\startitem\deffnoindex{iterate::#1}{}{Variable}
                                        \lispindex{#1}\finishitem}
\newcommand{\defun}[2]{\startitem\deff{#1}{#2}{Function}\finishitem}
\newcommand{\defmacro}[2]{\startitem\deff{#1}{#2}{Macro}\finishitem}
\newcommand{\defunx}[3]{\startitem\deffx{#1}{#2}{#3}{Function}\finishitem}
\newcommand{\defmacrox}[3]{\startitem\deffx{#1}{#2}{#3}{Macro}\finishitem}
\newcommand{\defunxx}[4]{\startitem\deffxx{#1}{#2}{#3}{#4}{Function}\finishitem}
\newcommand{\defmacroxx}[4]{\startitem\deffxx{#1}{#2}{#3}{#4}{Macro}\finishitem}

\newcommand{\deff}[3]{\deffnoindex{#1}{#2}{#3}\lispindex{#1}}
\newcommand{\deffnoindex}[3]{\hbox to \hsize{{\tt #1} {\it #2}\hfill [{\it #3}]}}
\newcommand{\deffx}[4]{\deff{#1}{#2}{#4}\moreargs{#1}{#3}}
\newcommand{\deffxx}[5]{\deffx{#1}{#2}{#3}{#5}\moreargs{#1}{#4}}

\newcommand{\moreargs}[2]{\hbox to \hsize{\phantom{\lisp #1} {\it #2}\hfill}}

\newcommand{\lispindex}[1]{\index{{\lisp #1}}}

 \newenvironment{note}[1]{\pagebreak[2]\bigskip
                          \hrule\smallskip\small
                         {\setlength{\parindent}{0pt}
                           \par{\bf #1:}}}{\smallskip\hrule\bigskip}



\makeindex

%%%%%%%%%%%%%%%%%%%%%%%%%%%%%%%%%%%%%%%%%%%%%%%%%%%%%%%%%%%%%%%%

%\thispagestyle{empty}

%\vspace*{2in}

%\title{The |Iterate| Manual}
%\author{Jonathan Amsterdam}

%\maketitle

%\thispagestyle{empty}

%\begin{center}
%\Large *** DRAFT ***
%\end{center}
%This manual is in beta-test. It will become an AI Memo in a couple of
%months.  I would appreciate any comments or
%corrections.  You can reach me at email address |jba@ai.ai.mit.edu|.

\begin{document}

\def\writtenby{Jonathan Amsterdam}
\def\contractno{This report describes research done at the Artificial
Intelligence Laboratory of the Massachusetts Institute of Technology.
Support for the laboratory's artificial intelligence research is
provided in part by the Advanced Research Projects Agency of the
Department of Defense under Office
of Naval Research contract N00014-85-K-0124.}

\AIMEMO{1236}{\Large The Iterate Manual \\[1in]}
{This is the manual for version 1.4 of \iter, a powerful iteration
macro for Common  
Lisp. \iter\ is similar to {\tt loop} but provides numerous additional
features, is well integrated with Lisp, and is extensible. 
\vfill }


% use ~ for \em
\catcode`\~=13
\newif\ifem
\def~{\ifem
         \/\endgroup\emfalse
      \else 
         \begingroup\em\emtrue 
      \fi}



%\pagebreak
%\setcounter{page}{1}

%\thispagestyle{empty}


\tableofcontents

\pagebreak
\pagestyle{headings}

\section{Introduction}

\begin{sloppypar}
This manual describes
\iter, a powerful iteration facility for Common Lisp.
\iter\ provides abstractions for many common iteration
patterns and allows for the definition of additional patterns.  \iter\ is
a macro that expands into ordinary Lisp at
compile-time, so it is more efficient than higher-order
functions like |map| and |reduce.|  
While it is similar to |loop|, \iter\ offers a more Lisp-like syntax
and enhanced extensibility.
(For a more complete
comparison of \iter\ with other iteration constructs, see MIT AI Lab Working
Paper No. 324, ~Don't Loop, Iterate.~)
\end{sloppypar}

An \iter\ form consists of the symbol |iter|\footnote{You can also use
|iterate|, but |iter| is preferred because it avoids potential conflicts with
possible future additions to Common Lisp, and because it saves
horizontal space when writing code.}
followed by one or more
forms, some of which may be \iter\ 
~clauses.~  Here is a simple example of \iter\ which collects the
numbers from 1 to 10 into a list, and returns the list.  The return
value is shown following the arrow.
\begin{program}
(iter (for i from 1 to 10)
      (collect i))          \yields (1 2 3 4 5 6 7 8 9 10)
\end{program}
This form contains two clauses: a |for| clause that steps the
variable |i| over the integers from 1 to 10, and a |collect|
clause that accumulates its argument into a list.  With a few
exceptions, all \iter\ clauses have the same format: alternating
symbols (called ~keywords~) and expressions (called
~arguments~).  The syntax and terminology are those of Common
Lisp's keyword lambda lists.  One difference is that \iter's keywords
do not have to begin with a colon---though they may, except for the
first symbol of a clause.  So you can also write  |(for i :from
1 :to 10)| if you prefer.

Any Lisp form can appear in the body of an \iter, where it will have
its usual meaning.  \iter\ walks the entire body, expanding macros,
and recognizing clauses at any level.  This example collects all the
odd numbers in a list:
\begin{program}
(iter (for el in list)
      (if (and (numberp el) (oddp el))
          (collect el)))
\end{program}

There are clauses for iterating over numbers, lists, arrays and other
objects, and for
collecting, summing, counting, maximizing and other useful operations.
\iter\ also supports the creation of new variable bindings, stepping
over multiple sequences at once, destructuring, and compiler
declarations of variable types.  The following example illustrates
some of these features: 
\begin{program}
(iter (for (key . item) in alist)
         (for i from 0)
         (declare (fixnum i))
         (collect (cons i key)))
\end{program}

This loop takes the keys of an alist and returns a new alist
associating the keys with their positions in the original list. The
compiler declaration for |i| will appear in the generated code
in the appropriate place.

%%%%%%%%%%%%%%%%%%%%%%%%%%%%%%%%%%%%%%%%%%%%%%%%%%%%%%%%%%%%%%%%
\section{Clauses}

Most of \iter's clauses will be familiar to |loop| programmers.  
(|loop| is an iteration macro that has been incorporated into Common
Lisp.  Seen Guy Steele's {\em Common Lisp, 2nd Edition}.)
In
nearly all cases they behave the same as their |loop| counterparts, so
a |loop| user can switch to \iter\ with little pain (and much gain).

All clauses with the standard keyword-argument syntax consist of two
parts: a ~required~ part, 
containing keywords that must be present and in the right order; and
an ~optional~ part, containing keywords that may be omitted and,
if present, may occur in any order.  In the descriptions below, the
parts are separated by the Lisp lambda-list keyword \opt.


\subsection{Drivers}

An iteration-driving clause
conceptually causes the iteration to go forward.  Driver clauses in
\iter\ allow iteration over numbers, lists, vectors, hashtables, packages,
files and streams.  Iteration-driving clauses must
appear at the top level of an \iter\ form; they cannot be nested
inside another clause.  The driver variable is updated at the point
where the driver clause occurs.  Before the clause is executed for the
first time, the value of the variable is undefined.

%Also, regardless of where the driver clause
%appears in the body, the driver variable is stepped at the top of the
%loop; hence it is stylistically preferable, though
%not required, to place driver clauses at the beginning of the \iter.

Multiple drivers may appear in a single \iter\ form, in which case all
of the driver variables are updated each time through the loop, in the
order in which the clauses appear.  The first driver to terminate will
terminate the entire loop.

In all cases, the value of the driver variable on exit from the loop,
including within the epilogue code (see the |finally| clause), is
undefined.

All the parameters of a driver clause are evaluated once, before the
loop begins.  Hence it is not possible to change the bounds or other
properties of an iteration by side-effect from within the loop.

With one exception, driver clauses begin with the word |for| (or
the synonym |as|) and mention an iteration variable, which is
given a binding within the \iter\ form.
The exception is |repeat|, which just executes a loop a
specified number of times:

\begin{clauses}

\clausex{repeat}{~n~}
Repeats the loop ~n~ times.  For example:
\begin{program}
(iter (repeat 100)
      (print "I will not talk in class."))
\end{program}
\cpar If $n \leq 0$, the
loop will never be executed.  If ~n~ is not an integer, the
actual number of executions will be $\lceil n \rceil$.

\end{clauses}

\pagebreak[3]

\subsubsection{Numerical Iteration}

\begin{clauses}
\clausex{for}{~var~ |\&sequence|} 
The general form for iterating over a sequence of numbers requires a
variable and, optionally, one or more keywords that provide the bounds
and step size of the iteration. The |\&sequence|
lambda-list keyword is a shorthand for these sequence keywords.  They are:
|from|, |upfrom|, |downfrom|, |to|, |downto|, |above|, |below| and |by|.
|from| provides the starting value for ~var~
and defaults to zero. |to| provides a final value and implies that the
successive values of ~var~ will be increasing; |downto| implies that
they will be decreasing.  The loop terminates when ~var~ passes
the final value (i.e. becomes smaller or larger than it, depending on
the direction of iteration); in other  words, the loop body
will never be 
executed for values of ~var~ past the final value.
|below| and |above| are similar to |to| and
|downto|, except that the loop terminates when ~var~ equals or passes
the final value.

\cpar If no final value is specified, the variable will be stepped
forever.  Using |from| or |upfrom| will result in increasing
values, while |downfrom| will give decreasing values.

\cpar On each iteration,
~var~ is incremented or decremented by the value of the sequence
keyword |by|, which defaults to 1.  It should always be a positive
number, even for downward iterations.

\cpar In the following examples, the sequence of numbers generated is shown
next to the clause.

\begin{program}
(for i upfrom 0) \yields\ 0 1 2 \ldots
(for i from 5)   \yields\ 5 6 7 \ldots    ; either from or upfrom is okay
(for i downfrom 0) \yields\ 0 -1 -2 \ldots
(for i from 1 to 3) \yields\ 1 2 3
(for i from 1 below 3) \yields\ 1 2
(for i from 1 to 3 by 2) \yields\ 1 3
(for i from 1 below 3 by 2) \yields\ 1
(for i from 5 downto 3) \yields\ 5 4 3
\end{program}

\end{clauses}

\subsubsection{Sequence Iteration}

There are a number of clauses for iterating over sequences.  In all of
them, the argument following |for| may be a list instead of a
symbol, in which case destructuring is performed.  See section
\ref{destructuring}. 

\begin{clauses}

\clause{|for| ~var~ |in| ~list~ \opt\ |by| ~step-function~}
\clindexx{for}{in}
~var~ is set to successive elements of list.  
~step-function,~ which 
defaults to |cdr|, is used to obtain the next sublist. 

\clause{|for| ~var~ |on| ~list~ \opt\ |by| ~step-function~}
\clindexx{for}{on}
~var~ is set to successive sublists of list. 
~step-function~ (default |cdr|) is used as in |for\dots in|.

\end{clauses}

\medskip

These two clauses use |atom| to test for the end
of a list.  Hence, given a list whose final |cdr| is not \nil,
they will silently ignore the last |cdr|.  Other choices are
|endp|,  which would signal an error, and |null|, which
would probably result in an error somewhere else.  If you wish to use
an end-test other than |atom|, set the variable
|iterate::*list-end-test*|\lispindex{*list-end-test*} to the name of
the desired function. 

\begin{clauses}
\clause{|for| ~var~ |in-vector| ~vector~ |\&sequence|}
\clindexx{for}{in-vector}
~var~ takes on successive elements from ~vector.~   The vector's
fill-pointer is observed.  Here and in subsequent clauses, the |\&sequence|
keywords include |with-index|, which takes a symbol as argument
and uses it for the index variable instead of an internally generated
symbol.  The other |\&sequence| keywords behave as in numerical
iteration, except that the default iteration bounds are the bounds of
the vector.  E.g. in |(for i in-vector v downto 3)|,
|i| will start off being bound to the last element in |v|,
and will  
be set to preceding elements down to and including the element with
index 3.

\clause{|for| ~var~ |in-sequence| ~seq~ |\&sequence|}
\clindexx{for}{in-sequence}
This uses Common Lisp's generalized sequence functions, |elt|
and |length|, to obtain elements and determine the length of
~seq.~   Hence it will work for any sequence, including lists, and will
observe the fill-pointers of vectors.

\clause{|for| ~var~ |in-string| ~string~ |\&sequence|}
\clindexx{for}{in-string}
~var~ is set to successive characters of ~string.~

\startitem
\Clause{|for| ~var~ |index-of-vector| ~vector~ |\&sequence|}
\clindexx{for}{index-of-vector}
\Clause{|for| ~var~ |index-of-sequence| ~sequence~ |\&sequence|}
\clindexx{for}{index-of-sequence}
\Clause{|for| ~var~ |index-of-string| ~string~ |\&sequence|}
\clindexx{for}{index-of-string}
\finishitem
~var~ is set to successive indices of the sequence.
These clauses avoid the overhead of accessing the sequence elements
for those applications where they do not need to be examined, or are
examined rarely.  They admit all the optional keywords of the other
sequence drivers except the (redundant) |with-index| keyword.

\clause{|for| ~(key value)~ |in-hashtable| ~table~}
\clindexx{for}{in-hashtable}

~key~ and ~value,~ which must appear as shown in a list
and may be destructuring templates, are set to the keys and values of
~table~.  If ~key~ is |nil|, then the hashtable's keys will be ignored;
similarly for ~value~.
The order in which 
elements of ~table~ will be retrieved is unpredictable.

\clause{|for| ~var~ |in-package| ~package~ \opt\ |external-only| ~ext~}
\clindexx{for}{in-package}
Iterates over all the symbols in ~package,~ or over only the
external symbols if ~ext~ is specified and non-|nil|.  ~ext~ is not
evaluated.  The same symbol may appear more than once.

\clause{|for| ~var~ |in-packages| ~packages~ \opt\ |having-access| ~symbol-types~}
\clindexx{for}{in-packages}
Iterates over all the symbols from the list of packages denoted by the
descriptor ~packages~ and having accessibility (or visibility) given by
~symbol-types~.  This defaults to the list |(:external :internal :inherited)|
and is not evaluated.  ~var~ must be a list of up to three variables: in each
iteration, these will be set to a symbol, its access-type and package (as per
|with-package-iterator| in ANSI-CL).  The same symbol may appear more than
once.

\end{clauses}

\begin{clauses}

\clause{|for| ~var~ |in-file| ~name~ \opt\ |using| ~reader~}
\clindexx{for}{in-file}
Opens the file ~name~ (which may be a string or pathname) for input,
and iterates 
over its contents. ~reader~ defaults to |read|, so by default ~var~
will be bound to the successive forms in the file.  The \iter\ body is
wrapped in an unwind-protect to ensure that the file is closed no
matter how the \iter\ is exited.

\clause{|for| ~var~ |in-stream| ~stream~ \opt\ |using| ~reader~}
\clindexx{for}{in-stream}
Like |for\dots in-file|, except that ~stream~ should be an existing
stream object that supports input operations.

\end{clauses}

\subsubsection{Generalized Drivers}

These are primarily useful for writing drivers that can also be used
as generators (see section \ref{generators}, below).
 
\begin{clauses}

\clause{|for| ~var~ |next| ~expr~}
\clindexx{for}{next}

~var~ is set to ~expr~ each time through the loop.  Destructuring is
performed.  When the clause is used as a generator, ~expr~ is the code
that is executed when |(next ~var~)| is 
encountered (see section \ref{generators}, below).
~expr~ should compute the first value for ~var~, as well
as all subsequent values, and is responsible for terminating the loop.
For compatibility with future versions of \iter, this termination
should be done with |terminate|\clindex{terminate}, which can be
considered a synonym for |finish| (see section \ref{control-flow}).

\cpar As an example, the following clauses are equivalent to |(for
i from 1 to 10)|:
\begin{program}
(initially (setq i 0))
(for i next (if (> i 10) (terminate) (incf i)))
\end{program}

\clause{|for| ~var~ |do-next| ~form~}
\clindexx{for}{do-next}
~form~ is evaluated each time through the loop.  Its value is ~not~
set to ~var~; that is ~form's~ job. ~var~ is only present so that
\iter\ knows it is a driver variable.  \linebreak |(for ~var~ next
~expr~)| is equivalent to |(for ~var~ do-next (dsetq ~var~ ~expr~))|.
(See section \ref{destructuring} for an explanation of |dsetq|.) 

\end{clauses}

\subsubsection{Generators}
\label{generators}

In all of the above clauses, the driver variable is updated on each
iteration.
Sometimes it is desirable to have greater control over updating.  
For instance, consider the problem of associating numbers, in
increasing order and with no gaps, with the
\nonnil\ elements of a list.  One obvious first pass at writing this is:

\begin{program}
(iter (for el in list)
      (for i upfrom 1)
      (if el (collect (cons el i))))
\end{program}
But on the list |(a b nil c)| this produces |((a . 1) (b . 2) (c .
4))| instead of the desired |((a . 1) (b . 2) (c . 3))|.  The problem
is that |i| is incremented each time through the loop, even when |el|
is |nil|.  

The problem could be solved elegantly if we could step |i| only when
we wished to.  This
can be accomplished for any \iter\ driver by writing |generate| (or
its synonym |generating|)
instead of |for|.  Doing so produces a ~generator~---a driver whose
values are yielded explicitly.  To obtain the next value of a
generator variable ~v~, write \linebreak |(next ~v~)|.  The value of a |next|
form is the next value of ~v~, as determined by its associated driver
clause.  |next| also has the side-effect of updating ~v~ to that
value.  If there is no next value, |next| will terminate the loop,
just as with a normal driver.


Using generators, we can now write our example like this:
\begin{program}
(iter (for el in list)
      (generate i upfrom 1)
      (if el (collect (cons el (next i)))))
\end{program}
Now |i| is updated only when |(next i)| is executed, and this occurs
only when |el| is \nonnil.

To better understand the relationship between ordinary drivers and
generators, observe that we can rewrite an ordinary driver using its
generator form immediately followed by |next|, as this example shows:
\begin{program}
(iter (generating i from 1 to 10)
      (next i)
      ...)
\end{program}
Provided that the loop body contains no |(next i)| forms, this will
behave just as if we had written |(for i from 1 to 10)|.

We can still refer to a driver variable ~v~ without using |next|; in
this case, its value is that given to it by the last evaluation of
|(next ~v~)|.  Before |(next ~v~)| has been called the first time, the
value of ~v~ is undefined.  

This semantics is more flexible than
one in which ~v~ begins the loop bound to its first value and calls
of |next| supply subsequent values, because it means the loop will not
terminate too soon if the generator's sequence is empty.  For
instance, consider the following code, which tags \nonnil\ elements of
a list using a list of tags, and also counts the null elements.
(We assume there are at least as many tags as \nonnil\ elements.)
\begin{program}
(let* ((counter 0)
       (tagged-list (iter (for el in list)
                          (generating tag in tag-list) 
                          (if (null el)
                              (incf counter)
                              (collect (cons el (next tag)))))))
  ...)
\end{program}
It may be that there are just as many tags as non-null elements of
|list|.  If all the elements of |list| are null, we still want the
counting to proceed, even though |tag-list| is \nil.  If |tag| had to be
assigned its first value before the loop begins, we would have had to
terminate the loop before the first iteration, since when |tag-list|
is \nil, |tag| 
has no first value.  With the existing semantics, however, |(next
tag)| will never execute, so the iteration will cover all the elements of
|list|. 
        
When the ``variable'' of a driver clause is actually a destructuring
template containing several variables, all the variables are eligible
for use with |next|.  As before, |(next ~v~)| evaluates to ~v's~ next
value; but the effect is to update all of the template's variables.
For instance, the following code will return the list |(a 2 c)|.
\begin{program}
(iter (generating (key . item) in '((a . 1) (b . 2) (c . 3)))
      (collect (next key))
      (collect (next item)))
\end{program}

Only driver clauses with variables can be made into generators.  This
includes all clauses mentioned so far except for |repeat|.  It does
~not~ include |for\dots previous|, |for\dots =|,
|for\dots initially\dots then| or |for\dots first\dots then| (see
below). 

\subsubsection{Previous Values of Driver Variables}

Often one would like to access the value of a variable on a previous
iteration.  \iter\ provides a special clause for accomplishing this.

\begin{clauses}

\clause{|for| ~pvar~ |previous| ~var~ \opt\ |initially| ~init~
|back| ~n~}
\clindexx{for}{previous}
Sets ~pvar~ to the previous value of ~var,~ which should be a driver
variable, a variable from another |for\dots previous| clause, or a
variable established by a |for\dots =|, 
|for\dots initially\dots then| or |for\dots first\dots then| clause
(see section \ref{setting}).
Initially, ~pvar~ is given the value ~init~ (which defaults to \nil).  
The ~init~ expression will be moved outside the loop body, so it
should not depend on anything computed within the loop.
~pvar~ retains the value of ~init~ until ~var~ is set to its second
value, at which point ~pvar~ is set to ~var's~ first value; and so on.  

\cpar The
argument ~n~ to |back| must be a constant, positive integer, and
defaults to 1.  It determines how many iterations back ~pvar~ should
track ~var~.  For example, when ~n~ is 2, then ~pvar~ will be assigned
~var's~ first value when ~var~ is set to its third value.

\cpar A |for\dots previous| clause may occur after or before its
associated driver clause. |for\dots previous| works with generators as
well as ordinary drivers.

\pagebreak[3]

\cpar Example:
\begin{program}
(iter (for el in '(1 2 3 4))
      (for p-el previous el)
      (for pp-el previous p-el initially 0)
      (collect pp-el))
\end{program}
This evaluates to |(0 0 1 2)|.  It could have been written more
economically as 
\begin{program}
(iter (for el in '(1 2 3 4))
      (for pp-el previous el back 2 initially 0)
      (collect pp-el))
\end{program}


\end{clauses}

\subsection{Variable Binding and Setting}
\label{setting}

Several clauses exist for establishing new variable bindings or for
setting variables in the loop.  They all support destructuring.

\begin{clauses}

\clausex{with}{~var~ \opt\ |=| ~value~}
Causes ~var~ to be bound to value before the loop body is entered.
If ~value~ is not supplied, ~var~ assumes a default
binding, which will be 
\nil\ in the absence of declarations.  Also, if ~value~ is not
supplied, no destructuring is performed; instead, ~var~ may be a list of
symbols, all of which are given default 
bindings.  If ~value~ is supplied, ~var~ is bound to it, with
destructuring.  

\cpar Because |with|
creates bindings whose scope includes the entire \iter\ form, it is
good style to put all |with| clauses at the beginning.

\cpar Successive occurrences of |with| result in sequential
bindings (as with 
|let*|).  There is no way to obtain parallel bindings; see
section \ref{bindings} for a rationale.  


\clause{|for| ~var~ |=| ~expr~}
\clindexx{for}{=}
On each iteration, ~expr~ is evaluated and ~var~ is set
to its value. 

\cpar This clause may appear to do the same thing as |for\dots next|.
In fact, they are quite different.  |for\dots =| provides only three
services: it sets up a binding for ~var~, sets it to ~expr~ on each
iteration, and makes it possible to use |for\dots previous| with
~var~.  |for\dots next| provides these services in addition to the
ability to turn the driver into a generator.
%Also, the code which sets ~var~ appears in the loop body in the same
%place as the |for\dots =| clause; the code for |for\dots next| appears
%at the top of the loop, as with other drivers (except when being used
%as a generator).

\clause{|for| ~var~ |initially| ~init-expr~ |then| ~then-expr~}
\clindexxx{for}{initially}{then}
Before the loop begins, ~var~ is set to ~init-expr;~ on all
iterations after the first it is set to ~then-expr.~  This clause
must occur at top-level.  ~init-expr~ will be moved outside the loop
body and ~then-expr~ will be moved to the end of the loop body, so
they are subject to code motion problems (see section
\ref{code-movement}). 

\cpar This clause may appear to be similar to |for\dots next|, but in
fact they differ significantly.  |for\dots initially\dots then| is
typically used to give ~var~ its first value before the loop begins,
and subsequent values on following iterations.  This is incompatible
with generators, whose first value and subsequent values must all be
computed by |(next ~var~)|.  Also, the update of ~var~ in 
|for\dots initially\dots then| does not occur at the location of the
clause. 
Use |for\dots initially\dots then| for
one-shot computations where its idiom is more convenient, but use
|for\dots next| for extending \iter\ with new drivers (see section
\ref{extend}).

\clause{|for| ~var~ |first| ~first-expr~ |then| ~then-expr~}
\clindexxx{for}{first}{then}
The first time through the loop, ~var~ is set to ~first-expr;~ on
subsequent iterations, it is set to ~then-expr.~  This differs from
|for\dots initially| in that ~var~ is set to ~first-expr~
inside the loop body, 
so ~first-expr~ may depend on the results of other clauses.  For
instance, 
\begin{program}
(iter (for num in list)
      (for i first num then (1+ i))
      ...)
\end{program}
will set |i| to the first element of |list| on the first
iteration, whereas 
\begin{program}
(iter (for num in list)
      (for i initially num then (1+ i))
      ...)
\end{program}
is probably erroneous; |i| will be bound to |num|'s
default binding (usually \nil) for the first iteration.

\end{clauses}

\begin{note}{Compatibility note}
|loop|'s |for\dots =| works like \iter's, but |loop| used the syntax
|for\dots =\dots then| to mean |for\dots initially\dots then|.  It was
felt that these two operations were sufficiently different to warrant
different keywords.

Also, the |for| in the above three clauses is misleading,
since none is true driver (e.g. none has a corresponding |generate|
form).  |setting| would have been a better choice, but |for| was used
to retain some compatibility with |loop|.  
\end{note}


%%%%%%%%%%%%%%%%%%%%%%%%%%%%%%%%
\subsection{Gathering Clauses}
Many of \iter's clauses accumulate values into a variable, or set a
variable under certain conditions.  At the end of the loop, this
variable contains the desired result.  All these clauses have an
optional |into| keyword, whose argument should be a symbol.  If the
|into| keyword is not supplied, the accumulation variable will be
internally generated and its value will be returned at the end of the
loop; if a variable is specified, that variable is used for the
accumulation, and is not returned as a result---it is up to the user
to return it explicitly, in the loop's epilogue code (see |finally|).
It is safe to examine the accumulation variable during the loop, but
it should not be modified.

These clauses all begin with a verb.  When the verb does
not conflict with an existing Common Lisp function, then it may be
used in either its infinitival or present-participle form (e.g. |sum|,
|summing|).  However, when there is a conflict with Common Lisp, only
the present-participle form may be used (e.g. |unioning|).  This is to
prevent \iter\ clauses from clashing with Common Lisp
functions. 

%although these clauses are described as ``producing a value,'' it is a
%mistake to think of the lisp list representing the clause as a
%value-producing form in the usual way.  clauses may legally be written
%where a value is expected, e.g. |(setq x (sum i))|, but the
%lisp value of a clause in such a context is undefined.

\subsubsection{Reductions}

~Reduction~ is an extremely common iteration pattern in which
the results of successive applications of a
binary operation are accumulated.
For example, a loop that computes the sum of the
elements of a list is performing a reduction with the addition
operation.  This could be written in Common Lisp as  |(reduce \#'+
list)| or with \iter\ as
\begin{program}
(iter (for el in list)
      (sum el))
\end{program}

\begin{clauses}

\clausex{sum}{~expr~ \opt\ |into| ~var~}
Each time through the loop, ~expr~ is evaluated and added to a
variable, which is bound initially to zero.  If ~expr~ has a type,
it is ~not~ used as the type of the sum variable, which is always
|number|.  To get the result variable to be of a more specific
type, use an explicit variable, as in
\begin{program}
(iter (for el in number-list)
      (sum el into x)
      (declare (fixnum x))
      (finally (return x)))
\end{program}

\clausex{multiply}{~expr~ \opt\ |into| ~var~}
Like |sum|, but the initial value of the result variable is $1$,
and the variable is updated by multiplying ~expr~ into it.

\clausex{counting}{~expr~ \opt\ |into| ~var~}
~expr~ is evaluated on each iteration.  If it is \nonnil, the
accumulation variable, initially zero, is incremented.

\startitem
\Clausex{maximize}{~expr~ \opt\ |into| ~var~}
\Clausex{minimize}{~expr~ \opt\ |into| ~var~}
\finishitem
~expr~ is evaluated on each iteration and its extremum (maximum or
minimum) is stored in the accumulation variable.  If ~expr~ is never
evaluated, then the result is \nil\ (if the accumulation variable is
untyped) or $0$ (if it has a numeric type).

\clausex{reducing}{~expr~ |by| ~func~ \opt\ |initial-value| ~init-val~ |into| ~var~}
This is a general way to perform reductions. ~func~ should be a
function of two arguments, the first of which will be the value
computed so far and
the second of which will be the value of ~expr~.  It should return the
new value.
|reducing| is roughly equivalent to the
Common Lisp |(reduce ~func~ ~sequence~ :key ~expr-function~)|, where
~expr-function~ is used to derive values from the successive elements
of ~sequence~.

\cpar If the |reducing| clause is never executed, the result is
undefined.

\cpar It is not necessary to provide an
initial value, but better code can be generated if one is supplied.
Regardless of its location in the \iter\ body, ~init-val~ will be
evaluated before the loop is entered, so it should not depend on any
value computed inside the \iter\ form.  

%\cpar if a ~var~ is not specified, you can get \iter\ to declare the
%type of the internal variable by putting a |the| expression around
%~func.~  see section \ref{types}.

\end{clauses}

%\begin{note}{implementation note}
%in principle, |maximize| and |minimize| can be thought of
%as reductions where the initial value is the smallest (or largest)
%value that the accumulation variable can assume.  because lisp's
%bignums can represent arbitrary integers, these clauses cannot be
%implemented as reductions in general.  if, however, the type of 
%~expr~ or ~var~ can be determined to be a fixnum
%or a float, \iter\ will implement the clause as a true reduction,
%using one of the constants |most-negative-fixnum|, |%most-positive-fixnum|, 
%|most-negative-short-float|, etc. as appropriate.
%\end{note}

\subsubsection{Accumulations}

All the predefined accumulation clauses add values to a sequence.  If
the sequence is a list, they all 
have the property that the partial list is kept in the correct order
and available for inspection at any point in the loop.

\begin{clauses}

\clausex{collect}{~expr~ \opt\ |into| ~var~ |at| ~place~ |result-type| ~type~}
Produces a sequence of the values of ~expr~ on each iteration. ~place~
indicates where the next value of ~expr~ is added to the list and may
be one of the symbols |start|, |beginning| (a synonym for |start|) or
|end|.  The symbol may be quoted, but need not be.  The default is
|end|.  For example,
\begin{program}
(iter (for i from 1 to 5)
      (collect i))
\end{program}
produces |(1 2 3 4 5)|, whereas
\begin{program}
(iter (for i from 1 to 5)
      (collect i at beginning))
\end{program}
produces |(5 4 3 2 1)| (and is likely to be faster in most Common Lisp
implementations).

\cpar If ~type~ is provided, it should be a subtype of |sequence|.
The default is |list|.  Specifying a type other than |list| will
result in |collect| returning a sequence of that type.  ~However,~ the
type of the sequence being constructed when inside the loop body is
undefined when a non-|list| type is specified.  (As with ~place~,
quoting ~type~ is optional.)

\clausex{adjoining}{~expr~ \opt\ |into| ~var~ |test| ~test~ |at| ~place~ |result-type| ~type~}
Like |collect|, but only adds the value of ~expr~ if it is not
already present.  ~test,~ which defaults to |\#'eql|, is
the test to be used with |member|.

\startitem
\Clausex{appending}{~expr~ \opt\ |into| ~var~ |at| ~place~}
\Clausex{nconcing}{\ ~expr~ \opt\ |into| ~var~ |at| ~place~}
\Clausex{unioning}{\ ~expr~ \opt\ |into| ~var~ |test| ~test~ |at| ~place~}
\Clausex{nunioning}{~expr~ \opt\ |into| ~var~ |test| ~test~ |at| ~place~}
\finishitem
These are like |collect|, but behave like the Common Lisp functions
|append|, |nconc|, |union| or |nunion|.  
As in Common Lisp, they work only on lists.  Also as in Common Lisp,
|unioning| and |nunioning| assume that the value of ~expr~ contains no
duplicates. 

\clausex{accumulate}{~expr~ |by| ~func~ \opt\ |initial-value| ~init-val~ |into| ~var~}
This is a general-purpose accumulation clause. ~func~ should be a
function of two arguments, the value of ~expr~ and the value
accumulated so far in the iteration, and it should return the updated
value.  If no initial value is supplied, \nil\ is used.  

%\cpar If a ~var~ is not specified, you can get \iter\ to declare the
%type of the internal variable by putting a |the| expression around
%~func.~  see section \ref{types}.

\cpar The differences between |accumulate| and |reducing| are slight.
One difference is that the functions take their arguments in a
different order.  Another is that in the absence of ~init-val~,
|accumulate| will use \nil, whereas |reducing| will generate different
code that avoids any dependence on the initial value.
The reason for having both clauses is that one usually
thinks of reductions (like |sum|) and accumulations (like |collect|)
as different beasts.

\end{clauses}


\subsubsection{Finders}

A ~finder~ is a clause whose value is an expression that meets some
condition.

\begin{clauses}

\clause{|finding| ~expr~ |such-that| ~test~ \opt\ |into| ~var~ |on-failure| ~failure-value~}
\clindexx{finding}{such-that}
If ~test~ (which is an expression) ever evaluates to \nonnil, the loop
is terminated, the 
epilogue code is run and the value of ~expr~ is returned.  Otherwise,
\nil\ (or ~failure-value,~ if provided) is returned.  If ~var~ is
provided, it will have either the \nonnil\ value of ~expr~ or
~failure-value~ when the epilogue code is run.

\cpar As a special case, if the ~test~ expression is a sharp-quoted
function, then it is applied to ~expr~ instead of being simply
evaluated.  E.g. |(finding x such-that \#'evenp)| is equivalent to
|(finding x such-that (evenp x))|.  

%\cpar although ~test~ need have nothing to do with ~%expr~ as in
%|(finding j such-that (> i 3))|, it usually 
%will: |(finding (length el) such-that (oddp (length el)))|.  to
%avoid performing the |length| computation twice, you could write
%|(finding (length el) such-that \#'oddp)| or |(finding (length
%el) such-that 'oddp)|; for these cases, \iter\ generates code that
%executes ~expr~ only once.  the code for |\#'oddp|
%is slightly different from that for {\lisp 'oddp}; see the discussion
%under {\lisp for\dots in} and {\lisp for\dots on}.

\cpar |On-failure| is a misnomer. Because it is always evaluated, it behaves
more like the default third argument to the |gethash| function. As a result,
|on-failure (error "Not found")| makes no sense. Instead, the clauses |leave|
or |thereis| can be used in conjunction with |finally| as follows:
\begin{program}
(iter (for x in '(1 2 3))
      (if (evenp x) (leave x))
      (finally (error "not found")))
\end{program}

\cpar This clause may appear multiple times when all defaults are
identical. It can also be used together with either |always|/|never| or
|thereis| if their defaults match. More specifically, |on-failure nil| is
compatible with |thereis|, while |on-failure t| is compatible with |always|
and |never| clauses.
\begin{program}
(iter (for i in '(7 -4 2 -3))
      (if (plusp i)
	  (finding    i  such-that (evenp i))
          (finding (- i) such-that (oddp i))))
\end{program}

\startitem
\Clause{|finding| ~expr~ |maximizing| ~m-expr~ \opt\ |into| ~var~}
\clindexx{finding}{maximizing}
\Clause{|finding| ~expr~ |minimizing| ~m-expr~ \opt\ |into| ~var~}
\clindexx{finding}{minimizing}
\finishitem
Computes the extremum (maximum or minimum) value of ~m-expr~ over all
iterations, and returns the value of ~expr~ corresponding to the
extremum.  ~expr~ is evaluated inside the loop at the time the new
extremum is established.  If ~m-expr~ is never evaluated (due to, for
example, being embedded in a conditional clause), then the returned
value depends on the type, if any, of ~expr~ (or ~var,~ if
one is supplied).  If there is no type, the returned
value will be \nil; if the type is numeric, the returned value will be
zero.

\cpar For these two clauses, ~var~ may be a list of two
symbols; in that case, the first is used to record ~expr~ and
the second, ~m-expr.~  

\cpar As with |finding\dots such-that|, if ~m-expr~ is a sharp-quoted
function, then it is called on ~expr~ instead of being evaluated.

\end{clauses}

\subsubsection{Boolean Tests}

\begin{clauses}

\clausex{first-iteration-p}{}
Returns |t| in the first cycle of the loop, otherwise \nil.

\clausex{first-time-p}{}
Returns |t| the first time the expression is evaluated, and then \nil\ forever.
This clause comes handy when printing (optional) elements separated
by a comma:

\begin{program}
(iter (for el in '(nil 1 2 nil 3))
      (when el
        (unless (first-time-p)
          (princ ", "))
        (princ el)))
\end{program}
produces |"1, 2, 3"|.

\end{clauses}

\subsubsection{Aggregated Boolean Tests}

\begin{clauses}

\clausex{always}{~expr~}
If ~expr~ ever evaluates to
\nil, then \nil\ is immediately returned; the epilogue code is not
executed.  If ~expr~ never evaluates to \nil, the epilogue code
is executed and the last value of ~expr~ (or |t| if ~expr~ was never
evaluated) is returned (whereas |loop| would constantly return |t|).

% mention last evaluated clause when multiple always clauses?

\clausex{never}{~expr~}
Like |(always (not ~expr~))|, except it does not influence the last
value returned by a possible other |always| clause. That is,
\begin{program}
(iter (repeat 2)
      (always 2)
      (never nil)) \yields 2 ; not t
\end{program}

\clausex{thereis}{~expr~}
If ~expr~ is ever \nonnil,
its value is immediately returned without running epilogue code.
Otherwise, the epilogue code is performed and \nil\ is returned.

This clause cannot be used together with |always| or |never|, because their
defaults are opposed (similarly, |(loop always 3 thereis nil)| refuses to
compile in some implementations of |loop|).

\end{clauses}


\subsection{Control Flow}
\label{control-flow}
Several clauses can be used to alter the usual flow of control in a loop.

Note: the clauses of this and subsequent sections don't adhere to \iter's
usual syntax, but instead use standard Common Lisp syntax.  Hence the
format for describing syntax subsequently is
like the standard format used in the Common Lisp manual, not like the
descriptions of clauses above.

\begin{clauses}

\clausex{finish}{}
Stops the loop and runs the epilogue code.

%for example:
%
%\begin{program}
%(iter (with answer = nil)
%         (initially (make-a-mess))
%         (for i from 1 to 10)
%         (when (correct? i)
%           (setq answer i)
%           (finish))
%         (finally (cleanup)))
%\end{program}
%
%this code will execute |cleanup| whether or not the test |(correct?
%i)| ever succeeds. 
%the (more elegant) formulation,
%\begin{program}
%(iter (initially (make-a-mess))
%         (for i from 1 to 10)
%         (finding i such-that (correct? i))
%         (finally (cleanup)))
%\end{program}
%would not execute |cleanup| if |(correct? i)| succeeded; it
%would do an immediate return.

\clausex{leave}{\opt\ ~value~}
Immediately returns ~value~ (default \nil) from the current \iter\
form, skipping the epilogue code.  Equivalent to using |return-from|.

\clausex{next-iteration}{}
Skips the remainder of the loop body and begins the next iteration of
the loop.

\clausex{while}{~expr~}
If ~expr~ ever evaluates to \nil, the loop is terminated and the
epilogue code executed.  Equivalent to |(if (not ~expr~) (finish))|.

\clausex{until}{~expr~}
Equivalent to |(if ~expr~ (finish))|.

\clausex{if-first-time}{~then~ \opt\ ~else~}
If this clause is being executed for the first time in this invocation
of the \iter\ form, then the ~then~ code is evaluated; otherwise the
~else~ code is evaluated.

\cpar |(for ~var~ first ~expr1~ then ~expr2~)| is almost equivalent to
\begin{program}
(if-first-time (dsetq ~var~ ~expr1~)
               (dsetq ~var~ ~expr2~))
\end{program}
The only difference is that the |for| version makes ~var~ available
for use with |for\dots previous|.

\end{clauses}

\subsection{Code Placement}
When fine control is desired over where code appears in a loop
generated by \iter, the following special clauses may be useful.
They are all subject to code-motion problems (see section
\ref{code-movement}).  

\begin{clauses}

\clausex{initially}{|\&rest| ~forms~}
The lisp ~forms~ are placed in the prologue section of the loop, where
they are executed once, before the loop body is entered.

\clausex{after-each}{|\&rest| ~forms~}
The ~forms~ are placed at the end of the loop body, where they
are executed after each iteration.  Unlike the other clauses in this
section, ~forms~ may contain \iter\ clauses.

\clausex{else}{|\&rest| ~forms~}
The lisp ~forms~ are placed in the epilogue section of the loop, where they
are executed if this |else| clause is never met during execution of the
loop and the loop terminates normally.

\clausex{finally}{|\&rest| ~forms~}
The lisp ~forms~ are placed in the epilogue section of the loop, where
they are executed after the loop has terminated normally.

\clausex{finally-protected}{|\&rest| ~forms~}
The lisp ~forms~ are placed in the second form of an unwind-protect
outside the loop.  They are always executed after the loop has
terminated, regardless of how the termination occurred.

\end{clauses}


%%%%%%%%%%%%%%%%%%%%%%%%%%%%%%%%%%%%%%%%%%%%%%%%%%%%%%%%%%%%%%%%
\section{Other Features}

\subsection{Multiple Accumulations}
\label{multiple}

\begin{sloppypar}
It is permitted to have more than one clause accumulate into the same
variable, as in the following:
\begin{program}
(iter (for i from 1 to 10)
      (collect i into nums)
      (collect (sqrt i) into nums)
      (finally (return nums)))
\end{program}
Clauses can only accumulate into the same variable if they are
compatible.  |collect|, |adjoining|, |appending|, |nconcing|,
|unioning| and |nunioning| are compatible with each other; |sum|,
|multiply| and |counting| are compatible; |always| and |never| are compatible;
|finding| \dots |such-that| is compatible with either |thereis| or |always|
and |never| when their defaults match; and |maximize| and |minimize| clauses
are compatible only with other |maximize| and |minimize| clauses,
respectively.

%note that the same variable ~cannot~ be both an accumulation
%variable and an ordinary variable; there can be only one variable with
%a given name within an \iter\ form. 
\end{sloppypar}

\subsection{Named Blocks}

Like Common Lisp |block|s, \iter\ forms can be given names.  The
name should be a single symbol, and it must be the first form in the
\iter.  The generated code behaves exactly like a named block; in
particular, |(return-from ~name~)| can be used to exit it:
\begin{program}
(iter fred
      (for i from 1 to 10)
      (iter barney
            (for j from i to 10)
            (if (> (* i j) 17)
                (return-from fred j))))
\end{program}
An \iter\ form that is not given a name is implicitly named \nil.

Sometimes one would like to write an expression in an inner \iter\ form,
but have it processed by an outer \iter\ form.  This is possible with
the |in| clause.  

\begin{clauses}

\clausex{in}{~name~ |\&rest| ~forms~}
Evaluates ~forms~ as if they were part of the \iter\ form named
~name~.  In other words, \iter\ clauses are processed by the \iter\
form named ~name,~ and not by any \iter\ forms that occur inside ~name.~

\cpar As an example, consider the problem of collecting a list of the
elements in a two-dimensional array.  The naive solution,
\begin{program}
(iter (for i below (array-dimension ar 0))
      (iter (for j below (array-dimension ar 1))
            (collect (aref ar i j))))
\end{program}
\noindent is wrong because the list created by the inner \iter\ is simply
ignored by the outer one.  But using |in| we can write:
\begin{program}
(iter outer (for i below (array-dimension ar 0))
      (iter (for j below (array-dimension ar 1))
            (in outer (collect (aref ar i j)))))
\end{program}
\noindent which has the desired result.

\end{clauses}

\subsection{Destructuring}
\label{destructuring}

In many places within \iter\ clauses where a variable is expected, a
list can be written instead.  In these cases, the value to be assigned
is ~destructured~ according to the pattern
described by the list.  As a simple example, the clause
\begin{program}
(for (key . item) in alist)
\end{program}
\noindent will result in |key| being set to the |car| of
each element in |alist|, and |item| being set to the |cdr|.  The
pattern list may be nested to arbitrary depth, and (as the example
shows) need not be terminated with \nil; the only requirement is that
each leaf be a bindable symbol (or \nil, in which case no binding is
generated for that piece of the structure).

Sometimes, you might like to do the equivalent of a
|multiple-value-setq| in a clause.  This 
``multiple-value destructuring'' can be expressed by writing 
\linebreak |(values $pat_1$ $pat_2 \ldots$)| for a destructuring
pattern, as in 
\begin{program}
(for (values (a . b) c d) = (three-valued-function ...))
\end{program}
\begin{sloppypar}
Note that the $pat_i$ can themselves be destructuring patterns (though
not multiple-value destructuring patterns).  You can't do multiple-value
destructuring in a |with| clause; instead wrap the whole \iter\
form in a |multiple-value-bind|.
\end{sloppypar}

\begin{note}{Rationale}
There are subtle interactions between variable declarations and
evaluation order that make the correct implementation of
multiple-value destructuring in a |with| somewhat tricky.
\end{note}

The destructuring feature of \iter\ is available as a separate
mechanism, using the |dsetq| macro:

\begin{clauses}
\defmacro{dsetq}{template expr}
Performs destructuring of ~expr~ using ~template~.   May be used
outside of an \iter\ form. Yields the primary value of ~expr~.

\end{clauses}

\subsection{On-line Help}

\begin{sloppypar}
There is a limited facility for on-line help, in the form of the 
|display-iterate-clauses| function.
\end{sloppypar}

\begin{clauses}

\defun{display-iterate-clauses}{\opt\ clause-spec}
Displays a list of \iter\ clauses.  If ~clause-spec~ is not
provided, all clauses are shown; if it is a symbol, all clauses
beginning with that symbol are shown; and if it is a list of symbols,
all clauses for which ~clause-spec~ is a prefix are shown.

\end{clauses}

\subsection{Parallel Binding and Stepping}
\label{bindings}

The parallel binding and stepping of variables is a feature that
\iter\ does ~not~ have.  This section attempts to provide a rationale.

We say that two variables are bound ~in parallel~ if neither
binding shadows the other.  This is the usual semantics of |let|
(as opposed to |let*|).  Similarly, we can say that iteration
variables are stepped in parallel if neither variable is updated
before the other, conceptually speaking; in other words, if the code
to update each variable can reference the old values of both variables.

|loop| allows parallel binding of variables and parallel stepping of
driver variables.  My view is that if you are depending on the 
serial/parallel distinction, you are doing something obscure.  If you
need to bind 
variables in parallel using |with|, then you must be using a
variable name that shadows a name in the existing lexical environment.
Don't do that.  The most common use for parallel stepping is to track
the values of variables on the previous iteration, but in fact this
does not require parallel stepping at all; the following will work:
\begin{program}
(iter (for current in list)
      (for prev previous current)
      ...)
\end{program}


%%%%%%%%%%%%%%%%%%%%%%%%%%%%%%%%%%%%%%%%%%%%%%%%%%%%%%%%%%%%%%%%
\section{Types and Declarations}
\label{types}

\subsection{Discussion}

Sometimes efficiency dictates that the types of variables be declared.
This type information needs to be communicated to \iter\ so it can
bind variables to appropriate values.  Furthermore, \iter\ must often
generate internal variables invisible to the user; there needs to be a
way for these to be declared.

As an example, consider this code, which will return the number of
odd elements in |number-list|:
\begin{program}
(iter (for el in number-list)
      (count (oddp el)))
\end{program}
In processing this form,
\iter\ will create an internal variable, let us call it |list17|, to
hold the successive |cdr|s of |number-list|, and  
will bind the variable to |number-list|.  It will also generate a
default binding for |el|; only inside the body of the loop will |el|
be set to the |car| of |list17|.  Finally, \iter\ will generate a
variable, call it |result|, to hold the result of the count, and will
bind it to zero.

When dealing with type declarations, \iter\ observes one simple rule:
~it will never generate a declaration unless requested to do so.~  The
reason is that such declarations might mask errors in compiled code by
avoiding error-checks; the resulting problems would be doubly hard to
track down because the declarations would be hidden from the
programmer.  Of course, a compiler might omit error-checks even in the
absence of declarations, though this behavior can usually be avoided,
e.g. by saying |(declaim (optimize (safety 3)))|.

So, the above \iter\ form will generate code with no declarations.
But say we wish to declare the types of |el| and the internal
variables |list17| and |result|.  How is this done?

Declaring the type of |el| is easy, since the programmer knows
the variable's name:
\begin{program}
(iter (for el in number-list)
      (declare (fixnum el))
      (counting (oddp el)))
\end{program}
\iter\ can read variable type declarations like this one.  Before
processing any clauses, it scans the entire top-level form for type
declarations 
and records the types, so that variable bindings can be performed
correctly.  In this case, |el| will be bound to zero
instead of \nil.  Also, \iter\ collects all the top-level declarations
and puts them at the begining of the generated code, so it is not
necessary to place all declarations at the beginning of an \iter\
form; instead, they can be written near the variables whose types they
declare.  

Since \iter\ is not part of the compiler, it will not know
about declarations that occur outside an \iter\ form; these
declarations must be repeated inside the form.

Here is another way we could have declared the type of |el|:
\begin{program}
(iter (for (the fixnum el) in number-list)
      (counting (oddp el)))
\end{program}
\iter\ extends the Common Lisp |the|\lispindex{the}
form to apply to variables as well as value-producing forms; anywhere
a variable is allowed---in a |with| clause, as the iteration
variable in a driver clause, as the |into| argument of an
accumulation clause, even inside a destructuring template---you can
write |(the ~type~ ~symbol~)| instead.

There is one crucial difference between using a |the| form and
actually declaring the variable: explicit declarations are always
placed in the generated code, but type information from a |the|
form is not turned into an actual declaration unless you tell \iter\
to do so using |iterate:declare-variables|.  See below.

\begin{sloppypar}
Declaring the types of internal variables is harder than declaring the
types of explicitly mentioned variables, since their names
are unknown.  You do it by declaring |iterate:declare-variables|
somewhere inside the top level of the \iter\ form.  (This will also
generate declarations for variables declared using |the|.)
\iter\ does not provide much selectivity here: it's all or none. 
And unfortunately, since \iter\ is not privy to compiler information
but instead reads declarations itself, it will not hear if you
|(declaim (iterate:declare-variables))|.  Instead, set the variable 
|iterate::*always-declare-variables*| to |t| at
compile-time, using |eval-when|.
\end{sloppypar}

To determine the appropriate types for internal variables, \iter\ uses
three sources of information:

\begin{itemize}

\item Often, the particular clause dictates a certain type for a
variable; \iter\ will use this information when available.  In the
current example, the variable |list17| will be given the type
|list|, since that is the only type that makes sense; and the
variable |result| will be given the type |fixnum|, on the
assumption that you will not be counting high enough to need bignums.
You can override this assumption only by using and explicitly declaring a
variable: 
\begin{verbatim}
(iter (declare (iterate:declare-variables))
      (for el in number-list)
      (count (oddp el) into my-result)
      (declare (integer my-result))
      (finally (return my-result)))
\end{verbatim}

\begin{sloppypar}
Other examples of the type assumptions that \iter\ makes are: type
|list| for |into| variables of collection clauses; type |list| for
expressions that are to be destructured; type |vector| for the
variable holding the vector in a |for\dots in-vector| clause, and
similarly for |string| and the |for\dots in-string| clause;
and the implementation-dependent type
|(type-of array-dimension-limit)| for the index and limit
variables generated by sequence iteration drivers like |for\dots
in-vector| and |for\dots in-string| (but not |for\dots in-sequence|,
because it may be used to iterate over a list). 
\end{sloppypar}

\item Sometimes, \iter\ will examine expressions and try to determine
their types in a simple-minded way.  If the expression is
self-evaluating (like a number, for instance), \iter\ knows that the
expression's type is the same as the type of the value it denotes, so
it can use that type.  If the expression is of the form |(the ~type~
~expr~)|, \iter\ is smart enough to extract ~type~ and use it.
However, the current version of \iter\ does not  
examine declarations of function result types or do any type
inference.  It will not determine, for 
example, that the type of |(+ 3 4)| is |fixnum|, or even
|number|.

\item In some cases, the type of an internal variable should match the
type of some other variable.  For instance, \iter\ generates an
internal variable for |(f x)| in the
clause |(for i from 1 to (f x))|, and in the absence of other
information will give it the same type as |i|.  If, however, the
expression had been written |(the fixnum (f x))|, then \iter\
would have given the internal variable the type |fixnum|
regardless of |i|'s type.  The type incompatibility errors that
could arise in this situation are not checked for.

\end{itemize}

Note that if you do declare |iterate:declare-variables|, then
\iter\ may declare user variables as well as internal ones if they do
not already have declarations, though only for variables that it
binds.  For instance, in this code:

\begin{program}
(iter (declare (iterate:declare-variables))
      (for i from 1 to 10)
      (collect i into var))
\end{program}
the variable |var| will be declared to be of type |list|.


\subsection{Summary}

\iter\ understands standard Common Lisp variable type declarations
that occur within an \iter\ form and
will pass them through to the generated code.  If the declaration
|(iterate:declare-variables)|\lispindex{declare-variables}
appears at the top level of an 
\iter\ form, or if 
|iterate::*always-declare-variables*|\lispindex{*always-declare-variables*} 
is \nonnil, then \iter\ will use the type information gleaned from user
declarations, self-evaluating expressions and |the| expressions,
combined with reasonable assumptions, to determine variable
types and declare them.


%%%%%%%%%%%%%%%%%%%%%%%%%%%%%%%%%%%%%%%%%%%%%%%%%%%%%%%%%%%%%%%%
\section{Problems with Code Movement}
\label{code-movement}

Some \iter\ clauses, or parts of clauses, result in code being
moved from the location of the clause to other parts of the loop.
Drivers behave this way, as do code-placement clauses like |initially|
and |finally|.  When using these clauses, there is a danger of writing
an expression that makes sense in its apparent location but will be
invalid or have a different meaning in another location.  For example:
\begin{program}
(iter (for i from 1 to 10)
      (let ((x 3))
        (initially (setq x 4))))
\end{program}
While it may appear that the |x| of |(initially (setq x 4))| is the
same as the |x| of |(let ((x 3)) \dots|, in fact they are not:
|initially| moves its code outside the loop body, so |x| would refer
to a global variable.  Here is another example of the same problem:
\begin{program}
(iter (for i from 1 to 10)
      (let ((x 3))
        (collect i into x)))
\end{program}
If this code were executed, |collect| would create a binding for its
|x| at the top level of the \iter\ form that the |let| will shadow.

Happily, \iter\ is smart enough to catch these errors; it
walks all problematical code to ensure that free variables are not
bound inside the loop body, and checks all variables it binds for the
same problem.

However, some errors cannot be caught:

\begin{program}
(iter (with x = 3)
      (for el in list)
      (setq x 1)
      (reducing el by \#'+ initial-value x))
\end{program}
|reducing| moves its |initial-value| argument to the initialization
part of the loop in order to produce more efficient code.  Since
\iter\ does not perform data-flow analysis, it cannot determine that
|x| is changed inside the loop; all it can establish is that |x| is
not bound internally.  Hence this code will not signal an
error and will use $3$ as the initial value of the reduction.

The following list summarizes all cases that are subject to these code
motion and variable-shadowing problems.
\begin{itemize}
\item Any variable for which \iter\ creates a binding, including those
used in |with| and the |into| keyword of many clauses.

\begin{sloppypar}
\item The special clauses which place code: |initially|, |after-each|, |else|,
|finally| and |finally-protected|. 
\end{sloppypar}

\item The variables of a |next| or |do-next| form.

\item The |initially| arguments of |for\dots initially\dots then| and
|for\dots previous|. 

\item The |then| argument of |for\dots initially\dots then|.

\item The |initial-value| arguments of |reducing| and |accumulate|.

\item The |on-failure| argument of |finding\dots such-that|.
\end{itemize}

%%%%%%%%%%%%%%%%%%%%%%%%%%%%%%%%%%%%%%%%%%%%%%%%%%%%%%%%%%%%%%%%
\section{Differences Between {\tt Iterate} and {\tt Loop}}
|loop| contains a great deal of complexity which \iter\ tries to
avoid.  Hence many esoteric features of |loop| don't exist in \iter.
Other features have been carried over, but in a cleaned-up form.  
And of course, many new features have been added; they are not
mentioned in this list.

\begin{itemize}

\item \iter's syntax is more Lisp-like than |loop|'s, having a higher
density of parens.

\item The current implementation of \iter, unlike the current version
of  |loop| (as documented in {\em Common Lisp, 2nd Ed.\/}), is
extensible (see section \ref{extend}).  

\item |loop| puts the updates of all driver variables at the top of
the loop; \iter\ leaves them where the driver clauses appear.

\item While for the most part \iter\ clauses that resemble |loop| clauses
behave similarly, there are some differences.  For instance, there is
no |for\dots =\dots then| in \iter; instead use 
|for\dots initially\dots then|.  

\item |loop| binds the variable |it| at certain times
to allow pseudo-English expressions like |when ~expr~ return it|. 
In \iter, you must bind ~expr~ to a variable yourself.  Note that
|when ~expr~ return it| is like |thereis ~expr~| except that the latter is an
accumulation clause and therefore competes with other accumulations
(remember section \ref{multiple} above).

% repeat different behaviour of |always| clause here?

\item |loop| has a special |return| clause, illustrated in the
previous item.  \iter\ doesn't need one,  since an ordinary Lisp
|return| has the same effect.

\item |loop| allows for parallel binding and stepping of iteration
variables.  \iter\ does not.  (See section \ref{bindings}.)

\item |loop| and \iter\ handle variable type declarations very
differently.  |loop| provides a special syntax for declaring variable
types, and does not examine declarations.  Moreover, the standard
implementation of |loop| will
generate declarations when none are requested.
\iter\ parses standard Common Lisp type declarations, and will never
declare a variable itself unless declarations are 
specifically requested.

\end{itemize}

%%%%%%%%%%%%%%%%%%%%%%%%%%%%%%%%%%%%%%%%%%%%%%%%%%%%%%%%%%%%%%%%
\section{Rolling Your Own}
\label{extend}

\subsection{Introduction}

\iter\ is extensible---you can write new clauses that embody new
iteration patterns.  You might want to write a new driver clause for a
data structure of your own, or you might want to write a clause that
collects or manipulates elements in a way not provided by \iter. 

This section describes how to write clauses for \iter.  Writing a
clause is like writing a macro.  In fact, writing a clause ~is~
writing a macro: since \iter\ code-walks its body and macroexpands,
you can add new abstractions to \iter\ with good old |defmacro|.

Actually, there are two extensions you can make to \iter\ that are
even easier than writing a macro.  They are adding a synonym for an
existing clause and defining a driver clause for an indexable
sequence.  These can be done with |defsynonym| and
|defclause-sequence|, respectively.  See section \ref{aids}, below.

The rest of this section explains how to write macros that expand into
\iter\ clauses.
Here's how you could add a simplified version of \iter's |multiply|
clause, if \iter\ didn't already have one:
\begin{program}
(defmacro multiply (expr)
  `(reducing ,expr by \#'* initial-value 1))
\end{program}

If you found yourself summing the square of an expression often, you
might want to write a macro for that.  A first cut might be
\begin{program}
(defmacro sum-of-squares (expr)
  `(sum (* ,expr ,expr)))
\end{program}
but if you are an experienced macro writer, you will realize that this
code will evaluate ~expr~ twice, which is probably a bad idea.  A
better version would use a temporary:
\begin{program}
(defmacro sum-of-squares (expr)
  (let ((temp (gensym)))
    `(let ((,temp ,expr))
       (sum (* ,temp ,temp)))))
\end{program}
Although this may seem complex, it is just the sort of thing you'd
have to go through to write any macro, which illustrates the point of
this section: if you can write macros, you can extend \iter.

Our macros don't use \iter's keyword-argument
syntax.  We could just use keywords with |defmacro|, but we would
still not be using \iter's clause indexing mechanism.  Unlike Lisp,
which uses just the first symbol of a form to determine what function
to call, \iter\ individuates clauses by the list of required keywords.
For instance, |for\dots in| and |for\dots in-vector| are different clauses
implemented by distinct Lisp functions.  

To buy into this indexing scheme, as well as the keyword-argument
syntax, use |defmacro-clause|:

\begin{clauses}

\defmacro{defmacro-clause}{arglist |\&body| body}
Defines a new \iter\ clause.  ~arglist~ is a list of symbols which are
alternating keywords and arguments.  \opt\ may be used, and the list
may be terminated by |\&sequence|.  ~body~ is an ordinary macro body,
as with |defmacro|.  If the first form of ~body~ is a string, it is
considered a documentation string and will be shown by
|display-iterate-clauses|. |defmacro-clause| will signal an error if 
defining the clause would result in an ambiguity.  E.g. you cannot
define the clause |for\dots from| because there would be no way to
distinguish it from a use of the |for| clause with optional keyword |from|.

\end{clauses}

\medskip

Here is |multiply| using |defmacro-clause|.  The keywords are capitalized
for readability.
\begin{program}
(defmacro-clause (MULTIPLY expr \opt\ INTO var)
  `(reducing ,expr by \#'* into ,var initial-value 1))
\end{program}
You don't have to worry about the case when |var| is not supplied; for
any clause with an |into| keyword, saying |into nil| is equivalent to
omitting the |into| entirely.

As another, more extended example, consider the fairly common
iteration pattern that 
involves finding the sequence element that maximizes (or minimizes) some
function.  \iter\ provides this as |finding\dots maximizing|, but it's
instructive to see how to write it.
Here, in pseudocode, is how you might write such a loop for
maximizing a function F:

\begin{quote}
\begin{tabbing}
set variable MAX-VAL to NIL; \\
set variable WINNER to NIL; \\
for\=\ each element EL in the sequence \\
\>  if\=\ MAX-VAL\=\ is NIL or F(EL) $>$ MAX-VAL then \\
\>\>    set MAX-VAL to F(EL); \\
\>\>    set WINNER to EL; \\
\>  end if; \\
end for; \\
return WINNER.
\end{tabbing}
\end{quote}

Here is the macro:
\begin{program}
(defmacro-clause (FINDING expr MAXIMIZING func \opt\ INTO var)
  (let ((max-val (gensym))
        (temp1 (gensym))
        (temp2 (gensym))
        (winner (or var iterate::*result-var*)))
    `(progn 
       (with ,max-val = nil)
       (with ,winner = nil)
       (cond 
        ((null ,max-val)
         (setq ,winner ,expr)
         (setq ,max-val (funcall ,func ,winner))
        (t
         (let* ((,temp1 ,expr)
                (,temp2 (funcall ,func ,temp1)))
           (when (> ,temp2 ,max-val)
             (setq ,max-val ,temp2)
             (setq ,winner ,temp1))))))
       (finally (leave ,winner)))))
\end{program}
Note that if no |into| variable is supplied, we use
|iterate::*result-var*|, which contains the internal variable into
which all clauses place their results.  If this variable is bound by
some clause, then \iter\ will return its value automatically;
otherwise, \nil\ will be returned.  

\subsection{Writing Drivers}

In principle, drivers can be implemented just as easily as other
\iter\ clauses.  In practice, they are a little harder to get right. 
As an example, consider writing a driver that
iterates over all the 
elements of a vector, ignoring its fill-pointer.  |for\dots in-vector|
won't work for this, because it observes the fill-pointer.  It's necessary to
use |array-dimension| instead of |length| to obtain the size of the
vector.  Here is one approach:

\begin{program}
(defmacro-clause (FOR var IN-WHOLE-VECTOR v)
  "All the elements of a vector (disregards fill-pointer)"
  (let ((vect (gensym))
        (index (gensym)))
    `(progn
       (with ,vect = ,v)
       (for ,index from 0 below (array-dimension ,vect 0))
       (for ,var = (aref ,vect ,index)))))
\end{program}
Note that we immediately put |v| in a variable, in case it is an
expression.  Again, this is just good Lisp macrology.  It also has a
subtle effect on the semantics of the driver: |v| is evaluated only
once, at the beginning of the loop, so changes to |v| in the loop have
no effect on the driver.  Similarly, the bounds for numerical iteration
e.g. the above |array-dimension| are also evaluated once only.  This is how
all of \iter's drivers work.


There is an important point concerning the |progn| in this code.  We
need the |progn|, of course, because we are returning several forms,
one of which is a driver.  But \iter\ drivers must occur at top-level.
Is this code in error?  No, because ~top-level~ is defined in \iter\
to include forms inside a |progn|.  This is just the definition of
top-level that Common Lisp uses, and for the same reason: to allow
macros to return multiple forms at top-level.

While our |for\dots in-whole-vector| clause will work, it is not
ideal.  In particular, it does not support generating.  Do do so, we
need to use |for\dots next| or |for\dots do-next|.  The job is
simplified by the |defmacro-driver| macro.

\begin{clauses}

\defmacro{defmacro-driver}{arglist |\&body| body}
Defines a driver clause in
both the |for| and |generate| forms, and provides a parameter
|generate| which ~body~ can examine to determine how it was invoked.
~arglist~ is as in |defmacro-clause|, and should begin with the symbol
|for|. 

\end{clauses}

With |defmacro-driver|, our driver looks like this:
\begin{program}
(defmacro-driver (FOR var IN-WHOLE-VECTOR v)
  "All the elements of a vector (disregards fill-pointer)"
   (let ((vect (gensym))
         (end (gensym))
         (index (gensym))
         (kwd (if generate 'generate 'for)))
     `(progn
        (with ,vect = ,v)
        (with ,end = (array-dimension ,vect 0))
        (with ,index = -1)
        (,kwd ,var next (progn (incf ,index)
                               (if (>= ,index ,end) (terminate))
                               (aref ,vect ,index))))))
\end{program}


We are still missing one thing: the |\&sequence| keywords.  
We can get them easily enough, by writing 
\begin{program}
(defmacro-driver (FOR var IN-WHOLE-VECTOR v \&sequence) 
  ...)
\end{program}
We can now refer to parameters |from|, |to|, |by|, etc. which contain
either the values for the corresponding keyword, or \nil\ if the
keyword was not supplied.  Implementing the right code for these
keywords is cumbersome but not difficult; it is left as an exercise.
But before you begin, see |defclause-sequence| below for an easier way.
       
\subsection{Extensibility Aids}
\label{aids}

This section documents assorted features that may be of use in
extending \iter.

\begin{clauses}

\defunexpvar{*result-var*}
Holds the variable that is used to return a value as a result of the
\iter\ form.  You may examine this and use it in a |with| clause, but
you should not change it.

\defmacro{defsynonym}{syn word}
 Makes ~syn~ a synonym for the existing \iter\ keyword ~word.~  Only
the first word in each clause can have synonyms. 

\defmacroxx{defclause-sequence}{element-name index-name}{|\&key|
access-fn size-fn sequence-type}{element-type
element-doc-string index-doc-string}  
Provides
a simple way to define sequence clauses.  Generates two
clauses, one for iterating over the sequence's elements, the other
for iterating over its indices.  The first symbol of both
clauses will have print-name |for|.
~element-name~  and ~index-name~ should be symbols.
~element-name~ is the second keyword of the element iterator (typically of
the form 
|in-~sequence-type~|), and ~index-name~ is the second keyword
of the index-iterator (typically of the form
|index-of-~sequence-type~|).  Either name may be 
\nil, in which case the corresponding clause is not defined.  If both
symbols are supplied, they should be in the same package.  The |for|
that begins the clauses will be in this package.

\cpar ~access-fn~ is the function to be used to
access elements of the sequence in the element iterator.  The function
should take two 
arguments, a sequence and an index, and return the appropriate element.
~size-fn~ should denote a function of one argument, a sequence, that
returns its size.  Both ~access-fn~ and ~size-fn~ are required for the
element iterator, but only ~size-fn~ is needed for the index iterator.

\cpar The ~sequence-type~ and ~element-type~ keywords are used to
suggest types for the variables 
used to hold the sequence and the
sequence elements, respectively.  The usual rules about \iter's
treatment of variable type declarations apply (see section \ref{types}).

\cpar ~element-doc-string~ and ~index-doc-string~ are
the documentation strings, for use with |display-iterate-clauses|. 

\cpar The generated element-iterator performs destructuring on the
element variable.

\cpar As an example, the above |for\dots in-whole-vector| example
could have been written:
\begin{program}
(defclause-sequence IN-WHOLE-VECTOR INDEX-OF-WHOLE-VECTOR
  :access-fn 'aref
  :size-fn \#'(lambda (v) (array-dimension v 0))
  :sequence-type 'vector
  :element-type t
  :element-doc-string 
     "Elements of a vector, disregarding fill-pointer"
  :index-doc-string 
     "Indices of vector, disregarding fill-pointer")
\end{program}

\end{clauses}

\subsection{Subtleties}

There are some subtleties to be aware of when writing \iter\ clauses.
First, the code returned by your macros may be |nconc|'ed into a list,
so you should always returned freshly consed lists, rather than
constants.  Second, \iter\ matches clauses by using |eq| on the first
symbol and |string=| on the subsequent ones, so the package of the
first symbol of a clause is relevant.  All of the clauses in this manual
have their first word in the \iter\ package.  
You can use the package system in the usual way to shadow
\iter\ clauses without replacing them.

%%% say more here, about the badness that only the first word of a
%%% clause is packagey.

\section{Non-portable Extensions to Iterate (Contribs)}
\label{contribs}

Currently, there is only one non-portable extension to iterate in the
distribution: iterate-pg. If you have made an extension that depends
on non-portable features, feel free to send them to |asf@boinkor.net|
for inclusion in the iterate distribution.

\subsection{An SQL query driver for iterate}

The pg package by Eric Marsden (see |http://cliki.net/pg|) provides an
interface to the PostgreSQL database. Using the \iterpg\ extension, it
is possible to handle the results of SQL queries using \iter.

This usage example should give you an idea of how to use it:

\begin{program}
(pg:with-pg-connection (c "somedb" "someuser")
  (iter (for (impl version date) in-relation "select * from version"
                                 on-connection *dbconn*)
        (collect version)))
\end{program}

To use the extension via |ASDF|, simply make your system depend on the
|iterate-pg| system instead of the |iterate| system. To load it
manually, use:

\begin{program}
  (asdf:oos 'asdf:load-op :iterate-pg)
\end{program}

%%%%%%%%%%%%%%%%%%%%%%%%%%%%%%%%%%%%%%%%%%%%%%%%%%%%%%%%%%%%%%%%
\section{Obtaining {\tt Iterate}}

\begin{sloppypar}
The information in this chapter is obsolete but included for
completeness's sake; Currently, the most up-to-date information on
\iter\ can be found at |http://boinkor.net/iterate.html|.
\end{sloppypar}

\begin{sloppypar}
\iter\ currently runs on Lisp Machines, and on
HP's, Sun3's and Sparcstations under Lucid.  
\iter\ source and binaries are available at the MIT AI Lab in the
subdirectories of |/src/local/lisplib/|.  The source file,
|iterate.lisp|, is also available for anonymous FTP in the directory
|/com/fpt/pub/| on the machine |TRIX.AI.MIT.EDU| (Internet number
128.52.32.6).  If you are unable to obtain |iterate| in one of these
ways, send mail to |jba@ai.mit.edu| and I will send you the source
file. 
\end{sloppypar}

\begin{sloppypar}
\iter\ resides in the |iterate| package (nickname |iter|).  Just say
\linebreak |(use-package :iterate)| to make all the necessary symbols
available. 
If a symbol is not exported, it appears in this manual with an
``|iterate::|'' prefix.
\end{sloppypar}

Send bug reports to |bug-iterate@ai.mit.edu|.  The |info-iterate|
mailing list will have notices of changes and problems; to have
yourself added, send mail to |info-iterate-request@ai.mit.edu|.


\medskip

\begin{flushleft}
 \bf Acknowledgements
\end{flushleft}
\smallskip

Richard Waters provided invaluable criticism which spurred me to improve
\iter\ greatly.  As early users, David Clemens, Oren Etzioni and Jeff
Siskind helped ferret out many bugs.

%\begin{theindex}
% The index files must be generated with the genindex program
% or from makeindex iterate-manual.idx which creates iterate-manual.ind.

%\baselineskip\normalbaselineskip
%\advance\baselineskip by -2pt

\input{iterate-manual.ind}

%\underline{Clauses}
%\input{iterate-manual.clindex}
%
%\indexspace
%\underline{Lisp}
%\input{iterate-manual.lispindex}

%\end{theindex}


\end{document}


% arch-tag: "c0f871a6-313c-11d8-abb9-000c76244c24"


%\end{theindex}


\end{document}


% arch-tag: "c0f871a6-313c-11d8-abb9-000c76244c24"


%\end{theindex}


\end{document}


% arch-tag: "c0f871a6-313c-11d8-abb9-000c76244c24"


%\end{theindex}


\end{document}


% arch-tag: "c0f871a6-313c-11d8-abb9-000c76244c24"
